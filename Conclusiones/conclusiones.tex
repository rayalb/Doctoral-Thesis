\chapter{Conclusiones}\label{chap:conclusiones}

En esta tesis se estudiaron y se propusieron nuevos métodos para hacer frente a dos grandes dificultades que aparecen en problemas de estimación espectral, la estimación del orden del modelo, y la estabilidad numérica de las soluciones.

Para  el problema de la selección del orden del modelo para señales compuestas por una suma de exponenciales complejas, se desarrollaron nuevas interpretaciones de los métodos existentes basados en ángulos entre subespacios. Este nuevo análisis permite exponer las desventajas que presentan los métodos anteriormente publicados cuando la señal está contaminada ruido. Además, gracias al análisis mediante ángulos entre subespacios, se obtuvieron nuevos resultados sobre las perturbaciones de un haz matricial con estructural Hankel. Motivado por la falta de robustez de los métodos analizados previamente en regímenes de alto ruido, se obtuvo un umbral para obtener la cantidad de valores singulares relevantes. Para hallar este umbral se obtuvo una cota probabilística de la norma espectral de una matriz Hankel aleatoria. A partir de estos desarrollos se propuso una nueva estrategia para la estimación del orden. La nueva propuesta explota la propiedad de invariancia de la matriz de Hankel, que es beneficiosa cuando existe una buena separación entre el ruido y el subespacio de la señal. Sin embargo, también impone una restricción sobre los valores singulares de la matriz de Hankel observada para penalizar las estimaciones de orden pequeño cuando la SNR es baja. Al resolver un problema de optimización con restricciones, el objetivo es resolver ambos problemas simultáneamente: descartar los componentes ruidosos de la señal y preservar la estructura de la señal. La restricción del problema de optimización se construye acotando superiormente la norma espectral de la matriz de Hankel asociada a la señal contaminada con ruido. En particular, cuando el ruido es Gaussiano, obtenemos una expresión analítica para su función de distribución. Usando aproximaciones numéricas, también se obtuvieron límites apropiados cuando se desconoce el nivel de potencia del ruido. Para probar el rendimiento del esquema propuesto, se han comparado con reglas de selección previamente conocidas usando ejemplos que ya publicados. Tomando varios ejemplos de sumas exponenciales de diferentes órdenes con frecuencias aleatorias no amortiguadas y frecuencias amortiguadas que podrían estar muy agrupadas muy cerca entre sí. Cada ejemplo se testeó con diferentes valores de SNR. En todos los casos, esta nueva propuesta mostró el mejor rendimiento para una amplia gama de valores de SNR.

Se analizó el comportamiento de la estimación de la frecuencia cuando los datos observados están sujetos a pequeñas perturbaciones. En particular, se estudió la estabilidad  numérica de las técnicas de estimación espectral. Se demostró que las estimaciones se vuelven vulnerables a pequeñas perturbaciones en los datos observados cuando la energía asociada a las frecuencias es pequeña y/o las frecuencias están muy cerca entre sí. La   técnica \emph{shift-and-zoom} hace frente a este problema cuando se dispone de información a priori. Se demostró que esta técnica supera a los enfoques basados en la estructura de la matriz de Hankel que no se benefician de la mejora en el espaciado entre las frecuencias complejas introducido por la  técnica \emph{shift-and-zoom}. Se demostró que este esquema es más eficiente que los esquemas tradicionales cuando se trabaja en un régimen de SNR bajas. En cambio cuando la SNR es alta, esta nueva técnica es tan eficiente como los métodos tradicionales. En este caso, la disminución en el número de muestras debido al paso de decimación se vuelve relevante y \emph{shift-and-zoom} requiere una cantidad de datos más grande para converger a límite de Cramér-Rao. Por lo tanto, existe un compromiso entre el número de muestras y el factor de diezmado. 

Los métodos aquí desarrollados se aplicaron a señales reales para identificar las frecuencias naturales de dos cilindros de misma forma pero con diferente concentración de agua, con el fin de encontrar características que posibiliten poder clasificar entre los diferentes blancos. Estas señales fueron obtenidas a partir de una plataforma de radar impulsivo UWB desarrollada en el Centro de simulación computacional (CSC).

Se propuso un método de clasificación basado en el concepto de rango numérico de un haz de matrices. Además, utilizando la estructura Hankel del haz, se pueden hallar expresiones cerradas para una comprobación rápida de la inclusión de las frecuencias naturales en el rango numérico. El algoritmo de clasificación planteado, permite clasificar la señal sin obtener explícitamente el espectro del haz matricial. El algoritmo consiste en tener una conjunto de frecuencias candidatas e ir probando una a una si se encuentran en el rango numérico del haz asociado a una observación. Si se encuentra que existe una frecuencia natural que no se encuentra en el rango numérico, se obtiene una clasificación incorrecta. En cambio si todas las frecuencias naturales están en el rango numérico, se obtiene una clasificación correcta. Los resultados obtenidos muestran la capacidad del método para clasificar esferas de diferentes materiales dieléctricos. 

%Sin embargo, como se vio a lo largo del trabajo, las matrices que conforman el haz suelen estar mal condicionadas, lo que puede causar que el rango numérico asociado al haz sea muy grande y la clasificación se deteriore. Por lo tanto, a futuro se debe trabajar en rediseñar la estrategia de clasificación. Una alternativa es usar el Pseudoespectro de una matriz en lugar del rango numérico. El Pseudoespectro es un conjunto que ``muestra'' como se mueven los autovalores bajo cierta perturbación. Si bien este conjunto suele ser una mejor aproximación del espectro que el rango numérico, el Pseudoespectro muchas veces suele ser un conjunto no convexo, por lo que definir estrategias de clasificación puede ser un desafío. 

Finalmente, en los últimos años, los algoritmos de aprendizaje automático como la inteligencia artificial han ganado atención como medio para resolver problemas no lineales complejos. Se espera  que estos algoritmos adecuadamente diseñados sean capaces de extraer la información necesaria  para lograr un buen desempeño tanto para la estimación del orden del modelo como para la clasificación de señales a partir de sus frecuencias \cite{Moon2021, Mhaska2020}. En particular, los modelos de entrenamiento para el aprendizaje requieren grandes cantidades de datos para minimizar el error de generalización. Desafortunadamente, muchas veces los datos de entrenamiento no están disponibles para todas las aplicaciones. En los últimos años se han explorado técnicas de aprendizaje automático para la clasificación de objetos de diferentes formas y/o composición \cite{Agresti2019, WOOD2020, Yasmeen2023}.  Aunque algunos de estos trabajos han mostrado resultados prometedores, estos esquemas no explotan los beneficios del modelo exponencial. Además, varios enfoques se basan únicamente en los datos y no hay indicación de como pueden generalizarse a otro tipo de escenarios. Se deja como trabajo a futuro el estudio de estas técnicas, tanto para la estimación del orden del modelo como para la clasificación de señales a partir de sus frecuencias. En particular, poder explotar la clasificación de materiales aprovechando las propiedades físicas de la dispersión electromagnética, incorporando el análisis basados en técnicas de aprendizaje utilizando las frecuencias naturales como descriptores de clases para diferentes objetos. Otro tema que podría agregarse a futuro es el estudio de técnicas avanzadas de reducción de ruido para mejorar el rendimiento de las técnicas de clasificación.

%Finalmente, dado los avances de los métodos de aprendizaje profundo, se debe mencionar que estos puede usarse para la estimación del orden del modelo \cite{Moon2021} y la estimación de los parámetros \cite{Mhaska2020}. En particular, los modelos de entrenamiento para el aprendizaje requieren grandes cantidades de datos para minimizar el error de generalización. Desafortunadamente, muchas veces los datos de entrenamiento no está disponibles para todas las aplicaciones.

\section{Trabajos publicados}

\subsection*{Artículos revistas internacionales}
\begin{itemize}
	\item Raymundo Albert, Cecilia Galarza. \textit{A constraint optimization problem for model order estimation}. Signal Processing, 2023.
	\item Raymundo Albert, Cecilia Galarza. \textit{Spectrum estimation using frequency shifting and decimation}. IET Signal Processing, 2020.
\end{itemize}
\subsection*{Artículos en conferencia internacionales}
\begin{itemize}
	\item Raymundo Albert, Cecilia Galarza. \textit{Model order selection for sum of complex exponentials}. IEEE URUCON, 2021.
\end{itemize}
\subsection*{Artículos en conferencia nacionales}
\begin{itemize}
	\item Raymundo Albert, Cecilia Galarza. \textit{Classification of matrices using their numerical range}. Congreso bienal de la sección Argentina del IEEE (ARGENCON) 2018.
	\item Raymundo Albert, Magdalena Bouza, Andrés Altieri, Cecilia Galarza. \textit{Dielectric classification by sensing scattering field}. XVII Reunión de trabajo en procesamiento de la información y control 2017.
\end{itemize}
\subsection*{Artículos en seminarios de divulgación}
\begin{itemize}
	\item Raymundo Albert, Cecilia Galarza. \textit{Técnicas ``Shift-and-Zoom´´ para la identificación de materiales dieléctrico}. Seminarios de Vinculación y Transferencia, Facultad de Ingeniería, UBA 2021.
	\item  Raymundo Albert, Cecilia Galarza. \textit{Estudio de técnicas de procesamiento para la clasificación de materiales dieléctricos}. Seminarios de Vinculación y Transferencia, Facultad de Ingeniería, UBA 2019.
	\item Raymundo Albert, Cecilia Galarza. \textit{Identificación de target mediante radares UWB}. VI Workshop en procesamiento de señales y comunicaciones, Red Universitaria de Telecomunicaciones, 2015.
\end{itemize}




%En el capitulo \ref{chap:sinalUWB} el objetivo fue describir brevemente el comportamiento físico de las ondas electromagnéticas cuando impactan contra un cierto blanco. La señal reflejada se puede caracterizar como una mezcla de señales sinusoidales donde sus frecuencias transporta una gran cantidad de información sobre las características del objeto. Se aplicaron los métodos propuestos en los capítulos anteriores para identificar las frecuencias naturales de dos cilindros de misma forma pero con diferente concentración de agua con el fin de encontrar características que posibiliten poder clasificar entre los diferentes blancos.


%En el capítulo \ref{chap:RangoNumerico} se propuso un método de clasificación basado en el concepto de rango numérico de un haz de matrices. Además, utilizando la estructura Hankel del pencil, se pueden hallar expresiones cerradas para una comprobación rápida de la inclusión de las frecuencias naturales en el rango numérico. El algoritmo de clasificación planteado, permite clasificar la señal sin obtener explícitamente el espectro del haz matricial. El algoritmo consiste en tener una conjunto de frecuencias candidatas e ir probando una a una si se encuentran en el rango numérico del haz asociado a una observación. Si se encuentra que existe una frecuencia natural que no se encuentra en el rango numérico, se obtiene una clasificación incorrecta. En cambio si todas las frecuencias naturales están en el rango numérico, se obtiene una clasificación correcta. Los resultados obtenidos muestran la capacidad del método para clasificar esferas de diferentes materiales dieléctricos. 

%Sin embargo, como se vio a lo largo del trabajo las matrices que conforman el haz suelen estar mal condicionadas, lo que puede causar que el rango numérico asociado a la haz sea muy grande y la clasificación se deteriore. Por lo tanto, a futuro se debe trabajar en rediseñar la estrategia de clasificación. Una alternativa es usar el Pseudoespectro de una matriz en lugar del rango numérico. El Pseudoespectro es un conjunto que ``muestra'' como se mueven los autovalores bajo cierta perturbación. Si bien este conjunto suele ser una mejor aproximación del espectro que el rango numérico, el Pseudoespectro muchas veces suele ser un conjunto no convexo, por lo que definir estrategias de clasificación puede ser un desafío. 

%Finalmente, con el arribo de la inteligencia artificial y los avances en los métodos de aprendizaje automático, se deja como trabajo a futuro el estudio de estas técnicas  tanto para la estimación del orden del modelo como para para la clasificación de señales a partir de sus frecuencias. En particular, poder explotar la extracción de modos naturales para la identificación y/o clasificación de diferentes materiales dieléctricos.





