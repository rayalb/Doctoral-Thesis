
\chapter{Estabilidad del algoritmo de estimación espectral} 
\label{chap:EstabilidadNumerica}

    
	Incluso cuando los algoritmos destinados a mitigar el ruido demuestran eficacia, el cálculo de los autovalores utilizando técnicas como ESPRIT o MPM  puede presentar desafíos en términos de estabilidad. Este problema se agudiza notablemente en situaciones en las cuales la cantidad de modos de oscilación es considerable o cuando las frecuencias complejas se encuentran en proximidad cercana unas de otras. El enfoque aquí propuesto se enfoca en el análisis del comportamiento del proceso de estimación de frecuencias cuando los datos experimentales se ven sujetos a pequeñas perturbaciones. En particular, se centra en el estudio de la estabilidad numérica asociada al problema de autovalores generalizados, que se deriva a partir de la matriz de Hankel vinculada con la señal observada. Como se puso de manifiesto en el capítulo \eqref{chap:ModeloSumExp}, se observa que la sensibilidad inherente de cada autovalor guarda una relación inversamente proporcional con el producto escalar entre los autovectores izquierdo y derecho, ponderados por la matriz de Hankel. Este producto escalar se relaciona directamente con la amplitud correspondiente a la frecuencia amortiguada asociada al respectivo autovalor.

    %Para el caso de suma de exponenciales complejas no amortiguadas en \cite{BATENKOV2018} se realiza un estudio del condicionamiento numérico de las estimaciones y se propone una estrategia de decimación de la señal dado un solo grupo de frecuencias muy cercanas entre sí. Esta estrategia permite aumentar artificialmente la distancia entre frecuencias. En \cite{Morren2003} también se propone un esquema de decimación para luego resolver un problema de cuadrados mínimos totales. Sin embargo, en estos dos trabajos el factor de decimación se elige para asegurarse de que no se introduzca solapamiento (\emph{aliasing}).  Para superar esta restricción, la señal original debe ser sobremuestreada.
        
    %En un ángulo diferente, en \cite{Bolt1979} se analiza el contenido espectral de series de tiempo geofísicas haciendo desplazamiento de frecuencia y posteriormente filtrando la señal observada. Al hacer esto, es posible identificar las frecuencias de resonancias contenidas en la señal observada.
 
    En este capítulo, se extiende los resultados obtenidos en el capítulo \ref{chap:OrdenModelo} sección \ref{sec:EstNum1} y se demuestra que la aproximación de primer orden de los autovalores perturbados no sólo depende de la amplitud asociada a los autovalores, sino también a la distancia entre los autovalores. Por lo tanto, los autovalores que estén muy cerca uno del otro son propensos a exhibir grandes desviaciones de sus valores reales, incluso cuando la señal observada está ligeramente perturbada. Además, usando la reinterpretación de la propiedad de invariancia a partir de los ángulos entre subespacios, también se obtiene una cota para la máxima perturbación. Estos resultados, fueron publicados en \cite{Albert2020,ALBERT2023}.
    
    Finalmente, se propone una estrategia para preprocesar la señal observada mediante un procesamiento multitasa. Esto tiene la capacidad de ``acomodar'' las frecuencias complejas para un cálculo más estable. Esta idea es validada por experimentos numéricos y se compara con los resultados obtenidos en publicaciones anteriores. Estos resultados fueron publicados en \cite{Albert2020}
	
	\section{Derivadas de los autovalores generalizados}
	
		Sea $\lambda_i$ el $i-$ésimo autovalor generalizado del par $(\matA,\matB)\in\C^{m\times n}$ con autovectores derechos e izquierdos $\v_i\in\C^n$ y $\vecpsi_i\in\C^m$ respectivamente. Se asume que $\matA$ y $\matB$ son matrices de rango completo y que todos los $\lambda_i$ son distintos. Por definición se tiene que
		\begin{equation}
			\matA\v_i = \lambda_i\matB\v_i,\qquad \vecpsi_i^H\matA = \lambda_i\vecpsi_i^H\matB.
			\label{Eq:Estabilidad1}
		\end{equation}
		A continuación, se consideran variaciones suaves en las matrices $\matA$ y $\matB$  dado un parámetro $\varepsilon\in\R$ en el entorno del origen, es decir, $\varepsilon\in\mathcal{B}(0) = \big\{\varepsilon\in\R:|\varepsilon-0|\le r\big\}$, para algún $r>0$. 
	
		Se asume que existen funciones diferenciables $\v_i(\varepsilon)$, $\vecpsi_i(\varepsilon)$, $\lambda_1(\varepsilon),\ldots,\lambda_n(\varepsilon)$, para todo $\varepsilon\in\mathcal{B}(0)$
		\begin{equation}
			\begin{aligned} 
				& \matA(\varepsilon)\v_i(\varepsilon) = \lambda_i(\varepsilon)\matB(\varepsilon)\v_i(\varepsilon),\\[0.3em] 
                & \vecpsi_i^H(\varepsilon)\matA(\varepsilon) = \lambda_i(\varepsilon)\vecpsi_i^H(\epsilon)\matB(\varepsilon).
			\end{aligned}
			\label{Eq:Estabilidad2}
		\end{equation}
		tal que $\lambda_i(0) = \lambda_i$, $\v_i(0) = \v_i$, $\vecpsi_i(0) = \vecpsi_i$, con $i = 1,\ldots, n$.
	
		\begin{theorem}\label{Th:Estabilidad_Autovalores}
			Dados $\matA(\varepsilon),\matB(\varepsilon)\in\C^{m\times n}$ tal que se satisface \eqref{Eq:Estabilidad2} para todo $\varepsilon\in\mathcal{B}(0)$, se tiene que
			\begin{equation}
				\dot{\lambda}_i = \frac{\mathrm{d}\lambda_i(\varepsilon)}{\mathrm{d}\varepsilon} = \frac{\vecpsi_i^H\big[\dot{\matA}-\lambda_i\dot{\matB}\big]\v_i}{\vecpsi_i^H\matB\v_i},
				\label{Eq:Eigenvalue_derivative}
			\end{equation} 
			donde $\dot{\matA}$ y $\dot{\matB}$ son las derivadas con respecto a $\varepsilon$ de $\matA(\varepsilon)$ y $\matB(\varepsilon)$.
		\end{theorem}
		\begin{proof}
			Sean $\matV = \big[\v_1,\cdots,\v_n\big]\in\C^{n\times n}$ y $\matW^H = \big[\w_1^H,\cdots,\w_n^H\big]\in\C^{n\times m}$ dos matrices con autovectores a derecha e izquierda respectivamente, y la matriz diagonal $\Lambdab = \diag(\lambda_i)$ con los autovalores. Luego, \eqref{Eq:Estabilidad1} se puede reescribir como
			\begin{equation}
				\matA\matV = \matB\matV\Lambdab
				\label{Eq:TeoremaEstabilidad_1}
			\end{equation}
			\begin{equation}
				\matW^H\matA = \Lambdab\matW^H\matB.
				\label{Eq:TeoremaEstabilidad_2}
			\end{equation}
			Multiplicando por $\matW^H$ a ambos lados en \eqref{Eq:TeoremaEstabilidad_1} se obtiene
			\[\matW^H\matA\matV = \matW^H\matB\matV\Lambdab = \Lambdab\matW^H\matB\matV,\]
			donde la última igualdad se debe a \eqref{Eq:TeoremaEstabilidad_2}. Dado que la matriz diagonal $\Lambdab$ conmuta con $\matW^H\matB\matV$, esta última matriz es diagonal, así como $\matW^H\matA\matV$ también.
		
			Si se consideran pequeñas perturbaciones definidas en \eqref{Eq:Estabilidad2}, diferenciando con respecto a $\varepsilon$, se obtiene
			\begin{equation}
				\dot{\matA}\matV + \matA\dot{\matV} = \dot{\matB}\matV\Lambdab + \matB\dot{\matV}\Lambdab + \matB\matV\dot{\Lambdab}
				\label{Eq:TeoremaEstabilidad_3}
			\end{equation}
			Como las columnas de $\matV$ son una base de $\C^n$, cualquier vector en este espacio se puede escribir como una combinación lineal de $\v_i$. Luego,
			\[\dot{\v}_j = \sum_{i=1}^n r_{ij}\v_i,\]
			definiendo $\matR = \big[r_{ij}\big]\in\C^{n\times n}$ tal que $\dot{\matV} = \matV\matR$. Reemplazando en \eqref{Eq:TeoremaEstabilidad_3} y multiplicando a izquierda ambos lados por $\matW^H$ se obtiene
			\begin{equation}
				\matW^H\dot{\matA}\matV + \matW^H\matA\matV\matR = \matW^H\dot{\matB}\matV\Lambdab + \matW^H\matB\matV\matR\Lambdab + \matW^H\matB\matV\dot{\Lambdab}.
				\label{Eq:TeoremaEstabilidad_4}
			\end{equation}
		
			Notar que 
			\[\matW^H\matA\matV\matR-\matW^H\matB\matV\matR\Lambdab = \matW^H\matB\matV\big[\Lambdab\matR-\matR\Lambdab\big].\]
			Los elementos en la diagonal de $\big[\Lambdab\matR-\matR\Lambdab\big]$ son ceros, debido a que $\Lambdab$ es una matriz diagonal. Además, como $\matW^H\matB\matV$ es una matriz diagonal, $\matW^H\matB\matV\big[\Lambdab\matR-\matR\Lambdab\big]$ también tiene ceros es su diagonal.
		
			La igualdad en \eqref{Eq:TeoremaEstabilidad_4} define  $m\times n$ ecuaciones. Evaluando aquellas correspondientes a los elementos diagonales se obtiene
			\[\w_k^H\matB\v_k\dot{\lambda}_k = \w_k^H\big[\dot{\matA}-\lambda_k\dot{\matB}\big]\v_k\]
			de igual forma que \eqref{Eq:Eigenvalue_derivative}
		\end{proof}
	
		Cuando pequeñas perturbaciones en las matrices $\matA$ y $\matB$ producen grandes desviaciones $\lambda_i$ de su verdadero valor, se dice que el autovalor está ``mal condicionado''. Según el Teorema \eqref{Th:Estabilidad_Autovalores} cuando $\vecpsi_1^H\matB\v_i$ es pequeño podemos decir que $\lambda_i$ está mal condicionado. Este es el caso cuando $\vecpsi_i$ está cerca de ser  perpendicular a $\matB\v_i$. Por otro lado, en \cite[ch. 7]{Golub1996} el producto $\vecpsi_i^H\matB\v_i$ ha sido asociado con la condición numérica de $\lambda_i$.
	
	\section{Perturbaciones en $\Hank_{\y}$}
	
		Cuando se analiza \eqref{MPM:eq}, se definen $\Hank_{\x,f}(\varepsilon)$ y $\Hank_{\x,l}(\varepsilon)$ a partir de $\Hank_{\x}(\epsilon)$, con $\varepsilon\in\R$. Para este caso, se enuncia el siguiente corolario
	
		\begin{Corollary}\label{Co:Estabilidad}
			Sean $\Hank_{\x,f}(\varepsilon)$ y $\Hank_{\x,l}(\varepsilon)$ dos funciones diferenciables que representan las versiones perturbadas del par $(\Hank_{\x,f},\Hank_{\x,l})$. Para cada $\varepsilon\in\mathcal{B}(0)$, los autovalores $z_i(\varepsilon)$, y los autovectores a derecho e izquierda $\v_i(\varepsilon)$ y $\w_i(\varepsilon)$ existen y satisfacen \eqref{Eq:Estabilidad2}. Luego,
			\begin{equation}
				|\dot{z}_i| = \bigg|\frac{\mathrm{d}z_i}{\mathrm{d}\varepsilon}\bigg|\le E \frac{1+|z_i|}{|c_i|}\frac{\varepsilon_i}{|P_i(z_i)|^2}\quad i = 1,\ldots,r
				\label{Eq:CorollaryEstabilidad1}
			\end{equation} 
			donde $E = \max(\|\dot{\Hank}_{\x,f}\|, \|\dot{\Hank}_{\x,l}\|)$, 
			\[ P_i(z) = \prod_{\stackrel{l=1}{l\neq i}}^{r}(z-z_l)\]
			es un polinomio en $z$ de grado $r-1$, y
			\[\varepsilon_i= \frac{1}{2\pi}\int_{-\pi}^{\pi}|P_i(e^{\jmath\omega})|^2\mathrm{d}\omega.\]
		\end{Corollary}
		\begin{proof}
			Aplicando el Teorema \eqref{Th:Estabilidad_Autovalores} al par $(\Hank_{\x,f},\Hank_{\x,l})$ , se obtiene
			\begin{equation}
				\begin{aligned}
					|\dot{z}_i| & \le \frac{\|\dot{\Hank}_{\x,f} - z_i\dot{\Hank}_{\x,l}\|_2\|\w_i\|_2\|\v_i\|_2}{|\w_i^H\Hank_{\x,l}\v_i|} \\[0.3em]
					& \le \max(\|\dot{\Hank}_{\x,f}\|_2, \|\dot{\Hank}_{\x,l}\|_2)\frac{(1+|z_i|)\|\w_i\|_2\|\v_i\|_2}{|\w_i^H\Hank_{\x,l}\v_i|},
				\end{aligned}
				\label{Eq:CorollaryEstabilidad2}
			\end{equation}
			con $\w_i$ y $\v_i$ son los autovectores a izquierda y derecha del par $(\Hank_{\x,f},\Hank_{\x,l})$.
			Recordando la descomposición en matrices de Vandermonde de la matriz de Hankel dada en \eqref{eq:VandermondeDecomposition}, los autovectores satisfacen
			\begin{equation}
				\matZ_{r-1}^T\v_i = \one_i\qquad \w_i^H\matZ_{m-1} = \one_i^T
				\label{Eq:CorollaryEstabilidad3}
			\end{equation}
			$\one_i$ es el $i-$ésimo vector unitario en $\R^r$. Además,
			\begin{equation}
				\w_i^H\Hank_{\x,l}\v_i = \w_i^H\matZ_{m-1}\matD\matZ_{r-1}^T\v_i = c_i
				\label{Eq:CorollaryEstabilidad4}
			\end{equation}
			Definiendo $\matZ_{m-1}^T = \big[\matZ_{r-1}^T\ \matM^T\big]$, con $\matM\in\C^{(m-r)\times r}$, eligiendo $\w_i^T = [\v_i^T\ \mathbf{0}_{1\times(m-r)}]$ satisface \eqref{Eq:CorollaryEstabilidad3}. Por otro lado, y dado que $\matZ_{r-1}$ es una matriz invertible, \eqref{Eq:CorollaryEstabilidad3} implica que $\v_i$, es la $i-$ésima columna de $\matZ_{r-1}^{-T}$. Luego, usando el resultado dado en \cite{Rawashdeh2018}, para $k=1,\ldots,n$ se obtiene
			\begin{equation}
				\begin{aligned}
					& (\v_i)_k = (-1)^{k-1}\frac{(\alpha_i)_k}{\prod_{\stackrel{l=1}{l\neq i}}^n(z_l-z_i)}\\[0.3em]
					& (\alpha_i)_k = \sum_{l=1}^{\binom{r-1}{r-k}}z_{s_1}\cdots z_{s_{n-k}},
				\end{aligned}
				\label{Eq:CorollaryEstabilidad5}
			\end{equation}
			donde $s_1\cdots s_{n-k}$ es una combinación de $n-k$ elementos tomados del conjunto $\{1,\ldots,i-1,i+1,\ldots,n\}.$ Dada la elección del autovector a izquierda, se obtiene que $\|\w_i\|_2=\|\v_i\|_2$. Luego,
			\begin{equation}
				\|\w_i\|_2\|\v_i\|_2 = \frac{\sum_{k=1}^{r}|(\alpha_i)_k|^2}{\prod_{\stackrel{l=1}{l\neq i}}^r|z_l-z_i|^2}.
				\label{Eq:CorollaryEstabilidad6}
			\end{equation}
			Además, $(\alpha_i)_k$ es el $k-$ésimo coeficiente del polinomio
			\[P_i(z) = \prod_{\stackrel{l=1}{l\neq i}}^r(z-z_l).\]
			Reordenando los términos, se obtiene que
			\[z^{-(r-1)}P_i(z) = \sum_{k=0}^{r-1}(\alpha_i)_{n-1-k}z^{-k}.\]
			Usando el teorema de Parseval entre la secuencia $\alpha_i{n-1-k}$ y su transformada de Fourier de tiempo discreta se obtiene
			\begin{equation}
				\sum_{k=0}^{r-1}|(\alpha_i)_k|^2 = \frac{1}{2\pi}\int_{-\pi}^{\pi}|P_i(e^{\jmath\omega})|^2\mathrm{d}\omega = \varepsilon_i
				\label{Eq:CorollaryEstabilidad7}
			\end{equation}
			donde en la integrar se evalúa el polinomio $P_i(z)$ en el círculo unitario $z=e^{\jmath\omega}$. Reemplazando \eqref{Eq:CorollaryEstabilidad7} en \eqref{Eq:CorollaryEstabilidad6}, y junto con \eqref{Eq:CorollaryEstabilidad4} en \eqref{Eq:CorollaryEstabilidad2} se obtiene
			\begin{equation}
				|\dot{z}_i|\le E\frac{(1+|z_i|)}{|c_i|}\frac{\varepsilon_i}{\prod_{\stackrel{l=1}{l\neq i}}^r|z_l-z_i|^2}
				\label{Eq:CorollaryEstabilidad8}
			\end{equation}
		\end{proof}
	
		El Corolario \eqref{Co:Estabilidad} muestra que los algoritmos de estimación espectral, como ESPRIT o MPM, la estimación de $z_i$ es sensible no solo a $|c_i|$ pero también a $\min_l|z_l-z_i|$. La constante $c_i$ representa el residuo asociado a $z_i$ y se relaciona con la energía observada de la frecuencia compleja. Por lo tanto, cuando $|c_i|$ es pequeña, el autovalor asociado puede ser fácilmente perturbado. Por otro lado, cuando un autovalor está localizado dentro de un cluster de autovalores, $\dot{z}_i$ puede tomar valores muy grandes. Como resultado, la sensibilidad a los errores numéricos de la frecuencia estimada $z_i$ se ve afectada no solo por la energía de cada frecuencia compleja, sino también por la proximidad entre los diferentes modos. 
		
		
	\subsection{Perturbaciones haz matricial usando ángulos principales}
	
	
	Una interpretación de como puede afectar las perturbaciones a la solución del problema de autovalores generalizados se puede realizar en función de la relación entre los ángulos principales y el principio de invariancia introducidos en el capítulo \eqref{chap:OrdenModelo}. 
	
	Sea el caso particular $s=r$ y se considera que $\matQ_l\matQ_l^\dagger = \matG(\matZ + \delta\matZ)\matG^{-1}$, donde $\delta\matZ$ es una matriz diagonal que contiene la perturbación asociada a cada autovalor y $\matG$ es la matriz de autovectores. Luego, de acuerdo con \eqref{eq:ESTER_cost},
	\begin{equation} J_{ESTER}(r) = \|\matQ_f -\matQ_l\matG(\matZ+\delta\matZ)\matG^{-1}\|_2\le \sin\vartheta_1. \label{eq:Jester_appendix}\end{equation}
	Por otro lado,
	\begin{equation}
	\begin{aligned} 
	J_{ESTER}(r) & \ge \frac{1}{\sqrt{r}}\big\|\left[ \matQ_f\ \matQ_l\right]\begin{bmatrix} \matI_r \\ -\matG(\matZ+\delta\matZ)\matG^{-1}\end{bmatrix} \big\|_F\\[0.3em]  & \ge \frac{1}{\sqrt{r}}\sigma_r\bigg(\begin{bmatrix} \matI_r \\ -\matG(\matZ+\delta\matZ)\matG^{-1}\end{bmatrix}\bigg)\sigma_{r+1}\big(\left[ \matQ_f\ \matQ_l\right]\big),
	\end{aligned}
	\label{Eq:vartheta2_1}
	\end{equation}
	donde la última desigualdad es debido a \cite[Teo. 2]{WANG1997}. También, usando la descomposición polar \cite{Horn1991}
	\[\left[\matQ_f\ \matQ_l\right] = \left[\hat{\matQ}_f\ \hat{\matQ}_l\right]\begin{bmatrix} \big(\matI_r-\u_f^H\u_f\big)^{\frac{1}{2}} & \mathbf{0} \\[0.3em] \mathbf{0} & \big(\matI_r-\u_l^H\u_l\big)^{\frac{1}{2}} 
	\end{bmatrix},\]
	por \cite{Horn1990} se obtiene
	\begin{equation}
	\sigma_{r+1}\big(\left[ \matQ_f\ \matQ_l\right]\big)\ge \Delta\sqrt{2}\sin\frac{\vartheta_1}{2}
	\label{Eq:sigma_r+1}
	\end{equation}
	con 
	\[\Delta = \min\big\{\sqrt{1-\|\u_f\|_2^2}, \sqrt{1-\|\u_l\|_2^2}\big\}. \quad \Delta\in(0,1).\]
	Además, usando el Teorema de Weyl \cite{Horn1991} se obtiene
	\begin{equation}
	\sigma_r\bigg(\begin{bmatrix} \matI_r \\ -\matG(\matZ+\delta\matZ)\matG^{-1}\end{bmatrix}\bigg)\ge \sqrt{1+\kappa^{-2}\big[\sigma_r(\matZ)-\sigma_1(\delta\matZ)\big]^2}
	\label{Eq:sigma_r}
	\end{equation}
	donde $\kappa$ es el número de condición de la matriz $\matG$. Luego, remplazando \eqref{Eq:sigma_r+1} y \eqref{Eq:sigma_r} en \eqref{Eq:vartheta2_1} y considerando  \eqref{eq:Jester_appendix}, se obtiene
	\[|\sigma_r(\matZ)-\sigma_1(\delta\matZ)|\le \kappa\sqrt{\frac{2r}{\Delta^2}\cos^2\frac{\vartheta}{2}-1}\]
	Luego, considerando que $\sigma_1(\delta\matZ) = \max_i|\delta z_i|$, $\sigma_r(\matZ) = \min_i|z_i|$, se obtiene
	\[\big|\min_i|z_i|-\max_i|\delta z_i|\big|\le \kappa\sqrt{\frac{2r}{\Delta^2}\cos^2\frac{\vartheta_1}{2}-1}.\]
	
	Finalmente, se obtiene que la perturbación del máximo autovalor esta acotada por
	\begin{equation}
	|\delta z_{max}|<|z_{min}| + \kappa\sqrt{\frac{2r}{\Delta^2}\cos^2\frac{\vartheta_1}{2}-1}.
	\label{bound1}
	\end{equation}
	Esta cota permite analizar la sensibilidad de los autovalores de una forma diferente a \eqref{Eq:CorollaryEstabilidad8}. A partir de \eqref{bound1}, es claro que desalineamiento entre los espacios columna de $\Hank_{\y,l}$ y $\Hank_{\y,f}$ contribuye a la perturbación del problema de autovalores generalizados.
	

	\section{Solución al problema de estabilidad numérica} 
		
		Como se demostró anteriormente, el desempeño de los algoritmos de estimación espectral se ve degradada  cuando las frecuencias están muy juntas entre sí. Para el caso de suma de exponenciales complejas no amortiguadas en \cite{BATENKOV2018} se realiza un estudio del condicionamiento numérico de las estimaciones y se propone una estrategia de decimación de la señal dado un solo grupo de frecuencias muy cercanas entre sí. Esta estrategia permite aumentar artificialmente la distancia entre frecuencias. En \cite{Morren2003} también se propone un esquema de decimación para luego resolver un problema de cuadrados mínimos totales. Sin embargo, en estos dos trabajos el factor de decimación se elige para asegurarse de que no se introduzca solapamiento (\emph{aliasing}).  Para superar esta restricción, la señal original debe ser sobremuestreada.
		
		En un ángulo diferente, en \cite{Bolt1979} se analiza el contenido espectral de series de tiempo geofísicas haciendo desplazamiento de frecuencia y posteriormente filtrando la señal observada. Al hacer esto, es posible identificar las frecuencias de resonancias contenidas en la señal observada.
		
		%Con base en los resultados obtenidos en el capítulo \ref{chap:ModeloSumExp}, se propone un estrategia para preprocesar la señal observada mediante un procesamiento multitasa. Esto tiene la capacidad de ``acomodar'' las frecuencias complejas para un cálculo más estable. Esta idea es validada por experimentos numéricos y se compara con los resultados obtenidos en publicaciones anteriores.
	
		\subsection{Estrategia \emph{Shift-and-Zoom}}
		
			Considerando que las muestras $y_k$ se obtienen después de muestrear uniformemente la versión en tiempo continuo 
			\begin{equation}
				y(t) = \sum_{i=1}^r c_ie^{\xi_i t} + w(t)
				\label{Eq:signalTimeCont}
			\end{equation}
			donde $\xi_i = \gamma_i+\jmath 2\pi\nu_i$. Se tiene que $y_k = y(kT_s)$, siendo $T_s$ es período de muestreo, y $z_i = e^{\xi T_s}$. Claramente, la ubicación de los diferentes $z_i$ en el plano complejo se puede controlar ajustando juiciosamente el tiempo de muestreo. Sin embargo, la elección del tiempo $T_s$ está sujeta a varias limitaciones al diseñar el sistema de adquisición de señales, y al agregar requisitos adicionales puede resultar que el diseño no sea factible. Sin embargo, la ubicación de los $z_i$ también puede cambiarse diezmando la señal observada \cite{Vaidyanathan1993}.
	
			Asumiendo que el contenido espectral de la señal observada está concentrado en el intervalo $\Upsilon = [\nu_{min},\nu_{max}]$, es decir que
			\begin{equation}
				z_i = e^{\gamma_iT_s}[\cos(2\pi\nu_iT_s)+\jmath\sin(2\pi\nu_iT_s)], \quad \nu_i\in\Upsilon,\ i=1,2,\ldots,r.
				\label{Eq:FrecInterval}
			\end{equation}
	
			Antes de realizar la estimación de los $z_i$ se quiere realizar un \emph{zoom} sobre $\Upsilon$ de manera de poder mejorar el rendimiento del algoritmo de estimación. Para ello se procede de la siguiente manera. Sea 
			\[\nu_c = \frac{\nu_{max}-\nu_{min}}{2},\]
			se define la siguiente operación
			\begin{equation}
				y_k^{bb} = y_ke^{-\jmath\nu_c kT_s}.
				\label{Eq:BaseBandSignal}
			\end{equation} 
			La señal $y_k^{bb}$ se conoce como el equivalente de banda base de $y_k$. A continuación, se considera un escalar $Q\in\R_{>0}$ tal que $\Upsilon Q\le \frac{2\pi}{T_s}$. Luego, usando el esquema de la Fig.~\ref{Fig:BlockDiagram2}, se construye la señal $y_k^{bb,D}$ después de filtrar y decimar la señal $y_k^{bb}$. Para evitar el efecto de \emph{aliasing}, la señal $y_k^{bb}$ es filtrada antes de realizar el submuestreo por un filtro de respuesta finita al impulso y fase lineal. Finalmente, la señal $y_k^{bb,D}$ es usada para estimar las frecuencias. Nos referimos a esta estrategia como \emph{Shift-and-zoom}.
	
			\begin{figure}[t]
				\centering
				%\includegraphics[width=0.5\textwidth]{filter.pdf}%
				\resizebox{\linewidth}{!}{\begin{tikzpicture}[cross/.style={path picture={ 
	\draw[black] (path picture bounding box.south east) -- (path picture bounding box.north west) (path picture bounding box.south west) -- (path picture bounding box.north east);
	}}]
  
	\node[rectangle, align=center] (fm) at (-2, 0) {$y_k$} ;
	
  	
  	
	\node[rectangle, align=center] (cos) at (-1, -0.9) {\\ $e^{-\jmath\nu_c}$};
	
	\node[circle, draw, cross, thick]  (mul1) at (-1, 0) {};
	\node[rectangle, draw, thick] (dwsample) at (2.5, 0) {$\downarrow Q$};
	\node[rectangle, draw, thick, align=center] (lp1) at (0.7, 0.0) {\tikz \draw[x=3.5ex, y=1ex, thick] (0, 0) sin (0.5, 0.5) cos (1, 0) sin (1.5, -0.5) cos (2, 0) (0.6, -0.5) -- (1.4, 0.5);\\ \tikz \draw[x=3.5ex, y=1ex, thick] (0, 0) sin (0.5, 0.5) cos (1, 0) sin (1.5, -0.5) cos (2, 0);};
	\node[rectangle, draw, thick] (alg) at (4.9, 0) {\scriptsize{$\begin{array}{ccc} \text{Algoritmo} \\ \text{Estimación} \\ \text{Espectral}\end{array}$}};
	\node[rectangle, align=center] (hatfreq) at (7.1, 0) {$\begin{aligned} & \hat{\nu}_i, \hat{\gamma}_i \\ &i = 1,\ldots, n\end{aligned}$};
	
	\draw[thick, -stealth] (cos) -- (mul1);
	\draw[thick, -stealth] (fm) -- (mul1);
	\draw[thick, -stealth] (mul1) --node[above]{$y^{bb}_k$} (lp1);
	\draw[thick, -stealth] (lp1) -- (dwsample);
	\draw[thick, -stealth] (dwsample) --node[above]{${y}_k^{bb,D}$} (alg);
	\draw[thick, -stealth] (alg) -- (hatfreq);

\end{tikzpicture}
}
				\caption{Esquema \emph{Shift-and-zoom}  para la estimación de frecuencias complejas}
				\label{Fig:BlockDiagram2}
			\end{figure}
	
			Usando el equivalente en banda base decimado $y_k^{bb,D}$, se estiman las frecuencias complejas usando los algoritmos de estimación espectral explicados en el capítulo \eqref{chap:ModeloSumExp}. Sean $z_i^{bb,D}$ las frecuencias estimadas obtenidas a partir de $y_k^{bb,D}$. Claramente
			\begin{equation}
				z_i = (z_i^{bb,D})^{\frac{1}{Q}}e^{\jmath\nu_cT_s}
				\label{Eq:EstimatedFrequencies}
			\end{equation}
	
			Al decimar o submuestrear la señal $y_k$ por un factor $Q$, las frecuencias complejas se mueven en el plano complejo. Para un $Q>1$, las frecuencias que están muy juntas en la región $\Upsilon$ pueden separarse, disminuyendo el valor de $\dot{z}_i$, haciendo que la estimación espectral sea más precisa. Es importante resaltar que el filtro pasa-bajos incluido en el esquema de \emph{Shift-and-Zoom} en la Fig.~\ref{Fig:BlockDiagram2} modifica las amplitudes asociadas a los modos complejas $z_i$ afectando el valor de la derivada de los $z_i$ en \eqref{Eq:Eigenvalue_derivative}.
	
			A continuación, se tiene que $\nu_i\in\cup_{l=1}^L\Upsilon_l$, para todo $i=1,\ldots,r$, donde $\Upsilon_l = [{\nu_l}_{min},{\nu_l}_{max}]$ son intervalos disjuntos. En este caso, los $z_i$ se separan en L grupos diferentes en el plano complejo. Para mejorar el rendimiento, el procedimiento descrito anteriormente puede repetirse para cada grupo $\Upsilon_l$, con $l=1,2,\ldots,L$. Además, dado que los intervalos $\Upsilon_l$ no necesariamente deben tener la misma longitud, el factor de decimación puede ser diferente para cada $l$. El procedimiento completo se muestra en la Fig.~\ref{Fig:BlockDiagram3}
	
			\begin{figure}[t]
				\centering
				%\includegraphics[width=0.5\textwidth]{filter_clusters.pdf} 
				\resizebox{\linewidth}{!}{\begin{tikzpicture}[cross/.style={path picture = { 	\draw[black] (path picture bounding box.south east) -- (path picture bounding box.north west) (path picture bounding box.south west) -- (path picture bounding box.north east);}}]
	
		\node[rectangle, align = center] (xt) at (-1, -1){};%{\\ $x = \sum_{i=1}^{25} x_{i}$};
		
		\node[circle, draw, cross, thick] (mul1) at (1, 1) {};
		\node[rectangle, align = center](cos1) at (1, -0.0){\\ $e^{-\jmath\nu_{c_1}}$};
		\node[rectangle, draw, thick, align = center] (lp1) at (2.3,1) {\tikz \draw[x=3.5ex, y=1ex, thick] (0, 0) sin (0.5, 0.5) cos (1, 0) sin (1.5, -0.5) cos (2, 0) (0.6, -0.5) -- (1.4, 0.5);\\ \tikz \draw[x=3.5ex, y=1ex, thick] (0, 0) sin (0.5, 0.5) cos (1, 0) sin (1.5, -0.5) cos (2, 0);};
		\node[rectangle, draw, thick, align = center] (dwsamp1) at (4.0, 1) {$\downarrow Q_1$};
		\node[rectangle, draw, thick] (alg1) at (6.5, 1) {\scriptsize{$\begin{array}{ccc} \text{Algoritmo} \\ \text{Estimación} \\ \text{Espectral}\end{array}$}};
		
		\node[rectangle, align = center] (hatfreq1) at (8.8,1) {$\hat{\gamma_i},\hat{\nu_i}\in\varUpsilon_1$};
		 
		
		\node[circle, draw, cross, thick] (mul2) at (1, -0.9)  {};
		\node[rectangle, align = center] (cos2) at (1, -1.9){\\ $e^{-\jmath\nu_{c_2}}$};
		\node[rectangle, draw, thick, align = center] (lp2) at (2.3,-0.9) {\tikz \draw[x=3.5ex, y=1ex, thick] (0, 0) sin (0.5, 0.5) cos (1, 0) sin (1.5, -0.5) cos (2, 0) (0.6, -0.5) -- (1.4, 0.5);\\ \tikz \draw[x=3.5ex, y=1ex, thick] (0, 0) sin (0.5, 0.5) cos (1, 0) sin (1.5, -0.5) cos (2, 0);};
		\node[rectangle, draw, thick, align = center] (dwsamp2) at (4.0, -0.9) {$\downarrow Q_2$};
		\node[rectangle, draw, thick] (alg2) at (6.5, -0.9){\scriptsize{$\begin{array}{ccc} \text{Algoritmo} \\ \text{Estimación} \\ \text{Espectral}\end{array}$}};
		\node[rectangle, align = center] (hatfreq2) at (8.8,-0.9) {$\hat{\gamma_i},\hat{\nu_i}\in\varUpsilon_2$};
		
		\node[draw = none] (ellipsis) at (2.5, -1.9) {\Huge{$\vdots$}};
		
		\node[circle, draw, cross, thick] (mul3) at (1, -3) {};
		\node[rectangle, align = center] (cos3) at (1, -4.0){\\ $e^{-\jmath\nu_{c_L}}$};
		\node[rectangle, draw, thick, align = center] (lp3) at (2.3,-3) {\tikz \draw[x=3.5ex, y=1ex, thick] (0, 0) sin (0.5, 0.5) cos (1, 0) sin (1.5, -0.5) cos (2, 0) (0.6, -0.5) -- (1.4, 0.5);\\ \tikz \draw[x=3.5ex, y=1ex, thick] (0, 0) sin (0.5, 0.5) cos (1, 0) sin (1.5, -0.5) cos (2, 0);};
		\node[rectangle, draw, thick, align = center] (dwsamp3) at (4.0, -3) {$\downarrow Q_L$};
		\node[rectangle, draw, thick] (algL) at (6.5, -3) {\scriptsize{$\begin{array}{ccc} \text{Algoritmo} \\ \text{Estimación} \\ \text{Espectral}\end{array}$}};
		\node[rectangle, align = center] (hatfreqL) at (8.8,-3) {$\hat{\gamma_i},\hat{\nu_i}\in\varUpsilon_L$}; 
		
%% CAMBIO		
%		\draw[thick, -stealth] (xt) -- node[above]{$\sum_{i=1}^{6}x_i(t) + w(t)$}(0, -1);
			\draw[thick, -stealth] (xt) -- node[above]{$y_k$}(0, -1);
		
		\draw[thick, -stealth] (0, -2) -- (0, 1) -- (mul1);
		%\draw[thick, -stealth] (0, 1) -- (mul1);
		\draw[thick, -stealth] (cos1) -- (mul1);
		\draw[thick, -stealth] (mul1) -- (lp1);
		\draw[thick, -stealth] (lp1) -- (dwsamp1);
		\draw[thick, -stealth] (dwsamp1) -- node[above]{${y_1}_k^{bb,D}$} (alg1);
		\draw[thick, -stealth] (alg1) -- (8.0, 1);
		
		\draw[thick, -stealth] (0, -0.9) -- (mul2);
		\draw[thick, -stealth] (cos2) -- (mul2);
		\draw[thick, -stealth] (mul2) -- (lp2);
		\draw[thick, -stealth] (lp2) -- (dwsamp2);
		\draw[thick, -stealth] (dwsamp2) -- node[above]{${y_2}_k^{bb,D}$} (alg2);
		\draw[thick, -stealth] (alg2) -- (8.0, -0.9);
				
		\draw[thick, -stealth] (0, -2) -- (0, -3) -- (mul3);
		%\draw[thick, -stealth] (0, -4) -- (mul3);
		\draw[thick, -stealth] (cos3) -- (mul3);
		\draw[thick, -stealth] (mul3) -- (lp3);
		\draw[thick, -stealth] (lp3) -- (dwsamp3);
		\draw[thick, -stealth] (dwsamp3) -- node[above]{${y_L}_k^{bb,D}$} (algL);
		\draw[thick, -stealth] (algL) -- (8.0, -3);
				
\end{tikzpicture}
}
				\caption{Diagrama en bloques para procesar la señal compuesta por $L$ conjuntos de frecuencias.}
				\label{Fig:BlockDiagram3}
			\end{figure}
	
	\section{Experimentos numéricos}
	
		Para evaluar el rendimiento de la estrategia \emph{Shift-and-Zoom} descrita en la sección anterior, se simularán dos ejemplos de suma de exponenciales complejas con diferente relación Señal a Ruido (SNR). En ambos casos, se estiman los modos complejos $\xi = \gamma_i+\jmath\nu_i$ y se comparan los resultados usando el procedimiento en presentado en \cite{Andersson2014}. De modo de valorar el desempeño de ambos algoritmo, se repetirán los experimentos $K$ veces y se calculará los errores cuadrático medio (RMSE) para las frecuencias y los factores de amortiguamiento estimados.
		\begin{equation}
			\begin{aligned}
				& \hat{\sigma}_{\nu} = \sqrt{\frac{1}{K}\sum_{k=1}^{K}\sum_{i=1}^{r}(\nu_i - \hat{\nu}_{i_k})^2} \\[0.3em]
				& \hat{\sigma}_{\gamma} = \sqrt{\frac{1}{K}\sum_{k=1}^{K}\sum_{i=1}^{r}(\gamma_i-\hat{\gamma}_{i_k})^2}
			\end{aligned}
			\label{Eq:RMSE}	
		\end{equation}
		siendo $\hat{\nu}_{i_k}$ y $\hat{\gamma}_{i_k}$ son las estimaciones obtenidas luego de $k-$ésima corrida en la simulación de Monte Carlo.

        \subsection{Cota Inferior de Cramér-Rao}\label{App:CRB}
		
			Tomando $N$ muestras de la señal \eqref{Eq:signalTimeCont}, con $c_i = \eta_ie^{\jmath\theta_i}$, $z_i = e^{(\gamma_i+\jmath2\pi\omega_i)T_s}$. Se asume que el ruido $w_k$ es un proceso Gaussiano complejo, circularmente simétrico, idénticamente distribuido, con parte real e imaginaria no correlacionadas de media cero con varianza $\sigma_w^2$.
			
			Siguiendo \cite{Yao1995}, la función de verosimilitud logarítmica se puede expresar como
			\begin{equation}
				\L(\y\mid \vecalpha) = -N\log(2\pi) - N\log(\sigma_w^2) - \frac{(\y - \x)(\y-\x)^H}{\sigma_w^2}
				\label{Eq:LogLikelihood}
			\end{equation}
			donde $\vecalpha = [\nu_1,\ldots,\nu_r,\gamma_i,\ldots,\gamma_r,\eta_1,\ldots,\eta_r,\theta_1,\ldots,\theta_r]^T\in\R^{4r}$. 
			
			La cota de Cramér-Rao de las estimaciones de los parámetros en el vector $\vecalpha$ forman los elementos en la diagonal de la inversa de la matriz de información de Fisher.
			\begin{equation}
				\I(\vecalpha) = \E\bigg[\frac{\partial\L(\y\mid\vecalpha)}{\partial\vecalpha}\bigg(\frac{\partial\L(\y\mid\vecalpha)}{\partial\vecalpha}\bigg)^T\bigg]
				\label{Eq:FisherInformationMatrix}
			\end{equation}
			
			Luego,
			
			\begin{equation}
				\I(\vecalpha) =\frac{1}{ \sigma_w^2}\bigg(2\Re\bigg[\frac{\partial\x}{\partial\vecalpha}\bigg(\frac{\partial\x}{\partial\vecalpha}\bigg)^H\bigg]\bigg).
				\label{Eq:FisherInformationMatrix1}
			\end{equation}
			
			Calculando la derivada, 
			\begin{equation}
				\begin{aligned} \frac{\partial\x}{\partial\vecalpha} = \big[& \jmath kT_sc_1z_1^k,\ldots,\jmath kT_sc_rz_r^k,kT_sc_1z_1^k,\ldots,kT_sc_rz_r^k,e^{\jmath\theta_1}z_1^k,\ldots,e^{\jmath\theta_r}z_r^k,\\[0.3em]
				& \jmath c_1z_1^k,\ldots,\jmath c_rz_r^k\big] \quad k=0,1,\ldots,N-1. \end{aligned}
				\label{Eq:Partialderivatives}
			\end{equation}
			se puede reescribir como
			\begin{equation}
				\frac{\partial\x}{\partial\vecalpha} = \diag\begin{bmatrix} \Lambdab & \Lambdab & \matI & \Lambdab			\end{bmatrix}\begin{bmatrix}\jmath\Thetab\matT\matZ_{N-1}^T\\[0.3em] \Thetab\matT\matZ_{N-1}^T\\[0.3em] \Thetab\matZ_{N-1}^T \\[0.3em] \jmath\Thetab\matZ_{N-1}^T
				\end{bmatrix} = \matS\matZ
			\end{equation}
			donde $\matZ_{N-1}$ es la matriz de Vandermonde definida en \eqref{Eq:VandermondeMatrix} y
			\[\begin{aligned} & \matT =T_s\diag \begin{bmatrix}0 & 1 & \cdots & N-1
			\end{bmatrix}\\[0.3em] 
			& \Lambdab = \diag\begin{bmatrix} \eta_1 & \eta_2 & \cdots & \eta_r 	\end{bmatrix}\\[0.3em]
			& \Thetab = \diag\begin{bmatrix}e^{\jmath\theta_1} & \cdots & e^{\jmath\theta_r}
			\end{bmatrix}.\end{aligned}\]
			
			Luego, la inversa de la matriz de Información de Fisher se puede escribir como
			\begin{equation}
				\I^{-1}(\vecalpha) = \sigma_w^2\matS^{-1}\big[2\Re(\matZ\matZ^H)\big]^{-1}\matS^{-1} = \sigma_w^2\matS^{-1}\tilde{\matQ}\matS^{-1}.
				\label{Eq:CRLB1}
			\end{equation}
			
			Por lo tanto, se obtiene las cotas de Cramér-Rao como
			
			\begin{equation}
				\mathrm{CRLB}(\nu_i) = \frac{\tilde{\matQ}_{ii}}{\mathrm{snr}_i}\quad i=1,\ldots,r
				\label{Eq:CRLB_nu}
			\end{equation}
			
			\begin{equation}
				\mathrm{CRLB}(\gamma_i) = \frac{\tilde{\matQ}_{(i+r)(i+r)}}{\mathrm{snr}_i}\quad i=1,\ldots,r
				\label{Eq:CRLB_gamma}
			\end{equation}
			
			\begin{equation}
				\mathrm{CRLB}(\eta_i) = \frac{\tilde{\matQ}_{(i+2r)(i+2r)}\eta_i^2}{\mathrm{snr}_i}\quad i=1,\ldots,r
				\label{Eq:CRLB_eta}
			\end{equation}
			
			\begin{equation}
				\mathrm{CRLB}(\theta_i) = \frac{\tilde{\matQ}_{(i+3r)(i+3r)}}{\mathrm{snr}_i}\quad i=1,\ldots,r
				\label{Eq:CRLB_theta}
			\end{equation}
			donde $\mathrm{snr}_i = \eta_i^2/\sigma_w^2$.
			
		
		\subsection{Modos débiles}
			
			Como primer ejemplo, se toma la combinación lineal de 4 modos descritos en la tabla \ref{Table:Anderson}. Este ejemplo se tomó de \cite{Andersson2014}. 
			
			\begin{table}[h!]
				\centering
				\begin{tabular}{lllll}
					$i$          & 1      & 2      & 3     & 4      \\ \hline 
					$\nu_i$      & -7.68  & 39.68  & 40.96 & 99.84  \\ 
					$\gamma_i$   & -0.274 & -0.150 & 0.133 & -0.221 \\ 
					$|c_i|$      & 0.4    & 1.2    & 1.0   & 0.9    \\ 
					$\angle c_i$ & -0.93  & -1.55  & -0.83 & 0.07   \\ 
					\hline
				\end{tabular}
				\caption{Parámetros para la suma de exponenciales \cite{Andersson2014}}
			\label{Table:Anderson}
			\end{table}
			
			Para simular las muestras $y_k$ se usó un periodo de muestreo $T_s = 0.0039$ segundos. Para usar la estrategia \emph{Shift-and-Zoom}, se consideran tres intervalos de frecuencia disjuntos, $\Upsilon_1 = [0\,\,\, 15.39]Hz$, $\Upsilon_2 = [32.62\,\,\, 48.01]Hz$, y $\Upsilon_3 = [92.14\,\,\, 107.54]Hz$. Es claro que, $\nu_1\in \Upsilon_1$, $\nu_2, \nu_3 \in \Upsilon_2$, y $\nu_4\in \Upsilon_3$.
			
			Para este ejemplo, se asume que la ubicación de los distintos clusters se conoce a-priori. Por el contrario, si esta información no está disponible, se puede realizar una corrida preliminar del algoritmo de estimación para obtener estimaciones aproximadas de las bandas de frecuencias.
			
			Los tres intervalos definidos tiene el mismo ancho de banda. Aunque, ésta no es una condición necesaria para el algoritmo, simplifica su implementación. Usando este hecho, se usa un único filtro pasa-bajos y coeficiente de decimación para los tres intervalos de frecuencia definidos. En particular, se diseña un filtro FIR de fase lineal con una ventana rectangular de 16 coeficientes, con ancho de banda de 15.39Hz. El coeficiente de submuestreo es $Q=4$. Tomando $K-200$ se calculan $\hat{\sigma}_{\nu}$ y $\hat{\sigma}_{\gamma}$ para cada SNR. Los resultados se muestran en la Figura \ref{Fig:RMSE_ex1}.
	
			\begin{figure}[t]
				%\centering
				\begin{subfigure}{0.5\textwidth}
					\centering	
					\resizebox{\linewidth}{!}{%% Creator: Matplotlib, PGF backend
%%
%% To include the figure in your LaTeX document, write
%%   \input{<filename>.pgf}
%%
%% Make sure the required packages are loaded in your preamble
%%   \usepackage{pgf}
%%
%% and, on pdftex
%%   \usepackage[utf8]{inputenc}\DeclareUnicodeCharacter{2212}{-}
%%
%% or, on luatex and xetex
%%   \usepackage{unicode-math}
%%
%% Figures using additional raster images can only be included by \input if
%% they are in the same directory as the main LaTeX file. For loading figures
%% from other directories you can use the `import` package
%%   \usepackage{import}
%%
%% and then include the figures with
%%   \import{<path to file>}{<filename>.pgf}
%%
%% Matplotlib used the following preamble
%%   \usepackage[utf8x]{inputenc}
%%   \usepackage[T1]{fontenc}
%%   \usepackage{amsmath,amssymb,amsfonts}
%%
\begingroup%
\makeatletter%
\begin{pgfpicture}%
\pgfpathrectangle{\pgfpointorigin}{\pgfqpoint{4.136389in}{2.495314in}}%
\pgfusepath{use as bounding box, clip}%
\begin{pgfscope}%
\pgfsetbuttcap%
\pgfsetmiterjoin%
\definecolor{currentfill}{rgb}{1.000000,1.000000,1.000000}%
\pgfsetfillcolor{currentfill}%
\pgfsetlinewidth{0.000000pt}%
\definecolor{currentstroke}{rgb}{1.000000,1.000000,1.000000}%
\pgfsetstrokecolor{currentstroke}%
\pgfsetdash{}{0pt}%
\pgfpathmoveto{\pgfqpoint{-0.000000in}{0.000000in}}%
\pgfpathlineto{\pgfqpoint{4.136389in}{0.000000in}}%
\pgfpathlineto{\pgfqpoint{4.136389in}{2.495314in}}%
\pgfpathlineto{\pgfqpoint{-0.000000in}{2.495314in}}%
\pgfpathclose%
\pgfusepath{fill}%
\end{pgfscope}%
\begin{pgfscope}%
\pgfsetbuttcap%
\pgfsetmiterjoin%
\definecolor{currentfill}{rgb}{1.000000,1.000000,1.000000}%
\pgfsetfillcolor{currentfill}%
\pgfsetlinewidth{0.000000pt}%
\definecolor{currentstroke}{rgb}{0.000000,0.000000,0.000000}%
\pgfsetstrokecolor{currentstroke}%
\pgfsetstrokeopacity{0.000000}%
\pgfsetdash{}{0pt}%
\pgfpathmoveto{\pgfqpoint{0.740433in}{0.566590in}}%
\pgfpathlineto{\pgfqpoint{4.036389in}{0.566590in}}%
\pgfpathlineto{\pgfqpoint{4.036389in}{2.395314in}}%
\pgfpathlineto{\pgfqpoint{0.740433in}{2.395314in}}%
\pgfpathclose%
\pgfusepath{fill}%
\end{pgfscope}%
\begin{pgfscope}%
\pgfpathrectangle{\pgfqpoint{0.740433in}{0.566590in}}{\pgfqpoint{3.295956in}{1.828724in}}%
\pgfusepath{clip}%
\pgfsetrectcap%
\pgfsetroundjoin%
\pgfsetlinewidth{0.803000pt}%
\definecolor{currentstroke}{rgb}{0.690196,0.690196,0.690196}%
\pgfsetstrokecolor{currentstroke}%
\pgfsetdash{}{0pt}%
\pgfpathmoveto{\pgfqpoint{0.740433in}{0.566590in}}%
\pgfpathlineto{\pgfqpoint{0.740433in}{2.395314in}}%
\pgfusepath{stroke}%
\end{pgfscope}%
\begin{pgfscope}%
\pgfsetbuttcap%
\pgfsetroundjoin%
\definecolor{currentfill}{rgb}{0.000000,0.000000,0.000000}%
\pgfsetfillcolor{currentfill}%
\pgfsetlinewidth{0.803000pt}%
\definecolor{currentstroke}{rgb}{0.000000,0.000000,0.000000}%
\pgfsetstrokecolor{currentstroke}%
\pgfsetdash{}{0pt}%
\pgfsys@defobject{currentmarker}{\pgfqpoint{0.000000in}{-0.048611in}}{\pgfqpoint{0.000000in}{0.000000in}}{%
\pgfpathmoveto{\pgfqpoint{0.000000in}{0.000000in}}%
\pgfpathlineto{\pgfqpoint{0.000000in}{-0.048611in}}%
\pgfusepath{stroke,fill}%
}%
\begin{pgfscope}%
\pgfsys@transformshift{0.740433in}{0.566590in}%
\pgfsys@useobject{currentmarker}{}%
\end{pgfscope}%
\end{pgfscope}%
\begin{pgfscope}%
\definecolor{textcolor}{rgb}{0.000000,0.000000,0.000000}%
\pgfsetstrokecolor{textcolor}%
\pgfsetfillcolor{textcolor}%
\pgftext[x=0.740433in,y=0.469368in,,top]{\color{textcolor}\rmfamily\fontsize{12.000000}{14.400000}\selectfont \(\displaystyle {-10}\)}%
\end{pgfscope}%
\begin{pgfscope}%
\pgfpathrectangle{\pgfqpoint{0.740433in}{0.566590in}}{\pgfqpoint{3.295956in}{1.828724in}}%
\pgfusepath{clip}%
\pgfsetrectcap%
\pgfsetroundjoin%
\pgfsetlinewidth{0.803000pt}%
\definecolor{currentstroke}{rgb}{0.690196,0.690196,0.690196}%
\pgfsetstrokecolor{currentstroke}%
\pgfsetdash{}{0pt}%
\pgfpathmoveto{\pgfqpoint{1.682134in}{0.566590in}}%
\pgfpathlineto{\pgfqpoint{1.682134in}{2.395314in}}%
\pgfusepath{stroke}%
\end{pgfscope}%
\begin{pgfscope}%
\pgfsetbuttcap%
\pgfsetroundjoin%
\definecolor{currentfill}{rgb}{0.000000,0.000000,0.000000}%
\pgfsetfillcolor{currentfill}%
\pgfsetlinewidth{0.803000pt}%
\definecolor{currentstroke}{rgb}{0.000000,0.000000,0.000000}%
\pgfsetstrokecolor{currentstroke}%
\pgfsetdash{}{0pt}%
\pgfsys@defobject{currentmarker}{\pgfqpoint{0.000000in}{-0.048611in}}{\pgfqpoint{0.000000in}{0.000000in}}{%
\pgfpathmoveto{\pgfqpoint{0.000000in}{0.000000in}}%
\pgfpathlineto{\pgfqpoint{0.000000in}{-0.048611in}}%
\pgfusepath{stroke,fill}%
}%
\begin{pgfscope}%
\pgfsys@transformshift{1.682134in}{0.566590in}%
\pgfsys@useobject{currentmarker}{}%
\end{pgfscope}%
\end{pgfscope}%
\begin{pgfscope}%
\definecolor{textcolor}{rgb}{0.000000,0.000000,0.000000}%
\pgfsetstrokecolor{textcolor}%
\pgfsetfillcolor{textcolor}%
\pgftext[x=1.682134in,y=0.469368in,,top]{\color{textcolor}\rmfamily\fontsize{12.000000}{14.400000}\selectfont \(\displaystyle {0}\)}%
\end{pgfscope}%
\begin{pgfscope}%
\pgfpathrectangle{\pgfqpoint{0.740433in}{0.566590in}}{\pgfqpoint{3.295956in}{1.828724in}}%
\pgfusepath{clip}%
\pgfsetrectcap%
\pgfsetroundjoin%
\pgfsetlinewidth{0.803000pt}%
\definecolor{currentstroke}{rgb}{0.690196,0.690196,0.690196}%
\pgfsetstrokecolor{currentstroke}%
\pgfsetdash{}{0pt}%
\pgfpathmoveto{\pgfqpoint{2.623836in}{0.566590in}}%
\pgfpathlineto{\pgfqpoint{2.623836in}{2.395314in}}%
\pgfusepath{stroke}%
\end{pgfscope}%
\begin{pgfscope}%
\pgfsetbuttcap%
\pgfsetroundjoin%
\definecolor{currentfill}{rgb}{0.000000,0.000000,0.000000}%
\pgfsetfillcolor{currentfill}%
\pgfsetlinewidth{0.803000pt}%
\definecolor{currentstroke}{rgb}{0.000000,0.000000,0.000000}%
\pgfsetstrokecolor{currentstroke}%
\pgfsetdash{}{0pt}%
\pgfsys@defobject{currentmarker}{\pgfqpoint{0.000000in}{-0.048611in}}{\pgfqpoint{0.000000in}{0.000000in}}{%
\pgfpathmoveto{\pgfqpoint{0.000000in}{0.000000in}}%
\pgfpathlineto{\pgfqpoint{0.000000in}{-0.048611in}}%
\pgfusepath{stroke,fill}%
}%
\begin{pgfscope}%
\pgfsys@transformshift{2.623836in}{0.566590in}%
\pgfsys@useobject{currentmarker}{}%
\end{pgfscope}%
\end{pgfscope}%
\begin{pgfscope}%
\definecolor{textcolor}{rgb}{0.000000,0.000000,0.000000}%
\pgfsetstrokecolor{textcolor}%
\pgfsetfillcolor{textcolor}%
\pgftext[x=2.623836in,y=0.469368in,,top]{\color{textcolor}\rmfamily\fontsize{12.000000}{14.400000}\selectfont \(\displaystyle {10}\)}%
\end{pgfscope}%
\begin{pgfscope}%
\pgfpathrectangle{\pgfqpoint{0.740433in}{0.566590in}}{\pgfqpoint{3.295956in}{1.828724in}}%
\pgfusepath{clip}%
\pgfsetrectcap%
\pgfsetroundjoin%
\pgfsetlinewidth{0.803000pt}%
\definecolor{currentstroke}{rgb}{0.690196,0.690196,0.690196}%
\pgfsetstrokecolor{currentstroke}%
\pgfsetdash{}{0pt}%
\pgfpathmoveto{\pgfqpoint{3.565538in}{0.566590in}}%
\pgfpathlineto{\pgfqpoint{3.565538in}{2.395314in}}%
\pgfusepath{stroke}%
\end{pgfscope}%
\begin{pgfscope}%
\pgfsetbuttcap%
\pgfsetroundjoin%
\definecolor{currentfill}{rgb}{0.000000,0.000000,0.000000}%
\pgfsetfillcolor{currentfill}%
\pgfsetlinewidth{0.803000pt}%
\definecolor{currentstroke}{rgb}{0.000000,0.000000,0.000000}%
\pgfsetstrokecolor{currentstroke}%
\pgfsetdash{}{0pt}%
\pgfsys@defobject{currentmarker}{\pgfqpoint{0.000000in}{-0.048611in}}{\pgfqpoint{0.000000in}{0.000000in}}{%
\pgfpathmoveto{\pgfqpoint{0.000000in}{0.000000in}}%
\pgfpathlineto{\pgfqpoint{0.000000in}{-0.048611in}}%
\pgfusepath{stroke,fill}%
}%
\begin{pgfscope}%
\pgfsys@transformshift{3.565538in}{0.566590in}%
\pgfsys@useobject{currentmarker}{}%
\end{pgfscope}%
\end{pgfscope}%
\begin{pgfscope}%
\definecolor{textcolor}{rgb}{0.000000,0.000000,0.000000}%
\pgfsetstrokecolor{textcolor}%
\pgfsetfillcolor{textcolor}%
\pgftext[x=3.565538in,y=0.469368in,,top]{\color{textcolor}\rmfamily\fontsize{12.000000}{14.400000}\selectfont \(\displaystyle {20}\)}%
\end{pgfscope}%
\begin{pgfscope}%
\definecolor{textcolor}{rgb}{0.000000,0.000000,0.000000}%
\pgfsetstrokecolor{textcolor}%
\pgfsetfillcolor{textcolor}%
\pgftext[x=2.388411in,y=0.266626in,,top]{\color{textcolor}\rmfamily\fontsize{12.000000}{14.400000}\selectfont SNR [dB]}%
\end{pgfscope}%
\begin{pgfscope}%
\pgfpathrectangle{\pgfqpoint{0.740433in}{0.566590in}}{\pgfqpoint{3.295956in}{1.828724in}}%
\pgfusepath{clip}%
\pgfsetrectcap%
\pgfsetroundjoin%
\pgfsetlinewidth{0.803000pt}%
\definecolor{currentstroke}{rgb}{0.690196,0.690196,0.690196}%
\pgfsetstrokecolor{currentstroke}%
\pgfsetdash{}{0pt}%
\pgfpathmoveto{\pgfqpoint{0.740433in}{1.011993in}}%
\pgfpathlineto{\pgfqpoint{4.036389in}{1.011993in}}%
\pgfusepath{stroke}%
\end{pgfscope}%
\begin{pgfscope}%
\pgfsetbuttcap%
\pgfsetroundjoin%
\definecolor{currentfill}{rgb}{0.000000,0.000000,0.000000}%
\pgfsetfillcolor{currentfill}%
\pgfsetlinewidth{0.803000pt}%
\definecolor{currentstroke}{rgb}{0.000000,0.000000,0.000000}%
\pgfsetstrokecolor{currentstroke}%
\pgfsetdash{}{0pt}%
\pgfsys@defobject{currentmarker}{\pgfqpoint{-0.048611in}{0.000000in}}{\pgfqpoint{-0.000000in}{0.000000in}}{%
\pgfpathmoveto{\pgfqpoint{-0.000000in}{0.000000in}}%
\pgfpathlineto{\pgfqpoint{-0.048611in}{0.000000in}}%
\pgfusepath{stroke,fill}%
}%
\begin{pgfscope}%
\pgfsys@transformshift{0.740433in}{1.011993in}%
\pgfsys@useobject{currentmarker}{}%
\end{pgfscope}%
\end{pgfscope}%
\begin{pgfscope}%
\definecolor{textcolor}{rgb}{0.000000,0.000000,0.000000}%
\pgfsetstrokecolor{textcolor}%
\pgfsetfillcolor{textcolor}%
\pgftext[x=0.322222in, y=0.954600in, left, base]{\color{textcolor}\rmfamily\fontsize{12.000000}{14.400000}\selectfont \(\displaystyle {10^{-2}}\)}%
\end{pgfscope}%
\begin{pgfscope}%
\pgfpathrectangle{\pgfqpoint{0.740433in}{0.566590in}}{\pgfqpoint{3.295956in}{1.828724in}}%
\pgfusepath{clip}%
\pgfsetrectcap%
\pgfsetroundjoin%
\pgfsetlinewidth{0.803000pt}%
\definecolor{currentstroke}{rgb}{0.690196,0.690196,0.690196}%
\pgfsetstrokecolor{currentstroke}%
\pgfsetdash{}{0pt}%
\pgfpathmoveto{\pgfqpoint{0.740433in}{1.660062in}}%
\pgfpathlineto{\pgfqpoint{4.036389in}{1.660062in}}%
\pgfusepath{stroke}%
\end{pgfscope}%
\begin{pgfscope}%
\pgfsetbuttcap%
\pgfsetroundjoin%
\definecolor{currentfill}{rgb}{0.000000,0.000000,0.000000}%
\pgfsetfillcolor{currentfill}%
\pgfsetlinewidth{0.803000pt}%
\definecolor{currentstroke}{rgb}{0.000000,0.000000,0.000000}%
\pgfsetstrokecolor{currentstroke}%
\pgfsetdash{}{0pt}%
\pgfsys@defobject{currentmarker}{\pgfqpoint{-0.048611in}{0.000000in}}{\pgfqpoint{-0.000000in}{0.000000in}}{%
\pgfpathmoveto{\pgfqpoint{-0.000000in}{0.000000in}}%
\pgfpathlineto{\pgfqpoint{-0.048611in}{0.000000in}}%
\pgfusepath{stroke,fill}%
}%
\begin{pgfscope}%
\pgfsys@transformshift{0.740433in}{1.660062in}%
\pgfsys@useobject{currentmarker}{}%
\end{pgfscope}%
\end{pgfscope}%
\begin{pgfscope}%
\definecolor{textcolor}{rgb}{0.000000,0.000000,0.000000}%
\pgfsetstrokecolor{textcolor}%
\pgfsetfillcolor{textcolor}%
\pgftext[x=0.414045in, y=1.602669in, left, base]{\color{textcolor}\rmfamily\fontsize{12.000000}{14.400000}\selectfont \(\displaystyle {10^{0}}\)}%
\end{pgfscope}%
\begin{pgfscope}%
\pgfpathrectangle{\pgfqpoint{0.740433in}{0.566590in}}{\pgfqpoint{3.295956in}{1.828724in}}%
\pgfusepath{clip}%
\pgfsetrectcap%
\pgfsetroundjoin%
\pgfsetlinewidth{0.803000pt}%
\definecolor{currentstroke}{rgb}{0.690196,0.690196,0.690196}%
\pgfsetstrokecolor{currentstroke}%
\pgfsetdash{}{0pt}%
\pgfpathmoveto{\pgfqpoint{0.740433in}{2.308131in}}%
\pgfpathlineto{\pgfqpoint{4.036389in}{2.308131in}}%
\pgfusepath{stroke}%
\end{pgfscope}%
\begin{pgfscope}%
\pgfsetbuttcap%
\pgfsetroundjoin%
\definecolor{currentfill}{rgb}{0.000000,0.000000,0.000000}%
\pgfsetfillcolor{currentfill}%
\pgfsetlinewidth{0.803000pt}%
\definecolor{currentstroke}{rgb}{0.000000,0.000000,0.000000}%
\pgfsetstrokecolor{currentstroke}%
\pgfsetdash{}{0pt}%
\pgfsys@defobject{currentmarker}{\pgfqpoint{-0.048611in}{0.000000in}}{\pgfqpoint{-0.000000in}{0.000000in}}{%
\pgfpathmoveto{\pgfqpoint{-0.000000in}{0.000000in}}%
\pgfpathlineto{\pgfqpoint{-0.048611in}{0.000000in}}%
\pgfusepath{stroke,fill}%
}%
\begin{pgfscope}%
\pgfsys@transformshift{0.740433in}{2.308131in}%
\pgfsys@useobject{currentmarker}{}%
\end{pgfscope}%
\end{pgfscope}%
\begin{pgfscope}%
\definecolor{textcolor}{rgb}{0.000000,0.000000,0.000000}%
\pgfsetstrokecolor{textcolor}%
\pgfsetfillcolor{textcolor}%
\pgftext[x=0.414045in, y=2.250738in, left, base]{\color{textcolor}\rmfamily\fontsize{12.000000}{14.400000}\selectfont \(\displaystyle {10^{2}}\)}%
\end{pgfscope}%
\begin{pgfscope}%
\definecolor{textcolor}{rgb}{0.000000,0.000000,0.000000}%
\pgfsetstrokecolor{textcolor}%
\pgfsetfillcolor{textcolor}%
\pgftext[x=0.266667in,y=1.480952in,,bottom,rotate=90.000000]{\color{textcolor}\rmfamily\fontsize{12.000000}{14.400000}\selectfont \(\displaystyle \hat{\sigma}_{\nu}(\mathrm{SNR})\)}%
\end{pgfscope}%
\begin{pgfscope}%
\pgfpathrectangle{\pgfqpoint{0.740433in}{0.566590in}}{\pgfqpoint{3.295956in}{1.828724in}}%
\pgfusepath{clip}%
\pgfsetbuttcap%
\pgfsetroundjoin%
\pgfsetlinewidth{1.505625pt}%
\definecolor{currentstroke}{rgb}{0.000000,0.447000,0.741000}%
\pgfsetstrokecolor{currentstroke}%
\pgfsetdash{{5.550000pt}{2.400000pt}}{0.000000pt}%
\pgfpathmoveto{\pgfqpoint{0.740433in}{2.162226in}}%
\pgfpathlineto{\pgfqpoint{0.791932in}{2.165514in}}%
\pgfpathlineto{\pgfqpoint{0.843431in}{2.165065in}}%
\pgfpathlineto{\pgfqpoint{0.894931in}{2.174326in}}%
\pgfpathlineto{\pgfqpoint{0.946430in}{2.170150in}}%
\pgfpathlineto{\pgfqpoint{0.997929in}{2.169649in}}%
\pgfpathlineto{\pgfqpoint{1.049429in}{2.170402in}}%
\pgfpathlineto{\pgfqpoint{1.100928in}{2.164602in}}%
\pgfpathlineto{\pgfqpoint{1.152427in}{2.172352in}}%
\pgfpathlineto{\pgfqpoint{1.203926in}{2.165266in}}%
\pgfpathlineto{\pgfqpoint{1.255426in}{2.151594in}}%
\pgfpathlineto{\pgfqpoint{1.306925in}{2.159374in}}%
\pgfpathlineto{\pgfqpoint{1.358424in}{2.139163in}}%
\pgfpathlineto{\pgfqpoint{1.409924in}{2.135724in}}%
\pgfpathlineto{\pgfqpoint{1.461423in}{2.115931in}}%
\pgfpathlineto{\pgfqpoint{1.512922in}{2.102634in}}%
\pgfpathlineto{\pgfqpoint{1.564422in}{2.068164in}}%
\pgfpathlineto{\pgfqpoint{1.615921in}{2.027186in}}%
\pgfpathlineto{\pgfqpoint{1.667420in}{2.028740in}}%
\pgfpathlineto{\pgfqpoint{1.718920in}{1.969492in}}%
\pgfpathlineto{\pgfqpoint{1.770419in}{1.929454in}}%
\pgfpathlineto{\pgfqpoint{1.821918in}{1.354060in}}%
\pgfpathlineto{\pgfqpoint{1.873417in}{1.835975in}}%
\pgfpathlineto{\pgfqpoint{1.924917in}{1.830283in}}%
\pgfpathlineto{\pgfqpoint{1.976416in}{1.324017in}}%
\pgfpathlineto{\pgfqpoint{2.027915in}{1.329980in}}%
\pgfpathlineto{\pgfqpoint{2.079415in}{1.293535in}}%
\pgfpathlineto{\pgfqpoint{2.130914in}{1.304884in}}%
\pgfpathlineto{\pgfqpoint{2.182413in}{1.277455in}}%
\pgfpathlineto{\pgfqpoint{2.233913in}{1.261650in}}%
\pgfpathlineto{\pgfqpoint{2.285412in}{1.264149in}}%
\pgfpathlineto{\pgfqpoint{2.336911in}{1.255331in}}%
\pgfpathlineto{\pgfqpoint{2.388411in}{1.261157in}}%
\pgfpathlineto{\pgfqpoint{2.439910in}{1.236574in}}%
\pgfpathlineto{\pgfqpoint{2.491409in}{1.222071in}}%
\pgfpathlineto{\pgfqpoint{2.542909in}{1.227555in}}%
\pgfpathlineto{\pgfqpoint{2.594408in}{1.203736in}}%
\pgfpathlineto{\pgfqpoint{2.645907in}{1.206128in}}%
\pgfpathlineto{\pgfqpoint{2.697406in}{1.195786in}}%
\pgfpathlineto{\pgfqpoint{2.748906in}{1.185994in}}%
\pgfpathlineto{\pgfqpoint{2.800405in}{1.167743in}}%
\pgfpathlineto{\pgfqpoint{2.851904in}{1.149869in}}%
\pgfpathlineto{\pgfqpoint{2.903404in}{1.142813in}}%
\pgfpathlineto{\pgfqpoint{2.954903in}{1.134572in}}%
\pgfpathlineto{\pgfqpoint{3.006402in}{1.135846in}}%
\pgfpathlineto{\pgfqpoint{3.057902in}{1.133227in}}%
\pgfpathlineto{\pgfqpoint{3.109401in}{1.124380in}}%
\pgfpathlineto{\pgfqpoint{3.160900in}{1.111411in}}%
\pgfpathlineto{\pgfqpoint{3.212400in}{1.097951in}}%
\pgfpathlineto{\pgfqpoint{3.263899in}{1.101898in}}%
\pgfpathlineto{\pgfqpoint{3.315398in}{1.077790in}}%
\pgfpathlineto{\pgfqpoint{3.366897in}{1.069797in}}%
\pgfpathlineto{\pgfqpoint{3.418397in}{1.055154in}}%
\pgfpathlineto{\pgfqpoint{3.469896in}{1.056606in}}%
\pgfpathlineto{\pgfqpoint{3.521395in}{1.053497in}}%
\pgfpathlineto{\pgfqpoint{3.572895in}{1.035085in}}%
\pgfpathlineto{\pgfqpoint{3.624394in}{1.032282in}}%
\pgfpathlineto{\pgfqpoint{3.675893in}{1.023886in}}%
\pgfpathlineto{\pgfqpoint{3.727393in}{1.018114in}}%
\pgfpathlineto{\pgfqpoint{3.778892in}{0.999516in}}%
\pgfpathlineto{\pgfqpoint{3.830391in}{1.004743in}}%
\pgfpathlineto{\pgfqpoint{3.881891in}{0.993577in}}%
\pgfpathlineto{\pgfqpoint{3.933390in}{0.981748in}}%
\pgfpathlineto{\pgfqpoint{3.984889in}{0.978206in}}%
\pgfpathlineto{\pgfqpoint{4.036389in}{0.948966in}}%
\pgfusepath{stroke}%
\end{pgfscope}%
\begin{pgfscope}%
\pgfpathrectangle{\pgfqpoint{0.740433in}{0.566590in}}{\pgfqpoint{3.295956in}{1.828724in}}%
\pgfusepath{clip}%
\pgfsetbuttcap%
\pgfsetroundjoin%
\definecolor{currentfill}{rgb}{0.000000,0.000000,0.000000}%
\pgfsetfillcolor{currentfill}%
\pgfsetfillopacity{0.000000}%
\pgfsetlinewidth{1.003750pt}%
\definecolor{currentstroke}{rgb}{0.000000,0.447000,0.741000}%
\pgfsetstrokecolor{currentstroke}%
\pgfsetdash{}{0pt}%
\pgfsys@defobject{currentmarker}{\pgfqpoint{-0.041667in}{-0.041667in}}{\pgfqpoint{0.041667in}{0.041667in}}{%
\pgfpathmoveto{\pgfqpoint{0.000000in}{-0.041667in}}%
\pgfpathcurveto{\pgfqpoint{0.011050in}{-0.041667in}}{\pgfqpoint{0.021649in}{-0.037276in}}{\pgfqpoint{0.029463in}{-0.029463in}}%
\pgfpathcurveto{\pgfqpoint{0.037276in}{-0.021649in}}{\pgfqpoint{0.041667in}{-0.011050in}}{\pgfqpoint{0.041667in}{0.000000in}}%
\pgfpathcurveto{\pgfqpoint{0.041667in}{0.011050in}}{\pgfqpoint{0.037276in}{0.021649in}}{\pgfqpoint{0.029463in}{0.029463in}}%
\pgfpathcurveto{\pgfqpoint{0.021649in}{0.037276in}}{\pgfqpoint{0.011050in}{0.041667in}}{\pgfqpoint{0.000000in}{0.041667in}}%
\pgfpathcurveto{\pgfqpoint{-0.011050in}{0.041667in}}{\pgfqpoint{-0.021649in}{0.037276in}}{\pgfqpoint{-0.029463in}{0.029463in}}%
\pgfpathcurveto{\pgfqpoint{-0.037276in}{0.021649in}}{\pgfqpoint{-0.041667in}{0.011050in}}{\pgfqpoint{-0.041667in}{0.000000in}}%
\pgfpathcurveto{\pgfqpoint{-0.041667in}{-0.011050in}}{\pgfqpoint{-0.037276in}{-0.021649in}}{\pgfqpoint{-0.029463in}{-0.029463in}}%
\pgfpathcurveto{\pgfqpoint{-0.021649in}{-0.037276in}}{\pgfqpoint{-0.011050in}{-0.041667in}}{\pgfqpoint{0.000000in}{-0.041667in}}%
\pgfpathclose%
\pgfusepath{stroke,fill}%
}%
\begin{pgfscope}%
\pgfsys@transformshift{0.740433in}{2.162226in}%
\pgfsys@useobject{currentmarker}{}%
\end{pgfscope}%
\begin{pgfscope}%
\pgfsys@transformshift{0.946430in}{2.170150in}%
\pgfsys@useobject{currentmarker}{}%
\end{pgfscope}%
\begin{pgfscope}%
\pgfsys@transformshift{1.152427in}{2.172352in}%
\pgfsys@useobject{currentmarker}{}%
\end{pgfscope}%
\begin{pgfscope}%
\pgfsys@transformshift{1.358424in}{2.139163in}%
\pgfsys@useobject{currentmarker}{}%
\end{pgfscope}%
\begin{pgfscope}%
\pgfsys@transformshift{1.564422in}{2.068164in}%
\pgfsys@useobject{currentmarker}{}%
\end{pgfscope}%
\begin{pgfscope}%
\pgfsys@transformshift{1.770419in}{1.929454in}%
\pgfsys@useobject{currentmarker}{}%
\end{pgfscope}%
\begin{pgfscope}%
\pgfsys@transformshift{1.976416in}{1.324017in}%
\pgfsys@useobject{currentmarker}{}%
\end{pgfscope}%
\begin{pgfscope}%
\pgfsys@transformshift{2.182413in}{1.277455in}%
\pgfsys@useobject{currentmarker}{}%
\end{pgfscope}%
\begin{pgfscope}%
\pgfsys@transformshift{2.388411in}{1.261157in}%
\pgfsys@useobject{currentmarker}{}%
\end{pgfscope}%
\begin{pgfscope}%
\pgfsys@transformshift{2.594408in}{1.203736in}%
\pgfsys@useobject{currentmarker}{}%
\end{pgfscope}%
\begin{pgfscope}%
\pgfsys@transformshift{2.800405in}{1.167743in}%
\pgfsys@useobject{currentmarker}{}%
\end{pgfscope}%
\begin{pgfscope}%
\pgfsys@transformshift{3.006402in}{1.135846in}%
\pgfsys@useobject{currentmarker}{}%
\end{pgfscope}%
\begin{pgfscope}%
\pgfsys@transformshift{3.212400in}{1.097951in}%
\pgfsys@useobject{currentmarker}{}%
\end{pgfscope}%
\begin{pgfscope}%
\pgfsys@transformshift{3.418397in}{1.055154in}%
\pgfsys@useobject{currentmarker}{}%
\end{pgfscope}%
\begin{pgfscope}%
\pgfsys@transformshift{3.624394in}{1.032282in}%
\pgfsys@useobject{currentmarker}{}%
\end{pgfscope}%
\begin{pgfscope}%
\pgfsys@transformshift{3.830391in}{1.004743in}%
\pgfsys@useobject{currentmarker}{}%
\end{pgfscope}%
\begin{pgfscope}%
\pgfsys@transformshift{4.036389in}{0.948966in}%
\pgfsys@useobject{currentmarker}{}%
\end{pgfscope}%
\end{pgfscope}%
\begin{pgfscope}%
\pgfpathrectangle{\pgfqpoint{0.740433in}{0.566590in}}{\pgfqpoint{3.295956in}{1.828724in}}%
\pgfusepath{clip}%
\pgfsetbuttcap%
\pgfsetroundjoin%
\pgfsetlinewidth{1.505625pt}%
\definecolor{currentstroke}{rgb}{0.850000,0.324000,0.098000}%
\pgfsetstrokecolor{currentstroke}%
\pgfsetdash{{5.550000pt}{2.400000pt}}{0.000000pt}%
\pgfpathmoveto{\pgfqpoint{0.740433in}{2.157608in}}%
\pgfpathlineto{\pgfqpoint{0.791932in}{2.132570in}}%
\pgfpathlineto{\pgfqpoint{0.843431in}{2.110057in}}%
\pgfpathlineto{\pgfqpoint{0.894931in}{2.087097in}}%
\pgfpathlineto{\pgfqpoint{0.946430in}{2.019527in}}%
\pgfpathlineto{\pgfqpoint{0.997929in}{2.068307in}}%
\pgfpathlineto{\pgfqpoint{1.049429in}{2.013145in}}%
\pgfpathlineto{\pgfqpoint{1.100928in}{1.996185in}}%
\pgfpathlineto{\pgfqpoint{1.152427in}{1.934626in}}%
\pgfpathlineto{\pgfqpoint{1.203926in}{1.912109in}}%
\pgfpathlineto{\pgfqpoint{1.255426in}{1.905050in}}%
\pgfpathlineto{\pgfqpoint{1.306925in}{1.991528in}}%
\pgfpathlineto{\pgfqpoint{1.358424in}{1.625560in}}%
\pgfpathlineto{\pgfqpoint{1.409924in}{1.626731in}}%
\pgfpathlineto{\pgfqpoint{1.461423in}{1.599114in}}%
\pgfpathlineto{\pgfqpoint{1.512922in}{1.593273in}}%
\pgfpathlineto{\pgfqpoint{1.564422in}{1.554716in}}%
\pgfpathlineto{\pgfqpoint{1.615921in}{1.515462in}}%
\pgfpathlineto{\pgfqpoint{1.667420in}{1.508217in}}%
\pgfpathlineto{\pgfqpoint{1.718920in}{1.465308in}}%
\pgfpathlineto{\pgfqpoint{1.770419in}{1.427940in}}%
\pgfpathlineto{\pgfqpoint{1.821918in}{1.244419in}}%
\pgfpathlineto{\pgfqpoint{1.873417in}{1.348303in}}%
\pgfpathlineto{\pgfqpoint{1.924917in}{1.337637in}}%
\pgfpathlineto{\pgfqpoint{1.976416in}{1.211203in}}%
\pgfpathlineto{\pgfqpoint{2.027915in}{1.200781in}}%
\pgfpathlineto{\pgfqpoint{2.079415in}{1.194895in}}%
\pgfpathlineto{\pgfqpoint{2.130914in}{1.186716in}}%
\pgfpathlineto{\pgfqpoint{2.182413in}{1.179173in}}%
\pgfpathlineto{\pgfqpoint{2.233913in}{1.156988in}}%
\pgfpathlineto{\pgfqpoint{2.285412in}{1.158176in}}%
\pgfpathlineto{\pgfqpoint{2.336911in}{1.142954in}}%
\pgfpathlineto{\pgfqpoint{2.388411in}{1.143563in}}%
\pgfpathlineto{\pgfqpoint{2.439910in}{1.139934in}}%
\pgfpathlineto{\pgfqpoint{2.491409in}{1.120881in}}%
\pgfpathlineto{\pgfqpoint{2.542909in}{1.120537in}}%
\pgfpathlineto{\pgfqpoint{2.594408in}{1.110879in}}%
\pgfpathlineto{\pgfqpoint{2.645907in}{1.099650in}}%
\pgfpathlineto{\pgfqpoint{2.697406in}{1.088264in}}%
\pgfpathlineto{\pgfqpoint{2.748906in}{1.085575in}}%
\pgfpathlineto{\pgfqpoint{2.800405in}{1.057704in}}%
\pgfpathlineto{\pgfqpoint{2.851904in}{1.054435in}}%
\pgfpathlineto{\pgfqpoint{2.903404in}{1.063775in}}%
\pgfpathlineto{\pgfqpoint{2.954903in}{1.040211in}}%
\pgfpathlineto{\pgfqpoint{3.006402in}{1.040828in}}%
\pgfpathlineto{\pgfqpoint{3.057902in}{1.029179in}}%
\pgfpathlineto{\pgfqpoint{3.109401in}{1.024409in}}%
\pgfpathlineto{\pgfqpoint{3.160900in}{1.004144in}}%
\pgfpathlineto{\pgfqpoint{3.212400in}{0.998036in}}%
\pgfpathlineto{\pgfqpoint{3.263899in}{0.989047in}}%
\pgfpathlineto{\pgfqpoint{3.315398in}{0.979522in}}%
\pgfpathlineto{\pgfqpoint{3.366897in}{0.965060in}}%
\pgfpathlineto{\pgfqpoint{3.418397in}{0.957606in}}%
\pgfpathlineto{\pgfqpoint{3.469896in}{0.966562in}}%
\pgfpathlineto{\pgfqpoint{3.521395in}{0.937132in}}%
\pgfpathlineto{\pgfqpoint{3.572895in}{0.919154in}}%
\pgfpathlineto{\pgfqpoint{3.624394in}{0.933245in}}%
\pgfpathlineto{\pgfqpoint{3.675893in}{0.903419in}}%
\pgfpathlineto{\pgfqpoint{3.727393in}{0.914119in}}%
\pgfpathlineto{\pgfqpoint{3.778892in}{0.906377in}}%
\pgfpathlineto{\pgfqpoint{3.830391in}{0.885048in}}%
\pgfpathlineto{\pgfqpoint{3.881891in}{0.882293in}}%
\pgfpathlineto{\pgfqpoint{3.933390in}{0.879602in}}%
\pgfpathlineto{\pgfqpoint{3.984889in}{0.881853in}}%
\pgfpathlineto{\pgfqpoint{4.036389in}{0.856232in}}%
\pgfusepath{stroke}%
\end{pgfscope}%
\begin{pgfscope}%
\pgfpathrectangle{\pgfqpoint{0.740433in}{0.566590in}}{\pgfqpoint{3.295956in}{1.828724in}}%
\pgfusepath{clip}%
\pgfsetbuttcap%
\pgfsetroundjoin%
\definecolor{currentfill}{rgb}{0.850000,0.324000,0.098000}%
\pgfsetfillcolor{currentfill}%
\pgfsetlinewidth{1.003750pt}%
\definecolor{currentstroke}{rgb}{0.850000,0.324000,0.098000}%
\pgfsetstrokecolor{currentstroke}%
\pgfsetdash{}{0pt}%
\pgfsys@defobject{currentmarker}{\pgfqpoint{-0.041667in}{-0.041667in}}{\pgfqpoint{0.041667in}{0.041667in}}{%
\pgfpathmoveto{\pgfqpoint{-0.041667in}{0.000000in}}%
\pgfpathlineto{\pgfqpoint{0.041667in}{0.000000in}}%
\pgfpathmoveto{\pgfqpoint{0.000000in}{-0.041667in}}%
\pgfpathlineto{\pgfqpoint{0.000000in}{0.041667in}}%
\pgfusepath{stroke,fill}%
}%
\begin{pgfscope}%
\pgfsys@transformshift{0.740433in}{2.157608in}%
\pgfsys@useobject{currentmarker}{}%
\end{pgfscope}%
\begin{pgfscope}%
\pgfsys@transformshift{0.894931in}{2.087097in}%
\pgfsys@useobject{currentmarker}{}%
\end{pgfscope}%
\begin{pgfscope}%
\pgfsys@transformshift{1.049429in}{2.013145in}%
\pgfsys@useobject{currentmarker}{}%
\end{pgfscope}%
\begin{pgfscope}%
\pgfsys@transformshift{1.203926in}{1.912109in}%
\pgfsys@useobject{currentmarker}{}%
\end{pgfscope}%
\begin{pgfscope}%
\pgfsys@transformshift{1.358424in}{1.625560in}%
\pgfsys@useobject{currentmarker}{}%
\end{pgfscope}%
\begin{pgfscope}%
\pgfsys@transformshift{1.512922in}{1.593273in}%
\pgfsys@useobject{currentmarker}{}%
\end{pgfscope}%
\begin{pgfscope}%
\pgfsys@transformshift{1.667420in}{1.508217in}%
\pgfsys@useobject{currentmarker}{}%
\end{pgfscope}%
\begin{pgfscope}%
\pgfsys@transformshift{1.821918in}{1.244419in}%
\pgfsys@useobject{currentmarker}{}%
\end{pgfscope}%
\begin{pgfscope}%
\pgfsys@transformshift{1.976416in}{1.211203in}%
\pgfsys@useobject{currentmarker}{}%
\end{pgfscope}%
\begin{pgfscope}%
\pgfsys@transformshift{2.130914in}{1.186716in}%
\pgfsys@useobject{currentmarker}{}%
\end{pgfscope}%
\begin{pgfscope}%
\pgfsys@transformshift{2.285412in}{1.158176in}%
\pgfsys@useobject{currentmarker}{}%
\end{pgfscope}%
\begin{pgfscope}%
\pgfsys@transformshift{2.439910in}{1.139934in}%
\pgfsys@useobject{currentmarker}{}%
\end{pgfscope}%
\begin{pgfscope}%
\pgfsys@transformshift{2.594408in}{1.110879in}%
\pgfsys@useobject{currentmarker}{}%
\end{pgfscope}%
\begin{pgfscope}%
\pgfsys@transformshift{2.748906in}{1.085575in}%
\pgfsys@useobject{currentmarker}{}%
\end{pgfscope}%
\begin{pgfscope}%
\pgfsys@transformshift{2.903404in}{1.063775in}%
\pgfsys@useobject{currentmarker}{}%
\end{pgfscope}%
\begin{pgfscope}%
\pgfsys@transformshift{3.057902in}{1.029179in}%
\pgfsys@useobject{currentmarker}{}%
\end{pgfscope}%
\begin{pgfscope}%
\pgfsys@transformshift{3.212400in}{0.998036in}%
\pgfsys@useobject{currentmarker}{}%
\end{pgfscope}%
\begin{pgfscope}%
\pgfsys@transformshift{3.366897in}{0.965060in}%
\pgfsys@useobject{currentmarker}{}%
\end{pgfscope}%
\begin{pgfscope}%
\pgfsys@transformshift{3.521395in}{0.937132in}%
\pgfsys@useobject{currentmarker}{}%
\end{pgfscope}%
\begin{pgfscope}%
\pgfsys@transformshift{3.675893in}{0.903419in}%
\pgfsys@useobject{currentmarker}{}%
\end{pgfscope}%
\begin{pgfscope}%
\pgfsys@transformshift{3.830391in}{0.885048in}%
\pgfsys@useobject{currentmarker}{}%
\end{pgfscope}%
\begin{pgfscope}%
\pgfsys@transformshift{3.984889in}{0.881853in}%
\pgfsys@useobject{currentmarker}{}%
\end{pgfscope}%
\end{pgfscope}%
\begin{pgfscope}%
\pgfpathrectangle{\pgfqpoint{0.740433in}{0.566590in}}{\pgfqpoint{3.295956in}{1.828724in}}%
\pgfusepath{clip}%
\pgfsetbuttcap%
\pgfsetroundjoin%
\pgfsetlinewidth{1.505625pt}%
\definecolor{currentstroke}{rgb}{0.000000,0.500000,0.000000}%
\pgfsetstrokecolor{currentstroke}%
\pgfsetdash{{5.550000pt}{2.400000pt}}{0.000000pt}%
\pgfpathmoveto{\pgfqpoint{0.740433in}{2.252770in}}%
\pgfpathlineto{\pgfqpoint{0.791932in}{2.241377in}}%
\pgfpathlineto{\pgfqpoint{0.843431in}{2.230198in}}%
\pgfpathlineto{\pgfqpoint{0.894931in}{2.249755in}}%
\pgfpathlineto{\pgfqpoint{0.946430in}{2.233450in}}%
\pgfpathlineto{\pgfqpoint{0.997929in}{2.237998in}}%
\pgfpathlineto{\pgfqpoint{1.049429in}{2.235474in}}%
\pgfpathlineto{\pgfqpoint{1.100928in}{2.234783in}}%
\pgfpathlineto{\pgfqpoint{1.152427in}{2.225058in}}%
\pgfpathlineto{\pgfqpoint{1.203926in}{2.218879in}}%
\pgfpathlineto{\pgfqpoint{1.255426in}{2.215610in}}%
\pgfpathlineto{\pgfqpoint{1.306925in}{2.210714in}}%
\pgfpathlineto{\pgfqpoint{1.358424in}{2.204654in}}%
\pgfpathlineto{\pgfqpoint{1.409924in}{2.203007in}}%
\pgfpathlineto{\pgfqpoint{1.461423in}{2.171209in}}%
\pgfpathlineto{\pgfqpoint{1.512922in}{2.167184in}}%
\pgfpathlineto{\pgfqpoint{1.564422in}{2.119465in}}%
\pgfpathlineto{\pgfqpoint{1.615921in}{2.108342in}}%
\pgfpathlineto{\pgfqpoint{1.667420in}{2.099888in}}%
\pgfpathlineto{\pgfqpoint{1.718920in}{2.065446in}}%
\pgfpathlineto{\pgfqpoint{1.770419in}{2.018542in}}%
\pgfpathlineto{\pgfqpoint{1.821918in}{1.267767in}}%
\pgfpathlineto{\pgfqpoint{1.873417in}{1.948265in}}%
\pgfpathlineto{\pgfqpoint{1.924917in}{1.954220in}}%
\pgfpathlineto{\pgfqpoint{1.976416in}{1.233496in}}%
\pgfpathlineto{\pgfqpoint{2.027915in}{1.224944in}}%
\pgfpathlineto{\pgfqpoint{2.079415in}{1.217823in}}%
\pgfpathlineto{\pgfqpoint{2.130914in}{1.220644in}}%
\pgfpathlineto{\pgfqpoint{2.182413in}{1.211178in}}%
\pgfpathlineto{\pgfqpoint{2.233913in}{1.177043in}}%
\pgfpathlineto{\pgfqpoint{2.285412in}{1.186696in}}%
\pgfpathlineto{\pgfqpoint{2.336911in}{1.177131in}}%
\pgfpathlineto{\pgfqpoint{2.388411in}{1.162991in}}%
\pgfpathlineto{\pgfqpoint{2.439910in}{1.157661in}}%
\pgfpathlineto{\pgfqpoint{2.491409in}{1.145618in}}%
\pgfpathlineto{\pgfqpoint{2.542909in}{1.147915in}}%
\pgfpathlineto{\pgfqpoint{2.594408in}{1.125957in}}%
\pgfpathlineto{\pgfqpoint{2.645907in}{1.118744in}}%
\pgfpathlineto{\pgfqpoint{2.697406in}{1.115574in}}%
\pgfpathlineto{\pgfqpoint{2.748906in}{1.110354in}}%
\pgfpathlineto{\pgfqpoint{2.800405in}{1.089557in}}%
\pgfpathlineto{\pgfqpoint{2.851904in}{1.080399in}}%
\pgfpathlineto{\pgfqpoint{2.903404in}{1.078290in}}%
\pgfpathlineto{\pgfqpoint{2.954903in}{1.068957in}}%
\pgfpathlineto{\pgfqpoint{3.006402in}{1.067940in}}%
\pgfpathlineto{\pgfqpoint{3.057902in}{1.051374in}}%
\pgfpathlineto{\pgfqpoint{3.109401in}{1.041086in}}%
\pgfpathlineto{\pgfqpoint{3.160900in}{1.012938in}}%
\pgfpathlineto{\pgfqpoint{3.212400in}{1.021736in}}%
\pgfpathlineto{\pgfqpoint{3.263899in}{1.013121in}}%
\pgfpathlineto{\pgfqpoint{3.315398in}{1.002245in}}%
\pgfpathlineto{\pgfqpoint{3.366897in}{0.995996in}}%
\pgfpathlineto{\pgfqpoint{3.418397in}{0.991611in}}%
\pgfpathlineto{\pgfqpoint{3.469896in}{0.983154in}}%
\pgfpathlineto{\pgfqpoint{3.521395in}{0.978005in}}%
\pgfpathlineto{\pgfqpoint{3.572895in}{0.956400in}}%
\pgfpathlineto{\pgfqpoint{3.624394in}{0.952963in}}%
\pgfpathlineto{\pgfqpoint{3.675893in}{0.933291in}}%
\pgfpathlineto{\pgfqpoint{3.727393in}{0.930852in}}%
\pgfpathlineto{\pgfqpoint{3.778892in}{0.927526in}}%
\pgfpathlineto{\pgfqpoint{3.830391in}{0.919947in}}%
\pgfpathlineto{\pgfqpoint{3.881891in}{0.909414in}}%
\pgfpathlineto{\pgfqpoint{3.933390in}{0.901101in}}%
\pgfpathlineto{\pgfqpoint{3.984889in}{0.895750in}}%
\pgfpathlineto{\pgfqpoint{4.036389in}{0.879925in}}%
\pgfusepath{stroke}%
\end{pgfscope}%
\begin{pgfscope}%
\pgfpathrectangle{\pgfqpoint{0.740433in}{0.566590in}}{\pgfqpoint{3.295956in}{1.828724in}}%
\pgfusepath{clip}%
\pgfsetbuttcap%
\pgfsetmiterjoin%
\definecolor{currentfill}{rgb}{0.000000,0.000000,0.000000}%
\pgfsetfillcolor{currentfill}%
\pgfsetfillopacity{0.000000}%
\pgfsetlinewidth{1.003750pt}%
\definecolor{currentstroke}{rgb}{0.000000,0.500000,0.000000}%
\pgfsetstrokecolor{currentstroke}%
\pgfsetdash{}{0pt}%
\pgfsys@defobject{currentmarker}{\pgfqpoint{-0.041667in}{-0.041667in}}{\pgfqpoint{0.041667in}{0.041667in}}{%
\pgfpathmoveto{\pgfqpoint{-0.041667in}{-0.041667in}}%
\pgfpathlineto{\pgfqpoint{0.041667in}{-0.041667in}}%
\pgfpathlineto{\pgfqpoint{0.041667in}{0.041667in}}%
\pgfpathlineto{\pgfqpoint{-0.041667in}{0.041667in}}%
\pgfpathclose%
\pgfusepath{stroke,fill}%
}%
\begin{pgfscope}%
\pgfsys@transformshift{0.740433in}{2.252770in}%
\pgfsys@useobject{currentmarker}{}%
\end{pgfscope}%
\begin{pgfscope}%
\pgfsys@transformshift{0.997929in}{2.237998in}%
\pgfsys@useobject{currentmarker}{}%
\end{pgfscope}%
\begin{pgfscope}%
\pgfsys@transformshift{1.255426in}{2.215610in}%
\pgfsys@useobject{currentmarker}{}%
\end{pgfscope}%
\begin{pgfscope}%
\pgfsys@transformshift{1.512922in}{2.167184in}%
\pgfsys@useobject{currentmarker}{}%
\end{pgfscope}%
\begin{pgfscope}%
\pgfsys@transformshift{1.770419in}{2.018542in}%
\pgfsys@useobject{currentmarker}{}%
\end{pgfscope}%
\begin{pgfscope}%
\pgfsys@transformshift{2.027915in}{1.224944in}%
\pgfsys@useobject{currentmarker}{}%
\end{pgfscope}%
\begin{pgfscope}%
\pgfsys@transformshift{2.285412in}{1.186696in}%
\pgfsys@useobject{currentmarker}{}%
\end{pgfscope}%
\begin{pgfscope}%
\pgfsys@transformshift{2.542909in}{1.147915in}%
\pgfsys@useobject{currentmarker}{}%
\end{pgfscope}%
\begin{pgfscope}%
\pgfsys@transformshift{2.800405in}{1.089557in}%
\pgfsys@useobject{currentmarker}{}%
\end{pgfscope}%
\begin{pgfscope}%
\pgfsys@transformshift{3.057902in}{1.051374in}%
\pgfsys@useobject{currentmarker}{}%
\end{pgfscope}%
\begin{pgfscope}%
\pgfsys@transformshift{3.315398in}{1.002245in}%
\pgfsys@useobject{currentmarker}{}%
\end{pgfscope}%
\begin{pgfscope}%
\pgfsys@transformshift{3.572895in}{0.956400in}%
\pgfsys@useobject{currentmarker}{}%
\end{pgfscope}%
\begin{pgfscope}%
\pgfsys@transformshift{3.830391in}{0.919947in}%
\pgfsys@useobject{currentmarker}{}%
\end{pgfscope}%
\end{pgfscope}%
\begin{pgfscope}%
\pgfpathrectangle{\pgfqpoint{0.740433in}{0.566590in}}{\pgfqpoint{3.295956in}{1.828724in}}%
\pgfusepath{clip}%
\pgfsetbuttcap%
\pgfsetroundjoin%
\pgfsetlinewidth{1.505625pt}%
\definecolor{currentstroke}{rgb}{0.494000,0.184000,0.556000}%
\pgfsetstrokecolor{currentstroke}%
\pgfsetdash{{5.550000pt}{2.400000pt}}{0.000000pt}%
\pgfpathmoveto{\pgfqpoint{0.740433in}{2.312190in}}%
\pgfpathlineto{\pgfqpoint{0.791932in}{2.293616in}}%
\pgfpathlineto{\pgfqpoint{0.843431in}{2.286510in}}%
\pgfpathlineto{\pgfqpoint{0.894931in}{2.264386in}}%
\pgfpathlineto{\pgfqpoint{0.946430in}{2.276097in}}%
\pgfpathlineto{\pgfqpoint{0.997929in}{2.251210in}}%
\pgfpathlineto{\pgfqpoint{1.049429in}{2.242105in}}%
\pgfpathlineto{\pgfqpoint{1.100928in}{2.212281in}}%
\pgfpathlineto{\pgfqpoint{1.152427in}{2.233010in}}%
\pgfpathlineto{\pgfqpoint{1.203926in}{2.257143in}}%
\pgfpathlineto{\pgfqpoint{1.255426in}{2.221324in}}%
\pgfpathlineto{\pgfqpoint{1.306925in}{2.245390in}}%
\pgfpathlineto{\pgfqpoint{1.358424in}{2.226076in}}%
\pgfpathlineto{\pgfqpoint{1.409924in}{2.205163in}}%
\pgfpathlineto{\pgfqpoint{1.461423in}{2.155926in}}%
\pgfpathlineto{\pgfqpoint{1.512922in}{2.121956in}}%
\pgfpathlineto{\pgfqpoint{1.564422in}{2.141438in}}%
\pgfpathlineto{\pgfqpoint{1.615921in}{2.117616in}}%
\pgfpathlineto{\pgfqpoint{1.667420in}{2.033654in}}%
\pgfpathlineto{\pgfqpoint{1.718920in}{1.241818in}}%
\pgfpathlineto{\pgfqpoint{1.770419in}{1.669829in}}%
\pgfpathlineto{\pgfqpoint{1.821918in}{1.234688in}}%
\pgfpathlineto{\pgfqpoint{1.873417in}{1.225000in}}%
\pgfpathlineto{\pgfqpoint{1.924917in}{1.220334in}}%
\pgfpathlineto{\pgfqpoint{1.976416in}{1.193377in}}%
\pgfpathlineto{\pgfqpoint{2.027915in}{1.194898in}}%
\pgfpathlineto{\pgfqpoint{2.079415in}{1.186136in}}%
\pgfpathlineto{\pgfqpoint{2.130914in}{1.175693in}}%
\pgfpathlineto{\pgfqpoint{2.182413in}{1.166905in}}%
\pgfpathlineto{\pgfqpoint{2.233913in}{1.151660in}}%
\pgfpathlineto{\pgfqpoint{2.285412in}{1.144976in}}%
\pgfpathlineto{\pgfqpoint{2.336911in}{1.126972in}}%
\pgfpathlineto{\pgfqpoint{2.388411in}{1.122031in}}%
\pgfpathlineto{\pgfqpoint{2.439910in}{1.123051in}}%
\pgfpathlineto{\pgfqpoint{2.491409in}{1.106857in}}%
\pgfpathlineto{\pgfqpoint{2.542909in}{1.102454in}}%
\pgfpathlineto{\pgfqpoint{2.594408in}{1.101180in}}%
\pgfpathlineto{\pgfqpoint{2.645907in}{1.075323in}}%
\pgfpathlineto{\pgfqpoint{2.697406in}{1.076156in}}%
\pgfpathlineto{\pgfqpoint{2.748906in}{1.066884in}}%
\pgfpathlineto{\pgfqpoint{2.800405in}{1.062968in}}%
\pgfpathlineto{\pgfqpoint{2.851904in}{1.050628in}}%
\pgfpathlineto{\pgfqpoint{2.903404in}{1.041775in}}%
\pgfpathlineto{\pgfqpoint{2.954903in}{1.040053in}}%
\pgfpathlineto{\pgfqpoint{3.006402in}{1.014828in}}%
\pgfpathlineto{\pgfqpoint{3.057902in}{1.009986in}}%
\pgfpathlineto{\pgfqpoint{3.109401in}{1.009158in}}%
\pgfpathlineto{\pgfqpoint{3.160900in}{0.997837in}}%
\pgfpathlineto{\pgfqpoint{3.212400in}{0.981118in}}%
\pgfpathlineto{\pgfqpoint{3.263899in}{0.980474in}}%
\pgfpathlineto{\pgfqpoint{3.315398in}{0.970782in}}%
\pgfpathlineto{\pgfqpoint{3.366897in}{0.954948in}}%
\pgfpathlineto{\pgfqpoint{3.418397in}{0.943725in}}%
\pgfpathlineto{\pgfqpoint{3.469896in}{0.934875in}}%
\pgfpathlineto{\pgfqpoint{3.521395in}{0.934531in}}%
\pgfpathlineto{\pgfqpoint{3.572895in}{0.910029in}}%
\pgfpathlineto{\pgfqpoint{3.624394in}{0.914885in}}%
\pgfpathlineto{\pgfqpoint{3.675893in}{0.893210in}}%
\pgfpathlineto{\pgfqpoint{3.727393in}{0.899613in}}%
\pgfpathlineto{\pgfqpoint{3.778892in}{0.890057in}}%
\pgfpathlineto{\pgfqpoint{3.830391in}{0.870322in}}%
\pgfpathlineto{\pgfqpoint{3.881891in}{0.863422in}}%
\pgfpathlineto{\pgfqpoint{3.933390in}{0.850528in}}%
\pgfpathlineto{\pgfqpoint{3.984889in}{0.854209in}}%
\pgfpathlineto{\pgfqpoint{4.036389in}{0.845163in}}%
\pgfusepath{stroke}%
\end{pgfscope}%
\begin{pgfscope}%
\pgfpathrectangle{\pgfqpoint{0.740433in}{0.566590in}}{\pgfqpoint{3.295956in}{1.828724in}}%
\pgfusepath{clip}%
\pgfsetbuttcap%
\pgfsetroundjoin%
\definecolor{currentfill}{rgb}{0.494000,0.184000,0.556000}%
\pgfsetfillcolor{currentfill}%
\pgfsetlinewidth{1.003750pt}%
\definecolor{currentstroke}{rgb}{0.494000,0.184000,0.556000}%
\pgfsetstrokecolor{currentstroke}%
\pgfsetdash{}{0pt}%
\pgfsys@defobject{currentmarker}{\pgfqpoint{-0.041667in}{-0.041667in}}{\pgfqpoint{0.041667in}{0.041667in}}{%
\pgfpathmoveto{\pgfqpoint{-0.041667in}{-0.041667in}}%
\pgfpathlineto{\pgfqpoint{0.041667in}{0.041667in}}%
\pgfpathmoveto{\pgfqpoint{-0.041667in}{0.041667in}}%
\pgfpathlineto{\pgfqpoint{0.041667in}{-0.041667in}}%
\pgfusepath{stroke,fill}%
}%
\begin{pgfscope}%
\pgfsys@transformshift{0.740433in}{2.312190in}%
\pgfsys@useobject{currentmarker}{}%
\end{pgfscope}%
\begin{pgfscope}%
\pgfsys@transformshift{0.946430in}{2.276097in}%
\pgfsys@useobject{currentmarker}{}%
\end{pgfscope}%
\begin{pgfscope}%
\pgfsys@transformshift{1.152427in}{2.233010in}%
\pgfsys@useobject{currentmarker}{}%
\end{pgfscope}%
\begin{pgfscope}%
\pgfsys@transformshift{1.358424in}{2.226076in}%
\pgfsys@useobject{currentmarker}{}%
\end{pgfscope}%
\begin{pgfscope}%
\pgfsys@transformshift{1.564422in}{2.141438in}%
\pgfsys@useobject{currentmarker}{}%
\end{pgfscope}%
\begin{pgfscope}%
\pgfsys@transformshift{1.770419in}{1.669829in}%
\pgfsys@useobject{currentmarker}{}%
\end{pgfscope}%
\begin{pgfscope}%
\pgfsys@transformshift{1.976416in}{1.193377in}%
\pgfsys@useobject{currentmarker}{}%
\end{pgfscope}%
\begin{pgfscope}%
\pgfsys@transformshift{2.182413in}{1.166905in}%
\pgfsys@useobject{currentmarker}{}%
\end{pgfscope}%
\begin{pgfscope}%
\pgfsys@transformshift{2.388411in}{1.122031in}%
\pgfsys@useobject{currentmarker}{}%
\end{pgfscope}%
\begin{pgfscope}%
\pgfsys@transformshift{2.594408in}{1.101180in}%
\pgfsys@useobject{currentmarker}{}%
\end{pgfscope}%
\begin{pgfscope}%
\pgfsys@transformshift{2.800405in}{1.062968in}%
\pgfsys@useobject{currentmarker}{}%
\end{pgfscope}%
\begin{pgfscope}%
\pgfsys@transformshift{3.006402in}{1.014828in}%
\pgfsys@useobject{currentmarker}{}%
\end{pgfscope}%
\begin{pgfscope}%
\pgfsys@transformshift{3.212400in}{0.981118in}%
\pgfsys@useobject{currentmarker}{}%
\end{pgfscope}%
\begin{pgfscope}%
\pgfsys@transformshift{3.418397in}{0.943725in}%
\pgfsys@useobject{currentmarker}{}%
\end{pgfscope}%
\begin{pgfscope}%
\pgfsys@transformshift{3.624394in}{0.914885in}%
\pgfsys@useobject{currentmarker}{}%
\end{pgfscope}%
\begin{pgfscope}%
\pgfsys@transformshift{3.830391in}{0.870322in}%
\pgfsys@useobject{currentmarker}{}%
\end{pgfscope}%
\begin{pgfscope}%
\pgfsys@transformshift{4.036389in}{0.845163in}%
\pgfsys@useobject{currentmarker}{}%
\end{pgfscope}%
\end{pgfscope}%
\begin{pgfscope}%
\pgfpathrectangle{\pgfqpoint{0.740433in}{0.566590in}}{\pgfqpoint{3.295956in}{1.828724in}}%
\pgfusepath{clip}%
\pgfsetrectcap%
\pgfsetroundjoin%
\pgfsetlinewidth{1.505625pt}%
\definecolor{currentstroke}{rgb}{0.000000,0.447000,0.741000}%
\pgfsetstrokecolor{currentstroke}%
\pgfsetdash{}{0pt}%
\pgfpathmoveto{\pgfqpoint{0.740433in}{1.762838in}}%
\pgfpathlineto{\pgfqpoint{0.791932in}{1.748511in}}%
\pgfpathlineto{\pgfqpoint{0.843431in}{1.753916in}}%
\pgfpathlineto{\pgfqpoint{0.894931in}{1.749913in}}%
\pgfpathlineto{\pgfqpoint{0.946430in}{1.734808in}}%
\pgfpathlineto{\pgfqpoint{0.997929in}{1.740537in}}%
\pgfpathlineto{\pgfqpoint{1.049429in}{1.749631in}}%
\pgfpathlineto{\pgfqpoint{1.100928in}{1.736105in}}%
\pgfpathlineto{\pgfqpoint{1.152427in}{1.749038in}}%
\pgfpathlineto{\pgfqpoint{1.203926in}{1.707965in}}%
\pgfpathlineto{\pgfqpoint{1.255426in}{1.705456in}}%
\pgfpathlineto{\pgfqpoint{1.306925in}{1.680176in}}%
\pgfpathlineto{\pgfqpoint{1.358424in}{1.677236in}}%
\pgfpathlineto{\pgfqpoint{1.409924in}{1.643566in}}%
\pgfpathlineto{\pgfqpoint{1.461423in}{1.654967in}}%
\pgfpathlineto{\pgfqpoint{1.512922in}{1.645091in}}%
\pgfpathlineto{\pgfqpoint{1.564422in}{1.571662in}}%
\pgfpathlineto{\pgfqpoint{1.615921in}{1.502113in}}%
\pgfpathlineto{\pgfqpoint{1.667420in}{1.504216in}}%
\pgfpathlineto{\pgfqpoint{1.718920in}{1.278904in}}%
\pgfpathlineto{\pgfqpoint{1.770419in}{1.262854in}}%
\pgfpathlineto{\pgfqpoint{1.821918in}{1.240546in}}%
\pgfpathlineto{\pgfqpoint{1.873417in}{1.226601in}}%
\pgfpathlineto{\pgfqpoint{1.924917in}{1.232341in}}%
\pgfpathlineto{\pgfqpoint{1.976416in}{1.216284in}}%
\pgfpathlineto{\pgfqpoint{2.027915in}{1.202192in}}%
\pgfpathlineto{\pgfqpoint{2.079415in}{1.179281in}}%
\pgfpathlineto{\pgfqpoint{2.130914in}{1.169611in}}%
\pgfpathlineto{\pgfqpoint{2.182413in}{1.166728in}}%
\pgfpathlineto{\pgfqpoint{2.233913in}{1.151969in}}%
\pgfpathlineto{\pgfqpoint{2.285412in}{1.132077in}}%
\pgfpathlineto{\pgfqpoint{2.336911in}{1.103598in}}%
\pgfpathlineto{\pgfqpoint{2.388411in}{1.116438in}}%
\pgfpathlineto{\pgfqpoint{2.439910in}{1.080358in}}%
\pgfpathlineto{\pgfqpoint{2.491409in}{1.058327in}}%
\pgfpathlineto{\pgfqpoint{2.542909in}{1.057530in}}%
\pgfpathlineto{\pgfqpoint{2.594408in}{1.035086in}}%
\pgfpathlineto{\pgfqpoint{2.645907in}{1.031890in}}%
\pgfpathlineto{\pgfqpoint{2.697406in}{1.024453in}}%
\pgfpathlineto{\pgfqpoint{2.748906in}{1.008044in}}%
\pgfpathlineto{\pgfqpoint{2.800405in}{0.995382in}}%
\pgfpathlineto{\pgfqpoint{2.851904in}{0.978053in}}%
\pgfpathlineto{\pgfqpoint{2.903404in}{0.979817in}}%
\pgfpathlineto{\pgfqpoint{2.954903in}{0.950981in}}%
\pgfpathlineto{\pgfqpoint{3.006402in}{0.951177in}}%
\pgfpathlineto{\pgfqpoint{3.057902in}{0.949654in}}%
\pgfpathlineto{\pgfqpoint{3.109401in}{0.939496in}}%
\pgfpathlineto{\pgfqpoint{3.160900in}{0.927167in}}%
\pgfpathlineto{\pgfqpoint{3.212400in}{0.914755in}}%
\pgfpathlineto{\pgfqpoint{3.263899in}{0.919308in}}%
\pgfpathlineto{\pgfqpoint{3.315398in}{0.891479in}}%
\pgfpathlineto{\pgfqpoint{3.366897in}{0.882281in}}%
\pgfpathlineto{\pgfqpoint{3.418397in}{0.874989in}}%
\pgfpathlineto{\pgfqpoint{3.469896in}{0.871778in}}%
\pgfpathlineto{\pgfqpoint{3.521395in}{0.865204in}}%
\pgfpathlineto{\pgfqpoint{3.572895in}{0.847364in}}%
\pgfpathlineto{\pgfqpoint{3.624394in}{0.846901in}}%
\pgfpathlineto{\pgfqpoint{3.675893in}{0.841023in}}%
\pgfpathlineto{\pgfqpoint{3.727393in}{0.832625in}}%
\pgfpathlineto{\pgfqpoint{3.778892in}{0.810731in}}%
\pgfpathlineto{\pgfqpoint{3.830391in}{0.820629in}}%
\pgfpathlineto{\pgfqpoint{3.881891in}{0.803988in}}%
\pgfpathlineto{\pgfqpoint{3.933390in}{0.794428in}}%
\pgfpathlineto{\pgfqpoint{3.984889in}{0.789066in}}%
\pgfpathlineto{\pgfqpoint{4.036389in}{0.761537in}}%
\pgfusepath{stroke}%
\end{pgfscope}%
\begin{pgfscope}%
\pgfpathrectangle{\pgfqpoint{0.740433in}{0.566590in}}{\pgfqpoint{3.295956in}{1.828724in}}%
\pgfusepath{clip}%
\pgfsetbuttcap%
\pgfsetroundjoin%
\definecolor{currentfill}{rgb}{0.000000,0.000000,0.000000}%
\pgfsetfillcolor{currentfill}%
\pgfsetfillopacity{0.000000}%
\pgfsetlinewidth{1.003750pt}%
\definecolor{currentstroke}{rgb}{0.000000,0.447000,0.741000}%
\pgfsetstrokecolor{currentstroke}%
\pgfsetdash{}{0pt}%
\pgfsys@defobject{currentmarker}{\pgfqpoint{-0.041667in}{-0.041667in}}{\pgfqpoint{0.041667in}{0.041667in}}{%
\pgfpathmoveto{\pgfqpoint{0.000000in}{-0.041667in}}%
\pgfpathcurveto{\pgfqpoint{0.011050in}{-0.041667in}}{\pgfqpoint{0.021649in}{-0.037276in}}{\pgfqpoint{0.029463in}{-0.029463in}}%
\pgfpathcurveto{\pgfqpoint{0.037276in}{-0.021649in}}{\pgfqpoint{0.041667in}{-0.011050in}}{\pgfqpoint{0.041667in}{0.000000in}}%
\pgfpathcurveto{\pgfqpoint{0.041667in}{0.011050in}}{\pgfqpoint{0.037276in}{0.021649in}}{\pgfqpoint{0.029463in}{0.029463in}}%
\pgfpathcurveto{\pgfqpoint{0.021649in}{0.037276in}}{\pgfqpoint{0.011050in}{0.041667in}}{\pgfqpoint{0.000000in}{0.041667in}}%
\pgfpathcurveto{\pgfqpoint{-0.011050in}{0.041667in}}{\pgfqpoint{-0.021649in}{0.037276in}}{\pgfqpoint{-0.029463in}{0.029463in}}%
\pgfpathcurveto{\pgfqpoint{-0.037276in}{0.021649in}}{\pgfqpoint{-0.041667in}{0.011050in}}{\pgfqpoint{-0.041667in}{0.000000in}}%
\pgfpathcurveto{\pgfqpoint{-0.041667in}{-0.011050in}}{\pgfqpoint{-0.037276in}{-0.021649in}}{\pgfqpoint{-0.029463in}{-0.029463in}}%
\pgfpathcurveto{\pgfqpoint{-0.021649in}{-0.037276in}}{\pgfqpoint{-0.011050in}{-0.041667in}}{\pgfqpoint{0.000000in}{-0.041667in}}%
\pgfpathclose%
\pgfusepath{stroke,fill}%
}%
\begin{pgfscope}%
\pgfsys@transformshift{0.740433in}{1.762838in}%
\pgfsys@useobject{currentmarker}{}%
\end{pgfscope}%
\begin{pgfscope}%
\pgfsys@transformshift{0.946430in}{1.734808in}%
\pgfsys@useobject{currentmarker}{}%
\end{pgfscope}%
\begin{pgfscope}%
\pgfsys@transformshift{1.152427in}{1.749038in}%
\pgfsys@useobject{currentmarker}{}%
\end{pgfscope}%
\begin{pgfscope}%
\pgfsys@transformshift{1.358424in}{1.677236in}%
\pgfsys@useobject{currentmarker}{}%
\end{pgfscope}%
\begin{pgfscope}%
\pgfsys@transformshift{1.564422in}{1.571662in}%
\pgfsys@useobject{currentmarker}{}%
\end{pgfscope}%
\begin{pgfscope}%
\pgfsys@transformshift{1.770419in}{1.262854in}%
\pgfsys@useobject{currentmarker}{}%
\end{pgfscope}%
\begin{pgfscope}%
\pgfsys@transformshift{1.976416in}{1.216284in}%
\pgfsys@useobject{currentmarker}{}%
\end{pgfscope}%
\begin{pgfscope}%
\pgfsys@transformshift{2.182413in}{1.166728in}%
\pgfsys@useobject{currentmarker}{}%
\end{pgfscope}%
\begin{pgfscope}%
\pgfsys@transformshift{2.388411in}{1.116438in}%
\pgfsys@useobject{currentmarker}{}%
\end{pgfscope}%
\begin{pgfscope}%
\pgfsys@transformshift{2.594408in}{1.035086in}%
\pgfsys@useobject{currentmarker}{}%
\end{pgfscope}%
\begin{pgfscope}%
\pgfsys@transformshift{2.800405in}{0.995382in}%
\pgfsys@useobject{currentmarker}{}%
\end{pgfscope}%
\begin{pgfscope}%
\pgfsys@transformshift{3.006402in}{0.951177in}%
\pgfsys@useobject{currentmarker}{}%
\end{pgfscope}%
\begin{pgfscope}%
\pgfsys@transformshift{3.212400in}{0.914755in}%
\pgfsys@useobject{currentmarker}{}%
\end{pgfscope}%
\begin{pgfscope}%
\pgfsys@transformshift{3.418397in}{0.874989in}%
\pgfsys@useobject{currentmarker}{}%
\end{pgfscope}%
\begin{pgfscope}%
\pgfsys@transformshift{3.624394in}{0.846901in}%
\pgfsys@useobject{currentmarker}{}%
\end{pgfscope}%
\begin{pgfscope}%
\pgfsys@transformshift{3.830391in}{0.820629in}%
\pgfsys@useobject{currentmarker}{}%
\end{pgfscope}%
\begin{pgfscope}%
\pgfsys@transformshift{4.036389in}{0.761537in}%
\pgfsys@useobject{currentmarker}{}%
\end{pgfscope}%
\end{pgfscope}%
\begin{pgfscope}%
\pgfpathrectangle{\pgfqpoint{0.740433in}{0.566590in}}{\pgfqpoint{3.295956in}{1.828724in}}%
\pgfusepath{clip}%
\pgfsetrectcap%
\pgfsetroundjoin%
\pgfsetlinewidth{1.505625pt}%
\definecolor{currentstroke}{rgb}{0.850000,0.324000,0.098000}%
\pgfsetstrokecolor{currentstroke}%
\pgfsetdash{}{0pt}%
\pgfpathmoveto{\pgfqpoint{0.740433in}{1.578208in}}%
\pgfpathlineto{\pgfqpoint{0.791932in}{1.544580in}}%
\pgfpathlineto{\pgfqpoint{0.843431in}{1.521081in}}%
\pgfpathlineto{\pgfqpoint{0.894931in}{1.493345in}}%
\pgfpathlineto{\pgfqpoint{0.946430in}{1.474081in}}%
\pgfpathlineto{\pgfqpoint{0.997929in}{1.458654in}}%
\pgfpathlineto{\pgfqpoint{1.049429in}{1.407107in}}%
\pgfpathlineto{\pgfqpoint{1.100928in}{1.368766in}}%
\pgfpathlineto{\pgfqpoint{1.152427in}{1.286367in}}%
\pgfpathlineto{\pgfqpoint{1.203926in}{1.286540in}}%
\pgfpathlineto{\pgfqpoint{1.255426in}{1.225780in}}%
\pgfpathlineto{\pgfqpoint{1.306925in}{1.253272in}}%
\pgfpathlineto{\pgfqpoint{1.358424in}{1.205467in}}%
\pgfpathlineto{\pgfqpoint{1.409924in}{1.203918in}}%
\pgfpathlineto{\pgfqpoint{1.461423in}{1.171159in}}%
\pgfpathlineto{\pgfqpoint{1.512922in}{1.139482in}}%
\pgfpathlineto{\pgfqpoint{1.564422in}{1.131744in}}%
\pgfpathlineto{\pgfqpoint{1.615921in}{1.140884in}}%
\pgfpathlineto{\pgfqpoint{1.667420in}{1.106290in}}%
\pgfpathlineto{\pgfqpoint{1.718920in}{1.083631in}}%
\pgfpathlineto{\pgfqpoint{1.770419in}{1.083800in}}%
\pgfpathlineto{\pgfqpoint{1.821918in}{1.067060in}}%
\pgfpathlineto{\pgfqpoint{1.873417in}{1.058819in}}%
\pgfpathlineto{\pgfqpoint{1.924917in}{1.049609in}}%
\pgfpathlineto{\pgfqpoint{1.976416in}{1.031374in}}%
\pgfpathlineto{\pgfqpoint{2.027915in}{1.014481in}}%
\pgfpathlineto{\pgfqpoint{2.079415in}{1.011754in}}%
\pgfpathlineto{\pgfqpoint{2.130914in}{1.008269in}}%
\pgfpathlineto{\pgfqpoint{2.182413in}{0.992049in}}%
\pgfpathlineto{\pgfqpoint{2.233913in}{0.965657in}}%
\pgfpathlineto{\pgfqpoint{2.285412in}{0.973177in}}%
\pgfpathlineto{\pgfqpoint{2.336911in}{0.964356in}}%
\pgfpathlineto{\pgfqpoint{2.388411in}{0.952908in}}%
\pgfpathlineto{\pgfqpoint{2.439910in}{0.953597in}}%
\pgfpathlineto{\pgfqpoint{2.491409in}{0.933530in}}%
\pgfpathlineto{\pgfqpoint{2.542909in}{0.931384in}}%
\pgfpathlineto{\pgfqpoint{2.594408in}{0.917119in}}%
\pgfpathlineto{\pgfqpoint{2.645907in}{0.911078in}}%
\pgfpathlineto{\pgfqpoint{2.697406in}{0.904893in}}%
\pgfpathlineto{\pgfqpoint{2.748906in}{0.895455in}}%
\pgfpathlineto{\pgfqpoint{2.800405in}{0.879132in}}%
\pgfpathlineto{\pgfqpoint{2.851904in}{0.864010in}}%
\pgfpathlineto{\pgfqpoint{2.903404in}{0.871956in}}%
\pgfpathlineto{\pgfqpoint{2.954903in}{0.845396in}}%
\pgfpathlineto{\pgfqpoint{3.006402in}{0.853883in}}%
\pgfpathlineto{\pgfqpoint{3.057902in}{0.837215in}}%
\pgfpathlineto{\pgfqpoint{3.109401in}{0.831885in}}%
\pgfpathlineto{\pgfqpoint{3.160900in}{0.817616in}}%
\pgfpathlineto{\pgfqpoint{3.212400in}{0.806829in}}%
\pgfpathlineto{\pgfqpoint{3.263899in}{0.799679in}}%
\pgfpathlineto{\pgfqpoint{3.315398in}{0.789624in}}%
\pgfpathlineto{\pgfqpoint{3.366897in}{0.781219in}}%
\pgfpathlineto{\pgfqpoint{3.418397in}{0.768105in}}%
\pgfpathlineto{\pgfqpoint{3.469896in}{0.780317in}}%
\pgfpathlineto{\pgfqpoint{3.521395in}{0.751607in}}%
\pgfpathlineto{\pgfqpoint{3.572895in}{0.731832in}}%
\pgfpathlineto{\pgfqpoint{3.624394in}{0.740354in}}%
\pgfpathlineto{\pgfqpoint{3.675893in}{0.721168in}}%
\pgfpathlineto{\pgfqpoint{3.727393in}{0.729691in}}%
\pgfpathlineto{\pgfqpoint{3.778892in}{0.721941in}}%
\pgfpathlineto{\pgfqpoint{3.830391in}{0.691625in}}%
\pgfpathlineto{\pgfqpoint{3.881891in}{0.696227in}}%
\pgfpathlineto{\pgfqpoint{3.933390in}{0.690713in}}%
\pgfpathlineto{\pgfqpoint{3.984889in}{0.691268in}}%
\pgfpathlineto{\pgfqpoint{4.036389in}{0.669097in}}%
\pgfusepath{stroke}%
\end{pgfscope}%
\begin{pgfscope}%
\pgfpathrectangle{\pgfqpoint{0.740433in}{0.566590in}}{\pgfqpoint{3.295956in}{1.828724in}}%
\pgfusepath{clip}%
\pgfsetbuttcap%
\pgfsetroundjoin%
\definecolor{currentfill}{rgb}{0.850000,0.324000,0.098000}%
\pgfsetfillcolor{currentfill}%
\pgfsetlinewidth{1.003750pt}%
\definecolor{currentstroke}{rgb}{0.850000,0.324000,0.098000}%
\pgfsetstrokecolor{currentstroke}%
\pgfsetdash{}{0pt}%
\pgfsys@defobject{currentmarker}{\pgfqpoint{-0.041667in}{-0.041667in}}{\pgfqpoint{0.041667in}{0.041667in}}{%
\pgfpathmoveto{\pgfqpoint{-0.041667in}{0.000000in}}%
\pgfpathlineto{\pgfqpoint{0.041667in}{0.000000in}}%
\pgfpathmoveto{\pgfqpoint{0.000000in}{-0.041667in}}%
\pgfpathlineto{\pgfqpoint{0.000000in}{0.041667in}}%
\pgfusepath{stroke,fill}%
}%
\begin{pgfscope}%
\pgfsys@transformshift{0.740433in}{1.578208in}%
\pgfsys@useobject{currentmarker}{}%
\end{pgfscope}%
\begin{pgfscope}%
\pgfsys@transformshift{0.894931in}{1.493345in}%
\pgfsys@useobject{currentmarker}{}%
\end{pgfscope}%
\begin{pgfscope}%
\pgfsys@transformshift{1.049429in}{1.407107in}%
\pgfsys@useobject{currentmarker}{}%
\end{pgfscope}%
\begin{pgfscope}%
\pgfsys@transformshift{1.203926in}{1.286540in}%
\pgfsys@useobject{currentmarker}{}%
\end{pgfscope}%
\begin{pgfscope}%
\pgfsys@transformshift{1.358424in}{1.205467in}%
\pgfsys@useobject{currentmarker}{}%
\end{pgfscope}%
\begin{pgfscope}%
\pgfsys@transformshift{1.512922in}{1.139482in}%
\pgfsys@useobject{currentmarker}{}%
\end{pgfscope}%
\begin{pgfscope}%
\pgfsys@transformshift{1.667420in}{1.106290in}%
\pgfsys@useobject{currentmarker}{}%
\end{pgfscope}%
\begin{pgfscope}%
\pgfsys@transformshift{1.821918in}{1.067060in}%
\pgfsys@useobject{currentmarker}{}%
\end{pgfscope}%
\begin{pgfscope}%
\pgfsys@transformshift{1.976416in}{1.031374in}%
\pgfsys@useobject{currentmarker}{}%
\end{pgfscope}%
\begin{pgfscope}%
\pgfsys@transformshift{2.130914in}{1.008269in}%
\pgfsys@useobject{currentmarker}{}%
\end{pgfscope}%
\begin{pgfscope}%
\pgfsys@transformshift{2.285412in}{0.973177in}%
\pgfsys@useobject{currentmarker}{}%
\end{pgfscope}%
\begin{pgfscope}%
\pgfsys@transformshift{2.439910in}{0.953597in}%
\pgfsys@useobject{currentmarker}{}%
\end{pgfscope}%
\begin{pgfscope}%
\pgfsys@transformshift{2.594408in}{0.917119in}%
\pgfsys@useobject{currentmarker}{}%
\end{pgfscope}%
\begin{pgfscope}%
\pgfsys@transformshift{2.748906in}{0.895455in}%
\pgfsys@useobject{currentmarker}{}%
\end{pgfscope}%
\begin{pgfscope}%
\pgfsys@transformshift{2.903404in}{0.871956in}%
\pgfsys@useobject{currentmarker}{}%
\end{pgfscope}%
\begin{pgfscope}%
\pgfsys@transformshift{3.057902in}{0.837215in}%
\pgfsys@useobject{currentmarker}{}%
\end{pgfscope}%
\begin{pgfscope}%
\pgfsys@transformshift{3.212400in}{0.806829in}%
\pgfsys@useobject{currentmarker}{}%
\end{pgfscope}%
\begin{pgfscope}%
\pgfsys@transformshift{3.366897in}{0.781219in}%
\pgfsys@useobject{currentmarker}{}%
\end{pgfscope}%
\begin{pgfscope}%
\pgfsys@transformshift{3.521395in}{0.751607in}%
\pgfsys@useobject{currentmarker}{}%
\end{pgfscope}%
\begin{pgfscope}%
\pgfsys@transformshift{3.675893in}{0.721168in}%
\pgfsys@useobject{currentmarker}{}%
\end{pgfscope}%
\begin{pgfscope}%
\pgfsys@transformshift{3.830391in}{0.691625in}%
\pgfsys@useobject{currentmarker}{}%
\end{pgfscope}%
\begin{pgfscope}%
\pgfsys@transformshift{3.984889in}{0.691268in}%
\pgfsys@useobject{currentmarker}{}%
\end{pgfscope}%
\end{pgfscope}%
\begin{pgfscope}%
\pgfpathrectangle{\pgfqpoint{0.740433in}{0.566590in}}{\pgfqpoint{3.295956in}{1.828724in}}%
\pgfusepath{clip}%
\pgfsetrectcap%
\pgfsetroundjoin%
\pgfsetlinewidth{1.505625pt}%
\definecolor{currentstroke}{rgb}{0.000000,0.500000,0.000000}%
\pgfsetstrokecolor{currentstroke}%
\pgfsetdash{}{0pt}%
\pgfpathmoveto{\pgfqpoint{0.740433in}{1.577935in}}%
\pgfpathlineto{\pgfqpoint{0.791932in}{1.570503in}}%
\pgfpathlineto{\pgfqpoint{0.843431in}{1.475447in}}%
\pgfpathlineto{\pgfqpoint{0.894931in}{1.481282in}}%
\pgfpathlineto{\pgfqpoint{0.946430in}{1.424590in}}%
\pgfpathlineto{\pgfqpoint{0.997929in}{1.398253in}}%
\pgfpathlineto{\pgfqpoint{1.049429in}{1.476520in}}%
\pgfpathlineto{\pgfqpoint{1.100928in}{1.373634in}}%
\pgfpathlineto{\pgfqpoint{1.152427in}{1.309273in}}%
\pgfpathlineto{\pgfqpoint{1.203926in}{1.461687in}}%
\pgfpathlineto{\pgfqpoint{1.255426in}{1.236835in}}%
\pgfpathlineto{\pgfqpoint{1.306925in}{1.235882in}}%
\pgfpathlineto{\pgfqpoint{1.358424in}{1.212774in}}%
\pgfpathlineto{\pgfqpoint{1.409924in}{1.195294in}}%
\pgfpathlineto{\pgfqpoint{1.461423in}{1.156535in}}%
\pgfpathlineto{\pgfqpoint{1.512922in}{1.149414in}}%
\pgfpathlineto{\pgfqpoint{1.564422in}{1.125585in}}%
\pgfpathlineto{\pgfqpoint{1.615921in}{1.148160in}}%
\pgfpathlineto{\pgfqpoint{1.667420in}{1.122526in}}%
\pgfpathlineto{\pgfqpoint{1.718920in}{1.081282in}}%
\pgfpathlineto{\pgfqpoint{1.770419in}{1.084930in}}%
\pgfpathlineto{\pgfqpoint{1.821918in}{1.077125in}}%
\pgfpathlineto{\pgfqpoint{1.873417in}{1.069324in}}%
\pgfpathlineto{\pgfqpoint{1.924917in}{1.047499in}}%
\pgfpathlineto{\pgfqpoint{1.976416in}{1.035979in}}%
\pgfpathlineto{\pgfqpoint{2.027915in}{1.021304in}}%
\pgfpathlineto{\pgfqpoint{2.079415in}{1.021387in}}%
\pgfpathlineto{\pgfqpoint{2.130914in}{1.026062in}}%
\pgfpathlineto{\pgfqpoint{2.182413in}{1.006747in}}%
\pgfpathlineto{\pgfqpoint{2.233913in}{0.976885in}}%
\pgfpathlineto{\pgfqpoint{2.285412in}{0.995029in}}%
\pgfpathlineto{\pgfqpoint{2.336911in}{0.982554in}}%
\pgfpathlineto{\pgfqpoint{2.388411in}{0.960245in}}%
\pgfpathlineto{\pgfqpoint{2.439910in}{0.958306in}}%
\pgfpathlineto{\pgfqpoint{2.491409in}{0.944448in}}%
\pgfpathlineto{\pgfqpoint{2.542909in}{0.942842in}}%
\pgfpathlineto{\pgfqpoint{2.594408in}{0.920451in}}%
\pgfpathlineto{\pgfqpoint{2.645907in}{0.920498in}}%
\pgfpathlineto{\pgfqpoint{2.697406in}{0.921661in}}%
\pgfpathlineto{\pgfqpoint{2.748906in}{0.901137in}}%
\pgfpathlineto{\pgfqpoint{2.800405in}{0.900836in}}%
\pgfpathlineto{\pgfqpoint{2.851904in}{0.881308in}}%
\pgfpathlineto{\pgfqpoint{2.903404in}{0.874286in}}%
\pgfpathlineto{\pgfqpoint{2.954903in}{0.863304in}}%
\pgfpathlineto{\pgfqpoint{3.006402in}{0.864554in}}%
\pgfpathlineto{\pgfqpoint{3.057902in}{0.850716in}}%
\pgfpathlineto{\pgfqpoint{3.109401in}{0.836834in}}%
\pgfpathlineto{\pgfqpoint{3.160900in}{0.820125in}}%
\pgfpathlineto{\pgfqpoint{3.212400in}{0.814872in}}%
\pgfpathlineto{\pgfqpoint{3.263899in}{0.812400in}}%
\pgfpathlineto{\pgfqpoint{3.315398in}{0.801743in}}%
\pgfpathlineto{\pgfqpoint{3.366897in}{0.800185in}}%
\pgfpathlineto{\pgfqpoint{3.418397in}{0.788520in}}%
\pgfpathlineto{\pgfqpoint{3.469896in}{0.778921in}}%
\pgfpathlineto{\pgfqpoint{3.521395in}{0.777280in}}%
\pgfpathlineto{\pgfqpoint{3.572895in}{0.749821in}}%
\pgfpathlineto{\pgfqpoint{3.624394in}{0.749302in}}%
\pgfpathlineto{\pgfqpoint{3.675893in}{0.740512in}}%
\pgfpathlineto{\pgfqpoint{3.727393in}{0.735027in}}%
\pgfpathlineto{\pgfqpoint{3.778892in}{0.728206in}}%
\pgfpathlineto{\pgfqpoint{3.830391in}{0.712709in}}%
\pgfpathlineto{\pgfqpoint{3.881891in}{0.708673in}}%
\pgfpathlineto{\pgfqpoint{3.933390in}{0.690917in}}%
\pgfpathlineto{\pgfqpoint{3.984889in}{0.699236in}}%
\pgfpathlineto{\pgfqpoint{4.036389in}{0.673067in}}%
\pgfusepath{stroke}%
\end{pgfscope}%
\begin{pgfscope}%
\pgfpathrectangle{\pgfqpoint{0.740433in}{0.566590in}}{\pgfqpoint{3.295956in}{1.828724in}}%
\pgfusepath{clip}%
\pgfsetbuttcap%
\pgfsetmiterjoin%
\definecolor{currentfill}{rgb}{0.000000,0.000000,0.000000}%
\pgfsetfillcolor{currentfill}%
\pgfsetfillopacity{0.000000}%
\pgfsetlinewidth{1.003750pt}%
\definecolor{currentstroke}{rgb}{0.000000,0.500000,0.000000}%
\pgfsetstrokecolor{currentstroke}%
\pgfsetdash{}{0pt}%
\pgfsys@defobject{currentmarker}{\pgfqpoint{-0.041667in}{-0.041667in}}{\pgfqpoint{0.041667in}{0.041667in}}{%
\pgfpathmoveto{\pgfqpoint{-0.041667in}{-0.041667in}}%
\pgfpathlineto{\pgfqpoint{0.041667in}{-0.041667in}}%
\pgfpathlineto{\pgfqpoint{0.041667in}{0.041667in}}%
\pgfpathlineto{\pgfqpoint{-0.041667in}{0.041667in}}%
\pgfpathclose%
\pgfusepath{stroke,fill}%
}%
\begin{pgfscope}%
\pgfsys@transformshift{0.740433in}{1.577935in}%
\pgfsys@useobject{currentmarker}{}%
\end{pgfscope}%
\begin{pgfscope}%
\pgfsys@transformshift{0.997929in}{1.398253in}%
\pgfsys@useobject{currentmarker}{}%
\end{pgfscope}%
\begin{pgfscope}%
\pgfsys@transformshift{1.255426in}{1.236835in}%
\pgfsys@useobject{currentmarker}{}%
\end{pgfscope}%
\begin{pgfscope}%
\pgfsys@transformshift{1.512922in}{1.149414in}%
\pgfsys@useobject{currentmarker}{}%
\end{pgfscope}%
\begin{pgfscope}%
\pgfsys@transformshift{1.770419in}{1.084930in}%
\pgfsys@useobject{currentmarker}{}%
\end{pgfscope}%
\begin{pgfscope}%
\pgfsys@transformshift{2.027915in}{1.021304in}%
\pgfsys@useobject{currentmarker}{}%
\end{pgfscope}%
\begin{pgfscope}%
\pgfsys@transformshift{2.285412in}{0.995029in}%
\pgfsys@useobject{currentmarker}{}%
\end{pgfscope}%
\begin{pgfscope}%
\pgfsys@transformshift{2.542909in}{0.942842in}%
\pgfsys@useobject{currentmarker}{}%
\end{pgfscope}%
\begin{pgfscope}%
\pgfsys@transformshift{2.800405in}{0.900836in}%
\pgfsys@useobject{currentmarker}{}%
\end{pgfscope}%
\begin{pgfscope}%
\pgfsys@transformshift{3.057902in}{0.850716in}%
\pgfsys@useobject{currentmarker}{}%
\end{pgfscope}%
\begin{pgfscope}%
\pgfsys@transformshift{3.315398in}{0.801743in}%
\pgfsys@useobject{currentmarker}{}%
\end{pgfscope}%
\begin{pgfscope}%
\pgfsys@transformshift{3.572895in}{0.749821in}%
\pgfsys@useobject{currentmarker}{}%
\end{pgfscope}%
\begin{pgfscope}%
\pgfsys@transformshift{3.830391in}{0.712709in}%
\pgfsys@useobject{currentmarker}{}%
\end{pgfscope}%
\end{pgfscope}%
\begin{pgfscope}%
\pgfpathrectangle{\pgfqpoint{0.740433in}{0.566590in}}{\pgfqpoint{3.295956in}{1.828724in}}%
\pgfusepath{clip}%
\pgfsetrectcap%
\pgfsetroundjoin%
\pgfsetlinewidth{1.505625pt}%
\definecolor{currentstroke}{rgb}{0.494000,0.184000,0.556000}%
\pgfsetstrokecolor{currentstroke}%
\pgfsetdash{}{0pt}%
\pgfpathmoveto{\pgfqpoint{0.740433in}{1.660776in}}%
\pgfpathlineto{\pgfqpoint{0.791932in}{1.654240in}}%
\pgfpathlineto{\pgfqpoint{0.843431in}{1.614137in}}%
\pgfpathlineto{\pgfqpoint{0.894931in}{1.531579in}}%
\pgfpathlineto{\pgfqpoint{0.946430in}{1.604957in}}%
\pgfpathlineto{\pgfqpoint{0.997929in}{1.527487in}}%
\pgfpathlineto{\pgfqpoint{1.049429in}{1.855892in}}%
\pgfpathlineto{\pgfqpoint{1.100928in}{1.414660in}}%
\pgfpathlineto{\pgfqpoint{1.152427in}{1.263340in}}%
\pgfpathlineto{\pgfqpoint{1.203926in}{1.365593in}}%
\pgfpathlineto{\pgfqpoint{1.255426in}{1.205910in}}%
\pgfpathlineto{\pgfqpoint{1.306925in}{1.176877in}}%
\pgfpathlineto{\pgfqpoint{1.358424in}{1.168008in}}%
\pgfpathlineto{\pgfqpoint{1.409924in}{1.178729in}}%
\pgfpathlineto{\pgfqpoint{1.461423in}{1.132653in}}%
\pgfpathlineto{\pgfqpoint{1.512922in}{1.138408in}}%
\pgfpathlineto{\pgfqpoint{1.564422in}{1.133508in}}%
\pgfpathlineto{\pgfqpoint{1.615921in}{1.111781in}}%
\pgfpathlineto{\pgfqpoint{1.667420in}{1.100606in}}%
\pgfpathlineto{\pgfqpoint{1.718920in}{1.073572in}}%
\pgfpathlineto{\pgfqpoint{1.770419in}{1.063099in}}%
\pgfpathlineto{\pgfqpoint{1.821918in}{1.057079in}}%
\pgfpathlineto{\pgfqpoint{1.873417in}{1.045046in}}%
\pgfpathlineto{\pgfqpoint{1.924917in}{1.050340in}}%
\pgfpathlineto{\pgfqpoint{1.976416in}{1.015611in}}%
\pgfpathlineto{\pgfqpoint{2.027915in}{1.006075in}}%
\pgfpathlineto{\pgfqpoint{2.079415in}{0.998790in}}%
\pgfpathlineto{\pgfqpoint{2.130914in}{0.984915in}}%
\pgfpathlineto{\pgfqpoint{2.182413in}{0.973080in}}%
\pgfpathlineto{\pgfqpoint{2.233913in}{0.961722in}}%
\pgfpathlineto{\pgfqpoint{2.285412in}{0.952883in}}%
\pgfpathlineto{\pgfqpoint{2.336911in}{0.934417in}}%
\pgfpathlineto{\pgfqpoint{2.388411in}{0.929068in}}%
\pgfpathlineto{\pgfqpoint{2.439910in}{0.930195in}}%
\pgfpathlineto{\pgfqpoint{2.491409in}{0.913738in}}%
\pgfpathlineto{\pgfqpoint{2.542909in}{0.908369in}}%
\pgfpathlineto{\pgfqpoint{2.594408in}{0.906875in}}%
\pgfpathlineto{\pgfqpoint{2.645907in}{0.882457in}}%
\pgfpathlineto{\pgfqpoint{2.697406in}{0.884473in}}%
\pgfpathlineto{\pgfqpoint{2.748906in}{0.874380in}}%
\pgfpathlineto{\pgfqpoint{2.800405in}{0.870102in}}%
\pgfpathlineto{\pgfqpoint{2.851904in}{0.859182in}}%
\pgfpathlineto{\pgfqpoint{2.903404in}{0.848390in}}%
\pgfpathlineto{\pgfqpoint{2.954903in}{0.848450in}}%
\pgfpathlineto{\pgfqpoint{3.006402in}{0.820098in}}%
\pgfpathlineto{\pgfqpoint{3.057902in}{0.816103in}}%
\pgfpathlineto{\pgfqpoint{3.109401in}{0.816713in}}%
\pgfpathlineto{\pgfqpoint{3.160900in}{0.806028in}}%
\pgfpathlineto{\pgfqpoint{3.212400in}{0.789473in}}%
\pgfpathlineto{\pgfqpoint{3.263899in}{0.787515in}}%
\pgfpathlineto{\pgfqpoint{3.315398in}{0.777272in}}%
\pgfpathlineto{\pgfqpoint{3.366897in}{0.764249in}}%
\pgfpathlineto{\pgfqpoint{3.418397in}{0.752278in}}%
\pgfpathlineto{\pgfqpoint{3.469896in}{0.741691in}}%
\pgfpathlineto{\pgfqpoint{3.521395in}{0.740391in}}%
\pgfpathlineto{\pgfqpoint{3.572895in}{0.715789in}}%
\pgfpathlineto{\pgfqpoint{3.624394in}{0.722412in}}%
\pgfpathlineto{\pgfqpoint{3.675893in}{0.698780in}}%
\pgfpathlineto{\pgfqpoint{3.727393in}{0.706688in}}%
\pgfpathlineto{\pgfqpoint{3.778892in}{0.697758in}}%
\pgfpathlineto{\pgfqpoint{3.830391in}{0.677887in}}%
\pgfpathlineto{\pgfqpoint{3.881891in}{0.672137in}}%
\pgfpathlineto{\pgfqpoint{3.933390in}{0.655081in}}%
\pgfpathlineto{\pgfqpoint{3.984889in}{0.660574in}}%
\pgfpathlineto{\pgfqpoint{4.036389in}{0.649714in}}%
\pgfusepath{stroke}%
\end{pgfscope}%
\begin{pgfscope}%
\pgfpathrectangle{\pgfqpoint{0.740433in}{0.566590in}}{\pgfqpoint{3.295956in}{1.828724in}}%
\pgfusepath{clip}%
\pgfsetbuttcap%
\pgfsetroundjoin%
\definecolor{currentfill}{rgb}{0.494000,0.184000,0.556000}%
\pgfsetfillcolor{currentfill}%
\pgfsetlinewidth{1.003750pt}%
\definecolor{currentstroke}{rgb}{0.494000,0.184000,0.556000}%
\pgfsetstrokecolor{currentstroke}%
\pgfsetdash{}{0pt}%
\pgfsys@defobject{currentmarker}{\pgfqpoint{-0.041667in}{-0.041667in}}{\pgfqpoint{0.041667in}{0.041667in}}{%
\pgfpathmoveto{\pgfqpoint{-0.041667in}{-0.041667in}}%
\pgfpathlineto{\pgfqpoint{0.041667in}{0.041667in}}%
\pgfpathmoveto{\pgfqpoint{-0.041667in}{0.041667in}}%
\pgfpathlineto{\pgfqpoint{0.041667in}{-0.041667in}}%
\pgfusepath{stroke,fill}%
}%
\begin{pgfscope}%
\pgfsys@transformshift{0.740433in}{1.660776in}%
\pgfsys@useobject{currentmarker}{}%
\end{pgfscope}%
\begin{pgfscope}%
\pgfsys@transformshift{0.946430in}{1.604957in}%
\pgfsys@useobject{currentmarker}{}%
\end{pgfscope}%
\begin{pgfscope}%
\pgfsys@transformshift{1.152427in}{1.263340in}%
\pgfsys@useobject{currentmarker}{}%
\end{pgfscope}%
\begin{pgfscope}%
\pgfsys@transformshift{1.358424in}{1.168008in}%
\pgfsys@useobject{currentmarker}{}%
\end{pgfscope}%
\begin{pgfscope}%
\pgfsys@transformshift{1.564422in}{1.133508in}%
\pgfsys@useobject{currentmarker}{}%
\end{pgfscope}%
\begin{pgfscope}%
\pgfsys@transformshift{1.770419in}{1.063099in}%
\pgfsys@useobject{currentmarker}{}%
\end{pgfscope}%
\begin{pgfscope}%
\pgfsys@transformshift{1.976416in}{1.015611in}%
\pgfsys@useobject{currentmarker}{}%
\end{pgfscope}%
\begin{pgfscope}%
\pgfsys@transformshift{2.182413in}{0.973080in}%
\pgfsys@useobject{currentmarker}{}%
\end{pgfscope}%
\begin{pgfscope}%
\pgfsys@transformshift{2.388411in}{0.929068in}%
\pgfsys@useobject{currentmarker}{}%
\end{pgfscope}%
\begin{pgfscope}%
\pgfsys@transformshift{2.594408in}{0.906875in}%
\pgfsys@useobject{currentmarker}{}%
\end{pgfscope}%
\begin{pgfscope}%
\pgfsys@transformshift{2.800405in}{0.870102in}%
\pgfsys@useobject{currentmarker}{}%
\end{pgfscope}%
\begin{pgfscope}%
\pgfsys@transformshift{3.006402in}{0.820098in}%
\pgfsys@useobject{currentmarker}{}%
\end{pgfscope}%
\begin{pgfscope}%
\pgfsys@transformshift{3.212400in}{0.789473in}%
\pgfsys@useobject{currentmarker}{}%
\end{pgfscope}%
\begin{pgfscope}%
\pgfsys@transformshift{3.418397in}{0.752278in}%
\pgfsys@useobject{currentmarker}{}%
\end{pgfscope}%
\begin{pgfscope}%
\pgfsys@transformshift{3.624394in}{0.722412in}%
\pgfsys@useobject{currentmarker}{}%
\end{pgfscope}%
\begin{pgfscope}%
\pgfsys@transformshift{3.830391in}{0.677887in}%
\pgfsys@useobject{currentmarker}{}%
\end{pgfscope}%
\begin{pgfscope}%
\pgfsys@transformshift{4.036389in}{0.649714in}%
\pgfsys@useobject{currentmarker}{}%
\end{pgfscope}%
\end{pgfscope}%
\begin{pgfscope}%
\pgfsetrectcap%
\pgfsetmiterjoin%
\pgfsetlinewidth{0.803000pt}%
\definecolor{currentstroke}{rgb}{0.000000,0.000000,0.000000}%
\pgfsetstrokecolor{currentstroke}%
\pgfsetdash{}{0pt}%
\pgfpathmoveto{\pgfqpoint{0.740433in}{0.566590in}}%
\pgfpathlineto{\pgfqpoint{0.740433in}{2.395314in}}%
\pgfusepath{stroke}%
\end{pgfscope}%
\begin{pgfscope}%
\pgfsetrectcap%
\pgfsetmiterjoin%
\pgfsetlinewidth{0.803000pt}%
\definecolor{currentstroke}{rgb}{0.000000,0.000000,0.000000}%
\pgfsetstrokecolor{currentstroke}%
\pgfsetdash{}{0pt}%
\pgfpathmoveto{\pgfqpoint{4.036389in}{0.566590in}}%
\pgfpathlineto{\pgfqpoint{4.036389in}{2.395314in}}%
\pgfusepath{stroke}%
\end{pgfscope}%
\begin{pgfscope}%
\pgfsetrectcap%
\pgfsetmiterjoin%
\pgfsetlinewidth{0.803000pt}%
\definecolor{currentstroke}{rgb}{0.000000,0.000000,0.000000}%
\pgfsetstrokecolor{currentstroke}%
\pgfsetdash{}{0pt}%
\pgfpathmoveto{\pgfqpoint{0.740433in}{0.566590in}}%
\pgfpathlineto{\pgfqpoint{4.036389in}{0.566590in}}%
\pgfusepath{stroke}%
\end{pgfscope}%
\begin{pgfscope}%
\pgfsetrectcap%
\pgfsetmiterjoin%
\pgfsetlinewidth{0.803000pt}%
\definecolor{currentstroke}{rgb}{0.000000,0.000000,0.000000}%
\pgfsetstrokecolor{currentstroke}%
\pgfsetdash{}{0pt}%
\pgfpathmoveto{\pgfqpoint{0.740433in}{2.395314in}}%
\pgfpathlineto{\pgfqpoint{4.036389in}{2.395314in}}%
\pgfusepath{stroke}%
\end{pgfscope}%
\begin{pgfscope}%
\pgfsetbuttcap%
\pgfsetmiterjoin%
\definecolor{currentfill}{rgb}{1.000000,1.000000,1.000000}%
\pgfsetfillcolor{currentfill}%
\pgfsetfillopacity{0.800000}%
\pgfsetlinewidth{1.003750pt}%
\definecolor{currentstroke}{rgb}{0.800000,0.800000,0.800000}%
\pgfsetstrokecolor{currentstroke}%
\pgfsetstrokeopacity{0.800000}%
\pgfsetdash{}{0pt}%
\pgfpathmoveto{\pgfqpoint{2.906752in}{1.598116in}}%
\pgfpathlineto{\pgfqpoint{3.948889in}{1.598116in}}%
\pgfpathquadraticcurveto{\pgfqpoint{3.973889in}{1.598116in}}{\pgfqpoint{3.973889in}{1.623116in}}%
\pgfpathlineto{\pgfqpoint{3.973889in}{2.307814in}}%
\pgfpathquadraticcurveto{\pgfqpoint{3.973889in}{2.332814in}}{\pgfqpoint{3.948889in}{2.332814in}}%
\pgfpathlineto{\pgfqpoint{2.906752in}{2.332814in}}%
\pgfpathquadraticcurveto{\pgfqpoint{2.881752in}{2.332814in}}{\pgfqpoint{2.881752in}{2.307814in}}%
\pgfpathlineto{\pgfqpoint{2.881752in}{1.623116in}}%
\pgfpathquadraticcurveto{\pgfqpoint{2.881752in}{1.598116in}}{\pgfqpoint{2.906752in}{1.598116in}}%
\pgfpathclose%
\pgfusepath{stroke,fill}%
\end{pgfscope}%
\begin{pgfscope}%
\pgfsetbuttcap%
\pgfsetroundjoin%
\definecolor{currentfill}{rgb}{0.000000,0.000000,0.000000}%
\pgfsetfillcolor{currentfill}%
\pgfsetfillopacity{0.000000}%
\pgfsetlinewidth{1.003750pt}%
\definecolor{currentstroke}{rgb}{0.000000,0.447000,0.741000}%
\pgfsetstrokecolor{currentstroke}%
\pgfsetdash{}{0pt}%
\pgfsys@defobject{currentmarker}{\pgfqpoint{-0.041667in}{-0.041667in}}{\pgfqpoint{0.041667in}{0.041667in}}{%
\pgfpathmoveto{\pgfqpoint{0.000000in}{-0.041667in}}%
\pgfpathcurveto{\pgfqpoint{0.011050in}{-0.041667in}}{\pgfqpoint{0.021649in}{-0.037276in}}{\pgfqpoint{0.029463in}{-0.029463in}}%
\pgfpathcurveto{\pgfqpoint{0.037276in}{-0.021649in}}{\pgfqpoint{0.041667in}{-0.011050in}}{\pgfqpoint{0.041667in}{0.000000in}}%
\pgfpathcurveto{\pgfqpoint{0.041667in}{0.011050in}}{\pgfqpoint{0.037276in}{0.021649in}}{\pgfqpoint{0.029463in}{0.029463in}}%
\pgfpathcurveto{\pgfqpoint{0.021649in}{0.037276in}}{\pgfqpoint{0.011050in}{0.041667in}}{\pgfqpoint{0.000000in}{0.041667in}}%
\pgfpathcurveto{\pgfqpoint{-0.011050in}{0.041667in}}{\pgfqpoint{-0.021649in}{0.037276in}}{\pgfqpoint{-0.029463in}{0.029463in}}%
\pgfpathcurveto{\pgfqpoint{-0.037276in}{0.021649in}}{\pgfqpoint{-0.041667in}{0.011050in}}{\pgfqpoint{-0.041667in}{0.000000in}}%
\pgfpathcurveto{\pgfqpoint{-0.041667in}{-0.011050in}}{\pgfqpoint{-0.037276in}{-0.021649in}}{\pgfqpoint{-0.029463in}{-0.029463in}}%
\pgfpathcurveto{\pgfqpoint{-0.021649in}{-0.037276in}}{\pgfqpoint{-0.011050in}{-0.041667in}}{\pgfqpoint{0.000000in}{-0.041667in}}%
\pgfpathclose%
\pgfusepath{stroke,fill}%
}%
\begin{pgfscope}%
\pgfsys@transformshift{3.056752in}{2.239064in}%
\pgfsys@useobject{currentmarker}{}%
\end{pgfscope}%
\end{pgfscope}%
\begin{pgfscope}%
\definecolor{textcolor}{rgb}{0.000000,0.000000,0.000000}%
\pgfsetstrokecolor{textcolor}%
\pgfsetfillcolor{textcolor}%
\pgftext[x=3.281752in,y=2.195314in,left,base]{\color{textcolor}\rmfamily\fontsize{9.000000}{10.800000}\selectfont \(\displaystyle \nu_1 =\) -7.68 }%
\end{pgfscope}%
\begin{pgfscope}%
\pgfsetbuttcap%
\pgfsetroundjoin%
\definecolor{currentfill}{rgb}{0.850000,0.324000,0.098000}%
\pgfsetfillcolor{currentfill}%
\pgfsetlinewidth{1.003750pt}%
\definecolor{currentstroke}{rgb}{0.850000,0.324000,0.098000}%
\pgfsetstrokecolor{currentstroke}%
\pgfsetdash{}{0pt}%
\pgfsys@defobject{currentmarker}{\pgfqpoint{-0.041667in}{-0.041667in}}{\pgfqpoint{0.041667in}{0.041667in}}{%
\pgfpathmoveto{\pgfqpoint{-0.041667in}{0.000000in}}%
\pgfpathlineto{\pgfqpoint{0.041667in}{0.000000in}}%
\pgfpathmoveto{\pgfqpoint{0.000000in}{-0.041667in}}%
\pgfpathlineto{\pgfqpoint{0.000000in}{0.041667in}}%
\pgfusepath{stroke,fill}%
}%
\begin{pgfscope}%
\pgfsys@transformshift{3.056752in}{2.064765in}%
\pgfsys@useobject{currentmarker}{}%
\end{pgfscope}%
\end{pgfscope}%
\begin{pgfscope}%
\definecolor{textcolor}{rgb}{0.000000,0.000000,0.000000}%
\pgfsetstrokecolor{textcolor}%
\pgfsetfillcolor{textcolor}%
\pgftext[x=3.281752in,y=2.021015in,left,base]{\color{textcolor}\rmfamily\fontsize{9.000000}{10.800000}\selectfont \(\displaystyle \nu_2 =\) 39.68}%
\end{pgfscope}%
\begin{pgfscope}%
\pgfsetbuttcap%
\pgfsetmiterjoin%
\definecolor{currentfill}{rgb}{0.000000,0.000000,0.000000}%
\pgfsetfillcolor{currentfill}%
\pgfsetfillopacity{0.000000}%
\pgfsetlinewidth{1.003750pt}%
\definecolor{currentstroke}{rgb}{0.000000,0.500000,0.000000}%
\pgfsetstrokecolor{currentstroke}%
\pgfsetdash{}{0pt}%
\pgfsys@defobject{currentmarker}{\pgfqpoint{-0.041667in}{-0.041667in}}{\pgfqpoint{0.041667in}{0.041667in}}{%
\pgfpathmoveto{\pgfqpoint{-0.041667in}{-0.041667in}}%
\pgfpathlineto{\pgfqpoint{0.041667in}{-0.041667in}}%
\pgfpathlineto{\pgfqpoint{0.041667in}{0.041667in}}%
\pgfpathlineto{\pgfqpoint{-0.041667in}{0.041667in}}%
\pgfpathclose%
\pgfusepath{stroke,fill}%
}%
\begin{pgfscope}%
\pgfsys@transformshift{3.056752in}{1.890465in}%
\pgfsys@useobject{currentmarker}{}%
\end{pgfscope}%
\end{pgfscope}%
\begin{pgfscope}%
\definecolor{textcolor}{rgb}{0.000000,0.000000,0.000000}%
\pgfsetstrokecolor{textcolor}%
\pgfsetfillcolor{textcolor}%
\pgftext[x=3.281752in,y=1.846715in,left,base]{\color{textcolor}\rmfamily\fontsize{9.000000}{10.800000}\selectfont \(\displaystyle \nu_3\) = 40.96 }%
\end{pgfscope}%
\begin{pgfscope}%
\pgfsetbuttcap%
\pgfsetroundjoin%
\definecolor{currentfill}{rgb}{0.494000,0.184000,0.556000}%
\pgfsetfillcolor{currentfill}%
\pgfsetlinewidth{1.003750pt}%
\definecolor{currentstroke}{rgb}{0.494000,0.184000,0.556000}%
\pgfsetstrokecolor{currentstroke}%
\pgfsetdash{}{0pt}%
\pgfsys@defobject{currentmarker}{\pgfqpoint{-0.041667in}{-0.041667in}}{\pgfqpoint{0.041667in}{0.041667in}}{%
\pgfpathmoveto{\pgfqpoint{-0.041667in}{-0.041667in}}%
\pgfpathlineto{\pgfqpoint{0.041667in}{0.041667in}}%
\pgfpathmoveto{\pgfqpoint{-0.041667in}{0.041667in}}%
\pgfpathlineto{\pgfqpoint{0.041667in}{-0.041667in}}%
\pgfusepath{stroke,fill}%
}%
\begin{pgfscope}%
\pgfsys@transformshift{3.056752in}{1.716165in}%
\pgfsys@useobject{currentmarker}{}%
\end{pgfscope}%
\end{pgfscope}%
\begin{pgfscope}%
\definecolor{textcolor}{rgb}{0.000000,0.000000,0.000000}%
\pgfsetstrokecolor{textcolor}%
\pgfsetfillcolor{textcolor}%
\pgftext[x=3.281752in,y=1.672415in,left,base]{\color{textcolor}\rmfamily\fontsize{9.000000}{10.800000}\selectfont \(\displaystyle \nu_4 = \) 99.84}%
\end{pgfscope}%
\end{pgfpicture}%
\makeatother%
\endgroup%
}
					\caption{RMSE frecuencias.}
					\label{Fig:RMSE_nu_ex1}
				\end{subfigure}
				~
				\begin{subfigure}{0.5\textwidth}
					\centering
					\resizebox{\linewidth}{!}{%% Creator: Matplotlib, PGF backend
%%
%% To include the figure in your LaTeX document, write
%%   \input{<filename>.pgf}
%%
%% Make sure the required packages are loaded in your preamble
%%   \usepackage{pgf}
%%
%% and, on pdftex
%%   \usepackage[utf8]{inputenc}\DeclareUnicodeCharacter{2212}{-}
%%
%% or, on luatex and xetex
%%   \usepackage{unicode-math}
%%
%% Figures using additional raster images can only be included by \input if
%% they are in the same directory as the main LaTeX file. For loading figures
%% from other directories you can use the `import` package
%%   \usepackage{import}
%%
%% and then include the figures with
%%   \import{<path to file>}{<filename>.pgf}
%%
%% Matplotlib used the following preamble
%%   \usepackage[utf8x]{inputenc}
%%   \usepackage[T1]{fontenc}
%%   \usepackage{amsmath,amssymb,amsfonts}
%%
\begingroup%
\makeatletter%
\begin{pgfpicture}%
\pgfpathrectangle{\pgfpointorigin}{\pgfqpoint{4.136389in}{2.495314in}}%
\pgfusepath{use as bounding box, clip}%
\begin{pgfscope}%
\pgfsetbuttcap%
\pgfsetmiterjoin%
\definecolor{currentfill}{rgb}{1.000000,1.000000,1.000000}%
\pgfsetfillcolor{currentfill}%
\pgfsetlinewidth{0.000000pt}%
\definecolor{currentstroke}{rgb}{1.000000,1.000000,1.000000}%
\pgfsetstrokecolor{currentstroke}%
\pgfsetdash{}{0pt}%
\pgfpathmoveto{\pgfqpoint{0.000000in}{0.000000in}}%
\pgfpathlineto{\pgfqpoint{4.136389in}{0.000000in}}%
\pgfpathlineto{\pgfqpoint{4.136389in}{2.495314in}}%
\pgfpathlineto{\pgfqpoint{0.000000in}{2.495314in}}%
\pgfpathclose%
\pgfusepath{fill}%
\end{pgfscope}%
\begin{pgfscope}%
\pgfsetbuttcap%
\pgfsetmiterjoin%
\definecolor{currentfill}{rgb}{1.000000,1.000000,1.000000}%
\pgfsetfillcolor{currentfill}%
\pgfsetlinewidth{0.000000pt}%
\definecolor{currentstroke}{rgb}{0.000000,0.000000,0.000000}%
\pgfsetstrokecolor{currentstroke}%
\pgfsetstrokeopacity{0.000000}%
\pgfsetdash{}{0pt}%
\pgfpathmoveto{\pgfqpoint{0.745371in}{0.566590in}}%
\pgfpathlineto{\pgfqpoint{4.036389in}{0.566590in}}%
\pgfpathlineto{\pgfqpoint{4.036389in}{2.395314in}}%
\pgfpathlineto{\pgfqpoint{0.745371in}{2.395314in}}%
\pgfpathclose%
\pgfusepath{fill}%
\end{pgfscope}%
\begin{pgfscope}%
\pgfpathrectangle{\pgfqpoint{0.745371in}{0.566590in}}{\pgfqpoint{3.291018in}{1.828724in}}%
\pgfusepath{clip}%
\pgfsetrectcap%
\pgfsetroundjoin%
\pgfsetlinewidth{0.803000pt}%
\definecolor{currentstroke}{rgb}{0.690196,0.690196,0.690196}%
\pgfsetstrokecolor{currentstroke}%
\pgfsetdash{}{0pt}%
\pgfpathmoveto{\pgfqpoint{0.745371in}{0.566590in}}%
\pgfpathlineto{\pgfqpoint{0.745371in}{2.395314in}}%
\pgfusepath{stroke}%
\end{pgfscope}%
\begin{pgfscope}%
\pgfsetbuttcap%
\pgfsetroundjoin%
\definecolor{currentfill}{rgb}{0.000000,0.000000,0.000000}%
\pgfsetfillcolor{currentfill}%
\pgfsetlinewidth{0.803000pt}%
\definecolor{currentstroke}{rgb}{0.000000,0.000000,0.000000}%
\pgfsetstrokecolor{currentstroke}%
\pgfsetdash{}{0pt}%
\pgfsys@defobject{currentmarker}{\pgfqpoint{0.000000in}{-0.048611in}}{\pgfqpoint{0.000000in}{0.000000in}}{%
\pgfpathmoveto{\pgfqpoint{0.000000in}{0.000000in}}%
\pgfpathlineto{\pgfqpoint{0.000000in}{-0.048611in}}%
\pgfusepath{stroke,fill}%
}%
\begin{pgfscope}%
\pgfsys@transformshift{0.745371in}{0.566590in}%
\pgfsys@useobject{currentmarker}{}%
\end{pgfscope}%
\end{pgfscope}%
\begin{pgfscope}%
\definecolor{textcolor}{rgb}{0.000000,0.000000,0.000000}%
\pgfsetstrokecolor{textcolor}%
\pgfsetfillcolor{textcolor}%
\pgftext[x=0.745371in,y=0.469368in,,top]{\color{textcolor}\rmfamily\fontsize{12.000000}{14.400000}\selectfont \(\displaystyle {-10}\)}%
\end{pgfscope}%
\begin{pgfscope}%
\pgfpathrectangle{\pgfqpoint{0.745371in}{0.566590in}}{\pgfqpoint{3.291018in}{1.828724in}}%
\pgfusepath{clip}%
\pgfsetrectcap%
\pgfsetroundjoin%
\pgfsetlinewidth{0.803000pt}%
\definecolor{currentstroke}{rgb}{0.690196,0.690196,0.690196}%
\pgfsetstrokecolor{currentstroke}%
\pgfsetdash{}{0pt}%
\pgfpathmoveto{\pgfqpoint{1.685662in}{0.566590in}}%
\pgfpathlineto{\pgfqpoint{1.685662in}{2.395314in}}%
\pgfusepath{stroke}%
\end{pgfscope}%
\begin{pgfscope}%
\pgfsetbuttcap%
\pgfsetroundjoin%
\definecolor{currentfill}{rgb}{0.000000,0.000000,0.000000}%
\pgfsetfillcolor{currentfill}%
\pgfsetlinewidth{0.803000pt}%
\definecolor{currentstroke}{rgb}{0.000000,0.000000,0.000000}%
\pgfsetstrokecolor{currentstroke}%
\pgfsetdash{}{0pt}%
\pgfsys@defobject{currentmarker}{\pgfqpoint{0.000000in}{-0.048611in}}{\pgfqpoint{0.000000in}{0.000000in}}{%
\pgfpathmoveto{\pgfqpoint{0.000000in}{0.000000in}}%
\pgfpathlineto{\pgfqpoint{0.000000in}{-0.048611in}}%
\pgfusepath{stroke,fill}%
}%
\begin{pgfscope}%
\pgfsys@transformshift{1.685662in}{0.566590in}%
\pgfsys@useobject{currentmarker}{}%
\end{pgfscope}%
\end{pgfscope}%
\begin{pgfscope}%
\definecolor{textcolor}{rgb}{0.000000,0.000000,0.000000}%
\pgfsetstrokecolor{textcolor}%
\pgfsetfillcolor{textcolor}%
\pgftext[x=1.685662in,y=0.469368in,,top]{\color{textcolor}\rmfamily\fontsize{12.000000}{14.400000}\selectfont \(\displaystyle {0}\)}%
\end{pgfscope}%
\begin{pgfscope}%
\pgfpathrectangle{\pgfqpoint{0.745371in}{0.566590in}}{\pgfqpoint{3.291018in}{1.828724in}}%
\pgfusepath{clip}%
\pgfsetrectcap%
\pgfsetroundjoin%
\pgfsetlinewidth{0.803000pt}%
\definecolor{currentstroke}{rgb}{0.690196,0.690196,0.690196}%
\pgfsetstrokecolor{currentstroke}%
\pgfsetdash{}{0pt}%
\pgfpathmoveto{\pgfqpoint{2.625952in}{0.566590in}}%
\pgfpathlineto{\pgfqpoint{2.625952in}{2.395314in}}%
\pgfusepath{stroke}%
\end{pgfscope}%
\begin{pgfscope}%
\pgfsetbuttcap%
\pgfsetroundjoin%
\definecolor{currentfill}{rgb}{0.000000,0.000000,0.000000}%
\pgfsetfillcolor{currentfill}%
\pgfsetlinewidth{0.803000pt}%
\definecolor{currentstroke}{rgb}{0.000000,0.000000,0.000000}%
\pgfsetstrokecolor{currentstroke}%
\pgfsetdash{}{0pt}%
\pgfsys@defobject{currentmarker}{\pgfqpoint{0.000000in}{-0.048611in}}{\pgfqpoint{0.000000in}{0.000000in}}{%
\pgfpathmoveto{\pgfqpoint{0.000000in}{0.000000in}}%
\pgfpathlineto{\pgfqpoint{0.000000in}{-0.048611in}}%
\pgfusepath{stroke,fill}%
}%
\begin{pgfscope}%
\pgfsys@transformshift{2.625952in}{0.566590in}%
\pgfsys@useobject{currentmarker}{}%
\end{pgfscope}%
\end{pgfscope}%
\begin{pgfscope}%
\definecolor{textcolor}{rgb}{0.000000,0.000000,0.000000}%
\pgfsetstrokecolor{textcolor}%
\pgfsetfillcolor{textcolor}%
\pgftext[x=2.625952in,y=0.469368in,,top]{\color{textcolor}\rmfamily\fontsize{12.000000}{14.400000}\selectfont \(\displaystyle {10}\)}%
\end{pgfscope}%
\begin{pgfscope}%
\pgfpathrectangle{\pgfqpoint{0.745371in}{0.566590in}}{\pgfqpoint{3.291018in}{1.828724in}}%
\pgfusepath{clip}%
\pgfsetrectcap%
\pgfsetroundjoin%
\pgfsetlinewidth{0.803000pt}%
\definecolor{currentstroke}{rgb}{0.690196,0.690196,0.690196}%
\pgfsetstrokecolor{currentstroke}%
\pgfsetdash{}{0pt}%
\pgfpathmoveto{\pgfqpoint{3.566243in}{0.566590in}}%
\pgfpathlineto{\pgfqpoint{3.566243in}{2.395314in}}%
\pgfusepath{stroke}%
\end{pgfscope}%
\begin{pgfscope}%
\pgfsetbuttcap%
\pgfsetroundjoin%
\definecolor{currentfill}{rgb}{0.000000,0.000000,0.000000}%
\pgfsetfillcolor{currentfill}%
\pgfsetlinewidth{0.803000pt}%
\definecolor{currentstroke}{rgb}{0.000000,0.000000,0.000000}%
\pgfsetstrokecolor{currentstroke}%
\pgfsetdash{}{0pt}%
\pgfsys@defobject{currentmarker}{\pgfqpoint{0.000000in}{-0.048611in}}{\pgfqpoint{0.000000in}{0.000000in}}{%
\pgfpathmoveto{\pgfqpoint{0.000000in}{0.000000in}}%
\pgfpathlineto{\pgfqpoint{0.000000in}{-0.048611in}}%
\pgfusepath{stroke,fill}%
}%
\begin{pgfscope}%
\pgfsys@transformshift{3.566243in}{0.566590in}%
\pgfsys@useobject{currentmarker}{}%
\end{pgfscope}%
\end{pgfscope}%
\begin{pgfscope}%
\definecolor{textcolor}{rgb}{0.000000,0.000000,0.000000}%
\pgfsetstrokecolor{textcolor}%
\pgfsetfillcolor{textcolor}%
\pgftext[x=3.566243in,y=0.469368in,,top]{\color{textcolor}\rmfamily\fontsize{12.000000}{14.400000}\selectfont \(\displaystyle {20}\)}%
\end{pgfscope}%
\begin{pgfscope}%
\definecolor{textcolor}{rgb}{0.000000,0.000000,0.000000}%
\pgfsetstrokecolor{textcolor}%
\pgfsetfillcolor{textcolor}%
\pgftext[x=2.390880in,y=0.266626in,,top]{\color{textcolor}\rmfamily\fontsize{12.000000}{14.400000}\selectfont SNR [dB]}%
\end{pgfscope}%
\begin{pgfscope}%
\pgfpathrectangle{\pgfqpoint{0.745371in}{0.566590in}}{\pgfqpoint{3.291018in}{1.828724in}}%
\pgfusepath{clip}%
\pgfsetrectcap%
\pgfsetroundjoin%
\pgfsetlinewidth{0.803000pt}%
\definecolor{currentstroke}{rgb}{0.690196,0.690196,0.690196}%
\pgfsetstrokecolor{currentstroke}%
\pgfsetdash{}{0pt}%
\pgfpathmoveto{\pgfqpoint{0.745371in}{1.127599in}}%
\pgfpathlineto{\pgfqpoint{4.036389in}{1.127599in}}%
\pgfusepath{stroke}%
\end{pgfscope}%
\begin{pgfscope}%
\pgfsetbuttcap%
\pgfsetroundjoin%
\definecolor{currentfill}{rgb}{0.000000,0.000000,0.000000}%
\pgfsetfillcolor{currentfill}%
\pgfsetlinewidth{0.803000pt}%
\definecolor{currentstroke}{rgb}{0.000000,0.000000,0.000000}%
\pgfsetstrokecolor{currentstroke}%
\pgfsetdash{}{0pt}%
\pgfsys@defobject{currentmarker}{\pgfqpoint{-0.048611in}{0.000000in}}{\pgfqpoint{-0.000000in}{0.000000in}}{%
\pgfpathmoveto{\pgfqpoint{-0.000000in}{0.000000in}}%
\pgfpathlineto{\pgfqpoint{-0.048611in}{0.000000in}}%
\pgfusepath{stroke,fill}%
}%
\begin{pgfscope}%
\pgfsys@transformshift{0.745371in}{1.127599in}%
\pgfsys@useobject{currentmarker}{}%
\end{pgfscope}%
\end{pgfscope}%
\begin{pgfscope}%
\definecolor{textcolor}{rgb}{0.000000,0.000000,0.000000}%
\pgfsetstrokecolor{textcolor}%
\pgfsetfillcolor{textcolor}%
\pgftext[x=0.327160in, y=1.070206in, left, base]{\color{textcolor}\rmfamily\fontsize{12.000000}{14.400000}\selectfont \(\displaystyle {10^{-2}}\)}%
\end{pgfscope}%
\begin{pgfscope}%
\pgfpathrectangle{\pgfqpoint{0.745371in}{0.566590in}}{\pgfqpoint{3.291018in}{1.828724in}}%
\pgfusepath{clip}%
\pgfsetrectcap%
\pgfsetroundjoin%
\pgfsetlinewidth{0.803000pt}%
\definecolor{currentstroke}{rgb}{0.690196,0.690196,0.690196}%
\pgfsetstrokecolor{currentstroke}%
\pgfsetdash{}{0pt}%
\pgfpathmoveto{\pgfqpoint{0.745371in}{1.911917in}}%
\pgfpathlineto{\pgfqpoint{4.036389in}{1.911917in}}%
\pgfusepath{stroke}%
\end{pgfscope}%
\begin{pgfscope}%
\pgfsetbuttcap%
\pgfsetroundjoin%
\definecolor{currentfill}{rgb}{0.000000,0.000000,0.000000}%
\pgfsetfillcolor{currentfill}%
\pgfsetlinewidth{0.803000pt}%
\definecolor{currentstroke}{rgb}{0.000000,0.000000,0.000000}%
\pgfsetstrokecolor{currentstroke}%
\pgfsetdash{}{0pt}%
\pgfsys@defobject{currentmarker}{\pgfqpoint{-0.048611in}{0.000000in}}{\pgfqpoint{-0.000000in}{0.000000in}}{%
\pgfpathmoveto{\pgfqpoint{-0.000000in}{0.000000in}}%
\pgfpathlineto{\pgfqpoint{-0.048611in}{0.000000in}}%
\pgfusepath{stroke,fill}%
}%
\begin{pgfscope}%
\pgfsys@transformshift{0.745371in}{1.911917in}%
\pgfsys@useobject{currentmarker}{}%
\end{pgfscope}%
\end{pgfscope}%
\begin{pgfscope}%
\definecolor{textcolor}{rgb}{0.000000,0.000000,0.000000}%
\pgfsetstrokecolor{textcolor}%
\pgfsetfillcolor{textcolor}%
\pgftext[x=0.418983in, y=1.854524in, left, base]{\color{textcolor}\rmfamily\fontsize{12.000000}{14.400000}\selectfont \(\displaystyle {10^{0}}\)}%
\end{pgfscope}%
\begin{pgfscope}%
\definecolor{textcolor}{rgb}{0.000000,0.000000,0.000000}%
\pgfsetstrokecolor{textcolor}%
\pgfsetfillcolor{textcolor}%
\pgftext[x=0.271605in,y=1.480952in,,bottom,rotate=90.000000]{\color{textcolor}\rmfamily\fontsize{12.000000}{14.400000}\selectfont \(\displaystyle \hat{\sigma}_{\gamma}(\mathrm{SNR})\)}%
\end{pgfscope}%
\begin{pgfscope}%
\pgfpathrectangle{\pgfqpoint{0.745371in}{0.566590in}}{\pgfqpoint{3.291018in}{1.828724in}}%
\pgfusepath{clip}%
\pgfsetbuttcap%
\pgfsetroundjoin%
\pgfsetlinewidth{1.505625pt}%
\definecolor{currentstroke}{rgb}{0.000000,0.447000,0.741000}%
\pgfsetstrokecolor{currentstroke}%
\pgfsetdash{{5.550000pt}{2.400000pt}}{0.000000pt}%
\pgfpathmoveto{\pgfqpoint{0.745371in}{2.125911in}}%
\pgfpathlineto{\pgfqpoint{0.796793in}{2.140570in}}%
\pgfpathlineto{\pgfqpoint{0.848215in}{2.127018in}}%
\pgfpathlineto{\pgfqpoint{0.899637in}{2.165804in}}%
\pgfpathlineto{\pgfqpoint{0.951059in}{2.121393in}}%
\pgfpathlineto{\pgfqpoint{1.002482in}{2.134250in}}%
\pgfpathlineto{\pgfqpoint{1.053904in}{2.120625in}}%
\pgfpathlineto{\pgfqpoint{1.105326in}{2.178231in}}%
\pgfpathlineto{\pgfqpoint{1.156748in}{2.150216in}}%
\pgfpathlineto{\pgfqpoint{1.208170in}{2.150056in}}%
\pgfpathlineto{\pgfqpoint{1.259592in}{2.232885in}}%
\pgfpathlineto{\pgfqpoint{1.311015in}{2.149751in}}%
\pgfpathlineto{\pgfqpoint{1.362437in}{2.070922in}}%
\pgfpathlineto{\pgfqpoint{1.413859in}{2.140753in}}%
\pgfpathlineto{\pgfqpoint{1.465281in}{2.191780in}}%
\pgfpathlineto{\pgfqpoint{1.516703in}{2.131088in}}%
\pgfpathlineto{\pgfqpoint{1.568125in}{2.070248in}}%
\pgfpathlineto{\pgfqpoint{1.619547in}{2.131251in}}%
\pgfpathlineto{\pgfqpoint{1.670970in}{2.004144in}}%
\pgfpathlineto{\pgfqpoint{1.722392in}{2.032316in}}%
\pgfpathlineto{\pgfqpoint{1.773814in}{1.980741in}}%
\pgfpathlineto{\pgfqpoint{1.825236in}{1.709094in}}%
\pgfpathlineto{\pgfqpoint{1.876658in}{1.772055in}}%
\pgfpathlineto{\pgfqpoint{1.928080in}{1.567145in}}%
\pgfpathlineto{\pgfqpoint{1.979502in}{1.557718in}}%
\pgfpathlineto{\pgfqpoint{2.030925in}{1.503979in}}%
\pgfpathlineto{\pgfqpoint{2.082347in}{1.475010in}}%
\pgfpathlineto{\pgfqpoint{2.133769in}{1.493022in}}%
\pgfpathlineto{\pgfqpoint{2.185191in}{1.435771in}}%
\pgfpathlineto{\pgfqpoint{2.236613in}{1.446778in}}%
\pgfpathlineto{\pgfqpoint{2.288035in}{1.439161in}}%
\pgfpathlineto{\pgfqpoint{2.339458in}{1.425160in}}%
\pgfpathlineto{\pgfqpoint{2.390880in}{1.429817in}}%
\pgfpathlineto{\pgfqpoint{2.442302in}{1.388522in}}%
\pgfpathlineto{\pgfqpoint{2.493724in}{1.390501in}}%
\pgfpathlineto{\pgfqpoint{2.545146in}{1.374013in}}%
\pgfpathlineto{\pgfqpoint{2.596568in}{1.355530in}}%
\pgfpathlineto{\pgfqpoint{2.647990in}{1.344089in}}%
\pgfpathlineto{\pgfqpoint{2.699413in}{1.324521in}}%
\pgfpathlineto{\pgfqpoint{2.750835in}{1.325425in}}%
\pgfpathlineto{\pgfqpoint{2.802257in}{1.320909in}}%
\pgfpathlineto{\pgfqpoint{2.853679in}{1.298808in}}%
\pgfpathlineto{\pgfqpoint{2.905101in}{1.302984in}}%
\pgfpathlineto{\pgfqpoint{2.956523in}{1.279071in}}%
\pgfpathlineto{\pgfqpoint{3.007946in}{1.273519in}}%
\pgfpathlineto{\pgfqpoint{3.059368in}{1.273599in}}%
\pgfpathlineto{\pgfqpoint{3.110790in}{1.258123in}}%
\pgfpathlineto{\pgfqpoint{3.162212in}{1.242631in}}%
\pgfpathlineto{\pgfqpoint{3.213634in}{1.223874in}}%
\pgfpathlineto{\pgfqpoint{3.265056in}{1.230156in}}%
\pgfpathlineto{\pgfqpoint{3.316478in}{1.216415in}}%
\pgfpathlineto{\pgfqpoint{3.367901in}{1.196071in}}%
\pgfpathlineto{\pgfqpoint{3.419323in}{1.183280in}}%
\pgfpathlineto{\pgfqpoint{3.470745in}{1.169327in}}%
\pgfpathlineto{\pgfqpoint{3.522167in}{1.171704in}}%
\pgfpathlineto{\pgfqpoint{3.573589in}{1.154860in}}%
\pgfpathlineto{\pgfqpoint{3.625011in}{1.149304in}}%
\pgfpathlineto{\pgfqpoint{3.676433in}{1.139549in}}%
\pgfpathlineto{\pgfqpoint{3.727856in}{1.117613in}}%
\pgfpathlineto{\pgfqpoint{3.779278in}{1.111905in}}%
\pgfpathlineto{\pgfqpoint{3.830700in}{1.119335in}}%
\pgfpathlineto{\pgfqpoint{3.882122in}{1.085183in}}%
\pgfpathlineto{\pgfqpoint{3.933544in}{1.080439in}}%
\pgfpathlineto{\pgfqpoint{3.984966in}{1.061576in}}%
\pgfpathlineto{\pgfqpoint{4.036389in}{1.066873in}}%
\pgfusepath{stroke}%
\end{pgfscope}%
\begin{pgfscope}%
\pgfpathrectangle{\pgfqpoint{0.745371in}{0.566590in}}{\pgfqpoint{3.291018in}{1.828724in}}%
\pgfusepath{clip}%
\pgfsetbuttcap%
\pgfsetroundjoin%
\definecolor{currentfill}{rgb}{0.000000,0.000000,0.000000}%
\pgfsetfillcolor{currentfill}%
\pgfsetfillopacity{0.000000}%
\pgfsetlinewidth{1.003750pt}%
\definecolor{currentstroke}{rgb}{0.000000,0.447000,0.741000}%
\pgfsetstrokecolor{currentstroke}%
\pgfsetdash{}{0pt}%
\pgfsys@defobject{currentmarker}{\pgfqpoint{-0.041667in}{-0.041667in}}{\pgfqpoint{0.041667in}{0.041667in}}{%
\pgfpathmoveto{\pgfqpoint{0.000000in}{-0.041667in}}%
\pgfpathcurveto{\pgfqpoint{0.011050in}{-0.041667in}}{\pgfqpoint{0.021649in}{-0.037276in}}{\pgfqpoint{0.029463in}{-0.029463in}}%
\pgfpathcurveto{\pgfqpoint{0.037276in}{-0.021649in}}{\pgfqpoint{0.041667in}{-0.011050in}}{\pgfqpoint{0.041667in}{0.000000in}}%
\pgfpathcurveto{\pgfqpoint{0.041667in}{0.011050in}}{\pgfqpoint{0.037276in}{0.021649in}}{\pgfqpoint{0.029463in}{0.029463in}}%
\pgfpathcurveto{\pgfqpoint{0.021649in}{0.037276in}}{\pgfqpoint{0.011050in}{0.041667in}}{\pgfqpoint{0.000000in}{0.041667in}}%
\pgfpathcurveto{\pgfqpoint{-0.011050in}{0.041667in}}{\pgfqpoint{-0.021649in}{0.037276in}}{\pgfqpoint{-0.029463in}{0.029463in}}%
\pgfpathcurveto{\pgfqpoint{-0.037276in}{0.021649in}}{\pgfqpoint{-0.041667in}{0.011050in}}{\pgfqpoint{-0.041667in}{0.000000in}}%
\pgfpathcurveto{\pgfqpoint{-0.041667in}{-0.011050in}}{\pgfqpoint{-0.037276in}{-0.021649in}}{\pgfqpoint{-0.029463in}{-0.029463in}}%
\pgfpathcurveto{\pgfqpoint{-0.021649in}{-0.037276in}}{\pgfqpoint{-0.011050in}{-0.041667in}}{\pgfqpoint{0.000000in}{-0.041667in}}%
\pgfpathclose%
\pgfusepath{stroke,fill}%
}%
\begin{pgfscope}%
\pgfsys@transformshift{0.745371in}{2.125911in}%
\pgfsys@useobject{currentmarker}{}%
\end{pgfscope}%
\begin{pgfscope}%
\pgfsys@transformshift{0.951059in}{2.121393in}%
\pgfsys@useobject{currentmarker}{}%
\end{pgfscope}%
\begin{pgfscope}%
\pgfsys@transformshift{1.156748in}{2.150216in}%
\pgfsys@useobject{currentmarker}{}%
\end{pgfscope}%
\begin{pgfscope}%
\pgfsys@transformshift{1.362437in}{2.070922in}%
\pgfsys@useobject{currentmarker}{}%
\end{pgfscope}%
\begin{pgfscope}%
\pgfsys@transformshift{1.568125in}{2.070248in}%
\pgfsys@useobject{currentmarker}{}%
\end{pgfscope}%
\begin{pgfscope}%
\pgfsys@transformshift{1.773814in}{1.980741in}%
\pgfsys@useobject{currentmarker}{}%
\end{pgfscope}%
\begin{pgfscope}%
\pgfsys@transformshift{1.979502in}{1.557718in}%
\pgfsys@useobject{currentmarker}{}%
\end{pgfscope}%
\begin{pgfscope}%
\pgfsys@transformshift{2.185191in}{1.435771in}%
\pgfsys@useobject{currentmarker}{}%
\end{pgfscope}%
\begin{pgfscope}%
\pgfsys@transformshift{2.390880in}{1.429817in}%
\pgfsys@useobject{currentmarker}{}%
\end{pgfscope}%
\begin{pgfscope}%
\pgfsys@transformshift{2.596568in}{1.355530in}%
\pgfsys@useobject{currentmarker}{}%
\end{pgfscope}%
\begin{pgfscope}%
\pgfsys@transformshift{2.802257in}{1.320909in}%
\pgfsys@useobject{currentmarker}{}%
\end{pgfscope}%
\begin{pgfscope}%
\pgfsys@transformshift{3.007946in}{1.273519in}%
\pgfsys@useobject{currentmarker}{}%
\end{pgfscope}%
\begin{pgfscope}%
\pgfsys@transformshift{3.213634in}{1.223874in}%
\pgfsys@useobject{currentmarker}{}%
\end{pgfscope}%
\begin{pgfscope}%
\pgfsys@transformshift{3.419323in}{1.183280in}%
\pgfsys@useobject{currentmarker}{}%
\end{pgfscope}%
\begin{pgfscope}%
\pgfsys@transformshift{3.625011in}{1.149304in}%
\pgfsys@useobject{currentmarker}{}%
\end{pgfscope}%
\begin{pgfscope}%
\pgfsys@transformshift{3.830700in}{1.119335in}%
\pgfsys@useobject{currentmarker}{}%
\end{pgfscope}%
\begin{pgfscope}%
\pgfsys@transformshift{4.036389in}{1.066873in}%
\pgfsys@useobject{currentmarker}{}%
\end{pgfscope}%
\end{pgfscope}%
\begin{pgfscope}%
\pgfpathrectangle{\pgfqpoint{0.745371in}{0.566590in}}{\pgfqpoint{3.291018in}{1.828724in}}%
\pgfusepath{clip}%
\pgfsetbuttcap%
\pgfsetroundjoin%
\pgfsetlinewidth{1.505625pt}%
\definecolor{currentstroke}{rgb}{0.850000,0.324000,0.098000}%
\pgfsetstrokecolor{currentstroke}%
\pgfsetdash{{5.550000pt}{2.400000pt}}{0.000000pt}%
\pgfpathmoveto{\pgfqpoint{0.745371in}{2.211166in}}%
\pgfpathlineto{\pgfqpoint{0.796793in}{2.120639in}}%
\pgfpathlineto{\pgfqpoint{0.848215in}{2.131986in}}%
\pgfpathlineto{\pgfqpoint{0.899637in}{2.192017in}}%
\pgfpathlineto{\pgfqpoint{0.951059in}{1.962717in}}%
\pgfpathlineto{\pgfqpoint{1.002482in}{2.071547in}}%
\pgfpathlineto{\pgfqpoint{1.053904in}{2.008787in}}%
\pgfpathlineto{\pgfqpoint{1.105326in}{1.911269in}}%
\pgfpathlineto{\pgfqpoint{1.156748in}{1.981969in}}%
\pgfpathlineto{\pgfqpoint{1.208170in}{2.079483in}}%
\pgfpathlineto{\pgfqpoint{1.259592in}{1.692245in}}%
\pgfpathlineto{\pgfqpoint{1.311015in}{2.157432in}}%
\pgfpathlineto{\pgfqpoint{1.362437in}{1.637393in}}%
\pgfpathlineto{\pgfqpoint{1.413859in}{1.630569in}}%
\pgfpathlineto{\pgfqpoint{1.465281in}{1.598460in}}%
\pgfpathlineto{\pgfqpoint{1.516703in}{1.846988in}}%
\pgfpathlineto{\pgfqpoint{1.568125in}{1.559344in}}%
\pgfpathlineto{\pgfqpoint{1.619547in}{1.542507in}}%
\pgfpathlineto{\pgfqpoint{1.670970in}{1.523689in}}%
\pgfpathlineto{\pgfqpoint{1.722392in}{1.476630in}}%
\pgfpathlineto{\pgfqpoint{1.773814in}{1.460510in}}%
\pgfpathlineto{\pgfqpoint{1.825236in}{1.394947in}}%
\pgfpathlineto{\pgfqpoint{1.876658in}{1.407670in}}%
\pgfpathlineto{\pgfqpoint{1.928080in}{1.409306in}}%
\pgfpathlineto{\pgfqpoint{1.979502in}{1.368596in}}%
\pgfpathlineto{\pgfqpoint{2.030925in}{1.346930in}}%
\pgfpathlineto{\pgfqpoint{2.082347in}{1.350371in}}%
\pgfpathlineto{\pgfqpoint{2.133769in}{1.328721in}}%
\pgfpathlineto{\pgfqpoint{2.185191in}{1.340291in}}%
\pgfpathlineto{\pgfqpoint{2.236613in}{1.310192in}}%
\pgfpathlineto{\pgfqpoint{2.288035in}{1.311902in}}%
\pgfpathlineto{\pgfqpoint{2.339458in}{1.304674in}}%
\pgfpathlineto{\pgfqpoint{2.390880in}{1.293140in}}%
\pgfpathlineto{\pgfqpoint{2.442302in}{1.268226in}}%
\pgfpathlineto{\pgfqpoint{2.493724in}{1.268958in}}%
\pgfpathlineto{\pgfqpoint{2.545146in}{1.238597in}}%
\pgfpathlineto{\pgfqpoint{2.596568in}{1.252974in}}%
\pgfpathlineto{\pgfqpoint{2.647990in}{1.221422in}}%
\pgfpathlineto{\pgfqpoint{2.699413in}{1.217049in}}%
\pgfpathlineto{\pgfqpoint{2.750835in}{1.216647in}}%
\pgfpathlineto{\pgfqpoint{2.802257in}{1.195329in}}%
\pgfpathlineto{\pgfqpoint{2.853679in}{1.194991in}}%
\pgfpathlineto{\pgfqpoint{2.905101in}{1.181247in}}%
\pgfpathlineto{\pgfqpoint{2.956523in}{1.167075in}}%
\pgfpathlineto{\pgfqpoint{3.007946in}{1.145515in}}%
\pgfpathlineto{\pgfqpoint{3.059368in}{1.131580in}}%
\pgfpathlineto{\pgfqpoint{3.110790in}{1.124969in}}%
\pgfpathlineto{\pgfqpoint{3.162212in}{1.120540in}}%
\pgfpathlineto{\pgfqpoint{3.213634in}{1.116363in}}%
\pgfpathlineto{\pgfqpoint{3.265056in}{1.092413in}}%
\pgfpathlineto{\pgfqpoint{3.316478in}{1.098094in}}%
\pgfpathlineto{\pgfqpoint{3.367901in}{1.078774in}}%
\pgfpathlineto{\pgfqpoint{3.419323in}{1.064728in}}%
\pgfpathlineto{\pgfqpoint{3.470745in}{1.068857in}}%
\pgfpathlineto{\pgfqpoint{3.522167in}{1.057556in}}%
\pgfpathlineto{\pgfqpoint{3.573589in}{1.031436in}}%
\pgfpathlineto{\pgfqpoint{3.625011in}{1.028674in}}%
\pgfpathlineto{\pgfqpoint{3.676433in}{1.026233in}}%
\pgfpathlineto{\pgfqpoint{3.727856in}{1.008075in}}%
\pgfpathlineto{\pgfqpoint{3.779278in}{0.996000in}}%
\pgfpathlineto{\pgfqpoint{3.830700in}{0.986092in}}%
\pgfpathlineto{\pgfqpoint{3.882122in}{0.981598in}}%
\pgfpathlineto{\pgfqpoint{3.933544in}{0.948681in}}%
\pgfpathlineto{\pgfqpoint{3.984966in}{0.952229in}}%
\pgfpathlineto{\pgfqpoint{4.036389in}{0.951859in}}%
\pgfusepath{stroke}%
\end{pgfscope}%
\begin{pgfscope}%
\pgfpathrectangle{\pgfqpoint{0.745371in}{0.566590in}}{\pgfqpoint{3.291018in}{1.828724in}}%
\pgfusepath{clip}%
\pgfsetbuttcap%
\pgfsetroundjoin%
\definecolor{currentfill}{rgb}{0.850000,0.324000,0.098000}%
\pgfsetfillcolor{currentfill}%
\pgfsetlinewidth{1.003750pt}%
\definecolor{currentstroke}{rgb}{0.850000,0.324000,0.098000}%
\pgfsetstrokecolor{currentstroke}%
\pgfsetdash{}{0pt}%
\pgfsys@defobject{currentmarker}{\pgfqpoint{-0.041667in}{-0.041667in}}{\pgfqpoint{0.041667in}{0.041667in}}{%
\pgfpathmoveto{\pgfqpoint{-0.041667in}{0.000000in}}%
\pgfpathlineto{\pgfqpoint{0.041667in}{0.000000in}}%
\pgfpathmoveto{\pgfqpoint{0.000000in}{-0.041667in}}%
\pgfpathlineto{\pgfqpoint{0.000000in}{0.041667in}}%
\pgfusepath{stroke,fill}%
}%
\begin{pgfscope}%
\pgfsys@transformshift{0.745371in}{2.211166in}%
\pgfsys@useobject{currentmarker}{}%
\end{pgfscope}%
\begin{pgfscope}%
\pgfsys@transformshift{0.899637in}{2.192017in}%
\pgfsys@useobject{currentmarker}{}%
\end{pgfscope}%
\begin{pgfscope}%
\pgfsys@transformshift{1.053904in}{2.008787in}%
\pgfsys@useobject{currentmarker}{}%
\end{pgfscope}%
\begin{pgfscope}%
\pgfsys@transformshift{1.208170in}{2.079483in}%
\pgfsys@useobject{currentmarker}{}%
\end{pgfscope}%
\begin{pgfscope}%
\pgfsys@transformshift{1.362437in}{1.637393in}%
\pgfsys@useobject{currentmarker}{}%
\end{pgfscope}%
\begin{pgfscope}%
\pgfsys@transformshift{1.516703in}{1.846988in}%
\pgfsys@useobject{currentmarker}{}%
\end{pgfscope}%
\begin{pgfscope}%
\pgfsys@transformshift{1.670970in}{1.523689in}%
\pgfsys@useobject{currentmarker}{}%
\end{pgfscope}%
\begin{pgfscope}%
\pgfsys@transformshift{1.825236in}{1.394947in}%
\pgfsys@useobject{currentmarker}{}%
\end{pgfscope}%
\begin{pgfscope}%
\pgfsys@transformshift{1.979502in}{1.368596in}%
\pgfsys@useobject{currentmarker}{}%
\end{pgfscope}%
\begin{pgfscope}%
\pgfsys@transformshift{2.133769in}{1.328721in}%
\pgfsys@useobject{currentmarker}{}%
\end{pgfscope}%
\begin{pgfscope}%
\pgfsys@transformshift{2.288035in}{1.311902in}%
\pgfsys@useobject{currentmarker}{}%
\end{pgfscope}%
\begin{pgfscope}%
\pgfsys@transformshift{2.442302in}{1.268226in}%
\pgfsys@useobject{currentmarker}{}%
\end{pgfscope}%
\begin{pgfscope}%
\pgfsys@transformshift{2.596568in}{1.252974in}%
\pgfsys@useobject{currentmarker}{}%
\end{pgfscope}%
\begin{pgfscope}%
\pgfsys@transformshift{2.750835in}{1.216647in}%
\pgfsys@useobject{currentmarker}{}%
\end{pgfscope}%
\begin{pgfscope}%
\pgfsys@transformshift{2.905101in}{1.181247in}%
\pgfsys@useobject{currentmarker}{}%
\end{pgfscope}%
\begin{pgfscope}%
\pgfsys@transformshift{3.059368in}{1.131580in}%
\pgfsys@useobject{currentmarker}{}%
\end{pgfscope}%
\begin{pgfscope}%
\pgfsys@transformshift{3.213634in}{1.116363in}%
\pgfsys@useobject{currentmarker}{}%
\end{pgfscope}%
\begin{pgfscope}%
\pgfsys@transformshift{3.367901in}{1.078774in}%
\pgfsys@useobject{currentmarker}{}%
\end{pgfscope}%
\begin{pgfscope}%
\pgfsys@transformshift{3.522167in}{1.057556in}%
\pgfsys@useobject{currentmarker}{}%
\end{pgfscope}%
\begin{pgfscope}%
\pgfsys@transformshift{3.676433in}{1.026233in}%
\pgfsys@useobject{currentmarker}{}%
\end{pgfscope}%
\begin{pgfscope}%
\pgfsys@transformshift{3.830700in}{0.986092in}%
\pgfsys@useobject{currentmarker}{}%
\end{pgfscope}%
\begin{pgfscope}%
\pgfsys@transformshift{3.984966in}{0.952229in}%
\pgfsys@useobject{currentmarker}{}%
\end{pgfscope}%
\end{pgfscope}%
\begin{pgfscope}%
\pgfpathrectangle{\pgfqpoint{0.745371in}{0.566590in}}{\pgfqpoint{3.291018in}{1.828724in}}%
\pgfusepath{clip}%
\pgfsetbuttcap%
\pgfsetroundjoin%
\pgfsetlinewidth{1.505625pt}%
\definecolor{currentstroke}{rgb}{0.000000,0.500000,0.000000}%
\pgfsetstrokecolor{currentstroke}%
\pgfsetdash{{5.550000pt}{2.400000pt}}{0.000000pt}%
\pgfpathmoveto{\pgfqpoint{0.745371in}{2.254779in}}%
\pgfpathlineto{\pgfqpoint{0.796793in}{2.281685in}}%
\pgfpathlineto{\pgfqpoint{0.848215in}{2.305793in}}%
\pgfpathlineto{\pgfqpoint{0.899637in}{2.206464in}}%
\pgfpathlineto{\pgfqpoint{0.951059in}{2.268202in}}%
\pgfpathlineto{\pgfqpoint{1.002482in}{2.265269in}}%
\pgfpathlineto{\pgfqpoint{1.053904in}{2.284174in}}%
\pgfpathlineto{\pgfqpoint{1.105326in}{2.256561in}}%
\pgfpathlineto{\pgfqpoint{1.156748in}{2.293164in}}%
\pgfpathlineto{\pgfqpoint{1.208170in}{2.250973in}}%
\pgfpathlineto{\pgfqpoint{1.259592in}{2.197624in}}%
\pgfpathlineto{\pgfqpoint{1.311015in}{2.192900in}}%
\pgfpathlineto{\pgfqpoint{1.362437in}{2.233627in}}%
\pgfpathlineto{\pgfqpoint{1.413859in}{2.211092in}}%
\pgfpathlineto{\pgfqpoint{1.465281in}{2.221721in}}%
\pgfpathlineto{\pgfqpoint{1.516703in}{2.217510in}}%
\pgfpathlineto{\pgfqpoint{1.568125in}{2.232495in}}%
\pgfpathlineto{\pgfqpoint{1.619547in}{2.173582in}}%
\pgfpathlineto{\pgfqpoint{1.670970in}{1.982169in}}%
\pgfpathlineto{\pgfqpoint{1.722392in}{2.040469in}}%
\pgfpathlineto{\pgfqpoint{1.773814in}{1.926964in}}%
\pgfpathlineto{\pgfqpoint{1.825236in}{1.450160in}}%
\pgfpathlineto{\pgfqpoint{1.876658in}{1.467439in}}%
\pgfpathlineto{\pgfqpoint{1.928080in}{1.474985in}}%
\pgfpathlineto{\pgfqpoint{1.979502in}{1.397700in}}%
\pgfpathlineto{\pgfqpoint{2.030925in}{1.378410in}}%
\pgfpathlineto{\pgfqpoint{2.082347in}{1.389514in}}%
\pgfpathlineto{\pgfqpoint{2.133769in}{1.381939in}}%
\pgfpathlineto{\pgfqpoint{2.185191in}{1.355669in}}%
\pgfpathlineto{\pgfqpoint{2.236613in}{1.343571in}}%
\pgfpathlineto{\pgfqpoint{2.288035in}{1.340002in}}%
\pgfpathlineto{\pgfqpoint{2.339458in}{1.324889in}}%
\pgfpathlineto{\pgfqpoint{2.390880in}{1.307739in}}%
\pgfpathlineto{\pgfqpoint{2.442302in}{1.304507in}}%
\pgfpathlineto{\pgfqpoint{2.493724in}{1.293963in}}%
\pgfpathlineto{\pgfqpoint{2.545146in}{1.288286in}}%
\pgfpathlineto{\pgfqpoint{2.596568in}{1.267216in}}%
\pgfpathlineto{\pgfqpoint{2.647990in}{1.257691in}}%
\pgfpathlineto{\pgfqpoint{2.699413in}{1.249697in}}%
\pgfpathlineto{\pgfqpoint{2.750835in}{1.234129in}}%
\pgfpathlineto{\pgfqpoint{2.802257in}{1.213104in}}%
\pgfpathlineto{\pgfqpoint{2.853679in}{1.224574in}}%
\pgfpathlineto{\pgfqpoint{2.905101in}{1.201376in}}%
\pgfpathlineto{\pgfqpoint{2.956523in}{1.205237in}}%
\pgfpathlineto{\pgfqpoint{3.007946in}{1.181712in}}%
\pgfpathlineto{\pgfqpoint{3.059368in}{1.169498in}}%
\pgfpathlineto{\pgfqpoint{3.110790in}{1.170550in}}%
\pgfpathlineto{\pgfqpoint{3.162212in}{1.145950in}}%
\pgfpathlineto{\pgfqpoint{3.213634in}{1.141877in}}%
\pgfpathlineto{\pgfqpoint{3.265056in}{1.120705in}}%
\pgfpathlineto{\pgfqpoint{3.316478in}{1.116778in}}%
\pgfpathlineto{\pgfqpoint{3.367901in}{1.116859in}}%
\pgfpathlineto{\pgfqpoint{3.419323in}{1.086340in}}%
\pgfpathlineto{\pgfqpoint{3.470745in}{1.090686in}}%
\pgfpathlineto{\pgfqpoint{3.522167in}{1.072537in}}%
\pgfpathlineto{\pgfqpoint{3.573589in}{1.038427in}}%
\pgfpathlineto{\pgfqpoint{3.625011in}{1.064732in}}%
\pgfpathlineto{\pgfqpoint{3.676433in}{1.055506in}}%
\pgfpathlineto{\pgfqpoint{3.727856in}{1.042578in}}%
\pgfpathlineto{\pgfqpoint{3.779278in}{1.029351in}}%
\pgfpathlineto{\pgfqpoint{3.830700in}{1.019085in}}%
\pgfpathlineto{\pgfqpoint{3.882122in}{0.997429in}}%
\pgfpathlineto{\pgfqpoint{3.933544in}{0.991711in}}%
\pgfpathlineto{\pgfqpoint{3.984966in}{1.001454in}}%
\pgfpathlineto{\pgfqpoint{4.036389in}{0.964724in}}%
\pgfusepath{stroke}%
\end{pgfscope}%
\begin{pgfscope}%
\pgfpathrectangle{\pgfqpoint{0.745371in}{0.566590in}}{\pgfqpoint{3.291018in}{1.828724in}}%
\pgfusepath{clip}%
\pgfsetbuttcap%
\pgfsetmiterjoin%
\definecolor{currentfill}{rgb}{0.000000,0.000000,0.000000}%
\pgfsetfillcolor{currentfill}%
\pgfsetfillopacity{0.000000}%
\pgfsetlinewidth{1.003750pt}%
\definecolor{currentstroke}{rgb}{0.000000,0.500000,0.000000}%
\pgfsetstrokecolor{currentstroke}%
\pgfsetdash{}{0pt}%
\pgfsys@defobject{currentmarker}{\pgfqpoint{-0.041667in}{-0.041667in}}{\pgfqpoint{0.041667in}{0.041667in}}{%
\pgfpathmoveto{\pgfqpoint{-0.041667in}{-0.041667in}}%
\pgfpathlineto{\pgfqpoint{0.041667in}{-0.041667in}}%
\pgfpathlineto{\pgfqpoint{0.041667in}{0.041667in}}%
\pgfpathlineto{\pgfqpoint{-0.041667in}{0.041667in}}%
\pgfpathclose%
\pgfusepath{stroke,fill}%
}%
\begin{pgfscope}%
\pgfsys@transformshift{0.745371in}{2.254779in}%
\pgfsys@useobject{currentmarker}{}%
\end{pgfscope}%
\begin{pgfscope}%
\pgfsys@transformshift{1.002482in}{2.265269in}%
\pgfsys@useobject{currentmarker}{}%
\end{pgfscope}%
\begin{pgfscope}%
\pgfsys@transformshift{1.259592in}{2.197624in}%
\pgfsys@useobject{currentmarker}{}%
\end{pgfscope}%
\begin{pgfscope}%
\pgfsys@transformshift{1.516703in}{2.217510in}%
\pgfsys@useobject{currentmarker}{}%
\end{pgfscope}%
\begin{pgfscope}%
\pgfsys@transformshift{1.773814in}{1.926964in}%
\pgfsys@useobject{currentmarker}{}%
\end{pgfscope}%
\begin{pgfscope}%
\pgfsys@transformshift{2.030925in}{1.378410in}%
\pgfsys@useobject{currentmarker}{}%
\end{pgfscope}%
\begin{pgfscope}%
\pgfsys@transformshift{2.288035in}{1.340002in}%
\pgfsys@useobject{currentmarker}{}%
\end{pgfscope}%
\begin{pgfscope}%
\pgfsys@transformshift{2.545146in}{1.288286in}%
\pgfsys@useobject{currentmarker}{}%
\end{pgfscope}%
\begin{pgfscope}%
\pgfsys@transformshift{2.802257in}{1.213104in}%
\pgfsys@useobject{currentmarker}{}%
\end{pgfscope}%
\begin{pgfscope}%
\pgfsys@transformshift{3.059368in}{1.169498in}%
\pgfsys@useobject{currentmarker}{}%
\end{pgfscope}%
\begin{pgfscope}%
\pgfsys@transformshift{3.316478in}{1.116778in}%
\pgfsys@useobject{currentmarker}{}%
\end{pgfscope}%
\begin{pgfscope}%
\pgfsys@transformshift{3.573589in}{1.038427in}%
\pgfsys@useobject{currentmarker}{}%
\end{pgfscope}%
\begin{pgfscope}%
\pgfsys@transformshift{3.830700in}{1.019085in}%
\pgfsys@useobject{currentmarker}{}%
\end{pgfscope}%
\end{pgfscope}%
\begin{pgfscope}%
\pgfpathrectangle{\pgfqpoint{0.745371in}{0.566590in}}{\pgfqpoint{3.291018in}{1.828724in}}%
\pgfusepath{clip}%
\pgfsetbuttcap%
\pgfsetroundjoin%
\pgfsetlinewidth{1.505625pt}%
\definecolor{currentstroke}{rgb}{0.494000,0.184000,0.556000}%
\pgfsetstrokecolor{currentstroke}%
\pgfsetdash{{5.550000pt}{2.400000pt}}{0.000000pt}%
\pgfpathmoveto{\pgfqpoint{0.745371in}{2.248518in}}%
\pgfpathlineto{\pgfqpoint{0.796793in}{2.306891in}}%
\pgfpathlineto{\pgfqpoint{0.848215in}{2.306987in}}%
\pgfpathlineto{\pgfqpoint{0.899637in}{2.109802in}}%
\pgfpathlineto{\pgfqpoint{0.951059in}{2.291805in}}%
\pgfpathlineto{\pgfqpoint{1.002482in}{2.189216in}}%
\pgfpathlineto{\pgfqpoint{1.053904in}{2.253858in}}%
\pgfpathlineto{\pgfqpoint{1.105326in}{2.263187in}}%
\pgfpathlineto{\pgfqpoint{1.156748in}{2.252221in}}%
\pgfpathlineto{\pgfqpoint{1.208170in}{2.219770in}}%
\pgfpathlineto{\pgfqpoint{1.259592in}{2.213077in}}%
\pgfpathlineto{\pgfqpoint{1.311015in}{2.299691in}}%
\pgfpathlineto{\pgfqpoint{1.362437in}{2.047705in}}%
\pgfpathlineto{\pgfqpoint{1.413859in}{2.312190in}}%
\pgfpathlineto{\pgfqpoint{1.465281in}{2.146564in}}%
\pgfpathlineto{\pgfqpoint{1.516703in}{2.040046in}}%
\pgfpathlineto{\pgfqpoint{1.568125in}{2.006624in}}%
\pgfpathlineto{\pgfqpoint{1.619547in}{2.085087in}}%
\pgfpathlineto{\pgfqpoint{1.670970in}{1.642308in}}%
\pgfpathlineto{\pgfqpoint{1.722392in}{1.410021in}}%
\pgfpathlineto{\pgfqpoint{1.773814in}{1.787979in}}%
\pgfpathlineto{\pgfqpoint{1.825236in}{1.382838in}}%
\pgfpathlineto{\pgfqpoint{1.876658in}{1.385682in}}%
\pgfpathlineto{\pgfqpoint{1.928080in}{1.374791in}}%
\pgfpathlineto{\pgfqpoint{1.979502in}{1.365927in}}%
\pgfpathlineto{\pgfqpoint{2.030925in}{1.358869in}}%
\pgfpathlineto{\pgfqpoint{2.082347in}{1.336535in}}%
\pgfpathlineto{\pgfqpoint{2.133769in}{1.310989in}}%
\pgfpathlineto{\pgfqpoint{2.185191in}{1.316431in}}%
\pgfpathlineto{\pgfqpoint{2.236613in}{1.309054in}}%
\pgfpathlineto{\pgfqpoint{2.288035in}{1.302086in}}%
\pgfpathlineto{\pgfqpoint{2.339458in}{1.282053in}}%
\pgfpathlineto{\pgfqpoint{2.390880in}{1.271142in}}%
\pgfpathlineto{\pgfqpoint{2.442302in}{1.241723in}}%
\pgfpathlineto{\pgfqpoint{2.493724in}{1.251306in}}%
\pgfpathlineto{\pgfqpoint{2.545146in}{1.228595in}}%
\pgfpathlineto{\pgfqpoint{2.596568in}{1.229427in}}%
\pgfpathlineto{\pgfqpoint{2.647990in}{1.214583in}}%
\pgfpathlineto{\pgfqpoint{2.699413in}{1.178304in}}%
\pgfpathlineto{\pgfqpoint{2.750835in}{1.183674in}}%
\pgfpathlineto{\pgfqpoint{2.802257in}{1.167776in}}%
\pgfpathlineto{\pgfqpoint{2.853679in}{1.167731in}}%
\pgfpathlineto{\pgfqpoint{2.905101in}{1.174557in}}%
\pgfpathlineto{\pgfqpoint{2.956523in}{1.153311in}}%
\pgfpathlineto{\pgfqpoint{3.007946in}{1.141093in}}%
\pgfpathlineto{\pgfqpoint{3.059368in}{1.124088in}}%
\pgfpathlineto{\pgfqpoint{3.110790in}{1.117681in}}%
\pgfpathlineto{\pgfqpoint{3.162212in}{1.106356in}}%
\pgfpathlineto{\pgfqpoint{3.213634in}{1.095323in}}%
\pgfpathlineto{\pgfqpoint{3.265056in}{1.091052in}}%
\pgfpathlineto{\pgfqpoint{3.316478in}{1.072720in}}%
\pgfpathlineto{\pgfqpoint{3.367901in}{1.053317in}}%
\pgfpathlineto{\pgfqpoint{3.419323in}{1.050111in}}%
\pgfpathlineto{\pgfqpoint{3.470745in}{1.035768in}}%
\pgfpathlineto{\pgfqpoint{3.522167in}{1.027303in}}%
\pgfpathlineto{\pgfqpoint{3.573589in}{1.037668in}}%
\pgfpathlineto{\pgfqpoint{3.625011in}{0.993819in}}%
\pgfpathlineto{\pgfqpoint{3.676433in}{0.993105in}}%
\pgfpathlineto{\pgfqpoint{3.727856in}{0.994014in}}%
\pgfpathlineto{\pgfqpoint{3.779278in}{0.978207in}}%
\pgfpathlineto{\pgfqpoint{3.830700in}{0.966503in}}%
\pgfpathlineto{\pgfqpoint{3.882122in}{0.949835in}}%
\pgfpathlineto{\pgfqpoint{3.933544in}{0.925000in}}%
\pgfpathlineto{\pgfqpoint{3.984966in}{0.938891in}}%
\pgfpathlineto{\pgfqpoint{4.036389in}{0.907402in}}%
\pgfusepath{stroke}%
\end{pgfscope}%
\begin{pgfscope}%
\pgfpathrectangle{\pgfqpoint{0.745371in}{0.566590in}}{\pgfqpoint{3.291018in}{1.828724in}}%
\pgfusepath{clip}%
\pgfsetbuttcap%
\pgfsetroundjoin%
\definecolor{currentfill}{rgb}{0.494000,0.184000,0.556000}%
\pgfsetfillcolor{currentfill}%
\pgfsetlinewidth{1.003750pt}%
\definecolor{currentstroke}{rgb}{0.494000,0.184000,0.556000}%
\pgfsetstrokecolor{currentstroke}%
\pgfsetdash{}{0pt}%
\pgfsys@defobject{currentmarker}{\pgfqpoint{-0.041667in}{-0.041667in}}{\pgfqpoint{0.041667in}{0.041667in}}{%
\pgfpathmoveto{\pgfqpoint{-0.041667in}{-0.041667in}}%
\pgfpathlineto{\pgfqpoint{0.041667in}{0.041667in}}%
\pgfpathmoveto{\pgfqpoint{-0.041667in}{0.041667in}}%
\pgfpathlineto{\pgfqpoint{0.041667in}{-0.041667in}}%
\pgfusepath{stroke,fill}%
}%
\begin{pgfscope}%
\pgfsys@transformshift{0.745371in}{2.248518in}%
\pgfsys@useobject{currentmarker}{}%
\end{pgfscope}%
\begin{pgfscope}%
\pgfsys@transformshift{0.951059in}{2.291805in}%
\pgfsys@useobject{currentmarker}{}%
\end{pgfscope}%
\begin{pgfscope}%
\pgfsys@transformshift{1.156748in}{2.252221in}%
\pgfsys@useobject{currentmarker}{}%
\end{pgfscope}%
\begin{pgfscope}%
\pgfsys@transformshift{1.362437in}{2.047705in}%
\pgfsys@useobject{currentmarker}{}%
\end{pgfscope}%
\begin{pgfscope}%
\pgfsys@transformshift{1.568125in}{2.006624in}%
\pgfsys@useobject{currentmarker}{}%
\end{pgfscope}%
\begin{pgfscope}%
\pgfsys@transformshift{1.773814in}{1.787979in}%
\pgfsys@useobject{currentmarker}{}%
\end{pgfscope}%
\begin{pgfscope}%
\pgfsys@transformshift{1.979502in}{1.365927in}%
\pgfsys@useobject{currentmarker}{}%
\end{pgfscope}%
\begin{pgfscope}%
\pgfsys@transformshift{2.185191in}{1.316431in}%
\pgfsys@useobject{currentmarker}{}%
\end{pgfscope}%
\begin{pgfscope}%
\pgfsys@transformshift{2.390880in}{1.271142in}%
\pgfsys@useobject{currentmarker}{}%
\end{pgfscope}%
\begin{pgfscope}%
\pgfsys@transformshift{2.596568in}{1.229427in}%
\pgfsys@useobject{currentmarker}{}%
\end{pgfscope}%
\begin{pgfscope}%
\pgfsys@transformshift{2.802257in}{1.167776in}%
\pgfsys@useobject{currentmarker}{}%
\end{pgfscope}%
\begin{pgfscope}%
\pgfsys@transformshift{3.007946in}{1.141093in}%
\pgfsys@useobject{currentmarker}{}%
\end{pgfscope}%
\begin{pgfscope}%
\pgfsys@transformshift{3.213634in}{1.095323in}%
\pgfsys@useobject{currentmarker}{}%
\end{pgfscope}%
\begin{pgfscope}%
\pgfsys@transformshift{3.419323in}{1.050111in}%
\pgfsys@useobject{currentmarker}{}%
\end{pgfscope}%
\begin{pgfscope}%
\pgfsys@transformshift{3.625011in}{0.993819in}%
\pgfsys@useobject{currentmarker}{}%
\end{pgfscope}%
\begin{pgfscope}%
\pgfsys@transformshift{3.830700in}{0.966503in}%
\pgfsys@useobject{currentmarker}{}%
\end{pgfscope}%
\begin{pgfscope}%
\pgfsys@transformshift{4.036389in}{0.907402in}%
\pgfsys@useobject{currentmarker}{}%
\end{pgfscope}%
\end{pgfscope}%
\begin{pgfscope}%
\pgfpathrectangle{\pgfqpoint{0.745371in}{0.566590in}}{\pgfqpoint{3.291018in}{1.828724in}}%
\pgfusepath{clip}%
\pgfsetrectcap%
\pgfsetroundjoin%
\pgfsetlinewidth{1.505625pt}%
\definecolor{currentstroke}{rgb}{0.000000,0.447000,0.741000}%
\pgfsetstrokecolor{currentstroke}%
\pgfsetdash{}{0pt}%
\pgfpathmoveto{\pgfqpoint{0.745371in}{1.757196in}}%
\pgfpathlineto{\pgfqpoint{0.796793in}{1.759247in}}%
\pgfpathlineto{\pgfqpoint{0.848215in}{1.744599in}}%
\pgfpathlineto{\pgfqpoint{0.899637in}{1.729315in}}%
\pgfpathlineto{\pgfqpoint{0.951059in}{1.892476in}}%
\pgfpathlineto{\pgfqpoint{1.002482in}{1.736129in}}%
\pgfpathlineto{\pgfqpoint{1.053904in}{1.836039in}}%
\pgfpathlineto{\pgfqpoint{1.105326in}{1.691052in}}%
\pgfpathlineto{\pgfqpoint{1.156748in}{1.710833in}}%
\pgfpathlineto{\pgfqpoint{1.208170in}{1.782022in}}%
\pgfpathlineto{\pgfqpoint{1.259592in}{1.662668in}}%
\pgfpathlineto{\pgfqpoint{1.311015in}{1.653829in}}%
\pgfpathlineto{\pgfqpoint{1.362437in}{1.675969in}}%
\pgfpathlineto{\pgfqpoint{1.413859in}{1.578206in}}%
\pgfpathlineto{\pgfqpoint{1.465281in}{1.640704in}}%
\pgfpathlineto{\pgfqpoint{1.516703in}{1.601009in}}%
\pgfpathlineto{\pgfqpoint{1.568125in}{1.519770in}}%
\pgfpathlineto{\pgfqpoint{1.619547in}{1.468623in}}%
\pgfpathlineto{\pgfqpoint{1.670970in}{1.452380in}}%
\pgfpathlineto{\pgfqpoint{1.722392in}{1.355281in}}%
\pgfpathlineto{\pgfqpoint{1.773814in}{1.396259in}}%
\pgfpathlineto{\pgfqpoint{1.825236in}{1.343516in}}%
\pgfpathlineto{\pgfqpoint{1.876658in}{1.323224in}}%
\pgfpathlineto{\pgfqpoint{1.928080in}{1.311639in}}%
\pgfpathlineto{\pgfqpoint{1.979502in}{1.298904in}}%
\pgfpathlineto{\pgfqpoint{2.030925in}{1.263103in}}%
\pgfpathlineto{\pgfqpoint{2.082347in}{1.241569in}}%
\pgfpathlineto{\pgfqpoint{2.133769in}{1.241013in}}%
\pgfpathlineto{\pgfqpoint{2.185191in}{1.198538in}}%
\pgfpathlineto{\pgfqpoint{2.236613in}{1.218175in}}%
\pgfpathlineto{\pgfqpoint{2.288035in}{1.191181in}}%
\pgfpathlineto{\pgfqpoint{2.339458in}{1.164849in}}%
\pgfpathlineto{\pgfqpoint{2.390880in}{1.170416in}}%
\pgfpathlineto{\pgfqpoint{2.442302in}{1.124983in}}%
\pgfpathlineto{\pgfqpoint{2.493724in}{1.129331in}}%
\pgfpathlineto{\pgfqpoint{2.545146in}{1.108337in}}%
\pgfpathlineto{\pgfqpoint{2.596568in}{1.097467in}}%
\pgfpathlineto{\pgfqpoint{2.647990in}{1.077945in}}%
\pgfpathlineto{\pgfqpoint{2.699413in}{1.065646in}}%
\pgfpathlineto{\pgfqpoint{2.750835in}{1.059953in}}%
\pgfpathlineto{\pgfqpoint{2.802257in}{1.052829in}}%
\pgfpathlineto{\pgfqpoint{2.853679in}{1.036259in}}%
\pgfpathlineto{\pgfqpoint{2.905101in}{1.035166in}}%
\pgfpathlineto{\pgfqpoint{2.956523in}{1.014642in}}%
\pgfpathlineto{\pgfqpoint{3.007946in}{1.005913in}}%
\pgfpathlineto{\pgfqpoint{3.059368in}{1.004894in}}%
\pgfpathlineto{\pgfqpoint{3.110790in}{0.989851in}}%
\pgfpathlineto{\pgfqpoint{3.162212in}{0.979353in}}%
\pgfpathlineto{\pgfqpoint{3.213634in}{0.952620in}}%
\pgfpathlineto{\pgfqpoint{3.265056in}{0.966967in}}%
\pgfpathlineto{\pgfqpoint{3.316478in}{0.947895in}}%
\pgfpathlineto{\pgfqpoint{3.367901in}{0.920598in}}%
\pgfpathlineto{\pgfqpoint{3.419323in}{0.918811in}}%
\pgfpathlineto{\pgfqpoint{3.470745in}{0.907632in}}%
\pgfpathlineto{\pgfqpoint{3.522167in}{0.901869in}}%
\pgfpathlineto{\pgfqpoint{3.573589in}{0.887537in}}%
\pgfpathlineto{\pgfqpoint{3.625011in}{0.880432in}}%
\pgfpathlineto{\pgfqpoint{3.676433in}{0.873208in}}%
\pgfpathlineto{\pgfqpoint{3.727856in}{0.845688in}}%
\pgfpathlineto{\pgfqpoint{3.779278in}{0.841283in}}%
\pgfpathlineto{\pgfqpoint{3.830700in}{0.850173in}}%
\pgfpathlineto{\pgfqpoint{3.882122in}{0.811286in}}%
\pgfpathlineto{\pgfqpoint{3.933544in}{0.806663in}}%
\pgfpathlineto{\pgfqpoint{3.984966in}{0.793081in}}%
\pgfpathlineto{\pgfqpoint{4.036389in}{0.800390in}}%
\pgfusepath{stroke}%
\end{pgfscope}%
\begin{pgfscope}%
\pgfpathrectangle{\pgfqpoint{0.745371in}{0.566590in}}{\pgfqpoint{3.291018in}{1.828724in}}%
\pgfusepath{clip}%
\pgfsetbuttcap%
\pgfsetroundjoin%
\definecolor{currentfill}{rgb}{0.000000,0.000000,0.000000}%
\pgfsetfillcolor{currentfill}%
\pgfsetfillopacity{0.000000}%
\pgfsetlinewidth{1.003750pt}%
\definecolor{currentstroke}{rgb}{0.000000,0.447000,0.741000}%
\pgfsetstrokecolor{currentstroke}%
\pgfsetdash{}{0pt}%
\pgfsys@defobject{currentmarker}{\pgfqpoint{-0.041667in}{-0.041667in}}{\pgfqpoint{0.041667in}{0.041667in}}{%
\pgfpathmoveto{\pgfqpoint{0.000000in}{-0.041667in}}%
\pgfpathcurveto{\pgfqpoint{0.011050in}{-0.041667in}}{\pgfqpoint{0.021649in}{-0.037276in}}{\pgfqpoint{0.029463in}{-0.029463in}}%
\pgfpathcurveto{\pgfqpoint{0.037276in}{-0.021649in}}{\pgfqpoint{0.041667in}{-0.011050in}}{\pgfqpoint{0.041667in}{0.000000in}}%
\pgfpathcurveto{\pgfqpoint{0.041667in}{0.011050in}}{\pgfqpoint{0.037276in}{0.021649in}}{\pgfqpoint{0.029463in}{0.029463in}}%
\pgfpathcurveto{\pgfqpoint{0.021649in}{0.037276in}}{\pgfqpoint{0.011050in}{0.041667in}}{\pgfqpoint{0.000000in}{0.041667in}}%
\pgfpathcurveto{\pgfqpoint{-0.011050in}{0.041667in}}{\pgfqpoint{-0.021649in}{0.037276in}}{\pgfqpoint{-0.029463in}{0.029463in}}%
\pgfpathcurveto{\pgfqpoint{-0.037276in}{0.021649in}}{\pgfqpoint{-0.041667in}{0.011050in}}{\pgfqpoint{-0.041667in}{0.000000in}}%
\pgfpathcurveto{\pgfqpoint{-0.041667in}{-0.011050in}}{\pgfqpoint{-0.037276in}{-0.021649in}}{\pgfqpoint{-0.029463in}{-0.029463in}}%
\pgfpathcurveto{\pgfqpoint{-0.021649in}{-0.037276in}}{\pgfqpoint{-0.011050in}{-0.041667in}}{\pgfqpoint{0.000000in}{-0.041667in}}%
\pgfpathclose%
\pgfusepath{stroke,fill}%
}%
\begin{pgfscope}%
\pgfsys@transformshift{0.745371in}{1.757196in}%
\pgfsys@useobject{currentmarker}{}%
\end{pgfscope}%
\begin{pgfscope}%
\pgfsys@transformshift{0.951059in}{1.892476in}%
\pgfsys@useobject{currentmarker}{}%
\end{pgfscope}%
\begin{pgfscope}%
\pgfsys@transformshift{1.156748in}{1.710833in}%
\pgfsys@useobject{currentmarker}{}%
\end{pgfscope}%
\begin{pgfscope}%
\pgfsys@transformshift{1.362437in}{1.675969in}%
\pgfsys@useobject{currentmarker}{}%
\end{pgfscope}%
\begin{pgfscope}%
\pgfsys@transformshift{1.568125in}{1.519770in}%
\pgfsys@useobject{currentmarker}{}%
\end{pgfscope}%
\begin{pgfscope}%
\pgfsys@transformshift{1.773814in}{1.396259in}%
\pgfsys@useobject{currentmarker}{}%
\end{pgfscope}%
\begin{pgfscope}%
\pgfsys@transformshift{1.979502in}{1.298904in}%
\pgfsys@useobject{currentmarker}{}%
\end{pgfscope}%
\begin{pgfscope}%
\pgfsys@transformshift{2.185191in}{1.198538in}%
\pgfsys@useobject{currentmarker}{}%
\end{pgfscope}%
\begin{pgfscope}%
\pgfsys@transformshift{2.390880in}{1.170416in}%
\pgfsys@useobject{currentmarker}{}%
\end{pgfscope}%
\begin{pgfscope}%
\pgfsys@transformshift{2.596568in}{1.097467in}%
\pgfsys@useobject{currentmarker}{}%
\end{pgfscope}%
\begin{pgfscope}%
\pgfsys@transformshift{2.802257in}{1.052829in}%
\pgfsys@useobject{currentmarker}{}%
\end{pgfscope}%
\begin{pgfscope}%
\pgfsys@transformshift{3.007946in}{1.005913in}%
\pgfsys@useobject{currentmarker}{}%
\end{pgfscope}%
\begin{pgfscope}%
\pgfsys@transformshift{3.213634in}{0.952620in}%
\pgfsys@useobject{currentmarker}{}%
\end{pgfscope}%
\begin{pgfscope}%
\pgfsys@transformshift{3.419323in}{0.918811in}%
\pgfsys@useobject{currentmarker}{}%
\end{pgfscope}%
\begin{pgfscope}%
\pgfsys@transformshift{3.625011in}{0.880432in}%
\pgfsys@useobject{currentmarker}{}%
\end{pgfscope}%
\begin{pgfscope}%
\pgfsys@transformshift{3.830700in}{0.850173in}%
\pgfsys@useobject{currentmarker}{}%
\end{pgfscope}%
\begin{pgfscope}%
\pgfsys@transformshift{4.036389in}{0.800390in}%
\pgfsys@useobject{currentmarker}{}%
\end{pgfscope}%
\end{pgfscope}%
\begin{pgfscope}%
\pgfpathrectangle{\pgfqpoint{0.745371in}{0.566590in}}{\pgfqpoint{3.291018in}{1.828724in}}%
\pgfusepath{clip}%
\pgfsetrectcap%
\pgfsetroundjoin%
\pgfsetlinewidth{1.505625pt}%
\definecolor{currentstroke}{rgb}{0.850000,0.324000,0.098000}%
\pgfsetstrokecolor{currentstroke}%
\pgfsetdash{}{0pt}%
\pgfpathmoveto{\pgfqpoint{0.745371in}{1.568267in}}%
\pgfpathlineto{\pgfqpoint{0.796793in}{1.543038in}}%
\pgfpathlineto{\pgfqpoint{0.848215in}{1.529377in}}%
\pgfpathlineto{\pgfqpoint{0.899637in}{1.500188in}}%
\pgfpathlineto{\pgfqpoint{0.951059in}{1.490148in}}%
\pgfpathlineto{\pgfqpoint{1.002482in}{1.503696in}}%
\pgfpathlineto{\pgfqpoint{1.053904in}{1.480754in}}%
\pgfpathlineto{\pgfqpoint{1.105326in}{1.452934in}}%
\pgfpathlineto{\pgfqpoint{1.156748in}{1.405629in}}%
\pgfpathlineto{\pgfqpoint{1.208170in}{1.432009in}}%
\pgfpathlineto{\pgfqpoint{1.259592in}{1.353816in}}%
\pgfpathlineto{\pgfqpoint{1.311015in}{1.353548in}}%
\pgfpathlineto{\pgfqpoint{1.362437in}{1.311538in}}%
\pgfpathlineto{\pgfqpoint{1.413859in}{1.316708in}}%
\pgfpathlineto{\pgfqpoint{1.465281in}{1.265213in}}%
\pgfpathlineto{\pgfqpoint{1.516703in}{1.225278in}}%
\pgfpathlineto{\pgfqpoint{1.568125in}{1.209774in}}%
\pgfpathlineto{\pgfqpoint{1.619547in}{1.235544in}}%
\pgfpathlineto{\pgfqpoint{1.670970in}{1.186483in}}%
\pgfpathlineto{\pgfqpoint{1.722392in}{1.171095in}}%
\pgfpathlineto{\pgfqpoint{1.773814in}{1.145857in}}%
\pgfpathlineto{\pgfqpoint{1.825236in}{1.138912in}}%
\pgfpathlineto{\pgfqpoint{1.876658in}{1.123575in}}%
\pgfpathlineto{\pgfqpoint{1.928080in}{1.132101in}}%
\pgfpathlineto{\pgfqpoint{1.979502in}{1.107792in}}%
\pgfpathlineto{\pgfqpoint{2.030925in}{1.085237in}}%
\pgfpathlineto{\pgfqpoint{2.082347in}{1.091322in}}%
\pgfpathlineto{\pgfqpoint{2.133769in}{1.074280in}}%
\pgfpathlineto{\pgfqpoint{2.185191in}{1.073152in}}%
\pgfpathlineto{\pgfqpoint{2.236613in}{1.055838in}}%
\pgfpathlineto{\pgfqpoint{2.288035in}{1.047940in}}%
\pgfpathlineto{\pgfqpoint{2.339458in}{1.034566in}}%
\pgfpathlineto{\pgfqpoint{2.390880in}{1.030932in}}%
\pgfpathlineto{\pgfqpoint{2.442302in}{1.003608in}}%
\pgfpathlineto{\pgfqpoint{2.493724in}{1.006008in}}%
\pgfpathlineto{\pgfqpoint{2.545146in}{0.979720in}}%
\pgfpathlineto{\pgfqpoint{2.596568in}{0.988538in}}%
\pgfpathlineto{\pgfqpoint{2.647990in}{0.961282in}}%
\pgfpathlineto{\pgfqpoint{2.699413in}{0.956587in}}%
\pgfpathlineto{\pgfqpoint{2.750835in}{0.947748in}}%
\pgfpathlineto{\pgfqpoint{2.802257in}{0.931638in}}%
\pgfpathlineto{\pgfqpoint{2.853679in}{0.934600in}}%
\pgfpathlineto{\pgfqpoint{2.905101in}{0.914785in}}%
\pgfpathlineto{\pgfqpoint{2.956523in}{0.900445in}}%
\pgfpathlineto{\pgfqpoint{3.007946in}{0.888865in}}%
\pgfpathlineto{\pgfqpoint{3.059368in}{0.878353in}}%
\pgfpathlineto{\pgfqpoint{3.110790in}{0.856450in}}%
\pgfpathlineto{\pgfqpoint{3.162212in}{0.857901in}}%
\pgfpathlineto{\pgfqpoint{3.213634in}{0.843513in}}%
\pgfpathlineto{\pgfqpoint{3.265056in}{0.833080in}}%
\pgfpathlineto{\pgfqpoint{3.316478in}{0.825086in}}%
\pgfpathlineto{\pgfqpoint{3.367901in}{0.814705in}}%
\pgfpathlineto{\pgfqpoint{3.419323in}{0.795348in}}%
\pgfpathlineto{\pgfqpoint{3.470745in}{0.804693in}}%
\pgfpathlineto{\pgfqpoint{3.522167in}{0.790614in}}%
\pgfpathlineto{\pgfqpoint{3.573589in}{0.763647in}}%
\pgfpathlineto{\pgfqpoint{3.625011in}{0.758883in}}%
\pgfpathlineto{\pgfqpoint{3.676433in}{0.762346in}}%
\pgfpathlineto{\pgfqpoint{3.727856in}{0.749465in}}%
\pgfpathlineto{\pgfqpoint{3.779278in}{0.734626in}}%
\pgfpathlineto{\pgfqpoint{3.830700in}{0.723441in}}%
\pgfpathlineto{\pgfqpoint{3.882122in}{0.708348in}}%
\pgfpathlineto{\pgfqpoint{3.933544in}{0.684299in}}%
\pgfpathlineto{\pgfqpoint{3.984966in}{0.693650in}}%
\pgfpathlineto{\pgfqpoint{4.036389in}{0.679066in}}%
\pgfusepath{stroke}%
\end{pgfscope}%
\begin{pgfscope}%
\pgfpathrectangle{\pgfqpoint{0.745371in}{0.566590in}}{\pgfqpoint{3.291018in}{1.828724in}}%
\pgfusepath{clip}%
\pgfsetbuttcap%
\pgfsetroundjoin%
\definecolor{currentfill}{rgb}{0.850000,0.324000,0.098000}%
\pgfsetfillcolor{currentfill}%
\pgfsetlinewidth{1.003750pt}%
\definecolor{currentstroke}{rgb}{0.850000,0.324000,0.098000}%
\pgfsetstrokecolor{currentstroke}%
\pgfsetdash{}{0pt}%
\pgfsys@defobject{currentmarker}{\pgfqpoint{-0.041667in}{-0.041667in}}{\pgfqpoint{0.041667in}{0.041667in}}{%
\pgfpathmoveto{\pgfqpoint{-0.041667in}{0.000000in}}%
\pgfpathlineto{\pgfqpoint{0.041667in}{0.000000in}}%
\pgfpathmoveto{\pgfqpoint{0.000000in}{-0.041667in}}%
\pgfpathlineto{\pgfqpoint{0.000000in}{0.041667in}}%
\pgfusepath{stroke,fill}%
}%
\begin{pgfscope}%
\pgfsys@transformshift{0.745371in}{1.568267in}%
\pgfsys@useobject{currentmarker}{}%
\end{pgfscope}%
\begin{pgfscope}%
\pgfsys@transformshift{0.899637in}{1.500188in}%
\pgfsys@useobject{currentmarker}{}%
\end{pgfscope}%
\begin{pgfscope}%
\pgfsys@transformshift{1.053904in}{1.480754in}%
\pgfsys@useobject{currentmarker}{}%
\end{pgfscope}%
\begin{pgfscope}%
\pgfsys@transformshift{1.208170in}{1.432009in}%
\pgfsys@useobject{currentmarker}{}%
\end{pgfscope}%
\begin{pgfscope}%
\pgfsys@transformshift{1.362437in}{1.311538in}%
\pgfsys@useobject{currentmarker}{}%
\end{pgfscope}%
\begin{pgfscope}%
\pgfsys@transformshift{1.516703in}{1.225278in}%
\pgfsys@useobject{currentmarker}{}%
\end{pgfscope}%
\begin{pgfscope}%
\pgfsys@transformshift{1.670970in}{1.186483in}%
\pgfsys@useobject{currentmarker}{}%
\end{pgfscope}%
\begin{pgfscope}%
\pgfsys@transformshift{1.825236in}{1.138912in}%
\pgfsys@useobject{currentmarker}{}%
\end{pgfscope}%
\begin{pgfscope}%
\pgfsys@transformshift{1.979502in}{1.107792in}%
\pgfsys@useobject{currentmarker}{}%
\end{pgfscope}%
\begin{pgfscope}%
\pgfsys@transformshift{2.133769in}{1.074280in}%
\pgfsys@useobject{currentmarker}{}%
\end{pgfscope}%
\begin{pgfscope}%
\pgfsys@transformshift{2.288035in}{1.047940in}%
\pgfsys@useobject{currentmarker}{}%
\end{pgfscope}%
\begin{pgfscope}%
\pgfsys@transformshift{2.442302in}{1.003608in}%
\pgfsys@useobject{currentmarker}{}%
\end{pgfscope}%
\begin{pgfscope}%
\pgfsys@transformshift{2.596568in}{0.988538in}%
\pgfsys@useobject{currentmarker}{}%
\end{pgfscope}%
\begin{pgfscope}%
\pgfsys@transformshift{2.750835in}{0.947748in}%
\pgfsys@useobject{currentmarker}{}%
\end{pgfscope}%
\begin{pgfscope}%
\pgfsys@transformshift{2.905101in}{0.914785in}%
\pgfsys@useobject{currentmarker}{}%
\end{pgfscope}%
\begin{pgfscope}%
\pgfsys@transformshift{3.059368in}{0.878353in}%
\pgfsys@useobject{currentmarker}{}%
\end{pgfscope}%
\begin{pgfscope}%
\pgfsys@transformshift{3.213634in}{0.843513in}%
\pgfsys@useobject{currentmarker}{}%
\end{pgfscope}%
\begin{pgfscope}%
\pgfsys@transformshift{3.367901in}{0.814705in}%
\pgfsys@useobject{currentmarker}{}%
\end{pgfscope}%
\begin{pgfscope}%
\pgfsys@transformshift{3.522167in}{0.790614in}%
\pgfsys@useobject{currentmarker}{}%
\end{pgfscope}%
\begin{pgfscope}%
\pgfsys@transformshift{3.676433in}{0.762346in}%
\pgfsys@useobject{currentmarker}{}%
\end{pgfscope}%
\begin{pgfscope}%
\pgfsys@transformshift{3.830700in}{0.723441in}%
\pgfsys@useobject{currentmarker}{}%
\end{pgfscope}%
\begin{pgfscope}%
\pgfsys@transformshift{3.984966in}{0.693650in}%
\pgfsys@useobject{currentmarker}{}%
\end{pgfscope}%
\end{pgfscope}%
\begin{pgfscope}%
\pgfpathrectangle{\pgfqpoint{0.745371in}{0.566590in}}{\pgfqpoint{3.291018in}{1.828724in}}%
\pgfusepath{clip}%
\pgfsetrectcap%
\pgfsetroundjoin%
\pgfsetlinewidth{1.505625pt}%
\definecolor{currentstroke}{rgb}{0.000000,0.500000,0.000000}%
\pgfsetstrokecolor{currentstroke}%
\pgfsetdash{}{0pt}%
\pgfpathmoveto{\pgfqpoint{0.745371in}{1.584741in}}%
\pgfpathlineto{\pgfqpoint{0.796793in}{1.579150in}}%
\pgfpathlineto{\pgfqpoint{0.848215in}{1.537644in}}%
\pgfpathlineto{\pgfqpoint{0.899637in}{1.504713in}}%
\pgfpathlineto{\pgfqpoint{0.951059in}{1.515826in}}%
\pgfpathlineto{\pgfqpoint{1.002482in}{1.500869in}}%
\pgfpathlineto{\pgfqpoint{1.053904in}{1.503363in}}%
\pgfpathlineto{\pgfqpoint{1.105326in}{1.439797in}}%
\pgfpathlineto{\pgfqpoint{1.156748in}{1.402541in}}%
\pgfpathlineto{\pgfqpoint{1.208170in}{1.390444in}}%
\pgfpathlineto{\pgfqpoint{1.259592in}{1.363688in}}%
\pgfpathlineto{\pgfqpoint{1.311015in}{1.376343in}}%
\pgfpathlineto{\pgfqpoint{1.362437in}{1.347100in}}%
\pgfpathlineto{\pgfqpoint{1.413859in}{1.339841in}}%
\pgfpathlineto{\pgfqpoint{1.465281in}{1.335244in}}%
\pgfpathlineto{\pgfqpoint{1.516703in}{1.302148in}}%
\pgfpathlineto{\pgfqpoint{1.568125in}{1.286093in}}%
\pgfpathlineto{\pgfqpoint{1.619547in}{1.297024in}}%
\pgfpathlineto{\pgfqpoint{1.670970in}{1.255393in}}%
\pgfpathlineto{\pgfqpoint{1.722392in}{1.220109in}}%
\pgfpathlineto{\pgfqpoint{1.773814in}{1.190016in}}%
\pgfpathlineto{\pgfqpoint{1.825236in}{1.215133in}}%
\pgfpathlineto{\pgfqpoint{1.876658in}{1.172419in}}%
\pgfpathlineto{\pgfqpoint{1.928080in}{1.185061in}}%
\pgfpathlineto{\pgfqpoint{1.979502in}{1.156618in}}%
\pgfpathlineto{\pgfqpoint{2.030925in}{1.136561in}}%
\pgfpathlineto{\pgfqpoint{2.082347in}{1.149069in}}%
\pgfpathlineto{\pgfqpoint{2.133769in}{1.141124in}}%
\pgfpathlineto{\pgfqpoint{2.185191in}{1.099949in}}%
\pgfpathlineto{\pgfqpoint{2.236613in}{1.101940in}}%
\pgfpathlineto{\pgfqpoint{2.288035in}{1.084556in}}%
\pgfpathlineto{\pgfqpoint{2.339458in}{1.066799in}}%
\pgfpathlineto{\pgfqpoint{2.390880in}{1.063580in}}%
\pgfpathlineto{\pgfqpoint{2.442302in}{1.057910in}}%
\pgfpathlineto{\pgfqpoint{2.493724in}{1.043691in}}%
\pgfpathlineto{\pgfqpoint{2.545146in}{1.054639in}}%
\pgfpathlineto{\pgfqpoint{2.596568in}{1.009636in}}%
\pgfpathlineto{\pgfqpoint{2.647990in}{1.008500in}}%
\pgfpathlineto{\pgfqpoint{2.699413in}{0.996884in}}%
\pgfpathlineto{\pgfqpoint{2.750835in}{0.981680in}}%
\pgfpathlineto{\pgfqpoint{2.802257in}{0.969458in}}%
\pgfpathlineto{\pgfqpoint{2.853679in}{0.981443in}}%
\pgfpathlineto{\pgfqpoint{2.905101in}{0.961585in}}%
\pgfpathlineto{\pgfqpoint{2.956523in}{0.955017in}}%
\pgfpathlineto{\pgfqpoint{3.007946in}{0.952270in}}%
\pgfpathlineto{\pgfqpoint{3.059368in}{0.922366in}}%
\pgfpathlineto{\pgfqpoint{3.110790in}{0.915997in}}%
\pgfpathlineto{\pgfqpoint{3.162212in}{0.891197in}}%
\pgfpathlineto{\pgfqpoint{3.213634in}{0.888211in}}%
\pgfpathlineto{\pgfqpoint{3.265056in}{0.887743in}}%
\pgfpathlineto{\pgfqpoint{3.316478in}{0.869670in}}%
\pgfpathlineto{\pgfqpoint{3.367901in}{0.865340in}}%
\pgfpathlineto{\pgfqpoint{3.419323in}{0.842118in}}%
\pgfpathlineto{\pgfqpoint{3.470745in}{0.846778in}}%
\pgfpathlineto{\pgfqpoint{3.522167in}{0.835344in}}%
\pgfpathlineto{\pgfqpoint{3.573589in}{0.803226in}}%
\pgfpathlineto{\pgfqpoint{3.625011in}{0.809718in}}%
\pgfpathlineto{\pgfqpoint{3.676433in}{0.821200in}}%
\pgfpathlineto{\pgfqpoint{3.727856in}{0.792842in}}%
\pgfpathlineto{\pgfqpoint{3.779278in}{0.791994in}}%
\pgfpathlineto{\pgfqpoint{3.830700in}{0.786899in}}%
\pgfpathlineto{\pgfqpoint{3.882122in}{0.760797in}}%
\pgfpathlineto{\pgfqpoint{3.933544in}{0.750771in}}%
\pgfpathlineto{\pgfqpoint{3.984966in}{0.761898in}}%
\pgfpathlineto{\pgfqpoint{4.036389in}{0.726009in}}%
\pgfusepath{stroke}%
\end{pgfscope}%
\begin{pgfscope}%
\pgfpathrectangle{\pgfqpoint{0.745371in}{0.566590in}}{\pgfqpoint{3.291018in}{1.828724in}}%
\pgfusepath{clip}%
\pgfsetbuttcap%
\pgfsetmiterjoin%
\definecolor{currentfill}{rgb}{0.000000,0.000000,0.000000}%
\pgfsetfillcolor{currentfill}%
\pgfsetfillopacity{0.000000}%
\pgfsetlinewidth{1.003750pt}%
\definecolor{currentstroke}{rgb}{0.000000,0.500000,0.000000}%
\pgfsetstrokecolor{currentstroke}%
\pgfsetdash{}{0pt}%
\pgfsys@defobject{currentmarker}{\pgfqpoint{-0.041667in}{-0.041667in}}{\pgfqpoint{0.041667in}{0.041667in}}{%
\pgfpathmoveto{\pgfqpoint{-0.041667in}{-0.041667in}}%
\pgfpathlineto{\pgfqpoint{0.041667in}{-0.041667in}}%
\pgfpathlineto{\pgfqpoint{0.041667in}{0.041667in}}%
\pgfpathlineto{\pgfqpoint{-0.041667in}{0.041667in}}%
\pgfpathclose%
\pgfusepath{stroke,fill}%
}%
\begin{pgfscope}%
\pgfsys@transformshift{0.745371in}{1.584741in}%
\pgfsys@useobject{currentmarker}{}%
\end{pgfscope}%
\begin{pgfscope}%
\pgfsys@transformshift{1.002482in}{1.500869in}%
\pgfsys@useobject{currentmarker}{}%
\end{pgfscope}%
\begin{pgfscope}%
\pgfsys@transformshift{1.259592in}{1.363688in}%
\pgfsys@useobject{currentmarker}{}%
\end{pgfscope}%
\begin{pgfscope}%
\pgfsys@transformshift{1.516703in}{1.302148in}%
\pgfsys@useobject{currentmarker}{}%
\end{pgfscope}%
\begin{pgfscope}%
\pgfsys@transformshift{1.773814in}{1.190016in}%
\pgfsys@useobject{currentmarker}{}%
\end{pgfscope}%
\begin{pgfscope}%
\pgfsys@transformshift{2.030925in}{1.136561in}%
\pgfsys@useobject{currentmarker}{}%
\end{pgfscope}%
\begin{pgfscope}%
\pgfsys@transformshift{2.288035in}{1.084556in}%
\pgfsys@useobject{currentmarker}{}%
\end{pgfscope}%
\begin{pgfscope}%
\pgfsys@transformshift{2.545146in}{1.054639in}%
\pgfsys@useobject{currentmarker}{}%
\end{pgfscope}%
\begin{pgfscope}%
\pgfsys@transformshift{2.802257in}{0.969458in}%
\pgfsys@useobject{currentmarker}{}%
\end{pgfscope}%
\begin{pgfscope}%
\pgfsys@transformshift{3.059368in}{0.922366in}%
\pgfsys@useobject{currentmarker}{}%
\end{pgfscope}%
\begin{pgfscope}%
\pgfsys@transformshift{3.316478in}{0.869670in}%
\pgfsys@useobject{currentmarker}{}%
\end{pgfscope}%
\begin{pgfscope}%
\pgfsys@transformshift{3.573589in}{0.803226in}%
\pgfsys@useobject{currentmarker}{}%
\end{pgfscope}%
\begin{pgfscope}%
\pgfsys@transformshift{3.830700in}{0.786899in}%
\pgfsys@useobject{currentmarker}{}%
\end{pgfscope}%
\end{pgfscope}%
\begin{pgfscope}%
\pgfpathrectangle{\pgfqpoint{0.745371in}{0.566590in}}{\pgfqpoint{3.291018in}{1.828724in}}%
\pgfusepath{clip}%
\pgfsetrectcap%
\pgfsetroundjoin%
\pgfsetlinewidth{1.505625pt}%
\definecolor{currentstroke}{rgb}{0.494000,0.184000,0.556000}%
\pgfsetstrokecolor{currentstroke}%
\pgfsetdash{}{0pt}%
\pgfpathmoveto{\pgfqpoint{0.745371in}{1.651968in}}%
\pgfpathlineto{\pgfqpoint{0.796793in}{1.629177in}}%
\pgfpathlineto{\pgfqpoint{0.848215in}{1.561700in}}%
\pgfpathlineto{\pgfqpoint{0.899637in}{1.574013in}}%
\pgfpathlineto{\pgfqpoint{0.951059in}{1.526162in}}%
\pgfpathlineto{\pgfqpoint{1.002482in}{1.514915in}}%
\pgfpathlineto{\pgfqpoint{1.053904in}{1.611586in}}%
\pgfpathlineto{\pgfqpoint{1.105326in}{1.439503in}}%
\pgfpathlineto{\pgfqpoint{1.156748in}{1.394397in}}%
\pgfpathlineto{\pgfqpoint{1.208170in}{1.387878in}}%
\pgfpathlineto{\pgfqpoint{1.259592in}{1.305555in}}%
\pgfpathlineto{\pgfqpoint{1.311015in}{1.282886in}}%
\pgfpathlineto{\pgfqpoint{1.362437in}{1.271311in}}%
\pgfpathlineto{\pgfqpoint{1.413859in}{1.266862in}}%
\pgfpathlineto{\pgfqpoint{1.465281in}{1.227228in}}%
\pgfpathlineto{\pgfqpoint{1.516703in}{1.204829in}}%
\pgfpathlineto{\pgfqpoint{1.568125in}{1.221638in}}%
\pgfpathlineto{\pgfqpoint{1.619547in}{1.188819in}}%
\pgfpathlineto{\pgfqpoint{1.670970in}{1.163683in}}%
\pgfpathlineto{\pgfqpoint{1.722392in}{1.160996in}}%
\pgfpathlineto{\pgfqpoint{1.773814in}{1.154361in}}%
\pgfpathlineto{\pgfqpoint{1.825236in}{1.128806in}}%
\pgfpathlineto{\pgfqpoint{1.876658in}{1.125423in}}%
\pgfpathlineto{\pgfqpoint{1.928080in}{1.117742in}}%
\pgfpathlineto{\pgfqpoint{1.979502in}{1.109420in}}%
\pgfpathlineto{\pgfqpoint{2.030925in}{1.095625in}}%
\pgfpathlineto{\pgfqpoint{2.082347in}{1.077740in}}%
\pgfpathlineto{\pgfqpoint{2.133769in}{1.051485in}}%
\pgfpathlineto{\pgfqpoint{2.185191in}{1.054852in}}%
\pgfpathlineto{\pgfqpoint{2.236613in}{1.049329in}}%
\pgfpathlineto{\pgfqpoint{2.288035in}{1.041691in}}%
\pgfpathlineto{\pgfqpoint{2.339458in}{1.021314in}}%
\pgfpathlineto{\pgfqpoint{2.390880in}{1.012971in}}%
\pgfpathlineto{\pgfqpoint{2.442302in}{0.980888in}}%
\pgfpathlineto{\pgfqpoint{2.493724in}{0.990371in}}%
\pgfpathlineto{\pgfqpoint{2.545146in}{0.970095in}}%
\pgfpathlineto{\pgfqpoint{2.596568in}{0.967421in}}%
\pgfpathlineto{\pgfqpoint{2.647990in}{0.954134in}}%
\pgfpathlineto{\pgfqpoint{2.699413in}{0.917857in}}%
\pgfpathlineto{\pgfqpoint{2.750835in}{0.923800in}}%
\pgfpathlineto{\pgfqpoint{2.802257in}{0.908811in}}%
\pgfpathlineto{\pgfqpoint{2.853679in}{0.907790in}}%
\pgfpathlineto{\pgfqpoint{2.905101in}{0.914527in}}%
\pgfpathlineto{\pgfqpoint{2.956523in}{0.893673in}}%
\pgfpathlineto{\pgfqpoint{3.007946in}{0.877192in}}%
\pgfpathlineto{\pgfqpoint{3.059368in}{0.861323in}}%
\pgfpathlineto{\pgfqpoint{3.110790in}{0.857859in}}%
\pgfpathlineto{\pgfqpoint{3.162212in}{0.847448in}}%
\pgfpathlineto{\pgfqpoint{3.213634in}{0.834115in}}%
\pgfpathlineto{\pgfqpoint{3.265056in}{0.826701in}}%
\pgfpathlineto{\pgfqpoint{3.316478in}{0.810323in}}%
\pgfpathlineto{\pgfqpoint{3.367901in}{0.790628in}}%
\pgfpathlineto{\pgfqpoint{3.419323in}{0.793556in}}%
\pgfpathlineto{\pgfqpoint{3.470745in}{0.773442in}}%
\pgfpathlineto{\pgfqpoint{3.522167in}{0.766918in}}%
\pgfpathlineto{\pgfqpoint{3.573589in}{0.776593in}}%
\pgfpathlineto{\pgfqpoint{3.625011in}{0.731944in}}%
\pgfpathlineto{\pgfqpoint{3.676433in}{0.731514in}}%
\pgfpathlineto{\pgfqpoint{3.727856in}{0.731850in}}%
\pgfpathlineto{\pgfqpoint{3.779278in}{0.718420in}}%
\pgfpathlineto{\pgfqpoint{3.830700in}{0.705968in}}%
\pgfpathlineto{\pgfqpoint{3.882122in}{0.689561in}}%
\pgfpathlineto{\pgfqpoint{3.933544in}{0.662443in}}%
\pgfpathlineto{\pgfqpoint{3.984966in}{0.680457in}}%
\pgfpathlineto{\pgfqpoint{4.036389in}{0.649714in}}%
\pgfusepath{stroke}%
\end{pgfscope}%
\begin{pgfscope}%
\pgfpathrectangle{\pgfqpoint{0.745371in}{0.566590in}}{\pgfqpoint{3.291018in}{1.828724in}}%
\pgfusepath{clip}%
\pgfsetbuttcap%
\pgfsetroundjoin%
\definecolor{currentfill}{rgb}{0.494000,0.184000,0.556000}%
\pgfsetfillcolor{currentfill}%
\pgfsetlinewidth{1.003750pt}%
\definecolor{currentstroke}{rgb}{0.494000,0.184000,0.556000}%
\pgfsetstrokecolor{currentstroke}%
\pgfsetdash{}{0pt}%
\pgfsys@defobject{currentmarker}{\pgfqpoint{-0.041667in}{-0.041667in}}{\pgfqpoint{0.041667in}{0.041667in}}{%
\pgfpathmoveto{\pgfqpoint{-0.041667in}{-0.041667in}}%
\pgfpathlineto{\pgfqpoint{0.041667in}{0.041667in}}%
\pgfpathmoveto{\pgfqpoint{-0.041667in}{0.041667in}}%
\pgfpathlineto{\pgfqpoint{0.041667in}{-0.041667in}}%
\pgfusepath{stroke,fill}%
}%
\begin{pgfscope}%
\pgfsys@transformshift{0.745371in}{1.651968in}%
\pgfsys@useobject{currentmarker}{}%
\end{pgfscope}%
\begin{pgfscope}%
\pgfsys@transformshift{0.951059in}{1.526162in}%
\pgfsys@useobject{currentmarker}{}%
\end{pgfscope}%
\begin{pgfscope}%
\pgfsys@transformshift{1.156748in}{1.394397in}%
\pgfsys@useobject{currentmarker}{}%
\end{pgfscope}%
\begin{pgfscope}%
\pgfsys@transformshift{1.362437in}{1.271311in}%
\pgfsys@useobject{currentmarker}{}%
\end{pgfscope}%
\begin{pgfscope}%
\pgfsys@transformshift{1.568125in}{1.221638in}%
\pgfsys@useobject{currentmarker}{}%
\end{pgfscope}%
\begin{pgfscope}%
\pgfsys@transformshift{1.773814in}{1.154361in}%
\pgfsys@useobject{currentmarker}{}%
\end{pgfscope}%
\begin{pgfscope}%
\pgfsys@transformshift{1.979502in}{1.109420in}%
\pgfsys@useobject{currentmarker}{}%
\end{pgfscope}%
\begin{pgfscope}%
\pgfsys@transformshift{2.185191in}{1.054852in}%
\pgfsys@useobject{currentmarker}{}%
\end{pgfscope}%
\begin{pgfscope}%
\pgfsys@transformshift{2.390880in}{1.012971in}%
\pgfsys@useobject{currentmarker}{}%
\end{pgfscope}%
\begin{pgfscope}%
\pgfsys@transformshift{2.596568in}{0.967421in}%
\pgfsys@useobject{currentmarker}{}%
\end{pgfscope}%
\begin{pgfscope}%
\pgfsys@transformshift{2.802257in}{0.908811in}%
\pgfsys@useobject{currentmarker}{}%
\end{pgfscope}%
\begin{pgfscope}%
\pgfsys@transformshift{3.007946in}{0.877192in}%
\pgfsys@useobject{currentmarker}{}%
\end{pgfscope}%
\begin{pgfscope}%
\pgfsys@transformshift{3.213634in}{0.834115in}%
\pgfsys@useobject{currentmarker}{}%
\end{pgfscope}%
\begin{pgfscope}%
\pgfsys@transformshift{3.419323in}{0.793556in}%
\pgfsys@useobject{currentmarker}{}%
\end{pgfscope}%
\begin{pgfscope}%
\pgfsys@transformshift{3.625011in}{0.731944in}%
\pgfsys@useobject{currentmarker}{}%
\end{pgfscope}%
\begin{pgfscope}%
\pgfsys@transformshift{3.830700in}{0.705968in}%
\pgfsys@useobject{currentmarker}{}%
\end{pgfscope}%
\begin{pgfscope}%
\pgfsys@transformshift{4.036389in}{0.649714in}%
\pgfsys@useobject{currentmarker}{}%
\end{pgfscope}%
\end{pgfscope}%
\begin{pgfscope}%
\pgfsetrectcap%
\pgfsetmiterjoin%
\pgfsetlinewidth{0.803000pt}%
\definecolor{currentstroke}{rgb}{0.000000,0.000000,0.000000}%
\pgfsetstrokecolor{currentstroke}%
\pgfsetdash{}{0pt}%
\pgfpathmoveto{\pgfqpoint{0.745371in}{0.566590in}}%
\pgfpathlineto{\pgfqpoint{0.745371in}{2.395314in}}%
\pgfusepath{stroke}%
\end{pgfscope}%
\begin{pgfscope}%
\pgfsetrectcap%
\pgfsetmiterjoin%
\pgfsetlinewidth{0.803000pt}%
\definecolor{currentstroke}{rgb}{0.000000,0.000000,0.000000}%
\pgfsetstrokecolor{currentstroke}%
\pgfsetdash{}{0pt}%
\pgfpathmoveto{\pgfqpoint{4.036389in}{0.566590in}}%
\pgfpathlineto{\pgfqpoint{4.036389in}{2.395314in}}%
\pgfusepath{stroke}%
\end{pgfscope}%
\begin{pgfscope}%
\pgfsetrectcap%
\pgfsetmiterjoin%
\pgfsetlinewidth{0.803000pt}%
\definecolor{currentstroke}{rgb}{0.000000,0.000000,0.000000}%
\pgfsetstrokecolor{currentstroke}%
\pgfsetdash{}{0pt}%
\pgfpathmoveto{\pgfqpoint{0.745371in}{0.566590in}}%
\pgfpathlineto{\pgfqpoint{4.036389in}{0.566590in}}%
\pgfusepath{stroke}%
\end{pgfscope}%
\begin{pgfscope}%
\pgfsetrectcap%
\pgfsetmiterjoin%
\pgfsetlinewidth{0.803000pt}%
\definecolor{currentstroke}{rgb}{0.000000,0.000000,0.000000}%
\pgfsetstrokecolor{currentstroke}%
\pgfsetdash{}{0pt}%
\pgfpathmoveto{\pgfqpoint{0.745371in}{2.395314in}}%
\pgfpathlineto{\pgfqpoint{4.036389in}{2.395314in}}%
\pgfusepath{stroke}%
\end{pgfscope}%
\begin{pgfscope}%
\pgfsetbuttcap%
\pgfsetmiterjoin%
\definecolor{currentfill}{rgb}{1.000000,1.000000,1.000000}%
\pgfsetfillcolor{currentfill}%
\pgfsetfillopacity{0.800000}%
\pgfsetlinewidth{1.003750pt}%
\definecolor{currentstroke}{rgb}{0.800000,0.800000,0.800000}%
\pgfsetstrokecolor{currentstroke}%
\pgfsetstrokeopacity{0.800000}%
\pgfsetdash{}{0pt}%
\pgfpathmoveto{\pgfqpoint{2.910868in}{1.598092in}}%
\pgfpathlineto{\pgfqpoint{3.948889in}{1.598092in}}%
\pgfpathquadraticcurveto{\pgfqpoint{3.973889in}{1.598092in}}{\pgfqpoint{3.973889in}{1.623092in}}%
\pgfpathlineto{\pgfqpoint{3.973889in}{2.307814in}}%
\pgfpathquadraticcurveto{\pgfqpoint{3.973889in}{2.332814in}}{\pgfqpoint{3.948889in}{2.332814in}}%
\pgfpathlineto{\pgfqpoint{2.910868in}{2.332814in}}%
\pgfpathquadraticcurveto{\pgfqpoint{2.885868in}{2.332814in}}{\pgfqpoint{2.885868in}{2.307814in}}%
\pgfpathlineto{\pgfqpoint{2.885868in}{1.623092in}}%
\pgfpathquadraticcurveto{\pgfqpoint{2.885868in}{1.598092in}}{\pgfqpoint{2.910868in}{1.598092in}}%
\pgfpathclose%
\pgfusepath{stroke,fill}%
\end{pgfscope}%
\begin{pgfscope}%
\pgfsetbuttcap%
\pgfsetroundjoin%
\definecolor{currentfill}{rgb}{0.000000,0.000000,0.000000}%
\pgfsetfillcolor{currentfill}%
\pgfsetfillopacity{0.000000}%
\pgfsetlinewidth{1.003750pt}%
\definecolor{currentstroke}{rgb}{0.000000,0.447000,0.741000}%
\pgfsetstrokecolor{currentstroke}%
\pgfsetdash{}{0pt}%
\pgfsys@defobject{currentmarker}{\pgfqpoint{-0.041667in}{-0.041667in}}{\pgfqpoint{0.041667in}{0.041667in}}{%
\pgfpathmoveto{\pgfqpoint{0.000000in}{-0.041667in}}%
\pgfpathcurveto{\pgfqpoint{0.011050in}{-0.041667in}}{\pgfqpoint{0.021649in}{-0.037276in}}{\pgfqpoint{0.029463in}{-0.029463in}}%
\pgfpathcurveto{\pgfqpoint{0.037276in}{-0.021649in}}{\pgfqpoint{0.041667in}{-0.011050in}}{\pgfqpoint{0.041667in}{0.000000in}}%
\pgfpathcurveto{\pgfqpoint{0.041667in}{0.011050in}}{\pgfqpoint{0.037276in}{0.021649in}}{\pgfqpoint{0.029463in}{0.029463in}}%
\pgfpathcurveto{\pgfqpoint{0.021649in}{0.037276in}}{\pgfqpoint{0.011050in}{0.041667in}}{\pgfqpoint{0.000000in}{0.041667in}}%
\pgfpathcurveto{\pgfqpoint{-0.011050in}{0.041667in}}{\pgfqpoint{-0.021649in}{0.037276in}}{\pgfqpoint{-0.029463in}{0.029463in}}%
\pgfpathcurveto{\pgfqpoint{-0.037276in}{0.021649in}}{\pgfqpoint{-0.041667in}{0.011050in}}{\pgfqpoint{-0.041667in}{0.000000in}}%
\pgfpathcurveto{\pgfqpoint{-0.041667in}{-0.011050in}}{\pgfqpoint{-0.037276in}{-0.021649in}}{\pgfqpoint{-0.029463in}{-0.029463in}}%
\pgfpathcurveto{\pgfqpoint{-0.021649in}{-0.037276in}}{\pgfqpoint{-0.011050in}{-0.041667in}}{\pgfqpoint{0.000000in}{-0.041667in}}%
\pgfpathclose%
\pgfusepath{stroke,fill}%
}%
\begin{pgfscope}%
\pgfsys@transformshift{3.060868in}{2.239064in}%
\pgfsys@useobject{currentmarker}{}%
\end{pgfscope}%
\end{pgfscope}%
\begin{pgfscope}%
\definecolor{textcolor}{rgb}{0.000000,0.000000,0.000000}%
\pgfsetstrokecolor{textcolor}%
\pgfsetfillcolor{textcolor}%
\pgftext[x=3.285868in,y=2.195314in,left,base]{\color{textcolor}\rmfamily\fontsize{9.000000}{10.800000}\selectfont \(\displaystyle \gamma_1 = \) -0.274}%
\end{pgfscope}%
\begin{pgfscope}%
\pgfsetbuttcap%
\pgfsetroundjoin%
\definecolor{currentfill}{rgb}{0.850000,0.324000,0.098000}%
\pgfsetfillcolor{currentfill}%
\pgfsetlinewidth{1.003750pt}%
\definecolor{currentstroke}{rgb}{0.850000,0.324000,0.098000}%
\pgfsetstrokecolor{currentstroke}%
\pgfsetdash{}{0pt}%
\pgfsys@defobject{currentmarker}{\pgfqpoint{-0.041667in}{-0.041667in}}{\pgfqpoint{0.041667in}{0.041667in}}{%
\pgfpathmoveto{\pgfqpoint{-0.041667in}{0.000000in}}%
\pgfpathlineto{\pgfqpoint{0.041667in}{0.000000in}}%
\pgfpathmoveto{\pgfqpoint{0.000000in}{-0.041667in}}%
\pgfpathlineto{\pgfqpoint{0.000000in}{0.041667in}}%
\pgfusepath{stroke,fill}%
}%
\begin{pgfscope}%
\pgfsys@transformshift{3.060868in}{2.064759in}%
\pgfsys@useobject{currentmarker}{}%
\end{pgfscope}%
\end{pgfscope}%
\begin{pgfscope}%
\definecolor{textcolor}{rgb}{0.000000,0.000000,0.000000}%
\pgfsetstrokecolor{textcolor}%
\pgfsetfillcolor{textcolor}%
\pgftext[x=3.285868in,y=2.021009in,left,base]{\color{textcolor}\rmfamily\fontsize{9.000000}{10.800000}\selectfont \(\displaystyle \gamma_2 = \) -0.15}%
\end{pgfscope}%
\begin{pgfscope}%
\pgfsetbuttcap%
\pgfsetmiterjoin%
\definecolor{currentfill}{rgb}{0.000000,0.000000,0.000000}%
\pgfsetfillcolor{currentfill}%
\pgfsetfillopacity{0.000000}%
\pgfsetlinewidth{1.003750pt}%
\definecolor{currentstroke}{rgb}{0.000000,0.500000,0.000000}%
\pgfsetstrokecolor{currentstroke}%
\pgfsetdash{}{0pt}%
\pgfsys@defobject{currentmarker}{\pgfqpoint{-0.041667in}{-0.041667in}}{\pgfqpoint{0.041667in}{0.041667in}}{%
\pgfpathmoveto{\pgfqpoint{-0.041667in}{-0.041667in}}%
\pgfpathlineto{\pgfqpoint{0.041667in}{-0.041667in}}%
\pgfpathlineto{\pgfqpoint{0.041667in}{0.041667in}}%
\pgfpathlineto{\pgfqpoint{-0.041667in}{0.041667in}}%
\pgfpathclose%
\pgfusepath{stroke,fill}%
}%
\begin{pgfscope}%
\pgfsys@transformshift{3.060868in}{1.890453in}%
\pgfsys@useobject{currentmarker}{}%
\end{pgfscope}%
\end{pgfscope}%
\begin{pgfscope}%
\definecolor{textcolor}{rgb}{0.000000,0.000000,0.000000}%
\pgfsetstrokecolor{textcolor}%
\pgfsetfillcolor{textcolor}%
\pgftext[x=3.285868in,y=1.846703in,left,base]{\color{textcolor}\rmfamily\fontsize{9.000000}{10.800000}\selectfont \(\displaystyle \gamma_3 = \) 0.133}%
\end{pgfscope}%
\begin{pgfscope}%
\pgfsetbuttcap%
\pgfsetroundjoin%
\definecolor{currentfill}{rgb}{0.494000,0.184000,0.556000}%
\pgfsetfillcolor{currentfill}%
\pgfsetlinewidth{1.003750pt}%
\definecolor{currentstroke}{rgb}{0.494000,0.184000,0.556000}%
\pgfsetstrokecolor{currentstroke}%
\pgfsetdash{}{0pt}%
\pgfsys@defobject{currentmarker}{\pgfqpoint{-0.041667in}{-0.041667in}}{\pgfqpoint{0.041667in}{0.041667in}}{%
\pgfpathmoveto{\pgfqpoint{-0.041667in}{-0.041667in}}%
\pgfpathlineto{\pgfqpoint{0.041667in}{0.041667in}}%
\pgfpathmoveto{\pgfqpoint{-0.041667in}{0.041667in}}%
\pgfpathlineto{\pgfqpoint{0.041667in}{-0.041667in}}%
\pgfusepath{stroke,fill}%
}%
\begin{pgfscope}%
\pgfsys@transformshift{3.060868in}{1.716147in}%
\pgfsys@useobject{currentmarker}{}%
\end{pgfscope}%
\end{pgfscope}%
\begin{pgfscope}%
\definecolor{textcolor}{rgb}{0.000000,0.000000,0.000000}%
\pgfsetstrokecolor{textcolor}%
\pgfsetfillcolor{textcolor}%
\pgftext[x=3.285868in,y=1.672397in,left,base]{\color{textcolor}\rmfamily\fontsize{9.000000}{10.800000}\selectfont \(\displaystyle \gamma_4 = \) -0.221}%
\end{pgfscope}%
\end{pgfpicture}%
\makeatother%
\endgroup%
}
					\caption{RMSE coeficientes de amortiguación.}
					\label{Fig:RMSE_gamma_ex1}
				\end{subfigure}
				\caption{RMSE en función de SNR: Lineas punteadas corresponden a \cite{Andersson2014}; linea solida corresponde a \emph{Shift-and-Zoom}.}
				\label{Fig:RMSE_ex1}
			\end{figure}
			
			Para altas SNR, la técnica de \emph{Shift-and-Zoom} tiene rendimiento similar al método en \cite{Andersson2014}. Sin embargo, para SNR más bajas, el mejor número de condición de $\dot{z}_i$ se vuelve crítico, resultando en un rendimiento notablemente mejor de \emph{Shift-amd-Zoom} en comparación con enfoques más tradicionales. Sin embargo, es importante tener en cuenta que cuando se decima la señal, existe un compromiso entre la cantidad de muestras que quedan luego de la decimación y que tan separados resultan estar los modos complejos.
			
			Para evaluar el procedimiento, se calcula la cota de Cramér Rao (CRB) usando las expresiones desarrolladas en el apéndice \ref{App:CRB} \cite{Yao1995}. La CRB es una cota inferior de la varianza de cualquier estimador insesgado. Aunque no se demostró que los estimadores propuestos sean insesgados, se usará esta cota como indicador del margen de mejora en el procedimiento de estimación. Se observa que la CRB depende de $z_i$ y no $\xi_i$. Por lo tanto, cuando se submuestrea $y_k^{bb}$, la CRB cambiará. En la Figura \ref{Fig:CRB_ex1} se comparan las relaciones $\frac{\hat{\sigma}_{\nu}}{\sqrt{\mathrm{CRB}(\nu)}}$ y $\frac{\hat{\sigma}_{\gamma}}{\sqrt{\mathrm{CRB}(\gamma)}}$ para las estimaciones obtenidas a través del \emph{Shift-and-Zoom} y el procedimiento en \cite{Andersson2014}. 
			
			Notar que el factor de decimación tiene una cota superior determinada por el número de muestras en la señal original. Como se señaló en \cite{Yao1995} la CRB depende no sólo de la distancia entre frecuencias, sino también del número de muestras de la señal. De hecho, se señala que la CRB depende del producto de número de muestras y entre la separación de frecuencias.
			\begin{figure}[t]
				%\centering
				\begin{subfigure}{0.5\textwidth}
					\centering	
					\resizebox{\linewidth}{!}{%% Creator: Matplotlib, PGF backend
%%
%% To include the figure in your LaTeX document, write
%%   \input{<filename>.pgf}
%%
%% Make sure the required packages are loaded in your preamble
%%   \usepackage{pgf}
%%
%% and, on pdftex
%%   \usepackage[utf8]{inputenc}\DeclareUnicodeCharacter{2212}{-}
%%
%% or, on luatex and xetex
%%   \usepackage{unicode-math}
%%
%% Figures using additional raster images can only be included by \input if
%% they are in the same directory as the main LaTeX file. For loading figures
%% from other directories you can use the `import` package
%%   \usepackage{import}
%%
%% and then include the figures with
%%   \import{<path to file>}{<filename>.pgf}
%%
%% Matplotlib used the following preamble
%%   \usepackage[utf8x]{inputenc}
%%   \usepackage[T1]{fontenc}
%%   \usepackage{amsmath,amssymb,amsfonts}
%%
\begingroup%
\makeatletter%
\begin{pgfpicture}%
\pgfpathrectangle{\pgfpointorigin}{\pgfqpoint{4.136389in}{2.498642in}}%
\pgfusepath{use as bounding box, clip}%
\begin{pgfscope}%
\pgfsetbuttcap%
\pgfsetmiterjoin%
\definecolor{currentfill}{rgb}{1.000000,1.000000,1.000000}%
\pgfsetfillcolor{currentfill}%
\pgfsetlinewidth{0.000000pt}%
\definecolor{currentstroke}{rgb}{1.000000,1.000000,1.000000}%
\pgfsetstrokecolor{currentstroke}%
\pgfsetdash{}{0pt}%
\pgfpathmoveto{\pgfqpoint{0.000000in}{0.000000in}}%
\pgfpathlineto{\pgfqpoint{4.136389in}{0.000000in}}%
\pgfpathlineto{\pgfqpoint{4.136389in}{2.498642in}}%
\pgfpathlineto{\pgfqpoint{0.000000in}{2.498642in}}%
\pgfpathclose%
\pgfusepath{fill}%
\end{pgfscope}%
\begin{pgfscope}%
\pgfsetbuttcap%
\pgfsetmiterjoin%
\definecolor{currentfill}{rgb}{1.000000,1.000000,1.000000}%
\pgfsetfillcolor{currentfill}%
\pgfsetlinewidth{0.000000pt}%
\definecolor{currentstroke}{rgb}{0.000000,0.000000,0.000000}%
\pgfsetstrokecolor{currentstroke}%
\pgfsetstrokeopacity{0.000000}%
\pgfsetdash{}{0pt}%
\pgfpathmoveto{\pgfqpoint{0.898769in}{0.566590in}}%
\pgfpathlineto{\pgfqpoint{4.036389in}{0.566590in}}%
\pgfpathlineto{\pgfqpoint{4.036389in}{2.365629in}}%
\pgfpathlineto{\pgfqpoint{0.898769in}{2.365629in}}%
\pgfpathclose%
\pgfusepath{fill}%
\end{pgfscope}%
\begin{pgfscope}%
\pgfpathrectangle{\pgfqpoint{0.898769in}{0.566590in}}{\pgfqpoint{3.137619in}{1.799039in}}%
\pgfusepath{clip}%
\pgfsetrectcap%
\pgfsetroundjoin%
\pgfsetlinewidth{0.803000pt}%
\definecolor{currentstroke}{rgb}{0.690196,0.690196,0.690196}%
\pgfsetstrokecolor{currentstroke}%
\pgfsetdash{}{0pt}%
\pgfpathmoveto{\pgfqpoint{0.898769in}{0.566590in}}%
\pgfpathlineto{\pgfqpoint{0.898769in}{2.365629in}}%
\pgfusepath{stroke}%
\end{pgfscope}%
\begin{pgfscope}%
\pgfsetbuttcap%
\pgfsetroundjoin%
\definecolor{currentfill}{rgb}{0.000000,0.000000,0.000000}%
\pgfsetfillcolor{currentfill}%
\pgfsetlinewidth{0.803000pt}%
\definecolor{currentstroke}{rgb}{0.000000,0.000000,0.000000}%
\pgfsetstrokecolor{currentstroke}%
\pgfsetdash{}{0pt}%
\pgfsys@defobject{currentmarker}{\pgfqpoint{0.000000in}{-0.048611in}}{\pgfqpoint{0.000000in}{0.000000in}}{%
\pgfpathmoveto{\pgfqpoint{0.000000in}{0.000000in}}%
\pgfpathlineto{\pgfqpoint{0.000000in}{-0.048611in}}%
\pgfusepath{stroke,fill}%
}%
\begin{pgfscope}%
\pgfsys@transformshift{0.898769in}{0.566590in}%
\pgfsys@useobject{currentmarker}{}%
\end{pgfscope}%
\end{pgfscope}%
\begin{pgfscope}%
\definecolor{textcolor}{rgb}{0.000000,0.000000,0.000000}%
\pgfsetstrokecolor{textcolor}%
\pgfsetfillcolor{textcolor}%
\pgftext[x=0.898769in,y=0.469368in,,top]{\color{textcolor}\rmfamily\fontsize{12.000000}{14.400000}\selectfont \(\displaystyle {-10}\)}%
\end{pgfscope}%
\begin{pgfscope}%
\pgfpathrectangle{\pgfqpoint{0.898769in}{0.566590in}}{\pgfqpoint{3.137619in}{1.799039in}}%
\pgfusepath{clip}%
\pgfsetrectcap%
\pgfsetroundjoin%
\pgfsetlinewidth{0.803000pt}%
\definecolor{currentstroke}{rgb}{0.690196,0.690196,0.690196}%
\pgfsetstrokecolor{currentstroke}%
\pgfsetdash{}{0pt}%
\pgfpathmoveto{\pgfqpoint{1.795232in}{0.566590in}}%
\pgfpathlineto{\pgfqpoint{1.795232in}{2.365629in}}%
\pgfusepath{stroke}%
\end{pgfscope}%
\begin{pgfscope}%
\pgfsetbuttcap%
\pgfsetroundjoin%
\definecolor{currentfill}{rgb}{0.000000,0.000000,0.000000}%
\pgfsetfillcolor{currentfill}%
\pgfsetlinewidth{0.803000pt}%
\definecolor{currentstroke}{rgb}{0.000000,0.000000,0.000000}%
\pgfsetstrokecolor{currentstroke}%
\pgfsetdash{}{0pt}%
\pgfsys@defobject{currentmarker}{\pgfqpoint{0.000000in}{-0.048611in}}{\pgfqpoint{0.000000in}{0.000000in}}{%
\pgfpathmoveto{\pgfqpoint{0.000000in}{0.000000in}}%
\pgfpathlineto{\pgfqpoint{0.000000in}{-0.048611in}}%
\pgfusepath{stroke,fill}%
}%
\begin{pgfscope}%
\pgfsys@transformshift{1.795232in}{0.566590in}%
\pgfsys@useobject{currentmarker}{}%
\end{pgfscope}%
\end{pgfscope}%
\begin{pgfscope}%
\definecolor{textcolor}{rgb}{0.000000,0.000000,0.000000}%
\pgfsetstrokecolor{textcolor}%
\pgfsetfillcolor{textcolor}%
\pgftext[x=1.795232in,y=0.469368in,,top]{\color{textcolor}\rmfamily\fontsize{12.000000}{14.400000}\selectfont \(\displaystyle {0}\)}%
\end{pgfscope}%
\begin{pgfscope}%
\pgfpathrectangle{\pgfqpoint{0.898769in}{0.566590in}}{\pgfqpoint{3.137619in}{1.799039in}}%
\pgfusepath{clip}%
\pgfsetrectcap%
\pgfsetroundjoin%
\pgfsetlinewidth{0.803000pt}%
\definecolor{currentstroke}{rgb}{0.690196,0.690196,0.690196}%
\pgfsetstrokecolor{currentstroke}%
\pgfsetdash{}{0pt}%
\pgfpathmoveto{\pgfqpoint{2.691695in}{0.566590in}}%
\pgfpathlineto{\pgfqpoint{2.691695in}{2.365629in}}%
\pgfusepath{stroke}%
\end{pgfscope}%
\begin{pgfscope}%
\pgfsetbuttcap%
\pgfsetroundjoin%
\definecolor{currentfill}{rgb}{0.000000,0.000000,0.000000}%
\pgfsetfillcolor{currentfill}%
\pgfsetlinewidth{0.803000pt}%
\definecolor{currentstroke}{rgb}{0.000000,0.000000,0.000000}%
\pgfsetstrokecolor{currentstroke}%
\pgfsetdash{}{0pt}%
\pgfsys@defobject{currentmarker}{\pgfqpoint{0.000000in}{-0.048611in}}{\pgfqpoint{0.000000in}{0.000000in}}{%
\pgfpathmoveto{\pgfqpoint{0.000000in}{0.000000in}}%
\pgfpathlineto{\pgfqpoint{0.000000in}{-0.048611in}}%
\pgfusepath{stroke,fill}%
}%
\begin{pgfscope}%
\pgfsys@transformshift{2.691695in}{0.566590in}%
\pgfsys@useobject{currentmarker}{}%
\end{pgfscope}%
\end{pgfscope}%
\begin{pgfscope}%
\definecolor{textcolor}{rgb}{0.000000,0.000000,0.000000}%
\pgfsetstrokecolor{textcolor}%
\pgfsetfillcolor{textcolor}%
\pgftext[x=2.691695in,y=0.469368in,,top]{\color{textcolor}\rmfamily\fontsize{12.000000}{14.400000}\selectfont \(\displaystyle {10}\)}%
\end{pgfscope}%
\begin{pgfscope}%
\pgfpathrectangle{\pgfqpoint{0.898769in}{0.566590in}}{\pgfqpoint{3.137619in}{1.799039in}}%
\pgfusepath{clip}%
\pgfsetrectcap%
\pgfsetroundjoin%
\pgfsetlinewidth{0.803000pt}%
\definecolor{currentstroke}{rgb}{0.690196,0.690196,0.690196}%
\pgfsetstrokecolor{currentstroke}%
\pgfsetdash{}{0pt}%
\pgfpathmoveto{\pgfqpoint{3.588157in}{0.566590in}}%
\pgfpathlineto{\pgfqpoint{3.588157in}{2.365629in}}%
\pgfusepath{stroke}%
\end{pgfscope}%
\begin{pgfscope}%
\pgfsetbuttcap%
\pgfsetroundjoin%
\definecolor{currentfill}{rgb}{0.000000,0.000000,0.000000}%
\pgfsetfillcolor{currentfill}%
\pgfsetlinewidth{0.803000pt}%
\definecolor{currentstroke}{rgb}{0.000000,0.000000,0.000000}%
\pgfsetstrokecolor{currentstroke}%
\pgfsetdash{}{0pt}%
\pgfsys@defobject{currentmarker}{\pgfqpoint{0.000000in}{-0.048611in}}{\pgfqpoint{0.000000in}{0.000000in}}{%
\pgfpathmoveto{\pgfqpoint{0.000000in}{0.000000in}}%
\pgfpathlineto{\pgfqpoint{0.000000in}{-0.048611in}}%
\pgfusepath{stroke,fill}%
}%
\begin{pgfscope}%
\pgfsys@transformshift{3.588157in}{0.566590in}%
\pgfsys@useobject{currentmarker}{}%
\end{pgfscope}%
\end{pgfscope}%
\begin{pgfscope}%
\definecolor{textcolor}{rgb}{0.000000,0.000000,0.000000}%
\pgfsetstrokecolor{textcolor}%
\pgfsetfillcolor{textcolor}%
\pgftext[x=3.588157in,y=0.469368in,,top]{\color{textcolor}\rmfamily\fontsize{12.000000}{14.400000}\selectfont \(\displaystyle {20}\)}%
\end{pgfscope}%
\begin{pgfscope}%
\definecolor{textcolor}{rgb}{0.000000,0.000000,0.000000}%
\pgfsetstrokecolor{textcolor}%
\pgfsetfillcolor{textcolor}%
\pgftext[x=2.467579in,y=0.266626in,,top]{\color{textcolor}\rmfamily\fontsize{12.000000}{14.400000}\selectfont SNR [dB]}%
\end{pgfscope}%
\begin{pgfscope}%
\pgfpathrectangle{\pgfqpoint{0.898769in}{0.566590in}}{\pgfqpoint{3.137619in}{1.799039in}}%
\pgfusepath{clip}%
\pgfsetrectcap%
\pgfsetroundjoin%
\pgfsetlinewidth{0.803000pt}%
\definecolor{currentstroke}{rgb}{0.690196,0.690196,0.690196}%
\pgfsetstrokecolor{currentstroke}%
\pgfsetdash{}{0pt}%
\pgfpathmoveto{\pgfqpoint{0.898769in}{0.920122in}}%
\pgfpathlineto{\pgfqpoint{4.036389in}{0.920122in}}%
\pgfusepath{stroke}%
\end{pgfscope}%
\begin{pgfscope}%
\pgfsetbuttcap%
\pgfsetroundjoin%
\definecolor{currentfill}{rgb}{0.000000,0.000000,0.000000}%
\pgfsetfillcolor{currentfill}%
\pgfsetlinewidth{0.803000pt}%
\definecolor{currentstroke}{rgb}{0.000000,0.000000,0.000000}%
\pgfsetstrokecolor{currentstroke}%
\pgfsetdash{}{0pt}%
\pgfsys@defobject{currentmarker}{\pgfqpoint{-0.048611in}{0.000000in}}{\pgfqpoint{-0.000000in}{0.000000in}}{%
\pgfpathmoveto{\pgfqpoint{-0.000000in}{0.000000in}}%
\pgfpathlineto{\pgfqpoint{-0.048611in}{0.000000in}}%
\pgfusepath{stroke,fill}%
}%
\begin{pgfscope}%
\pgfsys@transformshift{0.898769in}{0.920122in}%
\pgfsys@useobject{currentmarker}{}%
\end{pgfscope}%
\end{pgfscope}%
\begin{pgfscope}%
\definecolor{textcolor}{rgb}{0.000000,0.000000,0.000000}%
\pgfsetstrokecolor{textcolor}%
\pgfsetfillcolor{textcolor}%
\pgftext[x=0.572381in, y=0.862729in, left, base]{\color{textcolor}\rmfamily\fontsize{12.000000}{14.400000}\selectfont \(\displaystyle {10^{0}}\)}%
\end{pgfscope}%
\begin{pgfscope}%
\pgfpathrectangle{\pgfqpoint{0.898769in}{0.566590in}}{\pgfqpoint{3.137619in}{1.799039in}}%
\pgfusepath{clip}%
\pgfsetrectcap%
\pgfsetroundjoin%
\pgfsetlinewidth{0.803000pt}%
\definecolor{currentstroke}{rgb}{0.690196,0.690196,0.690196}%
\pgfsetstrokecolor{currentstroke}%
\pgfsetdash{}{0pt}%
\pgfpathmoveto{\pgfqpoint{0.898769in}{1.393831in}}%
\pgfpathlineto{\pgfqpoint{4.036389in}{1.393831in}}%
\pgfusepath{stroke}%
\end{pgfscope}%
\begin{pgfscope}%
\pgfsetbuttcap%
\pgfsetroundjoin%
\definecolor{currentfill}{rgb}{0.000000,0.000000,0.000000}%
\pgfsetfillcolor{currentfill}%
\pgfsetlinewidth{0.803000pt}%
\definecolor{currentstroke}{rgb}{0.000000,0.000000,0.000000}%
\pgfsetstrokecolor{currentstroke}%
\pgfsetdash{}{0pt}%
\pgfsys@defobject{currentmarker}{\pgfqpoint{-0.048611in}{0.000000in}}{\pgfqpoint{-0.000000in}{0.000000in}}{%
\pgfpathmoveto{\pgfqpoint{-0.000000in}{0.000000in}}%
\pgfpathlineto{\pgfqpoint{-0.048611in}{0.000000in}}%
\pgfusepath{stroke,fill}%
}%
\begin{pgfscope}%
\pgfsys@transformshift{0.898769in}{1.393831in}%
\pgfsys@useobject{currentmarker}{}%
\end{pgfscope}%
\end{pgfscope}%
\begin{pgfscope}%
\definecolor{textcolor}{rgb}{0.000000,0.000000,0.000000}%
\pgfsetstrokecolor{textcolor}%
\pgfsetfillcolor{textcolor}%
\pgftext[x=0.572381in, y=1.336438in, left, base]{\color{textcolor}\rmfamily\fontsize{12.000000}{14.400000}\selectfont \(\displaystyle {10^{1}}\)}%
\end{pgfscope}%
\begin{pgfscope}%
\pgfpathrectangle{\pgfqpoint{0.898769in}{0.566590in}}{\pgfqpoint{3.137619in}{1.799039in}}%
\pgfusepath{clip}%
\pgfsetrectcap%
\pgfsetroundjoin%
\pgfsetlinewidth{0.803000pt}%
\definecolor{currentstroke}{rgb}{0.690196,0.690196,0.690196}%
\pgfsetstrokecolor{currentstroke}%
\pgfsetdash{}{0pt}%
\pgfpathmoveto{\pgfqpoint{0.898769in}{1.867540in}}%
\pgfpathlineto{\pgfqpoint{4.036389in}{1.867540in}}%
\pgfusepath{stroke}%
\end{pgfscope}%
\begin{pgfscope}%
\pgfsetbuttcap%
\pgfsetroundjoin%
\definecolor{currentfill}{rgb}{0.000000,0.000000,0.000000}%
\pgfsetfillcolor{currentfill}%
\pgfsetlinewidth{0.803000pt}%
\definecolor{currentstroke}{rgb}{0.000000,0.000000,0.000000}%
\pgfsetstrokecolor{currentstroke}%
\pgfsetdash{}{0pt}%
\pgfsys@defobject{currentmarker}{\pgfqpoint{-0.048611in}{0.000000in}}{\pgfqpoint{-0.000000in}{0.000000in}}{%
\pgfpathmoveto{\pgfqpoint{-0.000000in}{0.000000in}}%
\pgfpathlineto{\pgfqpoint{-0.048611in}{0.000000in}}%
\pgfusepath{stroke,fill}%
}%
\begin{pgfscope}%
\pgfsys@transformshift{0.898769in}{1.867540in}%
\pgfsys@useobject{currentmarker}{}%
\end{pgfscope}%
\end{pgfscope}%
\begin{pgfscope}%
\definecolor{textcolor}{rgb}{0.000000,0.000000,0.000000}%
\pgfsetstrokecolor{textcolor}%
\pgfsetfillcolor{textcolor}%
\pgftext[x=0.572381in, y=1.810146in, left, base]{\color{textcolor}\rmfamily\fontsize{12.000000}{14.400000}\selectfont \(\displaystyle {10^{2}}\)}%
\end{pgfscope}%
\begin{pgfscope}%
\pgfpathrectangle{\pgfqpoint{0.898769in}{0.566590in}}{\pgfqpoint{3.137619in}{1.799039in}}%
\pgfusepath{clip}%
\pgfsetrectcap%
\pgfsetroundjoin%
\pgfsetlinewidth{0.803000pt}%
\definecolor{currentstroke}{rgb}{0.690196,0.690196,0.690196}%
\pgfsetstrokecolor{currentstroke}%
\pgfsetdash{}{0pt}%
\pgfpathmoveto{\pgfqpoint{0.898769in}{2.341248in}}%
\pgfpathlineto{\pgfqpoint{4.036389in}{2.341248in}}%
\pgfusepath{stroke}%
\end{pgfscope}%
\begin{pgfscope}%
\pgfsetbuttcap%
\pgfsetroundjoin%
\definecolor{currentfill}{rgb}{0.000000,0.000000,0.000000}%
\pgfsetfillcolor{currentfill}%
\pgfsetlinewidth{0.803000pt}%
\definecolor{currentstroke}{rgb}{0.000000,0.000000,0.000000}%
\pgfsetstrokecolor{currentstroke}%
\pgfsetdash{}{0pt}%
\pgfsys@defobject{currentmarker}{\pgfqpoint{-0.048611in}{0.000000in}}{\pgfqpoint{-0.000000in}{0.000000in}}{%
\pgfpathmoveto{\pgfqpoint{-0.000000in}{0.000000in}}%
\pgfpathlineto{\pgfqpoint{-0.048611in}{0.000000in}}%
\pgfusepath{stroke,fill}%
}%
\begin{pgfscope}%
\pgfsys@transformshift{0.898769in}{2.341248in}%
\pgfsys@useobject{currentmarker}{}%
\end{pgfscope}%
\end{pgfscope}%
\begin{pgfscope}%
\definecolor{textcolor}{rgb}{0.000000,0.000000,0.000000}%
\pgfsetstrokecolor{textcolor}%
\pgfsetfillcolor{textcolor}%
\pgftext[x=0.572381in, y=2.283855in, left, base]{\color{textcolor}\rmfamily\fontsize{12.000000}{14.400000}\selectfont \(\displaystyle {10^{3}}\)}%
\end{pgfscope}%
\begin{pgfscope}%
\pgfsetbuttcap%
\pgfsetroundjoin%
\definecolor{currentfill}{rgb}{0.000000,0.000000,0.000000}%
\pgfsetfillcolor{currentfill}%
\pgfsetlinewidth{0.602250pt}%
\definecolor{currentstroke}{rgb}{0.000000,0.000000,0.000000}%
\pgfsetstrokecolor{currentstroke}%
\pgfsetdash{}{0pt}%
\pgfsys@defobject{currentmarker}{\pgfqpoint{-0.027778in}{0.000000in}}{\pgfqpoint{-0.000000in}{0.000000in}}{%
\pgfpathmoveto{\pgfqpoint{-0.000000in}{0.000000in}}%
\pgfpathlineto{\pgfqpoint{-0.027778in}{0.000000in}}%
\pgfusepath{stroke,fill}%
}%
\begin{pgfscope}%
\pgfsys@transformshift{0.898769in}{0.589014in}%
\pgfsys@useobject{currentmarker}{}%
\end{pgfscope}%
\end{pgfscope}%
\begin{pgfscope}%
\pgfsetbuttcap%
\pgfsetroundjoin%
\definecolor{currentfill}{rgb}{0.000000,0.000000,0.000000}%
\pgfsetfillcolor{currentfill}%
\pgfsetlinewidth{0.602250pt}%
\definecolor{currentstroke}{rgb}{0.000000,0.000000,0.000000}%
\pgfsetstrokecolor{currentstroke}%
\pgfsetdash{}{0pt}%
\pgfsys@defobject{currentmarker}{\pgfqpoint{-0.027778in}{0.000000in}}{\pgfqpoint{-0.000000in}{0.000000in}}{%
\pgfpathmoveto{\pgfqpoint{-0.000000in}{0.000000in}}%
\pgfpathlineto{\pgfqpoint{-0.027778in}{0.000000in}}%
\pgfusepath{stroke,fill}%
}%
\begin{pgfscope}%
\pgfsys@transformshift{0.898769in}{0.672430in}%
\pgfsys@useobject{currentmarker}{}%
\end{pgfscope}%
\end{pgfscope}%
\begin{pgfscope}%
\pgfsetbuttcap%
\pgfsetroundjoin%
\definecolor{currentfill}{rgb}{0.000000,0.000000,0.000000}%
\pgfsetfillcolor{currentfill}%
\pgfsetlinewidth{0.602250pt}%
\definecolor{currentstroke}{rgb}{0.000000,0.000000,0.000000}%
\pgfsetstrokecolor{currentstroke}%
\pgfsetdash{}{0pt}%
\pgfsys@defobject{currentmarker}{\pgfqpoint{-0.027778in}{0.000000in}}{\pgfqpoint{-0.000000in}{0.000000in}}{%
\pgfpathmoveto{\pgfqpoint{-0.000000in}{0.000000in}}%
\pgfpathlineto{\pgfqpoint{-0.027778in}{0.000000in}}%
\pgfusepath{stroke,fill}%
}%
\begin{pgfscope}%
\pgfsys@transformshift{0.898769in}{0.731615in}%
\pgfsys@useobject{currentmarker}{}%
\end{pgfscope}%
\end{pgfscope}%
\begin{pgfscope}%
\pgfsetbuttcap%
\pgfsetroundjoin%
\definecolor{currentfill}{rgb}{0.000000,0.000000,0.000000}%
\pgfsetfillcolor{currentfill}%
\pgfsetlinewidth{0.602250pt}%
\definecolor{currentstroke}{rgb}{0.000000,0.000000,0.000000}%
\pgfsetstrokecolor{currentstroke}%
\pgfsetdash{}{0pt}%
\pgfsys@defobject{currentmarker}{\pgfqpoint{-0.027778in}{0.000000in}}{\pgfqpoint{-0.000000in}{0.000000in}}{%
\pgfpathmoveto{\pgfqpoint{-0.000000in}{0.000000in}}%
\pgfpathlineto{\pgfqpoint{-0.027778in}{0.000000in}}%
\pgfusepath{stroke,fill}%
}%
\begin{pgfscope}%
\pgfsys@transformshift{0.898769in}{0.777522in}%
\pgfsys@useobject{currentmarker}{}%
\end{pgfscope}%
\end{pgfscope}%
\begin{pgfscope}%
\pgfsetbuttcap%
\pgfsetroundjoin%
\definecolor{currentfill}{rgb}{0.000000,0.000000,0.000000}%
\pgfsetfillcolor{currentfill}%
\pgfsetlinewidth{0.602250pt}%
\definecolor{currentstroke}{rgb}{0.000000,0.000000,0.000000}%
\pgfsetstrokecolor{currentstroke}%
\pgfsetdash{}{0pt}%
\pgfsys@defobject{currentmarker}{\pgfqpoint{-0.027778in}{0.000000in}}{\pgfqpoint{-0.000000in}{0.000000in}}{%
\pgfpathmoveto{\pgfqpoint{-0.000000in}{0.000000in}}%
\pgfpathlineto{\pgfqpoint{-0.027778in}{0.000000in}}%
\pgfusepath{stroke,fill}%
}%
\begin{pgfscope}%
\pgfsys@transformshift{0.898769in}{0.815031in}%
\pgfsys@useobject{currentmarker}{}%
\end{pgfscope}%
\end{pgfscope}%
\begin{pgfscope}%
\pgfsetbuttcap%
\pgfsetroundjoin%
\definecolor{currentfill}{rgb}{0.000000,0.000000,0.000000}%
\pgfsetfillcolor{currentfill}%
\pgfsetlinewidth{0.602250pt}%
\definecolor{currentstroke}{rgb}{0.000000,0.000000,0.000000}%
\pgfsetstrokecolor{currentstroke}%
\pgfsetdash{}{0pt}%
\pgfsys@defobject{currentmarker}{\pgfqpoint{-0.027778in}{0.000000in}}{\pgfqpoint{-0.000000in}{0.000000in}}{%
\pgfpathmoveto{\pgfqpoint{-0.000000in}{0.000000in}}%
\pgfpathlineto{\pgfqpoint{-0.027778in}{0.000000in}}%
\pgfusepath{stroke,fill}%
}%
\begin{pgfscope}%
\pgfsys@transformshift{0.898769in}{0.846744in}%
\pgfsys@useobject{currentmarker}{}%
\end{pgfscope}%
\end{pgfscope}%
\begin{pgfscope}%
\pgfsetbuttcap%
\pgfsetroundjoin%
\definecolor{currentfill}{rgb}{0.000000,0.000000,0.000000}%
\pgfsetfillcolor{currentfill}%
\pgfsetlinewidth{0.602250pt}%
\definecolor{currentstroke}{rgb}{0.000000,0.000000,0.000000}%
\pgfsetstrokecolor{currentstroke}%
\pgfsetdash{}{0pt}%
\pgfsys@defobject{currentmarker}{\pgfqpoint{-0.027778in}{0.000000in}}{\pgfqpoint{-0.000000in}{0.000000in}}{%
\pgfpathmoveto{\pgfqpoint{-0.000000in}{0.000000in}}%
\pgfpathlineto{\pgfqpoint{-0.027778in}{0.000000in}}%
\pgfusepath{stroke,fill}%
}%
\begin{pgfscope}%
\pgfsys@transformshift{0.898769in}{0.874215in}%
\pgfsys@useobject{currentmarker}{}%
\end{pgfscope}%
\end{pgfscope}%
\begin{pgfscope}%
\pgfsetbuttcap%
\pgfsetroundjoin%
\definecolor{currentfill}{rgb}{0.000000,0.000000,0.000000}%
\pgfsetfillcolor{currentfill}%
\pgfsetlinewidth{0.602250pt}%
\definecolor{currentstroke}{rgb}{0.000000,0.000000,0.000000}%
\pgfsetstrokecolor{currentstroke}%
\pgfsetdash{}{0pt}%
\pgfsys@defobject{currentmarker}{\pgfqpoint{-0.027778in}{0.000000in}}{\pgfqpoint{-0.000000in}{0.000000in}}{%
\pgfpathmoveto{\pgfqpoint{-0.000000in}{0.000000in}}%
\pgfpathlineto{\pgfqpoint{-0.027778in}{0.000000in}}%
\pgfusepath{stroke,fill}%
}%
\begin{pgfscope}%
\pgfsys@transformshift{0.898769in}{0.898447in}%
\pgfsys@useobject{currentmarker}{}%
\end{pgfscope}%
\end{pgfscope}%
\begin{pgfscope}%
\pgfsetbuttcap%
\pgfsetroundjoin%
\definecolor{currentfill}{rgb}{0.000000,0.000000,0.000000}%
\pgfsetfillcolor{currentfill}%
\pgfsetlinewidth{0.602250pt}%
\definecolor{currentstroke}{rgb}{0.000000,0.000000,0.000000}%
\pgfsetstrokecolor{currentstroke}%
\pgfsetdash{}{0pt}%
\pgfsys@defobject{currentmarker}{\pgfqpoint{-0.027778in}{0.000000in}}{\pgfqpoint{-0.000000in}{0.000000in}}{%
\pgfpathmoveto{\pgfqpoint{-0.000000in}{0.000000in}}%
\pgfpathlineto{\pgfqpoint{-0.027778in}{0.000000in}}%
\pgfusepath{stroke,fill}%
}%
\begin{pgfscope}%
\pgfsys@transformshift{0.898769in}{1.062723in}%
\pgfsys@useobject{currentmarker}{}%
\end{pgfscope}%
\end{pgfscope}%
\begin{pgfscope}%
\pgfsetbuttcap%
\pgfsetroundjoin%
\definecolor{currentfill}{rgb}{0.000000,0.000000,0.000000}%
\pgfsetfillcolor{currentfill}%
\pgfsetlinewidth{0.602250pt}%
\definecolor{currentstroke}{rgb}{0.000000,0.000000,0.000000}%
\pgfsetstrokecolor{currentstroke}%
\pgfsetdash{}{0pt}%
\pgfsys@defobject{currentmarker}{\pgfqpoint{-0.027778in}{0.000000in}}{\pgfqpoint{-0.000000in}{0.000000in}}{%
\pgfpathmoveto{\pgfqpoint{-0.000000in}{0.000000in}}%
\pgfpathlineto{\pgfqpoint{-0.027778in}{0.000000in}}%
\pgfusepath{stroke,fill}%
}%
\begin{pgfscope}%
\pgfsys@transformshift{0.898769in}{1.146139in}%
\pgfsys@useobject{currentmarker}{}%
\end{pgfscope}%
\end{pgfscope}%
\begin{pgfscope}%
\pgfsetbuttcap%
\pgfsetroundjoin%
\definecolor{currentfill}{rgb}{0.000000,0.000000,0.000000}%
\pgfsetfillcolor{currentfill}%
\pgfsetlinewidth{0.602250pt}%
\definecolor{currentstroke}{rgb}{0.000000,0.000000,0.000000}%
\pgfsetstrokecolor{currentstroke}%
\pgfsetdash{}{0pt}%
\pgfsys@defobject{currentmarker}{\pgfqpoint{-0.027778in}{0.000000in}}{\pgfqpoint{-0.000000in}{0.000000in}}{%
\pgfpathmoveto{\pgfqpoint{-0.000000in}{0.000000in}}%
\pgfpathlineto{\pgfqpoint{-0.027778in}{0.000000in}}%
\pgfusepath{stroke,fill}%
}%
\begin{pgfscope}%
\pgfsys@transformshift{0.898769in}{1.205323in}%
\pgfsys@useobject{currentmarker}{}%
\end{pgfscope}%
\end{pgfscope}%
\begin{pgfscope}%
\pgfsetbuttcap%
\pgfsetroundjoin%
\definecolor{currentfill}{rgb}{0.000000,0.000000,0.000000}%
\pgfsetfillcolor{currentfill}%
\pgfsetlinewidth{0.602250pt}%
\definecolor{currentstroke}{rgb}{0.000000,0.000000,0.000000}%
\pgfsetstrokecolor{currentstroke}%
\pgfsetdash{}{0pt}%
\pgfsys@defobject{currentmarker}{\pgfqpoint{-0.027778in}{0.000000in}}{\pgfqpoint{-0.000000in}{0.000000in}}{%
\pgfpathmoveto{\pgfqpoint{-0.000000in}{0.000000in}}%
\pgfpathlineto{\pgfqpoint{-0.027778in}{0.000000in}}%
\pgfusepath{stroke,fill}%
}%
\begin{pgfscope}%
\pgfsys@transformshift{0.898769in}{1.251231in}%
\pgfsys@useobject{currentmarker}{}%
\end{pgfscope}%
\end{pgfscope}%
\begin{pgfscope}%
\pgfsetbuttcap%
\pgfsetroundjoin%
\definecolor{currentfill}{rgb}{0.000000,0.000000,0.000000}%
\pgfsetfillcolor{currentfill}%
\pgfsetlinewidth{0.602250pt}%
\definecolor{currentstroke}{rgb}{0.000000,0.000000,0.000000}%
\pgfsetstrokecolor{currentstroke}%
\pgfsetdash{}{0pt}%
\pgfsys@defobject{currentmarker}{\pgfqpoint{-0.027778in}{0.000000in}}{\pgfqpoint{-0.000000in}{0.000000in}}{%
\pgfpathmoveto{\pgfqpoint{-0.000000in}{0.000000in}}%
\pgfpathlineto{\pgfqpoint{-0.027778in}{0.000000in}}%
\pgfusepath{stroke,fill}%
}%
\begin{pgfscope}%
\pgfsys@transformshift{0.898769in}{1.288739in}%
\pgfsys@useobject{currentmarker}{}%
\end{pgfscope}%
\end{pgfscope}%
\begin{pgfscope}%
\pgfsetbuttcap%
\pgfsetroundjoin%
\definecolor{currentfill}{rgb}{0.000000,0.000000,0.000000}%
\pgfsetfillcolor{currentfill}%
\pgfsetlinewidth{0.602250pt}%
\definecolor{currentstroke}{rgb}{0.000000,0.000000,0.000000}%
\pgfsetstrokecolor{currentstroke}%
\pgfsetdash{}{0pt}%
\pgfsys@defobject{currentmarker}{\pgfqpoint{-0.027778in}{0.000000in}}{\pgfqpoint{-0.000000in}{0.000000in}}{%
\pgfpathmoveto{\pgfqpoint{-0.000000in}{0.000000in}}%
\pgfpathlineto{\pgfqpoint{-0.027778in}{0.000000in}}%
\pgfusepath{stroke,fill}%
}%
\begin{pgfscope}%
\pgfsys@transformshift{0.898769in}{1.320453in}%
\pgfsys@useobject{currentmarker}{}%
\end{pgfscope}%
\end{pgfscope}%
\begin{pgfscope}%
\pgfsetbuttcap%
\pgfsetroundjoin%
\definecolor{currentfill}{rgb}{0.000000,0.000000,0.000000}%
\pgfsetfillcolor{currentfill}%
\pgfsetlinewidth{0.602250pt}%
\definecolor{currentstroke}{rgb}{0.000000,0.000000,0.000000}%
\pgfsetstrokecolor{currentstroke}%
\pgfsetdash{}{0pt}%
\pgfsys@defobject{currentmarker}{\pgfqpoint{-0.027778in}{0.000000in}}{\pgfqpoint{-0.000000in}{0.000000in}}{%
\pgfpathmoveto{\pgfqpoint{-0.000000in}{0.000000in}}%
\pgfpathlineto{\pgfqpoint{-0.027778in}{0.000000in}}%
\pgfusepath{stroke,fill}%
}%
\begin{pgfscope}%
\pgfsys@transformshift{0.898769in}{1.347924in}%
\pgfsys@useobject{currentmarker}{}%
\end{pgfscope}%
\end{pgfscope}%
\begin{pgfscope}%
\pgfsetbuttcap%
\pgfsetroundjoin%
\definecolor{currentfill}{rgb}{0.000000,0.000000,0.000000}%
\pgfsetfillcolor{currentfill}%
\pgfsetlinewidth{0.602250pt}%
\definecolor{currentstroke}{rgb}{0.000000,0.000000,0.000000}%
\pgfsetstrokecolor{currentstroke}%
\pgfsetdash{}{0pt}%
\pgfsys@defobject{currentmarker}{\pgfqpoint{-0.027778in}{0.000000in}}{\pgfqpoint{-0.000000in}{0.000000in}}{%
\pgfpathmoveto{\pgfqpoint{-0.000000in}{0.000000in}}%
\pgfpathlineto{\pgfqpoint{-0.027778in}{0.000000in}}%
\pgfusepath{stroke,fill}%
}%
\begin{pgfscope}%
\pgfsys@transformshift{0.898769in}{1.372155in}%
\pgfsys@useobject{currentmarker}{}%
\end{pgfscope}%
\end{pgfscope}%
\begin{pgfscope}%
\pgfsetbuttcap%
\pgfsetroundjoin%
\definecolor{currentfill}{rgb}{0.000000,0.000000,0.000000}%
\pgfsetfillcolor{currentfill}%
\pgfsetlinewidth{0.602250pt}%
\definecolor{currentstroke}{rgb}{0.000000,0.000000,0.000000}%
\pgfsetstrokecolor{currentstroke}%
\pgfsetdash{}{0pt}%
\pgfsys@defobject{currentmarker}{\pgfqpoint{-0.027778in}{0.000000in}}{\pgfqpoint{-0.000000in}{0.000000in}}{%
\pgfpathmoveto{\pgfqpoint{-0.000000in}{0.000000in}}%
\pgfpathlineto{\pgfqpoint{-0.027778in}{0.000000in}}%
\pgfusepath{stroke,fill}%
}%
\begin{pgfscope}%
\pgfsys@transformshift{0.898769in}{1.536432in}%
\pgfsys@useobject{currentmarker}{}%
\end{pgfscope}%
\end{pgfscope}%
\begin{pgfscope}%
\pgfsetbuttcap%
\pgfsetroundjoin%
\definecolor{currentfill}{rgb}{0.000000,0.000000,0.000000}%
\pgfsetfillcolor{currentfill}%
\pgfsetlinewidth{0.602250pt}%
\definecolor{currentstroke}{rgb}{0.000000,0.000000,0.000000}%
\pgfsetstrokecolor{currentstroke}%
\pgfsetdash{}{0pt}%
\pgfsys@defobject{currentmarker}{\pgfqpoint{-0.027778in}{0.000000in}}{\pgfqpoint{-0.000000in}{0.000000in}}{%
\pgfpathmoveto{\pgfqpoint{-0.000000in}{0.000000in}}%
\pgfpathlineto{\pgfqpoint{-0.027778in}{0.000000in}}%
\pgfusepath{stroke,fill}%
}%
\begin{pgfscope}%
\pgfsys@transformshift{0.898769in}{1.619848in}%
\pgfsys@useobject{currentmarker}{}%
\end{pgfscope}%
\end{pgfscope}%
\begin{pgfscope}%
\pgfsetbuttcap%
\pgfsetroundjoin%
\definecolor{currentfill}{rgb}{0.000000,0.000000,0.000000}%
\pgfsetfillcolor{currentfill}%
\pgfsetlinewidth{0.602250pt}%
\definecolor{currentstroke}{rgb}{0.000000,0.000000,0.000000}%
\pgfsetstrokecolor{currentstroke}%
\pgfsetdash{}{0pt}%
\pgfsys@defobject{currentmarker}{\pgfqpoint{-0.027778in}{0.000000in}}{\pgfqpoint{-0.000000in}{0.000000in}}{%
\pgfpathmoveto{\pgfqpoint{-0.000000in}{0.000000in}}%
\pgfpathlineto{\pgfqpoint{-0.027778in}{0.000000in}}%
\pgfusepath{stroke,fill}%
}%
\begin{pgfscope}%
\pgfsys@transformshift{0.898769in}{1.679032in}%
\pgfsys@useobject{currentmarker}{}%
\end{pgfscope}%
\end{pgfscope}%
\begin{pgfscope}%
\pgfsetbuttcap%
\pgfsetroundjoin%
\definecolor{currentfill}{rgb}{0.000000,0.000000,0.000000}%
\pgfsetfillcolor{currentfill}%
\pgfsetlinewidth{0.602250pt}%
\definecolor{currentstroke}{rgb}{0.000000,0.000000,0.000000}%
\pgfsetstrokecolor{currentstroke}%
\pgfsetdash{}{0pt}%
\pgfsys@defobject{currentmarker}{\pgfqpoint{-0.027778in}{0.000000in}}{\pgfqpoint{-0.000000in}{0.000000in}}{%
\pgfpathmoveto{\pgfqpoint{-0.000000in}{0.000000in}}%
\pgfpathlineto{\pgfqpoint{-0.027778in}{0.000000in}}%
\pgfusepath{stroke,fill}%
}%
\begin{pgfscope}%
\pgfsys@transformshift{0.898769in}{1.724939in}%
\pgfsys@useobject{currentmarker}{}%
\end{pgfscope}%
\end{pgfscope}%
\begin{pgfscope}%
\pgfsetbuttcap%
\pgfsetroundjoin%
\definecolor{currentfill}{rgb}{0.000000,0.000000,0.000000}%
\pgfsetfillcolor{currentfill}%
\pgfsetlinewidth{0.602250pt}%
\definecolor{currentstroke}{rgb}{0.000000,0.000000,0.000000}%
\pgfsetstrokecolor{currentstroke}%
\pgfsetdash{}{0pt}%
\pgfsys@defobject{currentmarker}{\pgfqpoint{-0.027778in}{0.000000in}}{\pgfqpoint{-0.000000in}{0.000000in}}{%
\pgfpathmoveto{\pgfqpoint{-0.000000in}{0.000000in}}%
\pgfpathlineto{\pgfqpoint{-0.027778in}{0.000000in}}%
\pgfusepath{stroke,fill}%
}%
\begin{pgfscope}%
\pgfsys@transformshift{0.898769in}{1.762448in}%
\pgfsys@useobject{currentmarker}{}%
\end{pgfscope}%
\end{pgfscope}%
\begin{pgfscope}%
\pgfsetbuttcap%
\pgfsetroundjoin%
\definecolor{currentfill}{rgb}{0.000000,0.000000,0.000000}%
\pgfsetfillcolor{currentfill}%
\pgfsetlinewidth{0.602250pt}%
\definecolor{currentstroke}{rgb}{0.000000,0.000000,0.000000}%
\pgfsetstrokecolor{currentstroke}%
\pgfsetdash{}{0pt}%
\pgfsys@defobject{currentmarker}{\pgfqpoint{-0.027778in}{0.000000in}}{\pgfqpoint{-0.000000in}{0.000000in}}{%
\pgfpathmoveto{\pgfqpoint{-0.000000in}{0.000000in}}%
\pgfpathlineto{\pgfqpoint{-0.027778in}{0.000000in}}%
\pgfusepath{stroke,fill}%
}%
\begin{pgfscope}%
\pgfsys@transformshift{0.898769in}{1.794161in}%
\pgfsys@useobject{currentmarker}{}%
\end{pgfscope}%
\end{pgfscope}%
\begin{pgfscope}%
\pgfsetbuttcap%
\pgfsetroundjoin%
\definecolor{currentfill}{rgb}{0.000000,0.000000,0.000000}%
\pgfsetfillcolor{currentfill}%
\pgfsetlinewidth{0.602250pt}%
\definecolor{currentstroke}{rgb}{0.000000,0.000000,0.000000}%
\pgfsetstrokecolor{currentstroke}%
\pgfsetdash{}{0pt}%
\pgfsys@defobject{currentmarker}{\pgfqpoint{-0.027778in}{0.000000in}}{\pgfqpoint{-0.000000in}{0.000000in}}{%
\pgfpathmoveto{\pgfqpoint{-0.000000in}{0.000000in}}%
\pgfpathlineto{\pgfqpoint{-0.027778in}{0.000000in}}%
\pgfusepath{stroke,fill}%
}%
\begin{pgfscope}%
\pgfsys@transformshift{0.898769in}{1.821633in}%
\pgfsys@useobject{currentmarker}{}%
\end{pgfscope}%
\end{pgfscope}%
\begin{pgfscope}%
\pgfsetbuttcap%
\pgfsetroundjoin%
\definecolor{currentfill}{rgb}{0.000000,0.000000,0.000000}%
\pgfsetfillcolor{currentfill}%
\pgfsetlinewidth{0.602250pt}%
\definecolor{currentstroke}{rgb}{0.000000,0.000000,0.000000}%
\pgfsetstrokecolor{currentstroke}%
\pgfsetdash{}{0pt}%
\pgfsys@defobject{currentmarker}{\pgfqpoint{-0.027778in}{0.000000in}}{\pgfqpoint{-0.000000in}{0.000000in}}{%
\pgfpathmoveto{\pgfqpoint{-0.000000in}{0.000000in}}%
\pgfpathlineto{\pgfqpoint{-0.027778in}{0.000000in}}%
\pgfusepath{stroke,fill}%
}%
\begin{pgfscope}%
\pgfsys@transformshift{0.898769in}{1.845864in}%
\pgfsys@useobject{currentmarker}{}%
\end{pgfscope}%
\end{pgfscope}%
\begin{pgfscope}%
\pgfsetbuttcap%
\pgfsetroundjoin%
\definecolor{currentfill}{rgb}{0.000000,0.000000,0.000000}%
\pgfsetfillcolor{currentfill}%
\pgfsetlinewidth{0.602250pt}%
\definecolor{currentstroke}{rgb}{0.000000,0.000000,0.000000}%
\pgfsetstrokecolor{currentstroke}%
\pgfsetdash{}{0pt}%
\pgfsys@defobject{currentmarker}{\pgfqpoint{-0.027778in}{0.000000in}}{\pgfqpoint{-0.000000in}{0.000000in}}{%
\pgfpathmoveto{\pgfqpoint{-0.000000in}{0.000000in}}%
\pgfpathlineto{\pgfqpoint{-0.027778in}{0.000000in}}%
\pgfusepath{stroke,fill}%
}%
\begin{pgfscope}%
\pgfsys@transformshift{0.898769in}{2.010140in}%
\pgfsys@useobject{currentmarker}{}%
\end{pgfscope}%
\end{pgfscope}%
\begin{pgfscope}%
\pgfsetbuttcap%
\pgfsetroundjoin%
\definecolor{currentfill}{rgb}{0.000000,0.000000,0.000000}%
\pgfsetfillcolor{currentfill}%
\pgfsetlinewidth{0.602250pt}%
\definecolor{currentstroke}{rgb}{0.000000,0.000000,0.000000}%
\pgfsetstrokecolor{currentstroke}%
\pgfsetdash{}{0pt}%
\pgfsys@defobject{currentmarker}{\pgfqpoint{-0.027778in}{0.000000in}}{\pgfqpoint{-0.000000in}{0.000000in}}{%
\pgfpathmoveto{\pgfqpoint{-0.000000in}{0.000000in}}%
\pgfpathlineto{\pgfqpoint{-0.027778in}{0.000000in}}%
\pgfusepath{stroke,fill}%
}%
\begin{pgfscope}%
\pgfsys@transformshift{0.898769in}{2.093556in}%
\pgfsys@useobject{currentmarker}{}%
\end{pgfscope}%
\end{pgfscope}%
\begin{pgfscope}%
\pgfsetbuttcap%
\pgfsetroundjoin%
\definecolor{currentfill}{rgb}{0.000000,0.000000,0.000000}%
\pgfsetfillcolor{currentfill}%
\pgfsetlinewidth{0.602250pt}%
\definecolor{currentstroke}{rgb}{0.000000,0.000000,0.000000}%
\pgfsetstrokecolor{currentstroke}%
\pgfsetdash{}{0pt}%
\pgfsys@defobject{currentmarker}{\pgfqpoint{-0.027778in}{0.000000in}}{\pgfqpoint{-0.000000in}{0.000000in}}{%
\pgfpathmoveto{\pgfqpoint{-0.000000in}{0.000000in}}%
\pgfpathlineto{\pgfqpoint{-0.027778in}{0.000000in}}%
\pgfusepath{stroke,fill}%
}%
\begin{pgfscope}%
\pgfsys@transformshift{0.898769in}{2.152741in}%
\pgfsys@useobject{currentmarker}{}%
\end{pgfscope}%
\end{pgfscope}%
\begin{pgfscope}%
\pgfsetbuttcap%
\pgfsetroundjoin%
\definecolor{currentfill}{rgb}{0.000000,0.000000,0.000000}%
\pgfsetfillcolor{currentfill}%
\pgfsetlinewidth{0.602250pt}%
\definecolor{currentstroke}{rgb}{0.000000,0.000000,0.000000}%
\pgfsetstrokecolor{currentstroke}%
\pgfsetdash{}{0pt}%
\pgfsys@defobject{currentmarker}{\pgfqpoint{-0.027778in}{0.000000in}}{\pgfqpoint{-0.000000in}{0.000000in}}{%
\pgfpathmoveto{\pgfqpoint{-0.000000in}{0.000000in}}%
\pgfpathlineto{\pgfqpoint{-0.027778in}{0.000000in}}%
\pgfusepath{stroke,fill}%
}%
\begin{pgfscope}%
\pgfsys@transformshift{0.898769in}{2.198648in}%
\pgfsys@useobject{currentmarker}{}%
\end{pgfscope}%
\end{pgfscope}%
\begin{pgfscope}%
\pgfsetbuttcap%
\pgfsetroundjoin%
\definecolor{currentfill}{rgb}{0.000000,0.000000,0.000000}%
\pgfsetfillcolor{currentfill}%
\pgfsetlinewidth{0.602250pt}%
\definecolor{currentstroke}{rgb}{0.000000,0.000000,0.000000}%
\pgfsetstrokecolor{currentstroke}%
\pgfsetdash{}{0pt}%
\pgfsys@defobject{currentmarker}{\pgfqpoint{-0.027778in}{0.000000in}}{\pgfqpoint{-0.000000in}{0.000000in}}{%
\pgfpathmoveto{\pgfqpoint{-0.000000in}{0.000000in}}%
\pgfpathlineto{\pgfqpoint{-0.027778in}{0.000000in}}%
\pgfusepath{stroke,fill}%
}%
\begin{pgfscope}%
\pgfsys@transformshift{0.898769in}{2.236157in}%
\pgfsys@useobject{currentmarker}{}%
\end{pgfscope}%
\end{pgfscope}%
\begin{pgfscope}%
\pgfsetbuttcap%
\pgfsetroundjoin%
\definecolor{currentfill}{rgb}{0.000000,0.000000,0.000000}%
\pgfsetfillcolor{currentfill}%
\pgfsetlinewidth{0.602250pt}%
\definecolor{currentstroke}{rgb}{0.000000,0.000000,0.000000}%
\pgfsetstrokecolor{currentstroke}%
\pgfsetdash{}{0pt}%
\pgfsys@defobject{currentmarker}{\pgfqpoint{-0.027778in}{0.000000in}}{\pgfqpoint{-0.000000in}{0.000000in}}{%
\pgfpathmoveto{\pgfqpoint{-0.000000in}{0.000000in}}%
\pgfpathlineto{\pgfqpoint{-0.027778in}{0.000000in}}%
\pgfusepath{stroke,fill}%
}%
\begin{pgfscope}%
\pgfsys@transformshift{0.898769in}{2.267870in}%
\pgfsys@useobject{currentmarker}{}%
\end{pgfscope}%
\end{pgfscope}%
\begin{pgfscope}%
\pgfsetbuttcap%
\pgfsetroundjoin%
\definecolor{currentfill}{rgb}{0.000000,0.000000,0.000000}%
\pgfsetfillcolor{currentfill}%
\pgfsetlinewidth{0.602250pt}%
\definecolor{currentstroke}{rgb}{0.000000,0.000000,0.000000}%
\pgfsetstrokecolor{currentstroke}%
\pgfsetdash{}{0pt}%
\pgfsys@defobject{currentmarker}{\pgfqpoint{-0.027778in}{0.000000in}}{\pgfqpoint{-0.000000in}{0.000000in}}{%
\pgfpathmoveto{\pgfqpoint{-0.000000in}{0.000000in}}%
\pgfpathlineto{\pgfqpoint{-0.027778in}{0.000000in}}%
\pgfusepath{stroke,fill}%
}%
\begin{pgfscope}%
\pgfsys@transformshift{0.898769in}{2.295341in}%
\pgfsys@useobject{currentmarker}{}%
\end{pgfscope}%
\end{pgfscope}%
\begin{pgfscope}%
\pgfsetbuttcap%
\pgfsetroundjoin%
\definecolor{currentfill}{rgb}{0.000000,0.000000,0.000000}%
\pgfsetfillcolor{currentfill}%
\pgfsetlinewidth{0.602250pt}%
\definecolor{currentstroke}{rgb}{0.000000,0.000000,0.000000}%
\pgfsetstrokecolor{currentstroke}%
\pgfsetdash{}{0pt}%
\pgfsys@defobject{currentmarker}{\pgfqpoint{-0.027778in}{0.000000in}}{\pgfqpoint{-0.000000in}{0.000000in}}{%
\pgfpathmoveto{\pgfqpoint{-0.000000in}{0.000000in}}%
\pgfpathlineto{\pgfqpoint{-0.027778in}{0.000000in}}%
\pgfusepath{stroke,fill}%
}%
\begin{pgfscope}%
\pgfsys@transformshift{0.898769in}{2.319573in}%
\pgfsys@useobject{currentmarker}{}%
\end{pgfscope}%
\end{pgfscope}%
\begin{pgfscope}%
\definecolor{textcolor}{rgb}{0.000000,0.000000,0.000000}%
\pgfsetstrokecolor{textcolor}%
\pgfsetfillcolor{textcolor}%
\pgftext[x=0.516826in,y=1.466109in,,bottom,rotate=90.000000]{\color{textcolor}\rmfamily\fontsize{12.000000}{14.400000}\selectfont \(\displaystyle \frac{\hat{\sigma}_{\nu}}{\sqrt{\mathrm{CRB}(\nu)}}\)}%
\end{pgfscope}%
\begin{pgfscope}%
\pgfpathrectangle{\pgfqpoint{0.898769in}{0.566590in}}{\pgfqpoint{3.137619in}{1.799039in}}%
\pgfusepath{clip}%
\pgfsetbuttcap%
\pgfsetroundjoin%
\pgfsetlinewidth{1.505625pt}%
\definecolor{currentstroke}{rgb}{0.000000,0.447000,0.741000}%
\pgfsetstrokecolor{currentstroke}%
\pgfsetdash{{5.550000pt}{2.400000pt}}{0.000000pt}%
\pgfpathmoveto{\pgfqpoint{0.898769in}{1.851974in}}%
\pgfpathlineto{\pgfqpoint{0.947794in}{1.869734in}}%
\pgfpathlineto{\pgfqpoint{0.996820in}{1.882032in}}%
\pgfpathlineto{\pgfqpoint{1.045845in}{1.908523in}}%
\pgfpathlineto{\pgfqpoint{1.094870in}{1.915371in}}%
\pgfpathlineto{\pgfqpoint{1.143896in}{1.927592in}}%
\pgfpathlineto{\pgfqpoint{1.192921in}{1.941645in}}%
\pgfpathlineto{\pgfqpoint{1.241946in}{1.946120in}}%
\pgfpathlineto{\pgfqpoint{1.290972in}{1.970402in}}%
\pgfpathlineto{\pgfqpoint{1.339997in}{1.972996in}}%
\pgfpathlineto{\pgfqpoint{1.389022in}{1.965961in}}%
\pgfpathlineto{\pgfqpoint{1.438047in}{1.990289in}}%
\pgfpathlineto{\pgfqpoint{1.487073in}{1.973695in}}%
\pgfpathlineto{\pgfqpoint{1.536098in}{1.981620in}}%
\pgfpathlineto{\pgfqpoint{1.585123in}{1.965638in}}%
\pgfpathlineto{\pgfqpoint{1.634149in}{1.959152in}}%
\pgfpathlineto{\pgfqpoint{1.683174in}{1.921713in}}%
\pgfpathlineto{\pgfqpoint{1.732199in}{1.874760in}}%
\pgfpathlineto{\pgfqpoint{1.781225in}{1.889985in}}%
\pgfpathlineto{\pgfqpoint{1.830250in}{1.816323in}}%
\pgfpathlineto{\pgfqpoint{1.879275in}{1.770743in}}%
\pgfpathlineto{\pgfqpoint{1.928301in}{0.942524in}}%
\pgfpathlineto{\pgfqpoint{1.977326in}{1.659992in}}%
\pgfpathlineto{\pgfqpoint{2.026351in}{1.664623in}}%
\pgfpathlineto{\pgfqpoint{2.075376in}{0.937462in}}%
\pgfpathlineto{\pgfqpoint{2.124402in}{0.959132in}}%
\pgfpathlineto{\pgfqpoint{2.173427in}{0.918806in}}%
\pgfpathlineto{\pgfqpoint{2.222452in}{0.948350in}}%
\pgfpathlineto{\pgfqpoint{2.271478in}{0.921204in}}%
\pgfpathlineto{\pgfqpoint{2.320503in}{0.911052in}}%
\pgfpathlineto{\pgfqpoint{2.369528in}{0.927658in}}%
\pgfpathlineto{\pgfqpoint{2.418554in}{0.927720in}}%
\pgfpathlineto{\pgfqpoint{2.467579in}{0.949190in}}%
\pgfpathlineto{\pgfqpoint{2.516604in}{0.926206in}}%
\pgfpathlineto{\pgfqpoint{2.565629in}{0.917956in}}%
\pgfpathlineto{\pgfqpoint{2.614655in}{0.938927in}}%
\pgfpathlineto{\pgfqpoint{2.663680in}{0.917058in}}%
\pgfpathlineto{\pgfqpoint{2.712705in}{0.933507in}}%
\pgfpathlineto{\pgfqpoint{2.761731in}{0.931342in}}%
\pgfpathlineto{\pgfqpoint{2.810756in}{0.929980in}}%
\pgfpathlineto{\pgfqpoint{2.859781in}{0.916252in}}%
\pgfpathlineto{\pgfqpoint{2.908807in}{0.903075in}}%
\pgfpathlineto{\pgfqpoint{2.957832in}{0.905711in}}%
\pgfpathlineto{\pgfqpoint{3.006857in}{0.906616in}}%
\pgfpathlineto{\pgfqpoint{3.055882in}{0.921433in}}%
\pgfpathlineto{\pgfqpoint{3.104908in}{0.930556in}}%
\pgfpathlineto{\pgfqpoint{3.153933in}{0.930576in}}%
\pgfpathlineto{\pgfqpoint{3.202958in}{0.924570in}}%
\pgfpathlineto{\pgfqpoint{3.251984in}{0.917845in}}%
\pgfpathlineto{\pgfqpoint{3.301009in}{0.936569in}}%
\pgfpathlineto{\pgfqpoint{3.350034in}{0.914277in}}%
\pgfpathlineto{\pgfqpoint{3.399060in}{0.915545in}}%
\pgfpathlineto{\pgfqpoint{3.448085in}{0.907092in}}%
\pgfpathlineto{\pgfqpoint{3.497110in}{0.922168in}}%
\pgfpathlineto{\pgfqpoint{3.546136in}{0.930575in}}%
\pgfpathlineto{\pgfqpoint{3.595161in}{0.916611in}}%
\pgfpathlineto{\pgfqpoint{3.644186in}{0.925467in}}%
\pgfpathlineto{\pgfqpoint{3.693211in}{0.926146in}}%
\pgfpathlineto{\pgfqpoint{3.742237in}{0.930661in}}%
\pgfpathlineto{\pgfqpoint{3.791262in}{0.916426in}}%
\pgfpathlineto{\pgfqpoint{3.840287in}{0.937020in}}%
\pgfpathlineto{\pgfqpoint{3.889313in}{0.933649in}}%
\pgfpathlineto{\pgfqpoint{3.938338in}{0.929309in}}%
\pgfpathlineto{\pgfqpoint{3.987363in}{0.937084in}}%
\pgfpathlineto{\pgfqpoint{4.036389in}{0.907291in}}%
\pgfusepath{stroke}%
\end{pgfscope}%
\begin{pgfscope}%
\pgfpathrectangle{\pgfqpoint{0.898769in}{0.566590in}}{\pgfqpoint{3.137619in}{1.799039in}}%
\pgfusepath{clip}%
\pgfsetbuttcap%
\pgfsetroundjoin%
\definecolor{currentfill}{rgb}{0.000000,0.000000,0.000000}%
\pgfsetfillcolor{currentfill}%
\pgfsetfillopacity{0.000000}%
\pgfsetlinewidth{1.003750pt}%
\definecolor{currentstroke}{rgb}{0.000000,0.447000,0.741000}%
\pgfsetstrokecolor{currentstroke}%
\pgfsetdash{}{0pt}%
\pgfsys@defobject{currentmarker}{\pgfqpoint{-0.041667in}{-0.041667in}}{\pgfqpoint{0.041667in}{0.041667in}}{%
\pgfpathmoveto{\pgfqpoint{0.000000in}{-0.041667in}}%
\pgfpathcurveto{\pgfqpoint{0.011050in}{-0.041667in}}{\pgfqpoint{0.021649in}{-0.037276in}}{\pgfqpoint{0.029463in}{-0.029463in}}%
\pgfpathcurveto{\pgfqpoint{0.037276in}{-0.021649in}}{\pgfqpoint{0.041667in}{-0.011050in}}{\pgfqpoint{0.041667in}{0.000000in}}%
\pgfpathcurveto{\pgfqpoint{0.041667in}{0.011050in}}{\pgfqpoint{0.037276in}{0.021649in}}{\pgfqpoint{0.029463in}{0.029463in}}%
\pgfpathcurveto{\pgfqpoint{0.021649in}{0.037276in}}{\pgfqpoint{0.011050in}{0.041667in}}{\pgfqpoint{0.000000in}{0.041667in}}%
\pgfpathcurveto{\pgfqpoint{-0.011050in}{0.041667in}}{\pgfqpoint{-0.021649in}{0.037276in}}{\pgfqpoint{-0.029463in}{0.029463in}}%
\pgfpathcurveto{\pgfqpoint{-0.037276in}{0.021649in}}{\pgfqpoint{-0.041667in}{0.011050in}}{\pgfqpoint{-0.041667in}{0.000000in}}%
\pgfpathcurveto{\pgfqpoint{-0.041667in}{-0.011050in}}{\pgfqpoint{-0.037276in}{-0.021649in}}{\pgfqpoint{-0.029463in}{-0.029463in}}%
\pgfpathcurveto{\pgfqpoint{-0.021649in}{-0.037276in}}{\pgfqpoint{-0.011050in}{-0.041667in}}{\pgfqpoint{0.000000in}{-0.041667in}}%
\pgfpathclose%
\pgfusepath{stroke,fill}%
}%
\begin{pgfscope}%
\pgfsys@transformshift{0.898769in}{1.851974in}%
\pgfsys@useobject{currentmarker}{}%
\end{pgfscope}%
\begin{pgfscope}%
\pgfsys@transformshift{1.094870in}{1.915371in}%
\pgfsys@useobject{currentmarker}{}%
\end{pgfscope}%
\begin{pgfscope}%
\pgfsys@transformshift{1.290972in}{1.970402in}%
\pgfsys@useobject{currentmarker}{}%
\end{pgfscope}%
\begin{pgfscope}%
\pgfsys@transformshift{1.487073in}{1.973695in}%
\pgfsys@useobject{currentmarker}{}%
\end{pgfscope}%
\begin{pgfscope}%
\pgfsys@transformshift{1.683174in}{1.921713in}%
\pgfsys@useobject{currentmarker}{}%
\end{pgfscope}%
\begin{pgfscope}%
\pgfsys@transformshift{1.879275in}{1.770743in}%
\pgfsys@useobject{currentmarker}{}%
\end{pgfscope}%
\begin{pgfscope}%
\pgfsys@transformshift{2.075376in}{0.937462in}%
\pgfsys@useobject{currentmarker}{}%
\end{pgfscope}%
\begin{pgfscope}%
\pgfsys@transformshift{2.271478in}{0.921204in}%
\pgfsys@useobject{currentmarker}{}%
\end{pgfscope}%
\begin{pgfscope}%
\pgfsys@transformshift{2.467579in}{0.949190in}%
\pgfsys@useobject{currentmarker}{}%
\end{pgfscope}%
\begin{pgfscope}%
\pgfsys@transformshift{2.663680in}{0.917058in}%
\pgfsys@useobject{currentmarker}{}%
\end{pgfscope}%
\begin{pgfscope}%
\pgfsys@transformshift{2.859781in}{0.916252in}%
\pgfsys@useobject{currentmarker}{}%
\end{pgfscope}%
\begin{pgfscope}%
\pgfsys@transformshift{3.055882in}{0.921433in}%
\pgfsys@useobject{currentmarker}{}%
\end{pgfscope}%
\begin{pgfscope}%
\pgfsys@transformshift{3.251984in}{0.917845in}%
\pgfsys@useobject{currentmarker}{}%
\end{pgfscope}%
\begin{pgfscope}%
\pgfsys@transformshift{3.448085in}{0.907092in}%
\pgfsys@useobject{currentmarker}{}%
\end{pgfscope}%
\begin{pgfscope}%
\pgfsys@transformshift{3.644186in}{0.925467in}%
\pgfsys@useobject{currentmarker}{}%
\end{pgfscope}%
\begin{pgfscope}%
\pgfsys@transformshift{3.840287in}{0.937020in}%
\pgfsys@useobject{currentmarker}{}%
\end{pgfscope}%
\begin{pgfscope}%
\pgfsys@transformshift{4.036389in}{0.907291in}%
\pgfsys@useobject{currentmarker}{}%
\end{pgfscope}%
\end{pgfscope}%
\begin{pgfscope}%
\pgfpathrectangle{\pgfqpoint{0.898769in}{0.566590in}}{\pgfqpoint{3.137619in}{1.799039in}}%
\pgfusepath{clip}%
\pgfsetbuttcap%
\pgfsetroundjoin%
\pgfsetlinewidth{1.505625pt}%
\definecolor{currentstroke}{rgb}{0.850000,0.324000,0.098000}%
\pgfsetstrokecolor{currentstroke}%
\pgfsetdash{{5.550000pt}{2.400000pt}}{0.000000pt}%
\pgfpathmoveto{\pgfqpoint{0.898769in}{1.991579in}}%
\pgfpathlineto{\pgfqpoint{0.947794in}{1.967928in}}%
\pgfpathlineto{\pgfqpoint{0.996820in}{1.947969in}}%
\pgfpathlineto{\pgfqpoint{1.045845in}{1.927357in}}%
\pgfpathlineto{\pgfqpoint{1.094870in}{1.841529in}}%
\pgfpathlineto{\pgfqpoint{1.143896in}{1.925794in}}%
\pgfpathlineto{\pgfqpoint{1.192921in}{1.858104in}}%
\pgfpathlineto{\pgfqpoint{1.241946in}{1.846264in}}%
\pgfpathlineto{\pgfqpoint{1.290972in}{1.769224in}}%
\pgfpathlineto{\pgfqpoint{1.339997in}{1.749258in}}%
\pgfpathlineto{\pgfqpoint{1.389022in}{1.751891in}}%
\pgfpathlineto{\pgfqpoint{1.438047in}{1.891268in}}%
\pgfpathlineto{\pgfqpoint{1.487073in}{1.369209in}}%
\pgfpathlineto{\pgfqpoint{1.536098in}{1.383873in}}%
\pgfpathlineto{\pgfqpoint{1.585123in}{1.356454in}}%
\pgfpathlineto{\pgfqpoint{1.634149in}{1.360867in}}%
\pgfpathlineto{\pgfqpoint{1.683174in}{1.317454in}}%
\pgfpathlineto{\pgfqpoint{1.732199in}{1.273021in}}%
\pgfpathlineto{\pgfqpoint{1.781225in}{1.275382in}}%
\pgfpathlineto{\pgfqpoint{1.830250in}{1.225606in}}%
\pgfpathlineto{\pgfqpoint{1.879275in}{1.183931in}}%
\pgfpathlineto{\pgfqpoint{1.928301in}{0.928593in}}%
\pgfpathlineto{\pgfqpoint{1.977326in}{1.093415in}}%
\pgfpathlineto{\pgfqpoint{2.026351in}{1.090775in}}%
\pgfpathlineto{\pgfqpoint{2.075376in}{0.918893in}}%
\pgfpathlineto{\pgfqpoint{2.124402in}{0.916609in}}%
\pgfpathlineto{\pgfqpoint{2.173427in}{0.920958in}}%
\pgfpathlineto{\pgfqpoint{2.222452in}{0.921954in}}%
\pgfpathlineto{\pgfqpoint{2.271478in}{0.923880in}}%
\pgfpathlineto{\pgfqpoint{2.320503in}{0.904401in}}%
\pgfpathlineto{\pgfqpoint{2.369528in}{0.919090in}}%
\pgfpathlineto{\pgfqpoint{2.418554in}{0.909789in}}%
\pgfpathlineto{\pgfqpoint{2.467579in}{0.923633in}}%
\pgfpathlineto{\pgfqpoint{2.516604in}{0.931280in}}%
\pgfpathlineto{\pgfqpoint{2.565629in}{0.916380in}}%
\pgfpathlineto{\pgfqpoint{2.614655in}{0.928831in}}%
\pgfpathlineto{\pgfqpoint{2.663680in}{0.927663in}}%
\pgfpathlineto{\pgfqpoint{2.712705in}{0.924202in}}%
\pgfpathlineto{\pgfqpoint{2.761731in}{0.920509in}}%
\pgfpathlineto{\pgfqpoint{2.810756in}{0.929531in}}%
\pgfpathlineto{\pgfqpoint{2.859781in}{0.901739in}}%
\pgfpathlineto{\pgfqpoint{2.908807in}{0.909913in}}%
\pgfpathlineto{\pgfqpoint{2.957832in}{0.936520in}}%
\pgfpathlineto{\pgfqpoint{3.006857in}{0.915025in}}%
\pgfpathlineto{\pgfqpoint{3.055882in}{0.928879in}}%
\pgfpathlineto{\pgfqpoint{3.104908in}{0.924802in}}%
\pgfpathlineto{\pgfqpoint{3.153933in}{0.930782in}}%
\pgfpathlineto{\pgfqpoint{3.202958in}{0.914110in}}%
\pgfpathlineto{\pgfqpoint{3.251984in}{0.918133in}}%
\pgfpathlineto{\pgfqpoint{3.301009in}{0.917945in}}%
\pgfpathlineto{\pgfqpoint{3.350034in}{0.916974in}}%
\pgfpathlineto{\pgfqpoint{3.399060in}{0.908784in}}%
\pgfpathlineto{\pgfqpoint{3.448085in}{0.910840in}}%
\pgfpathlineto{\pgfqpoint{3.497110in}{0.936887in}}%
\pgfpathlineto{\pgfqpoint{3.546136in}{0.906815in}}%
\pgfpathlineto{\pgfqpoint{3.595161in}{0.893486in}}%
\pgfpathlineto{\pgfqpoint{3.644186in}{0.927039in}}%
\pgfpathlineto{\pgfqpoint{3.693211in}{0.896389in}}%
\pgfpathlineto{\pgfqpoint{3.742237in}{0.924984in}}%
\pgfpathlineto{\pgfqpoint{3.791262in}{0.926620in}}%
\pgfpathlineto{\pgfqpoint{3.840287in}{0.908390in}}%
\pgfpathlineto{\pgfqpoint{3.889313in}{0.917316in}}%
\pgfpathlineto{\pgfqpoint{3.938338in}{0.926335in}}%
\pgfpathlineto{\pgfqpoint{3.987363in}{0.942579in}}%
\pgfpathlineto{\pgfqpoint{4.036389in}{0.918077in}}%
\pgfusepath{stroke}%
\end{pgfscope}%
\begin{pgfscope}%
\pgfpathrectangle{\pgfqpoint{0.898769in}{0.566590in}}{\pgfqpoint{3.137619in}{1.799039in}}%
\pgfusepath{clip}%
\pgfsetbuttcap%
\pgfsetroundjoin%
\definecolor{currentfill}{rgb}{0.850000,0.324000,0.098000}%
\pgfsetfillcolor{currentfill}%
\pgfsetlinewidth{1.003750pt}%
\definecolor{currentstroke}{rgb}{0.850000,0.324000,0.098000}%
\pgfsetstrokecolor{currentstroke}%
\pgfsetdash{}{0pt}%
\pgfsys@defobject{currentmarker}{\pgfqpoint{-0.041667in}{-0.041667in}}{\pgfqpoint{0.041667in}{0.041667in}}{%
\pgfpathmoveto{\pgfqpoint{-0.041667in}{0.000000in}}%
\pgfpathlineto{\pgfqpoint{0.041667in}{0.000000in}}%
\pgfpathmoveto{\pgfqpoint{0.000000in}{-0.041667in}}%
\pgfpathlineto{\pgfqpoint{0.000000in}{0.041667in}}%
\pgfusepath{stroke,fill}%
}%
\begin{pgfscope}%
\pgfsys@transformshift{0.898769in}{1.991579in}%
\pgfsys@useobject{currentmarker}{}%
\end{pgfscope}%
\begin{pgfscope}%
\pgfsys@transformshift{1.045845in}{1.927357in}%
\pgfsys@useobject{currentmarker}{}%
\end{pgfscope}%
\begin{pgfscope}%
\pgfsys@transformshift{1.192921in}{1.858104in}%
\pgfsys@useobject{currentmarker}{}%
\end{pgfscope}%
\begin{pgfscope}%
\pgfsys@transformshift{1.339997in}{1.749258in}%
\pgfsys@useobject{currentmarker}{}%
\end{pgfscope}%
\begin{pgfscope}%
\pgfsys@transformshift{1.487073in}{1.369209in}%
\pgfsys@useobject{currentmarker}{}%
\end{pgfscope}%
\begin{pgfscope}%
\pgfsys@transformshift{1.634149in}{1.360867in}%
\pgfsys@useobject{currentmarker}{}%
\end{pgfscope}%
\begin{pgfscope}%
\pgfsys@transformshift{1.781225in}{1.275382in}%
\pgfsys@useobject{currentmarker}{}%
\end{pgfscope}%
\begin{pgfscope}%
\pgfsys@transformshift{1.928301in}{0.928593in}%
\pgfsys@useobject{currentmarker}{}%
\end{pgfscope}%
\begin{pgfscope}%
\pgfsys@transformshift{2.075376in}{0.918893in}%
\pgfsys@useobject{currentmarker}{}%
\end{pgfscope}%
\begin{pgfscope}%
\pgfsys@transformshift{2.222452in}{0.921954in}%
\pgfsys@useobject{currentmarker}{}%
\end{pgfscope}%
\begin{pgfscope}%
\pgfsys@transformshift{2.369528in}{0.919090in}%
\pgfsys@useobject{currentmarker}{}%
\end{pgfscope}%
\begin{pgfscope}%
\pgfsys@transformshift{2.516604in}{0.931280in}%
\pgfsys@useobject{currentmarker}{}%
\end{pgfscope}%
\begin{pgfscope}%
\pgfsys@transformshift{2.663680in}{0.927663in}%
\pgfsys@useobject{currentmarker}{}%
\end{pgfscope}%
\begin{pgfscope}%
\pgfsys@transformshift{2.810756in}{0.929531in}%
\pgfsys@useobject{currentmarker}{}%
\end{pgfscope}%
\begin{pgfscope}%
\pgfsys@transformshift{2.957832in}{0.936520in}%
\pgfsys@useobject{currentmarker}{}%
\end{pgfscope}%
\begin{pgfscope}%
\pgfsys@transformshift{3.104908in}{0.924802in}%
\pgfsys@useobject{currentmarker}{}%
\end{pgfscope}%
\begin{pgfscope}%
\pgfsys@transformshift{3.251984in}{0.918133in}%
\pgfsys@useobject{currentmarker}{}%
\end{pgfscope}%
\begin{pgfscope}%
\pgfsys@transformshift{3.399060in}{0.908784in}%
\pgfsys@useobject{currentmarker}{}%
\end{pgfscope}%
\begin{pgfscope}%
\pgfsys@transformshift{3.546136in}{0.906815in}%
\pgfsys@useobject{currentmarker}{}%
\end{pgfscope}%
\begin{pgfscope}%
\pgfsys@transformshift{3.693211in}{0.896389in}%
\pgfsys@useobject{currentmarker}{}%
\end{pgfscope}%
\begin{pgfscope}%
\pgfsys@transformshift{3.840287in}{0.908390in}%
\pgfsys@useobject{currentmarker}{}%
\end{pgfscope}%
\begin{pgfscope}%
\pgfsys@transformshift{3.987363in}{0.942579in}%
\pgfsys@useobject{currentmarker}{}%
\end{pgfscope}%
\end{pgfscope}%
\begin{pgfscope}%
\pgfpathrectangle{\pgfqpoint{0.898769in}{0.566590in}}{\pgfqpoint{3.137619in}{1.799039in}}%
\pgfusepath{clip}%
\pgfsetbuttcap%
\pgfsetroundjoin%
\pgfsetlinewidth{1.505625pt}%
\definecolor{currentstroke}{rgb}{0.000000,0.500000,0.000000}%
\pgfsetstrokecolor{currentstroke}%
\pgfsetdash{{5.550000pt}{2.400000pt}}{0.000000pt}%
\pgfpathmoveto{\pgfqpoint{0.898769in}{2.093010in}}%
\pgfpathlineto{\pgfqpoint{0.947794in}{2.089308in}}%
\pgfpathlineto{\pgfqpoint{0.996820in}{2.085918in}}%
\pgfpathlineto{\pgfqpoint{1.045845in}{2.127462in}}%
\pgfpathlineto{\pgfqpoint{1.094870in}{2.116578in}}%
\pgfpathlineto{\pgfqpoint{1.143896in}{2.136180in}}%
\pgfpathlineto{\pgfqpoint{1.192921in}{2.145443in}}%
\pgfpathlineto{\pgfqpoint{1.241946in}{2.157385in}}%
\pgfpathlineto{\pgfqpoint{1.290972in}{2.156122in}}%
\pgfpathlineto{\pgfqpoint{1.339997in}{2.160042in}}%
\pgfpathlineto{\pgfqpoint{1.389022in}{2.168216in}}%
\pgfpathlineto{\pgfqpoint{1.438047in}{2.174010in}}%
\pgfpathlineto{\pgfqpoint{1.487073in}{2.178105in}}%
\pgfpathlineto{\pgfqpoint{1.536098in}{2.188650in}}%
\pgfpathlineto{\pgfqpoint{1.585123in}{2.155117in}}%
\pgfpathlineto{\pgfqpoint{1.634149in}{2.162185in}}%
\pgfpathlineto{\pgfqpoint{1.683174in}{2.105378in}}%
\pgfpathlineto{\pgfqpoint{1.732199in}{2.102071in}}%
\pgfpathlineto{\pgfqpoint{1.781225in}{2.102664in}}%
\pgfpathlineto{\pgfqpoint{1.830250in}{2.065267in}}%
\pgfpathlineto{\pgfqpoint{1.879275in}{2.009650in}}%
\pgfpathlineto{\pgfqpoint{1.928301in}{0.925038in}}%
\pgfpathlineto{\pgfqpoint{1.977326in}{1.932818in}}%
\pgfpathlineto{\pgfqpoint{2.026351in}{1.954476in}}%
\pgfpathlineto{\pgfqpoint{2.075376in}{0.913797in}}%
\pgfpathlineto{\pgfqpoint{2.124402in}{0.914247in}}%
\pgfpathlineto{\pgfqpoint{2.173427in}{0.916791in}}%
\pgfpathlineto{\pgfqpoint{2.222452in}{0.933868in}}%
\pgfpathlineto{\pgfqpoint{2.271478in}{0.932982in}}%
\pgfpathlineto{\pgfqpoint{2.320503in}{0.896033in}}%
\pgfpathlineto{\pgfqpoint{2.369528in}{0.923098in}}%
\pgfpathlineto{\pgfqpoint{2.418554in}{0.922067in}}%
\pgfpathlineto{\pgfqpoint{2.467579in}{0.914348in}}%
\pgfpathlineto{\pgfqpoint{2.516604in}{0.919510in}}%
\pgfpathlineto{\pgfqpoint{2.565629in}{0.914857in}}%
\pgfpathlineto{\pgfqpoint{2.614655in}{0.931168in}}%
\pgfpathlineto{\pgfqpoint{2.663680in}{0.912020in}}%
\pgfpathlineto{\pgfqpoint{2.712705in}{0.914429in}}%
\pgfpathlineto{\pgfqpoint{2.761731in}{0.922747in}}%
\pgfpathlineto{\pgfqpoint{2.810756in}{0.928069in}}%
\pgfpathlineto{\pgfqpoint{2.859781in}{0.910618in}}%
\pgfpathlineto{\pgfqpoint{2.908807in}{0.910184in}}%
\pgfpathlineto{\pgfqpoint{2.957832in}{0.920053in}}%
\pgfpathlineto{\pgfqpoint{3.006857in}{0.919363in}}%
\pgfpathlineto{\pgfqpoint{3.055882in}{0.930829in}}%
\pgfpathlineto{\pgfqpoint{3.104908in}{0.919564in}}%
\pgfpathlineto{\pgfqpoint{3.153933in}{0.917476in}}%
\pgfpathlineto{\pgfqpoint{3.202958in}{0.889280in}}%
\pgfpathlineto{\pgfqpoint{3.251984in}{0.915094in}}%
\pgfpathlineto{\pgfqpoint{3.301009in}{0.915453in}}%
\pgfpathlineto{\pgfqpoint{3.350034in}{0.912506in}}%
\pgfpathlineto{\pgfqpoint{3.399060in}{0.916324in}}%
\pgfpathlineto{\pgfqpoint{3.448085in}{0.922866in}}%
\pgfpathlineto{\pgfqpoint{3.497110in}{0.923456in}}%
\pgfpathlineto{\pgfqpoint{3.546136in}{0.928882in}}%
\pgfpathlineto{\pgfqpoint{3.595161in}{0.910249in}}%
\pgfpathlineto{\pgfqpoint{3.644186in}{0.918178in}}%
\pgfpathlineto{\pgfqpoint{3.693211in}{0.902372in}}%
\pgfpathlineto{\pgfqpoint{3.742237in}{0.911760in}}%
\pgfpathlineto{\pgfqpoint{3.791262in}{0.919851in}}%
\pgfpathlineto{\pgfqpoint{3.840287in}{0.921724in}}%
\pgfpathlineto{\pgfqpoint{3.889313in}{0.919278in}}%
\pgfpathlineto{\pgfqpoint{3.938338in}{0.920079in}}%
\pgfpathlineto{\pgfqpoint{3.987363in}{0.925209in}}%
\pgfpathlineto{\pgfqpoint{4.036389in}{0.915028in}}%
\pgfusepath{stroke}%
\end{pgfscope}%
\begin{pgfscope}%
\pgfpathrectangle{\pgfqpoint{0.898769in}{0.566590in}}{\pgfqpoint{3.137619in}{1.799039in}}%
\pgfusepath{clip}%
\pgfsetbuttcap%
\pgfsetmiterjoin%
\definecolor{currentfill}{rgb}{0.000000,0.000000,0.000000}%
\pgfsetfillcolor{currentfill}%
\pgfsetfillopacity{0.000000}%
\pgfsetlinewidth{1.003750pt}%
\definecolor{currentstroke}{rgb}{0.000000,0.500000,0.000000}%
\pgfsetstrokecolor{currentstroke}%
\pgfsetdash{}{0pt}%
\pgfsys@defobject{currentmarker}{\pgfqpoint{-0.041667in}{-0.041667in}}{\pgfqpoint{0.041667in}{0.041667in}}{%
\pgfpathmoveto{\pgfqpoint{-0.041667in}{-0.041667in}}%
\pgfpathlineto{\pgfqpoint{0.041667in}{-0.041667in}}%
\pgfpathlineto{\pgfqpoint{0.041667in}{0.041667in}}%
\pgfpathlineto{\pgfqpoint{-0.041667in}{0.041667in}}%
\pgfpathclose%
\pgfusepath{stroke,fill}%
}%
\begin{pgfscope}%
\pgfsys@transformshift{0.898769in}{2.093010in}%
\pgfsys@useobject{currentmarker}{}%
\end{pgfscope}%
\begin{pgfscope}%
\pgfsys@transformshift{1.143896in}{2.136180in}%
\pgfsys@useobject{currentmarker}{}%
\end{pgfscope}%
\begin{pgfscope}%
\pgfsys@transformshift{1.389022in}{2.168216in}%
\pgfsys@useobject{currentmarker}{}%
\end{pgfscope}%
\begin{pgfscope}%
\pgfsys@transformshift{1.634149in}{2.162185in}%
\pgfsys@useobject{currentmarker}{}%
\end{pgfscope}%
\begin{pgfscope}%
\pgfsys@transformshift{1.879275in}{2.009650in}%
\pgfsys@useobject{currentmarker}{}%
\end{pgfscope}%
\begin{pgfscope}%
\pgfsys@transformshift{2.124402in}{0.914247in}%
\pgfsys@useobject{currentmarker}{}%
\end{pgfscope}%
\begin{pgfscope}%
\pgfsys@transformshift{2.369528in}{0.923098in}%
\pgfsys@useobject{currentmarker}{}%
\end{pgfscope}%
\begin{pgfscope}%
\pgfsys@transformshift{2.614655in}{0.931168in}%
\pgfsys@useobject{currentmarker}{}%
\end{pgfscope}%
\begin{pgfscope}%
\pgfsys@transformshift{2.859781in}{0.910618in}%
\pgfsys@useobject{currentmarker}{}%
\end{pgfscope}%
\begin{pgfscope}%
\pgfsys@transformshift{3.104908in}{0.919564in}%
\pgfsys@useobject{currentmarker}{}%
\end{pgfscope}%
\begin{pgfscope}%
\pgfsys@transformshift{3.350034in}{0.912506in}%
\pgfsys@useobject{currentmarker}{}%
\end{pgfscope}%
\begin{pgfscope}%
\pgfsys@transformshift{3.595161in}{0.910249in}%
\pgfsys@useobject{currentmarker}{}%
\end{pgfscope}%
\begin{pgfscope}%
\pgfsys@transformshift{3.840287in}{0.921724in}%
\pgfsys@useobject{currentmarker}{}%
\end{pgfscope}%
\end{pgfscope}%
\begin{pgfscope}%
\pgfpathrectangle{\pgfqpoint{0.898769in}{0.566590in}}{\pgfqpoint{3.137619in}{1.799039in}}%
\pgfusepath{clip}%
\pgfsetbuttcap%
\pgfsetroundjoin%
\pgfsetlinewidth{1.505625pt}%
\definecolor{currentstroke}{rgb}{0.494000,0.184000,0.556000}%
\pgfsetstrokecolor{currentstroke}%
\pgfsetdash{{5.550000pt}{2.400000pt}}{0.000000pt}%
\pgfpathmoveto{\pgfqpoint{0.898769in}{2.239028in}}%
\pgfpathlineto{\pgfqpoint{0.947794in}{2.224827in}}%
\pgfpathlineto{\pgfqpoint{0.996820in}{2.227392in}}%
\pgfpathlineto{\pgfqpoint{1.045845in}{2.208001in}}%
\pgfpathlineto{\pgfqpoint{1.094870in}{2.238074in}}%
\pgfpathlineto{\pgfqpoint{1.143896in}{2.214646in}}%
\pgfpathlineto{\pgfqpoint{1.192921in}{2.214287in}}%
\pgfpathlineto{\pgfqpoint{1.241946in}{2.183640in}}%
\pgfpathlineto{\pgfqpoint{1.290972in}{2.226897in}}%
\pgfpathlineto{\pgfqpoint{1.339997in}{2.275131in}}%
\pgfpathlineto{\pgfqpoint{1.389022in}{2.235720in}}%
\pgfpathlineto{\pgfqpoint{1.438047in}{2.283854in}}%
\pgfpathlineto{\pgfqpoint{1.487073in}{2.268572in}}%
\pgfpathlineto{\pgfqpoint{1.536098in}{2.250952in}}%
\pgfpathlineto{\pgfqpoint{1.585123in}{2.191925in}}%
\pgfpathlineto{\pgfqpoint{1.634149in}{2.155218in}}%
\pgfpathlineto{\pgfqpoint{1.683174in}{2.196651in}}%
\pgfpathlineto{\pgfqpoint{1.732199in}{2.174779in}}%
\pgfpathlineto{\pgfqpoint{1.781225in}{2.064987in}}%
\pgfpathlineto{\pgfqpoint{1.830250in}{0.920348in}}%
\pgfpathlineto{\pgfqpoint{1.879275in}{1.559014in}}%
\pgfpathlineto{\pgfqpoint{1.928301in}{0.935830in}}%
\pgfpathlineto{\pgfqpoint{1.977326in}{0.934621in}}%
\pgfpathlineto{\pgfqpoint{2.026351in}{0.940753in}}%
\pgfpathlineto{\pgfqpoint{2.075376in}{0.914297in}}%
\pgfpathlineto{\pgfqpoint{2.124402in}{0.929473in}}%
\pgfpathlineto{\pgfqpoint{2.173427in}{0.929617in}}%
\pgfpathlineto{\pgfqpoint{2.222452in}{0.927303in}}%
\pgfpathlineto{\pgfqpoint{2.271478in}{0.927409in}}%
\pgfpathlineto{\pgfqpoint{2.320503in}{0.918075in}}%
\pgfpathlineto{\pgfqpoint{2.369528in}{0.921257in}}%
\pgfpathlineto{\pgfqpoint{2.418554in}{0.907890in}}%
\pgfpathlineto{\pgfqpoint{2.467579in}{0.913619in}}%
\pgfpathlineto{\pgfqpoint{2.516604in}{0.928063in}}%
\pgfpathlineto{\pgfqpoint{2.565629in}{0.917342in}}%
\pgfpathlineto{\pgfqpoint{2.614655in}{0.923858in}}%
\pgfpathlineto{\pgfqpoint{2.663680in}{0.934949in}}%
\pgfpathlineto{\pgfqpoint{2.712705in}{0.910101in}}%
\pgfpathlineto{\pgfqpoint{2.761731in}{0.924273in}}%
\pgfpathlineto{\pgfqpoint{2.810756in}{0.923671in}}%
\pgfpathlineto{\pgfqpoint{2.859781in}{0.930898in}}%
\pgfpathlineto{\pgfqpoint{2.908807in}{0.925812in}}%
\pgfpathlineto{\pgfqpoint{2.957832in}{0.925822in}}%
\pgfpathlineto{\pgfqpoint{3.006857in}{0.936258in}}%
\pgfpathlineto{\pgfqpoint{3.055882in}{0.912334in}}%
\pgfpathlineto{\pgfqpoint{3.104908in}{0.918209in}}%
\pgfpathlineto{\pgfqpoint{3.153933in}{0.929951in}}%
\pgfpathlineto{\pgfqpoint{3.202958in}{0.926354in}}%
\pgfpathlineto{\pgfqpoint{3.251984in}{0.914864in}}%
\pgfpathlineto{\pgfqpoint{3.301009in}{0.926877in}}%
\pgfpathlineto{\pgfqpoint{3.350034in}{0.925661in}}%
\pgfpathlineto{\pgfqpoint{3.399060in}{0.915467in}}%
\pgfpathlineto{\pgfqpoint{3.448085in}{0.912012in}}%
\pgfpathlineto{\pgfqpoint{3.497110in}{0.912026in}}%
\pgfpathlineto{\pgfqpoint{3.546136in}{0.924477in}}%
\pgfpathlineto{\pgfqpoint{3.595161in}{0.901610in}}%
\pgfpathlineto{\pgfqpoint{3.644186in}{0.921662in}}%
\pgfpathlineto{\pgfqpoint{3.693211in}{0.902929in}}%
\pgfpathlineto{\pgfqpoint{3.742237in}{0.925242in}}%
\pgfpathlineto{\pgfqpoint{3.791262in}{0.924225in}}%
\pgfpathlineto{\pgfqpoint{3.840287in}{0.908328in}}%
\pgfpathlineto{\pgfqpoint{3.889313in}{0.911192in}}%
\pgfpathlineto{\pgfqpoint{3.938338in}{0.905296in}}%
\pgfpathlineto{\pgfqpoint{3.987363in}{0.923630in}}%
\pgfpathlineto{\pgfqpoint{4.036389in}{0.923358in}}%
\pgfusepath{stroke}%
\end{pgfscope}%
\begin{pgfscope}%
\pgfpathrectangle{\pgfqpoint{0.898769in}{0.566590in}}{\pgfqpoint{3.137619in}{1.799039in}}%
\pgfusepath{clip}%
\pgfsetbuttcap%
\pgfsetroundjoin%
\definecolor{currentfill}{rgb}{0.494000,0.184000,0.556000}%
\pgfsetfillcolor{currentfill}%
\pgfsetlinewidth{1.003750pt}%
\definecolor{currentstroke}{rgb}{0.494000,0.184000,0.556000}%
\pgfsetstrokecolor{currentstroke}%
\pgfsetdash{}{0pt}%
\pgfsys@defobject{currentmarker}{\pgfqpoint{-0.041667in}{-0.041667in}}{\pgfqpoint{0.041667in}{0.041667in}}{%
\pgfpathmoveto{\pgfqpoint{-0.041667in}{-0.041667in}}%
\pgfpathlineto{\pgfqpoint{0.041667in}{0.041667in}}%
\pgfpathmoveto{\pgfqpoint{-0.041667in}{0.041667in}}%
\pgfpathlineto{\pgfqpoint{0.041667in}{-0.041667in}}%
\pgfusepath{stroke,fill}%
}%
\begin{pgfscope}%
\pgfsys@transformshift{0.898769in}{2.239028in}%
\pgfsys@useobject{currentmarker}{}%
\end{pgfscope}%
\begin{pgfscope}%
\pgfsys@transformshift{1.094870in}{2.238074in}%
\pgfsys@useobject{currentmarker}{}%
\end{pgfscope}%
\begin{pgfscope}%
\pgfsys@transformshift{1.290972in}{2.226897in}%
\pgfsys@useobject{currentmarker}{}%
\end{pgfscope}%
\begin{pgfscope}%
\pgfsys@transformshift{1.487073in}{2.268572in}%
\pgfsys@useobject{currentmarker}{}%
\end{pgfscope}%
\begin{pgfscope}%
\pgfsys@transformshift{1.683174in}{2.196651in}%
\pgfsys@useobject{currentmarker}{}%
\end{pgfscope}%
\begin{pgfscope}%
\pgfsys@transformshift{1.879275in}{1.559014in}%
\pgfsys@useobject{currentmarker}{}%
\end{pgfscope}%
\begin{pgfscope}%
\pgfsys@transformshift{2.075376in}{0.914297in}%
\pgfsys@useobject{currentmarker}{}%
\end{pgfscope}%
\begin{pgfscope}%
\pgfsys@transformshift{2.271478in}{0.927409in}%
\pgfsys@useobject{currentmarker}{}%
\end{pgfscope}%
\begin{pgfscope}%
\pgfsys@transformshift{2.467579in}{0.913619in}%
\pgfsys@useobject{currentmarker}{}%
\end{pgfscope}%
\begin{pgfscope}%
\pgfsys@transformshift{2.663680in}{0.934949in}%
\pgfsys@useobject{currentmarker}{}%
\end{pgfscope}%
\begin{pgfscope}%
\pgfsys@transformshift{2.859781in}{0.930898in}%
\pgfsys@useobject{currentmarker}{}%
\end{pgfscope}%
\begin{pgfscope}%
\pgfsys@transformshift{3.055882in}{0.912334in}%
\pgfsys@useobject{currentmarker}{}%
\end{pgfscope}%
\begin{pgfscope}%
\pgfsys@transformshift{3.251984in}{0.914864in}%
\pgfsys@useobject{currentmarker}{}%
\end{pgfscope}%
\begin{pgfscope}%
\pgfsys@transformshift{3.448085in}{0.912012in}%
\pgfsys@useobject{currentmarker}{}%
\end{pgfscope}%
\begin{pgfscope}%
\pgfsys@transformshift{3.644186in}{0.921662in}%
\pgfsys@useobject{currentmarker}{}%
\end{pgfscope}%
\begin{pgfscope}%
\pgfsys@transformshift{3.840287in}{0.908328in}%
\pgfsys@useobject{currentmarker}{}%
\end{pgfscope}%
\begin{pgfscope}%
\pgfsys@transformshift{4.036389in}{0.923358in}%
\pgfsys@useobject{currentmarker}{}%
\end{pgfscope}%
\end{pgfscope}%
\begin{pgfscope}%
\pgfpathrectangle{\pgfqpoint{0.898769in}{0.566590in}}{\pgfqpoint{3.137619in}{1.799039in}}%
\pgfusepath{clip}%
\pgfsetrectcap%
\pgfsetroundjoin%
\pgfsetlinewidth{1.505625pt}%
\definecolor{currentstroke}{rgb}{0.000000,0.447000,0.741000}%
\pgfsetstrokecolor{currentstroke}%
\pgfsetdash{}{0pt}%
\pgfpathmoveto{\pgfqpoint{0.898769in}{1.545418in}}%
\pgfpathlineto{\pgfqpoint{0.947794in}{1.537426in}}%
\pgfpathlineto{\pgfqpoint{0.996820in}{1.558281in}}%
\pgfpathlineto{\pgfqpoint{1.045845in}{1.565382in}}%
\pgfpathlineto{\pgfqpoint{1.094870in}{1.556253in}}%
\pgfpathlineto{\pgfqpoint{1.143896in}{1.577581in}}%
\pgfpathlineto{\pgfqpoint{1.192921in}{1.603829in}}%
\pgfpathlineto{\pgfqpoint{1.241946in}{1.597008in}}%
\pgfpathlineto{\pgfqpoint{1.290972in}{1.628868in}}%
\pgfpathlineto{\pgfqpoint{1.339997in}{1.581775in}}%
\pgfpathlineto{\pgfqpoint{1.389022in}{1.591060in}}%
\pgfpathlineto{\pgfqpoint{1.438047in}{1.567056in}}%
\pgfpathlineto{\pgfqpoint{1.487073in}{1.575711in}}%
\pgfpathlineto{\pgfqpoint{1.536098in}{1.539442in}}%
\pgfpathlineto{\pgfqpoint{1.585123in}{1.569061in}}%
\pgfpathlineto{\pgfqpoint{1.634149in}{1.567577in}}%
\pgfpathlineto{\pgfqpoint{1.683174in}{1.473184in}}%
\pgfpathlineto{\pgfqpoint{1.732199in}{1.384463in}}%
\pgfpathlineto{\pgfqpoint{1.781225in}{1.400490in}}%
\pgfpathlineto{\pgfqpoint{1.830250in}{1.084057in}}%
\pgfpathlineto{\pgfqpoint{1.879275in}{1.073547in}}%
\pgfpathlineto{\pgfqpoint{1.928301in}{1.053887in}}%
\pgfpathlineto{\pgfqpoint{1.977326in}{1.046453in}}%
\pgfpathlineto{\pgfqpoint{2.026351in}{1.067798in}}%
\pgfpathlineto{\pgfqpoint{2.075376in}{1.057277in}}%
\pgfpathlineto{\pgfqpoint{2.124402in}{1.049629in}}%
\pgfpathlineto{\pgfqpoint{2.173427in}{1.029088in}}%
\pgfpathlineto{\pgfqpoint{2.222452in}{1.027904in}}%
\pgfpathlineto{\pgfqpoint{2.271478in}{1.036643in}}%
\pgfpathlineto{\pgfqpoint{2.320503in}{1.028020in}}%
\pgfpathlineto{\pgfqpoint{2.369528in}{1.011892in}}%
\pgfpathlineto{\pgfqpoint{2.418554in}{0.983211in}}%
\pgfpathlineto{\pgfqpoint{2.467579in}{1.014935in}}%
\pgfpathlineto{\pgfqpoint{2.516604in}{0.975142in}}%
\pgfpathlineto{\pgfqpoint{2.565629in}{0.955888in}}%
\pgfpathlineto{\pgfqpoint{2.614655in}{0.967676in}}%
\pgfpathlineto{\pgfqpoint{2.663680in}{0.947819in}}%
\pgfpathlineto{\pgfqpoint{2.712705in}{0.956099in}}%
\pgfpathlineto{\pgfqpoint{2.761731in}{0.958180in}}%
\pgfpathlineto{\pgfqpoint{2.810756in}{0.947144in}}%
\pgfpathlineto{\pgfqpoint{2.859781in}{0.941586in}}%
\pgfpathlineto{\pgfqpoint{2.908807in}{0.929206in}}%
\pgfpathlineto{\pgfqpoint{2.957832in}{0.944738in}}%
\pgfpathlineto{\pgfqpoint{3.006857in}{0.915535in}}%
\pgfpathlineto{\pgfqpoint{3.055882in}{0.928775in}}%
\pgfpathlineto{\pgfqpoint{3.104908in}{0.939501in}}%
\pgfpathlineto{\pgfqpoint{3.153933in}{0.937605in}}%
\pgfpathlineto{\pgfqpoint{3.202958in}{0.932534in}}%
\pgfpathlineto{\pgfqpoint{3.251984in}{0.927341in}}%
\pgfpathlineto{\pgfqpoint{3.301009in}{0.946951in}}%
\pgfpathlineto{\pgfqpoint{3.350034in}{0.919220in}}%
\pgfpathlineto{\pgfqpoint{3.399060in}{0.918727in}}%
\pgfpathlineto{\pgfqpoint{3.448085in}{0.921019in}}%
\pgfpathlineto{\pgfqpoint{3.497110in}{0.929278in}}%
\pgfpathlineto{\pgfqpoint{3.546136in}{0.932620in}}%
\pgfpathlineto{\pgfqpoint{3.595161in}{0.919493in}}%
\pgfpathlineto{\pgfqpoint{3.644186in}{0.931769in}}%
\pgfpathlineto{\pgfqpoint{3.693211in}{0.936129in}}%
\pgfpathlineto{\pgfqpoint{3.742237in}{0.936805in}}%
\pgfpathlineto{\pgfqpoint{3.791262in}{0.917751in}}%
\pgfpathlineto{\pgfqpoint{3.840287in}{0.945174in}}%
\pgfpathlineto{\pgfqpoint{3.889313in}{0.933798in}}%
\pgfpathlineto{\pgfqpoint{3.938338in}{0.932775in}}%
\pgfpathlineto{\pgfqpoint{3.987363in}{0.937890in}}%
\pgfpathlineto{\pgfqpoint{4.036389in}{0.910599in}}%
\pgfusepath{stroke}%
\end{pgfscope}%
\begin{pgfscope}%
\pgfpathrectangle{\pgfqpoint{0.898769in}{0.566590in}}{\pgfqpoint{3.137619in}{1.799039in}}%
\pgfusepath{clip}%
\pgfsetbuttcap%
\pgfsetroundjoin%
\definecolor{currentfill}{rgb}{0.000000,0.000000,0.000000}%
\pgfsetfillcolor{currentfill}%
\pgfsetfillopacity{0.000000}%
\pgfsetlinewidth{1.003750pt}%
\definecolor{currentstroke}{rgb}{0.000000,0.447000,0.741000}%
\pgfsetstrokecolor{currentstroke}%
\pgfsetdash{}{0pt}%
\pgfsys@defobject{currentmarker}{\pgfqpoint{-0.041667in}{-0.041667in}}{\pgfqpoint{0.041667in}{0.041667in}}{%
\pgfpathmoveto{\pgfqpoint{0.000000in}{-0.041667in}}%
\pgfpathcurveto{\pgfqpoint{0.011050in}{-0.041667in}}{\pgfqpoint{0.021649in}{-0.037276in}}{\pgfqpoint{0.029463in}{-0.029463in}}%
\pgfpathcurveto{\pgfqpoint{0.037276in}{-0.021649in}}{\pgfqpoint{0.041667in}{-0.011050in}}{\pgfqpoint{0.041667in}{0.000000in}}%
\pgfpathcurveto{\pgfqpoint{0.041667in}{0.011050in}}{\pgfqpoint{0.037276in}{0.021649in}}{\pgfqpoint{0.029463in}{0.029463in}}%
\pgfpathcurveto{\pgfqpoint{0.021649in}{0.037276in}}{\pgfqpoint{0.011050in}{0.041667in}}{\pgfqpoint{0.000000in}{0.041667in}}%
\pgfpathcurveto{\pgfqpoint{-0.011050in}{0.041667in}}{\pgfqpoint{-0.021649in}{0.037276in}}{\pgfqpoint{-0.029463in}{0.029463in}}%
\pgfpathcurveto{\pgfqpoint{-0.037276in}{0.021649in}}{\pgfqpoint{-0.041667in}{0.011050in}}{\pgfqpoint{-0.041667in}{0.000000in}}%
\pgfpathcurveto{\pgfqpoint{-0.041667in}{-0.011050in}}{\pgfqpoint{-0.037276in}{-0.021649in}}{\pgfqpoint{-0.029463in}{-0.029463in}}%
\pgfpathcurveto{\pgfqpoint{-0.021649in}{-0.037276in}}{\pgfqpoint{-0.011050in}{-0.041667in}}{\pgfqpoint{0.000000in}{-0.041667in}}%
\pgfpathclose%
\pgfusepath{stroke,fill}%
}%
\begin{pgfscope}%
\pgfsys@transformshift{0.898769in}{1.545418in}%
\pgfsys@useobject{currentmarker}{}%
\end{pgfscope}%
\begin{pgfscope}%
\pgfsys@transformshift{1.094870in}{1.556253in}%
\pgfsys@useobject{currentmarker}{}%
\end{pgfscope}%
\begin{pgfscope}%
\pgfsys@transformshift{1.290972in}{1.628868in}%
\pgfsys@useobject{currentmarker}{}%
\end{pgfscope}%
\begin{pgfscope}%
\pgfsys@transformshift{1.487073in}{1.575711in}%
\pgfsys@useobject{currentmarker}{}%
\end{pgfscope}%
\begin{pgfscope}%
\pgfsys@transformshift{1.683174in}{1.473184in}%
\pgfsys@useobject{currentmarker}{}%
\end{pgfscope}%
\begin{pgfscope}%
\pgfsys@transformshift{1.879275in}{1.073547in}%
\pgfsys@useobject{currentmarker}{}%
\end{pgfscope}%
\begin{pgfscope}%
\pgfsys@transformshift{2.075376in}{1.057277in}%
\pgfsys@useobject{currentmarker}{}%
\end{pgfscope}%
\begin{pgfscope}%
\pgfsys@transformshift{2.271478in}{1.036643in}%
\pgfsys@useobject{currentmarker}{}%
\end{pgfscope}%
\begin{pgfscope}%
\pgfsys@transformshift{2.467579in}{1.014935in}%
\pgfsys@useobject{currentmarker}{}%
\end{pgfscope}%
\begin{pgfscope}%
\pgfsys@transformshift{2.663680in}{0.947819in}%
\pgfsys@useobject{currentmarker}{}%
\end{pgfscope}%
\begin{pgfscope}%
\pgfsys@transformshift{2.859781in}{0.941586in}%
\pgfsys@useobject{currentmarker}{}%
\end{pgfscope}%
\begin{pgfscope}%
\pgfsys@transformshift{3.055882in}{0.928775in}%
\pgfsys@useobject{currentmarker}{}%
\end{pgfscope}%
\begin{pgfscope}%
\pgfsys@transformshift{3.251984in}{0.927341in}%
\pgfsys@useobject{currentmarker}{}%
\end{pgfscope}%
\begin{pgfscope}%
\pgfsys@transformshift{3.448085in}{0.921019in}%
\pgfsys@useobject{currentmarker}{}%
\end{pgfscope}%
\begin{pgfscope}%
\pgfsys@transformshift{3.644186in}{0.931769in}%
\pgfsys@useobject{currentmarker}{}%
\end{pgfscope}%
\begin{pgfscope}%
\pgfsys@transformshift{3.840287in}{0.945174in}%
\pgfsys@useobject{currentmarker}{}%
\end{pgfscope}%
\begin{pgfscope}%
\pgfsys@transformshift{4.036389in}{0.910599in}%
\pgfsys@useobject{currentmarker}{}%
\end{pgfscope}%
\end{pgfscope}%
\begin{pgfscope}%
\pgfpathrectangle{\pgfqpoint{0.898769in}{0.566590in}}{\pgfqpoint{3.137619in}{1.799039in}}%
\pgfusepath{clip}%
\pgfsetrectcap%
\pgfsetroundjoin%
\pgfsetlinewidth{1.505625pt}%
\definecolor{currentstroke}{rgb}{0.850000,0.324000,0.098000}%
\pgfsetstrokecolor{currentstroke}%
\pgfsetdash{}{0pt}%
\pgfpathmoveto{\pgfqpoint{0.898769in}{1.268977in}}%
\pgfpathlineto{\pgfqpoint{0.947794in}{1.232769in}}%
\pgfpathlineto{\pgfqpoint{0.996820in}{1.211370in}}%
\pgfpathlineto{\pgfqpoint{1.045845in}{1.183775in}}%
\pgfpathlineto{\pgfqpoint{1.094870in}{1.168565in}}%
\pgfpathlineto{\pgfqpoint{1.143896in}{1.158966in}}%
\pgfpathlineto{\pgfqpoint{1.192921in}{1.096561in}}%
\pgfpathlineto{\pgfqpoint{1.241946in}{1.053464in}}%
\pgfpathlineto{\pgfqpoint{1.290972in}{0.945957in}}%
\pgfpathlineto{\pgfqpoint{1.339997in}{0.959162in}}%
\pgfpathlineto{\pgfqpoint{1.389022in}{0.883291in}}%
\pgfpathlineto{\pgfqpoint{1.438047in}{0.936434in}}%
\pgfpathlineto{\pgfqpoint{1.487073in}{0.879500in}}%
\pgfpathlineto{\pgfqpoint{1.536098in}{0.890189in}}%
\pgfpathlineto{\pgfqpoint{1.585123in}{0.855251in}}%
\pgfpathlineto{\pgfqpoint{1.634149in}{0.821896in}}%
\pgfpathlineto{\pgfqpoint{1.683174in}{0.823536in}}%
\pgfpathlineto{\pgfqpoint{1.732199in}{0.849850in}}%
\pgfpathlineto{\pgfqpoint{1.781225in}{0.812231in}}%
\pgfpathlineto{\pgfqpoint{1.830250in}{0.792058in}}%
\pgfpathlineto{\pgfqpoint{1.879275in}{0.805258in}}%
\pgfpathlineto{\pgfqpoint{1.928301in}{0.793739in}}%
\pgfpathlineto{\pgfqpoint{1.977326in}{0.794644in}}%
\pgfpathlineto{\pgfqpoint{2.026351in}{0.794134in}}%
\pgfpathlineto{\pgfqpoint{2.075376in}{0.780429in}}%
\pgfpathlineto{\pgfqpoint{2.124402in}{0.768685in}}%
\pgfpathlineto{\pgfqpoint{2.173427in}{0.777651in}}%
\pgfpathlineto{\pgfqpoint{2.222452in}{0.785510in}}%
\pgfpathlineto{\pgfqpoint{2.271478in}{0.774750in}}%
\pgfpathlineto{\pgfqpoint{2.320503in}{0.749121in}}%
\pgfpathlineto{\pgfqpoint{2.369528in}{0.773067in}}%
\pgfpathlineto{\pgfqpoint{2.418554in}{0.773124in}}%
\pgfpathlineto{\pgfqpoint{2.467579in}{0.769341in}}%
\pgfpathlineto{\pgfqpoint{2.516604in}{0.783303in}}%
\pgfpathlineto{\pgfqpoint{2.565629in}{0.766919in}}%
\pgfpathlineto{\pgfqpoint{2.614655in}{0.776734in}}%
\pgfpathlineto{\pgfqpoint{2.663680in}{0.768833in}}%
\pgfpathlineto{\pgfqpoint{2.712705in}{0.772955in}}%
\pgfpathlineto{\pgfqpoint{2.761731in}{0.776867in}}%
\pgfpathlineto{\pgfqpoint{2.810756in}{0.776022in}}%
\pgfpathlineto{\pgfqpoint{2.859781in}{0.765112in}}%
\pgfpathlineto{\pgfqpoint{2.908807in}{0.755958in}}%
\pgfpathlineto{\pgfqpoint{2.957832in}{0.780528in}}%
\pgfpathlineto{\pgfqpoint{3.006857in}{0.754651in}}%
\pgfpathlineto{\pgfqpoint{3.055882in}{0.780012in}}%
\pgfpathlineto{\pgfqpoint{3.104908in}{0.768598in}}%
\pgfpathlineto{\pgfqpoint{3.153933in}{0.773759in}}%
\pgfpathlineto{\pgfqpoint{3.202958in}{0.765852in}}%
\pgfpathlineto{\pgfqpoint{3.251984in}{0.763035in}}%
\pgfpathlineto{\pgfqpoint{3.301009in}{0.765535in}}%
\pgfpathlineto{\pgfqpoint{3.350034in}{0.763788in}}%
\pgfpathlineto{\pgfqpoint{3.399060in}{0.764455in}}%
\pgfpathlineto{\pgfqpoint{3.448085in}{0.758237in}}%
\pgfpathlineto{\pgfqpoint{3.497110in}{0.789042in}}%
\pgfpathlineto{\pgfqpoint{3.546136in}{0.760023in}}%
\pgfpathlineto{\pgfqpoint{3.595161in}{0.744068in}}%
\pgfpathlineto{\pgfqpoint{3.644186in}{0.769479in}}%
\pgfpathlineto{\pgfqpoint{3.693211in}{0.754383in}}%
\pgfpathlineto{\pgfqpoint{3.742237in}{0.779797in}}%
\pgfpathlineto{\pgfqpoint{3.791262in}{0.781419in}}%
\pgfpathlineto{\pgfqpoint{3.840287in}{0.750054in}}%
\pgfpathlineto{\pgfqpoint{3.889313in}{0.769733in}}%
\pgfpathlineto{\pgfqpoint{3.938338in}{0.774626in}}%
\pgfpathlineto{\pgfqpoint{3.987363in}{0.788390in}}%
\pgfpathlineto{\pgfqpoint{4.036389in}{0.768932in}}%
\pgfusepath{stroke}%
\end{pgfscope}%
\begin{pgfscope}%
\pgfpathrectangle{\pgfqpoint{0.898769in}{0.566590in}}{\pgfqpoint{3.137619in}{1.799039in}}%
\pgfusepath{clip}%
\pgfsetbuttcap%
\pgfsetroundjoin%
\definecolor{currentfill}{rgb}{0.850000,0.324000,0.098000}%
\pgfsetfillcolor{currentfill}%
\pgfsetlinewidth{1.003750pt}%
\definecolor{currentstroke}{rgb}{0.850000,0.324000,0.098000}%
\pgfsetstrokecolor{currentstroke}%
\pgfsetdash{}{0pt}%
\pgfsys@defobject{currentmarker}{\pgfqpoint{-0.041667in}{-0.041667in}}{\pgfqpoint{0.041667in}{0.041667in}}{%
\pgfpathmoveto{\pgfqpoint{-0.041667in}{0.000000in}}%
\pgfpathlineto{\pgfqpoint{0.041667in}{0.000000in}}%
\pgfpathmoveto{\pgfqpoint{0.000000in}{-0.041667in}}%
\pgfpathlineto{\pgfqpoint{0.000000in}{0.041667in}}%
\pgfusepath{stroke,fill}%
}%
\begin{pgfscope}%
\pgfsys@transformshift{0.898769in}{1.268977in}%
\pgfsys@useobject{currentmarker}{}%
\end{pgfscope}%
\begin{pgfscope}%
\pgfsys@transformshift{1.045845in}{1.183775in}%
\pgfsys@useobject{currentmarker}{}%
\end{pgfscope}%
\begin{pgfscope}%
\pgfsys@transformshift{1.192921in}{1.096561in}%
\pgfsys@useobject{currentmarker}{}%
\end{pgfscope}%
\begin{pgfscope}%
\pgfsys@transformshift{1.339997in}{0.959162in}%
\pgfsys@useobject{currentmarker}{}%
\end{pgfscope}%
\begin{pgfscope}%
\pgfsys@transformshift{1.487073in}{0.879500in}%
\pgfsys@useobject{currentmarker}{}%
\end{pgfscope}%
\begin{pgfscope}%
\pgfsys@transformshift{1.634149in}{0.821896in}%
\pgfsys@useobject{currentmarker}{}%
\end{pgfscope}%
\begin{pgfscope}%
\pgfsys@transformshift{1.781225in}{0.812231in}%
\pgfsys@useobject{currentmarker}{}%
\end{pgfscope}%
\begin{pgfscope}%
\pgfsys@transformshift{1.928301in}{0.793739in}%
\pgfsys@useobject{currentmarker}{}%
\end{pgfscope}%
\begin{pgfscope}%
\pgfsys@transformshift{2.075376in}{0.780429in}%
\pgfsys@useobject{currentmarker}{}%
\end{pgfscope}%
\begin{pgfscope}%
\pgfsys@transformshift{2.222452in}{0.785510in}%
\pgfsys@useobject{currentmarker}{}%
\end{pgfscope}%
\begin{pgfscope}%
\pgfsys@transformshift{2.369528in}{0.773067in}%
\pgfsys@useobject{currentmarker}{}%
\end{pgfscope}%
\begin{pgfscope}%
\pgfsys@transformshift{2.516604in}{0.783303in}%
\pgfsys@useobject{currentmarker}{}%
\end{pgfscope}%
\begin{pgfscope}%
\pgfsys@transformshift{2.663680in}{0.768833in}%
\pgfsys@useobject{currentmarker}{}%
\end{pgfscope}%
\begin{pgfscope}%
\pgfsys@transformshift{2.810756in}{0.776022in}%
\pgfsys@useobject{currentmarker}{}%
\end{pgfscope}%
\begin{pgfscope}%
\pgfsys@transformshift{2.957832in}{0.780528in}%
\pgfsys@useobject{currentmarker}{}%
\end{pgfscope}%
\begin{pgfscope}%
\pgfsys@transformshift{3.104908in}{0.768598in}%
\pgfsys@useobject{currentmarker}{}%
\end{pgfscope}%
\begin{pgfscope}%
\pgfsys@transformshift{3.251984in}{0.763035in}%
\pgfsys@useobject{currentmarker}{}%
\end{pgfscope}%
\begin{pgfscope}%
\pgfsys@transformshift{3.399060in}{0.764455in}%
\pgfsys@useobject{currentmarker}{}%
\end{pgfscope}%
\begin{pgfscope}%
\pgfsys@transformshift{3.546136in}{0.760023in}%
\pgfsys@useobject{currentmarker}{}%
\end{pgfscope}%
\begin{pgfscope}%
\pgfsys@transformshift{3.693211in}{0.754383in}%
\pgfsys@useobject{currentmarker}{}%
\end{pgfscope}%
\begin{pgfscope}%
\pgfsys@transformshift{3.840287in}{0.750054in}%
\pgfsys@useobject{currentmarker}{}%
\end{pgfscope}%
\begin{pgfscope}%
\pgfsys@transformshift{3.987363in}{0.788390in}%
\pgfsys@useobject{currentmarker}{}%
\end{pgfscope}%
\end{pgfscope}%
\begin{pgfscope}%
\pgfpathrectangle{\pgfqpoint{0.898769in}{0.566590in}}{\pgfqpoint{3.137619in}{1.799039in}}%
\pgfusepath{clip}%
\pgfsetrectcap%
\pgfsetroundjoin%
\pgfsetlinewidth{1.505625pt}%
\definecolor{currentstroke}{rgb}{0.000000,0.500000,0.000000}%
\pgfsetstrokecolor{currentstroke}%
\pgfsetdash{}{0pt}%
\pgfpathmoveto{\pgfqpoint{0.898769in}{1.151407in}}%
\pgfpathlineto{\pgfqpoint{0.947794in}{1.153496in}}%
\pgfpathlineto{\pgfqpoint{0.996820in}{1.027486in}}%
\pgfpathlineto{\pgfqpoint{1.045845in}{1.048969in}}%
\pgfpathlineto{\pgfqpoint{1.094870in}{0.979043in}}%
\pgfpathlineto{\pgfqpoint{1.143896in}{0.953495in}}%
\pgfpathlineto{\pgfqpoint{1.192921in}{1.080866in}}%
\pgfpathlineto{\pgfqpoint{1.241946in}{0.943409in}}%
\pgfpathlineto{\pgfqpoint{1.290972in}{0.862273in}}%
\pgfpathlineto{\pgfqpoint{1.339997in}{1.098041in}}%
\pgfpathlineto{\pgfqpoint{1.389022in}{0.782281in}}%
\pgfpathlineto{\pgfqpoint{1.438047in}{0.793840in}}%
\pgfpathlineto{\pgfqpoint{1.487073in}{0.773012in}}%
\pgfpathlineto{\pgfqpoint{1.536098in}{0.760410in}}%
\pgfpathlineto{\pgfqpoint{1.585123in}{0.716701in}}%
\pgfpathlineto{\pgfqpoint{1.634149in}{0.719243in}}%
\pgfpathlineto{\pgfqpoint{1.683174in}{0.697362in}}%
\pgfpathlineto{\pgfqpoint{1.732199in}{0.743317in}}%
\pgfpathlineto{\pgfqpoint{1.781225in}{0.718795in}}%
\pgfpathlineto{\pgfqpoint{1.830250in}{0.671453in}}%
\pgfpathlineto{\pgfqpoint{1.879275in}{0.689739in}}%
\pgfpathlineto{\pgfqpoint{1.928301in}{0.691282in}}%
\pgfpathlineto{\pgfqpoint{1.977326in}{0.692831in}}%
\pgfpathlineto{\pgfqpoint{2.026351in}{0.673878in}}%
\pgfpathlineto{\pgfqpoint{2.075376in}{0.669990in}}%
\pgfpathlineto{\pgfqpoint{2.124402in}{0.661489in}}%
\pgfpathlineto{\pgfqpoint{2.173427in}{0.674562in}}%
\pgfpathlineto{\pgfqpoint{2.222452in}{0.694350in}}%
\pgfpathlineto{\pgfqpoint{2.271478in}{0.679067in}}%
\pgfpathlineto{\pgfqpoint{2.320503in}{0.648364in}}%
\pgfpathlineto{\pgfqpoint{2.369528in}{0.687842in}}%
\pgfpathlineto{\pgfqpoint{2.418554in}{0.682557in}}%
\pgfpathlineto{\pgfqpoint{2.467579in}{0.662897in}}%
\pgfpathlineto{\pgfqpoint{2.516604in}{0.673015in}}%
\pgfpathlineto{\pgfqpoint{2.565629in}{0.665709in}}%
\pgfpathlineto{\pgfqpoint{2.614655in}{0.676315in}}%
\pgfpathlineto{\pgfqpoint{2.663680in}{0.656534in}}%
\pgfpathlineto{\pgfqpoint{2.712705in}{0.669555in}}%
\pgfpathlineto{\pgfqpoint{2.761731in}{0.684208in}}%
\pgfpathlineto{\pgfqpoint{2.810756in}{0.667157in}}%
\pgfpathlineto{\pgfqpoint{2.859781in}{0.679670in}}%
\pgfpathlineto{\pgfqpoint{2.908807in}{0.664075in}}%
\pgfpathlineto{\pgfqpoint{2.957832in}{0.666762in}}%
\pgfpathlineto{\pgfqpoint{3.006857in}{0.663660in}}%
\pgfpathlineto{\pgfqpoint{3.055882in}{0.678441in}}%
\pgfpathlineto{\pgfqpoint{3.104908in}{0.671164in}}%
\pgfpathlineto{\pgfqpoint{3.153933in}{0.663822in}}%
\pgfpathlineto{\pgfqpoint{3.202958in}{0.652349in}}%
\pgfpathlineto{\pgfqpoint{3.251984in}{0.657623in}}%
\pgfpathlineto{\pgfqpoint{3.301009in}{0.666961in}}%
\pgfpathlineto{\pgfqpoint{3.350034in}{0.664336in}}%
\pgfpathlineto{\pgfqpoint{3.399060in}{0.675011in}}%
\pgfpathlineto{\pgfqpoint{3.448085in}{0.670911in}}%
\pgfpathlineto{\pgfqpoint{3.497110in}{0.669831in}}%
\pgfpathlineto{\pgfqpoint{3.546136in}{0.680384in}}%
\pgfpathlineto{\pgfqpoint{3.595161in}{0.653195in}}%
\pgfpathlineto{\pgfqpoint{3.644186in}{0.665389in}}%
\pgfpathlineto{\pgfqpoint{3.693211in}{0.665492in}}%
\pgfpathlineto{\pgfqpoint{3.742237in}{0.670426in}}%
\pgfpathlineto{\pgfqpoint{3.791262in}{0.673408in}}%
\pgfpathlineto{\pgfqpoint{3.840287in}{0.663705in}}%
\pgfpathlineto{\pgfqpoint{3.889313in}{0.670758in}}%
\pgfpathlineto{\pgfqpoint{3.938338in}{0.657753in}}%
\pgfpathlineto{\pgfqpoint{3.987363in}{0.682868in}}%
\pgfpathlineto{\pgfqpoint{4.036389in}{0.657564in}}%
\pgfusepath{stroke}%
\end{pgfscope}%
\begin{pgfscope}%
\pgfpathrectangle{\pgfqpoint{0.898769in}{0.566590in}}{\pgfqpoint{3.137619in}{1.799039in}}%
\pgfusepath{clip}%
\pgfsetbuttcap%
\pgfsetmiterjoin%
\definecolor{currentfill}{rgb}{0.000000,0.000000,0.000000}%
\pgfsetfillcolor{currentfill}%
\pgfsetfillopacity{0.000000}%
\pgfsetlinewidth{1.003750pt}%
\definecolor{currentstroke}{rgb}{0.000000,0.500000,0.000000}%
\pgfsetstrokecolor{currentstroke}%
\pgfsetdash{}{0pt}%
\pgfsys@defobject{currentmarker}{\pgfqpoint{-0.041667in}{-0.041667in}}{\pgfqpoint{0.041667in}{0.041667in}}{%
\pgfpathmoveto{\pgfqpoint{-0.041667in}{-0.041667in}}%
\pgfpathlineto{\pgfqpoint{0.041667in}{-0.041667in}}%
\pgfpathlineto{\pgfqpoint{0.041667in}{0.041667in}}%
\pgfpathlineto{\pgfqpoint{-0.041667in}{0.041667in}}%
\pgfpathclose%
\pgfusepath{stroke,fill}%
}%
\begin{pgfscope}%
\pgfsys@transformshift{0.898769in}{1.151407in}%
\pgfsys@useobject{currentmarker}{}%
\end{pgfscope}%
\begin{pgfscope}%
\pgfsys@transformshift{1.143896in}{0.953495in}%
\pgfsys@useobject{currentmarker}{}%
\end{pgfscope}%
\begin{pgfscope}%
\pgfsys@transformshift{1.389022in}{0.782281in}%
\pgfsys@useobject{currentmarker}{}%
\end{pgfscope}%
\begin{pgfscope}%
\pgfsys@transformshift{1.634149in}{0.719243in}%
\pgfsys@useobject{currentmarker}{}%
\end{pgfscope}%
\begin{pgfscope}%
\pgfsys@transformshift{1.879275in}{0.689739in}%
\pgfsys@useobject{currentmarker}{}%
\end{pgfscope}%
\begin{pgfscope}%
\pgfsys@transformshift{2.124402in}{0.661489in}%
\pgfsys@useobject{currentmarker}{}%
\end{pgfscope}%
\begin{pgfscope}%
\pgfsys@transformshift{2.369528in}{0.687842in}%
\pgfsys@useobject{currentmarker}{}%
\end{pgfscope}%
\begin{pgfscope}%
\pgfsys@transformshift{2.614655in}{0.676315in}%
\pgfsys@useobject{currentmarker}{}%
\end{pgfscope}%
\begin{pgfscope}%
\pgfsys@transformshift{2.859781in}{0.679670in}%
\pgfsys@useobject{currentmarker}{}%
\end{pgfscope}%
\begin{pgfscope}%
\pgfsys@transformshift{3.104908in}{0.671164in}%
\pgfsys@useobject{currentmarker}{}%
\end{pgfscope}%
\begin{pgfscope}%
\pgfsys@transformshift{3.350034in}{0.664336in}%
\pgfsys@useobject{currentmarker}{}%
\end{pgfscope}%
\begin{pgfscope}%
\pgfsys@transformshift{3.595161in}{0.653195in}%
\pgfsys@useobject{currentmarker}{}%
\end{pgfscope}%
\begin{pgfscope}%
\pgfsys@transformshift{3.840287in}{0.663705in}%
\pgfsys@useobject{currentmarker}{}%
\end{pgfscope}%
\end{pgfscope}%
\begin{pgfscope}%
\pgfpathrectangle{\pgfqpoint{0.898769in}{0.566590in}}{\pgfqpoint{3.137619in}{1.799039in}}%
\pgfusepath{clip}%
\pgfsetrectcap%
\pgfsetroundjoin%
\pgfsetlinewidth{1.505625pt}%
\definecolor{currentstroke}{rgb}{0.494000,0.184000,0.556000}%
\pgfsetstrokecolor{currentstroke}%
\pgfsetdash{}{0pt}%
\pgfpathmoveto{\pgfqpoint{0.898769in}{1.412041in}}%
\pgfpathlineto{\pgfqpoint{0.947794in}{1.415439in}}%
\pgfpathlineto{\pgfqpoint{0.996820in}{1.369764in}}%
\pgfpathlineto{\pgfqpoint{1.045845in}{1.262025in}}%
\pgfpathlineto{\pgfqpoint{1.094870in}{1.382250in}}%
\pgfpathlineto{\pgfqpoint{1.143896in}{1.281949in}}%
\pgfpathlineto{\pgfqpoint{1.192921in}{1.774999in}}%
\pgfpathlineto{\pgfqpoint{1.241946in}{1.142913in}}%
\pgfpathlineto{\pgfqpoint{1.290972in}{0.934650in}}%
\pgfpathlineto{\pgfqpoint{1.339997in}{1.097087in}}%
\pgfpathlineto{\pgfqpoint{1.389022in}{0.876597in}}%
\pgfpathlineto{\pgfqpoint{1.438047in}{0.847107in}}%
\pgfpathlineto{\pgfqpoint{1.487073in}{0.847095in}}%
\pgfpathlineto{\pgfqpoint{1.536098in}{0.875720in}}%
\pgfpathlineto{\pgfqpoint{1.585123in}{0.821315in}}%
\pgfpathlineto{\pgfqpoint{1.634149in}{0.842681in}}%
\pgfpathlineto{\pgfqpoint{1.683174in}{0.848470in}}%
\pgfpathlineto{\pgfqpoint{1.732199in}{0.829660in}}%
\pgfpathlineto{\pgfqpoint{1.781225in}{0.826277in}}%
\pgfpathlineto{\pgfqpoint{1.830250in}{0.799709in}}%
\pgfpathlineto{\pgfqpoint{1.879275in}{0.797351in}}%
\pgfpathlineto{\pgfqpoint{1.928301in}{0.801503in}}%
\pgfpathlineto{\pgfqpoint{1.977326in}{0.796865in}}%
\pgfpathlineto{\pgfqpoint{2.026351in}{0.817558in}}%
\pgfpathlineto{\pgfqpoint{2.075376in}{0.779740in}}%
\pgfpathlineto{\pgfqpoint{2.124402in}{0.778752in}}%
\pgfpathlineto{\pgfqpoint{2.173427in}{0.781056in}}%
\pgfpathlineto{\pgfqpoint{2.222452in}{0.773723in}}%
\pgfpathlineto{\pgfqpoint{2.271478in}{0.769376in}}%
\pgfpathlineto{\pgfqpoint{2.320503in}{0.765723in}}%
\pgfpathlineto{\pgfqpoint{2.369528in}{0.765754in}}%
\pgfpathlineto{\pgfqpoint{2.418554in}{0.751712in}}%
\pgfpathlineto{\pgfqpoint{2.467579in}{0.756846in}}%
\pgfpathlineto{\pgfqpoint{2.516604in}{0.771446in}}%
\pgfpathlineto{\pgfqpoint{2.565629in}{0.760341in}}%
\pgfpathlineto{\pgfqpoint{2.614655in}{0.765445in}}%
\pgfpathlineto{\pgfqpoint{2.663680in}{0.776214in}}%
\pgfpathlineto{\pgfqpoint{2.712705in}{0.753469in}}%
\pgfpathlineto{\pgfqpoint{2.761731in}{0.769370in}}%
\pgfpathlineto{\pgfqpoint{2.810756in}{0.767567in}}%
\pgfpathlineto{\pgfqpoint{2.859781in}{0.774266in}}%
\pgfpathlineto{\pgfqpoint{2.908807in}{0.771255in}}%
\pgfpathlineto{\pgfqpoint{2.957832in}{0.768431in}}%
\pgfpathlineto{\pgfqpoint{3.006857in}{0.781472in}}%
\pgfpathlineto{\pgfqpoint{3.055882in}{0.752977in}}%
\pgfpathlineto{\pgfqpoint{3.104908in}{0.760090in}}%
\pgfpathlineto{\pgfqpoint{3.153933in}{0.773934in}}%
\pgfpathlineto{\pgfqpoint{3.202958in}{0.771267in}}%
\pgfpathlineto{\pgfqpoint{3.251984in}{0.760018in}}%
\pgfpathlineto{\pgfqpoint{3.301009in}{0.770109in}}%
\pgfpathlineto{\pgfqpoint{3.350034in}{0.768087in}}%
\pgfpathlineto{\pgfqpoint{3.399060in}{0.762002in}}%
\pgfpathlineto{\pgfqpoint{3.448085in}{0.757454in}}%
\pgfpathlineto{\pgfqpoint{3.497110in}{0.754930in}}%
\pgfpathlineto{\pgfqpoint{3.546136in}{0.765983in}}%
\pgfpathlineto{\pgfqpoint{3.595161in}{0.742969in}}%
\pgfpathlineto{\pgfqpoint{3.644186in}{0.765604in}}%
\pgfpathlineto{\pgfqpoint{3.693211in}{0.744010in}}%
\pgfpathlineto{\pgfqpoint{3.742237in}{0.768523in}}%
\pgfpathlineto{\pgfqpoint{3.791262in}{0.768421in}}%
\pgfpathlineto{\pgfqpoint{3.840287in}{0.752325in}}%
\pgfpathlineto{\pgfqpoint{3.889313in}{0.756872in}}%
\pgfpathlineto{\pgfqpoint{3.938338in}{0.744891in}}%
\pgfpathlineto{\pgfqpoint{3.987363in}{0.765873in}}%
\pgfpathlineto{\pgfqpoint{4.036389in}{0.762951in}}%
\pgfusepath{stroke}%
\end{pgfscope}%
\begin{pgfscope}%
\pgfpathrectangle{\pgfqpoint{0.898769in}{0.566590in}}{\pgfqpoint{3.137619in}{1.799039in}}%
\pgfusepath{clip}%
\pgfsetbuttcap%
\pgfsetroundjoin%
\definecolor{currentfill}{rgb}{0.494000,0.184000,0.556000}%
\pgfsetfillcolor{currentfill}%
\pgfsetlinewidth{1.003750pt}%
\definecolor{currentstroke}{rgb}{0.494000,0.184000,0.556000}%
\pgfsetstrokecolor{currentstroke}%
\pgfsetdash{}{0pt}%
\pgfsys@defobject{currentmarker}{\pgfqpoint{-0.041667in}{-0.041667in}}{\pgfqpoint{0.041667in}{0.041667in}}{%
\pgfpathmoveto{\pgfqpoint{-0.041667in}{-0.041667in}}%
\pgfpathlineto{\pgfqpoint{0.041667in}{0.041667in}}%
\pgfpathmoveto{\pgfqpoint{-0.041667in}{0.041667in}}%
\pgfpathlineto{\pgfqpoint{0.041667in}{-0.041667in}}%
\pgfusepath{stroke,fill}%
}%
\begin{pgfscope}%
\pgfsys@transformshift{0.898769in}{1.412041in}%
\pgfsys@useobject{currentmarker}{}%
\end{pgfscope}%
\begin{pgfscope}%
\pgfsys@transformshift{1.094870in}{1.382250in}%
\pgfsys@useobject{currentmarker}{}%
\end{pgfscope}%
\begin{pgfscope}%
\pgfsys@transformshift{1.290972in}{0.934650in}%
\pgfsys@useobject{currentmarker}{}%
\end{pgfscope}%
\begin{pgfscope}%
\pgfsys@transformshift{1.487073in}{0.847095in}%
\pgfsys@useobject{currentmarker}{}%
\end{pgfscope}%
\begin{pgfscope}%
\pgfsys@transformshift{1.683174in}{0.848470in}%
\pgfsys@useobject{currentmarker}{}%
\end{pgfscope}%
\begin{pgfscope}%
\pgfsys@transformshift{1.879275in}{0.797351in}%
\pgfsys@useobject{currentmarker}{}%
\end{pgfscope}%
\begin{pgfscope}%
\pgfsys@transformshift{2.075376in}{0.779740in}%
\pgfsys@useobject{currentmarker}{}%
\end{pgfscope}%
\begin{pgfscope}%
\pgfsys@transformshift{2.271478in}{0.769376in}%
\pgfsys@useobject{currentmarker}{}%
\end{pgfscope}%
\begin{pgfscope}%
\pgfsys@transformshift{2.467579in}{0.756846in}%
\pgfsys@useobject{currentmarker}{}%
\end{pgfscope}%
\begin{pgfscope}%
\pgfsys@transformshift{2.663680in}{0.776214in}%
\pgfsys@useobject{currentmarker}{}%
\end{pgfscope}%
\begin{pgfscope}%
\pgfsys@transformshift{2.859781in}{0.774266in}%
\pgfsys@useobject{currentmarker}{}%
\end{pgfscope}%
\begin{pgfscope}%
\pgfsys@transformshift{3.055882in}{0.752977in}%
\pgfsys@useobject{currentmarker}{}%
\end{pgfscope}%
\begin{pgfscope}%
\pgfsys@transformshift{3.251984in}{0.760018in}%
\pgfsys@useobject{currentmarker}{}%
\end{pgfscope}%
\begin{pgfscope}%
\pgfsys@transformshift{3.448085in}{0.757454in}%
\pgfsys@useobject{currentmarker}{}%
\end{pgfscope}%
\begin{pgfscope}%
\pgfsys@transformshift{3.644186in}{0.765604in}%
\pgfsys@useobject{currentmarker}{}%
\end{pgfscope}%
\begin{pgfscope}%
\pgfsys@transformshift{3.840287in}{0.752325in}%
\pgfsys@useobject{currentmarker}{}%
\end{pgfscope}%
\begin{pgfscope}%
\pgfsys@transformshift{4.036389in}{0.762951in}%
\pgfsys@useobject{currentmarker}{}%
\end{pgfscope}%
\end{pgfscope}%
\begin{pgfscope}%
\pgfsetrectcap%
\pgfsetmiterjoin%
\pgfsetlinewidth{0.803000pt}%
\definecolor{currentstroke}{rgb}{0.000000,0.000000,0.000000}%
\pgfsetstrokecolor{currentstroke}%
\pgfsetdash{}{0pt}%
\pgfpathmoveto{\pgfqpoint{0.898769in}{0.566590in}}%
\pgfpathlineto{\pgfqpoint{0.898769in}{2.365629in}}%
\pgfusepath{stroke}%
\end{pgfscope}%
\begin{pgfscope}%
\pgfsetrectcap%
\pgfsetmiterjoin%
\pgfsetlinewidth{0.803000pt}%
\definecolor{currentstroke}{rgb}{0.000000,0.000000,0.000000}%
\pgfsetstrokecolor{currentstroke}%
\pgfsetdash{}{0pt}%
\pgfpathmoveto{\pgfqpoint{4.036389in}{0.566590in}}%
\pgfpathlineto{\pgfqpoint{4.036389in}{2.365629in}}%
\pgfusepath{stroke}%
\end{pgfscope}%
\begin{pgfscope}%
\pgfsetrectcap%
\pgfsetmiterjoin%
\pgfsetlinewidth{0.803000pt}%
\definecolor{currentstroke}{rgb}{0.000000,0.000000,0.000000}%
\pgfsetstrokecolor{currentstroke}%
\pgfsetdash{}{0pt}%
\pgfpathmoveto{\pgfqpoint{0.898769in}{0.566590in}}%
\pgfpathlineto{\pgfqpoint{4.036389in}{0.566590in}}%
\pgfusepath{stroke}%
\end{pgfscope}%
\begin{pgfscope}%
\pgfsetrectcap%
\pgfsetmiterjoin%
\pgfsetlinewidth{0.803000pt}%
\definecolor{currentstroke}{rgb}{0.000000,0.000000,0.000000}%
\pgfsetstrokecolor{currentstroke}%
\pgfsetdash{}{0pt}%
\pgfpathmoveto{\pgfqpoint{0.898769in}{2.365629in}}%
\pgfpathlineto{\pgfqpoint{4.036389in}{2.365629in}}%
\pgfusepath{stroke}%
\end{pgfscope}%
\begin{pgfscope}%
\pgfsetbuttcap%
\pgfsetmiterjoin%
\definecolor{currentfill}{rgb}{1.000000,1.000000,1.000000}%
\pgfsetfillcolor{currentfill}%
\pgfsetfillopacity{0.800000}%
\pgfsetlinewidth{1.003750pt}%
\definecolor{currentstroke}{rgb}{0.800000,0.800000,0.800000}%
\pgfsetstrokecolor{currentstroke}%
\pgfsetstrokeopacity{0.800000}%
\pgfsetdash{}{0pt}%
\pgfpathmoveto{\pgfqpoint{2.906752in}{1.568430in}}%
\pgfpathlineto{\pgfqpoint{3.948889in}{1.568430in}}%
\pgfpathquadraticcurveto{\pgfqpoint{3.973889in}{1.568430in}}{\pgfqpoint{3.973889in}{1.593430in}}%
\pgfpathlineto{\pgfqpoint{3.973889in}{2.278129in}}%
\pgfpathquadraticcurveto{\pgfqpoint{3.973889in}{2.303129in}}{\pgfqpoint{3.948889in}{2.303129in}}%
\pgfpathlineto{\pgfqpoint{2.906752in}{2.303129in}}%
\pgfpathquadraticcurveto{\pgfqpoint{2.881752in}{2.303129in}}{\pgfqpoint{2.881752in}{2.278129in}}%
\pgfpathlineto{\pgfqpoint{2.881752in}{1.593430in}}%
\pgfpathquadraticcurveto{\pgfqpoint{2.881752in}{1.568430in}}{\pgfqpoint{2.906752in}{1.568430in}}%
\pgfpathclose%
\pgfusepath{stroke,fill}%
\end{pgfscope}%
\begin{pgfscope}%
\pgfsetbuttcap%
\pgfsetroundjoin%
\definecolor{currentfill}{rgb}{0.000000,0.000000,0.000000}%
\pgfsetfillcolor{currentfill}%
\pgfsetfillopacity{0.000000}%
\pgfsetlinewidth{1.003750pt}%
\definecolor{currentstroke}{rgb}{0.000000,0.447000,0.741000}%
\pgfsetstrokecolor{currentstroke}%
\pgfsetdash{}{0pt}%
\pgfsys@defobject{currentmarker}{\pgfqpoint{-0.041667in}{-0.041667in}}{\pgfqpoint{0.041667in}{0.041667in}}{%
\pgfpathmoveto{\pgfqpoint{0.000000in}{-0.041667in}}%
\pgfpathcurveto{\pgfqpoint{0.011050in}{-0.041667in}}{\pgfqpoint{0.021649in}{-0.037276in}}{\pgfqpoint{0.029463in}{-0.029463in}}%
\pgfpathcurveto{\pgfqpoint{0.037276in}{-0.021649in}}{\pgfqpoint{0.041667in}{-0.011050in}}{\pgfqpoint{0.041667in}{0.000000in}}%
\pgfpathcurveto{\pgfqpoint{0.041667in}{0.011050in}}{\pgfqpoint{0.037276in}{0.021649in}}{\pgfqpoint{0.029463in}{0.029463in}}%
\pgfpathcurveto{\pgfqpoint{0.021649in}{0.037276in}}{\pgfqpoint{0.011050in}{0.041667in}}{\pgfqpoint{0.000000in}{0.041667in}}%
\pgfpathcurveto{\pgfqpoint{-0.011050in}{0.041667in}}{\pgfqpoint{-0.021649in}{0.037276in}}{\pgfqpoint{-0.029463in}{0.029463in}}%
\pgfpathcurveto{\pgfqpoint{-0.037276in}{0.021649in}}{\pgfqpoint{-0.041667in}{0.011050in}}{\pgfqpoint{-0.041667in}{0.000000in}}%
\pgfpathcurveto{\pgfqpoint{-0.041667in}{-0.011050in}}{\pgfqpoint{-0.037276in}{-0.021649in}}{\pgfqpoint{-0.029463in}{-0.029463in}}%
\pgfpathcurveto{\pgfqpoint{-0.021649in}{-0.037276in}}{\pgfqpoint{-0.011050in}{-0.041667in}}{\pgfqpoint{0.000000in}{-0.041667in}}%
\pgfpathclose%
\pgfusepath{stroke,fill}%
}%
\begin{pgfscope}%
\pgfsys@transformshift{3.056752in}{2.209379in}%
\pgfsys@useobject{currentmarker}{}%
\end{pgfscope}%
\end{pgfscope}%
\begin{pgfscope}%
\definecolor{textcolor}{rgb}{0.000000,0.000000,0.000000}%
\pgfsetstrokecolor{textcolor}%
\pgfsetfillcolor{textcolor}%
\pgftext[x=3.281752in,y=2.165629in,left,base]{\color{textcolor}\rmfamily\fontsize{9.000000}{10.800000}\selectfont \(\displaystyle \nu_1 =\) -7.68 }%
\end{pgfscope}%
\begin{pgfscope}%
\pgfsetbuttcap%
\pgfsetroundjoin%
\definecolor{currentfill}{rgb}{0.850000,0.324000,0.098000}%
\pgfsetfillcolor{currentfill}%
\pgfsetlinewidth{1.003750pt}%
\definecolor{currentstroke}{rgb}{0.850000,0.324000,0.098000}%
\pgfsetstrokecolor{currentstroke}%
\pgfsetdash{}{0pt}%
\pgfsys@defobject{currentmarker}{\pgfqpoint{-0.041667in}{-0.041667in}}{\pgfqpoint{0.041667in}{0.041667in}}{%
\pgfpathmoveto{\pgfqpoint{-0.041667in}{0.000000in}}%
\pgfpathlineto{\pgfqpoint{0.041667in}{0.000000in}}%
\pgfpathmoveto{\pgfqpoint{0.000000in}{-0.041667in}}%
\pgfpathlineto{\pgfqpoint{0.000000in}{0.041667in}}%
\pgfusepath{stroke,fill}%
}%
\begin{pgfscope}%
\pgfsys@transformshift{3.056752in}{2.035079in}%
\pgfsys@useobject{currentmarker}{}%
\end{pgfscope}%
\end{pgfscope}%
\begin{pgfscope}%
\definecolor{textcolor}{rgb}{0.000000,0.000000,0.000000}%
\pgfsetstrokecolor{textcolor}%
\pgfsetfillcolor{textcolor}%
\pgftext[x=3.281752in,y=1.991329in,left,base]{\color{textcolor}\rmfamily\fontsize{9.000000}{10.800000}\selectfont \(\displaystyle \nu_2 =\) 39.68}%
\end{pgfscope}%
\begin{pgfscope}%
\pgfsetbuttcap%
\pgfsetmiterjoin%
\definecolor{currentfill}{rgb}{0.000000,0.000000,0.000000}%
\pgfsetfillcolor{currentfill}%
\pgfsetfillopacity{0.000000}%
\pgfsetlinewidth{1.003750pt}%
\definecolor{currentstroke}{rgb}{0.000000,0.500000,0.000000}%
\pgfsetstrokecolor{currentstroke}%
\pgfsetdash{}{0pt}%
\pgfsys@defobject{currentmarker}{\pgfqpoint{-0.041667in}{-0.041667in}}{\pgfqpoint{0.041667in}{0.041667in}}{%
\pgfpathmoveto{\pgfqpoint{-0.041667in}{-0.041667in}}%
\pgfpathlineto{\pgfqpoint{0.041667in}{-0.041667in}}%
\pgfpathlineto{\pgfqpoint{0.041667in}{0.041667in}}%
\pgfpathlineto{\pgfqpoint{-0.041667in}{0.041667in}}%
\pgfpathclose%
\pgfusepath{stroke,fill}%
}%
\begin{pgfscope}%
\pgfsys@transformshift{3.056752in}{1.860780in}%
\pgfsys@useobject{currentmarker}{}%
\end{pgfscope}%
\end{pgfscope}%
\begin{pgfscope}%
\definecolor{textcolor}{rgb}{0.000000,0.000000,0.000000}%
\pgfsetstrokecolor{textcolor}%
\pgfsetfillcolor{textcolor}%
\pgftext[x=3.281752in,y=1.817030in,left,base]{\color{textcolor}\rmfamily\fontsize{9.000000}{10.800000}\selectfont \(\displaystyle \nu_3\) = 40.96 }%
\end{pgfscope}%
\begin{pgfscope}%
\pgfsetbuttcap%
\pgfsetroundjoin%
\definecolor{currentfill}{rgb}{0.494000,0.184000,0.556000}%
\pgfsetfillcolor{currentfill}%
\pgfsetlinewidth{1.003750pt}%
\definecolor{currentstroke}{rgb}{0.494000,0.184000,0.556000}%
\pgfsetstrokecolor{currentstroke}%
\pgfsetdash{}{0pt}%
\pgfsys@defobject{currentmarker}{\pgfqpoint{-0.041667in}{-0.041667in}}{\pgfqpoint{0.041667in}{0.041667in}}{%
\pgfpathmoveto{\pgfqpoint{-0.041667in}{-0.041667in}}%
\pgfpathlineto{\pgfqpoint{0.041667in}{0.041667in}}%
\pgfpathmoveto{\pgfqpoint{-0.041667in}{0.041667in}}%
\pgfpathlineto{\pgfqpoint{0.041667in}{-0.041667in}}%
\pgfusepath{stroke,fill}%
}%
\begin{pgfscope}%
\pgfsys@transformshift{3.056752in}{1.686480in}%
\pgfsys@useobject{currentmarker}{}%
\end{pgfscope}%
\end{pgfscope}%
\begin{pgfscope}%
\definecolor{textcolor}{rgb}{0.000000,0.000000,0.000000}%
\pgfsetstrokecolor{textcolor}%
\pgfsetfillcolor{textcolor}%
\pgftext[x=3.281752in,y=1.642730in,left,base]{\color{textcolor}\rmfamily\fontsize{9.000000}{10.800000}\selectfont \(\displaystyle \nu_4 = \) 99.84}%
\end{pgfscope}%
\end{pgfpicture}%
\makeatother%
\endgroup%
}
					\caption{Relación ente RMSE y CRB para $\nu$.}
					\label{Fig:CRB_nu_ex1}
				\end{subfigure}
				~
				\begin{subfigure}{0.5\textwidth}
					\centering
					\resizebox{\linewidth}{!}{%% Creator: Matplotlib, PGF backend
%%
%% To include the figure in your LaTeX document, write
%%   \input{<filename>.pgf}
%%
%% Make sure the required packages are loaded in your preamble
%%   \usepackage{pgf}
%%
%% and, on pdftex
%%   \usepackage[utf8]{inputenc}\DeclareUnicodeCharacter{2212}{-}
%%
%% or, on luatex and xetex
%%   \usepackage{unicode-math}
%%
%% Figures using additional raster images can only be included by \input if
%% they are in the same directory as the main LaTeX file. For loading figures
%% from other directories you can use the `import` package
%%   \usepackage{import}
%%
%% and then include the figures with
%%   \import{<path to file>}{<filename>.pgf}
%%
%% Matplotlib used the following preamble
%%   \usepackage[utf8x]{inputenc}
%%   \usepackage[T1]{fontenc}
%%   \usepackage{amsmath,amssymb,amsfonts}
%%
\begingroup%
\makeatletter%
\begin{pgfpicture}%
\pgfpathrectangle{\pgfpointorigin}{\pgfqpoint{4.136389in}{2.495314in}}%
\pgfusepath{use as bounding box, clip}%
\begin{pgfscope}%
\pgfsetbuttcap%
\pgfsetmiterjoin%
\definecolor{currentfill}{rgb}{1.000000,1.000000,1.000000}%
\pgfsetfillcolor{currentfill}%
\pgfsetlinewidth{0.000000pt}%
\definecolor{currentstroke}{rgb}{1.000000,1.000000,1.000000}%
\pgfsetstrokecolor{currentstroke}%
\pgfsetdash{}{0pt}%
\pgfpathmoveto{\pgfqpoint{0.000000in}{0.000000in}}%
\pgfpathlineto{\pgfqpoint{4.136389in}{0.000000in}}%
\pgfpathlineto{\pgfqpoint{4.136389in}{2.495314in}}%
\pgfpathlineto{\pgfqpoint{0.000000in}{2.495314in}}%
\pgfpathclose%
\pgfusepath{fill}%
\end{pgfscope}%
\begin{pgfscope}%
\pgfsetbuttcap%
\pgfsetmiterjoin%
\definecolor{currentfill}{rgb}{1.000000,1.000000,1.000000}%
\pgfsetfillcolor{currentfill}%
\pgfsetlinewidth{0.000000pt}%
\definecolor{currentstroke}{rgb}{0.000000,0.000000,0.000000}%
\pgfsetstrokecolor{currentstroke}%
\pgfsetstrokeopacity{0.000000}%
\pgfsetdash{}{0pt}%
\pgfpathmoveto{\pgfqpoint{0.898769in}{0.566590in}}%
\pgfpathlineto{\pgfqpoint{4.036389in}{0.566590in}}%
\pgfpathlineto{\pgfqpoint{4.036389in}{2.395314in}}%
\pgfpathlineto{\pgfqpoint{0.898769in}{2.395314in}}%
\pgfpathclose%
\pgfusepath{fill}%
\end{pgfscope}%
\begin{pgfscope}%
\pgfpathrectangle{\pgfqpoint{0.898769in}{0.566590in}}{\pgfqpoint{3.137619in}{1.828724in}}%
\pgfusepath{clip}%
\pgfsetrectcap%
\pgfsetroundjoin%
\pgfsetlinewidth{0.803000pt}%
\definecolor{currentstroke}{rgb}{0.690196,0.690196,0.690196}%
\pgfsetstrokecolor{currentstroke}%
\pgfsetdash{}{0pt}%
\pgfpathmoveto{\pgfqpoint{0.898769in}{0.566590in}}%
\pgfpathlineto{\pgfqpoint{0.898769in}{2.395314in}}%
\pgfusepath{stroke}%
\end{pgfscope}%
\begin{pgfscope}%
\pgfsetbuttcap%
\pgfsetroundjoin%
\definecolor{currentfill}{rgb}{0.000000,0.000000,0.000000}%
\pgfsetfillcolor{currentfill}%
\pgfsetlinewidth{0.803000pt}%
\definecolor{currentstroke}{rgb}{0.000000,0.000000,0.000000}%
\pgfsetstrokecolor{currentstroke}%
\pgfsetdash{}{0pt}%
\pgfsys@defobject{currentmarker}{\pgfqpoint{0.000000in}{-0.048611in}}{\pgfqpoint{0.000000in}{0.000000in}}{%
\pgfpathmoveto{\pgfqpoint{0.000000in}{0.000000in}}%
\pgfpathlineto{\pgfqpoint{0.000000in}{-0.048611in}}%
\pgfusepath{stroke,fill}%
}%
\begin{pgfscope}%
\pgfsys@transformshift{0.898769in}{0.566590in}%
\pgfsys@useobject{currentmarker}{}%
\end{pgfscope}%
\end{pgfscope}%
\begin{pgfscope}%
\definecolor{textcolor}{rgb}{0.000000,0.000000,0.000000}%
\pgfsetstrokecolor{textcolor}%
\pgfsetfillcolor{textcolor}%
\pgftext[x=0.898769in,y=0.469368in,,top]{\color{textcolor}\rmfamily\fontsize{12.000000}{14.400000}\selectfont \(\displaystyle {-10}\)}%
\end{pgfscope}%
\begin{pgfscope}%
\pgfpathrectangle{\pgfqpoint{0.898769in}{0.566590in}}{\pgfqpoint{3.137619in}{1.828724in}}%
\pgfusepath{clip}%
\pgfsetrectcap%
\pgfsetroundjoin%
\pgfsetlinewidth{0.803000pt}%
\definecolor{currentstroke}{rgb}{0.690196,0.690196,0.690196}%
\pgfsetstrokecolor{currentstroke}%
\pgfsetdash{}{0pt}%
\pgfpathmoveto{\pgfqpoint{1.795232in}{0.566590in}}%
\pgfpathlineto{\pgfqpoint{1.795232in}{2.395314in}}%
\pgfusepath{stroke}%
\end{pgfscope}%
\begin{pgfscope}%
\pgfsetbuttcap%
\pgfsetroundjoin%
\definecolor{currentfill}{rgb}{0.000000,0.000000,0.000000}%
\pgfsetfillcolor{currentfill}%
\pgfsetlinewidth{0.803000pt}%
\definecolor{currentstroke}{rgb}{0.000000,0.000000,0.000000}%
\pgfsetstrokecolor{currentstroke}%
\pgfsetdash{}{0pt}%
\pgfsys@defobject{currentmarker}{\pgfqpoint{0.000000in}{-0.048611in}}{\pgfqpoint{0.000000in}{0.000000in}}{%
\pgfpathmoveto{\pgfqpoint{0.000000in}{0.000000in}}%
\pgfpathlineto{\pgfqpoint{0.000000in}{-0.048611in}}%
\pgfusepath{stroke,fill}%
}%
\begin{pgfscope}%
\pgfsys@transformshift{1.795232in}{0.566590in}%
\pgfsys@useobject{currentmarker}{}%
\end{pgfscope}%
\end{pgfscope}%
\begin{pgfscope}%
\definecolor{textcolor}{rgb}{0.000000,0.000000,0.000000}%
\pgfsetstrokecolor{textcolor}%
\pgfsetfillcolor{textcolor}%
\pgftext[x=1.795232in,y=0.469368in,,top]{\color{textcolor}\rmfamily\fontsize{12.000000}{14.400000}\selectfont \(\displaystyle {0}\)}%
\end{pgfscope}%
\begin{pgfscope}%
\pgfpathrectangle{\pgfqpoint{0.898769in}{0.566590in}}{\pgfqpoint{3.137619in}{1.828724in}}%
\pgfusepath{clip}%
\pgfsetrectcap%
\pgfsetroundjoin%
\pgfsetlinewidth{0.803000pt}%
\definecolor{currentstroke}{rgb}{0.690196,0.690196,0.690196}%
\pgfsetstrokecolor{currentstroke}%
\pgfsetdash{}{0pt}%
\pgfpathmoveto{\pgfqpoint{2.691695in}{0.566590in}}%
\pgfpathlineto{\pgfqpoint{2.691695in}{2.395314in}}%
\pgfusepath{stroke}%
\end{pgfscope}%
\begin{pgfscope}%
\pgfsetbuttcap%
\pgfsetroundjoin%
\definecolor{currentfill}{rgb}{0.000000,0.000000,0.000000}%
\pgfsetfillcolor{currentfill}%
\pgfsetlinewidth{0.803000pt}%
\definecolor{currentstroke}{rgb}{0.000000,0.000000,0.000000}%
\pgfsetstrokecolor{currentstroke}%
\pgfsetdash{}{0pt}%
\pgfsys@defobject{currentmarker}{\pgfqpoint{0.000000in}{-0.048611in}}{\pgfqpoint{0.000000in}{0.000000in}}{%
\pgfpathmoveto{\pgfqpoint{0.000000in}{0.000000in}}%
\pgfpathlineto{\pgfqpoint{0.000000in}{-0.048611in}}%
\pgfusepath{stroke,fill}%
}%
\begin{pgfscope}%
\pgfsys@transformshift{2.691695in}{0.566590in}%
\pgfsys@useobject{currentmarker}{}%
\end{pgfscope}%
\end{pgfscope}%
\begin{pgfscope}%
\definecolor{textcolor}{rgb}{0.000000,0.000000,0.000000}%
\pgfsetstrokecolor{textcolor}%
\pgfsetfillcolor{textcolor}%
\pgftext[x=2.691695in,y=0.469368in,,top]{\color{textcolor}\rmfamily\fontsize{12.000000}{14.400000}\selectfont \(\displaystyle {10}\)}%
\end{pgfscope}%
\begin{pgfscope}%
\pgfpathrectangle{\pgfqpoint{0.898769in}{0.566590in}}{\pgfqpoint{3.137619in}{1.828724in}}%
\pgfusepath{clip}%
\pgfsetrectcap%
\pgfsetroundjoin%
\pgfsetlinewidth{0.803000pt}%
\definecolor{currentstroke}{rgb}{0.690196,0.690196,0.690196}%
\pgfsetstrokecolor{currentstroke}%
\pgfsetdash{}{0pt}%
\pgfpathmoveto{\pgfqpoint{3.588157in}{0.566590in}}%
\pgfpathlineto{\pgfqpoint{3.588157in}{2.395314in}}%
\pgfusepath{stroke}%
\end{pgfscope}%
\begin{pgfscope}%
\pgfsetbuttcap%
\pgfsetroundjoin%
\definecolor{currentfill}{rgb}{0.000000,0.000000,0.000000}%
\pgfsetfillcolor{currentfill}%
\pgfsetlinewidth{0.803000pt}%
\definecolor{currentstroke}{rgb}{0.000000,0.000000,0.000000}%
\pgfsetstrokecolor{currentstroke}%
\pgfsetdash{}{0pt}%
\pgfsys@defobject{currentmarker}{\pgfqpoint{0.000000in}{-0.048611in}}{\pgfqpoint{0.000000in}{0.000000in}}{%
\pgfpathmoveto{\pgfqpoint{0.000000in}{0.000000in}}%
\pgfpathlineto{\pgfqpoint{0.000000in}{-0.048611in}}%
\pgfusepath{stroke,fill}%
}%
\begin{pgfscope}%
\pgfsys@transformshift{3.588157in}{0.566590in}%
\pgfsys@useobject{currentmarker}{}%
\end{pgfscope}%
\end{pgfscope}%
\begin{pgfscope}%
\definecolor{textcolor}{rgb}{0.000000,0.000000,0.000000}%
\pgfsetstrokecolor{textcolor}%
\pgfsetfillcolor{textcolor}%
\pgftext[x=3.588157in,y=0.469368in,,top]{\color{textcolor}\rmfamily\fontsize{12.000000}{14.400000}\selectfont \(\displaystyle {20}\)}%
\end{pgfscope}%
\begin{pgfscope}%
\definecolor{textcolor}{rgb}{0.000000,0.000000,0.000000}%
\pgfsetstrokecolor{textcolor}%
\pgfsetfillcolor{textcolor}%
\pgftext[x=2.467579in,y=0.266626in,,top]{\color{textcolor}\rmfamily\fontsize{12.000000}{14.400000}\selectfont SNR [dB]}%
\end{pgfscope}%
\begin{pgfscope}%
\pgfpathrectangle{\pgfqpoint{0.898769in}{0.566590in}}{\pgfqpoint{3.137619in}{1.828724in}}%
\pgfusepath{clip}%
\pgfsetrectcap%
\pgfsetroundjoin%
\pgfsetlinewidth{0.803000pt}%
\definecolor{currentstroke}{rgb}{0.690196,0.690196,0.690196}%
\pgfsetstrokecolor{currentstroke}%
\pgfsetdash{}{0pt}%
\pgfpathmoveto{\pgfqpoint{0.898769in}{1.025694in}}%
\pgfpathlineto{\pgfqpoint{4.036389in}{1.025694in}}%
\pgfusepath{stroke}%
\end{pgfscope}%
\begin{pgfscope}%
\pgfsetbuttcap%
\pgfsetroundjoin%
\definecolor{currentfill}{rgb}{0.000000,0.000000,0.000000}%
\pgfsetfillcolor{currentfill}%
\pgfsetlinewidth{0.803000pt}%
\definecolor{currentstroke}{rgb}{0.000000,0.000000,0.000000}%
\pgfsetstrokecolor{currentstroke}%
\pgfsetdash{}{0pt}%
\pgfsys@defobject{currentmarker}{\pgfqpoint{-0.048611in}{0.000000in}}{\pgfqpoint{-0.000000in}{0.000000in}}{%
\pgfpathmoveto{\pgfqpoint{-0.000000in}{0.000000in}}%
\pgfpathlineto{\pgfqpoint{-0.048611in}{0.000000in}}%
\pgfusepath{stroke,fill}%
}%
\begin{pgfscope}%
\pgfsys@transformshift{0.898769in}{1.025694in}%
\pgfsys@useobject{currentmarker}{}%
\end{pgfscope}%
\end{pgfscope}%
\begin{pgfscope}%
\definecolor{textcolor}{rgb}{0.000000,0.000000,0.000000}%
\pgfsetstrokecolor{textcolor}%
\pgfsetfillcolor{textcolor}%
\pgftext[x=0.572381in, y=0.968301in, left, base]{\color{textcolor}\rmfamily\fontsize{12.000000}{14.400000}\selectfont \(\displaystyle {10^{0}}\)}%
\end{pgfscope}%
\begin{pgfscope}%
\pgfpathrectangle{\pgfqpoint{0.898769in}{0.566590in}}{\pgfqpoint{3.137619in}{1.828724in}}%
\pgfusepath{clip}%
\pgfsetrectcap%
\pgfsetroundjoin%
\pgfsetlinewidth{0.803000pt}%
\definecolor{currentstroke}{rgb}{0.690196,0.690196,0.690196}%
\pgfsetstrokecolor{currentstroke}%
\pgfsetdash{}{0pt}%
\pgfpathmoveto{\pgfqpoint{0.898769in}{1.624665in}}%
\pgfpathlineto{\pgfqpoint{4.036389in}{1.624665in}}%
\pgfusepath{stroke}%
\end{pgfscope}%
\begin{pgfscope}%
\pgfsetbuttcap%
\pgfsetroundjoin%
\definecolor{currentfill}{rgb}{0.000000,0.000000,0.000000}%
\pgfsetfillcolor{currentfill}%
\pgfsetlinewidth{0.803000pt}%
\definecolor{currentstroke}{rgb}{0.000000,0.000000,0.000000}%
\pgfsetstrokecolor{currentstroke}%
\pgfsetdash{}{0pt}%
\pgfsys@defobject{currentmarker}{\pgfqpoint{-0.048611in}{0.000000in}}{\pgfqpoint{-0.000000in}{0.000000in}}{%
\pgfpathmoveto{\pgfqpoint{-0.000000in}{0.000000in}}%
\pgfpathlineto{\pgfqpoint{-0.048611in}{0.000000in}}%
\pgfusepath{stroke,fill}%
}%
\begin{pgfscope}%
\pgfsys@transformshift{0.898769in}{1.624665in}%
\pgfsys@useobject{currentmarker}{}%
\end{pgfscope}%
\end{pgfscope}%
\begin{pgfscope}%
\definecolor{textcolor}{rgb}{0.000000,0.000000,0.000000}%
\pgfsetstrokecolor{textcolor}%
\pgfsetfillcolor{textcolor}%
\pgftext[x=0.572381in, y=1.567272in, left, base]{\color{textcolor}\rmfamily\fontsize{12.000000}{14.400000}\selectfont \(\displaystyle {10^{1}}\)}%
\end{pgfscope}%
\begin{pgfscope}%
\pgfpathrectangle{\pgfqpoint{0.898769in}{0.566590in}}{\pgfqpoint{3.137619in}{1.828724in}}%
\pgfusepath{clip}%
\pgfsetrectcap%
\pgfsetroundjoin%
\pgfsetlinewidth{0.803000pt}%
\definecolor{currentstroke}{rgb}{0.690196,0.690196,0.690196}%
\pgfsetstrokecolor{currentstroke}%
\pgfsetdash{}{0pt}%
\pgfpathmoveto{\pgfqpoint{0.898769in}{2.223636in}}%
\pgfpathlineto{\pgfqpoint{4.036389in}{2.223636in}}%
\pgfusepath{stroke}%
\end{pgfscope}%
\begin{pgfscope}%
\pgfsetbuttcap%
\pgfsetroundjoin%
\definecolor{currentfill}{rgb}{0.000000,0.000000,0.000000}%
\pgfsetfillcolor{currentfill}%
\pgfsetlinewidth{0.803000pt}%
\definecolor{currentstroke}{rgb}{0.000000,0.000000,0.000000}%
\pgfsetstrokecolor{currentstroke}%
\pgfsetdash{}{0pt}%
\pgfsys@defobject{currentmarker}{\pgfqpoint{-0.048611in}{0.000000in}}{\pgfqpoint{-0.000000in}{0.000000in}}{%
\pgfpathmoveto{\pgfqpoint{-0.000000in}{0.000000in}}%
\pgfpathlineto{\pgfqpoint{-0.048611in}{0.000000in}}%
\pgfusepath{stroke,fill}%
}%
\begin{pgfscope}%
\pgfsys@transformshift{0.898769in}{2.223636in}%
\pgfsys@useobject{currentmarker}{}%
\end{pgfscope}%
\end{pgfscope}%
\begin{pgfscope}%
\definecolor{textcolor}{rgb}{0.000000,0.000000,0.000000}%
\pgfsetstrokecolor{textcolor}%
\pgfsetfillcolor{textcolor}%
\pgftext[x=0.572381in, y=2.166243in, left, base]{\color{textcolor}\rmfamily\fontsize{12.000000}{14.400000}\selectfont \(\displaystyle {10^{2}}\)}%
\end{pgfscope}%
\begin{pgfscope}%
\pgfsetbuttcap%
\pgfsetroundjoin%
\definecolor{currentfill}{rgb}{0.000000,0.000000,0.000000}%
\pgfsetfillcolor{currentfill}%
\pgfsetlinewidth{0.602250pt}%
\definecolor{currentstroke}{rgb}{0.000000,0.000000,0.000000}%
\pgfsetstrokecolor{currentstroke}%
\pgfsetdash{}{0pt}%
\pgfsys@defobject{currentmarker}{\pgfqpoint{-0.027778in}{0.000000in}}{\pgfqpoint{-0.000000in}{0.000000in}}{%
\pgfpathmoveto{\pgfqpoint{-0.000000in}{0.000000in}}%
\pgfpathlineto{\pgfqpoint{-0.027778in}{0.000000in}}%
\pgfusepath{stroke,fill}%
}%
\begin{pgfscope}%
\pgfsys@transformshift{0.898769in}{0.607031in}%
\pgfsys@useobject{currentmarker}{}%
\end{pgfscope}%
\end{pgfscope}%
\begin{pgfscope}%
\pgfsetbuttcap%
\pgfsetroundjoin%
\definecolor{currentfill}{rgb}{0.000000,0.000000,0.000000}%
\pgfsetfillcolor{currentfill}%
\pgfsetlinewidth{0.602250pt}%
\definecolor{currentstroke}{rgb}{0.000000,0.000000,0.000000}%
\pgfsetstrokecolor{currentstroke}%
\pgfsetdash{}{0pt}%
\pgfsys@defobject{currentmarker}{\pgfqpoint{-0.027778in}{0.000000in}}{\pgfqpoint{-0.000000in}{0.000000in}}{%
\pgfpathmoveto{\pgfqpoint{-0.000000in}{0.000000in}}%
\pgfpathlineto{\pgfqpoint{-0.027778in}{0.000000in}}%
\pgfusepath{stroke,fill}%
}%
\begin{pgfscope}%
\pgfsys@transformshift{0.898769in}{0.712505in}%
\pgfsys@useobject{currentmarker}{}%
\end{pgfscope}%
\end{pgfscope}%
\begin{pgfscope}%
\pgfsetbuttcap%
\pgfsetroundjoin%
\definecolor{currentfill}{rgb}{0.000000,0.000000,0.000000}%
\pgfsetfillcolor{currentfill}%
\pgfsetlinewidth{0.602250pt}%
\definecolor{currentstroke}{rgb}{0.000000,0.000000,0.000000}%
\pgfsetstrokecolor{currentstroke}%
\pgfsetdash{}{0pt}%
\pgfsys@defobject{currentmarker}{\pgfqpoint{-0.027778in}{0.000000in}}{\pgfqpoint{-0.000000in}{0.000000in}}{%
\pgfpathmoveto{\pgfqpoint{-0.000000in}{0.000000in}}%
\pgfpathlineto{\pgfqpoint{-0.027778in}{0.000000in}}%
\pgfusepath{stroke,fill}%
}%
\begin{pgfscope}%
\pgfsys@transformshift{0.898769in}{0.787339in}%
\pgfsys@useobject{currentmarker}{}%
\end{pgfscope}%
\end{pgfscope}%
\begin{pgfscope}%
\pgfsetbuttcap%
\pgfsetroundjoin%
\definecolor{currentfill}{rgb}{0.000000,0.000000,0.000000}%
\pgfsetfillcolor{currentfill}%
\pgfsetlinewidth{0.602250pt}%
\definecolor{currentstroke}{rgb}{0.000000,0.000000,0.000000}%
\pgfsetstrokecolor{currentstroke}%
\pgfsetdash{}{0pt}%
\pgfsys@defobject{currentmarker}{\pgfqpoint{-0.027778in}{0.000000in}}{\pgfqpoint{-0.000000in}{0.000000in}}{%
\pgfpathmoveto{\pgfqpoint{-0.000000in}{0.000000in}}%
\pgfpathlineto{\pgfqpoint{-0.027778in}{0.000000in}}%
\pgfusepath{stroke,fill}%
}%
\begin{pgfscope}%
\pgfsys@transformshift{0.898769in}{0.845386in}%
\pgfsys@useobject{currentmarker}{}%
\end{pgfscope}%
\end{pgfscope}%
\begin{pgfscope}%
\pgfsetbuttcap%
\pgfsetroundjoin%
\definecolor{currentfill}{rgb}{0.000000,0.000000,0.000000}%
\pgfsetfillcolor{currentfill}%
\pgfsetlinewidth{0.602250pt}%
\definecolor{currentstroke}{rgb}{0.000000,0.000000,0.000000}%
\pgfsetstrokecolor{currentstroke}%
\pgfsetdash{}{0pt}%
\pgfsys@defobject{currentmarker}{\pgfqpoint{-0.027778in}{0.000000in}}{\pgfqpoint{-0.000000in}{0.000000in}}{%
\pgfpathmoveto{\pgfqpoint{-0.000000in}{0.000000in}}%
\pgfpathlineto{\pgfqpoint{-0.027778in}{0.000000in}}%
\pgfusepath{stroke,fill}%
}%
\begin{pgfscope}%
\pgfsys@transformshift{0.898769in}{0.892813in}%
\pgfsys@useobject{currentmarker}{}%
\end{pgfscope}%
\end{pgfscope}%
\begin{pgfscope}%
\pgfsetbuttcap%
\pgfsetroundjoin%
\definecolor{currentfill}{rgb}{0.000000,0.000000,0.000000}%
\pgfsetfillcolor{currentfill}%
\pgfsetlinewidth{0.602250pt}%
\definecolor{currentstroke}{rgb}{0.000000,0.000000,0.000000}%
\pgfsetstrokecolor{currentstroke}%
\pgfsetdash{}{0pt}%
\pgfsys@defobject{currentmarker}{\pgfqpoint{-0.027778in}{0.000000in}}{\pgfqpoint{-0.000000in}{0.000000in}}{%
\pgfpathmoveto{\pgfqpoint{-0.000000in}{0.000000in}}%
\pgfpathlineto{\pgfqpoint{-0.027778in}{0.000000in}}%
\pgfusepath{stroke,fill}%
}%
\begin{pgfscope}%
\pgfsys@transformshift{0.898769in}{0.932912in}%
\pgfsys@useobject{currentmarker}{}%
\end{pgfscope}%
\end{pgfscope}%
\begin{pgfscope}%
\pgfsetbuttcap%
\pgfsetroundjoin%
\definecolor{currentfill}{rgb}{0.000000,0.000000,0.000000}%
\pgfsetfillcolor{currentfill}%
\pgfsetlinewidth{0.602250pt}%
\definecolor{currentstroke}{rgb}{0.000000,0.000000,0.000000}%
\pgfsetstrokecolor{currentstroke}%
\pgfsetdash{}{0pt}%
\pgfsys@defobject{currentmarker}{\pgfqpoint{-0.027778in}{0.000000in}}{\pgfqpoint{-0.000000in}{0.000000in}}{%
\pgfpathmoveto{\pgfqpoint{-0.000000in}{0.000000in}}%
\pgfpathlineto{\pgfqpoint{-0.027778in}{0.000000in}}%
\pgfusepath{stroke,fill}%
}%
\begin{pgfscope}%
\pgfsys@transformshift{0.898769in}{0.967648in}%
\pgfsys@useobject{currentmarker}{}%
\end{pgfscope}%
\end{pgfscope}%
\begin{pgfscope}%
\pgfsetbuttcap%
\pgfsetroundjoin%
\definecolor{currentfill}{rgb}{0.000000,0.000000,0.000000}%
\pgfsetfillcolor{currentfill}%
\pgfsetlinewidth{0.602250pt}%
\definecolor{currentstroke}{rgb}{0.000000,0.000000,0.000000}%
\pgfsetstrokecolor{currentstroke}%
\pgfsetdash{}{0pt}%
\pgfsys@defobject{currentmarker}{\pgfqpoint{-0.027778in}{0.000000in}}{\pgfqpoint{-0.000000in}{0.000000in}}{%
\pgfpathmoveto{\pgfqpoint{-0.000000in}{0.000000in}}%
\pgfpathlineto{\pgfqpoint{-0.027778in}{0.000000in}}%
\pgfusepath{stroke,fill}%
}%
\begin{pgfscope}%
\pgfsys@transformshift{0.898769in}{0.998287in}%
\pgfsys@useobject{currentmarker}{}%
\end{pgfscope}%
\end{pgfscope}%
\begin{pgfscope}%
\pgfsetbuttcap%
\pgfsetroundjoin%
\definecolor{currentfill}{rgb}{0.000000,0.000000,0.000000}%
\pgfsetfillcolor{currentfill}%
\pgfsetlinewidth{0.602250pt}%
\definecolor{currentstroke}{rgb}{0.000000,0.000000,0.000000}%
\pgfsetstrokecolor{currentstroke}%
\pgfsetdash{}{0pt}%
\pgfsys@defobject{currentmarker}{\pgfqpoint{-0.027778in}{0.000000in}}{\pgfqpoint{-0.000000in}{0.000000in}}{%
\pgfpathmoveto{\pgfqpoint{-0.000000in}{0.000000in}}%
\pgfpathlineto{\pgfqpoint{-0.027778in}{0.000000in}}%
\pgfusepath{stroke,fill}%
}%
\begin{pgfscope}%
\pgfsys@transformshift{0.898769in}{1.206002in}%
\pgfsys@useobject{currentmarker}{}%
\end{pgfscope}%
\end{pgfscope}%
\begin{pgfscope}%
\pgfsetbuttcap%
\pgfsetroundjoin%
\definecolor{currentfill}{rgb}{0.000000,0.000000,0.000000}%
\pgfsetfillcolor{currentfill}%
\pgfsetlinewidth{0.602250pt}%
\definecolor{currentstroke}{rgb}{0.000000,0.000000,0.000000}%
\pgfsetstrokecolor{currentstroke}%
\pgfsetdash{}{0pt}%
\pgfsys@defobject{currentmarker}{\pgfqpoint{-0.027778in}{0.000000in}}{\pgfqpoint{-0.000000in}{0.000000in}}{%
\pgfpathmoveto{\pgfqpoint{-0.000000in}{0.000000in}}%
\pgfpathlineto{\pgfqpoint{-0.027778in}{0.000000in}}%
\pgfusepath{stroke,fill}%
}%
\begin{pgfscope}%
\pgfsys@transformshift{0.898769in}{1.311476in}%
\pgfsys@useobject{currentmarker}{}%
\end{pgfscope}%
\end{pgfscope}%
\begin{pgfscope}%
\pgfsetbuttcap%
\pgfsetroundjoin%
\definecolor{currentfill}{rgb}{0.000000,0.000000,0.000000}%
\pgfsetfillcolor{currentfill}%
\pgfsetlinewidth{0.602250pt}%
\definecolor{currentstroke}{rgb}{0.000000,0.000000,0.000000}%
\pgfsetstrokecolor{currentstroke}%
\pgfsetdash{}{0pt}%
\pgfsys@defobject{currentmarker}{\pgfqpoint{-0.027778in}{0.000000in}}{\pgfqpoint{-0.000000in}{0.000000in}}{%
\pgfpathmoveto{\pgfqpoint{-0.000000in}{0.000000in}}%
\pgfpathlineto{\pgfqpoint{-0.027778in}{0.000000in}}%
\pgfusepath{stroke,fill}%
}%
\begin{pgfscope}%
\pgfsys@transformshift{0.898769in}{1.386310in}%
\pgfsys@useobject{currentmarker}{}%
\end{pgfscope}%
\end{pgfscope}%
\begin{pgfscope}%
\pgfsetbuttcap%
\pgfsetroundjoin%
\definecolor{currentfill}{rgb}{0.000000,0.000000,0.000000}%
\pgfsetfillcolor{currentfill}%
\pgfsetlinewidth{0.602250pt}%
\definecolor{currentstroke}{rgb}{0.000000,0.000000,0.000000}%
\pgfsetstrokecolor{currentstroke}%
\pgfsetdash{}{0pt}%
\pgfsys@defobject{currentmarker}{\pgfqpoint{-0.027778in}{0.000000in}}{\pgfqpoint{-0.000000in}{0.000000in}}{%
\pgfpathmoveto{\pgfqpoint{-0.000000in}{0.000000in}}%
\pgfpathlineto{\pgfqpoint{-0.027778in}{0.000000in}}%
\pgfusepath{stroke,fill}%
}%
\begin{pgfscope}%
\pgfsys@transformshift{0.898769in}{1.444357in}%
\pgfsys@useobject{currentmarker}{}%
\end{pgfscope}%
\end{pgfscope}%
\begin{pgfscope}%
\pgfsetbuttcap%
\pgfsetroundjoin%
\definecolor{currentfill}{rgb}{0.000000,0.000000,0.000000}%
\pgfsetfillcolor{currentfill}%
\pgfsetlinewidth{0.602250pt}%
\definecolor{currentstroke}{rgb}{0.000000,0.000000,0.000000}%
\pgfsetstrokecolor{currentstroke}%
\pgfsetdash{}{0pt}%
\pgfsys@defobject{currentmarker}{\pgfqpoint{-0.027778in}{0.000000in}}{\pgfqpoint{-0.000000in}{0.000000in}}{%
\pgfpathmoveto{\pgfqpoint{-0.000000in}{0.000000in}}%
\pgfpathlineto{\pgfqpoint{-0.027778in}{0.000000in}}%
\pgfusepath{stroke,fill}%
}%
\begin{pgfscope}%
\pgfsys@transformshift{0.898769in}{1.491784in}%
\pgfsys@useobject{currentmarker}{}%
\end{pgfscope}%
\end{pgfscope}%
\begin{pgfscope}%
\pgfsetbuttcap%
\pgfsetroundjoin%
\definecolor{currentfill}{rgb}{0.000000,0.000000,0.000000}%
\pgfsetfillcolor{currentfill}%
\pgfsetlinewidth{0.602250pt}%
\definecolor{currentstroke}{rgb}{0.000000,0.000000,0.000000}%
\pgfsetstrokecolor{currentstroke}%
\pgfsetdash{}{0pt}%
\pgfsys@defobject{currentmarker}{\pgfqpoint{-0.027778in}{0.000000in}}{\pgfqpoint{-0.000000in}{0.000000in}}{%
\pgfpathmoveto{\pgfqpoint{-0.000000in}{0.000000in}}%
\pgfpathlineto{\pgfqpoint{-0.027778in}{0.000000in}}%
\pgfusepath{stroke,fill}%
}%
\begin{pgfscope}%
\pgfsys@transformshift{0.898769in}{1.531883in}%
\pgfsys@useobject{currentmarker}{}%
\end{pgfscope}%
\end{pgfscope}%
\begin{pgfscope}%
\pgfsetbuttcap%
\pgfsetroundjoin%
\definecolor{currentfill}{rgb}{0.000000,0.000000,0.000000}%
\pgfsetfillcolor{currentfill}%
\pgfsetlinewidth{0.602250pt}%
\definecolor{currentstroke}{rgb}{0.000000,0.000000,0.000000}%
\pgfsetstrokecolor{currentstroke}%
\pgfsetdash{}{0pt}%
\pgfsys@defobject{currentmarker}{\pgfqpoint{-0.027778in}{0.000000in}}{\pgfqpoint{-0.000000in}{0.000000in}}{%
\pgfpathmoveto{\pgfqpoint{-0.000000in}{0.000000in}}%
\pgfpathlineto{\pgfqpoint{-0.027778in}{0.000000in}}%
\pgfusepath{stroke,fill}%
}%
\begin{pgfscope}%
\pgfsys@transformshift{0.898769in}{1.566619in}%
\pgfsys@useobject{currentmarker}{}%
\end{pgfscope}%
\end{pgfscope}%
\begin{pgfscope}%
\pgfsetbuttcap%
\pgfsetroundjoin%
\definecolor{currentfill}{rgb}{0.000000,0.000000,0.000000}%
\pgfsetfillcolor{currentfill}%
\pgfsetlinewidth{0.602250pt}%
\definecolor{currentstroke}{rgb}{0.000000,0.000000,0.000000}%
\pgfsetstrokecolor{currentstroke}%
\pgfsetdash{}{0pt}%
\pgfsys@defobject{currentmarker}{\pgfqpoint{-0.027778in}{0.000000in}}{\pgfqpoint{-0.000000in}{0.000000in}}{%
\pgfpathmoveto{\pgfqpoint{-0.000000in}{0.000000in}}%
\pgfpathlineto{\pgfqpoint{-0.027778in}{0.000000in}}%
\pgfusepath{stroke,fill}%
}%
\begin{pgfscope}%
\pgfsys@transformshift{0.898769in}{1.597258in}%
\pgfsys@useobject{currentmarker}{}%
\end{pgfscope}%
\end{pgfscope}%
\begin{pgfscope}%
\pgfsetbuttcap%
\pgfsetroundjoin%
\definecolor{currentfill}{rgb}{0.000000,0.000000,0.000000}%
\pgfsetfillcolor{currentfill}%
\pgfsetlinewidth{0.602250pt}%
\definecolor{currentstroke}{rgb}{0.000000,0.000000,0.000000}%
\pgfsetstrokecolor{currentstroke}%
\pgfsetdash{}{0pt}%
\pgfsys@defobject{currentmarker}{\pgfqpoint{-0.027778in}{0.000000in}}{\pgfqpoint{-0.000000in}{0.000000in}}{%
\pgfpathmoveto{\pgfqpoint{-0.000000in}{0.000000in}}%
\pgfpathlineto{\pgfqpoint{-0.027778in}{0.000000in}}%
\pgfusepath{stroke,fill}%
}%
\begin{pgfscope}%
\pgfsys@transformshift{0.898769in}{1.804973in}%
\pgfsys@useobject{currentmarker}{}%
\end{pgfscope}%
\end{pgfscope}%
\begin{pgfscope}%
\pgfsetbuttcap%
\pgfsetroundjoin%
\definecolor{currentfill}{rgb}{0.000000,0.000000,0.000000}%
\pgfsetfillcolor{currentfill}%
\pgfsetlinewidth{0.602250pt}%
\definecolor{currentstroke}{rgb}{0.000000,0.000000,0.000000}%
\pgfsetstrokecolor{currentstroke}%
\pgfsetdash{}{0pt}%
\pgfsys@defobject{currentmarker}{\pgfqpoint{-0.027778in}{0.000000in}}{\pgfqpoint{-0.000000in}{0.000000in}}{%
\pgfpathmoveto{\pgfqpoint{-0.000000in}{0.000000in}}%
\pgfpathlineto{\pgfqpoint{-0.027778in}{0.000000in}}%
\pgfusepath{stroke,fill}%
}%
\begin{pgfscope}%
\pgfsys@transformshift{0.898769in}{1.910447in}%
\pgfsys@useobject{currentmarker}{}%
\end{pgfscope}%
\end{pgfscope}%
\begin{pgfscope}%
\pgfsetbuttcap%
\pgfsetroundjoin%
\definecolor{currentfill}{rgb}{0.000000,0.000000,0.000000}%
\pgfsetfillcolor{currentfill}%
\pgfsetlinewidth{0.602250pt}%
\definecolor{currentstroke}{rgb}{0.000000,0.000000,0.000000}%
\pgfsetstrokecolor{currentstroke}%
\pgfsetdash{}{0pt}%
\pgfsys@defobject{currentmarker}{\pgfqpoint{-0.027778in}{0.000000in}}{\pgfqpoint{-0.000000in}{0.000000in}}{%
\pgfpathmoveto{\pgfqpoint{-0.000000in}{0.000000in}}%
\pgfpathlineto{\pgfqpoint{-0.027778in}{0.000000in}}%
\pgfusepath{stroke,fill}%
}%
\begin{pgfscope}%
\pgfsys@transformshift{0.898769in}{1.985281in}%
\pgfsys@useobject{currentmarker}{}%
\end{pgfscope}%
\end{pgfscope}%
\begin{pgfscope}%
\pgfsetbuttcap%
\pgfsetroundjoin%
\definecolor{currentfill}{rgb}{0.000000,0.000000,0.000000}%
\pgfsetfillcolor{currentfill}%
\pgfsetlinewidth{0.602250pt}%
\definecolor{currentstroke}{rgb}{0.000000,0.000000,0.000000}%
\pgfsetstrokecolor{currentstroke}%
\pgfsetdash{}{0pt}%
\pgfsys@defobject{currentmarker}{\pgfqpoint{-0.027778in}{0.000000in}}{\pgfqpoint{-0.000000in}{0.000000in}}{%
\pgfpathmoveto{\pgfqpoint{-0.000000in}{0.000000in}}%
\pgfpathlineto{\pgfqpoint{-0.027778in}{0.000000in}}%
\pgfusepath{stroke,fill}%
}%
\begin{pgfscope}%
\pgfsys@transformshift{0.898769in}{2.043328in}%
\pgfsys@useobject{currentmarker}{}%
\end{pgfscope}%
\end{pgfscope}%
\begin{pgfscope}%
\pgfsetbuttcap%
\pgfsetroundjoin%
\definecolor{currentfill}{rgb}{0.000000,0.000000,0.000000}%
\pgfsetfillcolor{currentfill}%
\pgfsetlinewidth{0.602250pt}%
\definecolor{currentstroke}{rgb}{0.000000,0.000000,0.000000}%
\pgfsetstrokecolor{currentstroke}%
\pgfsetdash{}{0pt}%
\pgfsys@defobject{currentmarker}{\pgfqpoint{-0.027778in}{0.000000in}}{\pgfqpoint{-0.000000in}{0.000000in}}{%
\pgfpathmoveto{\pgfqpoint{-0.000000in}{0.000000in}}%
\pgfpathlineto{\pgfqpoint{-0.027778in}{0.000000in}}%
\pgfusepath{stroke,fill}%
}%
\begin{pgfscope}%
\pgfsys@transformshift{0.898769in}{2.090755in}%
\pgfsys@useobject{currentmarker}{}%
\end{pgfscope}%
\end{pgfscope}%
\begin{pgfscope}%
\pgfsetbuttcap%
\pgfsetroundjoin%
\definecolor{currentfill}{rgb}{0.000000,0.000000,0.000000}%
\pgfsetfillcolor{currentfill}%
\pgfsetlinewidth{0.602250pt}%
\definecolor{currentstroke}{rgb}{0.000000,0.000000,0.000000}%
\pgfsetstrokecolor{currentstroke}%
\pgfsetdash{}{0pt}%
\pgfsys@defobject{currentmarker}{\pgfqpoint{-0.027778in}{0.000000in}}{\pgfqpoint{-0.000000in}{0.000000in}}{%
\pgfpathmoveto{\pgfqpoint{-0.000000in}{0.000000in}}%
\pgfpathlineto{\pgfqpoint{-0.027778in}{0.000000in}}%
\pgfusepath{stroke,fill}%
}%
\begin{pgfscope}%
\pgfsys@transformshift{0.898769in}{2.130854in}%
\pgfsys@useobject{currentmarker}{}%
\end{pgfscope}%
\end{pgfscope}%
\begin{pgfscope}%
\pgfsetbuttcap%
\pgfsetroundjoin%
\definecolor{currentfill}{rgb}{0.000000,0.000000,0.000000}%
\pgfsetfillcolor{currentfill}%
\pgfsetlinewidth{0.602250pt}%
\definecolor{currentstroke}{rgb}{0.000000,0.000000,0.000000}%
\pgfsetstrokecolor{currentstroke}%
\pgfsetdash{}{0pt}%
\pgfsys@defobject{currentmarker}{\pgfqpoint{-0.027778in}{0.000000in}}{\pgfqpoint{-0.000000in}{0.000000in}}{%
\pgfpathmoveto{\pgfqpoint{-0.000000in}{0.000000in}}%
\pgfpathlineto{\pgfqpoint{-0.027778in}{0.000000in}}%
\pgfusepath{stroke,fill}%
}%
\begin{pgfscope}%
\pgfsys@transformshift{0.898769in}{2.165590in}%
\pgfsys@useobject{currentmarker}{}%
\end{pgfscope}%
\end{pgfscope}%
\begin{pgfscope}%
\pgfsetbuttcap%
\pgfsetroundjoin%
\definecolor{currentfill}{rgb}{0.000000,0.000000,0.000000}%
\pgfsetfillcolor{currentfill}%
\pgfsetlinewidth{0.602250pt}%
\definecolor{currentstroke}{rgb}{0.000000,0.000000,0.000000}%
\pgfsetstrokecolor{currentstroke}%
\pgfsetdash{}{0pt}%
\pgfsys@defobject{currentmarker}{\pgfqpoint{-0.027778in}{0.000000in}}{\pgfqpoint{-0.000000in}{0.000000in}}{%
\pgfpathmoveto{\pgfqpoint{-0.000000in}{0.000000in}}%
\pgfpathlineto{\pgfqpoint{-0.027778in}{0.000000in}}%
\pgfusepath{stroke,fill}%
}%
\begin{pgfscope}%
\pgfsys@transformshift{0.898769in}{2.196229in}%
\pgfsys@useobject{currentmarker}{}%
\end{pgfscope}%
\end{pgfscope}%
\begin{pgfscope}%
\definecolor{textcolor}{rgb}{0.000000,0.000000,0.000000}%
\pgfsetstrokecolor{textcolor}%
\pgfsetfillcolor{textcolor}%
\pgftext[x=0.516826in,y=1.480952in,,bottom,rotate=90.000000]{\color{textcolor}\rmfamily\fontsize{12.000000}{14.400000}\selectfont \(\displaystyle \frac{\hat{\sigma}_{\gamma}}{\sqrt{\mathrm{CRB}(\gamma)}}\)}%
\end{pgfscope}%
\begin{pgfscope}%
\pgfpathrectangle{\pgfqpoint{0.898769in}{0.566590in}}{\pgfqpoint{3.137619in}{1.828724in}}%
\pgfusepath{clip}%
\pgfsetbuttcap%
\pgfsetroundjoin%
\pgfsetlinewidth{1.505625pt}%
\definecolor{currentstroke}{rgb}{0.000000,0.447000,0.741000}%
\pgfsetstrokecolor{currentstroke}%
\pgfsetdash{{5.550000pt}{2.400000pt}}{0.000000pt}%
\pgfpathmoveto{\pgfqpoint{0.898769in}{1.602562in}}%
\pgfpathlineto{\pgfqpoint{0.947794in}{1.641330in}}%
\pgfpathlineto{\pgfqpoint{0.996820in}{1.637010in}}%
\pgfpathlineto{\pgfqpoint{1.045845in}{1.712628in}}%
\pgfpathlineto{\pgfqpoint{1.094870in}{1.661175in}}%
\pgfpathlineto{\pgfqpoint{1.143896in}{1.697190in}}%
\pgfpathlineto{\pgfqpoint{1.192921in}{1.692758in}}%
\pgfpathlineto{\pgfqpoint{1.241946in}{1.797122in}}%
\pgfpathlineto{\pgfqpoint{1.290972in}{1.770711in}}%
\pgfpathlineto{\pgfqpoint{1.339997in}{1.786843in}}%
\pgfpathlineto{\pgfqpoint{1.389022in}{1.929733in}}%
\pgfpathlineto{\pgfqpoint{1.438047in}{1.819135in}}%
\pgfpathlineto{\pgfqpoint{1.487073in}{1.715112in}}%
\pgfpathlineto{\pgfqpoint{1.536098in}{1.838147in}}%
\pgfpathlineto{\pgfqpoint{1.585123in}{1.932463in}}%
\pgfpathlineto{\pgfqpoint{1.634149in}{1.856141in}}%
\pgfpathlineto{\pgfqpoint{1.683174in}{1.779595in}}%
\pgfpathlineto{\pgfqpoint{1.732199in}{1.889147in}}%
\pgfpathlineto{\pgfqpoint{1.781225in}{1.711386in}}%
\pgfpathlineto{\pgfqpoint{1.830250in}{1.770793in}}%
\pgfpathlineto{\pgfqpoint{1.879275in}{1.708396in}}%
\pgfpathlineto{\pgfqpoint{1.928301in}{1.309870in}}%
\pgfpathlineto{\pgfqpoint{1.977326in}{1.422412in}}%
\pgfpathlineto{\pgfqpoint{2.026351in}{1.125819in}}%
\pgfpathlineto{\pgfqpoint{2.075376in}{1.127797in}}%
\pgfpathlineto{\pgfqpoint{2.124402in}{1.062097in}}%
\pgfpathlineto{\pgfqpoint{2.173427in}{1.034229in}}%
\pgfpathlineto{\pgfqpoint{2.222452in}{1.078117in}}%
\pgfpathlineto{\pgfqpoint{2.271478in}{1.007053in}}%
\pgfpathlineto{\pgfqpoint{2.320503in}{1.040243in}}%
\pgfpathlineto{\pgfqpoint{2.369528in}{1.044986in}}%
\pgfpathlineto{\pgfqpoint{2.418554in}{1.039979in}}%
\pgfpathlineto{\pgfqpoint{2.467579in}{1.063470in}}%
\pgfpathlineto{\pgfqpoint{2.516604in}{1.016777in}}%
\pgfpathlineto{\pgfqpoint{2.565629in}{1.036177in}}%
\pgfpathlineto{\pgfqpoint{2.614655in}{1.027371in}}%
\pgfpathlineto{\pgfqpoint{2.663680in}{1.015520in}}%
\pgfpathlineto{\pgfqpoint{2.712705in}{1.014422in}}%
\pgfpathlineto{\pgfqpoint{2.761731in}{1.000913in}}%
\pgfpathlineto{\pgfqpoint{2.810756in}{1.018672in}}%
\pgfpathlineto{\pgfqpoint{2.859781in}{1.028153in}}%
\pgfpathlineto{\pgfqpoint{2.908807in}{1.010776in}}%
\pgfpathlineto{\pgfqpoint{2.957832in}{1.033531in}}%
\pgfpathlineto{\pgfqpoint{3.006857in}{1.013385in}}%
\pgfpathlineto{\pgfqpoint{3.055882in}{1.021283in}}%
\pgfpathlineto{\pgfqpoint{3.104908in}{1.037784in}}%
\pgfpathlineto{\pgfqpoint{3.153933in}{1.030525in}}%
\pgfpathlineto{\pgfqpoint{3.202958in}{1.023240in}}%
\pgfpathlineto{\pgfqpoint{3.251984in}{1.010970in}}%
\pgfpathlineto{\pgfqpoint{3.301009in}{1.036943in}}%
\pgfpathlineto{\pgfqpoint{3.350034in}{1.032333in}}%
\pgfpathlineto{\pgfqpoint{3.399060in}{1.017638in}}%
\pgfpathlineto{\pgfqpoint{3.448085in}{1.014480in}}%
\pgfpathlineto{\pgfqpoint{3.497110in}{1.009548in}}%
\pgfpathlineto{\pgfqpoint{3.546136in}{1.029555in}}%
\pgfpathlineto{\pgfqpoint{3.595161in}{1.020206in}}%
\pgfpathlineto{\pgfqpoint{3.644186in}{1.028098in}}%
\pgfpathlineto{\pgfqpoint{3.693211in}{1.029577in}}%
\pgfpathlineto{\pgfqpoint{3.742237in}{1.012451in}}%
\pgfpathlineto{\pgfqpoint{3.791262in}{1.020111in}}%
\pgfpathlineto{\pgfqpoint{3.840287in}{1.047838in}}%
\pgfpathlineto{\pgfqpoint{3.889313in}{1.012053in}}%
\pgfpathlineto{\pgfqpoint{3.938338in}{1.021186in}}%
\pgfpathlineto{\pgfqpoint{3.987363in}{1.008753in}}%
\pgfpathlineto{\pgfqpoint{4.036389in}{1.033222in}}%
\pgfusepath{stroke}%
\end{pgfscope}%
\begin{pgfscope}%
\pgfpathrectangle{\pgfqpoint{0.898769in}{0.566590in}}{\pgfqpoint{3.137619in}{1.828724in}}%
\pgfusepath{clip}%
\pgfsetbuttcap%
\pgfsetroundjoin%
\definecolor{currentfill}{rgb}{0.000000,0.000000,0.000000}%
\pgfsetfillcolor{currentfill}%
\pgfsetfillopacity{0.000000}%
\pgfsetlinewidth{1.003750pt}%
\definecolor{currentstroke}{rgb}{0.000000,0.447000,0.741000}%
\pgfsetstrokecolor{currentstroke}%
\pgfsetdash{}{0pt}%
\pgfsys@defobject{currentmarker}{\pgfqpoint{-0.041667in}{-0.041667in}}{\pgfqpoint{0.041667in}{0.041667in}}{%
\pgfpathmoveto{\pgfqpoint{0.000000in}{-0.041667in}}%
\pgfpathcurveto{\pgfqpoint{0.011050in}{-0.041667in}}{\pgfqpoint{0.021649in}{-0.037276in}}{\pgfqpoint{0.029463in}{-0.029463in}}%
\pgfpathcurveto{\pgfqpoint{0.037276in}{-0.021649in}}{\pgfqpoint{0.041667in}{-0.011050in}}{\pgfqpoint{0.041667in}{0.000000in}}%
\pgfpathcurveto{\pgfqpoint{0.041667in}{0.011050in}}{\pgfqpoint{0.037276in}{0.021649in}}{\pgfqpoint{0.029463in}{0.029463in}}%
\pgfpathcurveto{\pgfqpoint{0.021649in}{0.037276in}}{\pgfqpoint{0.011050in}{0.041667in}}{\pgfqpoint{0.000000in}{0.041667in}}%
\pgfpathcurveto{\pgfqpoint{-0.011050in}{0.041667in}}{\pgfqpoint{-0.021649in}{0.037276in}}{\pgfqpoint{-0.029463in}{0.029463in}}%
\pgfpathcurveto{\pgfqpoint{-0.037276in}{0.021649in}}{\pgfqpoint{-0.041667in}{0.011050in}}{\pgfqpoint{-0.041667in}{0.000000in}}%
\pgfpathcurveto{\pgfqpoint{-0.041667in}{-0.011050in}}{\pgfqpoint{-0.037276in}{-0.021649in}}{\pgfqpoint{-0.029463in}{-0.029463in}}%
\pgfpathcurveto{\pgfqpoint{-0.021649in}{-0.037276in}}{\pgfqpoint{-0.011050in}{-0.041667in}}{\pgfqpoint{0.000000in}{-0.041667in}}%
\pgfpathclose%
\pgfusepath{stroke,fill}%
}%
\begin{pgfscope}%
\pgfsys@transformshift{0.898769in}{1.602562in}%
\pgfsys@useobject{currentmarker}{}%
\end{pgfscope}%
\begin{pgfscope}%
\pgfsys@transformshift{1.094870in}{1.661175in}%
\pgfsys@useobject{currentmarker}{}%
\end{pgfscope}%
\begin{pgfscope}%
\pgfsys@transformshift{1.290972in}{1.770711in}%
\pgfsys@useobject{currentmarker}{}%
\end{pgfscope}%
\begin{pgfscope}%
\pgfsys@transformshift{1.487073in}{1.715112in}%
\pgfsys@useobject{currentmarker}{}%
\end{pgfscope}%
\begin{pgfscope}%
\pgfsys@transformshift{1.683174in}{1.779595in}%
\pgfsys@useobject{currentmarker}{}%
\end{pgfscope}%
\begin{pgfscope}%
\pgfsys@transformshift{1.879275in}{1.708396in}%
\pgfsys@useobject{currentmarker}{}%
\end{pgfscope}%
\begin{pgfscope}%
\pgfsys@transformshift{2.075376in}{1.127797in}%
\pgfsys@useobject{currentmarker}{}%
\end{pgfscope}%
\begin{pgfscope}%
\pgfsys@transformshift{2.271478in}{1.007053in}%
\pgfsys@useobject{currentmarker}{}%
\end{pgfscope}%
\begin{pgfscope}%
\pgfsys@transformshift{2.467579in}{1.063470in}%
\pgfsys@useobject{currentmarker}{}%
\end{pgfscope}%
\begin{pgfscope}%
\pgfsys@transformshift{2.663680in}{1.015520in}%
\pgfsys@useobject{currentmarker}{}%
\end{pgfscope}%
\begin{pgfscope}%
\pgfsys@transformshift{2.859781in}{1.028153in}%
\pgfsys@useobject{currentmarker}{}%
\end{pgfscope}%
\begin{pgfscope}%
\pgfsys@transformshift{3.055882in}{1.021283in}%
\pgfsys@useobject{currentmarker}{}%
\end{pgfscope}%
\begin{pgfscope}%
\pgfsys@transformshift{3.251984in}{1.010970in}%
\pgfsys@useobject{currentmarker}{}%
\end{pgfscope}%
\begin{pgfscope}%
\pgfsys@transformshift{3.448085in}{1.014480in}%
\pgfsys@useobject{currentmarker}{}%
\end{pgfscope}%
\begin{pgfscope}%
\pgfsys@transformshift{3.644186in}{1.028098in}%
\pgfsys@useobject{currentmarker}{}%
\end{pgfscope}%
\begin{pgfscope}%
\pgfsys@transformshift{3.840287in}{1.047838in}%
\pgfsys@useobject{currentmarker}{}%
\end{pgfscope}%
\begin{pgfscope}%
\pgfsys@transformshift{4.036389in}{1.033222in}%
\pgfsys@useobject{currentmarker}{}%
\end{pgfscope}%
\end{pgfscope}%
\begin{pgfscope}%
\pgfpathrectangle{\pgfqpoint{0.898769in}{0.566590in}}{\pgfqpoint{3.137619in}{1.828724in}}%
\pgfusepath{clip}%
\pgfsetbuttcap%
\pgfsetroundjoin%
\pgfsetlinewidth{1.505625pt}%
\definecolor{currentstroke}{rgb}{0.850000,0.324000,0.098000}%
\pgfsetstrokecolor{currentstroke}%
\pgfsetdash{{5.550000pt}{2.400000pt}}{0.000000pt}%
\pgfpathmoveto{\pgfqpoint{0.898769in}{1.917833in}}%
\pgfpathlineto{\pgfqpoint{0.947794in}{1.795944in}}%
\pgfpathlineto{\pgfqpoint{0.996820in}{1.829652in}}%
\pgfpathlineto{\pgfqpoint{1.045845in}{1.937721in}}%
\pgfpathlineto{\pgfqpoint{1.094870in}{1.603872in}}%
\pgfpathlineto{\pgfqpoint{1.143896in}{1.786475in}}%
\pgfpathlineto{\pgfqpoint{1.192921in}{1.706995in}}%
\pgfpathlineto{\pgfqpoint{1.241946in}{1.574427in}}%
\pgfpathlineto{\pgfqpoint{1.290972in}{1.698790in}}%
\pgfpathlineto{\pgfqpoint{1.339997in}{1.864107in}}%
\pgfpathlineto{\pgfqpoint{1.389022in}{1.289032in}}%
\pgfpathlineto{\pgfqpoint{1.438047in}{2.015922in}}%
\pgfpathlineto{\pgfqpoint{1.487073in}{1.238008in}}%
\pgfpathlineto{\pgfqpoint{1.536098in}{1.243963in}}%
\pgfpathlineto{\pgfqpoint{1.585123in}{1.211300in}}%
\pgfpathlineto{\pgfqpoint{1.634149in}{1.607271in}}%
\pgfpathlineto{\pgfqpoint{1.683174in}{1.184312in}}%
\pgfpathlineto{\pgfqpoint{1.732199in}{1.174973in}}%
\pgfpathlineto{\pgfqpoint{1.781225in}{1.162608in}}%
\pgfpathlineto{\pgfqpoint{1.830250in}{1.107111in}}%
\pgfpathlineto{\pgfqpoint{1.879275in}{1.098868in}}%
\pgfpathlineto{\pgfqpoint{1.928301in}{1.015107in}}%
\pgfpathlineto{\pgfqpoint{1.977326in}{1.050918in}}%
\pgfpathlineto{\pgfqpoint{2.026351in}{1.069795in}}%
\pgfpathlineto{\pgfqpoint{2.075376in}{1.023993in}}%
\pgfpathlineto{\pgfqpoint{2.124402in}{1.007280in}}%
\pgfpathlineto{\pgfqpoint{2.173427in}{1.028914in}}%
\pgfpathlineto{\pgfqpoint{2.222452in}{1.012224in}}%
\pgfpathlineto{\pgfqpoint{2.271478in}{1.046274in}}%
\pgfpathlineto{\pgfqpoint{2.320503in}{1.016680in}}%
\pgfpathlineto{\pgfqpoint{2.369528in}{1.035670in}}%
\pgfpathlineto{\pgfqpoint{2.418554in}{1.041008in}}%
\pgfpathlineto{\pgfqpoint{2.467579in}{1.039769in}}%
\pgfpathlineto{\pgfqpoint{2.516604in}{1.018094in}}%
\pgfpathlineto{\pgfqpoint{2.565629in}{1.035591in}}%
\pgfpathlineto{\pgfqpoint{2.614655in}{1.005596in}}%
\pgfpathlineto{\pgfqpoint{2.663680in}{1.043934in}}%
\pgfpathlineto{\pgfqpoint{2.712705in}{1.012120in}}%
\pgfpathlineto{\pgfqpoint{2.761731in}{1.021819in}}%
\pgfpathlineto{\pgfqpoint{2.810756in}{1.037583in}}%
\pgfpathlineto{\pgfqpoint{2.859781in}{1.021401in}}%
\pgfpathlineto{\pgfqpoint{2.908807in}{1.037263in}}%
\pgfpathlineto{\pgfqpoint{2.957832in}{1.032648in}}%
\pgfpathlineto{\pgfqpoint{3.006857in}{1.027381in}}%
\pgfpathlineto{\pgfqpoint{3.055882in}{1.010829in}}%
\pgfpathlineto{\pgfqpoint{3.104908in}{1.005923in}}%
\pgfpathlineto{\pgfqpoint{3.153933in}{1.012204in}}%
\pgfpathlineto{\pgfqpoint{3.202958in}{1.021817in}}%
\pgfpathlineto{\pgfqpoint{3.251984in}{1.031816in}}%
\pgfpathlineto{\pgfqpoint{3.301009in}{1.011613in}}%
\pgfpathlineto{\pgfqpoint{3.350034in}{1.036668in}}%
\pgfpathlineto{\pgfqpoint{3.399060in}{1.023538in}}%
\pgfpathlineto{\pgfqpoint{3.448085in}{1.018463in}}%
\pgfpathlineto{\pgfqpoint{3.497110in}{1.041147in}}%
\pgfpathlineto{\pgfqpoint{3.546136in}{1.040264in}}%
\pgfpathlineto{\pgfqpoint{3.595161in}{1.016748in}}%
\pgfpathlineto{\pgfqpoint{3.644186in}{1.028907in}}%
\pgfpathlineto{\pgfqpoint{3.693211in}{1.041557in}}%
\pgfpathlineto{\pgfqpoint{3.742237in}{1.030201in}}%
\pgfpathlineto{\pgfqpoint{3.791262in}{1.028136in}}%
\pgfpathlineto{\pgfqpoint{3.840287in}{1.029382in}}%
\pgfpathlineto{\pgfqpoint{3.889313in}{1.038896in}}%
\pgfpathlineto{\pgfqpoint{3.938338in}{1.004998in}}%
\pgfpathlineto{\pgfqpoint{3.987363in}{1.026794in}}%
\pgfpathlineto{\pgfqpoint{4.036389in}{1.042608in}}%
\pgfusepath{stroke}%
\end{pgfscope}%
\begin{pgfscope}%
\pgfpathrectangle{\pgfqpoint{0.898769in}{0.566590in}}{\pgfqpoint{3.137619in}{1.828724in}}%
\pgfusepath{clip}%
\pgfsetbuttcap%
\pgfsetroundjoin%
\definecolor{currentfill}{rgb}{0.850000,0.324000,0.098000}%
\pgfsetfillcolor{currentfill}%
\pgfsetlinewidth{1.003750pt}%
\definecolor{currentstroke}{rgb}{0.850000,0.324000,0.098000}%
\pgfsetstrokecolor{currentstroke}%
\pgfsetdash{}{0pt}%
\pgfsys@defobject{currentmarker}{\pgfqpoint{-0.041667in}{-0.041667in}}{\pgfqpoint{0.041667in}{0.041667in}}{%
\pgfpathmoveto{\pgfqpoint{-0.041667in}{0.000000in}}%
\pgfpathlineto{\pgfqpoint{0.041667in}{0.000000in}}%
\pgfpathmoveto{\pgfqpoint{0.000000in}{-0.041667in}}%
\pgfpathlineto{\pgfqpoint{0.000000in}{0.041667in}}%
\pgfusepath{stroke,fill}%
}%
\begin{pgfscope}%
\pgfsys@transformshift{0.898769in}{1.917833in}%
\pgfsys@useobject{currentmarker}{}%
\end{pgfscope}%
\begin{pgfscope}%
\pgfsys@transformshift{1.045845in}{1.937721in}%
\pgfsys@useobject{currentmarker}{}%
\end{pgfscope}%
\begin{pgfscope}%
\pgfsys@transformshift{1.192921in}{1.706995in}%
\pgfsys@useobject{currentmarker}{}%
\end{pgfscope}%
\begin{pgfscope}%
\pgfsys@transformshift{1.339997in}{1.864107in}%
\pgfsys@useobject{currentmarker}{}%
\end{pgfscope}%
\begin{pgfscope}%
\pgfsys@transformshift{1.487073in}{1.238008in}%
\pgfsys@useobject{currentmarker}{}%
\end{pgfscope}%
\begin{pgfscope}%
\pgfsys@transformshift{1.634149in}{1.607271in}%
\pgfsys@useobject{currentmarker}{}%
\end{pgfscope}%
\begin{pgfscope}%
\pgfsys@transformshift{1.781225in}{1.162608in}%
\pgfsys@useobject{currentmarker}{}%
\end{pgfscope}%
\begin{pgfscope}%
\pgfsys@transformshift{1.928301in}{1.015107in}%
\pgfsys@useobject{currentmarker}{}%
\end{pgfscope}%
\begin{pgfscope}%
\pgfsys@transformshift{2.075376in}{1.023993in}%
\pgfsys@useobject{currentmarker}{}%
\end{pgfscope}%
\begin{pgfscope}%
\pgfsys@transformshift{2.222452in}{1.012224in}%
\pgfsys@useobject{currentmarker}{}%
\end{pgfscope}%
\begin{pgfscope}%
\pgfsys@transformshift{2.369528in}{1.035670in}%
\pgfsys@useobject{currentmarker}{}%
\end{pgfscope}%
\begin{pgfscope}%
\pgfsys@transformshift{2.516604in}{1.018094in}%
\pgfsys@useobject{currentmarker}{}%
\end{pgfscope}%
\begin{pgfscope}%
\pgfsys@transformshift{2.663680in}{1.043934in}%
\pgfsys@useobject{currentmarker}{}%
\end{pgfscope}%
\begin{pgfscope}%
\pgfsys@transformshift{2.810756in}{1.037583in}%
\pgfsys@useobject{currentmarker}{}%
\end{pgfscope}%
\begin{pgfscope}%
\pgfsys@transformshift{2.957832in}{1.032648in}%
\pgfsys@useobject{currentmarker}{}%
\end{pgfscope}%
\begin{pgfscope}%
\pgfsys@transformshift{3.104908in}{1.005923in}%
\pgfsys@useobject{currentmarker}{}%
\end{pgfscope}%
\begin{pgfscope}%
\pgfsys@transformshift{3.251984in}{1.031816in}%
\pgfsys@useobject{currentmarker}{}%
\end{pgfscope}%
\begin{pgfscope}%
\pgfsys@transformshift{3.399060in}{1.023538in}%
\pgfsys@useobject{currentmarker}{}%
\end{pgfscope}%
\begin{pgfscope}%
\pgfsys@transformshift{3.546136in}{1.040264in}%
\pgfsys@useobject{currentmarker}{}%
\end{pgfscope}%
\begin{pgfscope}%
\pgfsys@transformshift{3.693211in}{1.041557in}%
\pgfsys@useobject{currentmarker}{}%
\end{pgfscope}%
\begin{pgfscope}%
\pgfsys@transformshift{3.840287in}{1.029382in}%
\pgfsys@useobject{currentmarker}{}%
\end{pgfscope}%
\begin{pgfscope}%
\pgfsys@transformshift{3.987363in}{1.026794in}%
\pgfsys@useobject{currentmarker}{}%
\end{pgfscope}%
\end{pgfscope}%
\begin{pgfscope}%
\pgfpathrectangle{\pgfqpoint{0.898769in}{0.566590in}}{\pgfqpoint{3.137619in}{1.828724in}}%
\pgfusepath{clip}%
\pgfsetbuttcap%
\pgfsetroundjoin%
\pgfsetlinewidth{1.505625pt}%
\definecolor{currentstroke}{rgb}{0.000000,0.500000,0.000000}%
\pgfsetstrokecolor{currentstroke}%
\pgfsetdash{{5.550000pt}{2.400000pt}}{0.000000pt}%
\pgfpathmoveto{\pgfqpoint{0.898769in}{1.936795in}}%
\pgfpathlineto{\pgfqpoint{0.947794in}{1.994268in}}%
\pgfpathlineto{\pgfqpoint{0.996820in}{2.047468in}}%
\pgfpathlineto{\pgfqpoint{1.045845in}{1.912134in}}%
\pgfpathlineto{\pgfqpoint{1.094870in}{2.022809in}}%
\pgfpathlineto{\pgfqpoint{1.143896in}{2.034707in}}%
\pgfpathlineto{\pgfqpoint{1.192921in}{2.079961in}}%
\pgfpathlineto{\pgfqpoint{1.241946in}{2.054164in}}%
\pgfpathlineto{\pgfqpoint{1.290972in}{2.126447in}}%
\pgfpathlineto{\pgfqpoint{1.339997in}{2.078384in}}%
\pgfpathlineto{\pgfqpoint{1.389022in}{2.013279in}}%
\pgfpathlineto{\pgfqpoint{1.438047in}{2.022442in}}%
\pgfpathlineto{\pgfqpoint{1.487073in}{2.101025in}}%
\pgfpathlineto{\pgfqpoint{1.536098in}{2.082985in}}%
\pgfpathlineto{\pgfqpoint{1.585123in}{2.115597in}}%
\pgfpathlineto{\pgfqpoint{1.634149in}{2.125543in}}%
\pgfpathlineto{\pgfqpoint{1.683174in}{2.164809in}}%
\pgfpathlineto{\pgfqpoint{1.732199in}{2.091205in}}%
\pgfpathlineto{\pgfqpoint{1.781225in}{1.815226in}}%
\pgfpathlineto{\pgfqpoint{1.830250in}{1.920649in}}%
\pgfpathlineto{\pgfqpoint{1.879275in}{1.763663in}}%
\pgfpathlineto{\pgfqpoint{1.928301in}{1.051786in}}%
\pgfpathlineto{\pgfqpoint{1.977326in}{1.094556in}}%
\pgfpathlineto{\pgfqpoint{2.026351in}{1.122459in}}%
\pgfpathlineto{\pgfqpoint{2.075376in}{1.020795in}}%
\pgfpathlineto{\pgfqpoint{2.124402in}{1.007709in}}%
\pgfpathlineto{\pgfqpoint{2.173427in}{1.041049in}}%
\pgfpathlineto{\pgfqpoint{2.222452in}{1.045856in}}%
\pgfpathlineto{\pgfqpoint{2.271478in}{1.022111in}}%
\pgfpathlineto{\pgfqpoint{2.320503in}{1.020010in}}%
\pgfpathlineto{\pgfqpoint{2.369528in}{1.030937in}}%
\pgfpathlineto{\pgfqpoint{2.418554in}{1.024232in}}%
\pgfpathlineto{\pgfqpoint{2.467579in}{1.014417in}}%
\pgfpathlineto{\pgfqpoint{2.516604in}{1.025858in}}%
\pgfpathlineto{\pgfqpoint{2.565629in}{1.026131in}}%
\pgfpathlineto{\pgfqpoint{2.614655in}{1.033839in}}%
\pgfpathlineto{\pgfqpoint{2.663680in}{1.018034in}}%
\pgfpathlineto{\pgfqpoint{2.712705in}{1.019865in}}%
\pgfpathlineto{\pgfqpoint{2.761731in}{1.024033in}}%
\pgfpathlineto{\pgfqpoint{2.810756in}{1.016633in}}%
\pgfpathlineto{\pgfqpoint{2.859781in}{1.000899in}}%
\pgfpathlineto{\pgfqpoint{2.908807in}{1.034796in}}%
\pgfpathlineto{\pgfqpoint{2.957832in}{1.015741in}}%
\pgfpathlineto{\pgfqpoint{3.006857in}{1.038016in}}%
\pgfpathlineto{\pgfqpoint{3.055882in}{1.018465in}}%
\pgfpathlineto{\pgfqpoint{3.104908in}{1.016186in}}%
\pgfpathlineto{\pgfqpoint{3.153933in}{1.034171in}}%
\pgfpathlineto{\pgfqpoint{3.202958in}{1.012976in}}%
\pgfpathlineto{\pgfqpoint{3.251984in}{1.023133in}}%
\pgfpathlineto{\pgfqpoint{3.301009in}{1.007173in}}%
\pgfpathlineto{\pgfqpoint{3.350034in}{1.017554in}}%
\pgfpathlineto{\pgfqpoint{3.399060in}{1.034056in}}%
\pgfpathlineto{\pgfqpoint{3.448085in}{1.003821in}}%
\pgfpathlineto{\pgfqpoint{3.497110in}{1.026836in}}%
\pgfpathlineto{\pgfqpoint{3.546136in}{1.015495in}}%
\pgfpathlineto{\pgfqpoint{3.595161in}{0.979774in}}%
\pgfpathlineto{\pgfqpoint{3.644186in}{1.036330in}}%
\pgfpathlineto{\pgfqpoint{3.693211in}{1.038616in}}%
\pgfpathlineto{\pgfqpoint{3.742237in}{1.035249in}}%
\pgfpathlineto{\pgfqpoint{3.791262in}{1.031424in}}%
\pgfpathlineto{\pgfqpoint{3.840287in}{1.032122in}}%
\pgfpathlineto{\pgfqpoint{3.889313in}{1.015423in}}%
\pgfpathlineto{\pgfqpoint{3.938338in}{1.023068in}}%
\pgfpathlineto{\pgfqpoint{3.987363in}{1.054328in}}%
\pgfpathlineto{\pgfqpoint{4.036389in}{1.014605in}}%
\pgfusepath{stroke}%
\end{pgfscope}%
\begin{pgfscope}%
\pgfpathrectangle{\pgfqpoint{0.898769in}{0.566590in}}{\pgfqpoint{3.137619in}{1.828724in}}%
\pgfusepath{clip}%
\pgfsetbuttcap%
\pgfsetmiterjoin%
\definecolor{currentfill}{rgb}{0.000000,0.000000,0.000000}%
\pgfsetfillcolor{currentfill}%
\pgfsetfillopacity{0.000000}%
\pgfsetlinewidth{1.003750pt}%
\definecolor{currentstroke}{rgb}{0.000000,0.500000,0.000000}%
\pgfsetstrokecolor{currentstroke}%
\pgfsetdash{}{0pt}%
\pgfsys@defobject{currentmarker}{\pgfqpoint{-0.041667in}{-0.041667in}}{\pgfqpoint{0.041667in}{0.041667in}}{%
\pgfpathmoveto{\pgfqpoint{-0.041667in}{-0.041667in}}%
\pgfpathlineto{\pgfqpoint{0.041667in}{-0.041667in}}%
\pgfpathlineto{\pgfqpoint{0.041667in}{0.041667in}}%
\pgfpathlineto{\pgfqpoint{-0.041667in}{0.041667in}}%
\pgfpathclose%
\pgfusepath{stroke,fill}%
}%
\begin{pgfscope}%
\pgfsys@transformshift{0.898769in}{1.936795in}%
\pgfsys@useobject{currentmarker}{}%
\end{pgfscope}%
\begin{pgfscope}%
\pgfsys@transformshift{1.143896in}{2.034707in}%
\pgfsys@useobject{currentmarker}{}%
\end{pgfscope}%
\begin{pgfscope}%
\pgfsys@transformshift{1.389022in}{2.013279in}%
\pgfsys@useobject{currentmarker}{}%
\end{pgfscope}%
\begin{pgfscope}%
\pgfsys@transformshift{1.634149in}{2.125543in}%
\pgfsys@useobject{currentmarker}{}%
\end{pgfscope}%
\begin{pgfscope}%
\pgfsys@transformshift{1.879275in}{1.763663in}%
\pgfsys@useobject{currentmarker}{}%
\end{pgfscope}%
\begin{pgfscope}%
\pgfsys@transformshift{2.124402in}{1.007709in}%
\pgfsys@useobject{currentmarker}{}%
\end{pgfscope}%
\begin{pgfscope}%
\pgfsys@transformshift{2.369528in}{1.030937in}%
\pgfsys@useobject{currentmarker}{}%
\end{pgfscope}%
\begin{pgfscope}%
\pgfsys@transformshift{2.614655in}{1.033839in}%
\pgfsys@useobject{currentmarker}{}%
\end{pgfscope}%
\begin{pgfscope}%
\pgfsys@transformshift{2.859781in}{1.000899in}%
\pgfsys@useobject{currentmarker}{}%
\end{pgfscope}%
\begin{pgfscope}%
\pgfsys@transformshift{3.104908in}{1.016186in}%
\pgfsys@useobject{currentmarker}{}%
\end{pgfscope}%
\begin{pgfscope}%
\pgfsys@transformshift{3.350034in}{1.017554in}%
\pgfsys@useobject{currentmarker}{}%
\end{pgfscope}%
\begin{pgfscope}%
\pgfsys@transformshift{3.595161in}{0.979774in}%
\pgfsys@useobject{currentmarker}{}%
\end{pgfscope}%
\begin{pgfscope}%
\pgfsys@transformshift{3.840287in}{1.032122in}%
\pgfsys@useobject{currentmarker}{}%
\end{pgfscope}%
\end{pgfscope}%
\begin{pgfscope}%
\pgfpathrectangle{\pgfqpoint{0.898769in}{0.566590in}}{\pgfqpoint{3.137619in}{1.828724in}}%
\pgfusepath{clip}%
\pgfsetbuttcap%
\pgfsetroundjoin%
\pgfsetlinewidth{1.505625pt}%
\definecolor{currentstroke}{rgb}{0.494000,0.184000,0.556000}%
\pgfsetstrokecolor{currentstroke}%
\pgfsetdash{{5.550000pt}{2.400000pt}}{0.000000pt}%
\pgfpathmoveto{\pgfqpoint{0.898769in}{2.002024in}}%
\pgfpathlineto{\pgfqpoint{0.947794in}{2.107559in}}%
\pgfpathlineto{\pgfqpoint{0.996820in}{2.124084in}}%
\pgfpathlineto{\pgfqpoint{1.045845in}{1.839288in}}%
\pgfpathlineto{\pgfqpoint{1.094870in}{2.133651in}}%
\pgfpathlineto{\pgfqpoint{1.143896in}{1.993338in}}%
\pgfpathlineto{\pgfqpoint{1.192921in}{2.108448in}}%
\pgfpathlineto{\pgfqpoint{1.241946in}{2.139075in}}%
\pgfpathlineto{\pgfqpoint{1.290972in}{2.138705in}}%
\pgfpathlineto{\pgfqpoint{1.339997in}{2.105518in}}%
\pgfpathlineto{\pgfqpoint{1.389022in}{2.111674in}}%
\pgfpathlineto{\pgfqpoint{1.438047in}{2.260343in}}%
\pgfpathlineto{\pgfqpoint{1.487073in}{1.891846in}}%
\pgfpathlineto{\pgfqpoint{1.536098in}{2.312190in}}%
\pgfpathlineto{\pgfqpoint{1.585123in}{2.075595in}}%
\pgfpathlineto{\pgfqpoint{1.634149in}{1.929281in}}%
\pgfpathlineto{\pgfqpoint{1.683174in}{1.894612in}}%
\pgfpathlineto{\pgfqpoint{1.732199in}{2.030832in}}%
\pgfpathlineto{\pgfqpoint{1.781225in}{1.370924in}}%
\pgfpathlineto{\pgfqpoint{1.830250in}{1.032514in}}%
\pgfpathlineto{\pgfqpoint{1.879275in}{1.626173in}}%
\pgfpathlineto{\pgfqpoint{1.928301in}{1.023752in}}%
\pgfpathlineto{\pgfqpoint{1.977326in}{1.044473in}}%
\pgfpathlineto{\pgfqpoint{2.026351in}{1.044217in}}%
\pgfpathlineto{\pgfqpoint{2.075376in}{1.047057in}}%
\pgfpathlineto{\pgfqpoint{2.124402in}{1.052655in}}%
\pgfpathlineto{\pgfqpoint{2.173427in}{1.034921in}}%
\pgfpathlineto{\pgfqpoint{2.222452in}{1.012282in}}%
\pgfpathlineto{\pgfqpoint{2.271478in}{1.036971in}}%
\pgfpathlineto{\pgfqpoint{2.320503in}{1.042082in}}%
\pgfpathlineto{\pgfqpoint{2.369528in}{1.047817in}}%
\pgfpathlineto{\pgfqpoint{2.418554in}{1.033597in}}%
\pgfpathlineto{\pgfqpoint{2.467579in}{1.033311in}}%
\pgfpathlineto{\pgfqpoint{2.516604in}{1.004754in}}%
\pgfpathlineto{\pgfqpoint{2.565629in}{1.035769in}}%
\pgfpathlineto{\pgfqpoint{2.614655in}{1.017461in}}%
\pgfpathlineto{\pgfqpoint{2.663680in}{1.035109in}}%
\pgfpathlineto{\pgfqpoint{2.712705in}{1.028814in}}%
\pgfpathlineto{\pgfqpoint{2.761731in}{0.989781in}}%
\pgfpathlineto{\pgfqpoint{2.810756in}{1.014361in}}%
\pgfpathlineto{\pgfqpoint{2.859781in}{1.006457in}}%
\pgfpathlineto{\pgfqpoint{2.908807in}{1.022767in}}%
\pgfpathlineto{\pgfqpoint{2.957832in}{1.049570in}}%
\pgfpathlineto{\pgfqpoint{3.006857in}{1.033499in}}%
\pgfpathlineto{\pgfqpoint{3.055882in}{1.031215in}}%
\pgfpathlineto{\pgfqpoint{3.104908in}{1.021620in}}%
\pgfpathlineto{\pgfqpoint{3.153933in}{1.028212in}}%
\pgfpathlineto{\pgfqpoint{3.202958in}{1.027293in}}%
\pgfpathlineto{\pgfqpoint{3.251984in}{1.026820in}}%
\pgfpathlineto{\pgfqpoint{3.301009in}{1.036674in}}%
\pgfpathlineto{\pgfqpoint{3.350034in}{1.025053in}}%
\pgfpathlineto{\pgfqpoint{3.399060in}{1.011796in}}%
\pgfpathlineto{\pgfqpoint{3.448085in}{1.023278in}}%
\pgfpathlineto{\pgfqpoint{3.497110in}{1.017748in}}%
\pgfpathlineto{\pgfqpoint{3.546136in}{1.021196in}}%
\pgfpathlineto{\pgfqpoint{3.595161in}{1.053406in}}%
\pgfpathlineto{\pgfqpoint{3.644186in}{1.002812in}}%
\pgfpathlineto{\pgfqpoint{3.693211in}{1.018099in}}%
\pgfpathlineto{\pgfqpoint{3.742237in}{1.035864in}}%
\pgfpathlineto{\pgfqpoint{3.791262in}{1.028100in}}%
\pgfpathlineto{\pgfqpoint{3.840287in}{1.026601in}}%
\pgfpathlineto{\pgfqpoint{3.889313in}{1.017521in}}%
\pgfpathlineto{\pgfqpoint{3.938338in}{0.995967in}}%
\pgfpathlineto{\pgfqpoint{3.987363in}{1.033562in}}%
\pgfpathlineto{\pgfqpoint{4.036389in}{1.001845in}}%
\pgfusepath{stroke}%
\end{pgfscope}%
\begin{pgfscope}%
\pgfpathrectangle{\pgfqpoint{0.898769in}{0.566590in}}{\pgfqpoint{3.137619in}{1.828724in}}%
\pgfusepath{clip}%
\pgfsetbuttcap%
\pgfsetroundjoin%
\definecolor{currentfill}{rgb}{0.494000,0.184000,0.556000}%
\pgfsetfillcolor{currentfill}%
\pgfsetlinewidth{1.003750pt}%
\definecolor{currentstroke}{rgb}{0.494000,0.184000,0.556000}%
\pgfsetstrokecolor{currentstroke}%
\pgfsetdash{}{0pt}%
\pgfsys@defobject{currentmarker}{\pgfqpoint{-0.041667in}{-0.041667in}}{\pgfqpoint{0.041667in}{0.041667in}}{%
\pgfpathmoveto{\pgfqpoint{-0.041667in}{-0.041667in}}%
\pgfpathlineto{\pgfqpoint{0.041667in}{0.041667in}}%
\pgfpathmoveto{\pgfqpoint{-0.041667in}{0.041667in}}%
\pgfpathlineto{\pgfqpoint{0.041667in}{-0.041667in}}%
\pgfusepath{stroke,fill}%
}%
\begin{pgfscope}%
\pgfsys@transformshift{0.898769in}{2.002024in}%
\pgfsys@useobject{currentmarker}{}%
\end{pgfscope}%
\begin{pgfscope}%
\pgfsys@transformshift{1.094870in}{2.133651in}%
\pgfsys@useobject{currentmarker}{}%
\end{pgfscope}%
\begin{pgfscope}%
\pgfsys@transformshift{1.290972in}{2.138705in}%
\pgfsys@useobject{currentmarker}{}%
\end{pgfscope}%
\begin{pgfscope}%
\pgfsys@transformshift{1.487073in}{1.891846in}%
\pgfsys@useobject{currentmarker}{}%
\end{pgfscope}%
\begin{pgfscope}%
\pgfsys@transformshift{1.683174in}{1.894612in}%
\pgfsys@useobject{currentmarker}{}%
\end{pgfscope}%
\begin{pgfscope}%
\pgfsys@transformshift{1.879275in}{1.626173in}%
\pgfsys@useobject{currentmarker}{}%
\end{pgfscope}%
\begin{pgfscope}%
\pgfsys@transformshift{2.075376in}{1.047057in}%
\pgfsys@useobject{currentmarker}{}%
\end{pgfscope}%
\begin{pgfscope}%
\pgfsys@transformshift{2.271478in}{1.036971in}%
\pgfsys@useobject{currentmarker}{}%
\end{pgfscope}%
\begin{pgfscope}%
\pgfsys@transformshift{2.467579in}{1.033311in}%
\pgfsys@useobject{currentmarker}{}%
\end{pgfscope}%
\begin{pgfscope}%
\pgfsys@transformshift{2.663680in}{1.035109in}%
\pgfsys@useobject{currentmarker}{}%
\end{pgfscope}%
\begin{pgfscope}%
\pgfsys@transformshift{2.859781in}{1.006457in}%
\pgfsys@useobject{currentmarker}{}%
\end{pgfscope}%
\begin{pgfscope}%
\pgfsys@transformshift{3.055882in}{1.031215in}%
\pgfsys@useobject{currentmarker}{}%
\end{pgfscope}%
\begin{pgfscope}%
\pgfsys@transformshift{3.251984in}{1.026820in}%
\pgfsys@useobject{currentmarker}{}%
\end{pgfscope}%
\begin{pgfscope}%
\pgfsys@transformshift{3.448085in}{1.023278in}%
\pgfsys@useobject{currentmarker}{}%
\end{pgfscope}%
\begin{pgfscope}%
\pgfsys@transformshift{3.644186in}{1.002812in}%
\pgfsys@useobject{currentmarker}{}%
\end{pgfscope}%
\begin{pgfscope}%
\pgfsys@transformshift{3.840287in}{1.026601in}%
\pgfsys@useobject{currentmarker}{}%
\end{pgfscope}%
\begin{pgfscope}%
\pgfsys@transformshift{4.036389in}{1.001845in}%
\pgfsys@useobject{currentmarker}{}%
\end{pgfscope}%
\end{pgfscope}%
\begin{pgfscope}%
\pgfpathrectangle{\pgfqpoint{0.898769in}{0.566590in}}{\pgfqpoint{3.137619in}{1.828724in}}%
\pgfusepath{clip}%
\pgfsetrectcap%
\pgfsetroundjoin%
\pgfsetlinewidth{1.505625pt}%
\definecolor{currentstroke}{rgb}{0.000000,0.447000,0.741000}%
\pgfsetstrokecolor{currentstroke}%
\pgfsetdash{}{0pt}%
\pgfpathmoveto{\pgfqpoint{0.898769in}{1.289313in}}%
\pgfpathlineto{\pgfqpoint{0.947794in}{1.308824in}}%
\pgfpathlineto{\pgfqpoint{0.996820in}{1.302828in}}%
\pgfpathlineto{\pgfqpoint{1.045845in}{1.295863in}}%
\pgfpathlineto{\pgfqpoint{1.094870in}{1.561448in}}%
\pgfpathlineto{\pgfqpoint{1.143896in}{1.339026in}}%
\pgfpathlineto{\pgfqpoint{1.192921in}{1.508003in}}%
\pgfpathlineto{\pgfqpoint{1.241946in}{1.302933in}}%
\pgfpathlineto{\pgfqpoint{1.290972in}{1.349525in}}%
\pgfpathlineto{\pgfqpoint{1.339997in}{1.474634in}}%
\pgfpathlineto{\pgfqpoint{1.389022in}{1.308715in}}%
\pgfpathlineto{\pgfqpoint{1.438047in}{1.311593in}}%
\pgfpathlineto{\pgfqpoint{1.487073in}{1.361786in}}%
\pgfpathlineto{\pgfqpoint{1.536098in}{1.228845in}}%
\pgfpathlineto{\pgfqpoint{1.585123in}{1.340681in}}%
\pgfpathlineto{\pgfqpoint{1.634149in}{1.296430in}}%
\pgfpathlineto{\pgfqpoint{1.683174in}{1.188726in}}%
\pgfpathlineto{\pgfqpoint{1.732199in}{1.126984in}}%
\pgfpathlineto{\pgfqpoint{1.781225in}{1.118552in}}%
\pgfpathlineto{\pgfqpoint{1.830250in}{0.986625in}}%
\pgfpathlineto{\pgfqpoint{1.879275in}{1.065592in}}%
\pgfpathlineto{\pgfqpoint{1.928301in}{1.001411in}}%
\pgfpathlineto{\pgfqpoint{1.977326in}{0.986797in}}%
\pgfpathlineto{\pgfqpoint{2.026351in}{0.985481in}}%
\pgfpathlineto{\pgfqpoint{2.075376in}{0.982407in}}%
\pgfpathlineto{\pgfqpoint{2.124402in}{0.944104in}}%
\pgfpathlineto{\pgfqpoint{2.173427in}{0.927591in}}%
\pgfpathlineto{\pgfqpoint{2.222452in}{0.943121in}}%
\pgfpathlineto{\pgfqpoint{2.271478in}{0.894624in}}%
\pgfpathlineto{\pgfqpoint{2.320503in}{0.940995in}}%
\pgfpathlineto{\pgfqpoint{2.369528in}{0.916144in}}%
\pgfpathlineto{\pgfqpoint{2.418554in}{0.892303in}}%
\pgfpathlineto{\pgfqpoint{2.467579in}{0.917183in}}%
\pgfpathlineto{\pgfqpoint{2.516604in}{0.864169in}}%
\pgfpathlineto{\pgfqpoint{2.565629in}{0.887187in}}%
\pgfpathlineto{\pgfqpoint{2.614655in}{0.871500in}}%
\pgfpathlineto{\pgfqpoint{2.663680in}{0.871275in}}%
\pgfpathlineto{\pgfqpoint{2.712705in}{0.857836in}}%
\pgfpathlineto{\pgfqpoint{2.761731in}{0.855430in}}%
\pgfpathlineto{\pgfqpoint{2.810756in}{0.863113in}}%
\pgfpathlineto{\pgfqpoint{2.859781in}{0.868609in}}%
\pgfpathlineto{\pgfqpoint{2.908807in}{0.859679in}}%
\pgfpathlineto{\pgfqpoint{2.957832in}{0.874388in}}%
\pgfpathlineto{\pgfqpoint{3.006857in}{0.859419in}}%
\pgfpathlineto{\pgfqpoint{3.055882in}{0.862465in}}%
\pgfpathlineto{\pgfqpoint{3.104908in}{0.877286in}}%
\pgfpathlineto{\pgfqpoint{3.153933in}{0.870688in}}%
\pgfpathlineto{\pgfqpoint{3.202958in}{0.871031in}}%
\pgfpathlineto{\pgfqpoint{3.251984in}{0.846579in}}%
\pgfpathlineto{\pgfqpoint{3.301009in}{0.884870in}}%
\pgfpathlineto{\pgfqpoint{3.350034in}{0.872119in}}%
\pgfpathlineto{\pgfqpoint{3.399060in}{0.846804in}}%
\pgfpathlineto{\pgfqpoint{3.448085in}{0.860452in}}%
\pgfpathlineto{\pgfqpoint{3.497110in}{0.859756in}}%
\pgfpathlineto{\pgfqpoint{3.546136in}{0.867332in}}%
\pgfpathlineto{\pgfqpoint{3.595161in}{0.861820in}}%
\pgfpathlineto{\pgfqpoint{3.644186in}{0.867345in}}%
\pgfpathlineto{\pgfqpoint{3.693211in}{0.872690in}}%
\pgfpathlineto{\pgfqpoint{3.742237in}{0.847035in}}%
\pgfpathlineto{\pgfqpoint{3.791262in}{0.856685in}}%
\pgfpathlineto{\pgfqpoint{3.840287in}{0.886642in}}%
\pgfpathlineto{\pgfqpoint{3.889313in}{0.843625in}}%
\pgfpathlineto{\pgfqpoint{3.938338in}{0.852942in}}%
\pgfpathlineto{\pgfqpoint{3.987363in}{0.848576in}}%
\pgfpathlineto{\pgfqpoint{4.036389in}{0.876117in}}%
\pgfusepath{stroke}%
\end{pgfscope}%
\begin{pgfscope}%
\pgfpathrectangle{\pgfqpoint{0.898769in}{0.566590in}}{\pgfqpoint{3.137619in}{1.828724in}}%
\pgfusepath{clip}%
\pgfsetbuttcap%
\pgfsetroundjoin%
\definecolor{currentfill}{rgb}{0.000000,0.000000,0.000000}%
\pgfsetfillcolor{currentfill}%
\pgfsetfillopacity{0.000000}%
\pgfsetlinewidth{1.003750pt}%
\definecolor{currentstroke}{rgb}{0.000000,0.447000,0.741000}%
\pgfsetstrokecolor{currentstroke}%
\pgfsetdash{}{0pt}%
\pgfsys@defobject{currentmarker}{\pgfqpoint{-0.041667in}{-0.041667in}}{\pgfqpoint{0.041667in}{0.041667in}}{%
\pgfpathmoveto{\pgfqpoint{0.000000in}{-0.041667in}}%
\pgfpathcurveto{\pgfqpoint{0.011050in}{-0.041667in}}{\pgfqpoint{0.021649in}{-0.037276in}}{\pgfqpoint{0.029463in}{-0.029463in}}%
\pgfpathcurveto{\pgfqpoint{0.037276in}{-0.021649in}}{\pgfqpoint{0.041667in}{-0.011050in}}{\pgfqpoint{0.041667in}{0.000000in}}%
\pgfpathcurveto{\pgfqpoint{0.041667in}{0.011050in}}{\pgfqpoint{0.037276in}{0.021649in}}{\pgfqpoint{0.029463in}{0.029463in}}%
\pgfpathcurveto{\pgfqpoint{0.021649in}{0.037276in}}{\pgfqpoint{0.011050in}{0.041667in}}{\pgfqpoint{0.000000in}{0.041667in}}%
\pgfpathcurveto{\pgfqpoint{-0.011050in}{0.041667in}}{\pgfqpoint{-0.021649in}{0.037276in}}{\pgfqpoint{-0.029463in}{0.029463in}}%
\pgfpathcurveto{\pgfqpoint{-0.037276in}{0.021649in}}{\pgfqpoint{-0.041667in}{0.011050in}}{\pgfqpoint{-0.041667in}{0.000000in}}%
\pgfpathcurveto{\pgfqpoint{-0.041667in}{-0.011050in}}{\pgfqpoint{-0.037276in}{-0.021649in}}{\pgfqpoint{-0.029463in}{-0.029463in}}%
\pgfpathcurveto{\pgfqpoint{-0.021649in}{-0.037276in}}{\pgfqpoint{-0.011050in}{-0.041667in}}{\pgfqpoint{0.000000in}{-0.041667in}}%
\pgfpathclose%
\pgfusepath{stroke,fill}%
}%
\begin{pgfscope}%
\pgfsys@transformshift{0.898769in}{1.289313in}%
\pgfsys@useobject{currentmarker}{}%
\end{pgfscope}%
\begin{pgfscope}%
\pgfsys@transformshift{1.094870in}{1.561448in}%
\pgfsys@useobject{currentmarker}{}%
\end{pgfscope}%
\begin{pgfscope}%
\pgfsys@transformshift{1.290972in}{1.349525in}%
\pgfsys@useobject{currentmarker}{}%
\end{pgfscope}%
\begin{pgfscope}%
\pgfsys@transformshift{1.487073in}{1.361786in}%
\pgfsys@useobject{currentmarker}{}%
\end{pgfscope}%
\begin{pgfscope}%
\pgfsys@transformshift{1.683174in}{1.188726in}%
\pgfsys@useobject{currentmarker}{}%
\end{pgfscope}%
\begin{pgfscope}%
\pgfsys@transformshift{1.879275in}{1.065592in}%
\pgfsys@useobject{currentmarker}{}%
\end{pgfscope}%
\begin{pgfscope}%
\pgfsys@transformshift{2.075376in}{0.982407in}%
\pgfsys@useobject{currentmarker}{}%
\end{pgfscope}%
\begin{pgfscope}%
\pgfsys@transformshift{2.271478in}{0.894624in}%
\pgfsys@useobject{currentmarker}{}%
\end{pgfscope}%
\begin{pgfscope}%
\pgfsys@transformshift{2.467579in}{0.917183in}%
\pgfsys@useobject{currentmarker}{}%
\end{pgfscope}%
\begin{pgfscope}%
\pgfsys@transformshift{2.663680in}{0.871275in}%
\pgfsys@useobject{currentmarker}{}%
\end{pgfscope}%
\begin{pgfscope}%
\pgfsys@transformshift{2.859781in}{0.868609in}%
\pgfsys@useobject{currentmarker}{}%
\end{pgfscope}%
\begin{pgfscope}%
\pgfsys@transformshift{3.055882in}{0.862465in}%
\pgfsys@useobject{currentmarker}{}%
\end{pgfscope}%
\begin{pgfscope}%
\pgfsys@transformshift{3.251984in}{0.846579in}%
\pgfsys@useobject{currentmarker}{}%
\end{pgfscope}%
\begin{pgfscope}%
\pgfsys@transformshift{3.448085in}{0.860452in}%
\pgfsys@useobject{currentmarker}{}%
\end{pgfscope}%
\begin{pgfscope}%
\pgfsys@transformshift{3.644186in}{0.867345in}%
\pgfsys@useobject{currentmarker}{}%
\end{pgfscope}%
\begin{pgfscope}%
\pgfsys@transformshift{3.840287in}{0.886642in}%
\pgfsys@useobject{currentmarker}{}%
\end{pgfscope}%
\begin{pgfscope}%
\pgfsys@transformshift{4.036389in}{0.876117in}%
\pgfsys@useobject{currentmarker}{}%
\end{pgfscope}%
\end{pgfscope}%
\begin{pgfscope}%
\pgfpathrectangle{\pgfqpoint{0.898769in}{0.566590in}}{\pgfqpoint{3.137619in}{1.828724in}}%
\pgfusepath{clip}%
\pgfsetrectcap%
\pgfsetroundjoin%
\pgfsetlinewidth{1.505625pt}%
\definecolor{currentstroke}{rgb}{0.850000,0.324000,0.098000}%
\pgfsetstrokecolor{currentstroke}%
\pgfsetdash{}{0pt}%
\pgfpathmoveto{\pgfqpoint{0.898769in}{0.992270in}}%
\pgfpathlineto{\pgfqpoint{0.947794in}{0.970114in}}%
\pgfpathlineto{\pgfqpoint{0.996820in}{0.965628in}}%
\pgfpathlineto{\pgfqpoint{1.045845in}{0.937423in}}%
\pgfpathlineto{\pgfqpoint{1.094870in}{0.938466in}}%
\pgfpathlineto{\pgfqpoint{1.143896in}{0.975537in}}%
\pgfpathlineto{\pgfqpoint{1.192921in}{0.956874in}}%
\pgfpathlineto{\pgfqpoint{1.241946in}{0.930761in}}%
\pgfpathlineto{\pgfqpoint{1.290972in}{0.874887in}}%
\pgfpathlineto{\pgfqpoint{1.339997in}{0.931558in}}%
\pgfpathlineto{\pgfqpoint{1.389022in}{0.828506in}}%
\pgfpathlineto{\pgfqpoint{1.438047in}{0.844474in}}%
\pgfpathlineto{\pgfqpoint{1.487073in}{0.796689in}}%
\pgfpathlineto{\pgfqpoint{1.536098in}{0.820962in}}%
\pgfpathlineto{\pgfqpoint{1.585123in}{0.758689in}}%
\pgfpathlineto{\pgfqpoint{1.634149in}{0.714071in}}%
\pgfpathlineto{\pgfqpoint{1.683174in}{0.706770in}}%
\pgfpathlineto{\pgfqpoint{1.732199in}{0.762509in}}%
\pgfpathlineto{\pgfqpoint{1.781225in}{0.703952in}}%
\pgfpathlineto{\pgfqpoint{1.830250in}{0.696826in}}%
\pgfpathlineto{\pgfqpoint{1.879275in}{0.674657in}}%
\pgfpathlineto{\pgfqpoint{1.928301in}{0.680427in}}%
\pgfpathlineto{\pgfqpoint{1.977326in}{0.673381in}}%
\pgfpathlineto{\pgfqpoint{2.026351in}{0.702781in}}%
\pgfpathlineto{\pgfqpoint{2.075376in}{0.682031in}}%
\pgfpathlineto{\pgfqpoint{2.124402in}{0.663959in}}%
\pgfpathlineto{\pgfqpoint{2.173427in}{0.689631in}}%
\pgfpathlineto{\pgfqpoint{2.222452in}{0.679979in}}%
\pgfpathlineto{\pgfqpoint{2.271478in}{0.694635in}}%
\pgfpathlineto{\pgfqpoint{2.320503in}{0.684568in}}%
\pgfpathlineto{\pgfqpoint{2.369528in}{0.688883in}}%
\pgfpathlineto{\pgfqpoint{2.418554in}{0.684835in}}%
\pgfpathlineto{\pgfqpoint{2.467579in}{0.695662in}}%
\pgfpathlineto{\pgfqpoint{2.516604in}{0.670307in}}%
\pgfpathlineto{\pgfqpoint{2.565629in}{0.690350in}}%
\pgfpathlineto{\pgfqpoint{2.614655in}{0.666576in}}%
\pgfpathlineto{\pgfqpoint{2.663680in}{0.696423in}}%
\pgfpathlineto{\pgfqpoint{2.712705in}{0.671171in}}%
\pgfpathlineto{\pgfqpoint{2.761731in}{0.680378in}}%
\pgfpathlineto{\pgfqpoint{2.810756in}{0.683257in}}%
\pgfpathlineto{\pgfqpoint{2.859781in}{0.675028in}}%
\pgfpathlineto{\pgfqpoint{2.908807in}{0.695930in}}%
\pgfpathlineto{\pgfqpoint{2.957832in}{0.682043in}}%
\pgfpathlineto{\pgfqpoint{3.006857in}{0.676520in}}%
\pgfpathlineto{\pgfqpoint{3.055882in}{0.675210in}}%
\pgfpathlineto{\pgfqpoint{3.104908in}{0.675532in}}%
\pgfpathlineto{\pgfqpoint{3.153933in}{0.658457in}}%
\pgfpathlineto{\pgfqpoint{3.202958in}{0.677052in}}%
\pgfpathlineto{\pgfqpoint{3.251984in}{0.671454in}}%
\pgfpathlineto{\pgfqpoint{3.301009in}{0.671897in}}%
\pgfpathlineto{\pgfqpoint{3.350034in}{0.676065in}}%
\pgfpathlineto{\pgfqpoint{3.399060in}{0.676587in}}%
\pgfpathlineto{\pgfqpoint{3.448085in}{0.663401in}}%
\pgfpathlineto{\pgfqpoint{3.497110in}{0.694052in}}%
\pgfpathlineto{\pgfqpoint{3.546136in}{0.688927in}}%
\pgfpathlineto{\pgfqpoint{3.595161in}{0.664116in}}%
\pgfpathlineto{\pgfqpoint{3.644186in}{0.673218in}}%
\pgfpathlineto{\pgfqpoint{3.693211in}{0.694885in}}%
\pgfpathlineto{\pgfqpoint{3.742237in}{0.691589in}}%
\pgfpathlineto{\pgfqpoint{3.791262in}{0.685302in}}%
\pgfpathlineto{\pgfqpoint{3.840287in}{0.684598in}}%
\pgfpathlineto{\pgfqpoint{3.889313in}{0.677922in}}%
\pgfpathlineto{\pgfqpoint{3.938338in}{0.657568in}}%
\pgfpathlineto{\pgfqpoint{3.987363in}{0.688230in}}%
\pgfpathlineto{\pgfqpoint{4.036389in}{0.682333in}}%
\pgfusepath{stroke}%
\end{pgfscope}%
\begin{pgfscope}%
\pgfpathrectangle{\pgfqpoint{0.898769in}{0.566590in}}{\pgfqpoint{3.137619in}{1.828724in}}%
\pgfusepath{clip}%
\pgfsetbuttcap%
\pgfsetroundjoin%
\definecolor{currentfill}{rgb}{0.850000,0.324000,0.098000}%
\pgfsetfillcolor{currentfill}%
\pgfsetlinewidth{1.003750pt}%
\definecolor{currentstroke}{rgb}{0.850000,0.324000,0.098000}%
\pgfsetstrokecolor{currentstroke}%
\pgfsetdash{}{0pt}%
\pgfsys@defobject{currentmarker}{\pgfqpoint{-0.041667in}{-0.041667in}}{\pgfqpoint{0.041667in}{0.041667in}}{%
\pgfpathmoveto{\pgfqpoint{-0.041667in}{0.000000in}}%
\pgfpathlineto{\pgfqpoint{0.041667in}{0.000000in}}%
\pgfpathmoveto{\pgfqpoint{0.000000in}{-0.041667in}}%
\pgfpathlineto{\pgfqpoint{0.000000in}{0.041667in}}%
\pgfusepath{stroke,fill}%
}%
\begin{pgfscope}%
\pgfsys@transformshift{0.898769in}{0.992270in}%
\pgfsys@useobject{currentmarker}{}%
\end{pgfscope}%
\begin{pgfscope}%
\pgfsys@transformshift{1.045845in}{0.937423in}%
\pgfsys@useobject{currentmarker}{}%
\end{pgfscope}%
\begin{pgfscope}%
\pgfsys@transformshift{1.192921in}{0.956874in}%
\pgfsys@useobject{currentmarker}{}%
\end{pgfscope}%
\begin{pgfscope}%
\pgfsys@transformshift{1.339997in}{0.931558in}%
\pgfsys@useobject{currentmarker}{}%
\end{pgfscope}%
\begin{pgfscope}%
\pgfsys@transformshift{1.487073in}{0.796689in}%
\pgfsys@useobject{currentmarker}{}%
\end{pgfscope}%
\begin{pgfscope}%
\pgfsys@transformshift{1.634149in}{0.714071in}%
\pgfsys@useobject{currentmarker}{}%
\end{pgfscope}%
\begin{pgfscope}%
\pgfsys@transformshift{1.781225in}{0.703952in}%
\pgfsys@useobject{currentmarker}{}%
\end{pgfscope}%
\begin{pgfscope}%
\pgfsys@transformshift{1.928301in}{0.680427in}%
\pgfsys@useobject{currentmarker}{}%
\end{pgfscope}%
\begin{pgfscope}%
\pgfsys@transformshift{2.075376in}{0.682031in}%
\pgfsys@useobject{currentmarker}{}%
\end{pgfscope}%
\begin{pgfscope}%
\pgfsys@transformshift{2.222452in}{0.679979in}%
\pgfsys@useobject{currentmarker}{}%
\end{pgfscope}%
\begin{pgfscope}%
\pgfsys@transformshift{2.369528in}{0.688883in}%
\pgfsys@useobject{currentmarker}{}%
\end{pgfscope}%
\begin{pgfscope}%
\pgfsys@transformshift{2.516604in}{0.670307in}%
\pgfsys@useobject{currentmarker}{}%
\end{pgfscope}%
\begin{pgfscope}%
\pgfsys@transformshift{2.663680in}{0.696423in}%
\pgfsys@useobject{currentmarker}{}%
\end{pgfscope}%
\begin{pgfscope}%
\pgfsys@transformshift{2.810756in}{0.683257in}%
\pgfsys@useobject{currentmarker}{}%
\end{pgfscope}%
\begin{pgfscope}%
\pgfsys@transformshift{2.957832in}{0.682043in}%
\pgfsys@useobject{currentmarker}{}%
\end{pgfscope}%
\begin{pgfscope}%
\pgfsys@transformshift{3.104908in}{0.675532in}%
\pgfsys@useobject{currentmarker}{}%
\end{pgfscope}%
\begin{pgfscope}%
\pgfsys@transformshift{3.251984in}{0.671454in}%
\pgfsys@useobject{currentmarker}{}%
\end{pgfscope}%
\begin{pgfscope}%
\pgfsys@transformshift{3.399060in}{0.676587in}%
\pgfsys@useobject{currentmarker}{}%
\end{pgfscope}%
\begin{pgfscope}%
\pgfsys@transformshift{3.546136in}{0.688927in}%
\pgfsys@useobject{currentmarker}{}%
\end{pgfscope}%
\begin{pgfscope}%
\pgfsys@transformshift{3.693211in}{0.694885in}%
\pgfsys@useobject{currentmarker}{}%
\end{pgfscope}%
\begin{pgfscope}%
\pgfsys@transformshift{3.840287in}{0.684598in}%
\pgfsys@useobject{currentmarker}{}%
\end{pgfscope}%
\begin{pgfscope}%
\pgfsys@transformshift{3.987363in}{0.688230in}%
\pgfsys@useobject{currentmarker}{}%
\end{pgfscope}%
\end{pgfscope}%
\begin{pgfscope}%
\pgfpathrectangle{\pgfqpoint{0.898769in}{0.566590in}}{\pgfqpoint{3.137619in}{1.828724in}}%
\pgfusepath{clip}%
\pgfsetrectcap%
\pgfsetroundjoin%
\pgfsetlinewidth{1.505625pt}%
\definecolor{currentstroke}{rgb}{0.000000,0.500000,0.000000}%
\pgfsetstrokecolor{currentstroke}%
\pgfsetdash{}{0pt}%
\pgfpathmoveto{\pgfqpoint{0.898769in}{1.072204in}}%
\pgfpathlineto{\pgfqpoint{0.947794in}{1.080042in}}%
\pgfpathlineto{\pgfqpoint{0.996820in}{1.033026in}}%
\pgfpathlineto{\pgfqpoint{1.045845in}{0.999106in}}%
\pgfpathlineto{\pgfqpoint{1.094870in}{1.032458in}}%
\pgfpathlineto{\pgfqpoint{1.143896in}{1.025991in}}%
\pgfpathlineto{\pgfqpoint{1.192921in}{1.046178in}}%
\pgfpathlineto{\pgfqpoint{1.241946in}{0.965468in}}%
\pgfpathlineto{\pgfqpoint{1.290972in}{0.924942in}}%
\pgfpathlineto{\pgfqpoint{1.339997in}{0.922844in}}%
\pgfpathlineto{\pgfqpoint{1.389022in}{0.898356in}}%
\pgfpathlineto{\pgfqpoint{1.438047in}{0.934063in}}%
\pgfpathlineto{\pgfqpoint{1.487073in}{0.905776in}}%
\pgfpathlineto{\pgfqpoint{1.536098in}{0.911067in}}%
\pgfpathlineto{\pgfqpoint{1.585123in}{0.920423in}}%
\pgfpathlineto{\pgfqpoint{1.634149in}{0.886252in}}%
\pgfpathlineto{\pgfqpoint{1.683174in}{0.878109in}}%
\pgfpathlineto{\pgfqpoint{1.732199in}{0.911181in}}%
\pgfpathlineto{\pgfqpoint{1.781225in}{0.863974in}}%
\pgfpathlineto{\pgfqpoint{1.830250in}{0.826461in}}%
\pgfpathlineto{\pgfqpoint{1.879275in}{0.796876in}}%
\pgfpathlineto{\pgfqpoint{1.928301in}{0.851616in}}%
\pgfpathlineto{\pgfqpoint{1.977326in}{0.802754in}}%
\pgfpathlineto{\pgfqpoint{2.026351in}{0.838441in}}%
\pgfpathlineto{\pgfqpoint{2.075376in}{0.811377in}}%
\pgfpathlineto{\pgfqpoint{2.124402in}{0.797122in}}%
\pgfpathlineto{\pgfqpoint{2.173427in}{0.832603in}}%
\pgfpathlineto{\pgfqpoint{2.222452in}{0.836847in}}%
\pgfpathlineto{\pgfqpoint{2.271478in}{0.790336in}}%
\pgfpathlineto{\pgfqpoint{2.320503in}{0.809755in}}%
\pgfpathlineto{\pgfqpoint{2.369528in}{0.799580in}}%
\pgfpathlineto{\pgfqpoint{2.418554in}{0.788837in}}%
\pgfpathlineto{\pgfqpoint{2.467579in}{0.800298in}}%
\pgfpathlineto{\pgfqpoint{2.516604in}{0.808016in}}%
\pgfpathlineto{\pgfqpoint{2.565629in}{0.802678in}}%
\pgfpathlineto{\pgfqpoint{2.614655in}{0.835777in}}%
\pgfpathlineto{\pgfqpoint{2.663680in}{0.783419in}}%
\pgfpathlineto{\pgfqpoint{2.712705in}{0.798062in}}%
\pgfpathlineto{\pgfqpoint{2.761731in}{0.796698in}}%
\pgfpathlineto{\pgfqpoint{2.810756in}{0.789854in}}%
\pgfpathlineto{\pgfqpoint{2.859781in}{0.787564in}}%
\pgfpathlineto{\pgfqpoint{2.908807in}{0.822249in}}%
\pgfpathlineto{\pgfqpoint{2.957832in}{0.808297in}}%
\pgfpathlineto{\pgfqpoint{3.006857in}{0.814642in}}%
\pgfpathlineto{\pgfqpoint{3.055882in}{0.826825in}}%
\pgfpathlineto{\pgfqpoint{3.104908in}{0.797528in}}%
\pgfpathlineto{\pgfqpoint{3.153933in}{0.804179in}}%
\pgfpathlineto{\pgfqpoint{3.202958in}{0.782679in}}%
\pgfpathlineto{\pgfqpoint{3.251984in}{0.794496in}}%
\pgfpathlineto{\pgfqpoint{3.301009in}{0.810158in}}%
\pgfpathlineto{\pgfqpoint{3.350034in}{0.798933in}}%
\pgfpathlineto{\pgfqpoint{3.399060in}{0.808698in}}%
\pgfpathlineto{\pgfqpoint{3.448085in}{0.789607in}}%
\pgfpathlineto{\pgfqpoint{3.497110in}{0.813102in}}%
\pgfpathlineto{\pgfqpoint{3.546136in}{0.812017in}}%
\pgfpathlineto{\pgfqpoint{3.595161in}{0.779340in}}%
\pgfpathlineto{\pgfqpoint{3.644186in}{0.805634in}}%
\pgfpathlineto{\pgfqpoint{3.693211in}{0.839548in}}%
\pgfpathlineto{\pgfqpoint{3.742237in}{0.812614in}}%
\pgfpathlineto{\pgfqpoint{3.791262in}{0.827697in}}%
\pgfpathlineto{\pgfqpoint{3.840287in}{0.836292in}}%
\pgfpathlineto{\pgfqpoint{3.889313in}{0.812802in}}%
\pgfpathlineto{\pgfqpoint{3.938338in}{0.813868in}}%
\pgfpathlineto{\pgfqpoint{3.987363in}{0.847241in}}%
\pgfpathlineto{\pgfqpoint{4.036389in}{0.808803in}}%
\pgfusepath{stroke}%
\end{pgfscope}%
\begin{pgfscope}%
\pgfpathrectangle{\pgfqpoint{0.898769in}{0.566590in}}{\pgfqpoint{3.137619in}{1.828724in}}%
\pgfusepath{clip}%
\pgfsetbuttcap%
\pgfsetmiterjoin%
\definecolor{currentfill}{rgb}{0.000000,0.000000,0.000000}%
\pgfsetfillcolor{currentfill}%
\pgfsetfillopacity{0.000000}%
\pgfsetlinewidth{1.003750pt}%
\definecolor{currentstroke}{rgb}{0.000000,0.500000,0.000000}%
\pgfsetstrokecolor{currentstroke}%
\pgfsetdash{}{0pt}%
\pgfsys@defobject{currentmarker}{\pgfqpoint{-0.041667in}{-0.041667in}}{\pgfqpoint{0.041667in}{0.041667in}}{%
\pgfpathmoveto{\pgfqpoint{-0.041667in}{-0.041667in}}%
\pgfpathlineto{\pgfqpoint{0.041667in}{-0.041667in}}%
\pgfpathlineto{\pgfqpoint{0.041667in}{0.041667in}}%
\pgfpathlineto{\pgfqpoint{-0.041667in}{0.041667in}}%
\pgfpathclose%
\pgfusepath{stroke,fill}%
}%
\begin{pgfscope}%
\pgfsys@transformshift{0.898769in}{1.072204in}%
\pgfsys@useobject{currentmarker}{}%
\end{pgfscope}%
\begin{pgfscope}%
\pgfsys@transformshift{1.143896in}{1.025991in}%
\pgfsys@useobject{currentmarker}{}%
\end{pgfscope}%
\begin{pgfscope}%
\pgfsys@transformshift{1.389022in}{0.898356in}%
\pgfsys@useobject{currentmarker}{}%
\end{pgfscope}%
\begin{pgfscope}%
\pgfsys@transformshift{1.634149in}{0.886252in}%
\pgfsys@useobject{currentmarker}{}%
\end{pgfscope}%
\begin{pgfscope}%
\pgfsys@transformshift{1.879275in}{0.796876in}%
\pgfsys@useobject{currentmarker}{}%
\end{pgfscope}%
\begin{pgfscope}%
\pgfsys@transformshift{2.124402in}{0.797122in}%
\pgfsys@useobject{currentmarker}{}%
\end{pgfscope}%
\begin{pgfscope}%
\pgfsys@transformshift{2.369528in}{0.799580in}%
\pgfsys@useobject{currentmarker}{}%
\end{pgfscope}%
\begin{pgfscope}%
\pgfsys@transformshift{2.614655in}{0.835777in}%
\pgfsys@useobject{currentmarker}{}%
\end{pgfscope}%
\begin{pgfscope}%
\pgfsys@transformshift{2.859781in}{0.787564in}%
\pgfsys@useobject{currentmarker}{}%
\end{pgfscope}%
\begin{pgfscope}%
\pgfsys@transformshift{3.104908in}{0.797528in}%
\pgfsys@useobject{currentmarker}{}%
\end{pgfscope}%
\begin{pgfscope}%
\pgfsys@transformshift{3.350034in}{0.798933in}%
\pgfsys@useobject{currentmarker}{}%
\end{pgfscope}%
\begin{pgfscope}%
\pgfsys@transformshift{3.595161in}{0.779340in}%
\pgfsys@useobject{currentmarker}{}%
\end{pgfscope}%
\begin{pgfscope}%
\pgfsys@transformshift{3.840287in}{0.836292in}%
\pgfsys@useobject{currentmarker}{}%
\end{pgfscope}%
\end{pgfscope}%
\begin{pgfscope}%
\pgfpathrectangle{\pgfqpoint{0.898769in}{0.566590in}}{\pgfqpoint{3.137619in}{1.828724in}}%
\pgfusepath{clip}%
\pgfsetrectcap%
\pgfsetroundjoin%
\pgfsetlinewidth{1.505625pt}%
\definecolor{currentstroke}{rgb}{0.494000,0.184000,0.556000}%
\pgfsetstrokecolor{currentstroke}%
\pgfsetdash{}{0pt}%
\pgfpathmoveto{\pgfqpoint{0.898769in}{1.148604in}}%
\pgfpathlineto{\pgfqpoint{0.947794in}{1.130172in}}%
\pgfpathlineto{\pgfqpoint{0.996820in}{1.043488in}}%
\pgfpathlineto{\pgfqpoint{1.045845in}{1.078673in}}%
\pgfpathlineto{\pgfqpoint{1.094870in}{1.021965in}}%
\pgfpathlineto{\pgfqpoint{1.143896in}{1.021164in}}%
\pgfpathlineto{\pgfqpoint{1.192921in}{1.185194in}}%
\pgfpathlineto{\pgfqpoint{1.241946in}{0.938739in}}%
\pgfpathlineto{\pgfqpoint{1.290972in}{0.886223in}}%
\pgfpathlineto{\pgfqpoint{1.339997in}{0.892645in}}%
\pgfpathlineto{\pgfqpoint{1.389022in}{0.783284in}}%
\pgfpathlineto{\pgfqpoint{1.438047in}{0.765039in}}%
\pgfpathlineto{\pgfqpoint{1.487073in}{0.763738in}}%
\pgfpathlineto{\pgfqpoint{1.536098in}{0.773321in}}%
\pgfpathlineto{\pgfqpoint{1.585123in}{0.729163in}}%
\pgfpathlineto{\pgfqpoint{1.634149in}{0.711330in}}%
\pgfpathlineto{\pgfqpoint{1.683174in}{0.753381in}}%
\pgfpathlineto{\pgfqpoint{1.732199in}{0.719632in}}%
\pgfpathlineto{\pgfqpoint{1.781225in}{0.697618in}}%
\pgfpathlineto{\pgfqpoint{1.830250in}{0.709894in}}%
\pgfpathlineto{\pgfqpoint{1.879275in}{0.716137in}}%
\pgfpathlineto{\pgfqpoint{1.928301in}{0.693483in}}%
\pgfpathlineto{\pgfqpoint{1.977326in}{0.704695in}}%
\pgfpathlineto{\pgfqpoint{2.026351in}{0.709341in}}%
\pgfpathlineto{\pgfqpoint{2.075376in}{0.713009in}}%
\pgfpathlineto{\pgfqpoint{2.124402in}{0.708316in}}%
\pgfpathlineto{\pgfqpoint{2.173427in}{0.697377in}}%
\pgfpathlineto{\pgfqpoint{2.222452in}{0.673655in}}%
\pgfpathlineto{\pgfqpoint{2.271478in}{0.695175in}}%
\pgfpathlineto{\pgfqpoint{2.320503in}{0.703118in}}%
\pgfpathlineto{\pgfqpoint{2.369528in}{0.707830in}}%
\pgfpathlineto{\pgfqpoint{2.418554in}{0.693085in}}%
\pgfpathlineto{\pgfqpoint{2.467579in}{0.696720in}}%
\pgfpathlineto{\pgfqpoint{2.516604in}{0.664095in}}%
\pgfpathlineto{\pgfqpoint{2.565629in}{0.694957in}}%
\pgfpathlineto{\pgfqpoint{2.614655in}{0.680367in}}%
\pgfpathlineto{\pgfqpoint{2.663680in}{0.692661in}}%
\pgfpathlineto{\pgfqpoint{2.712705in}{0.688745in}}%
\pgfpathlineto{\pgfqpoint{2.761731in}{0.649714in}}%
\pgfpathlineto{\pgfqpoint{2.810756in}{0.675170in}}%
\pgfpathlineto{\pgfqpoint{2.859781in}{0.668653in}}%
\pgfpathlineto{\pgfqpoint{2.908807in}{0.683473in}}%
\pgfpathlineto{\pgfqpoint{2.957832in}{0.710141in}}%
\pgfpathlineto{\pgfqpoint{3.006857in}{0.694668in}}%
\pgfpathlineto{\pgfqpoint{3.055882in}{0.685873in}}%
\pgfpathlineto{\pgfqpoint{3.104908in}{0.678014in}}%
\pgfpathlineto{\pgfqpoint{3.153933in}{0.689101in}}%
\pgfpathlineto{\pgfqpoint{3.202958in}{0.689577in}}%
\pgfpathlineto{\pgfqpoint{3.251984in}{0.685592in}}%
\pgfpathlineto{\pgfqpoint{3.301009in}{0.690645in}}%
\pgfpathlineto{\pgfqpoint{3.350034in}{0.682008in}}%
\pgfpathlineto{\pgfqpoint{3.399060in}{0.668304in}}%
\pgfpathlineto{\pgfqpoint{3.448085in}{0.689154in}}%
\pgfpathlineto{\pgfqpoint{3.497110in}{0.674812in}}%
\pgfpathlineto{\pgfqpoint{3.546136in}{0.681225in}}%
\pgfpathlineto{\pgfqpoint{3.595161in}{0.712380in}}%
\pgfpathlineto{\pgfqpoint{3.644186in}{0.660563in}}%
\pgfpathlineto{\pgfqpoint{3.693211in}{0.676285in}}%
\pgfpathlineto{\pgfqpoint{3.742237in}{0.693176in}}%
\pgfpathlineto{\pgfqpoint{3.791262in}{0.689041in}}%
\pgfpathlineto{\pgfqpoint{3.840287in}{0.686401in}}%
\pgfpathlineto{\pgfqpoint{3.889313in}{0.677719in}}%
\pgfpathlineto{\pgfqpoint{3.938338in}{0.652679in}}%
\pgfpathlineto{\pgfqpoint{3.987363in}{0.696570in}}%
\pgfpathlineto{\pgfqpoint{4.036389in}{0.665992in}}%
\pgfusepath{stroke}%
\end{pgfscope}%
\begin{pgfscope}%
\pgfpathrectangle{\pgfqpoint{0.898769in}{0.566590in}}{\pgfqpoint{3.137619in}{1.828724in}}%
\pgfusepath{clip}%
\pgfsetbuttcap%
\pgfsetroundjoin%
\definecolor{currentfill}{rgb}{0.494000,0.184000,0.556000}%
\pgfsetfillcolor{currentfill}%
\pgfsetlinewidth{1.003750pt}%
\definecolor{currentstroke}{rgb}{0.494000,0.184000,0.556000}%
\pgfsetstrokecolor{currentstroke}%
\pgfsetdash{}{0pt}%
\pgfsys@defobject{currentmarker}{\pgfqpoint{-0.041667in}{-0.041667in}}{\pgfqpoint{0.041667in}{0.041667in}}{%
\pgfpathmoveto{\pgfqpoint{-0.041667in}{-0.041667in}}%
\pgfpathlineto{\pgfqpoint{0.041667in}{0.041667in}}%
\pgfpathmoveto{\pgfqpoint{-0.041667in}{0.041667in}}%
\pgfpathlineto{\pgfqpoint{0.041667in}{-0.041667in}}%
\pgfusepath{stroke,fill}%
}%
\begin{pgfscope}%
\pgfsys@transformshift{0.898769in}{1.148604in}%
\pgfsys@useobject{currentmarker}{}%
\end{pgfscope}%
\begin{pgfscope}%
\pgfsys@transformshift{1.094870in}{1.021965in}%
\pgfsys@useobject{currentmarker}{}%
\end{pgfscope}%
\begin{pgfscope}%
\pgfsys@transformshift{1.290972in}{0.886223in}%
\pgfsys@useobject{currentmarker}{}%
\end{pgfscope}%
\begin{pgfscope}%
\pgfsys@transformshift{1.487073in}{0.763738in}%
\pgfsys@useobject{currentmarker}{}%
\end{pgfscope}%
\begin{pgfscope}%
\pgfsys@transformshift{1.683174in}{0.753381in}%
\pgfsys@useobject{currentmarker}{}%
\end{pgfscope}%
\begin{pgfscope}%
\pgfsys@transformshift{1.879275in}{0.716137in}%
\pgfsys@useobject{currentmarker}{}%
\end{pgfscope}%
\begin{pgfscope}%
\pgfsys@transformshift{2.075376in}{0.713009in}%
\pgfsys@useobject{currentmarker}{}%
\end{pgfscope}%
\begin{pgfscope}%
\pgfsys@transformshift{2.271478in}{0.695175in}%
\pgfsys@useobject{currentmarker}{}%
\end{pgfscope}%
\begin{pgfscope}%
\pgfsys@transformshift{2.467579in}{0.696720in}%
\pgfsys@useobject{currentmarker}{}%
\end{pgfscope}%
\begin{pgfscope}%
\pgfsys@transformshift{2.663680in}{0.692661in}%
\pgfsys@useobject{currentmarker}{}%
\end{pgfscope}%
\begin{pgfscope}%
\pgfsys@transformshift{2.859781in}{0.668653in}%
\pgfsys@useobject{currentmarker}{}%
\end{pgfscope}%
\begin{pgfscope}%
\pgfsys@transformshift{3.055882in}{0.685873in}%
\pgfsys@useobject{currentmarker}{}%
\end{pgfscope}%
\begin{pgfscope}%
\pgfsys@transformshift{3.251984in}{0.685592in}%
\pgfsys@useobject{currentmarker}{}%
\end{pgfscope}%
\begin{pgfscope}%
\pgfsys@transformshift{3.448085in}{0.689154in}%
\pgfsys@useobject{currentmarker}{}%
\end{pgfscope}%
\begin{pgfscope}%
\pgfsys@transformshift{3.644186in}{0.660563in}%
\pgfsys@useobject{currentmarker}{}%
\end{pgfscope}%
\begin{pgfscope}%
\pgfsys@transformshift{3.840287in}{0.686401in}%
\pgfsys@useobject{currentmarker}{}%
\end{pgfscope}%
\begin{pgfscope}%
\pgfsys@transformshift{4.036389in}{0.665992in}%
\pgfsys@useobject{currentmarker}{}%
\end{pgfscope}%
\end{pgfscope}%
\begin{pgfscope}%
\pgfsetrectcap%
\pgfsetmiterjoin%
\pgfsetlinewidth{0.803000pt}%
\definecolor{currentstroke}{rgb}{0.000000,0.000000,0.000000}%
\pgfsetstrokecolor{currentstroke}%
\pgfsetdash{}{0pt}%
\pgfpathmoveto{\pgfqpoint{0.898769in}{0.566590in}}%
\pgfpathlineto{\pgfqpoint{0.898769in}{2.395314in}}%
\pgfusepath{stroke}%
\end{pgfscope}%
\begin{pgfscope}%
\pgfsetrectcap%
\pgfsetmiterjoin%
\pgfsetlinewidth{0.803000pt}%
\definecolor{currentstroke}{rgb}{0.000000,0.000000,0.000000}%
\pgfsetstrokecolor{currentstroke}%
\pgfsetdash{}{0pt}%
\pgfpathmoveto{\pgfqpoint{4.036389in}{0.566590in}}%
\pgfpathlineto{\pgfqpoint{4.036389in}{2.395314in}}%
\pgfusepath{stroke}%
\end{pgfscope}%
\begin{pgfscope}%
\pgfsetrectcap%
\pgfsetmiterjoin%
\pgfsetlinewidth{0.803000pt}%
\definecolor{currentstroke}{rgb}{0.000000,0.000000,0.000000}%
\pgfsetstrokecolor{currentstroke}%
\pgfsetdash{}{0pt}%
\pgfpathmoveto{\pgfqpoint{0.898769in}{0.566590in}}%
\pgfpathlineto{\pgfqpoint{4.036389in}{0.566590in}}%
\pgfusepath{stroke}%
\end{pgfscope}%
\begin{pgfscope}%
\pgfsetrectcap%
\pgfsetmiterjoin%
\pgfsetlinewidth{0.803000pt}%
\definecolor{currentstroke}{rgb}{0.000000,0.000000,0.000000}%
\pgfsetstrokecolor{currentstroke}%
\pgfsetdash{}{0pt}%
\pgfpathmoveto{\pgfqpoint{0.898769in}{2.395314in}}%
\pgfpathlineto{\pgfqpoint{4.036389in}{2.395314in}}%
\pgfusepath{stroke}%
\end{pgfscope}%
\begin{pgfscope}%
\pgfsetbuttcap%
\pgfsetmiterjoin%
\definecolor{currentfill}{rgb}{1.000000,1.000000,1.000000}%
\pgfsetfillcolor{currentfill}%
\pgfsetfillopacity{0.800000}%
\pgfsetlinewidth{1.003750pt}%
\definecolor{currentstroke}{rgb}{0.800000,0.800000,0.800000}%
\pgfsetstrokecolor{currentstroke}%
\pgfsetstrokeopacity{0.800000}%
\pgfsetdash{}{0pt}%
\pgfpathmoveto{\pgfqpoint{2.868054in}{1.598092in}}%
\pgfpathlineto{\pgfqpoint{3.948889in}{1.598092in}}%
\pgfpathquadraticcurveto{\pgfqpoint{3.973889in}{1.598092in}}{\pgfqpoint{3.973889in}{1.623092in}}%
\pgfpathlineto{\pgfqpoint{3.973889in}{2.307814in}}%
\pgfpathquadraticcurveto{\pgfqpoint{3.973889in}{2.332814in}}{\pgfqpoint{3.948889in}{2.332814in}}%
\pgfpathlineto{\pgfqpoint{2.868054in}{2.332814in}}%
\pgfpathquadraticcurveto{\pgfqpoint{2.843054in}{2.332814in}}{\pgfqpoint{2.843054in}{2.307814in}}%
\pgfpathlineto{\pgfqpoint{2.843054in}{1.623092in}}%
\pgfpathquadraticcurveto{\pgfqpoint{2.843054in}{1.598092in}}{\pgfqpoint{2.868054in}{1.598092in}}%
\pgfpathclose%
\pgfusepath{stroke,fill}%
\end{pgfscope}%
\begin{pgfscope}%
\pgfsetbuttcap%
\pgfsetroundjoin%
\definecolor{currentfill}{rgb}{0.000000,0.000000,0.000000}%
\pgfsetfillcolor{currentfill}%
\pgfsetfillopacity{0.000000}%
\pgfsetlinewidth{1.003750pt}%
\definecolor{currentstroke}{rgb}{0.000000,0.447000,0.741000}%
\pgfsetstrokecolor{currentstroke}%
\pgfsetdash{}{0pt}%
\pgfsys@defobject{currentmarker}{\pgfqpoint{-0.041667in}{-0.041667in}}{\pgfqpoint{0.041667in}{0.041667in}}{%
\pgfpathmoveto{\pgfqpoint{0.000000in}{-0.041667in}}%
\pgfpathcurveto{\pgfqpoint{0.011050in}{-0.041667in}}{\pgfqpoint{0.021649in}{-0.037276in}}{\pgfqpoint{0.029463in}{-0.029463in}}%
\pgfpathcurveto{\pgfqpoint{0.037276in}{-0.021649in}}{\pgfqpoint{0.041667in}{-0.011050in}}{\pgfqpoint{0.041667in}{0.000000in}}%
\pgfpathcurveto{\pgfqpoint{0.041667in}{0.011050in}}{\pgfqpoint{0.037276in}{0.021649in}}{\pgfqpoint{0.029463in}{0.029463in}}%
\pgfpathcurveto{\pgfqpoint{0.021649in}{0.037276in}}{\pgfqpoint{0.011050in}{0.041667in}}{\pgfqpoint{0.000000in}{0.041667in}}%
\pgfpathcurveto{\pgfqpoint{-0.011050in}{0.041667in}}{\pgfqpoint{-0.021649in}{0.037276in}}{\pgfqpoint{-0.029463in}{0.029463in}}%
\pgfpathcurveto{\pgfqpoint{-0.037276in}{0.021649in}}{\pgfqpoint{-0.041667in}{0.011050in}}{\pgfqpoint{-0.041667in}{0.000000in}}%
\pgfpathcurveto{\pgfqpoint{-0.041667in}{-0.011050in}}{\pgfqpoint{-0.037276in}{-0.021649in}}{\pgfqpoint{-0.029463in}{-0.029463in}}%
\pgfpathcurveto{\pgfqpoint{-0.021649in}{-0.037276in}}{\pgfqpoint{-0.011050in}{-0.041667in}}{\pgfqpoint{0.000000in}{-0.041667in}}%
\pgfpathclose%
\pgfusepath{stroke,fill}%
}%
\begin{pgfscope}%
\pgfsys@transformshift{3.018054in}{2.239064in}%
\pgfsys@useobject{currentmarker}{}%
\end{pgfscope}%
\end{pgfscope}%
\begin{pgfscope}%
\definecolor{textcolor}{rgb}{0.000000,0.000000,0.000000}%
\pgfsetstrokecolor{textcolor}%
\pgfsetfillcolor{textcolor}%
\pgftext[x=3.243054in,y=2.195314in,left,base]{\color{textcolor}\rmfamily\fontsize{9.000000}{10.800000}\selectfont \(\displaystyle \gamma_1 =\) -0.274 }%
\end{pgfscope}%
\begin{pgfscope}%
\pgfsetbuttcap%
\pgfsetroundjoin%
\definecolor{currentfill}{rgb}{0.850000,0.324000,0.098000}%
\pgfsetfillcolor{currentfill}%
\pgfsetlinewidth{1.003750pt}%
\definecolor{currentstroke}{rgb}{0.850000,0.324000,0.098000}%
\pgfsetstrokecolor{currentstroke}%
\pgfsetdash{}{0pt}%
\pgfsys@defobject{currentmarker}{\pgfqpoint{-0.041667in}{-0.041667in}}{\pgfqpoint{0.041667in}{0.041667in}}{%
\pgfpathmoveto{\pgfqpoint{-0.041667in}{0.000000in}}%
\pgfpathlineto{\pgfqpoint{0.041667in}{0.000000in}}%
\pgfpathmoveto{\pgfqpoint{0.000000in}{-0.041667in}}%
\pgfpathlineto{\pgfqpoint{0.000000in}{0.041667in}}%
\pgfusepath{stroke,fill}%
}%
\begin{pgfscope}%
\pgfsys@transformshift{3.018054in}{2.064759in}%
\pgfsys@useobject{currentmarker}{}%
\end{pgfscope}%
\end{pgfscope}%
\begin{pgfscope}%
\definecolor{textcolor}{rgb}{0.000000,0.000000,0.000000}%
\pgfsetstrokecolor{textcolor}%
\pgfsetfillcolor{textcolor}%
\pgftext[x=3.243054in,y=2.021009in,left,base]{\color{textcolor}\rmfamily\fontsize{9.000000}{10.800000}\selectfont \(\displaystyle \gamma_2 =\) -0.15}%
\end{pgfscope}%
\begin{pgfscope}%
\pgfsetbuttcap%
\pgfsetmiterjoin%
\definecolor{currentfill}{rgb}{0.000000,0.000000,0.000000}%
\pgfsetfillcolor{currentfill}%
\pgfsetfillopacity{0.000000}%
\pgfsetlinewidth{1.003750pt}%
\definecolor{currentstroke}{rgb}{0.000000,0.500000,0.000000}%
\pgfsetstrokecolor{currentstroke}%
\pgfsetdash{}{0pt}%
\pgfsys@defobject{currentmarker}{\pgfqpoint{-0.041667in}{-0.041667in}}{\pgfqpoint{0.041667in}{0.041667in}}{%
\pgfpathmoveto{\pgfqpoint{-0.041667in}{-0.041667in}}%
\pgfpathlineto{\pgfqpoint{0.041667in}{-0.041667in}}%
\pgfpathlineto{\pgfqpoint{0.041667in}{0.041667in}}%
\pgfpathlineto{\pgfqpoint{-0.041667in}{0.041667in}}%
\pgfpathclose%
\pgfusepath{stroke,fill}%
}%
\begin{pgfscope}%
\pgfsys@transformshift{3.018054in}{1.890453in}%
\pgfsys@useobject{currentmarker}{}%
\end{pgfscope}%
\end{pgfscope}%
\begin{pgfscope}%
\definecolor{textcolor}{rgb}{0.000000,0.000000,0.000000}%
\pgfsetstrokecolor{textcolor}%
\pgfsetfillcolor{textcolor}%
\pgftext[x=3.243054in,y=1.846703in,left,base]{\color{textcolor}\rmfamily\fontsize{9.000000}{10.800000}\selectfont \(\displaystyle \gamma_3\) = 0.133 }%
\end{pgfscope}%
\begin{pgfscope}%
\pgfsetbuttcap%
\pgfsetroundjoin%
\definecolor{currentfill}{rgb}{0.494000,0.184000,0.556000}%
\pgfsetfillcolor{currentfill}%
\pgfsetlinewidth{1.003750pt}%
\definecolor{currentstroke}{rgb}{0.494000,0.184000,0.556000}%
\pgfsetstrokecolor{currentstroke}%
\pgfsetdash{}{0pt}%
\pgfsys@defobject{currentmarker}{\pgfqpoint{-0.041667in}{-0.041667in}}{\pgfqpoint{0.041667in}{0.041667in}}{%
\pgfpathmoveto{\pgfqpoint{-0.041667in}{-0.041667in}}%
\pgfpathlineto{\pgfqpoint{0.041667in}{0.041667in}}%
\pgfpathmoveto{\pgfqpoint{-0.041667in}{0.041667in}}%
\pgfpathlineto{\pgfqpoint{0.041667in}{-0.041667in}}%
\pgfusepath{stroke,fill}%
}%
\begin{pgfscope}%
\pgfsys@transformshift{3.018054in}{1.716147in}%
\pgfsys@useobject{currentmarker}{}%
\end{pgfscope}%
\end{pgfscope}%
\begin{pgfscope}%
\definecolor{textcolor}{rgb}{0.000000,0.000000,0.000000}%
\pgfsetstrokecolor{textcolor}%
\pgfsetfillcolor{textcolor}%
\pgftext[x=3.243054in,y=1.672397in,left,base]{\color{textcolor}\rmfamily\fontsize{9.000000}{10.800000}\selectfont \(\displaystyle \gamma_4 = \) -0.221}%
\end{pgfscope}%
\end{pgfpicture}%
\makeatother%
\endgroup%
}
					\caption{Relación entre RMSE y CRB para $\gamma$.}
					\label{Fig:CRB_gamma_ex1}
				\end{subfigure}
				\caption{Relación entre RMSE y CRB en función de SNR. Lineas punteadas corresponden a \cite{Andersson2014}; linea solida corresponde a \emph{Shift-and-Zoom}.}
				\label{Fig:CRB_ex1}
			\end{figure}
			
			La longitud de la señal observada también es significante cuando se estiman los factores de amortiguamiento. Este problema se vuelve crítico cuando el orden del modelo estimado en mayor que el real. En este caso, los factores de amortiguamiento erróneos pueden afectar la estimación de los parámetros reales. En \cite{Hasan1982} se aborda este problema utilizando $\gamma_i' = \gamma_i/(m+n-1)$.
			
			Para analizar el efecto del factor de decimación $Q$, se comparó el RMSE entre la señal estimada y la señal verdadera para diferentes valores de $Q$. Los resultados se muestran en la tabla \ref{Table:RMSE_Q}. El mínimo RMSE se obtiene para $Q=4$. A medida que se aumenta $Q$, el RMSE aumenta debido a que se usan menos muestras. Por otro lado, cuando $Q$ decrece, aunque se analicen más muestras, se pierde el efecto de separación de frecuencia.
			
			\begin{table}[h!]
				\centering
				\begin{tabular}{llllllll}
					$Q$ & 1 & 2 & 4 & 7 & 9 & 14 & 15 \\ \hline
					RMSE & 15.27 & 15.26 & 7.82 & 11.25 & 12.09& 14.54 & 15.82 \\
				\end{tabular}
				\caption{RMSE en función del factor de decimación.}
				\label{Table:RMSE_Q}
			\end{table}						
			
		\subsection{Modos agrupados}
		
			Como segundo ejemplo, se considera la suma de 25 exponenciales amortiguadas con distintas amplitudes. Este ejemplo fue presentado en \cite{Cuyt2018} y los parámetros corresponden a una espectroscopia de resonancia magnética (MRS). El período de muestreo es $T_s = 0.4883\cdot 10^{-3}$ segundos. En la tabla \ref{Table:Cuyt} se muestran las frecuencias, factores de amortiguamiento y amplitudes para cada modo complejo. En la Fig.~\ref{Fig:loc_ava} se muestran las posiciones de los distintos $z_i = e^{(\gamma_i+\jmath 2\pi\nu_i)T_s}$, $i =1,\ldots,25$. Los 25 modos se pueden agrupar en 6 grupos disjuntos.	
			
			\begin{figure}[t]
				\centering
				%% Creator: Matplotlib, PGF backend
%%
%% To include the figure in your LaTeX document, write
%%   \input{<filename>.pgf}
%%
%% Make sure the required packages are loaded in your preamble
%%   \usepackage{pgf}
%%
%% and, on pdftex
%%   \usepackage[utf8]{inputenc}\DeclareUnicodeCharacter{2212}{-}
%%
%% or, on luatex and xetex
%%   \usepackage{unicode-math}
%%
%% Figures using additional raster images can only be included by \input if
%% they are in the same directory as the main LaTeX file. For loading figures
%% from other directories you can use the `import` package
%%   \usepackage{import}
%%
%% and then include the figures with
%%   \import{<path to file>}{<filename>.pgf}
%%
%% Matplotlib used the following preamble
%%   \usepackage[utf8x]{inputenc}
%%   \usepackage[T1]{fontenc}
%%   \usepackage{amsmath,amssymb,amsfonts}
%%
\begingroup%
\makeatletter%
\begin{pgfpicture}%
\pgfpathrectangle{\pgfpointorigin}{\pgfqpoint{4.137360in}{2.555314in}}%
\pgfusepath{use as bounding box, clip}%
\begin{pgfscope}%
\pgfsetbuttcap%
\pgfsetmiterjoin%
\definecolor{currentfill}{rgb}{1.000000,1.000000,1.000000}%
\pgfsetfillcolor{currentfill}%
\pgfsetlinewidth{0.000000pt}%
\definecolor{currentstroke}{rgb}{1.000000,1.000000,1.000000}%
\pgfsetstrokecolor{currentstroke}%
\pgfsetdash{}{0pt}%
\pgfpathmoveto{\pgfqpoint{0.000000in}{0.000000in}}%
\pgfpathlineto{\pgfqpoint{4.137360in}{0.000000in}}%
\pgfpathlineto{\pgfqpoint{4.137360in}{2.555314in}}%
\pgfpathlineto{\pgfqpoint{0.000000in}{2.555314in}}%
\pgfpathclose%
\pgfusepath{fill}%
\end{pgfscope}%
\begin{pgfscope}%
\pgfsetbuttcap%
\pgfsetmiterjoin%
\definecolor{currentfill}{rgb}{1.000000,1.000000,1.000000}%
\pgfsetfillcolor{currentfill}%
\pgfsetlinewidth{0.000000pt}%
\definecolor{currentstroke}{rgb}{0.000000,0.000000,0.000000}%
\pgfsetstrokecolor{currentstroke}%
\pgfsetstrokeopacity{0.000000}%
\pgfsetdash{}{0pt}%
\pgfpathmoveto{\pgfqpoint{0.634340in}{0.489791in}}%
\pgfpathlineto{\pgfqpoint{4.037360in}{0.489791in}}%
\pgfpathlineto{\pgfqpoint{4.037360in}{2.455314in}}%
\pgfpathlineto{\pgfqpoint{0.634340in}{2.455314in}}%
\pgfpathclose%
\pgfusepath{fill}%
\end{pgfscope}%
\begin{pgfscope}%
\pgfpathrectangle{\pgfqpoint{0.634340in}{0.489791in}}{\pgfqpoint{3.403020in}{1.965523in}}%
\pgfusepath{clip}%
\pgfsetrectcap%
\pgfsetroundjoin%
\pgfsetlinewidth{0.803000pt}%
\definecolor{currentstroke}{rgb}{0.690196,0.690196,0.690196}%
\pgfsetstrokecolor{currentstroke}%
\pgfsetdash{}{0pt}%
\pgfpathmoveto{\pgfqpoint{0.634340in}{0.489791in}}%
\pgfpathlineto{\pgfqpoint{0.634340in}{2.455314in}}%
\pgfusepath{stroke}%
\end{pgfscope}%
\begin{pgfscope}%
\pgfsetbuttcap%
\pgfsetroundjoin%
\definecolor{currentfill}{rgb}{0.000000,0.000000,0.000000}%
\pgfsetfillcolor{currentfill}%
\pgfsetlinewidth{0.803000pt}%
\definecolor{currentstroke}{rgb}{0.000000,0.000000,0.000000}%
\pgfsetstrokecolor{currentstroke}%
\pgfsetdash{}{0pt}%
\pgfsys@defobject{currentmarker}{\pgfqpoint{0.000000in}{-0.048611in}}{\pgfqpoint{0.000000in}{0.000000in}}{%
\pgfpathmoveto{\pgfqpoint{0.000000in}{0.000000in}}%
\pgfpathlineto{\pgfqpoint{0.000000in}{-0.048611in}}%
\pgfusepath{stroke,fill}%
}%
\begin{pgfscope}%
\pgfsys@transformshift{0.634340in}{0.489791in}%
\pgfsys@useobject{currentmarker}{}%
\end{pgfscope}%
\end{pgfscope}%
\begin{pgfscope}%
\definecolor{textcolor}{rgb}{0.000000,0.000000,0.000000}%
\pgfsetstrokecolor{textcolor}%
\pgfsetfillcolor{textcolor}%
\pgftext[x=0.634340in,y=0.392569in,,top]{\color{textcolor}\rmfamily\fontsize{8.000000}{9.600000}\selectfont \(\displaystyle {0.2}\)}%
\end{pgfscope}%
\begin{pgfscope}%
\pgfpathrectangle{\pgfqpoint{0.634340in}{0.489791in}}{\pgfqpoint{3.403020in}{1.965523in}}%
\pgfusepath{clip}%
\pgfsetrectcap%
\pgfsetroundjoin%
\pgfsetlinewidth{0.803000pt}%
\definecolor{currentstroke}{rgb}{0.690196,0.690196,0.690196}%
\pgfsetstrokecolor{currentstroke}%
\pgfsetdash{}{0pt}%
\pgfpathmoveto{\pgfqpoint{1.390567in}{0.489791in}}%
\pgfpathlineto{\pgfqpoint{1.390567in}{2.455314in}}%
\pgfusepath{stroke}%
\end{pgfscope}%
\begin{pgfscope}%
\pgfsetbuttcap%
\pgfsetroundjoin%
\definecolor{currentfill}{rgb}{0.000000,0.000000,0.000000}%
\pgfsetfillcolor{currentfill}%
\pgfsetlinewidth{0.803000pt}%
\definecolor{currentstroke}{rgb}{0.000000,0.000000,0.000000}%
\pgfsetstrokecolor{currentstroke}%
\pgfsetdash{}{0pt}%
\pgfsys@defobject{currentmarker}{\pgfqpoint{0.000000in}{-0.048611in}}{\pgfqpoint{0.000000in}{0.000000in}}{%
\pgfpathmoveto{\pgfqpoint{0.000000in}{0.000000in}}%
\pgfpathlineto{\pgfqpoint{0.000000in}{-0.048611in}}%
\pgfusepath{stroke,fill}%
}%
\begin{pgfscope}%
\pgfsys@transformshift{1.390567in}{0.489791in}%
\pgfsys@useobject{currentmarker}{}%
\end{pgfscope}%
\end{pgfscope}%
\begin{pgfscope}%
\definecolor{textcolor}{rgb}{0.000000,0.000000,0.000000}%
\pgfsetstrokecolor{textcolor}%
\pgfsetfillcolor{textcolor}%
\pgftext[x=1.390567in,y=0.392569in,,top]{\color{textcolor}\rmfamily\fontsize{8.000000}{9.600000}\selectfont \(\displaystyle {0.4}\)}%
\end{pgfscope}%
\begin{pgfscope}%
\pgfpathrectangle{\pgfqpoint{0.634340in}{0.489791in}}{\pgfqpoint{3.403020in}{1.965523in}}%
\pgfusepath{clip}%
\pgfsetrectcap%
\pgfsetroundjoin%
\pgfsetlinewidth{0.803000pt}%
\definecolor{currentstroke}{rgb}{0.690196,0.690196,0.690196}%
\pgfsetstrokecolor{currentstroke}%
\pgfsetdash{}{0pt}%
\pgfpathmoveto{\pgfqpoint{2.146793in}{0.489791in}}%
\pgfpathlineto{\pgfqpoint{2.146793in}{2.455314in}}%
\pgfusepath{stroke}%
\end{pgfscope}%
\begin{pgfscope}%
\pgfsetbuttcap%
\pgfsetroundjoin%
\definecolor{currentfill}{rgb}{0.000000,0.000000,0.000000}%
\pgfsetfillcolor{currentfill}%
\pgfsetlinewidth{0.803000pt}%
\definecolor{currentstroke}{rgb}{0.000000,0.000000,0.000000}%
\pgfsetstrokecolor{currentstroke}%
\pgfsetdash{}{0pt}%
\pgfsys@defobject{currentmarker}{\pgfqpoint{0.000000in}{-0.048611in}}{\pgfqpoint{0.000000in}{0.000000in}}{%
\pgfpathmoveto{\pgfqpoint{0.000000in}{0.000000in}}%
\pgfpathlineto{\pgfqpoint{0.000000in}{-0.048611in}}%
\pgfusepath{stroke,fill}%
}%
\begin{pgfscope}%
\pgfsys@transformshift{2.146793in}{0.489791in}%
\pgfsys@useobject{currentmarker}{}%
\end{pgfscope}%
\end{pgfscope}%
\begin{pgfscope}%
\definecolor{textcolor}{rgb}{0.000000,0.000000,0.000000}%
\pgfsetstrokecolor{textcolor}%
\pgfsetfillcolor{textcolor}%
\pgftext[x=2.146793in,y=0.392569in,,top]{\color{textcolor}\rmfamily\fontsize{8.000000}{9.600000}\selectfont \(\displaystyle {0.6}\)}%
\end{pgfscope}%
\begin{pgfscope}%
\pgfpathrectangle{\pgfqpoint{0.634340in}{0.489791in}}{\pgfqpoint{3.403020in}{1.965523in}}%
\pgfusepath{clip}%
\pgfsetrectcap%
\pgfsetroundjoin%
\pgfsetlinewidth{0.803000pt}%
\definecolor{currentstroke}{rgb}{0.690196,0.690196,0.690196}%
\pgfsetstrokecolor{currentstroke}%
\pgfsetdash{}{0pt}%
\pgfpathmoveto{\pgfqpoint{2.903020in}{0.489791in}}%
\pgfpathlineto{\pgfqpoint{2.903020in}{2.455314in}}%
\pgfusepath{stroke}%
\end{pgfscope}%
\begin{pgfscope}%
\pgfsetbuttcap%
\pgfsetroundjoin%
\definecolor{currentfill}{rgb}{0.000000,0.000000,0.000000}%
\pgfsetfillcolor{currentfill}%
\pgfsetlinewidth{0.803000pt}%
\definecolor{currentstroke}{rgb}{0.000000,0.000000,0.000000}%
\pgfsetstrokecolor{currentstroke}%
\pgfsetdash{}{0pt}%
\pgfsys@defobject{currentmarker}{\pgfqpoint{0.000000in}{-0.048611in}}{\pgfqpoint{0.000000in}{0.000000in}}{%
\pgfpathmoveto{\pgfqpoint{0.000000in}{0.000000in}}%
\pgfpathlineto{\pgfqpoint{0.000000in}{-0.048611in}}%
\pgfusepath{stroke,fill}%
}%
\begin{pgfscope}%
\pgfsys@transformshift{2.903020in}{0.489791in}%
\pgfsys@useobject{currentmarker}{}%
\end{pgfscope}%
\end{pgfscope}%
\begin{pgfscope}%
\definecolor{textcolor}{rgb}{0.000000,0.000000,0.000000}%
\pgfsetstrokecolor{textcolor}%
\pgfsetfillcolor{textcolor}%
\pgftext[x=2.903020in,y=0.392569in,,top]{\color{textcolor}\rmfamily\fontsize{8.000000}{9.600000}\selectfont \(\displaystyle {0.8}\)}%
\end{pgfscope}%
\begin{pgfscope}%
\pgfpathrectangle{\pgfqpoint{0.634340in}{0.489791in}}{\pgfqpoint{3.403020in}{1.965523in}}%
\pgfusepath{clip}%
\pgfsetrectcap%
\pgfsetroundjoin%
\pgfsetlinewidth{0.803000pt}%
\definecolor{currentstroke}{rgb}{0.690196,0.690196,0.690196}%
\pgfsetstrokecolor{currentstroke}%
\pgfsetdash{}{0pt}%
\pgfpathmoveto{\pgfqpoint{3.659247in}{0.489791in}}%
\pgfpathlineto{\pgfqpoint{3.659247in}{2.455314in}}%
\pgfusepath{stroke}%
\end{pgfscope}%
\begin{pgfscope}%
\pgfsetbuttcap%
\pgfsetroundjoin%
\definecolor{currentfill}{rgb}{0.000000,0.000000,0.000000}%
\pgfsetfillcolor{currentfill}%
\pgfsetlinewidth{0.803000pt}%
\definecolor{currentstroke}{rgb}{0.000000,0.000000,0.000000}%
\pgfsetstrokecolor{currentstroke}%
\pgfsetdash{}{0pt}%
\pgfsys@defobject{currentmarker}{\pgfqpoint{0.000000in}{-0.048611in}}{\pgfqpoint{0.000000in}{0.000000in}}{%
\pgfpathmoveto{\pgfqpoint{0.000000in}{0.000000in}}%
\pgfpathlineto{\pgfqpoint{0.000000in}{-0.048611in}}%
\pgfusepath{stroke,fill}%
}%
\begin{pgfscope}%
\pgfsys@transformshift{3.659247in}{0.489791in}%
\pgfsys@useobject{currentmarker}{}%
\end{pgfscope}%
\end{pgfscope}%
\begin{pgfscope}%
\definecolor{textcolor}{rgb}{0.000000,0.000000,0.000000}%
\pgfsetstrokecolor{textcolor}%
\pgfsetfillcolor{textcolor}%
\pgftext[x=3.659247in,y=0.392569in,,top]{\color{textcolor}\rmfamily\fontsize{8.000000}{9.600000}\selectfont \(\displaystyle {1.0}\)}%
\end{pgfscope}%
\begin{pgfscope}%
\definecolor{textcolor}{rgb}{0.000000,0.000000,0.000000}%
\pgfsetstrokecolor{textcolor}%
\pgfsetfillcolor{textcolor}%
\pgftext[x=2.335850in,y=0.238889in,,top]{\color{textcolor}\rmfamily\fontsize{10.000000}{12.000000}\selectfont \(\displaystyle \Re(z)\)}%
\end{pgfscope}%
\begin{pgfscope}%
\pgfpathrectangle{\pgfqpoint{0.634340in}{0.489791in}}{\pgfqpoint{3.403020in}{1.965523in}}%
\pgfusepath{clip}%
\pgfsetrectcap%
\pgfsetroundjoin%
\pgfsetlinewidth{0.803000pt}%
\definecolor{currentstroke}{rgb}{0.690196,0.690196,0.690196}%
\pgfsetstrokecolor{currentstroke}%
\pgfsetdash{}{0pt}%
\pgfpathmoveto{\pgfqpoint{0.634340in}{0.489791in}}%
\pgfpathlineto{\pgfqpoint{4.037360in}{0.489791in}}%
\pgfusepath{stroke}%
\end{pgfscope}%
\begin{pgfscope}%
\pgfsetbuttcap%
\pgfsetroundjoin%
\definecolor{currentfill}{rgb}{0.000000,0.000000,0.000000}%
\pgfsetfillcolor{currentfill}%
\pgfsetlinewidth{0.803000pt}%
\definecolor{currentstroke}{rgb}{0.000000,0.000000,0.000000}%
\pgfsetstrokecolor{currentstroke}%
\pgfsetdash{}{0pt}%
\pgfsys@defobject{currentmarker}{\pgfqpoint{-0.048611in}{0.000000in}}{\pgfqpoint{-0.000000in}{0.000000in}}{%
\pgfpathmoveto{\pgfqpoint{-0.000000in}{0.000000in}}%
\pgfpathlineto{\pgfqpoint{-0.048611in}{0.000000in}}%
\pgfusepath{stroke,fill}%
}%
\begin{pgfscope}%
\pgfsys@transformshift{0.634340in}{0.489791in}%
\pgfsys@useobject{currentmarker}{}%
\end{pgfscope}%
\end{pgfscope}%
\begin{pgfscope}%
\definecolor{textcolor}{rgb}{0.000000,0.000000,0.000000}%
\pgfsetstrokecolor{textcolor}%
\pgfsetfillcolor{textcolor}%
\pgftext[x=0.294444in, y=0.451529in, left, base]{\color{textcolor}\rmfamily\fontsize{8.000000}{9.600000}\selectfont \(\displaystyle {-1.0}\)}%
\end{pgfscope}%
\begin{pgfscope}%
\pgfpathrectangle{\pgfqpoint{0.634340in}{0.489791in}}{\pgfqpoint{3.403020in}{1.965523in}}%
\pgfusepath{clip}%
\pgfsetrectcap%
\pgfsetroundjoin%
\pgfsetlinewidth{0.803000pt}%
\definecolor{currentstroke}{rgb}{0.690196,0.690196,0.690196}%
\pgfsetstrokecolor{currentstroke}%
\pgfsetdash{}{0pt}%
\pgfpathmoveto{\pgfqpoint{0.634340in}{0.957773in}}%
\pgfpathlineto{\pgfqpoint{4.037360in}{0.957773in}}%
\pgfusepath{stroke}%
\end{pgfscope}%
\begin{pgfscope}%
\pgfsetbuttcap%
\pgfsetroundjoin%
\definecolor{currentfill}{rgb}{0.000000,0.000000,0.000000}%
\pgfsetfillcolor{currentfill}%
\pgfsetlinewidth{0.803000pt}%
\definecolor{currentstroke}{rgb}{0.000000,0.000000,0.000000}%
\pgfsetstrokecolor{currentstroke}%
\pgfsetdash{}{0pt}%
\pgfsys@defobject{currentmarker}{\pgfqpoint{-0.048611in}{0.000000in}}{\pgfqpoint{-0.000000in}{0.000000in}}{%
\pgfpathmoveto{\pgfqpoint{-0.000000in}{0.000000in}}%
\pgfpathlineto{\pgfqpoint{-0.048611in}{0.000000in}}%
\pgfusepath{stroke,fill}%
}%
\begin{pgfscope}%
\pgfsys@transformshift{0.634340in}{0.957773in}%
\pgfsys@useobject{currentmarker}{}%
\end{pgfscope}%
\end{pgfscope}%
\begin{pgfscope}%
\definecolor{textcolor}{rgb}{0.000000,0.000000,0.000000}%
\pgfsetstrokecolor{textcolor}%
\pgfsetfillcolor{textcolor}%
\pgftext[x=0.294444in, y=0.919510in, left, base]{\color{textcolor}\rmfamily\fontsize{8.000000}{9.600000}\selectfont \(\displaystyle {-0.5}\)}%
\end{pgfscope}%
\begin{pgfscope}%
\pgfpathrectangle{\pgfqpoint{0.634340in}{0.489791in}}{\pgfqpoint{3.403020in}{1.965523in}}%
\pgfusepath{clip}%
\pgfsetrectcap%
\pgfsetroundjoin%
\pgfsetlinewidth{0.803000pt}%
\definecolor{currentstroke}{rgb}{0.690196,0.690196,0.690196}%
\pgfsetstrokecolor{currentstroke}%
\pgfsetdash{}{0pt}%
\pgfpathmoveto{\pgfqpoint{0.634340in}{1.425754in}}%
\pgfpathlineto{\pgfqpoint{4.037360in}{1.425754in}}%
\pgfusepath{stroke}%
\end{pgfscope}%
\begin{pgfscope}%
\pgfsetbuttcap%
\pgfsetroundjoin%
\definecolor{currentfill}{rgb}{0.000000,0.000000,0.000000}%
\pgfsetfillcolor{currentfill}%
\pgfsetlinewidth{0.803000pt}%
\definecolor{currentstroke}{rgb}{0.000000,0.000000,0.000000}%
\pgfsetstrokecolor{currentstroke}%
\pgfsetdash{}{0pt}%
\pgfsys@defobject{currentmarker}{\pgfqpoint{-0.048611in}{0.000000in}}{\pgfqpoint{-0.000000in}{0.000000in}}{%
\pgfpathmoveto{\pgfqpoint{-0.000000in}{0.000000in}}%
\pgfpathlineto{\pgfqpoint{-0.048611in}{0.000000in}}%
\pgfusepath{stroke,fill}%
}%
\begin{pgfscope}%
\pgfsys@transformshift{0.634340in}{1.425754in}%
\pgfsys@useobject{currentmarker}{}%
\end{pgfscope}%
\end{pgfscope}%
\begin{pgfscope}%
\definecolor{textcolor}{rgb}{0.000000,0.000000,0.000000}%
\pgfsetstrokecolor{textcolor}%
\pgfsetfillcolor{textcolor}%
\pgftext[x=0.386267in, y=1.387492in, left, base]{\color{textcolor}\rmfamily\fontsize{8.000000}{9.600000}\selectfont \(\displaystyle {0.0}\)}%
\end{pgfscope}%
\begin{pgfscope}%
\pgfpathrectangle{\pgfqpoint{0.634340in}{0.489791in}}{\pgfqpoint{3.403020in}{1.965523in}}%
\pgfusepath{clip}%
\pgfsetrectcap%
\pgfsetroundjoin%
\pgfsetlinewidth{0.803000pt}%
\definecolor{currentstroke}{rgb}{0.690196,0.690196,0.690196}%
\pgfsetstrokecolor{currentstroke}%
\pgfsetdash{}{0pt}%
\pgfpathmoveto{\pgfqpoint{0.634340in}{1.893736in}}%
\pgfpathlineto{\pgfqpoint{4.037360in}{1.893736in}}%
\pgfusepath{stroke}%
\end{pgfscope}%
\begin{pgfscope}%
\pgfsetbuttcap%
\pgfsetroundjoin%
\definecolor{currentfill}{rgb}{0.000000,0.000000,0.000000}%
\pgfsetfillcolor{currentfill}%
\pgfsetlinewidth{0.803000pt}%
\definecolor{currentstroke}{rgb}{0.000000,0.000000,0.000000}%
\pgfsetstrokecolor{currentstroke}%
\pgfsetdash{}{0pt}%
\pgfsys@defobject{currentmarker}{\pgfqpoint{-0.048611in}{0.000000in}}{\pgfqpoint{-0.000000in}{0.000000in}}{%
\pgfpathmoveto{\pgfqpoint{-0.000000in}{0.000000in}}%
\pgfpathlineto{\pgfqpoint{-0.048611in}{0.000000in}}%
\pgfusepath{stroke,fill}%
}%
\begin{pgfscope}%
\pgfsys@transformshift{0.634340in}{1.893736in}%
\pgfsys@useobject{currentmarker}{}%
\end{pgfscope}%
\end{pgfscope}%
\begin{pgfscope}%
\definecolor{textcolor}{rgb}{0.000000,0.000000,0.000000}%
\pgfsetstrokecolor{textcolor}%
\pgfsetfillcolor{textcolor}%
\pgftext[x=0.386267in, y=1.855474in, left, base]{\color{textcolor}\rmfamily\fontsize{8.000000}{9.600000}\selectfont \(\displaystyle {0.5}\)}%
\end{pgfscope}%
\begin{pgfscope}%
\pgfpathrectangle{\pgfqpoint{0.634340in}{0.489791in}}{\pgfqpoint{3.403020in}{1.965523in}}%
\pgfusepath{clip}%
\pgfsetrectcap%
\pgfsetroundjoin%
\pgfsetlinewidth{0.803000pt}%
\definecolor{currentstroke}{rgb}{0.690196,0.690196,0.690196}%
\pgfsetstrokecolor{currentstroke}%
\pgfsetdash{}{0pt}%
\pgfpathmoveto{\pgfqpoint{0.634340in}{2.361718in}}%
\pgfpathlineto{\pgfqpoint{4.037360in}{2.361718in}}%
\pgfusepath{stroke}%
\end{pgfscope}%
\begin{pgfscope}%
\pgfsetbuttcap%
\pgfsetroundjoin%
\definecolor{currentfill}{rgb}{0.000000,0.000000,0.000000}%
\pgfsetfillcolor{currentfill}%
\pgfsetlinewidth{0.803000pt}%
\definecolor{currentstroke}{rgb}{0.000000,0.000000,0.000000}%
\pgfsetstrokecolor{currentstroke}%
\pgfsetdash{}{0pt}%
\pgfsys@defobject{currentmarker}{\pgfqpoint{-0.048611in}{0.000000in}}{\pgfqpoint{-0.000000in}{0.000000in}}{%
\pgfpathmoveto{\pgfqpoint{-0.000000in}{0.000000in}}%
\pgfpathlineto{\pgfqpoint{-0.048611in}{0.000000in}}%
\pgfusepath{stroke,fill}%
}%
\begin{pgfscope}%
\pgfsys@transformshift{0.634340in}{2.361718in}%
\pgfsys@useobject{currentmarker}{}%
\end{pgfscope}%
\end{pgfscope}%
\begin{pgfscope}%
\definecolor{textcolor}{rgb}{0.000000,0.000000,0.000000}%
\pgfsetstrokecolor{textcolor}%
\pgfsetfillcolor{textcolor}%
\pgftext[x=0.386267in, y=2.323456in, left, base]{\color{textcolor}\rmfamily\fontsize{8.000000}{9.600000}\selectfont \(\displaystyle {1.0}\)}%
\end{pgfscope}%
\begin{pgfscope}%
\definecolor{textcolor}{rgb}{0.000000,0.000000,0.000000}%
\pgfsetstrokecolor{textcolor}%
\pgfsetfillcolor{textcolor}%
\pgftext[x=0.238889in,y=1.472553in,,bottom,rotate=90.000000]{\color{textcolor}\rmfamily\fontsize{10.000000}{12.000000}\selectfont \(\displaystyle \Im(z)\)}%
\end{pgfscope}%
\begin{pgfscope}%
\pgfpathrectangle{\pgfqpoint{0.634340in}{0.489791in}}{\pgfqpoint{3.403020in}{1.965523in}}%
\pgfusepath{clip}%
\pgfsetbuttcap%
\pgfsetroundjoin%
\definecolor{currentfill}{rgb}{0.121569,0.466667,0.705882}%
\pgfsetfillcolor{currentfill}%
\pgfsetlinewidth{1.003750pt}%
\definecolor{currentstroke}{rgb}{0.121569,0.466667,0.705882}%
\pgfsetstrokecolor{currentstroke}%
\pgfsetdash{}{0pt}%
\pgfsys@defobject{currentmarker}{\pgfqpoint{-0.041667in}{-0.041667in}}{\pgfqpoint{0.041667in}{0.041667in}}{%
\pgfpathmoveto{\pgfqpoint{-0.041667in}{-0.041667in}}%
\pgfpathlineto{\pgfqpoint{0.041667in}{0.041667in}}%
\pgfpathmoveto{\pgfqpoint{-0.041667in}{0.041667in}}%
\pgfpathlineto{\pgfqpoint{0.041667in}{-0.041667in}}%
\pgfusepath{stroke,fill}%
}%
\begin{pgfscope}%
\pgfsys@transformshift{1.640999in}{0.597779in}%
\pgfsys@useobject{currentmarker}{}%
\end{pgfscope}%
\begin{pgfscope}%
\pgfsys@transformshift{1.660208in}{0.600334in}%
\pgfsys@useobject{currentmarker}{}%
\end{pgfscope}%
\begin{pgfscope}%
\pgfsys@transformshift{1.686778in}{0.603867in}%
\pgfsys@useobject{currentmarker}{}%
\end{pgfscope}%
\begin{pgfscope}%
\pgfsys@transformshift{1.701115in}{0.605815in}%
\pgfsys@useobject{currentmarker}{}%
\end{pgfscope}%
\begin{pgfscope}%
\pgfsys@transformshift{1.711260in}{0.607207in}%
\pgfsys@useobject{currentmarker}{}%
\end{pgfscope}%
\begin{pgfscope}%
\pgfsys@transformshift{1.731652in}{0.610006in}%
\pgfsys@useobject{currentmarker}{}%
\end{pgfscope}%
\begin{pgfscope}%
\pgfsys@transformshift{3.350936in}{1.055692in}%
\pgfsys@useobject{currentmarker}{}%
\end{pgfscope}%
\begin{pgfscope}%
\pgfsys@transformshift{3.355923in}{1.058600in}%
\pgfsys@useobject{currentmarker}{}%
\end{pgfscope}%
\begin{pgfscope}%
\pgfsys@transformshift{3.362090in}{1.062287in}%
\pgfsys@useobject{currentmarker}{}%
\end{pgfscope}%
\begin{pgfscope}%
\pgfsys@transformshift{3.365936in}{1.064503in}%
\pgfsys@useobject{currentmarker}{}%
\end{pgfscope}%
\begin{pgfscope}%
\pgfsys@transformshift{3.370701in}{1.067371in}%
\pgfsys@useobject{currentmarker}{}%
\end{pgfscope}%
\begin{pgfscope}%
\pgfsys@transformshift{3.655521in}{1.466228in}%
\pgfsys@useobject{currentmarker}{}%
\end{pgfscope}%
\begin{pgfscope}%
\pgfsys@transformshift{3.654565in}{1.471132in}%
\pgfsys@useobject{currentmarker}{}%
\end{pgfscope}%
\begin{pgfscope}%
\pgfsys@transformshift{3.454579in}{1.729407in}%
\pgfsys@useobject{currentmarker}{}%
\end{pgfscope}%
\begin{pgfscope}%
\pgfsys@transformshift{3.444885in}{1.736294in}%
\pgfsys@useobject{currentmarker}{}%
\end{pgfscope}%
\begin{pgfscope}%
\pgfsys@transformshift{3.436085in}{1.742407in}%
\pgfsys@useobject{currentmarker}{}%
\end{pgfscope}%
\begin{pgfscope}%
\pgfsys@transformshift{3.430195in}{1.746460in}%
\pgfsys@useobject{currentmarker}{}%
\end{pgfscope}%
\begin{pgfscope}%
\pgfsys@transformshift{3.388515in}{1.773285in}%
\pgfsys@useobject{currentmarker}{}%
\end{pgfscope}%
\begin{pgfscope}%
\pgfsys@transformshift{3.381641in}{1.777583in}%
\pgfsys@useobject{currentmarker}{}%
\end{pgfscope}%
\begin{pgfscope}%
\pgfsys@transformshift{3.375640in}{1.781194in}%
\pgfsys@useobject{currentmarker}{}%
\end{pgfscope}%
\begin{pgfscope}%
\pgfsys@transformshift{0.774283in}{2.334998in}%
\pgfsys@useobject{currentmarker}{}%
\end{pgfscope}%
\begin{pgfscope}%
\pgfsys@transformshift{0.758719in}{2.335924in}%
\pgfsys@useobject{currentmarker}{}%
\end{pgfscope}%
\begin{pgfscope}%
\pgfsys@transformshift{0.749563in}{2.336447in}%
\pgfsys@useobject{currentmarker}{}%
\end{pgfscope}%
\begin{pgfscope}%
\pgfsys@transformshift{0.729475in}{2.337628in}%
\pgfsys@useobject{currentmarker}{}%
\end{pgfscope}%
\begin{pgfscope}%
\pgfsys@transformshift{0.712036in}{2.338586in}%
\pgfsys@useobject{currentmarker}{}%
\end{pgfscope}%
\end{pgfscope}%
\begin{pgfscope}%
\pgfsetrectcap%
\pgfsetmiterjoin%
\pgfsetlinewidth{0.803000pt}%
\definecolor{currentstroke}{rgb}{0.000000,0.000000,0.000000}%
\pgfsetstrokecolor{currentstroke}%
\pgfsetdash{}{0pt}%
\pgfpathmoveto{\pgfqpoint{0.634340in}{0.489791in}}%
\pgfpathlineto{\pgfqpoint{0.634340in}{2.455314in}}%
\pgfusepath{stroke}%
\end{pgfscope}%
\begin{pgfscope}%
\pgfsetrectcap%
\pgfsetmiterjoin%
\pgfsetlinewidth{0.803000pt}%
\definecolor{currentstroke}{rgb}{0.000000,0.000000,0.000000}%
\pgfsetstrokecolor{currentstroke}%
\pgfsetdash{}{0pt}%
\pgfpathmoveto{\pgfqpoint{4.037360in}{0.489791in}}%
\pgfpathlineto{\pgfqpoint{4.037360in}{2.455314in}}%
\pgfusepath{stroke}%
\end{pgfscope}%
\begin{pgfscope}%
\pgfsetrectcap%
\pgfsetmiterjoin%
\pgfsetlinewidth{0.803000pt}%
\definecolor{currentstroke}{rgb}{0.000000,0.000000,0.000000}%
\pgfsetstrokecolor{currentstroke}%
\pgfsetdash{}{0pt}%
\pgfpathmoveto{\pgfqpoint{0.634340in}{0.489791in}}%
\pgfpathlineto{\pgfqpoint{4.037360in}{0.489791in}}%
\pgfusepath{stroke}%
\end{pgfscope}%
\begin{pgfscope}%
\pgfsetrectcap%
\pgfsetmiterjoin%
\pgfsetlinewidth{0.803000pt}%
\definecolor{currentstroke}{rgb}{0.000000,0.000000,0.000000}%
\pgfsetstrokecolor{currentstroke}%
\pgfsetdash{}{0pt}%
\pgfpathmoveto{\pgfqpoint{0.634340in}{2.455314in}}%
\pgfpathlineto{\pgfqpoint{4.037360in}{2.455314in}}%
\pgfusepath{stroke}%
\end{pgfscope}%
\begin{pgfscope}%
\definecolor{textcolor}{rgb}{0.000000,0.000000,0.000000}%
\pgfsetstrokecolor{textcolor}%
\pgfsetfillcolor{textcolor}%
\pgftext[x=1.579623in,y=0.676984in,left,base]{\color{textcolor}\rmfamily\fontsize{10.000000}{12.000000}\selectfont Cluster I}%
\end{pgfscope}%
\begin{pgfscope}%
\definecolor{textcolor}{rgb}{0.000000,0.000000,0.000000}%
\pgfsetstrokecolor{textcolor}%
\pgfsetfillcolor{textcolor}%
\pgftext[x=3.092077in,y=1.144965in,left,base]{\color{textcolor}\rmfamily\fontsize{10.000000}{12.000000}\selectfont Cluster II}%
\end{pgfscope}%
\begin{pgfscope}%
\definecolor{textcolor}{rgb}{0.000000,0.000000,0.000000}%
\pgfsetstrokecolor{textcolor}%
\pgfsetfillcolor{textcolor}%
\pgftext[x=2.903020in,y=1.467873in,left,base]{\color{textcolor}\rmfamily\fontsize{10.000000}{12.000000}\selectfont Cluster III}%
\end{pgfscope}%
\begin{pgfscope}%
\definecolor{textcolor}{rgb}{0.000000,0.000000,0.000000}%
\pgfsetstrokecolor{textcolor}%
\pgfsetfillcolor{textcolor}%
\pgftext[x=3.281133in,y=1.846938in,left,base]{\color{textcolor}\rmfamily\fontsize{10.000000}{12.000000}\selectfont Cluster IV}%
\end{pgfscope}%
\begin{pgfscope}%
\definecolor{textcolor}{rgb}{0.000000,0.000000,0.000000}%
\pgfsetstrokecolor{textcolor}%
\pgfsetfillcolor{textcolor}%
\pgftext[x=0.823397in,y=2.277481in,left,base]{\color{textcolor}\rmfamily\fontsize{10.000000}{12.000000}\selectfont Cluster V}%
\end{pgfscope}%
\end{pgfpicture}%
\makeatother%
\endgroup%

				\caption{Ubicación de los modos descriptos en la tabla \ref{Table:Cuyt} cuando $T_s = 0.4883\cdot 10^{-3}$ segundos.}
				\label{Fig:loc_ava}		
			\end{figure}
			
			
			Con el fin de utilizar el procedimiento \emph{Shift-and-Zoom}, se consideran $L=5$ intervalos de frecuencia diferentes. Los límites de los intervalos y los factores de diezmado se encuentran en la tabla~\ref{Table:segments2}.  
		
			
			\begin{table}[t]
				\centering
				\begin{tabular}{lllll}
				 & $\nu_{c_i}$(Hz)       & $\nu_{i_{min}}$(Hz)      & $\nu_{i_{max}}$(Hz) & $Q_i$\\ 
				 \hline
				 
				$\Upsilon_1$   & -349.45 & -390.41  & -308.49 & 4 \\ 
				$\Upsilon_2$   & -130.29 & -191.73  & -68.85  & 3 \\ 
				$\Upsilon_3$   & 14.95   &  4.71    & 25.19   & 4 \\ 
				$\Upsilon_{4}$   & 140.95  & 100.61   & 166.45  & 5 \\ 
					%$\Upsilon_5$   & 125.49  & 84.53    & 166.45  & 5 \\ \hline
				$\Upsilon_5$   & 436.75  & 416.27   & 457.23  & 6 \\ 
				\hline
				\end{tabular}
			\caption{Intervalos de frecuencia y factores de diezmado.}\label{Table:segments2}
			\end{table}
			
			
			Se procesa las muestras de $y_k$ con el algoritmo \emph{Shift-and-zoom} y se comparan las estimaciones con las obtenidas sin el preprocesamiento. Las figuras \ref{Fig:RMSE_Cluster_nu} y \ref{Fig:RMSE_Cluster_gamma} muestran el RMSE de las frecuencias y los factores de amortiguamiento estimados para los diferentes clústeres usando ambas técnicas. Para una mejor presentación de los resultados, el cluster $IV$ se divide en dos. Cluster $IV(a)$ con 4 frecuencias y el Cluster $IV(b)$ con 3 frecuencias.
			
			En el caso de bajas $SNR$ el procedimiento \emph{Shift-and-Zoom} presenta un mejor rendimiento que el enfoque tradicional. Dado que \emph{Shift-and-Zoom} aumenta la separación de las frecuencias, mejorando $\dot{z}_i$ hace que el algoritmo de estimación sea más resistente a perturbaciones de ruido. Esta observación es aún más relevante cuando la amplitud $|c_i|$ es pequeña. Este es el caos de los modos $\xi_{14}$ a $\xi_{17}$ en el cluster $IV(a)$. 
			
			En este ejemplo, es fundamental diezmar $y^{bb}_k$ en lugar de $y_k$ directamente. Si se hubiera hecho esto último, el factor de diezmado máximo habría sido $Q =2$ para evitar \emph{aliasing}. Dicho valor para $Q$ da como resultado una estimación deficiente de algunos parámetros, en particular los del cluster $IV(a)$. Al implementar la estimación en paralelo para los distintos clusters, se pudo separar aún más las frecuencias complejas, mejorando el rendimiento final. Usando este enfoque, se transformó un problema mal condicionado en cinco problemas diferentes que están mejor condicionados.		
			%\newpage
			\begin{figure}[h!]
				%\centering
				\begin{subfigure}[h]{0.5\textwidth}
					\centering
					\resizebox{\linewidth}{!}{%% Creator: Matplotlib, PGF backend
%%
%% To include the figure in your LaTeX document, write
%%   \input{<filename>.pgf}
%%
%% Make sure the required packages are loaded in your preamble
%%   \usepackage{pgf}
%%
%% and, on pdftex
%%   \usepackage[utf8]{inputenc}\DeclareUnicodeCharacter{2212}{-}
%%
%% or, on luatex and xetex
%%   \usepackage{unicode-math}
%%
%% Figures using additional raster images can only be included by \input if
%% they are in the same directory as the main LaTeX file. For loading figures
%% from other directories you can use the `import` package
%%   \usepackage{import}
%%
%% and then include the figures with
%%   \import{<path to file>}{<filename>.pgf}
%%
%% Matplotlib used the following preamble
%%   \usepackage[utf8x]{inputenc}
%%   \usepackage[T1]{fontenc}
%%   \usepackage{amsmath,amssymb,amsfonts}
%%
\begingroup%
\makeatletter%
\begin{pgfpicture}%
\pgfpathrectangle{\pgfpointorigin}{\pgfqpoint{4.136389in}{2.495314in}}%
\pgfusepath{use as bounding box, clip}%
\begin{pgfscope}%
\pgfsetbuttcap%
\pgfsetmiterjoin%
\definecolor{currentfill}{rgb}{1.000000,1.000000,1.000000}%
\pgfsetfillcolor{currentfill}%
\pgfsetlinewidth{0.000000pt}%
\definecolor{currentstroke}{rgb}{1.000000,1.000000,1.000000}%
\pgfsetstrokecolor{currentstroke}%
\pgfsetdash{}{0pt}%
\pgfpathmoveto{\pgfqpoint{-0.000000in}{0.000000in}}%
\pgfpathlineto{\pgfqpoint{4.136389in}{0.000000in}}%
\pgfpathlineto{\pgfqpoint{4.136389in}{2.495314in}}%
\pgfpathlineto{\pgfqpoint{-0.000000in}{2.495314in}}%
\pgfpathclose%
\pgfusepath{fill}%
\end{pgfscope}%
\begin{pgfscope}%
\pgfsetbuttcap%
\pgfsetmiterjoin%
\definecolor{currentfill}{rgb}{1.000000,1.000000,1.000000}%
\pgfsetfillcolor{currentfill}%
\pgfsetlinewidth{0.000000pt}%
\definecolor{currentstroke}{rgb}{0.000000,0.000000,0.000000}%
\pgfsetstrokecolor{currentstroke}%
\pgfsetstrokeopacity{0.000000}%
\pgfsetdash{}{0pt}%
\pgfpathmoveto{\pgfqpoint{0.740433in}{0.566590in}}%
\pgfpathlineto{\pgfqpoint{4.036389in}{0.566590in}}%
\pgfpathlineto{\pgfqpoint{4.036389in}{2.395314in}}%
\pgfpathlineto{\pgfqpoint{0.740433in}{2.395314in}}%
\pgfpathclose%
\pgfusepath{fill}%
\end{pgfscope}%
\begin{pgfscope}%
\pgfpathrectangle{\pgfqpoint{0.740433in}{0.566590in}}{\pgfqpoint{3.295956in}{1.828724in}}%
\pgfusepath{clip}%
\pgfsetrectcap%
\pgfsetroundjoin%
\pgfsetlinewidth{0.803000pt}%
\definecolor{currentstroke}{rgb}{0.690196,0.690196,0.690196}%
\pgfsetstrokecolor{currentstroke}%
\pgfsetdash{}{0pt}%
\pgfpathmoveto{\pgfqpoint{0.740433in}{0.566590in}}%
\pgfpathlineto{\pgfqpoint{0.740433in}{2.395314in}}%
\pgfusepath{stroke}%
\end{pgfscope}%
\begin{pgfscope}%
\pgfsetbuttcap%
\pgfsetroundjoin%
\definecolor{currentfill}{rgb}{0.000000,0.000000,0.000000}%
\pgfsetfillcolor{currentfill}%
\pgfsetlinewidth{0.803000pt}%
\definecolor{currentstroke}{rgb}{0.000000,0.000000,0.000000}%
\pgfsetstrokecolor{currentstroke}%
\pgfsetdash{}{0pt}%
\pgfsys@defobject{currentmarker}{\pgfqpoint{0.000000in}{-0.048611in}}{\pgfqpoint{0.000000in}{0.000000in}}{%
\pgfpathmoveto{\pgfqpoint{0.000000in}{0.000000in}}%
\pgfpathlineto{\pgfqpoint{0.000000in}{-0.048611in}}%
\pgfusepath{stroke,fill}%
}%
\begin{pgfscope}%
\pgfsys@transformshift{0.740433in}{0.566590in}%
\pgfsys@useobject{currentmarker}{}%
\end{pgfscope}%
\end{pgfscope}%
\begin{pgfscope}%
\definecolor{textcolor}{rgb}{0.000000,0.000000,0.000000}%
\pgfsetstrokecolor{textcolor}%
\pgfsetfillcolor{textcolor}%
\pgftext[x=0.740433in,y=0.469368in,,top]{\color{textcolor}\rmfamily\fontsize{12.000000}{14.400000}\selectfont \(\displaystyle {-10}\)}%
\end{pgfscope}%
\begin{pgfscope}%
\pgfpathrectangle{\pgfqpoint{0.740433in}{0.566590in}}{\pgfqpoint{3.295956in}{1.828724in}}%
\pgfusepath{clip}%
\pgfsetrectcap%
\pgfsetroundjoin%
\pgfsetlinewidth{0.803000pt}%
\definecolor{currentstroke}{rgb}{0.690196,0.690196,0.690196}%
\pgfsetstrokecolor{currentstroke}%
\pgfsetdash{}{0pt}%
\pgfpathmoveto{\pgfqpoint{1.247503in}{0.566590in}}%
\pgfpathlineto{\pgfqpoint{1.247503in}{2.395314in}}%
\pgfusepath{stroke}%
\end{pgfscope}%
\begin{pgfscope}%
\pgfsetbuttcap%
\pgfsetroundjoin%
\definecolor{currentfill}{rgb}{0.000000,0.000000,0.000000}%
\pgfsetfillcolor{currentfill}%
\pgfsetlinewidth{0.803000pt}%
\definecolor{currentstroke}{rgb}{0.000000,0.000000,0.000000}%
\pgfsetstrokecolor{currentstroke}%
\pgfsetdash{}{0pt}%
\pgfsys@defobject{currentmarker}{\pgfqpoint{0.000000in}{-0.048611in}}{\pgfqpoint{0.000000in}{0.000000in}}{%
\pgfpathmoveto{\pgfqpoint{0.000000in}{0.000000in}}%
\pgfpathlineto{\pgfqpoint{0.000000in}{-0.048611in}}%
\pgfusepath{stroke,fill}%
}%
\begin{pgfscope}%
\pgfsys@transformshift{1.247503in}{0.566590in}%
\pgfsys@useobject{currentmarker}{}%
\end{pgfscope}%
\end{pgfscope}%
\begin{pgfscope}%
\definecolor{textcolor}{rgb}{0.000000,0.000000,0.000000}%
\pgfsetstrokecolor{textcolor}%
\pgfsetfillcolor{textcolor}%
\pgftext[x=1.247503in,y=0.469368in,,top]{\color{textcolor}\rmfamily\fontsize{12.000000}{14.400000}\selectfont \(\displaystyle {0}\)}%
\end{pgfscope}%
\begin{pgfscope}%
\pgfpathrectangle{\pgfqpoint{0.740433in}{0.566590in}}{\pgfqpoint{3.295956in}{1.828724in}}%
\pgfusepath{clip}%
\pgfsetrectcap%
\pgfsetroundjoin%
\pgfsetlinewidth{0.803000pt}%
\definecolor{currentstroke}{rgb}{0.690196,0.690196,0.690196}%
\pgfsetstrokecolor{currentstroke}%
\pgfsetdash{}{0pt}%
\pgfpathmoveto{\pgfqpoint{1.754573in}{0.566590in}}%
\pgfpathlineto{\pgfqpoint{1.754573in}{2.395314in}}%
\pgfusepath{stroke}%
\end{pgfscope}%
\begin{pgfscope}%
\pgfsetbuttcap%
\pgfsetroundjoin%
\definecolor{currentfill}{rgb}{0.000000,0.000000,0.000000}%
\pgfsetfillcolor{currentfill}%
\pgfsetlinewidth{0.803000pt}%
\definecolor{currentstroke}{rgb}{0.000000,0.000000,0.000000}%
\pgfsetstrokecolor{currentstroke}%
\pgfsetdash{}{0pt}%
\pgfsys@defobject{currentmarker}{\pgfqpoint{0.000000in}{-0.048611in}}{\pgfqpoint{0.000000in}{0.000000in}}{%
\pgfpathmoveto{\pgfqpoint{0.000000in}{0.000000in}}%
\pgfpathlineto{\pgfqpoint{0.000000in}{-0.048611in}}%
\pgfusepath{stroke,fill}%
}%
\begin{pgfscope}%
\pgfsys@transformshift{1.754573in}{0.566590in}%
\pgfsys@useobject{currentmarker}{}%
\end{pgfscope}%
\end{pgfscope}%
\begin{pgfscope}%
\definecolor{textcolor}{rgb}{0.000000,0.000000,0.000000}%
\pgfsetstrokecolor{textcolor}%
\pgfsetfillcolor{textcolor}%
\pgftext[x=1.754573in,y=0.469368in,,top]{\color{textcolor}\rmfamily\fontsize{12.000000}{14.400000}\selectfont \(\displaystyle {10}\)}%
\end{pgfscope}%
\begin{pgfscope}%
\pgfpathrectangle{\pgfqpoint{0.740433in}{0.566590in}}{\pgfqpoint{3.295956in}{1.828724in}}%
\pgfusepath{clip}%
\pgfsetrectcap%
\pgfsetroundjoin%
\pgfsetlinewidth{0.803000pt}%
\definecolor{currentstroke}{rgb}{0.690196,0.690196,0.690196}%
\pgfsetstrokecolor{currentstroke}%
\pgfsetdash{}{0pt}%
\pgfpathmoveto{\pgfqpoint{2.261643in}{0.566590in}}%
\pgfpathlineto{\pgfqpoint{2.261643in}{2.395314in}}%
\pgfusepath{stroke}%
\end{pgfscope}%
\begin{pgfscope}%
\pgfsetbuttcap%
\pgfsetroundjoin%
\definecolor{currentfill}{rgb}{0.000000,0.000000,0.000000}%
\pgfsetfillcolor{currentfill}%
\pgfsetlinewidth{0.803000pt}%
\definecolor{currentstroke}{rgb}{0.000000,0.000000,0.000000}%
\pgfsetstrokecolor{currentstroke}%
\pgfsetdash{}{0pt}%
\pgfsys@defobject{currentmarker}{\pgfqpoint{0.000000in}{-0.048611in}}{\pgfqpoint{0.000000in}{0.000000in}}{%
\pgfpathmoveto{\pgfqpoint{0.000000in}{0.000000in}}%
\pgfpathlineto{\pgfqpoint{0.000000in}{-0.048611in}}%
\pgfusepath{stroke,fill}%
}%
\begin{pgfscope}%
\pgfsys@transformshift{2.261643in}{0.566590in}%
\pgfsys@useobject{currentmarker}{}%
\end{pgfscope}%
\end{pgfscope}%
\begin{pgfscope}%
\definecolor{textcolor}{rgb}{0.000000,0.000000,0.000000}%
\pgfsetstrokecolor{textcolor}%
\pgfsetfillcolor{textcolor}%
\pgftext[x=2.261643in,y=0.469368in,,top]{\color{textcolor}\rmfamily\fontsize{12.000000}{14.400000}\selectfont \(\displaystyle {20}\)}%
\end{pgfscope}%
\begin{pgfscope}%
\pgfpathrectangle{\pgfqpoint{0.740433in}{0.566590in}}{\pgfqpoint{3.295956in}{1.828724in}}%
\pgfusepath{clip}%
\pgfsetrectcap%
\pgfsetroundjoin%
\pgfsetlinewidth{0.803000pt}%
\definecolor{currentstroke}{rgb}{0.690196,0.690196,0.690196}%
\pgfsetstrokecolor{currentstroke}%
\pgfsetdash{}{0pt}%
\pgfpathmoveto{\pgfqpoint{2.768713in}{0.566590in}}%
\pgfpathlineto{\pgfqpoint{2.768713in}{2.395314in}}%
\pgfusepath{stroke}%
\end{pgfscope}%
\begin{pgfscope}%
\pgfsetbuttcap%
\pgfsetroundjoin%
\definecolor{currentfill}{rgb}{0.000000,0.000000,0.000000}%
\pgfsetfillcolor{currentfill}%
\pgfsetlinewidth{0.803000pt}%
\definecolor{currentstroke}{rgb}{0.000000,0.000000,0.000000}%
\pgfsetstrokecolor{currentstroke}%
\pgfsetdash{}{0pt}%
\pgfsys@defobject{currentmarker}{\pgfqpoint{0.000000in}{-0.048611in}}{\pgfqpoint{0.000000in}{0.000000in}}{%
\pgfpathmoveto{\pgfqpoint{0.000000in}{0.000000in}}%
\pgfpathlineto{\pgfqpoint{0.000000in}{-0.048611in}}%
\pgfusepath{stroke,fill}%
}%
\begin{pgfscope}%
\pgfsys@transformshift{2.768713in}{0.566590in}%
\pgfsys@useobject{currentmarker}{}%
\end{pgfscope}%
\end{pgfscope}%
\begin{pgfscope}%
\definecolor{textcolor}{rgb}{0.000000,0.000000,0.000000}%
\pgfsetstrokecolor{textcolor}%
\pgfsetfillcolor{textcolor}%
\pgftext[x=2.768713in,y=0.469368in,,top]{\color{textcolor}\rmfamily\fontsize{12.000000}{14.400000}\selectfont \(\displaystyle {30}\)}%
\end{pgfscope}%
\begin{pgfscope}%
\pgfpathrectangle{\pgfqpoint{0.740433in}{0.566590in}}{\pgfqpoint{3.295956in}{1.828724in}}%
\pgfusepath{clip}%
\pgfsetrectcap%
\pgfsetroundjoin%
\pgfsetlinewidth{0.803000pt}%
\definecolor{currentstroke}{rgb}{0.690196,0.690196,0.690196}%
\pgfsetstrokecolor{currentstroke}%
\pgfsetdash{}{0pt}%
\pgfpathmoveto{\pgfqpoint{3.275783in}{0.566590in}}%
\pgfpathlineto{\pgfqpoint{3.275783in}{2.395314in}}%
\pgfusepath{stroke}%
\end{pgfscope}%
\begin{pgfscope}%
\pgfsetbuttcap%
\pgfsetroundjoin%
\definecolor{currentfill}{rgb}{0.000000,0.000000,0.000000}%
\pgfsetfillcolor{currentfill}%
\pgfsetlinewidth{0.803000pt}%
\definecolor{currentstroke}{rgb}{0.000000,0.000000,0.000000}%
\pgfsetstrokecolor{currentstroke}%
\pgfsetdash{}{0pt}%
\pgfsys@defobject{currentmarker}{\pgfqpoint{0.000000in}{-0.048611in}}{\pgfqpoint{0.000000in}{0.000000in}}{%
\pgfpathmoveto{\pgfqpoint{0.000000in}{0.000000in}}%
\pgfpathlineto{\pgfqpoint{0.000000in}{-0.048611in}}%
\pgfusepath{stroke,fill}%
}%
\begin{pgfscope}%
\pgfsys@transformshift{3.275783in}{0.566590in}%
\pgfsys@useobject{currentmarker}{}%
\end{pgfscope}%
\end{pgfscope}%
\begin{pgfscope}%
\definecolor{textcolor}{rgb}{0.000000,0.000000,0.000000}%
\pgfsetstrokecolor{textcolor}%
\pgfsetfillcolor{textcolor}%
\pgftext[x=3.275783in,y=0.469368in,,top]{\color{textcolor}\rmfamily\fontsize{12.000000}{14.400000}\selectfont \(\displaystyle {40}\)}%
\end{pgfscope}%
\begin{pgfscope}%
\pgfpathrectangle{\pgfqpoint{0.740433in}{0.566590in}}{\pgfqpoint{3.295956in}{1.828724in}}%
\pgfusepath{clip}%
\pgfsetrectcap%
\pgfsetroundjoin%
\pgfsetlinewidth{0.803000pt}%
\definecolor{currentstroke}{rgb}{0.690196,0.690196,0.690196}%
\pgfsetstrokecolor{currentstroke}%
\pgfsetdash{}{0pt}%
\pgfpathmoveto{\pgfqpoint{3.782853in}{0.566590in}}%
\pgfpathlineto{\pgfqpoint{3.782853in}{2.395314in}}%
\pgfusepath{stroke}%
\end{pgfscope}%
\begin{pgfscope}%
\pgfsetbuttcap%
\pgfsetroundjoin%
\definecolor{currentfill}{rgb}{0.000000,0.000000,0.000000}%
\pgfsetfillcolor{currentfill}%
\pgfsetlinewidth{0.803000pt}%
\definecolor{currentstroke}{rgb}{0.000000,0.000000,0.000000}%
\pgfsetstrokecolor{currentstroke}%
\pgfsetdash{}{0pt}%
\pgfsys@defobject{currentmarker}{\pgfqpoint{0.000000in}{-0.048611in}}{\pgfqpoint{0.000000in}{0.000000in}}{%
\pgfpathmoveto{\pgfqpoint{0.000000in}{0.000000in}}%
\pgfpathlineto{\pgfqpoint{0.000000in}{-0.048611in}}%
\pgfusepath{stroke,fill}%
}%
\begin{pgfscope}%
\pgfsys@transformshift{3.782853in}{0.566590in}%
\pgfsys@useobject{currentmarker}{}%
\end{pgfscope}%
\end{pgfscope}%
\begin{pgfscope}%
\definecolor{textcolor}{rgb}{0.000000,0.000000,0.000000}%
\pgfsetstrokecolor{textcolor}%
\pgfsetfillcolor{textcolor}%
\pgftext[x=3.782853in,y=0.469368in,,top]{\color{textcolor}\rmfamily\fontsize{12.000000}{14.400000}\selectfont \(\displaystyle {50}\)}%
\end{pgfscope}%
\begin{pgfscope}%
\definecolor{textcolor}{rgb}{0.000000,0.000000,0.000000}%
\pgfsetstrokecolor{textcolor}%
\pgfsetfillcolor{textcolor}%
\pgftext[x=2.388411in,y=0.266626in,,top]{\color{textcolor}\rmfamily\fontsize{12.000000}{14.400000}\selectfont SNR [dB]}%
\end{pgfscope}%
\begin{pgfscope}%
\pgfpathrectangle{\pgfqpoint{0.740433in}{0.566590in}}{\pgfqpoint{3.295956in}{1.828724in}}%
\pgfusepath{clip}%
\pgfsetrectcap%
\pgfsetroundjoin%
\pgfsetlinewidth{0.803000pt}%
\definecolor{currentstroke}{rgb}{0.690196,0.690196,0.690196}%
\pgfsetstrokecolor{currentstroke}%
\pgfsetdash{}{0pt}%
\pgfpathmoveto{\pgfqpoint{0.740433in}{0.752967in}}%
\pgfpathlineto{\pgfqpoint{4.036389in}{0.752967in}}%
\pgfusepath{stroke}%
\end{pgfscope}%
\begin{pgfscope}%
\pgfsetbuttcap%
\pgfsetroundjoin%
\definecolor{currentfill}{rgb}{0.000000,0.000000,0.000000}%
\pgfsetfillcolor{currentfill}%
\pgfsetlinewidth{0.803000pt}%
\definecolor{currentstroke}{rgb}{0.000000,0.000000,0.000000}%
\pgfsetstrokecolor{currentstroke}%
\pgfsetdash{}{0pt}%
\pgfsys@defobject{currentmarker}{\pgfqpoint{-0.048611in}{0.000000in}}{\pgfqpoint{-0.000000in}{0.000000in}}{%
\pgfpathmoveto{\pgfqpoint{-0.000000in}{0.000000in}}%
\pgfpathlineto{\pgfqpoint{-0.048611in}{0.000000in}}%
\pgfusepath{stroke,fill}%
}%
\begin{pgfscope}%
\pgfsys@transformshift{0.740433in}{0.752967in}%
\pgfsys@useobject{currentmarker}{}%
\end{pgfscope}%
\end{pgfscope}%
\begin{pgfscope}%
\definecolor{textcolor}{rgb}{0.000000,0.000000,0.000000}%
\pgfsetstrokecolor{textcolor}%
\pgfsetfillcolor{textcolor}%
\pgftext[x=0.322222in, y=0.695574in, left, base]{\color{textcolor}\rmfamily\fontsize{12.000000}{14.400000}\selectfont \(\displaystyle {10^{-4}}\)}%
\end{pgfscope}%
\begin{pgfscope}%
\pgfpathrectangle{\pgfqpoint{0.740433in}{0.566590in}}{\pgfqpoint{3.295956in}{1.828724in}}%
\pgfusepath{clip}%
\pgfsetrectcap%
\pgfsetroundjoin%
\pgfsetlinewidth{0.803000pt}%
\definecolor{currentstroke}{rgb}{0.690196,0.690196,0.690196}%
\pgfsetstrokecolor{currentstroke}%
\pgfsetdash{}{0pt}%
\pgfpathmoveto{\pgfqpoint{0.740433in}{1.236303in}}%
\pgfpathlineto{\pgfqpoint{4.036389in}{1.236303in}}%
\pgfusepath{stroke}%
\end{pgfscope}%
\begin{pgfscope}%
\pgfsetbuttcap%
\pgfsetroundjoin%
\definecolor{currentfill}{rgb}{0.000000,0.000000,0.000000}%
\pgfsetfillcolor{currentfill}%
\pgfsetlinewidth{0.803000pt}%
\definecolor{currentstroke}{rgb}{0.000000,0.000000,0.000000}%
\pgfsetstrokecolor{currentstroke}%
\pgfsetdash{}{0pt}%
\pgfsys@defobject{currentmarker}{\pgfqpoint{-0.048611in}{0.000000in}}{\pgfqpoint{-0.000000in}{0.000000in}}{%
\pgfpathmoveto{\pgfqpoint{-0.000000in}{0.000000in}}%
\pgfpathlineto{\pgfqpoint{-0.048611in}{0.000000in}}%
\pgfusepath{stroke,fill}%
}%
\begin{pgfscope}%
\pgfsys@transformshift{0.740433in}{1.236303in}%
\pgfsys@useobject{currentmarker}{}%
\end{pgfscope}%
\end{pgfscope}%
\begin{pgfscope}%
\definecolor{textcolor}{rgb}{0.000000,0.000000,0.000000}%
\pgfsetstrokecolor{textcolor}%
\pgfsetfillcolor{textcolor}%
\pgftext[x=0.322222in, y=1.178910in, left, base]{\color{textcolor}\rmfamily\fontsize{12.000000}{14.400000}\selectfont \(\displaystyle {10^{-2}}\)}%
\end{pgfscope}%
\begin{pgfscope}%
\pgfpathrectangle{\pgfqpoint{0.740433in}{0.566590in}}{\pgfqpoint{3.295956in}{1.828724in}}%
\pgfusepath{clip}%
\pgfsetrectcap%
\pgfsetroundjoin%
\pgfsetlinewidth{0.803000pt}%
\definecolor{currentstroke}{rgb}{0.690196,0.690196,0.690196}%
\pgfsetstrokecolor{currentstroke}%
\pgfsetdash{}{0pt}%
\pgfpathmoveto{\pgfqpoint{0.740433in}{1.719639in}}%
\pgfpathlineto{\pgfqpoint{4.036389in}{1.719639in}}%
\pgfusepath{stroke}%
\end{pgfscope}%
\begin{pgfscope}%
\pgfsetbuttcap%
\pgfsetroundjoin%
\definecolor{currentfill}{rgb}{0.000000,0.000000,0.000000}%
\pgfsetfillcolor{currentfill}%
\pgfsetlinewidth{0.803000pt}%
\definecolor{currentstroke}{rgb}{0.000000,0.000000,0.000000}%
\pgfsetstrokecolor{currentstroke}%
\pgfsetdash{}{0pt}%
\pgfsys@defobject{currentmarker}{\pgfqpoint{-0.048611in}{0.000000in}}{\pgfqpoint{-0.000000in}{0.000000in}}{%
\pgfpathmoveto{\pgfqpoint{-0.000000in}{0.000000in}}%
\pgfpathlineto{\pgfqpoint{-0.048611in}{0.000000in}}%
\pgfusepath{stroke,fill}%
}%
\begin{pgfscope}%
\pgfsys@transformshift{0.740433in}{1.719639in}%
\pgfsys@useobject{currentmarker}{}%
\end{pgfscope}%
\end{pgfscope}%
\begin{pgfscope}%
\definecolor{textcolor}{rgb}{0.000000,0.000000,0.000000}%
\pgfsetstrokecolor{textcolor}%
\pgfsetfillcolor{textcolor}%
\pgftext[x=0.414045in, y=1.662246in, left, base]{\color{textcolor}\rmfamily\fontsize{12.000000}{14.400000}\selectfont \(\displaystyle {10^{0}}\)}%
\end{pgfscope}%
\begin{pgfscope}%
\pgfpathrectangle{\pgfqpoint{0.740433in}{0.566590in}}{\pgfqpoint{3.295956in}{1.828724in}}%
\pgfusepath{clip}%
\pgfsetrectcap%
\pgfsetroundjoin%
\pgfsetlinewidth{0.803000pt}%
\definecolor{currentstroke}{rgb}{0.690196,0.690196,0.690196}%
\pgfsetstrokecolor{currentstroke}%
\pgfsetdash{}{0pt}%
\pgfpathmoveto{\pgfqpoint{0.740433in}{2.202975in}}%
\pgfpathlineto{\pgfqpoint{4.036389in}{2.202975in}}%
\pgfusepath{stroke}%
\end{pgfscope}%
\begin{pgfscope}%
\pgfsetbuttcap%
\pgfsetroundjoin%
\definecolor{currentfill}{rgb}{0.000000,0.000000,0.000000}%
\pgfsetfillcolor{currentfill}%
\pgfsetlinewidth{0.803000pt}%
\definecolor{currentstroke}{rgb}{0.000000,0.000000,0.000000}%
\pgfsetstrokecolor{currentstroke}%
\pgfsetdash{}{0pt}%
\pgfsys@defobject{currentmarker}{\pgfqpoint{-0.048611in}{0.000000in}}{\pgfqpoint{-0.000000in}{0.000000in}}{%
\pgfpathmoveto{\pgfqpoint{-0.000000in}{0.000000in}}%
\pgfpathlineto{\pgfqpoint{-0.048611in}{0.000000in}}%
\pgfusepath{stroke,fill}%
}%
\begin{pgfscope}%
\pgfsys@transformshift{0.740433in}{2.202975in}%
\pgfsys@useobject{currentmarker}{}%
\end{pgfscope}%
\end{pgfscope}%
\begin{pgfscope}%
\definecolor{textcolor}{rgb}{0.000000,0.000000,0.000000}%
\pgfsetstrokecolor{textcolor}%
\pgfsetfillcolor{textcolor}%
\pgftext[x=0.414045in, y=2.145582in, left, base]{\color{textcolor}\rmfamily\fontsize{12.000000}{14.400000}\selectfont \(\displaystyle {10^{2}}\)}%
\end{pgfscope}%
\begin{pgfscope}%
\definecolor{textcolor}{rgb}{0.000000,0.000000,0.000000}%
\pgfsetstrokecolor{textcolor}%
\pgfsetfillcolor{textcolor}%
\pgftext[x=0.266667in,y=1.480952in,,bottom,rotate=90.000000]{\color{textcolor}\rmfamily\fontsize{12.000000}{14.400000}\selectfont \(\displaystyle \hat{\sigma}_{\nu}(\mathrm{SNR})\)}%
\end{pgfscope}%
\begin{pgfscope}%
\pgfpathrectangle{\pgfqpoint{0.740433in}{0.566590in}}{\pgfqpoint{3.295956in}{1.828724in}}%
\pgfusepath{clip}%
\pgfsetbuttcap%
\pgfsetroundjoin%
\pgfsetlinewidth{1.505625pt}%
\definecolor{currentstroke}{rgb}{0.000000,0.447000,0.741000}%
\pgfsetstrokecolor{currentstroke}%
\pgfsetdash{{5.550000pt}{2.400000pt}}{0.000000pt}%
\pgfpathmoveto{\pgfqpoint{0.740433in}{2.286753in}}%
\pgfpathlineto{\pgfqpoint{0.837373in}{2.283142in}}%
\pgfpathlineto{\pgfqpoint{0.934312in}{2.282542in}}%
\pgfpathlineto{\pgfqpoint{1.031252in}{2.272757in}}%
\pgfpathlineto{\pgfqpoint{1.128192in}{2.278788in}}%
\pgfpathlineto{\pgfqpoint{1.225132in}{2.275197in}}%
\pgfpathlineto{\pgfqpoint{1.322072in}{2.280438in}}%
\pgfpathlineto{\pgfqpoint{1.419012in}{2.272222in}}%
\pgfpathlineto{\pgfqpoint{1.515952in}{2.259562in}}%
\pgfpathlineto{\pgfqpoint{1.612892in}{2.260827in}}%
\pgfpathlineto{\pgfqpoint{1.709831in}{2.258675in}}%
\pgfpathlineto{\pgfqpoint{1.806771in}{2.253748in}}%
\pgfpathlineto{\pgfqpoint{1.903711in}{2.233501in}}%
\pgfpathlineto{\pgfqpoint{2.000651in}{2.231452in}}%
\pgfpathlineto{\pgfqpoint{2.097591in}{2.199703in}}%
\pgfpathlineto{\pgfqpoint{2.194531in}{2.185879in}}%
\pgfpathlineto{\pgfqpoint{2.291471in}{2.179856in}}%
\pgfpathlineto{\pgfqpoint{2.388411in}{2.096321in}}%
\pgfpathlineto{\pgfqpoint{2.485350in}{2.078328in}}%
\pgfpathlineto{\pgfqpoint{2.582290in}{1.963907in}}%
\pgfpathlineto{\pgfqpoint{2.679230in}{1.346186in}}%
\pgfpathlineto{\pgfqpoint{2.776170in}{1.328449in}}%
\pgfpathlineto{\pgfqpoint{2.873110in}{1.304822in}}%
\pgfpathlineto{\pgfqpoint{2.970050in}{1.279447in}}%
\pgfpathlineto{\pgfqpoint{3.066990in}{1.254941in}}%
\pgfpathlineto{\pgfqpoint{3.163930in}{1.226196in}}%
\pgfpathlineto{\pgfqpoint{3.260870in}{1.213608in}}%
\pgfpathlineto{\pgfqpoint{3.357809in}{1.177408in}}%
\pgfpathlineto{\pgfqpoint{3.454749in}{1.164917in}}%
\pgfpathlineto{\pgfqpoint{3.551689in}{1.136221in}}%
\pgfpathlineto{\pgfqpoint{3.648629in}{1.111982in}}%
\pgfpathlineto{\pgfqpoint{3.745569in}{1.093846in}}%
\pgfpathlineto{\pgfqpoint{3.842509in}{1.068950in}}%
\pgfpathlineto{\pgfqpoint{3.939449in}{1.040409in}}%
\pgfpathlineto{\pgfqpoint{4.036389in}{1.025767in}}%
\pgfusepath{stroke}%
\end{pgfscope}%
\begin{pgfscope}%
\pgfpathrectangle{\pgfqpoint{0.740433in}{0.566590in}}{\pgfqpoint{3.295956in}{1.828724in}}%
\pgfusepath{clip}%
\pgfsetbuttcap%
\pgfsetroundjoin%
\definecolor{currentfill}{rgb}{0.000000,0.000000,0.000000}%
\pgfsetfillcolor{currentfill}%
\pgfsetfillopacity{0.000000}%
\pgfsetlinewidth{1.003750pt}%
\definecolor{currentstroke}{rgb}{0.000000,0.447000,0.741000}%
\pgfsetstrokecolor{currentstroke}%
\pgfsetdash{}{0pt}%
\pgfsys@defobject{currentmarker}{\pgfqpoint{-0.041667in}{-0.041667in}}{\pgfqpoint{0.041667in}{0.041667in}}{%
\pgfpathmoveto{\pgfqpoint{0.000000in}{-0.041667in}}%
\pgfpathcurveto{\pgfqpoint{0.011050in}{-0.041667in}}{\pgfqpoint{0.021649in}{-0.037276in}}{\pgfqpoint{0.029463in}{-0.029463in}}%
\pgfpathcurveto{\pgfqpoint{0.037276in}{-0.021649in}}{\pgfqpoint{0.041667in}{-0.011050in}}{\pgfqpoint{0.041667in}{0.000000in}}%
\pgfpathcurveto{\pgfqpoint{0.041667in}{0.011050in}}{\pgfqpoint{0.037276in}{0.021649in}}{\pgfqpoint{0.029463in}{0.029463in}}%
\pgfpathcurveto{\pgfqpoint{0.021649in}{0.037276in}}{\pgfqpoint{0.011050in}{0.041667in}}{\pgfqpoint{0.000000in}{0.041667in}}%
\pgfpathcurveto{\pgfqpoint{-0.011050in}{0.041667in}}{\pgfqpoint{-0.021649in}{0.037276in}}{\pgfqpoint{-0.029463in}{0.029463in}}%
\pgfpathcurveto{\pgfqpoint{-0.037276in}{0.021649in}}{\pgfqpoint{-0.041667in}{0.011050in}}{\pgfqpoint{-0.041667in}{0.000000in}}%
\pgfpathcurveto{\pgfqpoint{-0.041667in}{-0.011050in}}{\pgfqpoint{-0.037276in}{-0.021649in}}{\pgfqpoint{-0.029463in}{-0.029463in}}%
\pgfpathcurveto{\pgfqpoint{-0.021649in}{-0.037276in}}{\pgfqpoint{-0.011050in}{-0.041667in}}{\pgfqpoint{0.000000in}{-0.041667in}}%
\pgfpathclose%
\pgfusepath{stroke,fill}%
}%
\begin{pgfscope}%
\pgfsys@transformshift{0.740433in}{2.286753in}%
\pgfsys@useobject{currentmarker}{}%
\end{pgfscope}%
\begin{pgfscope}%
\pgfsys@transformshift{1.128192in}{2.278788in}%
\pgfsys@useobject{currentmarker}{}%
\end{pgfscope}%
\begin{pgfscope}%
\pgfsys@transformshift{1.515952in}{2.259562in}%
\pgfsys@useobject{currentmarker}{}%
\end{pgfscope}%
\begin{pgfscope}%
\pgfsys@transformshift{1.903711in}{2.233501in}%
\pgfsys@useobject{currentmarker}{}%
\end{pgfscope}%
\begin{pgfscope}%
\pgfsys@transformshift{2.291471in}{2.179856in}%
\pgfsys@useobject{currentmarker}{}%
\end{pgfscope}%
\begin{pgfscope}%
\pgfsys@transformshift{2.679230in}{1.346186in}%
\pgfsys@useobject{currentmarker}{}%
\end{pgfscope}%
\begin{pgfscope}%
\pgfsys@transformshift{3.066990in}{1.254941in}%
\pgfsys@useobject{currentmarker}{}%
\end{pgfscope}%
\begin{pgfscope}%
\pgfsys@transformshift{3.454749in}{1.164917in}%
\pgfsys@useobject{currentmarker}{}%
\end{pgfscope}%
\begin{pgfscope}%
\pgfsys@transformshift{3.842509in}{1.068950in}%
\pgfsys@useobject{currentmarker}{}%
\end{pgfscope}%
\end{pgfscope}%
\begin{pgfscope}%
\pgfpathrectangle{\pgfqpoint{0.740433in}{0.566590in}}{\pgfqpoint{3.295956in}{1.828724in}}%
\pgfusepath{clip}%
\pgfsetbuttcap%
\pgfsetroundjoin%
\pgfsetlinewidth{1.505625pt}%
\definecolor{currentstroke}{rgb}{0.850000,0.324000,0.098000}%
\pgfsetstrokecolor{currentstroke}%
\pgfsetdash{{5.550000pt}{2.400000pt}}{0.000000pt}%
\pgfpathmoveto{\pgfqpoint{0.740433in}{2.247892in}}%
\pgfpathlineto{\pgfqpoint{0.837373in}{2.234293in}}%
\pgfpathlineto{\pgfqpoint{0.934312in}{2.236247in}}%
\pgfpathlineto{\pgfqpoint{1.031252in}{2.217708in}}%
\pgfpathlineto{\pgfqpoint{1.128192in}{2.219008in}}%
\pgfpathlineto{\pgfqpoint{1.225132in}{2.212531in}}%
\pgfpathlineto{\pgfqpoint{1.322072in}{2.212859in}}%
\pgfpathlineto{\pgfqpoint{1.419012in}{2.201486in}}%
\pgfpathlineto{\pgfqpoint{1.515952in}{2.178458in}}%
\pgfpathlineto{\pgfqpoint{1.612892in}{2.177679in}}%
\pgfpathlineto{\pgfqpoint{1.709831in}{2.162156in}}%
\pgfpathlineto{\pgfqpoint{1.806771in}{2.139962in}}%
\pgfpathlineto{\pgfqpoint{1.903711in}{2.119378in}}%
\pgfpathlineto{\pgfqpoint{2.000651in}{2.128625in}}%
\pgfpathlineto{\pgfqpoint{2.097591in}{2.047699in}}%
\pgfpathlineto{\pgfqpoint{2.194531in}{1.800957in}}%
\pgfpathlineto{\pgfqpoint{2.291471in}{1.764253in}}%
\pgfpathlineto{\pgfqpoint{2.388411in}{1.701432in}}%
\pgfpathlineto{\pgfqpoint{2.485350in}{1.664769in}}%
\pgfpathlineto{\pgfqpoint{2.582290in}{1.617127in}}%
\pgfpathlineto{\pgfqpoint{2.679230in}{1.145530in}}%
\pgfpathlineto{\pgfqpoint{2.776170in}{1.115400in}}%
\pgfpathlineto{\pgfqpoint{2.873110in}{1.088905in}}%
\pgfpathlineto{\pgfqpoint{2.970050in}{1.071149in}}%
\pgfpathlineto{\pgfqpoint{3.066990in}{1.042784in}}%
\pgfpathlineto{\pgfqpoint{3.163930in}{1.019180in}}%
\pgfpathlineto{\pgfqpoint{3.260870in}{1.000978in}}%
\pgfpathlineto{\pgfqpoint{3.357809in}{0.977573in}}%
\pgfpathlineto{\pgfqpoint{3.454749in}{0.953740in}}%
\pgfpathlineto{\pgfqpoint{3.551689in}{0.936186in}}%
\pgfpathlineto{\pgfqpoint{3.648629in}{0.905112in}}%
\pgfpathlineto{\pgfqpoint{3.745569in}{0.877239in}}%
\pgfpathlineto{\pgfqpoint{3.842509in}{0.863278in}}%
\pgfpathlineto{\pgfqpoint{3.939449in}{0.831160in}}%
\pgfpathlineto{\pgfqpoint{4.036389in}{0.816339in}}%
\pgfusepath{stroke}%
\end{pgfscope}%
\begin{pgfscope}%
\pgfpathrectangle{\pgfqpoint{0.740433in}{0.566590in}}{\pgfqpoint{3.295956in}{1.828724in}}%
\pgfusepath{clip}%
\pgfsetbuttcap%
\pgfsetroundjoin%
\definecolor{currentfill}{rgb}{0.850000,0.324000,0.098000}%
\pgfsetfillcolor{currentfill}%
\pgfsetlinewidth{1.003750pt}%
\definecolor{currentstroke}{rgb}{0.850000,0.324000,0.098000}%
\pgfsetstrokecolor{currentstroke}%
\pgfsetdash{}{0pt}%
\pgfsys@defobject{currentmarker}{\pgfqpoint{-0.041667in}{-0.041667in}}{\pgfqpoint{0.041667in}{0.041667in}}{%
\pgfpathmoveto{\pgfqpoint{-0.041667in}{0.000000in}}%
\pgfpathlineto{\pgfqpoint{0.041667in}{0.000000in}}%
\pgfpathmoveto{\pgfqpoint{0.000000in}{-0.041667in}}%
\pgfpathlineto{\pgfqpoint{0.000000in}{0.041667in}}%
\pgfusepath{stroke,fill}%
}%
\begin{pgfscope}%
\pgfsys@transformshift{0.740433in}{2.247892in}%
\pgfsys@useobject{currentmarker}{}%
\end{pgfscope}%
\begin{pgfscope}%
\pgfsys@transformshift{1.031252in}{2.217708in}%
\pgfsys@useobject{currentmarker}{}%
\end{pgfscope}%
\begin{pgfscope}%
\pgfsys@transformshift{1.322072in}{2.212859in}%
\pgfsys@useobject{currentmarker}{}%
\end{pgfscope}%
\begin{pgfscope}%
\pgfsys@transformshift{1.612892in}{2.177679in}%
\pgfsys@useobject{currentmarker}{}%
\end{pgfscope}%
\begin{pgfscope}%
\pgfsys@transformshift{1.903711in}{2.119378in}%
\pgfsys@useobject{currentmarker}{}%
\end{pgfscope}%
\begin{pgfscope}%
\pgfsys@transformshift{2.194531in}{1.800957in}%
\pgfsys@useobject{currentmarker}{}%
\end{pgfscope}%
\begin{pgfscope}%
\pgfsys@transformshift{2.485350in}{1.664769in}%
\pgfsys@useobject{currentmarker}{}%
\end{pgfscope}%
\begin{pgfscope}%
\pgfsys@transformshift{2.776170in}{1.115400in}%
\pgfsys@useobject{currentmarker}{}%
\end{pgfscope}%
\begin{pgfscope}%
\pgfsys@transformshift{3.066990in}{1.042784in}%
\pgfsys@useobject{currentmarker}{}%
\end{pgfscope}%
\begin{pgfscope}%
\pgfsys@transformshift{3.357809in}{0.977573in}%
\pgfsys@useobject{currentmarker}{}%
\end{pgfscope}%
\begin{pgfscope}%
\pgfsys@transformshift{3.648629in}{0.905112in}%
\pgfsys@useobject{currentmarker}{}%
\end{pgfscope}%
\begin{pgfscope}%
\pgfsys@transformshift{3.939449in}{0.831160in}%
\pgfsys@useobject{currentmarker}{}%
\end{pgfscope}%
\end{pgfscope}%
\begin{pgfscope}%
\pgfpathrectangle{\pgfqpoint{0.740433in}{0.566590in}}{\pgfqpoint{3.295956in}{1.828724in}}%
\pgfusepath{clip}%
\pgfsetbuttcap%
\pgfsetroundjoin%
\pgfsetlinewidth{1.505625pt}%
\definecolor{currentstroke}{rgb}{0.000000,0.500000,0.000000}%
\pgfsetstrokecolor{currentstroke}%
\pgfsetdash{{5.550000pt}{2.400000pt}}{0.000000pt}%
\pgfpathmoveto{\pgfqpoint{0.740433in}{2.212032in}}%
\pgfpathlineto{\pgfqpoint{0.837373in}{2.182918in}}%
\pgfpathlineto{\pgfqpoint{0.934312in}{2.176216in}}%
\pgfpathlineto{\pgfqpoint{1.031252in}{2.143058in}}%
\pgfpathlineto{\pgfqpoint{1.128192in}{2.148113in}}%
\pgfpathlineto{\pgfqpoint{1.225132in}{2.140298in}}%
\pgfpathlineto{\pgfqpoint{1.322072in}{2.135531in}}%
\pgfpathlineto{\pgfqpoint{1.419012in}{2.132674in}}%
\pgfpathlineto{\pgfqpoint{1.515952in}{2.074793in}}%
\pgfpathlineto{\pgfqpoint{1.612892in}{2.071161in}}%
\pgfpathlineto{\pgfqpoint{1.709831in}{2.080392in}}%
\pgfpathlineto{\pgfqpoint{1.806771in}{2.013116in}}%
\pgfpathlineto{\pgfqpoint{1.903711in}{2.005779in}}%
\pgfpathlineto{\pgfqpoint{2.000651in}{1.970527in}}%
\pgfpathlineto{\pgfqpoint{2.097591in}{1.839420in}}%
\pgfpathlineto{\pgfqpoint{2.194531in}{1.814340in}}%
\pgfpathlineto{\pgfqpoint{2.291471in}{1.798297in}}%
\pgfpathlineto{\pgfqpoint{2.388411in}{1.736142in}}%
\pgfpathlineto{\pgfqpoint{2.485350in}{1.699869in}}%
\pgfpathlineto{\pgfqpoint{2.582290in}{1.652439in}}%
\pgfpathlineto{\pgfqpoint{2.679230in}{1.444204in}}%
\pgfpathlineto{\pgfqpoint{2.776170in}{1.410538in}}%
\pgfpathlineto{\pgfqpoint{2.873110in}{1.388062in}}%
\pgfpathlineto{\pgfqpoint{2.970050in}{1.355139in}}%
\pgfpathlineto{\pgfqpoint{3.066990in}{1.344956in}}%
\pgfpathlineto{\pgfqpoint{3.163930in}{1.323462in}}%
\pgfpathlineto{\pgfqpoint{3.260870in}{1.299165in}}%
\pgfpathlineto{\pgfqpoint{3.357809in}{1.278856in}}%
\pgfpathlineto{\pgfqpoint{3.454749in}{1.253244in}}%
\pgfpathlineto{\pgfqpoint{3.551689in}{1.217683in}}%
\pgfpathlineto{\pgfqpoint{3.648629in}{1.205258in}}%
\pgfpathlineto{\pgfqpoint{3.745569in}{1.181498in}}%
\pgfpathlineto{\pgfqpoint{3.842509in}{1.156600in}}%
\pgfpathlineto{\pgfqpoint{3.939449in}{1.131184in}}%
\pgfpathlineto{\pgfqpoint{4.036389in}{1.118823in}}%
\pgfusepath{stroke}%
\end{pgfscope}%
\begin{pgfscope}%
\pgfpathrectangle{\pgfqpoint{0.740433in}{0.566590in}}{\pgfqpoint{3.295956in}{1.828724in}}%
\pgfusepath{clip}%
\pgfsetbuttcap%
\pgfsetmiterjoin%
\definecolor{currentfill}{rgb}{0.000000,0.000000,0.000000}%
\pgfsetfillcolor{currentfill}%
\pgfsetfillopacity{0.000000}%
\pgfsetlinewidth{1.003750pt}%
\definecolor{currentstroke}{rgb}{0.000000,0.500000,0.000000}%
\pgfsetstrokecolor{currentstroke}%
\pgfsetdash{}{0pt}%
\pgfsys@defobject{currentmarker}{\pgfqpoint{-0.041667in}{-0.041667in}}{\pgfqpoint{0.041667in}{0.041667in}}{%
\pgfpathmoveto{\pgfqpoint{-0.041667in}{-0.041667in}}%
\pgfpathlineto{\pgfqpoint{0.041667in}{-0.041667in}}%
\pgfpathlineto{\pgfqpoint{0.041667in}{0.041667in}}%
\pgfpathlineto{\pgfqpoint{-0.041667in}{0.041667in}}%
\pgfpathclose%
\pgfusepath{stroke,fill}%
}%
\begin{pgfscope}%
\pgfsys@transformshift{0.740433in}{2.212032in}%
\pgfsys@useobject{currentmarker}{}%
\end{pgfscope}%
\begin{pgfscope}%
\pgfsys@transformshift{1.225132in}{2.140298in}%
\pgfsys@useobject{currentmarker}{}%
\end{pgfscope}%
\begin{pgfscope}%
\pgfsys@transformshift{1.709831in}{2.080392in}%
\pgfsys@useobject{currentmarker}{}%
\end{pgfscope}%
\begin{pgfscope}%
\pgfsys@transformshift{2.194531in}{1.814340in}%
\pgfsys@useobject{currentmarker}{}%
\end{pgfscope}%
\begin{pgfscope}%
\pgfsys@transformshift{2.679230in}{1.444204in}%
\pgfsys@useobject{currentmarker}{}%
\end{pgfscope}%
\begin{pgfscope}%
\pgfsys@transformshift{3.163930in}{1.323462in}%
\pgfsys@useobject{currentmarker}{}%
\end{pgfscope}%
\begin{pgfscope}%
\pgfsys@transformshift{3.648629in}{1.205258in}%
\pgfsys@useobject{currentmarker}{}%
\end{pgfscope}%
\end{pgfscope}%
\begin{pgfscope}%
\pgfpathrectangle{\pgfqpoint{0.740433in}{0.566590in}}{\pgfqpoint{3.295956in}{1.828724in}}%
\pgfusepath{clip}%
\pgfsetbuttcap%
\pgfsetroundjoin%
\pgfsetlinewidth{1.505625pt}%
\definecolor{currentstroke}{rgb}{0.494000,0.184000,0.556000}%
\pgfsetstrokecolor{currentstroke}%
\pgfsetdash{{5.550000pt}{2.400000pt}}{0.000000pt}%
\pgfpathmoveto{\pgfqpoint{0.740433in}{2.202626in}}%
\pgfpathlineto{\pgfqpoint{0.837373in}{2.192500in}}%
\pgfpathlineto{\pgfqpoint{0.934312in}{2.173829in}}%
\pgfpathlineto{\pgfqpoint{1.031252in}{2.183647in}}%
\pgfpathlineto{\pgfqpoint{1.128192in}{2.179288in}}%
\pgfpathlineto{\pgfqpoint{1.225132in}{2.135836in}}%
\pgfpathlineto{\pgfqpoint{1.322072in}{2.170687in}}%
\pgfpathlineto{\pgfqpoint{1.419012in}{2.159284in}}%
\pgfpathlineto{\pgfqpoint{1.515952in}{2.103556in}}%
\pgfpathlineto{\pgfqpoint{1.612892in}{1.952807in}}%
\pgfpathlineto{\pgfqpoint{1.709831in}{1.983864in}}%
\pgfpathlineto{\pgfqpoint{1.806771in}{1.887543in}}%
\pgfpathlineto{\pgfqpoint{1.903711in}{1.840527in}}%
\pgfpathlineto{\pgfqpoint{2.000651in}{1.841944in}}%
\pgfpathlineto{\pgfqpoint{2.097591in}{1.795729in}}%
\pgfpathlineto{\pgfqpoint{2.194531in}{1.756191in}}%
\pgfpathlineto{\pgfqpoint{2.291471in}{1.736524in}}%
\pgfpathlineto{\pgfqpoint{2.388411in}{1.673802in}}%
\pgfpathlineto{\pgfqpoint{2.485350in}{1.637984in}}%
\pgfpathlineto{\pgfqpoint{2.582290in}{1.584625in}}%
\pgfpathlineto{\pgfqpoint{2.679230in}{1.363660in}}%
\pgfpathlineto{\pgfqpoint{2.776170in}{1.330579in}}%
\pgfpathlineto{\pgfqpoint{2.873110in}{1.313411in}}%
\pgfpathlineto{\pgfqpoint{2.970050in}{1.281412in}}%
\pgfpathlineto{\pgfqpoint{3.066990in}{1.263107in}}%
\pgfpathlineto{\pgfqpoint{3.163930in}{1.250109in}}%
\pgfpathlineto{\pgfqpoint{3.260870in}{1.221513in}}%
\pgfpathlineto{\pgfqpoint{3.357809in}{1.201235in}}%
\pgfpathlineto{\pgfqpoint{3.454749in}{1.169899in}}%
\pgfpathlineto{\pgfqpoint{3.551689in}{1.144517in}}%
\pgfpathlineto{\pgfqpoint{3.648629in}{1.128852in}}%
\pgfpathlineto{\pgfqpoint{3.745569in}{1.102767in}}%
\pgfpathlineto{\pgfqpoint{3.842509in}{1.072018in}}%
\pgfpathlineto{\pgfqpoint{3.939449in}{1.055385in}}%
\pgfpathlineto{\pgfqpoint{4.036389in}{1.037695in}}%
\pgfusepath{stroke}%
\end{pgfscope}%
\begin{pgfscope}%
\pgfpathrectangle{\pgfqpoint{0.740433in}{0.566590in}}{\pgfqpoint{3.295956in}{1.828724in}}%
\pgfusepath{clip}%
\pgfsetbuttcap%
\pgfsetroundjoin%
\definecolor{currentfill}{rgb}{0.494000,0.184000,0.556000}%
\pgfsetfillcolor{currentfill}%
\pgfsetlinewidth{1.003750pt}%
\definecolor{currentstroke}{rgb}{0.494000,0.184000,0.556000}%
\pgfsetstrokecolor{currentstroke}%
\pgfsetdash{}{0pt}%
\pgfsys@defobject{currentmarker}{\pgfqpoint{-0.041667in}{-0.041667in}}{\pgfqpoint{0.041667in}{0.041667in}}{%
\pgfpathmoveto{\pgfqpoint{-0.041667in}{-0.041667in}}%
\pgfpathlineto{\pgfqpoint{0.041667in}{0.041667in}}%
\pgfpathmoveto{\pgfqpoint{-0.041667in}{0.041667in}}%
\pgfpathlineto{\pgfqpoint{0.041667in}{-0.041667in}}%
\pgfusepath{stroke,fill}%
}%
\begin{pgfscope}%
\pgfsys@transformshift{0.740433in}{2.202626in}%
\pgfsys@useobject{currentmarker}{}%
\end{pgfscope}%
\begin{pgfscope}%
\pgfsys@transformshift{1.128192in}{2.179288in}%
\pgfsys@useobject{currentmarker}{}%
\end{pgfscope}%
\begin{pgfscope}%
\pgfsys@transformshift{1.515952in}{2.103556in}%
\pgfsys@useobject{currentmarker}{}%
\end{pgfscope}%
\begin{pgfscope}%
\pgfsys@transformshift{1.903711in}{1.840527in}%
\pgfsys@useobject{currentmarker}{}%
\end{pgfscope}%
\begin{pgfscope}%
\pgfsys@transformshift{2.291471in}{1.736524in}%
\pgfsys@useobject{currentmarker}{}%
\end{pgfscope}%
\begin{pgfscope}%
\pgfsys@transformshift{2.679230in}{1.363660in}%
\pgfsys@useobject{currentmarker}{}%
\end{pgfscope}%
\begin{pgfscope}%
\pgfsys@transformshift{3.066990in}{1.263107in}%
\pgfsys@useobject{currentmarker}{}%
\end{pgfscope}%
\begin{pgfscope}%
\pgfsys@transformshift{3.454749in}{1.169899in}%
\pgfsys@useobject{currentmarker}{}%
\end{pgfscope}%
\begin{pgfscope}%
\pgfsys@transformshift{3.842509in}{1.072018in}%
\pgfsys@useobject{currentmarker}{}%
\end{pgfscope}%
\end{pgfscope}%
\begin{pgfscope}%
\pgfpathrectangle{\pgfqpoint{0.740433in}{0.566590in}}{\pgfqpoint{3.295956in}{1.828724in}}%
\pgfusepath{clip}%
\pgfsetbuttcap%
\pgfsetroundjoin%
\pgfsetlinewidth{1.505625pt}%
\definecolor{currentstroke}{rgb}{0.635000,0.078000,0.184000}%
\pgfsetstrokecolor{currentstroke}%
\pgfsetdash{{5.550000pt}{2.400000pt}}{0.000000pt}%
\pgfpathmoveto{\pgfqpoint{0.740433in}{2.248607in}}%
\pgfpathlineto{\pgfqpoint{0.837373in}{2.247783in}}%
\pgfpathlineto{\pgfqpoint{0.934312in}{2.248506in}}%
\pgfpathlineto{\pgfqpoint{1.031252in}{2.254433in}}%
\pgfpathlineto{\pgfqpoint{1.128192in}{2.253739in}}%
\pgfpathlineto{\pgfqpoint{1.225132in}{2.255346in}}%
\pgfpathlineto{\pgfqpoint{1.322072in}{2.245762in}}%
\pgfpathlineto{\pgfqpoint{1.419012in}{2.248698in}}%
\pgfpathlineto{\pgfqpoint{1.515952in}{2.213633in}}%
\pgfpathlineto{\pgfqpoint{1.612892in}{2.138066in}}%
\pgfpathlineto{\pgfqpoint{1.709831in}{2.139935in}}%
\pgfpathlineto{\pgfqpoint{1.806771in}{1.825790in}}%
\pgfpathlineto{\pgfqpoint{1.903711in}{1.800296in}}%
\pgfpathlineto{\pgfqpoint{2.000651in}{1.803398in}}%
\pgfpathlineto{\pgfqpoint{2.097591in}{1.752008in}}%
\pgfpathlineto{\pgfqpoint{2.194531in}{1.717395in}}%
\pgfpathlineto{\pgfqpoint{2.291471in}{1.699806in}}%
\pgfpathlineto{\pgfqpoint{2.388411in}{1.636254in}}%
\pgfpathlineto{\pgfqpoint{2.485350in}{1.600232in}}%
\pgfpathlineto{\pgfqpoint{2.582290in}{1.549783in}}%
\pgfpathlineto{\pgfqpoint{2.679230in}{1.310696in}}%
\pgfpathlineto{\pgfqpoint{2.776170in}{1.280593in}}%
\pgfpathlineto{\pgfqpoint{2.873110in}{1.262954in}}%
\pgfpathlineto{\pgfqpoint{2.970050in}{1.227472in}}%
\pgfpathlineto{\pgfqpoint{3.066990in}{1.214667in}}%
\pgfpathlineto{\pgfqpoint{3.163930in}{1.199751in}}%
\pgfpathlineto{\pgfqpoint{3.260870in}{1.170082in}}%
\pgfpathlineto{\pgfqpoint{3.357809in}{1.150396in}}%
\pgfpathlineto{\pgfqpoint{3.454749in}{1.118535in}}%
\pgfpathlineto{\pgfqpoint{3.551689in}{1.093328in}}%
\pgfpathlineto{\pgfqpoint{3.648629in}{1.074522in}}%
\pgfpathlineto{\pgfqpoint{3.745569in}{1.051063in}}%
\pgfpathlineto{\pgfqpoint{3.842509in}{1.021520in}}%
\pgfpathlineto{\pgfqpoint{3.939449in}{1.005973in}}%
\pgfpathlineto{\pgfqpoint{4.036389in}{0.987393in}}%
\pgfusepath{stroke}%
\end{pgfscope}%
\begin{pgfscope}%
\pgfpathrectangle{\pgfqpoint{0.740433in}{0.566590in}}{\pgfqpoint{3.295956in}{1.828724in}}%
\pgfusepath{clip}%
\pgfsetbuttcap%
\pgfsetmiterjoin%
\definecolor{currentfill}{rgb}{0.000000,0.000000,0.000000}%
\pgfsetfillcolor{currentfill}%
\pgfsetfillopacity{0.000000}%
\pgfsetlinewidth{1.003750pt}%
\definecolor{currentstroke}{rgb}{0.635000,0.078000,0.184000}%
\pgfsetstrokecolor{currentstroke}%
\pgfsetdash{}{0pt}%
\pgfsys@defobject{currentmarker}{\pgfqpoint{-0.035355in}{-0.058926in}}{\pgfqpoint{0.035355in}{0.058926in}}{%
\pgfpathmoveto{\pgfqpoint{-0.000000in}{-0.058926in}}%
\pgfpathlineto{\pgfqpoint{0.035355in}{0.000000in}}%
\pgfpathlineto{\pgfqpoint{0.000000in}{0.058926in}}%
\pgfpathlineto{\pgfqpoint{-0.035355in}{0.000000in}}%
\pgfpathclose%
\pgfusepath{stroke,fill}%
}%
\begin{pgfscope}%
\pgfsys@transformshift{0.740433in}{2.248607in}%
\pgfsys@useobject{currentmarker}{}%
\end{pgfscope}%
\begin{pgfscope}%
\pgfsys@transformshift{1.031252in}{2.254433in}%
\pgfsys@useobject{currentmarker}{}%
\end{pgfscope}%
\begin{pgfscope}%
\pgfsys@transformshift{1.322072in}{2.245762in}%
\pgfsys@useobject{currentmarker}{}%
\end{pgfscope}%
\begin{pgfscope}%
\pgfsys@transformshift{1.612892in}{2.138066in}%
\pgfsys@useobject{currentmarker}{}%
\end{pgfscope}%
\begin{pgfscope}%
\pgfsys@transformshift{1.903711in}{1.800296in}%
\pgfsys@useobject{currentmarker}{}%
\end{pgfscope}%
\begin{pgfscope}%
\pgfsys@transformshift{2.194531in}{1.717395in}%
\pgfsys@useobject{currentmarker}{}%
\end{pgfscope}%
\begin{pgfscope}%
\pgfsys@transformshift{2.485350in}{1.600232in}%
\pgfsys@useobject{currentmarker}{}%
\end{pgfscope}%
\begin{pgfscope}%
\pgfsys@transformshift{2.776170in}{1.280593in}%
\pgfsys@useobject{currentmarker}{}%
\end{pgfscope}%
\begin{pgfscope}%
\pgfsys@transformshift{3.066990in}{1.214667in}%
\pgfsys@useobject{currentmarker}{}%
\end{pgfscope}%
\begin{pgfscope}%
\pgfsys@transformshift{3.357809in}{1.150396in}%
\pgfsys@useobject{currentmarker}{}%
\end{pgfscope}%
\begin{pgfscope}%
\pgfsys@transformshift{3.648629in}{1.074522in}%
\pgfsys@useobject{currentmarker}{}%
\end{pgfscope}%
\begin{pgfscope}%
\pgfsys@transformshift{3.939449in}{1.005973in}%
\pgfsys@useobject{currentmarker}{}%
\end{pgfscope}%
\end{pgfscope}%
\begin{pgfscope}%
\pgfpathrectangle{\pgfqpoint{0.740433in}{0.566590in}}{\pgfqpoint{3.295956in}{1.828724in}}%
\pgfusepath{clip}%
\pgfsetbuttcap%
\pgfsetroundjoin%
\pgfsetlinewidth{1.505625pt}%
\definecolor{currentstroke}{rgb}{0.301000,0.745000,0.741000}%
\pgfsetstrokecolor{currentstroke}%
\pgfsetdash{{5.550000pt}{2.400000pt}}{0.000000pt}%
\pgfpathmoveto{\pgfqpoint{0.740433in}{2.289027in}}%
\pgfpathlineto{\pgfqpoint{0.837373in}{2.288338in}}%
\pgfpathlineto{\pgfqpoint{0.934312in}{2.293331in}}%
\pgfpathlineto{\pgfqpoint{1.031252in}{2.303686in}}%
\pgfpathlineto{\pgfqpoint{1.128192in}{2.295404in}}%
\pgfpathlineto{\pgfqpoint{1.225132in}{2.300583in}}%
\pgfpathlineto{\pgfqpoint{1.322072in}{2.297694in}}%
\pgfpathlineto{\pgfqpoint{1.419012in}{2.289923in}}%
\pgfpathlineto{\pgfqpoint{1.515952in}{2.281437in}}%
\pgfpathlineto{\pgfqpoint{1.612892in}{2.260382in}}%
\pgfpathlineto{\pgfqpoint{1.709831in}{2.240466in}}%
\pgfpathlineto{\pgfqpoint{1.806771in}{2.175083in}}%
\pgfpathlineto{\pgfqpoint{1.903711in}{2.162356in}}%
\pgfpathlineto{\pgfqpoint{2.000651in}{2.084108in}}%
\pgfpathlineto{\pgfqpoint{2.097591in}{1.809384in}}%
\pgfpathlineto{\pgfqpoint{2.194531in}{1.786822in}}%
\pgfpathlineto{\pgfqpoint{2.291471in}{1.771771in}}%
\pgfpathlineto{\pgfqpoint{2.388411in}{1.708770in}}%
\pgfpathlineto{\pgfqpoint{2.485350in}{1.672773in}}%
\pgfpathlineto{\pgfqpoint{2.582290in}{1.623872in}}%
\pgfpathlineto{\pgfqpoint{2.679230in}{1.360040in}}%
\pgfpathlineto{\pgfqpoint{2.776170in}{1.340973in}}%
\pgfpathlineto{\pgfqpoint{2.873110in}{1.307617in}}%
\pgfpathlineto{\pgfqpoint{2.970050in}{1.281865in}}%
\pgfpathlineto{\pgfqpoint{3.066990in}{1.269203in}}%
\pgfpathlineto{\pgfqpoint{3.163930in}{1.254411in}}%
\pgfpathlineto{\pgfqpoint{3.260870in}{1.210852in}}%
\pgfpathlineto{\pgfqpoint{3.357809in}{1.199648in}}%
\pgfpathlineto{\pgfqpoint{3.454749in}{1.178532in}}%
\pgfpathlineto{\pgfqpoint{3.551689in}{1.155464in}}%
\pgfpathlineto{\pgfqpoint{3.648629in}{1.129940in}}%
\pgfpathlineto{\pgfqpoint{3.745569in}{1.110658in}}%
\pgfpathlineto{\pgfqpoint{3.842509in}{1.078643in}}%
\pgfpathlineto{\pgfqpoint{3.939449in}{1.052770in}}%
\pgfpathlineto{\pgfqpoint{4.036389in}{1.034533in}}%
\pgfusepath{stroke}%
\end{pgfscope}%
\begin{pgfscope}%
\pgfpathrectangle{\pgfqpoint{0.740433in}{0.566590in}}{\pgfqpoint{3.295956in}{1.828724in}}%
\pgfusepath{clip}%
\pgfsetbuttcap%
\pgfsetmiterjoin%
\definecolor{currentfill}{rgb}{0.000000,0.000000,0.000000}%
\pgfsetfillcolor{currentfill}%
\pgfsetfillopacity{0.000000}%
\pgfsetlinewidth{1.003750pt}%
\definecolor{currentstroke}{rgb}{0.301000,0.745000,0.741000}%
\pgfsetstrokecolor{currentstroke}%
\pgfsetdash{}{0pt}%
\pgfsys@defobject{currentmarker}{\pgfqpoint{-0.041667in}{-0.041667in}}{\pgfqpoint{0.041667in}{0.041667in}}{%
\pgfpathmoveto{\pgfqpoint{0.000000in}{0.041667in}}%
\pgfpathlineto{\pgfqpoint{-0.041667in}{-0.041667in}}%
\pgfpathlineto{\pgfqpoint{0.041667in}{-0.041667in}}%
\pgfpathclose%
\pgfusepath{stroke,fill}%
}%
\begin{pgfscope}%
\pgfsys@transformshift{0.740433in}{2.289027in}%
\pgfsys@useobject{currentmarker}{}%
\end{pgfscope}%
\begin{pgfscope}%
\pgfsys@transformshift{1.225132in}{2.300583in}%
\pgfsys@useobject{currentmarker}{}%
\end{pgfscope}%
\begin{pgfscope}%
\pgfsys@transformshift{1.709831in}{2.240466in}%
\pgfsys@useobject{currentmarker}{}%
\end{pgfscope}%
\begin{pgfscope}%
\pgfsys@transformshift{2.194531in}{1.786822in}%
\pgfsys@useobject{currentmarker}{}%
\end{pgfscope}%
\begin{pgfscope}%
\pgfsys@transformshift{2.679230in}{1.360040in}%
\pgfsys@useobject{currentmarker}{}%
\end{pgfscope}%
\begin{pgfscope}%
\pgfsys@transformshift{3.163930in}{1.254411in}%
\pgfsys@useobject{currentmarker}{}%
\end{pgfscope}%
\begin{pgfscope}%
\pgfsys@transformshift{3.648629in}{1.129940in}%
\pgfsys@useobject{currentmarker}{}%
\end{pgfscope}%
\end{pgfscope}%
\begin{pgfscope}%
\pgfpathrectangle{\pgfqpoint{0.740433in}{0.566590in}}{\pgfqpoint{3.295956in}{1.828724in}}%
\pgfusepath{clip}%
\pgfsetrectcap%
\pgfsetroundjoin%
\pgfsetlinewidth{1.505625pt}%
\definecolor{currentstroke}{rgb}{0.000000,0.447000,0.741000}%
\pgfsetstrokecolor{currentstroke}%
\pgfsetdash{}{0pt}%
\pgfpathmoveto{\pgfqpoint{0.740433in}{1.893416in}}%
\pgfpathlineto{\pgfqpoint{0.975858in}{1.882000in}}%
\pgfpathlineto{\pgfqpoint{1.211283in}{1.845105in}}%
\pgfpathlineto{\pgfqpoint{1.446709in}{1.740717in}}%
\pgfpathlineto{\pgfqpoint{1.682134in}{1.445281in}}%
\pgfpathlineto{\pgfqpoint{1.917560in}{1.251519in}}%
\pgfpathlineto{\pgfqpoint{2.152985in}{1.196325in}}%
\pgfpathlineto{\pgfqpoint{2.388411in}{1.133343in}}%
\pgfpathlineto{\pgfqpoint{2.623836in}{1.082166in}}%
\pgfpathlineto{\pgfqpoint{2.859261in}{1.017601in}}%
\pgfpathlineto{\pgfqpoint{3.094687in}{0.959403in}}%
\pgfpathlineto{\pgfqpoint{3.330112in}{0.922494in}}%
\pgfpathlineto{\pgfqpoint{3.565538in}{0.861979in}}%
\pgfpathlineto{\pgfqpoint{3.800963in}{0.801417in}}%
\pgfpathlineto{\pgfqpoint{4.036389in}{0.739205in}}%
\pgfusepath{stroke}%
\end{pgfscope}%
\begin{pgfscope}%
\pgfpathrectangle{\pgfqpoint{0.740433in}{0.566590in}}{\pgfqpoint{3.295956in}{1.828724in}}%
\pgfusepath{clip}%
\pgfsetbuttcap%
\pgfsetroundjoin%
\definecolor{currentfill}{rgb}{0.000000,0.000000,0.000000}%
\pgfsetfillcolor{currentfill}%
\pgfsetfillopacity{0.000000}%
\pgfsetlinewidth{1.003750pt}%
\definecolor{currentstroke}{rgb}{0.000000,0.447000,0.741000}%
\pgfsetstrokecolor{currentstroke}%
\pgfsetdash{}{0pt}%
\pgfsys@defobject{currentmarker}{\pgfqpoint{-0.041667in}{-0.041667in}}{\pgfqpoint{0.041667in}{0.041667in}}{%
\pgfpathmoveto{\pgfqpoint{0.000000in}{-0.041667in}}%
\pgfpathcurveto{\pgfqpoint{0.011050in}{-0.041667in}}{\pgfqpoint{0.021649in}{-0.037276in}}{\pgfqpoint{0.029463in}{-0.029463in}}%
\pgfpathcurveto{\pgfqpoint{0.037276in}{-0.021649in}}{\pgfqpoint{0.041667in}{-0.011050in}}{\pgfqpoint{0.041667in}{0.000000in}}%
\pgfpathcurveto{\pgfqpoint{0.041667in}{0.011050in}}{\pgfqpoint{0.037276in}{0.021649in}}{\pgfqpoint{0.029463in}{0.029463in}}%
\pgfpathcurveto{\pgfqpoint{0.021649in}{0.037276in}}{\pgfqpoint{0.011050in}{0.041667in}}{\pgfqpoint{0.000000in}{0.041667in}}%
\pgfpathcurveto{\pgfqpoint{-0.011050in}{0.041667in}}{\pgfqpoint{-0.021649in}{0.037276in}}{\pgfqpoint{-0.029463in}{0.029463in}}%
\pgfpathcurveto{\pgfqpoint{-0.037276in}{0.021649in}}{\pgfqpoint{-0.041667in}{0.011050in}}{\pgfqpoint{-0.041667in}{0.000000in}}%
\pgfpathcurveto{\pgfqpoint{-0.041667in}{-0.011050in}}{\pgfqpoint{-0.037276in}{-0.021649in}}{\pgfqpoint{-0.029463in}{-0.029463in}}%
\pgfpathcurveto{\pgfqpoint{-0.021649in}{-0.037276in}}{\pgfqpoint{-0.011050in}{-0.041667in}}{\pgfqpoint{0.000000in}{-0.041667in}}%
\pgfpathclose%
\pgfusepath{stroke,fill}%
}%
\begin{pgfscope}%
\pgfsys@transformshift{0.740433in}{1.893416in}%
\pgfsys@useobject{currentmarker}{}%
\end{pgfscope}%
\begin{pgfscope}%
\pgfsys@transformshift{0.975858in}{1.882000in}%
\pgfsys@useobject{currentmarker}{}%
\end{pgfscope}%
\begin{pgfscope}%
\pgfsys@transformshift{1.211283in}{1.845105in}%
\pgfsys@useobject{currentmarker}{}%
\end{pgfscope}%
\begin{pgfscope}%
\pgfsys@transformshift{1.446709in}{1.740717in}%
\pgfsys@useobject{currentmarker}{}%
\end{pgfscope}%
\begin{pgfscope}%
\pgfsys@transformshift{1.682134in}{1.445281in}%
\pgfsys@useobject{currentmarker}{}%
\end{pgfscope}%
\begin{pgfscope}%
\pgfsys@transformshift{1.917560in}{1.251519in}%
\pgfsys@useobject{currentmarker}{}%
\end{pgfscope}%
\begin{pgfscope}%
\pgfsys@transformshift{2.152985in}{1.196325in}%
\pgfsys@useobject{currentmarker}{}%
\end{pgfscope}%
\begin{pgfscope}%
\pgfsys@transformshift{2.388411in}{1.133343in}%
\pgfsys@useobject{currentmarker}{}%
\end{pgfscope}%
\begin{pgfscope}%
\pgfsys@transformshift{2.623836in}{1.082166in}%
\pgfsys@useobject{currentmarker}{}%
\end{pgfscope}%
\begin{pgfscope}%
\pgfsys@transformshift{2.859261in}{1.017601in}%
\pgfsys@useobject{currentmarker}{}%
\end{pgfscope}%
\begin{pgfscope}%
\pgfsys@transformshift{3.094687in}{0.959403in}%
\pgfsys@useobject{currentmarker}{}%
\end{pgfscope}%
\begin{pgfscope}%
\pgfsys@transformshift{3.330112in}{0.922494in}%
\pgfsys@useobject{currentmarker}{}%
\end{pgfscope}%
\begin{pgfscope}%
\pgfsys@transformshift{3.565538in}{0.861979in}%
\pgfsys@useobject{currentmarker}{}%
\end{pgfscope}%
\begin{pgfscope}%
\pgfsys@transformshift{3.800963in}{0.801417in}%
\pgfsys@useobject{currentmarker}{}%
\end{pgfscope}%
\begin{pgfscope}%
\pgfsys@transformshift{4.036389in}{0.739205in}%
\pgfsys@useobject{currentmarker}{}%
\end{pgfscope}%
\end{pgfscope}%
\begin{pgfscope}%
\pgfpathrectangle{\pgfqpoint{0.740433in}{0.566590in}}{\pgfqpoint{3.295956in}{1.828724in}}%
\pgfusepath{clip}%
\pgfsetrectcap%
\pgfsetroundjoin%
\pgfsetlinewidth{1.505625pt}%
\definecolor{currentstroke}{rgb}{0.850000,0.324000,0.098000}%
\pgfsetstrokecolor{currentstroke}%
\pgfsetdash{}{0pt}%
\pgfpathmoveto{\pgfqpoint{0.740433in}{1.839270in}}%
\pgfpathlineto{\pgfqpoint{0.975858in}{1.799877in}}%
\pgfpathlineto{\pgfqpoint{1.211283in}{1.727348in}}%
\pgfpathlineto{\pgfqpoint{1.446709in}{1.554836in}}%
\pgfpathlineto{\pgfqpoint{1.682134in}{1.218743in}}%
\pgfpathlineto{\pgfqpoint{1.917560in}{1.106954in}}%
\pgfpathlineto{\pgfqpoint{2.152985in}{1.045424in}}%
\pgfpathlineto{\pgfqpoint{2.388411in}{0.984752in}}%
\pgfpathlineto{\pgfqpoint{2.623836in}{0.929010in}}%
\pgfpathlineto{\pgfqpoint{2.859261in}{0.876269in}}%
\pgfpathlineto{\pgfqpoint{3.094687in}{0.806986in}}%
\pgfpathlineto{\pgfqpoint{3.330112in}{0.773329in}}%
\pgfpathlineto{\pgfqpoint{3.565538in}{0.701049in}}%
\pgfpathlineto{\pgfqpoint{3.800963in}{0.645043in}}%
\pgfpathlineto{\pgfqpoint{4.036389in}{0.589134in}}%
\pgfusepath{stroke}%
\end{pgfscope}%
\begin{pgfscope}%
\pgfpathrectangle{\pgfqpoint{0.740433in}{0.566590in}}{\pgfqpoint{3.295956in}{1.828724in}}%
\pgfusepath{clip}%
\pgfsetbuttcap%
\pgfsetroundjoin%
\definecolor{currentfill}{rgb}{0.850000,0.324000,0.098000}%
\pgfsetfillcolor{currentfill}%
\pgfsetlinewidth{1.003750pt}%
\definecolor{currentstroke}{rgb}{0.850000,0.324000,0.098000}%
\pgfsetstrokecolor{currentstroke}%
\pgfsetdash{}{0pt}%
\pgfsys@defobject{currentmarker}{\pgfqpoint{-0.041667in}{-0.041667in}}{\pgfqpoint{0.041667in}{0.041667in}}{%
\pgfpathmoveto{\pgfqpoint{-0.041667in}{0.000000in}}%
\pgfpathlineto{\pgfqpoint{0.041667in}{0.000000in}}%
\pgfpathmoveto{\pgfqpoint{0.000000in}{-0.041667in}}%
\pgfpathlineto{\pgfqpoint{0.000000in}{0.041667in}}%
\pgfusepath{stroke,fill}%
}%
\begin{pgfscope}%
\pgfsys@transformshift{0.740433in}{1.839270in}%
\pgfsys@useobject{currentmarker}{}%
\end{pgfscope}%
\begin{pgfscope}%
\pgfsys@transformshift{0.975858in}{1.799877in}%
\pgfsys@useobject{currentmarker}{}%
\end{pgfscope}%
\begin{pgfscope}%
\pgfsys@transformshift{1.211283in}{1.727348in}%
\pgfsys@useobject{currentmarker}{}%
\end{pgfscope}%
\begin{pgfscope}%
\pgfsys@transformshift{1.446709in}{1.554836in}%
\pgfsys@useobject{currentmarker}{}%
\end{pgfscope}%
\begin{pgfscope}%
\pgfsys@transformshift{1.682134in}{1.218743in}%
\pgfsys@useobject{currentmarker}{}%
\end{pgfscope}%
\begin{pgfscope}%
\pgfsys@transformshift{1.917560in}{1.106954in}%
\pgfsys@useobject{currentmarker}{}%
\end{pgfscope}%
\begin{pgfscope}%
\pgfsys@transformshift{2.152985in}{1.045424in}%
\pgfsys@useobject{currentmarker}{}%
\end{pgfscope}%
\begin{pgfscope}%
\pgfsys@transformshift{2.388411in}{0.984752in}%
\pgfsys@useobject{currentmarker}{}%
\end{pgfscope}%
\begin{pgfscope}%
\pgfsys@transformshift{2.623836in}{0.929010in}%
\pgfsys@useobject{currentmarker}{}%
\end{pgfscope}%
\begin{pgfscope}%
\pgfsys@transformshift{2.859261in}{0.876269in}%
\pgfsys@useobject{currentmarker}{}%
\end{pgfscope}%
\begin{pgfscope}%
\pgfsys@transformshift{3.094687in}{0.806986in}%
\pgfsys@useobject{currentmarker}{}%
\end{pgfscope}%
\begin{pgfscope}%
\pgfsys@transformshift{3.330112in}{0.773329in}%
\pgfsys@useobject{currentmarker}{}%
\end{pgfscope}%
\begin{pgfscope}%
\pgfsys@transformshift{3.565538in}{0.701049in}%
\pgfsys@useobject{currentmarker}{}%
\end{pgfscope}%
\begin{pgfscope}%
\pgfsys@transformshift{3.800963in}{0.645043in}%
\pgfsys@useobject{currentmarker}{}%
\end{pgfscope}%
\begin{pgfscope}%
\pgfsys@transformshift{4.036389in}{0.589134in}%
\pgfsys@useobject{currentmarker}{}%
\end{pgfscope}%
\end{pgfscope}%
\begin{pgfscope}%
\pgfpathrectangle{\pgfqpoint{0.740433in}{0.566590in}}{\pgfqpoint{3.295956in}{1.828724in}}%
\pgfusepath{clip}%
\pgfsetrectcap%
\pgfsetroundjoin%
\pgfsetlinewidth{1.505625pt}%
\definecolor{currentstroke}{rgb}{0.000000,0.500000,0.000000}%
\pgfsetstrokecolor{currentstroke}%
\pgfsetdash{}{0pt}%
\pgfpathmoveto{\pgfqpoint{0.740433in}{1.733058in}}%
\pgfpathlineto{\pgfqpoint{0.975858in}{1.695975in}}%
\pgfpathlineto{\pgfqpoint{1.211283in}{1.602655in}}%
\pgfpathlineto{\pgfqpoint{1.446709in}{1.546163in}}%
\pgfpathlineto{\pgfqpoint{1.682134in}{1.411849in}}%
\pgfpathlineto{\pgfqpoint{1.917560in}{1.314984in}}%
\pgfpathlineto{\pgfqpoint{2.152985in}{1.256770in}}%
\pgfpathlineto{\pgfqpoint{2.388411in}{1.190237in}}%
\pgfpathlineto{\pgfqpoint{2.623836in}{1.126179in}}%
\pgfpathlineto{\pgfqpoint{2.859261in}{1.081910in}}%
\pgfpathlineto{\pgfqpoint{3.094687in}{1.024233in}}%
\pgfpathlineto{\pgfqpoint{3.330112in}{0.970410in}}%
\pgfpathlineto{\pgfqpoint{3.565538in}{0.919011in}}%
\pgfpathlineto{\pgfqpoint{3.800963in}{0.861258in}}%
\pgfpathlineto{\pgfqpoint{4.036389in}{0.800847in}}%
\pgfusepath{stroke}%
\end{pgfscope}%
\begin{pgfscope}%
\pgfpathrectangle{\pgfqpoint{0.740433in}{0.566590in}}{\pgfqpoint{3.295956in}{1.828724in}}%
\pgfusepath{clip}%
\pgfsetbuttcap%
\pgfsetmiterjoin%
\definecolor{currentfill}{rgb}{0.000000,0.000000,0.000000}%
\pgfsetfillcolor{currentfill}%
\pgfsetfillopacity{0.000000}%
\pgfsetlinewidth{1.003750pt}%
\definecolor{currentstroke}{rgb}{0.000000,0.500000,0.000000}%
\pgfsetstrokecolor{currentstroke}%
\pgfsetdash{}{0pt}%
\pgfsys@defobject{currentmarker}{\pgfqpoint{-0.041667in}{-0.041667in}}{\pgfqpoint{0.041667in}{0.041667in}}{%
\pgfpathmoveto{\pgfqpoint{-0.041667in}{-0.041667in}}%
\pgfpathlineto{\pgfqpoint{0.041667in}{-0.041667in}}%
\pgfpathlineto{\pgfqpoint{0.041667in}{0.041667in}}%
\pgfpathlineto{\pgfqpoint{-0.041667in}{0.041667in}}%
\pgfpathclose%
\pgfusepath{stroke,fill}%
}%
\begin{pgfscope}%
\pgfsys@transformshift{0.740433in}{1.733058in}%
\pgfsys@useobject{currentmarker}{}%
\end{pgfscope}%
\begin{pgfscope}%
\pgfsys@transformshift{0.975858in}{1.695975in}%
\pgfsys@useobject{currentmarker}{}%
\end{pgfscope}%
\begin{pgfscope}%
\pgfsys@transformshift{1.211283in}{1.602655in}%
\pgfsys@useobject{currentmarker}{}%
\end{pgfscope}%
\begin{pgfscope}%
\pgfsys@transformshift{1.446709in}{1.546163in}%
\pgfsys@useobject{currentmarker}{}%
\end{pgfscope}%
\begin{pgfscope}%
\pgfsys@transformshift{1.682134in}{1.411849in}%
\pgfsys@useobject{currentmarker}{}%
\end{pgfscope}%
\begin{pgfscope}%
\pgfsys@transformshift{1.917560in}{1.314984in}%
\pgfsys@useobject{currentmarker}{}%
\end{pgfscope}%
\begin{pgfscope}%
\pgfsys@transformshift{2.152985in}{1.256770in}%
\pgfsys@useobject{currentmarker}{}%
\end{pgfscope}%
\begin{pgfscope}%
\pgfsys@transformshift{2.388411in}{1.190237in}%
\pgfsys@useobject{currentmarker}{}%
\end{pgfscope}%
\begin{pgfscope}%
\pgfsys@transformshift{2.623836in}{1.126179in}%
\pgfsys@useobject{currentmarker}{}%
\end{pgfscope}%
\begin{pgfscope}%
\pgfsys@transformshift{2.859261in}{1.081910in}%
\pgfsys@useobject{currentmarker}{}%
\end{pgfscope}%
\begin{pgfscope}%
\pgfsys@transformshift{3.094687in}{1.024233in}%
\pgfsys@useobject{currentmarker}{}%
\end{pgfscope}%
\begin{pgfscope}%
\pgfsys@transformshift{3.330112in}{0.970410in}%
\pgfsys@useobject{currentmarker}{}%
\end{pgfscope}%
\begin{pgfscope}%
\pgfsys@transformshift{3.565538in}{0.919011in}%
\pgfsys@useobject{currentmarker}{}%
\end{pgfscope}%
\begin{pgfscope}%
\pgfsys@transformshift{3.800963in}{0.861258in}%
\pgfsys@useobject{currentmarker}{}%
\end{pgfscope}%
\begin{pgfscope}%
\pgfsys@transformshift{4.036389in}{0.800847in}%
\pgfsys@useobject{currentmarker}{}%
\end{pgfscope}%
\end{pgfscope}%
\begin{pgfscope}%
\pgfpathrectangle{\pgfqpoint{0.740433in}{0.566590in}}{\pgfqpoint{3.295956in}{1.828724in}}%
\pgfusepath{clip}%
\pgfsetrectcap%
\pgfsetroundjoin%
\pgfsetlinewidth{1.505625pt}%
\definecolor{currentstroke}{rgb}{0.494000,0.184000,0.556000}%
\pgfsetstrokecolor{currentstroke}%
\pgfsetdash{}{0pt}%
\pgfpathmoveto{\pgfqpoint{0.740433in}{1.754319in}}%
\pgfpathlineto{\pgfqpoint{0.975858in}{1.653913in}}%
\pgfpathlineto{\pgfqpoint{1.211283in}{1.602252in}}%
\pgfpathlineto{\pgfqpoint{1.446709in}{1.533872in}}%
\pgfpathlineto{\pgfqpoint{1.682134in}{1.326726in}}%
\pgfpathlineto{\pgfqpoint{1.917560in}{1.218703in}}%
\pgfpathlineto{\pgfqpoint{2.152985in}{1.161805in}}%
\pgfpathlineto{\pgfqpoint{2.388411in}{1.102409in}}%
\pgfpathlineto{\pgfqpoint{2.623836in}{1.048927in}}%
\pgfpathlineto{\pgfqpoint{2.859261in}{1.001396in}}%
\pgfpathlineto{\pgfqpoint{3.094687in}{0.939798in}}%
\pgfpathlineto{\pgfqpoint{3.330112in}{0.883734in}}%
\pgfpathlineto{\pgfqpoint{3.565538in}{0.837594in}}%
\pgfpathlineto{\pgfqpoint{3.800963in}{0.768867in}}%
\pgfpathlineto{\pgfqpoint{4.036389in}{0.708199in}}%
\pgfusepath{stroke}%
\end{pgfscope}%
\begin{pgfscope}%
\pgfpathrectangle{\pgfqpoint{0.740433in}{0.566590in}}{\pgfqpoint{3.295956in}{1.828724in}}%
\pgfusepath{clip}%
\pgfsetbuttcap%
\pgfsetroundjoin%
\definecolor{currentfill}{rgb}{0.494000,0.184000,0.556000}%
\pgfsetfillcolor{currentfill}%
\pgfsetlinewidth{1.003750pt}%
\definecolor{currentstroke}{rgb}{0.494000,0.184000,0.556000}%
\pgfsetstrokecolor{currentstroke}%
\pgfsetdash{}{0pt}%
\pgfsys@defobject{currentmarker}{\pgfqpoint{-0.041667in}{-0.041667in}}{\pgfqpoint{0.041667in}{0.041667in}}{%
\pgfpathmoveto{\pgfqpoint{-0.041667in}{-0.041667in}}%
\pgfpathlineto{\pgfqpoint{0.041667in}{0.041667in}}%
\pgfpathmoveto{\pgfqpoint{-0.041667in}{0.041667in}}%
\pgfpathlineto{\pgfqpoint{0.041667in}{-0.041667in}}%
\pgfusepath{stroke,fill}%
}%
\begin{pgfscope}%
\pgfsys@transformshift{0.740433in}{1.754319in}%
\pgfsys@useobject{currentmarker}{}%
\end{pgfscope}%
\begin{pgfscope}%
\pgfsys@transformshift{0.975858in}{1.653913in}%
\pgfsys@useobject{currentmarker}{}%
\end{pgfscope}%
\begin{pgfscope}%
\pgfsys@transformshift{1.211283in}{1.602252in}%
\pgfsys@useobject{currentmarker}{}%
\end{pgfscope}%
\begin{pgfscope}%
\pgfsys@transformshift{1.446709in}{1.533872in}%
\pgfsys@useobject{currentmarker}{}%
\end{pgfscope}%
\begin{pgfscope}%
\pgfsys@transformshift{1.682134in}{1.326726in}%
\pgfsys@useobject{currentmarker}{}%
\end{pgfscope}%
\begin{pgfscope}%
\pgfsys@transformshift{1.917560in}{1.218703in}%
\pgfsys@useobject{currentmarker}{}%
\end{pgfscope}%
\begin{pgfscope}%
\pgfsys@transformshift{2.152985in}{1.161805in}%
\pgfsys@useobject{currentmarker}{}%
\end{pgfscope}%
\begin{pgfscope}%
\pgfsys@transformshift{2.388411in}{1.102409in}%
\pgfsys@useobject{currentmarker}{}%
\end{pgfscope}%
\begin{pgfscope}%
\pgfsys@transformshift{2.623836in}{1.048927in}%
\pgfsys@useobject{currentmarker}{}%
\end{pgfscope}%
\begin{pgfscope}%
\pgfsys@transformshift{2.859261in}{1.001396in}%
\pgfsys@useobject{currentmarker}{}%
\end{pgfscope}%
\begin{pgfscope}%
\pgfsys@transformshift{3.094687in}{0.939798in}%
\pgfsys@useobject{currentmarker}{}%
\end{pgfscope}%
\begin{pgfscope}%
\pgfsys@transformshift{3.330112in}{0.883734in}%
\pgfsys@useobject{currentmarker}{}%
\end{pgfscope}%
\begin{pgfscope}%
\pgfsys@transformshift{3.565538in}{0.837594in}%
\pgfsys@useobject{currentmarker}{}%
\end{pgfscope}%
\begin{pgfscope}%
\pgfsys@transformshift{3.800963in}{0.768867in}%
\pgfsys@useobject{currentmarker}{}%
\end{pgfscope}%
\begin{pgfscope}%
\pgfsys@transformshift{4.036389in}{0.708199in}%
\pgfsys@useobject{currentmarker}{}%
\end{pgfscope}%
\end{pgfscope}%
\begin{pgfscope}%
\pgfpathrectangle{\pgfqpoint{0.740433in}{0.566590in}}{\pgfqpoint{3.295956in}{1.828724in}}%
\pgfusepath{clip}%
\pgfsetrectcap%
\pgfsetroundjoin%
\pgfsetlinewidth{1.505625pt}%
\definecolor{currentstroke}{rgb}{0.635000,0.078000,0.184000}%
\pgfsetstrokecolor{currentstroke}%
\pgfsetdash{}{0pt}%
\pgfpathmoveto{\pgfqpoint{0.740433in}{1.830149in}}%
\pgfpathlineto{\pgfqpoint{0.975858in}{1.789722in}}%
\pgfpathlineto{\pgfqpoint{1.211283in}{1.717629in}}%
\pgfpathlineto{\pgfqpoint{1.446709in}{1.505538in}}%
\pgfpathlineto{\pgfqpoint{1.682134in}{1.220327in}}%
\pgfpathlineto{\pgfqpoint{1.917560in}{1.122277in}}%
\pgfpathlineto{\pgfqpoint{2.152985in}{1.071316in}}%
\pgfpathlineto{\pgfqpoint{2.388411in}{1.006944in}}%
\pgfpathlineto{\pgfqpoint{2.623836in}{0.956811in}}%
\pgfpathlineto{\pgfqpoint{2.859261in}{0.909678in}}%
\pgfpathlineto{\pgfqpoint{3.094687in}{0.852827in}}%
\pgfpathlineto{\pgfqpoint{3.330112in}{0.792984in}}%
\pgfpathlineto{\pgfqpoint{3.565538in}{0.738614in}}%
\pgfpathlineto{\pgfqpoint{3.800963in}{0.681089in}}%
\pgfpathlineto{\pgfqpoint{4.036389in}{0.614741in}}%
\pgfusepath{stroke}%
\end{pgfscope}%
\begin{pgfscope}%
\pgfpathrectangle{\pgfqpoint{0.740433in}{0.566590in}}{\pgfqpoint{3.295956in}{1.828724in}}%
\pgfusepath{clip}%
\pgfsetbuttcap%
\pgfsetmiterjoin%
\definecolor{currentfill}{rgb}{0.000000,0.000000,0.000000}%
\pgfsetfillcolor{currentfill}%
\pgfsetfillopacity{0.000000}%
\pgfsetlinewidth{1.003750pt}%
\definecolor{currentstroke}{rgb}{0.635000,0.078000,0.184000}%
\pgfsetstrokecolor{currentstroke}%
\pgfsetdash{}{0pt}%
\pgfsys@defobject{currentmarker}{\pgfqpoint{-0.035355in}{-0.058926in}}{\pgfqpoint{0.035355in}{0.058926in}}{%
\pgfpathmoveto{\pgfqpoint{-0.000000in}{-0.058926in}}%
\pgfpathlineto{\pgfqpoint{0.035355in}{0.000000in}}%
\pgfpathlineto{\pgfqpoint{0.000000in}{0.058926in}}%
\pgfpathlineto{\pgfqpoint{-0.035355in}{0.000000in}}%
\pgfpathclose%
\pgfusepath{stroke,fill}%
}%
\begin{pgfscope}%
\pgfsys@transformshift{0.740433in}{1.830149in}%
\pgfsys@useobject{currentmarker}{}%
\end{pgfscope}%
\begin{pgfscope}%
\pgfsys@transformshift{0.975858in}{1.789722in}%
\pgfsys@useobject{currentmarker}{}%
\end{pgfscope}%
\begin{pgfscope}%
\pgfsys@transformshift{1.211283in}{1.717629in}%
\pgfsys@useobject{currentmarker}{}%
\end{pgfscope}%
\begin{pgfscope}%
\pgfsys@transformshift{1.446709in}{1.505538in}%
\pgfsys@useobject{currentmarker}{}%
\end{pgfscope}%
\begin{pgfscope}%
\pgfsys@transformshift{1.682134in}{1.220327in}%
\pgfsys@useobject{currentmarker}{}%
\end{pgfscope}%
\begin{pgfscope}%
\pgfsys@transformshift{1.917560in}{1.122277in}%
\pgfsys@useobject{currentmarker}{}%
\end{pgfscope}%
\begin{pgfscope}%
\pgfsys@transformshift{2.152985in}{1.071316in}%
\pgfsys@useobject{currentmarker}{}%
\end{pgfscope}%
\begin{pgfscope}%
\pgfsys@transformshift{2.388411in}{1.006944in}%
\pgfsys@useobject{currentmarker}{}%
\end{pgfscope}%
\begin{pgfscope}%
\pgfsys@transformshift{2.623836in}{0.956811in}%
\pgfsys@useobject{currentmarker}{}%
\end{pgfscope}%
\begin{pgfscope}%
\pgfsys@transformshift{2.859261in}{0.909678in}%
\pgfsys@useobject{currentmarker}{}%
\end{pgfscope}%
\begin{pgfscope}%
\pgfsys@transformshift{3.094687in}{0.852827in}%
\pgfsys@useobject{currentmarker}{}%
\end{pgfscope}%
\begin{pgfscope}%
\pgfsys@transformshift{3.330112in}{0.792984in}%
\pgfsys@useobject{currentmarker}{}%
\end{pgfscope}%
\begin{pgfscope}%
\pgfsys@transformshift{3.565538in}{0.738614in}%
\pgfsys@useobject{currentmarker}{}%
\end{pgfscope}%
\begin{pgfscope}%
\pgfsys@transformshift{3.800963in}{0.681089in}%
\pgfsys@useobject{currentmarker}{}%
\end{pgfscope}%
\begin{pgfscope}%
\pgfsys@transformshift{4.036389in}{0.614741in}%
\pgfsys@useobject{currentmarker}{}%
\end{pgfscope}%
\end{pgfscope}%
\begin{pgfscope}%
\pgfpathrectangle{\pgfqpoint{0.740433in}{0.566590in}}{\pgfqpoint{3.295956in}{1.828724in}}%
\pgfusepath{clip}%
\pgfsetrectcap%
\pgfsetroundjoin%
\pgfsetlinewidth{1.505625pt}%
\definecolor{currentstroke}{rgb}{0.301000,0.745000,0.741000}%
\pgfsetstrokecolor{currentstroke}%
\pgfsetdash{}{0pt}%
\pgfpathmoveto{\pgfqpoint{0.740433in}{1.894198in}}%
\pgfpathlineto{\pgfqpoint{0.975858in}{1.885931in}}%
\pgfpathlineto{\pgfqpoint{1.211283in}{1.845028in}}%
\pgfpathlineto{\pgfqpoint{1.446709in}{1.759513in}}%
\pgfpathlineto{\pgfqpoint{1.682134in}{1.444719in}}%
\pgfpathlineto{\pgfqpoint{1.917560in}{1.346558in}}%
\pgfpathlineto{\pgfqpoint{2.152985in}{1.252842in}}%
\pgfpathlineto{\pgfqpoint{2.388411in}{1.175809in}}%
\pgfpathlineto{\pgfqpoint{2.623836in}{1.124633in}}%
\pgfpathlineto{\pgfqpoint{2.859261in}{1.071432in}}%
\pgfpathlineto{\pgfqpoint{3.094687in}{1.011919in}}%
\pgfpathlineto{\pgfqpoint{3.330112in}{0.959407in}}%
\pgfpathlineto{\pgfqpoint{3.565538in}{0.902650in}}%
\pgfpathlineto{\pgfqpoint{3.800963in}{0.843270in}}%
\pgfpathlineto{\pgfqpoint{4.036389in}{0.802290in}}%
\pgfusepath{stroke}%
\end{pgfscope}%
\begin{pgfscope}%
\pgfpathrectangle{\pgfqpoint{0.740433in}{0.566590in}}{\pgfqpoint{3.295956in}{1.828724in}}%
\pgfusepath{clip}%
\pgfsetbuttcap%
\pgfsetmiterjoin%
\definecolor{currentfill}{rgb}{0.000000,0.000000,0.000000}%
\pgfsetfillcolor{currentfill}%
\pgfsetfillopacity{0.000000}%
\pgfsetlinewidth{1.003750pt}%
\definecolor{currentstroke}{rgb}{0.301000,0.745000,0.741000}%
\pgfsetstrokecolor{currentstroke}%
\pgfsetdash{}{0pt}%
\pgfsys@defobject{currentmarker}{\pgfqpoint{-0.041667in}{-0.041667in}}{\pgfqpoint{0.041667in}{0.041667in}}{%
\pgfpathmoveto{\pgfqpoint{0.000000in}{0.041667in}}%
\pgfpathlineto{\pgfqpoint{-0.041667in}{-0.041667in}}%
\pgfpathlineto{\pgfqpoint{0.041667in}{-0.041667in}}%
\pgfpathclose%
\pgfusepath{stroke,fill}%
}%
\begin{pgfscope}%
\pgfsys@transformshift{0.740433in}{1.894198in}%
\pgfsys@useobject{currentmarker}{}%
\end{pgfscope}%
\begin{pgfscope}%
\pgfsys@transformshift{0.975858in}{1.885931in}%
\pgfsys@useobject{currentmarker}{}%
\end{pgfscope}%
\begin{pgfscope}%
\pgfsys@transformshift{1.211283in}{1.845028in}%
\pgfsys@useobject{currentmarker}{}%
\end{pgfscope}%
\begin{pgfscope}%
\pgfsys@transformshift{1.446709in}{1.759513in}%
\pgfsys@useobject{currentmarker}{}%
\end{pgfscope}%
\begin{pgfscope}%
\pgfsys@transformshift{1.682134in}{1.444719in}%
\pgfsys@useobject{currentmarker}{}%
\end{pgfscope}%
\begin{pgfscope}%
\pgfsys@transformshift{1.917560in}{1.346558in}%
\pgfsys@useobject{currentmarker}{}%
\end{pgfscope}%
\begin{pgfscope}%
\pgfsys@transformshift{2.152985in}{1.252842in}%
\pgfsys@useobject{currentmarker}{}%
\end{pgfscope}%
\begin{pgfscope}%
\pgfsys@transformshift{2.388411in}{1.175809in}%
\pgfsys@useobject{currentmarker}{}%
\end{pgfscope}%
\begin{pgfscope}%
\pgfsys@transformshift{2.623836in}{1.124633in}%
\pgfsys@useobject{currentmarker}{}%
\end{pgfscope}%
\begin{pgfscope}%
\pgfsys@transformshift{2.859261in}{1.071432in}%
\pgfsys@useobject{currentmarker}{}%
\end{pgfscope}%
\begin{pgfscope}%
\pgfsys@transformshift{3.094687in}{1.011919in}%
\pgfsys@useobject{currentmarker}{}%
\end{pgfscope}%
\begin{pgfscope}%
\pgfsys@transformshift{3.330112in}{0.959407in}%
\pgfsys@useobject{currentmarker}{}%
\end{pgfscope}%
\begin{pgfscope}%
\pgfsys@transformshift{3.565538in}{0.902650in}%
\pgfsys@useobject{currentmarker}{}%
\end{pgfscope}%
\begin{pgfscope}%
\pgfsys@transformshift{3.800963in}{0.843270in}%
\pgfsys@useobject{currentmarker}{}%
\end{pgfscope}%
\begin{pgfscope}%
\pgfsys@transformshift{4.036389in}{0.802290in}%
\pgfsys@useobject{currentmarker}{}%
\end{pgfscope}%
\end{pgfscope}%
\begin{pgfscope}%
\pgfsetrectcap%
\pgfsetmiterjoin%
\pgfsetlinewidth{0.803000pt}%
\definecolor{currentstroke}{rgb}{0.000000,0.000000,0.000000}%
\pgfsetstrokecolor{currentstroke}%
\pgfsetdash{}{0pt}%
\pgfpathmoveto{\pgfqpoint{0.740433in}{0.566590in}}%
\pgfpathlineto{\pgfqpoint{0.740433in}{2.395314in}}%
\pgfusepath{stroke}%
\end{pgfscope}%
\begin{pgfscope}%
\pgfsetrectcap%
\pgfsetmiterjoin%
\pgfsetlinewidth{0.803000pt}%
\definecolor{currentstroke}{rgb}{0.000000,0.000000,0.000000}%
\pgfsetstrokecolor{currentstroke}%
\pgfsetdash{}{0pt}%
\pgfpathmoveto{\pgfqpoint{4.036389in}{0.566590in}}%
\pgfpathlineto{\pgfqpoint{4.036389in}{2.395314in}}%
\pgfusepath{stroke}%
\end{pgfscope}%
\begin{pgfscope}%
\pgfsetrectcap%
\pgfsetmiterjoin%
\pgfsetlinewidth{0.803000pt}%
\definecolor{currentstroke}{rgb}{0.000000,0.000000,0.000000}%
\pgfsetstrokecolor{currentstroke}%
\pgfsetdash{}{0pt}%
\pgfpathmoveto{\pgfqpoint{0.740433in}{0.566590in}}%
\pgfpathlineto{\pgfqpoint{4.036389in}{0.566590in}}%
\pgfusepath{stroke}%
\end{pgfscope}%
\begin{pgfscope}%
\pgfsetrectcap%
\pgfsetmiterjoin%
\pgfsetlinewidth{0.803000pt}%
\definecolor{currentstroke}{rgb}{0.000000,0.000000,0.000000}%
\pgfsetstrokecolor{currentstroke}%
\pgfsetdash{}{0pt}%
\pgfpathmoveto{\pgfqpoint{0.740433in}{2.395314in}}%
\pgfpathlineto{\pgfqpoint{4.036389in}{2.395314in}}%
\pgfusepath{stroke}%
\end{pgfscope}%
\begin{pgfscope}%
\pgfsetbuttcap%
\pgfsetmiterjoin%
\definecolor{currentfill}{rgb}{1.000000,1.000000,1.000000}%
\pgfsetfillcolor{currentfill}%
\pgfsetfillopacity{0.800000}%
\pgfsetlinewidth{1.003750pt}%
\definecolor{currentstroke}{rgb}{0.800000,0.800000,0.800000}%
\pgfsetstrokecolor{currentstroke}%
\pgfsetstrokeopacity{0.800000}%
\pgfsetdash{}{0pt}%
\pgfpathmoveto{\pgfqpoint{2.849635in}{1.249517in}}%
\pgfpathlineto{\pgfqpoint{3.948889in}{1.249517in}}%
\pgfpathquadraticcurveto{\pgfqpoint{3.973889in}{1.249517in}}{\pgfqpoint{3.973889in}{1.274517in}}%
\pgfpathlineto{\pgfqpoint{3.973889in}{2.307814in}}%
\pgfpathquadraticcurveto{\pgfqpoint{3.973889in}{2.332814in}}{\pgfqpoint{3.948889in}{2.332814in}}%
\pgfpathlineto{\pgfqpoint{2.849635in}{2.332814in}}%
\pgfpathquadraticcurveto{\pgfqpoint{2.824635in}{2.332814in}}{\pgfqpoint{2.824635in}{2.307814in}}%
\pgfpathlineto{\pgfqpoint{2.824635in}{1.274517in}}%
\pgfpathquadraticcurveto{\pgfqpoint{2.824635in}{1.249517in}}{\pgfqpoint{2.849635in}{1.249517in}}%
\pgfpathclose%
\pgfusepath{stroke,fill}%
\end{pgfscope}%
\begin{pgfscope}%
\pgfsetbuttcap%
\pgfsetroundjoin%
\definecolor{currentfill}{rgb}{0.000000,0.000000,0.000000}%
\pgfsetfillcolor{currentfill}%
\pgfsetfillopacity{0.000000}%
\pgfsetlinewidth{1.003750pt}%
\definecolor{currentstroke}{rgb}{0.000000,0.447000,0.741000}%
\pgfsetstrokecolor{currentstroke}%
\pgfsetdash{}{0pt}%
\pgfsys@defobject{currentmarker}{\pgfqpoint{-0.041667in}{-0.041667in}}{\pgfqpoint{0.041667in}{0.041667in}}{%
\pgfpathmoveto{\pgfqpoint{0.000000in}{-0.041667in}}%
\pgfpathcurveto{\pgfqpoint{0.011050in}{-0.041667in}}{\pgfqpoint{0.021649in}{-0.037276in}}{\pgfqpoint{0.029463in}{-0.029463in}}%
\pgfpathcurveto{\pgfqpoint{0.037276in}{-0.021649in}}{\pgfqpoint{0.041667in}{-0.011050in}}{\pgfqpoint{0.041667in}{0.000000in}}%
\pgfpathcurveto{\pgfqpoint{0.041667in}{0.011050in}}{\pgfqpoint{0.037276in}{0.021649in}}{\pgfqpoint{0.029463in}{0.029463in}}%
\pgfpathcurveto{\pgfqpoint{0.021649in}{0.037276in}}{\pgfqpoint{0.011050in}{0.041667in}}{\pgfqpoint{0.000000in}{0.041667in}}%
\pgfpathcurveto{\pgfqpoint{-0.011050in}{0.041667in}}{\pgfqpoint{-0.021649in}{0.037276in}}{\pgfqpoint{-0.029463in}{0.029463in}}%
\pgfpathcurveto{\pgfqpoint{-0.037276in}{0.021649in}}{\pgfqpoint{-0.041667in}{0.011050in}}{\pgfqpoint{-0.041667in}{0.000000in}}%
\pgfpathcurveto{\pgfqpoint{-0.041667in}{-0.011050in}}{\pgfqpoint{-0.037276in}{-0.021649in}}{\pgfqpoint{-0.029463in}{-0.029463in}}%
\pgfpathcurveto{\pgfqpoint{-0.021649in}{-0.037276in}}{\pgfqpoint{-0.011050in}{-0.041667in}}{\pgfqpoint{0.000000in}{-0.041667in}}%
\pgfpathclose%
\pgfusepath{stroke,fill}%
}%
\begin{pgfscope}%
\pgfsys@transformshift{2.999635in}{2.239064in}%
\pgfsys@useobject{currentmarker}{}%
\end{pgfscope}%
\end{pgfscope}%
\begin{pgfscope}%
\definecolor{textcolor}{rgb}{0.000000,0.000000,0.000000}%
\pgfsetstrokecolor{textcolor}%
\pgfsetfillcolor{textcolor}%
\pgftext[x=3.224635in,y=2.195314in,left,base]{\color{textcolor}\rmfamily\fontsize{9.000000}{10.800000}\selectfont \(\displaystyle \nu_1 = \) -353.90}%
\end{pgfscope}%
\begin{pgfscope}%
\pgfsetbuttcap%
\pgfsetroundjoin%
\definecolor{currentfill}{rgb}{0.850000,0.324000,0.098000}%
\pgfsetfillcolor{currentfill}%
\pgfsetlinewidth{1.003750pt}%
\definecolor{currentstroke}{rgb}{0.850000,0.324000,0.098000}%
\pgfsetstrokecolor{currentstroke}%
\pgfsetdash{}{0pt}%
\pgfsys@defobject{currentmarker}{\pgfqpoint{-0.041667in}{-0.041667in}}{\pgfqpoint{0.041667in}{0.041667in}}{%
\pgfpathmoveto{\pgfqpoint{-0.041667in}{0.000000in}}%
\pgfpathlineto{\pgfqpoint{0.041667in}{0.000000in}}%
\pgfpathmoveto{\pgfqpoint{0.000000in}{-0.041667in}}%
\pgfpathlineto{\pgfqpoint{0.000000in}{0.041667in}}%
\pgfusepath{stroke,fill}%
}%
\begin{pgfscope}%
\pgfsys@transformshift{2.999635in}{2.064765in}%
\pgfsys@useobject{currentmarker}{}%
\end{pgfscope}%
\end{pgfscope}%
\begin{pgfscope}%
\definecolor{textcolor}{rgb}{0.000000,0.000000,0.000000}%
\pgfsetstrokecolor{textcolor}%
\pgfsetfillcolor{textcolor}%
\pgftext[x=3.224635in,y=2.021015in,left,base]{\color{textcolor}\rmfamily\fontsize{9.000000}{10.800000}\selectfont \(\displaystyle \nu_2 = \) -352.02}%
\end{pgfscope}%
\begin{pgfscope}%
\pgfsetbuttcap%
\pgfsetmiterjoin%
\definecolor{currentfill}{rgb}{0.000000,0.000000,0.000000}%
\pgfsetfillcolor{currentfill}%
\pgfsetfillopacity{0.000000}%
\pgfsetlinewidth{1.003750pt}%
\definecolor{currentstroke}{rgb}{0.000000,0.500000,0.000000}%
\pgfsetstrokecolor{currentstroke}%
\pgfsetdash{}{0pt}%
\pgfsys@defobject{currentmarker}{\pgfqpoint{-0.041667in}{-0.041667in}}{\pgfqpoint{0.041667in}{0.041667in}}{%
\pgfpathmoveto{\pgfqpoint{-0.041667in}{-0.041667in}}%
\pgfpathlineto{\pgfqpoint{0.041667in}{-0.041667in}}%
\pgfpathlineto{\pgfqpoint{0.041667in}{0.041667in}}%
\pgfpathlineto{\pgfqpoint{-0.041667in}{0.041667in}}%
\pgfpathclose%
\pgfusepath{stroke,fill}%
}%
\begin{pgfscope}%
\pgfsys@transformshift{2.999635in}{1.890465in}%
\pgfsys@useobject{currentmarker}{}%
\end{pgfscope}%
\end{pgfscope}%
\begin{pgfscope}%
\definecolor{textcolor}{rgb}{0.000000,0.000000,0.000000}%
\pgfsetstrokecolor{textcolor}%
\pgfsetfillcolor{textcolor}%
\pgftext[x=3.224635in,y=1.846715in,left,base]{\color{textcolor}\rmfamily\fontsize{9.000000}{10.800000}\selectfont \(\displaystyle \nu_3 = \) -349.42}%
\end{pgfscope}%
\begin{pgfscope}%
\pgfsetbuttcap%
\pgfsetroundjoin%
\definecolor{currentfill}{rgb}{0.494000,0.184000,0.556000}%
\pgfsetfillcolor{currentfill}%
\pgfsetlinewidth{1.003750pt}%
\definecolor{currentstroke}{rgb}{0.494000,0.184000,0.556000}%
\pgfsetstrokecolor{currentstroke}%
\pgfsetdash{}{0pt}%
\pgfsys@defobject{currentmarker}{\pgfqpoint{-0.041667in}{-0.041667in}}{\pgfqpoint{0.041667in}{0.041667in}}{%
\pgfpathmoveto{\pgfqpoint{-0.041667in}{-0.041667in}}%
\pgfpathlineto{\pgfqpoint{0.041667in}{0.041667in}}%
\pgfpathmoveto{\pgfqpoint{-0.041667in}{0.041667in}}%
\pgfpathlineto{\pgfqpoint{0.041667in}{-0.041667in}}%
\pgfusepath{stroke,fill}%
}%
\begin{pgfscope}%
\pgfsys@transformshift{2.999635in}{1.716165in}%
\pgfsys@useobject{currentmarker}{}%
\end{pgfscope}%
\end{pgfscope}%
\begin{pgfscope}%
\definecolor{textcolor}{rgb}{0.000000,0.000000,0.000000}%
\pgfsetstrokecolor{textcolor}%
\pgfsetfillcolor{textcolor}%
\pgftext[x=3.224635in,y=1.672415in,left,base]{\color{textcolor}\rmfamily\fontsize{9.000000}{10.800000}\selectfont \(\displaystyle \nu_4 = \) -348.01}%
\end{pgfscope}%
\begin{pgfscope}%
\pgfsetbuttcap%
\pgfsetmiterjoin%
\definecolor{currentfill}{rgb}{0.000000,0.000000,0.000000}%
\pgfsetfillcolor{currentfill}%
\pgfsetfillopacity{0.000000}%
\pgfsetlinewidth{1.003750pt}%
\definecolor{currentstroke}{rgb}{0.635000,0.078000,0.184000}%
\pgfsetstrokecolor{currentstroke}%
\pgfsetdash{}{0pt}%
\pgfsys@defobject{currentmarker}{\pgfqpoint{-0.035355in}{-0.058926in}}{\pgfqpoint{0.035355in}{0.058926in}}{%
\pgfpathmoveto{\pgfqpoint{-0.000000in}{-0.058926in}}%
\pgfpathlineto{\pgfqpoint{0.035355in}{0.000000in}}%
\pgfpathlineto{\pgfqpoint{0.000000in}{0.058926in}}%
\pgfpathlineto{\pgfqpoint{-0.035355in}{0.000000in}}%
\pgfpathclose%
\pgfusepath{stroke,fill}%
}%
\begin{pgfscope}%
\pgfsys@transformshift{2.999635in}{1.541866in}%
\pgfsys@useobject{currentmarker}{}%
\end{pgfscope}%
\end{pgfscope}%
\begin{pgfscope}%
\definecolor{textcolor}{rgb}{0.000000,0.000000,0.000000}%
\pgfsetstrokecolor{textcolor}%
\pgfsetfillcolor{textcolor}%
\pgftext[x=3.224635in,y=1.498116in,left,base]{\color{textcolor}\rmfamily\fontsize{9.000000}{10.800000}\selectfont \(\displaystyle \nu_5 = \) -347.01}%
\end{pgfscope}%
\begin{pgfscope}%
\pgfsetbuttcap%
\pgfsetmiterjoin%
\definecolor{currentfill}{rgb}{0.000000,0.000000,0.000000}%
\pgfsetfillcolor{currentfill}%
\pgfsetfillopacity{0.000000}%
\pgfsetlinewidth{1.003750pt}%
\definecolor{currentstroke}{rgb}{0.301000,0.745000,0.741000}%
\pgfsetstrokecolor{currentstroke}%
\pgfsetdash{}{0pt}%
\pgfsys@defobject{currentmarker}{\pgfqpoint{-0.041667in}{-0.041667in}}{\pgfqpoint{0.041667in}{0.041667in}}{%
\pgfpathmoveto{\pgfqpoint{0.000000in}{0.041667in}}%
\pgfpathlineto{\pgfqpoint{-0.041667in}{-0.041667in}}%
\pgfpathlineto{\pgfqpoint{0.041667in}{-0.041667in}}%
\pgfpathclose%
\pgfusepath{stroke,fill}%
}%
\begin{pgfscope}%
\pgfsys@transformshift{2.999635in}{1.367566in}%
\pgfsys@useobject{currentmarker}{}%
\end{pgfscope}%
\end{pgfscope}%
\begin{pgfscope}%
\definecolor{textcolor}{rgb}{0.000000,0.000000,0.000000}%
\pgfsetstrokecolor{textcolor}%
\pgfsetfillcolor{textcolor}%
\pgftext[x=3.224635in,y=1.323816in,left,base]{\color{textcolor}\rmfamily\fontsize{9.000000}{10.800000}\selectfont \(\displaystyle \nu_6 = \) -345.00}%
\end{pgfscope}%
\end{pgfpicture}%
\makeatother%
\endgroup%
}
					\caption{Cluster I}
					\label{SubFig:Cluster_I_imag}
				\end{subfigure}
				\begin{subfigure}[h]{0.5\textwidth}
					\centering
					\resizebox{\linewidth}{!}{%% Creator: Matplotlib, PGF backend
%%
%% To include the figure in your LaTeX document, write
%%   \input{<filename>.pgf}
%%
%% Make sure the required packages are loaded in your preamble
%%   \usepackage{pgf}
%%
%% and, on pdftex
%%   \usepackage[utf8]{inputenc}\DeclareUnicodeCharacter{2212}{-}
%%
%% or, on luatex and xetex
%%   \usepackage{unicode-math}
%%
%% Figures using additional raster images can only be included by \input if
%% they are in the same directory as the main LaTeX file. For loading figures
%% from other directories you can use the `import` package
%%   \usepackage{import}
%%
%% and then include the figures with
%%   \import{<path to file>}{<filename>.pgf}
%%
%% Matplotlib used the following preamble
%%   \usepackage[utf8x]{inputenc}
%%   \usepackage[T1]{fontenc}
%%   \usepackage{amsmath,amssymb,amsfonts}
%%
\begingroup%
\makeatletter%
\begin{pgfpicture}%
\pgfpathrectangle{\pgfpointorigin}{\pgfqpoint{4.136389in}{2.495314in}}%
\pgfusepath{use as bounding box, clip}%
\begin{pgfscope}%
\pgfsetbuttcap%
\pgfsetmiterjoin%
\definecolor{currentfill}{rgb}{1.000000,1.000000,1.000000}%
\pgfsetfillcolor{currentfill}%
\pgfsetlinewidth{0.000000pt}%
\definecolor{currentstroke}{rgb}{1.000000,1.000000,1.000000}%
\pgfsetstrokecolor{currentstroke}%
\pgfsetdash{}{0pt}%
\pgfpathmoveto{\pgfqpoint{-0.000000in}{0.000000in}}%
\pgfpathlineto{\pgfqpoint{4.136389in}{0.000000in}}%
\pgfpathlineto{\pgfqpoint{4.136389in}{2.495314in}}%
\pgfpathlineto{\pgfqpoint{-0.000000in}{2.495314in}}%
\pgfpathclose%
\pgfusepath{fill}%
\end{pgfscope}%
\begin{pgfscope}%
\pgfsetbuttcap%
\pgfsetmiterjoin%
\definecolor{currentfill}{rgb}{1.000000,1.000000,1.000000}%
\pgfsetfillcolor{currentfill}%
\pgfsetlinewidth{0.000000pt}%
\definecolor{currentstroke}{rgb}{0.000000,0.000000,0.000000}%
\pgfsetstrokecolor{currentstroke}%
\pgfsetstrokeopacity{0.000000}%
\pgfsetdash{}{0pt}%
\pgfpathmoveto{\pgfqpoint{0.740433in}{0.566590in}}%
\pgfpathlineto{\pgfqpoint{4.036389in}{0.566590in}}%
\pgfpathlineto{\pgfqpoint{4.036389in}{2.395314in}}%
\pgfpathlineto{\pgfqpoint{0.740433in}{2.395314in}}%
\pgfpathclose%
\pgfusepath{fill}%
\end{pgfscope}%
\begin{pgfscope}%
\pgfpathrectangle{\pgfqpoint{0.740433in}{0.566590in}}{\pgfqpoint{3.295956in}{1.828724in}}%
\pgfusepath{clip}%
\pgfsetrectcap%
\pgfsetroundjoin%
\pgfsetlinewidth{0.803000pt}%
\definecolor{currentstroke}{rgb}{0.690196,0.690196,0.690196}%
\pgfsetstrokecolor{currentstroke}%
\pgfsetdash{}{0pt}%
\pgfpathmoveto{\pgfqpoint{0.740433in}{0.566590in}}%
\pgfpathlineto{\pgfqpoint{0.740433in}{2.395314in}}%
\pgfusepath{stroke}%
\end{pgfscope}%
\begin{pgfscope}%
\pgfsetbuttcap%
\pgfsetroundjoin%
\definecolor{currentfill}{rgb}{0.000000,0.000000,0.000000}%
\pgfsetfillcolor{currentfill}%
\pgfsetlinewidth{0.803000pt}%
\definecolor{currentstroke}{rgb}{0.000000,0.000000,0.000000}%
\pgfsetstrokecolor{currentstroke}%
\pgfsetdash{}{0pt}%
\pgfsys@defobject{currentmarker}{\pgfqpoint{0.000000in}{-0.048611in}}{\pgfqpoint{0.000000in}{0.000000in}}{%
\pgfpathmoveto{\pgfqpoint{0.000000in}{0.000000in}}%
\pgfpathlineto{\pgfqpoint{0.000000in}{-0.048611in}}%
\pgfusepath{stroke,fill}%
}%
\begin{pgfscope}%
\pgfsys@transformshift{0.740433in}{0.566590in}%
\pgfsys@useobject{currentmarker}{}%
\end{pgfscope}%
\end{pgfscope}%
\begin{pgfscope}%
\definecolor{textcolor}{rgb}{0.000000,0.000000,0.000000}%
\pgfsetstrokecolor{textcolor}%
\pgfsetfillcolor{textcolor}%
\pgftext[x=0.740433in,y=0.469368in,,top]{\color{textcolor}\rmfamily\fontsize{12.000000}{14.400000}\selectfont \(\displaystyle {-10}\)}%
\end{pgfscope}%
\begin{pgfscope}%
\pgfpathrectangle{\pgfqpoint{0.740433in}{0.566590in}}{\pgfqpoint{3.295956in}{1.828724in}}%
\pgfusepath{clip}%
\pgfsetrectcap%
\pgfsetroundjoin%
\pgfsetlinewidth{0.803000pt}%
\definecolor{currentstroke}{rgb}{0.690196,0.690196,0.690196}%
\pgfsetstrokecolor{currentstroke}%
\pgfsetdash{}{0pt}%
\pgfpathmoveto{\pgfqpoint{1.247503in}{0.566590in}}%
\pgfpathlineto{\pgfqpoint{1.247503in}{2.395314in}}%
\pgfusepath{stroke}%
\end{pgfscope}%
\begin{pgfscope}%
\pgfsetbuttcap%
\pgfsetroundjoin%
\definecolor{currentfill}{rgb}{0.000000,0.000000,0.000000}%
\pgfsetfillcolor{currentfill}%
\pgfsetlinewidth{0.803000pt}%
\definecolor{currentstroke}{rgb}{0.000000,0.000000,0.000000}%
\pgfsetstrokecolor{currentstroke}%
\pgfsetdash{}{0pt}%
\pgfsys@defobject{currentmarker}{\pgfqpoint{0.000000in}{-0.048611in}}{\pgfqpoint{0.000000in}{0.000000in}}{%
\pgfpathmoveto{\pgfqpoint{0.000000in}{0.000000in}}%
\pgfpathlineto{\pgfqpoint{0.000000in}{-0.048611in}}%
\pgfusepath{stroke,fill}%
}%
\begin{pgfscope}%
\pgfsys@transformshift{1.247503in}{0.566590in}%
\pgfsys@useobject{currentmarker}{}%
\end{pgfscope}%
\end{pgfscope}%
\begin{pgfscope}%
\definecolor{textcolor}{rgb}{0.000000,0.000000,0.000000}%
\pgfsetstrokecolor{textcolor}%
\pgfsetfillcolor{textcolor}%
\pgftext[x=1.247503in,y=0.469368in,,top]{\color{textcolor}\rmfamily\fontsize{12.000000}{14.400000}\selectfont \(\displaystyle {0}\)}%
\end{pgfscope}%
\begin{pgfscope}%
\pgfpathrectangle{\pgfqpoint{0.740433in}{0.566590in}}{\pgfqpoint{3.295956in}{1.828724in}}%
\pgfusepath{clip}%
\pgfsetrectcap%
\pgfsetroundjoin%
\pgfsetlinewidth{0.803000pt}%
\definecolor{currentstroke}{rgb}{0.690196,0.690196,0.690196}%
\pgfsetstrokecolor{currentstroke}%
\pgfsetdash{}{0pt}%
\pgfpathmoveto{\pgfqpoint{1.754573in}{0.566590in}}%
\pgfpathlineto{\pgfqpoint{1.754573in}{2.395314in}}%
\pgfusepath{stroke}%
\end{pgfscope}%
\begin{pgfscope}%
\pgfsetbuttcap%
\pgfsetroundjoin%
\definecolor{currentfill}{rgb}{0.000000,0.000000,0.000000}%
\pgfsetfillcolor{currentfill}%
\pgfsetlinewidth{0.803000pt}%
\definecolor{currentstroke}{rgb}{0.000000,0.000000,0.000000}%
\pgfsetstrokecolor{currentstroke}%
\pgfsetdash{}{0pt}%
\pgfsys@defobject{currentmarker}{\pgfqpoint{0.000000in}{-0.048611in}}{\pgfqpoint{0.000000in}{0.000000in}}{%
\pgfpathmoveto{\pgfqpoint{0.000000in}{0.000000in}}%
\pgfpathlineto{\pgfqpoint{0.000000in}{-0.048611in}}%
\pgfusepath{stroke,fill}%
}%
\begin{pgfscope}%
\pgfsys@transformshift{1.754573in}{0.566590in}%
\pgfsys@useobject{currentmarker}{}%
\end{pgfscope}%
\end{pgfscope}%
\begin{pgfscope}%
\definecolor{textcolor}{rgb}{0.000000,0.000000,0.000000}%
\pgfsetstrokecolor{textcolor}%
\pgfsetfillcolor{textcolor}%
\pgftext[x=1.754573in,y=0.469368in,,top]{\color{textcolor}\rmfamily\fontsize{12.000000}{14.400000}\selectfont \(\displaystyle {10}\)}%
\end{pgfscope}%
\begin{pgfscope}%
\pgfpathrectangle{\pgfqpoint{0.740433in}{0.566590in}}{\pgfqpoint{3.295956in}{1.828724in}}%
\pgfusepath{clip}%
\pgfsetrectcap%
\pgfsetroundjoin%
\pgfsetlinewidth{0.803000pt}%
\definecolor{currentstroke}{rgb}{0.690196,0.690196,0.690196}%
\pgfsetstrokecolor{currentstroke}%
\pgfsetdash{}{0pt}%
\pgfpathmoveto{\pgfqpoint{2.261643in}{0.566590in}}%
\pgfpathlineto{\pgfqpoint{2.261643in}{2.395314in}}%
\pgfusepath{stroke}%
\end{pgfscope}%
\begin{pgfscope}%
\pgfsetbuttcap%
\pgfsetroundjoin%
\definecolor{currentfill}{rgb}{0.000000,0.000000,0.000000}%
\pgfsetfillcolor{currentfill}%
\pgfsetlinewidth{0.803000pt}%
\definecolor{currentstroke}{rgb}{0.000000,0.000000,0.000000}%
\pgfsetstrokecolor{currentstroke}%
\pgfsetdash{}{0pt}%
\pgfsys@defobject{currentmarker}{\pgfqpoint{0.000000in}{-0.048611in}}{\pgfqpoint{0.000000in}{0.000000in}}{%
\pgfpathmoveto{\pgfqpoint{0.000000in}{0.000000in}}%
\pgfpathlineto{\pgfqpoint{0.000000in}{-0.048611in}}%
\pgfusepath{stroke,fill}%
}%
\begin{pgfscope}%
\pgfsys@transformshift{2.261643in}{0.566590in}%
\pgfsys@useobject{currentmarker}{}%
\end{pgfscope}%
\end{pgfscope}%
\begin{pgfscope}%
\definecolor{textcolor}{rgb}{0.000000,0.000000,0.000000}%
\pgfsetstrokecolor{textcolor}%
\pgfsetfillcolor{textcolor}%
\pgftext[x=2.261643in,y=0.469368in,,top]{\color{textcolor}\rmfamily\fontsize{12.000000}{14.400000}\selectfont \(\displaystyle {20}\)}%
\end{pgfscope}%
\begin{pgfscope}%
\pgfpathrectangle{\pgfqpoint{0.740433in}{0.566590in}}{\pgfqpoint{3.295956in}{1.828724in}}%
\pgfusepath{clip}%
\pgfsetrectcap%
\pgfsetroundjoin%
\pgfsetlinewidth{0.803000pt}%
\definecolor{currentstroke}{rgb}{0.690196,0.690196,0.690196}%
\pgfsetstrokecolor{currentstroke}%
\pgfsetdash{}{0pt}%
\pgfpathmoveto{\pgfqpoint{2.768713in}{0.566590in}}%
\pgfpathlineto{\pgfqpoint{2.768713in}{2.395314in}}%
\pgfusepath{stroke}%
\end{pgfscope}%
\begin{pgfscope}%
\pgfsetbuttcap%
\pgfsetroundjoin%
\definecolor{currentfill}{rgb}{0.000000,0.000000,0.000000}%
\pgfsetfillcolor{currentfill}%
\pgfsetlinewidth{0.803000pt}%
\definecolor{currentstroke}{rgb}{0.000000,0.000000,0.000000}%
\pgfsetstrokecolor{currentstroke}%
\pgfsetdash{}{0pt}%
\pgfsys@defobject{currentmarker}{\pgfqpoint{0.000000in}{-0.048611in}}{\pgfqpoint{0.000000in}{0.000000in}}{%
\pgfpathmoveto{\pgfqpoint{0.000000in}{0.000000in}}%
\pgfpathlineto{\pgfqpoint{0.000000in}{-0.048611in}}%
\pgfusepath{stroke,fill}%
}%
\begin{pgfscope}%
\pgfsys@transformshift{2.768713in}{0.566590in}%
\pgfsys@useobject{currentmarker}{}%
\end{pgfscope}%
\end{pgfscope}%
\begin{pgfscope}%
\definecolor{textcolor}{rgb}{0.000000,0.000000,0.000000}%
\pgfsetstrokecolor{textcolor}%
\pgfsetfillcolor{textcolor}%
\pgftext[x=2.768713in,y=0.469368in,,top]{\color{textcolor}\rmfamily\fontsize{12.000000}{14.400000}\selectfont \(\displaystyle {30}\)}%
\end{pgfscope}%
\begin{pgfscope}%
\pgfpathrectangle{\pgfqpoint{0.740433in}{0.566590in}}{\pgfqpoint{3.295956in}{1.828724in}}%
\pgfusepath{clip}%
\pgfsetrectcap%
\pgfsetroundjoin%
\pgfsetlinewidth{0.803000pt}%
\definecolor{currentstroke}{rgb}{0.690196,0.690196,0.690196}%
\pgfsetstrokecolor{currentstroke}%
\pgfsetdash{}{0pt}%
\pgfpathmoveto{\pgfqpoint{3.275783in}{0.566590in}}%
\pgfpathlineto{\pgfqpoint{3.275783in}{2.395314in}}%
\pgfusepath{stroke}%
\end{pgfscope}%
\begin{pgfscope}%
\pgfsetbuttcap%
\pgfsetroundjoin%
\definecolor{currentfill}{rgb}{0.000000,0.000000,0.000000}%
\pgfsetfillcolor{currentfill}%
\pgfsetlinewidth{0.803000pt}%
\definecolor{currentstroke}{rgb}{0.000000,0.000000,0.000000}%
\pgfsetstrokecolor{currentstroke}%
\pgfsetdash{}{0pt}%
\pgfsys@defobject{currentmarker}{\pgfqpoint{0.000000in}{-0.048611in}}{\pgfqpoint{0.000000in}{0.000000in}}{%
\pgfpathmoveto{\pgfqpoint{0.000000in}{0.000000in}}%
\pgfpathlineto{\pgfqpoint{0.000000in}{-0.048611in}}%
\pgfusepath{stroke,fill}%
}%
\begin{pgfscope}%
\pgfsys@transformshift{3.275783in}{0.566590in}%
\pgfsys@useobject{currentmarker}{}%
\end{pgfscope}%
\end{pgfscope}%
\begin{pgfscope}%
\definecolor{textcolor}{rgb}{0.000000,0.000000,0.000000}%
\pgfsetstrokecolor{textcolor}%
\pgfsetfillcolor{textcolor}%
\pgftext[x=3.275783in,y=0.469368in,,top]{\color{textcolor}\rmfamily\fontsize{12.000000}{14.400000}\selectfont \(\displaystyle {40}\)}%
\end{pgfscope}%
\begin{pgfscope}%
\pgfpathrectangle{\pgfqpoint{0.740433in}{0.566590in}}{\pgfqpoint{3.295956in}{1.828724in}}%
\pgfusepath{clip}%
\pgfsetrectcap%
\pgfsetroundjoin%
\pgfsetlinewidth{0.803000pt}%
\definecolor{currentstroke}{rgb}{0.690196,0.690196,0.690196}%
\pgfsetstrokecolor{currentstroke}%
\pgfsetdash{}{0pt}%
\pgfpathmoveto{\pgfqpoint{3.782853in}{0.566590in}}%
\pgfpathlineto{\pgfqpoint{3.782853in}{2.395314in}}%
\pgfusepath{stroke}%
\end{pgfscope}%
\begin{pgfscope}%
\pgfsetbuttcap%
\pgfsetroundjoin%
\definecolor{currentfill}{rgb}{0.000000,0.000000,0.000000}%
\pgfsetfillcolor{currentfill}%
\pgfsetlinewidth{0.803000pt}%
\definecolor{currentstroke}{rgb}{0.000000,0.000000,0.000000}%
\pgfsetstrokecolor{currentstroke}%
\pgfsetdash{}{0pt}%
\pgfsys@defobject{currentmarker}{\pgfqpoint{0.000000in}{-0.048611in}}{\pgfqpoint{0.000000in}{0.000000in}}{%
\pgfpathmoveto{\pgfqpoint{0.000000in}{0.000000in}}%
\pgfpathlineto{\pgfqpoint{0.000000in}{-0.048611in}}%
\pgfusepath{stroke,fill}%
}%
\begin{pgfscope}%
\pgfsys@transformshift{3.782853in}{0.566590in}%
\pgfsys@useobject{currentmarker}{}%
\end{pgfscope}%
\end{pgfscope}%
\begin{pgfscope}%
\definecolor{textcolor}{rgb}{0.000000,0.000000,0.000000}%
\pgfsetstrokecolor{textcolor}%
\pgfsetfillcolor{textcolor}%
\pgftext[x=3.782853in,y=0.469368in,,top]{\color{textcolor}\rmfamily\fontsize{12.000000}{14.400000}\selectfont \(\displaystyle {50}\)}%
\end{pgfscope}%
\begin{pgfscope}%
\definecolor{textcolor}{rgb}{0.000000,0.000000,0.000000}%
\pgfsetstrokecolor{textcolor}%
\pgfsetfillcolor{textcolor}%
\pgftext[x=2.388411in,y=0.266626in,,top]{\color{textcolor}\rmfamily\fontsize{12.000000}{14.400000}\selectfont SNR [dB]}%
\end{pgfscope}%
\begin{pgfscope}%
\pgfpathrectangle{\pgfqpoint{0.740433in}{0.566590in}}{\pgfqpoint{3.295956in}{1.828724in}}%
\pgfusepath{clip}%
\pgfsetrectcap%
\pgfsetroundjoin%
\pgfsetlinewidth{0.803000pt}%
\definecolor{currentstroke}{rgb}{0.690196,0.690196,0.690196}%
\pgfsetstrokecolor{currentstroke}%
\pgfsetdash{}{0pt}%
\pgfpathmoveto{\pgfqpoint{0.740433in}{0.752967in}}%
\pgfpathlineto{\pgfqpoint{4.036389in}{0.752967in}}%
\pgfusepath{stroke}%
\end{pgfscope}%
\begin{pgfscope}%
\pgfsetbuttcap%
\pgfsetroundjoin%
\definecolor{currentfill}{rgb}{0.000000,0.000000,0.000000}%
\pgfsetfillcolor{currentfill}%
\pgfsetlinewidth{0.803000pt}%
\definecolor{currentstroke}{rgb}{0.000000,0.000000,0.000000}%
\pgfsetstrokecolor{currentstroke}%
\pgfsetdash{}{0pt}%
\pgfsys@defobject{currentmarker}{\pgfqpoint{-0.048611in}{0.000000in}}{\pgfqpoint{-0.000000in}{0.000000in}}{%
\pgfpathmoveto{\pgfqpoint{-0.000000in}{0.000000in}}%
\pgfpathlineto{\pgfqpoint{-0.048611in}{0.000000in}}%
\pgfusepath{stroke,fill}%
}%
\begin{pgfscope}%
\pgfsys@transformshift{0.740433in}{0.752967in}%
\pgfsys@useobject{currentmarker}{}%
\end{pgfscope}%
\end{pgfscope}%
\begin{pgfscope}%
\definecolor{textcolor}{rgb}{0.000000,0.000000,0.000000}%
\pgfsetstrokecolor{textcolor}%
\pgfsetfillcolor{textcolor}%
\pgftext[x=0.322222in, y=0.695574in, left, base]{\color{textcolor}\rmfamily\fontsize{12.000000}{14.400000}\selectfont \(\displaystyle {10^{-4}}\)}%
\end{pgfscope}%
\begin{pgfscope}%
\pgfpathrectangle{\pgfqpoint{0.740433in}{0.566590in}}{\pgfqpoint{3.295956in}{1.828724in}}%
\pgfusepath{clip}%
\pgfsetrectcap%
\pgfsetroundjoin%
\pgfsetlinewidth{0.803000pt}%
\definecolor{currentstroke}{rgb}{0.690196,0.690196,0.690196}%
\pgfsetstrokecolor{currentstroke}%
\pgfsetdash{}{0pt}%
\pgfpathmoveto{\pgfqpoint{0.740433in}{1.236303in}}%
\pgfpathlineto{\pgfqpoint{4.036389in}{1.236303in}}%
\pgfusepath{stroke}%
\end{pgfscope}%
\begin{pgfscope}%
\pgfsetbuttcap%
\pgfsetroundjoin%
\definecolor{currentfill}{rgb}{0.000000,0.000000,0.000000}%
\pgfsetfillcolor{currentfill}%
\pgfsetlinewidth{0.803000pt}%
\definecolor{currentstroke}{rgb}{0.000000,0.000000,0.000000}%
\pgfsetstrokecolor{currentstroke}%
\pgfsetdash{}{0pt}%
\pgfsys@defobject{currentmarker}{\pgfqpoint{-0.048611in}{0.000000in}}{\pgfqpoint{-0.000000in}{0.000000in}}{%
\pgfpathmoveto{\pgfqpoint{-0.000000in}{0.000000in}}%
\pgfpathlineto{\pgfqpoint{-0.048611in}{0.000000in}}%
\pgfusepath{stroke,fill}%
}%
\begin{pgfscope}%
\pgfsys@transformshift{0.740433in}{1.236303in}%
\pgfsys@useobject{currentmarker}{}%
\end{pgfscope}%
\end{pgfscope}%
\begin{pgfscope}%
\definecolor{textcolor}{rgb}{0.000000,0.000000,0.000000}%
\pgfsetstrokecolor{textcolor}%
\pgfsetfillcolor{textcolor}%
\pgftext[x=0.322222in, y=1.178910in, left, base]{\color{textcolor}\rmfamily\fontsize{12.000000}{14.400000}\selectfont \(\displaystyle {10^{-2}}\)}%
\end{pgfscope}%
\begin{pgfscope}%
\pgfpathrectangle{\pgfqpoint{0.740433in}{0.566590in}}{\pgfqpoint{3.295956in}{1.828724in}}%
\pgfusepath{clip}%
\pgfsetrectcap%
\pgfsetroundjoin%
\pgfsetlinewidth{0.803000pt}%
\definecolor{currentstroke}{rgb}{0.690196,0.690196,0.690196}%
\pgfsetstrokecolor{currentstroke}%
\pgfsetdash{}{0pt}%
\pgfpathmoveto{\pgfqpoint{0.740433in}{1.719639in}}%
\pgfpathlineto{\pgfqpoint{4.036389in}{1.719639in}}%
\pgfusepath{stroke}%
\end{pgfscope}%
\begin{pgfscope}%
\pgfsetbuttcap%
\pgfsetroundjoin%
\definecolor{currentfill}{rgb}{0.000000,0.000000,0.000000}%
\pgfsetfillcolor{currentfill}%
\pgfsetlinewidth{0.803000pt}%
\definecolor{currentstroke}{rgb}{0.000000,0.000000,0.000000}%
\pgfsetstrokecolor{currentstroke}%
\pgfsetdash{}{0pt}%
\pgfsys@defobject{currentmarker}{\pgfqpoint{-0.048611in}{0.000000in}}{\pgfqpoint{-0.000000in}{0.000000in}}{%
\pgfpathmoveto{\pgfqpoint{-0.000000in}{0.000000in}}%
\pgfpathlineto{\pgfqpoint{-0.048611in}{0.000000in}}%
\pgfusepath{stroke,fill}%
}%
\begin{pgfscope}%
\pgfsys@transformshift{0.740433in}{1.719639in}%
\pgfsys@useobject{currentmarker}{}%
\end{pgfscope}%
\end{pgfscope}%
\begin{pgfscope}%
\definecolor{textcolor}{rgb}{0.000000,0.000000,0.000000}%
\pgfsetstrokecolor{textcolor}%
\pgfsetfillcolor{textcolor}%
\pgftext[x=0.414045in, y=1.662246in, left, base]{\color{textcolor}\rmfamily\fontsize{12.000000}{14.400000}\selectfont \(\displaystyle {10^{0}}\)}%
\end{pgfscope}%
\begin{pgfscope}%
\pgfpathrectangle{\pgfqpoint{0.740433in}{0.566590in}}{\pgfqpoint{3.295956in}{1.828724in}}%
\pgfusepath{clip}%
\pgfsetrectcap%
\pgfsetroundjoin%
\pgfsetlinewidth{0.803000pt}%
\definecolor{currentstroke}{rgb}{0.690196,0.690196,0.690196}%
\pgfsetstrokecolor{currentstroke}%
\pgfsetdash{}{0pt}%
\pgfpathmoveto{\pgfqpoint{0.740433in}{2.202975in}}%
\pgfpathlineto{\pgfqpoint{4.036389in}{2.202975in}}%
\pgfusepath{stroke}%
\end{pgfscope}%
\begin{pgfscope}%
\pgfsetbuttcap%
\pgfsetroundjoin%
\definecolor{currentfill}{rgb}{0.000000,0.000000,0.000000}%
\pgfsetfillcolor{currentfill}%
\pgfsetlinewidth{0.803000pt}%
\definecolor{currentstroke}{rgb}{0.000000,0.000000,0.000000}%
\pgfsetstrokecolor{currentstroke}%
\pgfsetdash{}{0pt}%
\pgfsys@defobject{currentmarker}{\pgfqpoint{-0.048611in}{0.000000in}}{\pgfqpoint{-0.000000in}{0.000000in}}{%
\pgfpathmoveto{\pgfqpoint{-0.000000in}{0.000000in}}%
\pgfpathlineto{\pgfqpoint{-0.048611in}{0.000000in}}%
\pgfusepath{stroke,fill}%
}%
\begin{pgfscope}%
\pgfsys@transformshift{0.740433in}{2.202975in}%
\pgfsys@useobject{currentmarker}{}%
\end{pgfscope}%
\end{pgfscope}%
\begin{pgfscope}%
\definecolor{textcolor}{rgb}{0.000000,0.000000,0.000000}%
\pgfsetstrokecolor{textcolor}%
\pgfsetfillcolor{textcolor}%
\pgftext[x=0.414045in, y=2.145582in, left, base]{\color{textcolor}\rmfamily\fontsize{12.000000}{14.400000}\selectfont \(\displaystyle {10^{2}}\)}%
\end{pgfscope}%
\begin{pgfscope}%
\definecolor{textcolor}{rgb}{0.000000,0.000000,0.000000}%
\pgfsetstrokecolor{textcolor}%
\pgfsetfillcolor{textcolor}%
\pgftext[x=0.266667in,y=1.480952in,,bottom,rotate=90.000000]{\color{textcolor}\rmfamily\fontsize{12.000000}{14.400000}\selectfont \(\displaystyle \hat{\sigma}_{\nu}(\mathrm{SNR})\)}%
\end{pgfscope}%
\begin{pgfscope}%
\pgfpathrectangle{\pgfqpoint{0.740433in}{0.566590in}}{\pgfqpoint{3.295956in}{1.828724in}}%
\pgfusepath{clip}%
\pgfsetbuttcap%
\pgfsetroundjoin%
\pgfsetlinewidth{1.505625pt}%
\definecolor{currentstroke}{rgb}{0.000000,0.447000,0.741000}%
\pgfsetstrokecolor{currentstroke}%
\pgfsetdash{{5.550000pt}{2.400000pt}}{0.000000pt}%
\pgfpathmoveto{\pgfqpoint{0.740433in}{2.294665in}}%
\pgfpathlineto{\pgfqpoint{0.837373in}{2.283877in}}%
\pgfpathlineto{\pgfqpoint{0.934312in}{2.283189in}}%
\pgfpathlineto{\pgfqpoint{1.031252in}{2.271965in}}%
\pgfpathlineto{\pgfqpoint{1.128192in}{2.289702in}}%
\pgfpathlineto{\pgfqpoint{1.225132in}{2.270581in}}%
\pgfpathlineto{\pgfqpoint{1.322072in}{2.273031in}}%
\pgfpathlineto{\pgfqpoint{1.419012in}{2.285544in}}%
\pgfpathlineto{\pgfqpoint{1.515952in}{2.289912in}}%
\pgfpathlineto{\pgfqpoint{1.612892in}{2.308587in}}%
\pgfpathlineto{\pgfqpoint{1.709831in}{2.320247in}}%
\pgfpathlineto{\pgfqpoint{1.806771in}{2.323150in}}%
\pgfpathlineto{\pgfqpoint{1.903711in}{2.319686in}}%
\pgfpathlineto{\pgfqpoint{2.000651in}{2.321479in}}%
\pgfpathlineto{\pgfqpoint{2.097591in}{2.306561in}}%
\pgfpathlineto{\pgfqpoint{2.194531in}{2.295095in}}%
\pgfpathlineto{\pgfqpoint{2.291471in}{2.274926in}}%
\pgfpathlineto{\pgfqpoint{2.388411in}{2.218941in}}%
\pgfpathlineto{\pgfqpoint{2.485350in}{2.181813in}}%
\pgfpathlineto{\pgfqpoint{2.582290in}{2.113169in}}%
\pgfpathlineto{\pgfqpoint{2.679230in}{1.436981in}}%
\pgfpathlineto{\pgfqpoint{2.776170in}{1.401565in}}%
\pgfpathlineto{\pgfqpoint{2.873110in}{1.370685in}}%
\pgfpathlineto{\pgfqpoint{2.970050in}{1.360761in}}%
\pgfpathlineto{\pgfqpoint{3.066990in}{1.327026in}}%
\pgfpathlineto{\pgfqpoint{3.163930in}{1.309834in}}%
\pgfpathlineto{\pgfqpoint{3.260870in}{1.287123in}}%
\pgfpathlineto{\pgfqpoint{3.357809in}{1.244986in}}%
\pgfpathlineto{\pgfqpoint{3.454749in}{1.235818in}}%
\pgfpathlineto{\pgfqpoint{3.551689in}{1.218989in}}%
\pgfpathlineto{\pgfqpoint{3.648629in}{1.177403in}}%
\pgfpathlineto{\pgfqpoint{3.745569in}{1.152299in}}%
\pgfpathlineto{\pgfqpoint{3.842509in}{1.136869in}}%
\pgfpathlineto{\pgfqpoint{3.939449in}{1.110353in}}%
\pgfpathlineto{\pgfqpoint{4.036389in}{1.083029in}}%
\pgfusepath{stroke}%
\end{pgfscope}%
\begin{pgfscope}%
\pgfpathrectangle{\pgfqpoint{0.740433in}{0.566590in}}{\pgfqpoint{3.295956in}{1.828724in}}%
\pgfusepath{clip}%
\pgfsetbuttcap%
\pgfsetroundjoin%
\definecolor{currentfill}{rgb}{0.000000,0.000000,0.000000}%
\pgfsetfillcolor{currentfill}%
\pgfsetfillopacity{0.000000}%
\pgfsetlinewidth{1.003750pt}%
\definecolor{currentstroke}{rgb}{0.000000,0.447000,0.741000}%
\pgfsetstrokecolor{currentstroke}%
\pgfsetdash{}{0pt}%
\pgfsys@defobject{currentmarker}{\pgfqpoint{-0.041667in}{-0.041667in}}{\pgfqpoint{0.041667in}{0.041667in}}{%
\pgfpathmoveto{\pgfqpoint{0.000000in}{-0.041667in}}%
\pgfpathcurveto{\pgfqpoint{0.011050in}{-0.041667in}}{\pgfqpoint{0.021649in}{-0.037276in}}{\pgfqpoint{0.029463in}{-0.029463in}}%
\pgfpathcurveto{\pgfqpoint{0.037276in}{-0.021649in}}{\pgfqpoint{0.041667in}{-0.011050in}}{\pgfqpoint{0.041667in}{0.000000in}}%
\pgfpathcurveto{\pgfqpoint{0.041667in}{0.011050in}}{\pgfqpoint{0.037276in}{0.021649in}}{\pgfqpoint{0.029463in}{0.029463in}}%
\pgfpathcurveto{\pgfqpoint{0.021649in}{0.037276in}}{\pgfqpoint{0.011050in}{0.041667in}}{\pgfqpoint{0.000000in}{0.041667in}}%
\pgfpathcurveto{\pgfqpoint{-0.011050in}{0.041667in}}{\pgfqpoint{-0.021649in}{0.037276in}}{\pgfqpoint{-0.029463in}{0.029463in}}%
\pgfpathcurveto{\pgfqpoint{-0.037276in}{0.021649in}}{\pgfqpoint{-0.041667in}{0.011050in}}{\pgfqpoint{-0.041667in}{0.000000in}}%
\pgfpathcurveto{\pgfqpoint{-0.041667in}{-0.011050in}}{\pgfqpoint{-0.037276in}{-0.021649in}}{\pgfqpoint{-0.029463in}{-0.029463in}}%
\pgfpathcurveto{\pgfqpoint{-0.021649in}{-0.037276in}}{\pgfqpoint{-0.011050in}{-0.041667in}}{\pgfqpoint{0.000000in}{-0.041667in}}%
\pgfpathclose%
\pgfusepath{stroke,fill}%
}%
\begin{pgfscope}%
\pgfsys@transformshift{0.740433in}{2.294665in}%
\pgfsys@useobject{currentmarker}{}%
\end{pgfscope}%
\begin{pgfscope}%
\pgfsys@transformshift{1.128192in}{2.289702in}%
\pgfsys@useobject{currentmarker}{}%
\end{pgfscope}%
\begin{pgfscope}%
\pgfsys@transformshift{1.515952in}{2.289912in}%
\pgfsys@useobject{currentmarker}{}%
\end{pgfscope}%
\begin{pgfscope}%
\pgfsys@transformshift{1.903711in}{2.319686in}%
\pgfsys@useobject{currentmarker}{}%
\end{pgfscope}%
\begin{pgfscope}%
\pgfsys@transformshift{2.291471in}{2.274926in}%
\pgfsys@useobject{currentmarker}{}%
\end{pgfscope}%
\begin{pgfscope}%
\pgfsys@transformshift{2.679230in}{1.436981in}%
\pgfsys@useobject{currentmarker}{}%
\end{pgfscope}%
\begin{pgfscope}%
\pgfsys@transformshift{3.066990in}{1.327026in}%
\pgfsys@useobject{currentmarker}{}%
\end{pgfscope}%
\begin{pgfscope}%
\pgfsys@transformshift{3.454749in}{1.235818in}%
\pgfsys@useobject{currentmarker}{}%
\end{pgfscope}%
\begin{pgfscope}%
\pgfsys@transformshift{3.842509in}{1.136869in}%
\pgfsys@useobject{currentmarker}{}%
\end{pgfscope}%
\end{pgfscope}%
\begin{pgfscope}%
\pgfpathrectangle{\pgfqpoint{0.740433in}{0.566590in}}{\pgfqpoint{3.295956in}{1.828724in}}%
\pgfusepath{clip}%
\pgfsetbuttcap%
\pgfsetroundjoin%
\pgfsetlinewidth{1.505625pt}%
\definecolor{currentstroke}{rgb}{0.850000,0.324000,0.098000}%
\pgfsetstrokecolor{currentstroke}%
\pgfsetdash{{5.550000pt}{2.400000pt}}{0.000000pt}%
\pgfpathmoveto{\pgfqpoint{0.740433in}{2.264239in}}%
\pgfpathlineto{\pgfqpoint{0.837373in}{2.250341in}}%
\pgfpathlineto{\pgfqpoint{0.934312in}{2.232349in}}%
\pgfpathlineto{\pgfqpoint{1.031252in}{2.222292in}}%
\pgfpathlineto{\pgfqpoint{1.128192in}{2.248428in}}%
\pgfpathlineto{\pgfqpoint{1.225132in}{2.219278in}}%
\pgfpathlineto{\pgfqpoint{1.322072in}{2.203558in}}%
\pgfpathlineto{\pgfqpoint{1.419012in}{2.233296in}}%
\pgfpathlineto{\pgfqpoint{1.515952in}{2.243466in}}%
\pgfpathlineto{\pgfqpoint{1.612892in}{2.256521in}}%
\pgfpathlineto{\pgfqpoint{1.709831in}{2.274050in}}%
\pgfpathlineto{\pgfqpoint{1.806771in}{2.270730in}}%
\pgfpathlineto{\pgfqpoint{1.903711in}{2.259913in}}%
\pgfpathlineto{\pgfqpoint{2.000651in}{2.263291in}}%
\pgfpathlineto{\pgfqpoint{2.097591in}{2.207408in}}%
\pgfpathlineto{\pgfqpoint{2.194531in}{2.136809in}}%
\pgfpathlineto{\pgfqpoint{2.291471in}{2.032270in}}%
\pgfpathlineto{\pgfqpoint{2.388411in}{1.730705in}}%
\pgfpathlineto{\pgfqpoint{2.485350in}{1.697031in}}%
\pgfpathlineto{\pgfqpoint{2.582290in}{1.635546in}}%
\pgfpathlineto{\pgfqpoint{2.679230in}{1.607109in}}%
\pgfpathlineto{\pgfqpoint{2.776170in}{1.564913in}}%
\pgfpathlineto{\pgfqpoint{2.873110in}{1.535884in}}%
\pgfpathlineto{\pgfqpoint{2.970050in}{1.522917in}}%
\pgfpathlineto{\pgfqpoint{3.066990in}{1.488349in}}%
\pgfpathlineto{\pgfqpoint{3.163930in}{1.472513in}}%
\pgfpathlineto{\pgfqpoint{3.260870in}{1.447087in}}%
\pgfpathlineto{\pgfqpoint{3.357809in}{1.407007in}}%
\pgfpathlineto{\pgfqpoint{3.454749in}{1.397375in}}%
\pgfpathlineto{\pgfqpoint{3.551689in}{1.380693in}}%
\pgfpathlineto{\pgfqpoint{3.648629in}{1.338511in}}%
\pgfpathlineto{\pgfqpoint{3.745569in}{1.313368in}}%
\pgfpathlineto{\pgfqpoint{3.842509in}{1.298469in}}%
\pgfpathlineto{\pgfqpoint{3.939449in}{1.273719in}}%
\pgfpathlineto{\pgfqpoint{4.036389in}{1.247300in}}%
\pgfusepath{stroke}%
\end{pgfscope}%
\begin{pgfscope}%
\pgfpathrectangle{\pgfqpoint{0.740433in}{0.566590in}}{\pgfqpoint{3.295956in}{1.828724in}}%
\pgfusepath{clip}%
\pgfsetbuttcap%
\pgfsetroundjoin%
\definecolor{currentfill}{rgb}{0.850000,0.324000,0.098000}%
\pgfsetfillcolor{currentfill}%
\pgfsetlinewidth{1.003750pt}%
\definecolor{currentstroke}{rgb}{0.850000,0.324000,0.098000}%
\pgfsetstrokecolor{currentstroke}%
\pgfsetdash{}{0pt}%
\pgfsys@defobject{currentmarker}{\pgfqpoint{-0.041667in}{-0.041667in}}{\pgfqpoint{0.041667in}{0.041667in}}{%
\pgfpathmoveto{\pgfqpoint{-0.041667in}{0.000000in}}%
\pgfpathlineto{\pgfqpoint{0.041667in}{0.000000in}}%
\pgfpathmoveto{\pgfqpoint{0.000000in}{-0.041667in}}%
\pgfpathlineto{\pgfqpoint{0.000000in}{0.041667in}}%
\pgfusepath{stroke,fill}%
}%
\begin{pgfscope}%
\pgfsys@transformshift{0.740433in}{2.264239in}%
\pgfsys@useobject{currentmarker}{}%
\end{pgfscope}%
\begin{pgfscope}%
\pgfsys@transformshift{1.031252in}{2.222292in}%
\pgfsys@useobject{currentmarker}{}%
\end{pgfscope}%
\begin{pgfscope}%
\pgfsys@transformshift{1.322072in}{2.203558in}%
\pgfsys@useobject{currentmarker}{}%
\end{pgfscope}%
\begin{pgfscope}%
\pgfsys@transformshift{1.612892in}{2.256521in}%
\pgfsys@useobject{currentmarker}{}%
\end{pgfscope}%
\begin{pgfscope}%
\pgfsys@transformshift{1.903711in}{2.259913in}%
\pgfsys@useobject{currentmarker}{}%
\end{pgfscope}%
\begin{pgfscope}%
\pgfsys@transformshift{2.194531in}{2.136809in}%
\pgfsys@useobject{currentmarker}{}%
\end{pgfscope}%
\begin{pgfscope}%
\pgfsys@transformshift{2.485350in}{1.697031in}%
\pgfsys@useobject{currentmarker}{}%
\end{pgfscope}%
\begin{pgfscope}%
\pgfsys@transformshift{2.776170in}{1.564913in}%
\pgfsys@useobject{currentmarker}{}%
\end{pgfscope}%
\begin{pgfscope}%
\pgfsys@transformshift{3.066990in}{1.488349in}%
\pgfsys@useobject{currentmarker}{}%
\end{pgfscope}%
\begin{pgfscope}%
\pgfsys@transformshift{3.357809in}{1.407007in}%
\pgfsys@useobject{currentmarker}{}%
\end{pgfscope}%
\begin{pgfscope}%
\pgfsys@transformshift{3.648629in}{1.338511in}%
\pgfsys@useobject{currentmarker}{}%
\end{pgfscope}%
\begin{pgfscope}%
\pgfsys@transformshift{3.939449in}{1.273719in}%
\pgfsys@useobject{currentmarker}{}%
\end{pgfscope}%
\end{pgfscope}%
\begin{pgfscope}%
\pgfpathrectangle{\pgfqpoint{0.740433in}{0.566590in}}{\pgfqpoint{3.295956in}{1.828724in}}%
\pgfusepath{clip}%
\pgfsetbuttcap%
\pgfsetroundjoin%
\pgfsetlinewidth{1.505625pt}%
\definecolor{currentstroke}{rgb}{0.000000,0.500000,0.000000}%
\pgfsetstrokecolor{currentstroke}%
\pgfsetdash{{5.550000pt}{2.400000pt}}{0.000000pt}%
\pgfpathmoveto{\pgfqpoint{0.740433in}{2.233545in}}%
\pgfpathlineto{\pgfqpoint{0.837373in}{2.214662in}}%
\pgfpathlineto{\pgfqpoint{0.934312in}{2.192983in}}%
\pgfpathlineto{\pgfqpoint{1.031252in}{2.185786in}}%
\pgfpathlineto{\pgfqpoint{1.128192in}{2.195694in}}%
\pgfpathlineto{\pgfqpoint{1.225132in}{2.168305in}}%
\pgfpathlineto{\pgfqpoint{1.322072in}{2.166404in}}%
\pgfpathlineto{\pgfqpoint{1.419012in}{2.178576in}}%
\pgfpathlineto{\pgfqpoint{1.515952in}{2.166520in}}%
\pgfpathlineto{\pgfqpoint{1.612892in}{2.191595in}}%
\pgfpathlineto{\pgfqpoint{1.709831in}{2.203806in}}%
\pgfpathlineto{\pgfqpoint{1.806771in}{2.201948in}}%
\pgfpathlineto{\pgfqpoint{1.903711in}{2.180876in}}%
\pgfpathlineto{\pgfqpoint{2.000651in}{2.159322in}}%
\pgfpathlineto{\pgfqpoint{2.097591in}{1.952027in}}%
\pgfpathlineto{\pgfqpoint{2.194531in}{1.791559in}}%
\pgfpathlineto{\pgfqpoint{2.291471in}{1.770072in}}%
\pgfpathlineto{\pgfqpoint{2.388411in}{1.737656in}}%
\pgfpathlineto{\pgfqpoint{2.485350in}{1.722173in}}%
\pgfpathlineto{\pgfqpoint{2.582290in}{1.681653in}}%
\pgfpathlineto{\pgfqpoint{2.679230in}{1.649770in}}%
\pgfpathlineto{\pgfqpoint{2.776170in}{1.609870in}}%
\pgfpathlineto{\pgfqpoint{2.873110in}{1.583334in}}%
\pgfpathlineto{\pgfqpoint{2.970050in}{1.575221in}}%
\pgfpathlineto{\pgfqpoint{3.066990in}{1.540772in}}%
\pgfpathlineto{\pgfqpoint{3.163930in}{1.522027in}}%
\pgfpathlineto{\pgfqpoint{3.260870in}{1.490832in}}%
\pgfpathlineto{\pgfqpoint{3.357809in}{1.455187in}}%
\pgfpathlineto{\pgfqpoint{3.454749in}{1.445571in}}%
\pgfpathlineto{\pgfqpoint{3.551689in}{1.425625in}}%
\pgfpathlineto{\pgfqpoint{3.648629in}{1.387210in}}%
\pgfpathlineto{\pgfqpoint{3.745569in}{1.359274in}}%
\pgfpathlineto{\pgfqpoint{3.842509in}{1.345413in}}%
\pgfpathlineto{\pgfqpoint{3.939449in}{1.321320in}}%
\pgfpathlineto{\pgfqpoint{4.036389in}{1.300327in}}%
\pgfusepath{stroke}%
\end{pgfscope}%
\begin{pgfscope}%
\pgfpathrectangle{\pgfqpoint{0.740433in}{0.566590in}}{\pgfqpoint{3.295956in}{1.828724in}}%
\pgfusepath{clip}%
\pgfsetbuttcap%
\pgfsetmiterjoin%
\definecolor{currentfill}{rgb}{0.000000,0.000000,0.000000}%
\pgfsetfillcolor{currentfill}%
\pgfsetfillopacity{0.000000}%
\pgfsetlinewidth{1.003750pt}%
\definecolor{currentstroke}{rgb}{0.000000,0.500000,0.000000}%
\pgfsetstrokecolor{currentstroke}%
\pgfsetdash{}{0pt}%
\pgfsys@defobject{currentmarker}{\pgfqpoint{-0.041667in}{-0.041667in}}{\pgfqpoint{0.041667in}{0.041667in}}{%
\pgfpathmoveto{\pgfqpoint{-0.041667in}{-0.041667in}}%
\pgfpathlineto{\pgfqpoint{0.041667in}{-0.041667in}}%
\pgfpathlineto{\pgfqpoint{0.041667in}{0.041667in}}%
\pgfpathlineto{\pgfqpoint{-0.041667in}{0.041667in}}%
\pgfpathclose%
\pgfusepath{stroke,fill}%
}%
\begin{pgfscope}%
\pgfsys@transformshift{0.740433in}{2.233545in}%
\pgfsys@useobject{currentmarker}{}%
\end{pgfscope}%
\begin{pgfscope}%
\pgfsys@transformshift{1.225132in}{2.168305in}%
\pgfsys@useobject{currentmarker}{}%
\end{pgfscope}%
\begin{pgfscope}%
\pgfsys@transformshift{1.709831in}{2.203806in}%
\pgfsys@useobject{currentmarker}{}%
\end{pgfscope}%
\begin{pgfscope}%
\pgfsys@transformshift{2.194531in}{1.791559in}%
\pgfsys@useobject{currentmarker}{}%
\end{pgfscope}%
\begin{pgfscope}%
\pgfsys@transformshift{2.679230in}{1.649770in}%
\pgfsys@useobject{currentmarker}{}%
\end{pgfscope}%
\begin{pgfscope}%
\pgfsys@transformshift{3.163930in}{1.522027in}%
\pgfsys@useobject{currentmarker}{}%
\end{pgfscope}%
\begin{pgfscope}%
\pgfsys@transformshift{3.648629in}{1.387210in}%
\pgfsys@useobject{currentmarker}{}%
\end{pgfscope}%
\end{pgfscope}%
\begin{pgfscope}%
\pgfpathrectangle{\pgfqpoint{0.740433in}{0.566590in}}{\pgfqpoint{3.295956in}{1.828724in}}%
\pgfusepath{clip}%
\pgfsetbuttcap%
\pgfsetroundjoin%
\pgfsetlinewidth{1.505625pt}%
\definecolor{currentstroke}{rgb}{0.494000,0.184000,0.556000}%
\pgfsetstrokecolor{currentstroke}%
\pgfsetdash{{5.550000pt}{2.400000pt}}{0.000000pt}%
\pgfpathmoveto{\pgfqpoint{0.740433in}{2.240218in}}%
\pgfpathlineto{\pgfqpoint{0.837373in}{2.210934in}}%
\pgfpathlineto{\pgfqpoint{0.934312in}{2.208147in}}%
\pgfpathlineto{\pgfqpoint{1.031252in}{2.204052in}}%
\pgfpathlineto{\pgfqpoint{1.128192in}{2.200733in}}%
\pgfpathlineto{\pgfqpoint{1.225132in}{2.187108in}}%
\pgfpathlineto{\pgfqpoint{1.322072in}{2.195493in}}%
\pgfpathlineto{\pgfqpoint{1.419012in}{2.192093in}}%
\pgfpathlineto{\pgfqpoint{1.515952in}{2.167742in}}%
\pgfpathlineto{\pgfqpoint{1.612892in}{2.157921in}}%
\pgfpathlineto{\pgfqpoint{1.709831in}{2.146649in}}%
\pgfpathlineto{\pgfqpoint{1.806771in}{2.076105in}}%
\pgfpathlineto{\pgfqpoint{1.903711in}{2.107104in}}%
\pgfpathlineto{\pgfqpoint{2.000651in}{1.821611in}}%
\pgfpathlineto{\pgfqpoint{2.097591in}{1.791160in}}%
\pgfpathlineto{\pgfqpoint{2.194531in}{1.776610in}}%
\pgfpathlineto{\pgfqpoint{2.291471in}{1.765264in}}%
\pgfpathlineto{\pgfqpoint{2.388411in}{1.746588in}}%
\pgfpathlineto{\pgfqpoint{2.485350in}{1.712104in}}%
\pgfpathlineto{\pgfqpoint{2.582290in}{1.689325in}}%
\pgfpathlineto{\pgfqpoint{2.679230in}{1.657321in}}%
\pgfpathlineto{\pgfqpoint{2.776170in}{1.633004in}}%
\pgfpathlineto{\pgfqpoint{2.873110in}{1.607789in}}%
\pgfpathlineto{\pgfqpoint{2.970050in}{1.593030in}}%
\pgfpathlineto{\pgfqpoint{3.066990in}{1.554935in}}%
\pgfpathlineto{\pgfqpoint{3.163930in}{1.532945in}}%
\pgfpathlineto{\pgfqpoint{3.260870in}{1.510934in}}%
\pgfpathlineto{\pgfqpoint{3.357809in}{1.476247in}}%
\pgfpathlineto{\pgfqpoint{3.454749in}{1.460680in}}%
\pgfpathlineto{\pgfqpoint{3.551689in}{1.444437in}}%
\pgfpathlineto{\pgfqpoint{3.648629in}{1.404666in}}%
\pgfpathlineto{\pgfqpoint{3.745569in}{1.381949in}}%
\pgfpathlineto{\pgfqpoint{3.842509in}{1.366261in}}%
\pgfpathlineto{\pgfqpoint{3.939449in}{1.343372in}}%
\pgfpathlineto{\pgfqpoint{4.036389in}{1.316799in}}%
\pgfusepath{stroke}%
\end{pgfscope}%
\begin{pgfscope}%
\pgfpathrectangle{\pgfqpoint{0.740433in}{0.566590in}}{\pgfqpoint{3.295956in}{1.828724in}}%
\pgfusepath{clip}%
\pgfsetbuttcap%
\pgfsetroundjoin%
\definecolor{currentfill}{rgb}{0.494000,0.184000,0.556000}%
\pgfsetfillcolor{currentfill}%
\pgfsetlinewidth{1.003750pt}%
\definecolor{currentstroke}{rgb}{0.494000,0.184000,0.556000}%
\pgfsetstrokecolor{currentstroke}%
\pgfsetdash{}{0pt}%
\pgfsys@defobject{currentmarker}{\pgfqpoint{-0.041667in}{-0.041667in}}{\pgfqpoint{0.041667in}{0.041667in}}{%
\pgfpathmoveto{\pgfqpoint{-0.041667in}{-0.041667in}}%
\pgfpathlineto{\pgfqpoint{0.041667in}{0.041667in}}%
\pgfpathmoveto{\pgfqpoint{-0.041667in}{0.041667in}}%
\pgfpathlineto{\pgfqpoint{0.041667in}{-0.041667in}}%
\pgfusepath{stroke,fill}%
}%
\begin{pgfscope}%
\pgfsys@transformshift{0.740433in}{2.240218in}%
\pgfsys@useobject{currentmarker}{}%
\end{pgfscope}%
\begin{pgfscope}%
\pgfsys@transformshift{1.128192in}{2.200733in}%
\pgfsys@useobject{currentmarker}{}%
\end{pgfscope}%
\begin{pgfscope}%
\pgfsys@transformshift{1.515952in}{2.167742in}%
\pgfsys@useobject{currentmarker}{}%
\end{pgfscope}%
\begin{pgfscope}%
\pgfsys@transformshift{1.903711in}{2.107104in}%
\pgfsys@useobject{currentmarker}{}%
\end{pgfscope}%
\begin{pgfscope}%
\pgfsys@transformshift{2.291471in}{1.765264in}%
\pgfsys@useobject{currentmarker}{}%
\end{pgfscope}%
\begin{pgfscope}%
\pgfsys@transformshift{2.679230in}{1.657321in}%
\pgfsys@useobject{currentmarker}{}%
\end{pgfscope}%
\begin{pgfscope}%
\pgfsys@transformshift{3.066990in}{1.554935in}%
\pgfsys@useobject{currentmarker}{}%
\end{pgfscope}%
\begin{pgfscope}%
\pgfsys@transformshift{3.454749in}{1.460680in}%
\pgfsys@useobject{currentmarker}{}%
\end{pgfscope}%
\begin{pgfscope}%
\pgfsys@transformshift{3.842509in}{1.366261in}%
\pgfsys@useobject{currentmarker}{}%
\end{pgfscope}%
\end{pgfscope}%
\begin{pgfscope}%
\pgfpathrectangle{\pgfqpoint{0.740433in}{0.566590in}}{\pgfqpoint{3.295956in}{1.828724in}}%
\pgfusepath{clip}%
\pgfsetbuttcap%
\pgfsetroundjoin%
\pgfsetlinewidth{1.505625pt}%
\definecolor{currentstroke}{rgb}{0.635000,0.078000,0.184000}%
\pgfsetstrokecolor{currentstroke}%
\pgfsetdash{{5.550000pt}{2.400000pt}}{0.000000pt}%
\pgfpathmoveto{\pgfqpoint{0.740433in}{2.259354in}}%
\pgfpathlineto{\pgfqpoint{0.837373in}{2.250136in}}%
\pgfpathlineto{\pgfqpoint{0.934312in}{2.249454in}}%
\pgfpathlineto{\pgfqpoint{1.031252in}{2.248786in}}%
\pgfpathlineto{\pgfqpoint{1.128192in}{2.247052in}}%
\pgfpathlineto{\pgfqpoint{1.225132in}{2.243603in}}%
\pgfpathlineto{\pgfqpoint{1.322072in}{2.243988in}}%
\pgfpathlineto{\pgfqpoint{1.419012in}{2.242081in}}%
\pgfpathlineto{\pgfqpoint{1.515952in}{2.229005in}}%
\pgfpathlineto{\pgfqpoint{1.612892in}{2.206940in}}%
\pgfpathlineto{\pgfqpoint{1.709831in}{2.212440in}}%
\pgfpathlineto{\pgfqpoint{1.806771in}{2.194796in}}%
\pgfpathlineto{\pgfqpoint{1.903711in}{2.207953in}}%
\pgfpathlineto{\pgfqpoint{2.000651in}{2.218449in}}%
\pgfpathlineto{\pgfqpoint{2.097591in}{2.231917in}}%
\pgfpathlineto{\pgfqpoint{2.194531in}{2.249155in}}%
\pgfpathlineto{\pgfqpoint{2.291471in}{2.249281in}}%
\pgfpathlineto{\pgfqpoint{2.388411in}{2.223428in}}%
\pgfpathlineto{\pgfqpoint{2.485350in}{2.170885in}}%
\pgfpathlineto{\pgfqpoint{2.582290in}{2.034630in}}%
\pgfpathlineto{\pgfqpoint{2.679230in}{1.479786in}}%
\pgfpathlineto{\pgfqpoint{2.776170in}{1.468891in}}%
\pgfpathlineto{\pgfqpoint{2.873110in}{1.436297in}}%
\pgfpathlineto{\pgfqpoint{2.970050in}{1.404838in}}%
\pgfpathlineto{\pgfqpoint{3.066990in}{1.379445in}}%
\pgfpathlineto{\pgfqpoint{3.163930in}{1.352842in}}%
\pgfpathlineto{\pgfqpoint{3.260870in}{1.335166in}}%
\pgfpathlineto{\pgfqpoint{3.357809in}{1.299910in}}%
\pgfpathlineto{\pgfqpoint{3.454749in}{1.285060in}}%
\pgfpathlineto{\pgfqpoint{3.551689in}{1.258857in}}%
\pgfpathlineto{\pgfqpoint{3.648629in}{1.228249in}}%
\pgfpathlineto{\pgfqpoint{3.745569in}{1.204449in}}%
\pgfpathlineto{\pgfqpoint{3.842509in}{1.188978in}}%
\pgfpathlineto{\pgfqpoint{3.939449in}{1.161746in}}%
\pgfpathlineto{\pgfqpoint{4.036389in}{1.142095in}}%
\pgfusepath{stroke}%
\end{pgfscope}%
\begin{pgfscope}%
\pgfpathrectangle{\pgfqpoint{0.740433in}{0.566590in}}{\pgfqpoint{3.295956in}{1.828724in}}%
\pgfusepath{clip}%
\pgfsetbuttcap%
\pgfsetmiterjoin%
\definecolor{currentfill}{rgb}{0.000000,0.000000,0.000000}%
\pgfsetfillcolor{currentfill}%
\pgfsetfillopacity{0.000000}%
\pgfsetlinewidth{1.003750pt}%
\definecolor{currentstroke}{rgb}{0.635000,0.078000,0.184000}%
\pgfsetstrokecolor{currentstroke}%
\pgfsetdash{}{0pt}%
\pgfsys@defobject{currentmarker}{\pgfqpoint{-0.035355in}{-0.058926in}}{\pgfqpoint{0.035355in}{0.058926in}}{%
\pgfpathmoveto{\pgfqpoint{-0.000000in}{-0.058926in}}%
\pgfpathlineto{\pgfqpoint{0.035355in}{0.000000in}}%
\pgfpathlineto{\pgfqpoint{0.000000in}{0.058926in}}%
\pgfpathlineto{\pgfqpoint{-0.035355in}{0.000000in}}%
\pgfpathclose%
\pgfusepath{stroke,fill}%
}%
\begin{pgfscope}%
\pgfsys@transformshift{0.740433in}{2.259354in}%
\pgfsys@useobject{currentmarker}{}%
\end{pgfscope}%
\begin{pgfscope}%
\pgfsys@transformshift{1.031252in}{2.248786in}%
\pgfsys@useobject{currentmarker}{}%
\end{pgfscope}%
\begin{pgfscope}%
\pgfsys@transformshift{1.322072in}{2.243988in}%
\pgfsys@useobject{currentmarker}{}%
\end{pgfscope}%
\begin{pgfscope}%
\pgfsys@transformshift{1.612892in}{2.206940in}%
\pgfsys@useobject{currentmarker}{}%
\end{pgfscope}%
\begin{pgfscope}%
\pgfsys@transformshift{1.903711in}{2.207953in}%
\pgfsys@useobject{currentmarker}{}%
\end{pgfscope}%
\begin{pgfscope}%
\pgfsys@transformshift{2.194531in}{2.249155in}%
\pgfsys@useobject{currentmarker}{}%
\end{pgfscope}%
\begin{pgfscope}%
\pgfsys@transformshift{2.485350in}{2.170885in}%
\pgfsys@useobject{currentmarker}{}%
\end{pgfscope}%
\begin{pgfscope}%
\pgfsys@transformshift{2.776170in}{1.468891in}%
\pgfsys@useobject{currentmarker}{}%
\end{pgfscope}%
\begin{pgfscope}%
\pgfsys@transformshift{3.066990in}{1.379445in}%
\pgfsys@useobject{currentmarker}{}%
\end{pgfscope}%
\begin{pgfscope}%
\pgfsys@transformshift{3.357809in}{1.299910in}%
\pgfsys@useobject{currentmarker}{}%
\end{pgfscope}%
\begin{pgfscope}%
\pgfsys@transformshift{3.648629in}{1.228249in}%
\pgfsys@useobject{currentmarker}{}%
\end{pgfscope}%
\begin{pgfscope}%
\pgfsys@transformshift{3.939449in}{1.161746in}%
\pgfsys@useobject{currentmarker}{}%
\end{pgfscope}%
\end{pgfscope}%
\begin{pgfscope}%
\pgfpathrectangle{\pgfqpoint{0.740433in}{0.566590in}}{\pgfqpoint{3.295956in}{1.828724in}}%
\pgfusepath{clip}%
\pgfsetrectcap%
\pgfsetroundjoin%
\pgfsetlinewidth{1.505625pt}%
\definecolor{currentstroke}{rgb}{0.000000,0.447000,0.741000}%
\pgfsetstrokecolor{currentstroke}%
\pgfsetdash{}{0pt}%
\pgfpathmoveto{\pgfqpoint{0.740433in}{2.011447in}}%
\pgfpathlineto{\pgfqpoint{0.975858in}{2.006757in}}%
\pgfpathlineto{\pgfqpoint{1.211283in}{1.979933in}}%
\pgfpathlineto{\pgfqpoint{1.446709in}{1.874783in}}%
\pgfpathlineto{\pgfqpoint{1.682134in}{1.559128in}}%
\pgfpathlineto{\pgfqpoint{1.917560in}{1.546606in}}%
\pgfpathlineto{\pgfqpoint{2.152985in}{1.532840in}}%
\pgfpathlineto{\pgfqpoint{2.388411in}{1.498540in}}%
\pgfpathlineto{\pgfqpoint{2.623836in}{1.466190in}}%
\pgfpathlineto{\pgfqpoint{2.859261in}{1.333223in}}%
\pgfpathlineto{\pgfqpoint{3.094687in}{1.226553in}}%
\pgfpathlineto{\pgfqpoint{3.330112in}{1.130827in}}%
\pgfpathlineto{\pgfqpoint{3.565538in}{1.084586in}}%
\pgfpathlineto{\pgfqpoint{3.800963in}{1.032238in}}%
\pgfpathlineto{\pgfqpoint{4.036389in}{0.967958in}}%
\pgfusepath{stroke}%
\end{pgfscope}%
\begin{pgfscope}%
\pgfpathrectangle{\pgfqpoint{0.740433in}{0.566590in}}{\pgfqpoint{3.295956in}{1.828724in}}%
\pgfusepath{clip}%
\pgfsetbuttcap%
\pgfsetroundjoin%
\definecolor{currentfill}{rgb}{0.000000,0.000000,0.000000}%
\pgfsetfillcolor{currentfill}%
\pgfsetfillopacity{0.000000}%
\pgfsetlinewidth{1.003750pt}%
\definecolor{currentstroke}{rgb}{0.000000,0.447000,0.741000}%
\pgfsetstrokecolor{currentstroke}%
\pgfsetdash{}{0pt}%
\pgfsys@defobject{currentmarker}{\pgfqpoint{-0.041667in}{-0.041667in}}{\pgfqpoint{0.041667in}{0.041667in}}{%
\pgfpathmoveto{\pgfqpoint{0.000000in}{-0.041667in}}%
\pgfpathcurveto{\pgfqpoint{0.011050in}{-0.041667in}}{\pgfqpoint{0.021649in}{-0.037276in}}{\pgfqpoint{0.029463in}{-0.029463in}}%
\pgfpathcurveto{\pgfqpoint{0.037276in}{-0.021649in}}{\pgfqpoint{0.041667in}{-0.011050in}}{\pgfqpoint{0.041667in}{0.000000in}}%
\pgfpathcurveto{\pgfqpoint{0.041667in}{0.011050in}}{\pgfqpoint{0.037276in}{0.021649in}}{\pgfqpoint{0.029463in}{0.029463in}}%
\pgfpathcurveto{\pgfqpoint{0.021649in}{0.037276in}}{\pgfqpoint{0.011050in}{0.041667in}}{\pgfqpoint{0.000000in}{0.041667in}}%
\pgfpathcurveto{\pgfqpoint{-0.011050in}{0.041667in}}{\pgfqpoint{-0.021649in}{0.037276in}}{\pgfqpoint{-0.029463in}{0.029463in}}%
\pgfpathcurveto{\pgfqpoint{-0.037276in}{0.021649in}}{\pgfqpoint{-0.041667in}{0.011050in}}{\pgfqpoint{-0.041667in}{0.000000in}}%
\pgfpathcurveto{\pgfqpoint{-0.041667in}{-0.011050in}}{\pgfqpoint{-0.037276in}{-0.021649in}}{\pgfqpoint{-0.029463in}{-0.029463in}}%
\pgfpathcurveto{\pgfqpoint{-0.021649in}{-0.037276in}}{\pgfqpoint{-0.011050in}{-0.041667in}}{\pgfqpoint{0.000000in}{-0.041667in}}%
\pgfpathclose%
\pgfusepath{stroke,fill}%
}%
\begin{pgfscope}%
\pgfsys@transformshift{0.740433in}{2.011447in}%
\pgfsys@useobject{currentmarker}{}%
\end{pgfscope}%
\begin{pgfscope}%
\pgfsys@transformshift{0.975858in}{2.006757in}%
\pgfsys@useobject{currentmarker}{}%
\end{pgfscope}%
\begin{pgfscope}%
\pgfsys@transformshift{1.211283in}{1.979933in}%
\pgfsys@useobject{currentmarker}{}%
\end{pgfscope}%
\begin{pgfscope}%
\pgfsys@transformshift{1.446709in}{1.874783in}%
\pgfsys@useobject{currentmarker}{}%
\end{pgfscope}%
\begin{pgfscope}%
\pgfsys@transformshift{1.682134in}{1.559128in}%
\pgfsys@useobject{currentmarker}{}%
\end{pgfscope}%
\begin{pgfscope}%
\pgfsys@transformshift{1.917560in}{1.546606in}%
\pgfsys@useobject{currentmarker}{}%
\end{pgfscope}%
\begin{pgfscope}%
\pgfsys@transformshift{2.152985in}{1.532840in}%
\pgfsys@useobject{currentmarker}{}%
\end{pgfscope}%
\begin{pgfscope}%
\pgfsys@transformshift{2.388411in}{1.498540in}%
\pgfsys@useobject{currentmarker}{}%
\end{pgfscope}%
\begin{pgfscope}%
\pgfsys@transformshift{2.623836in}{1.466190in}%
\pgfsys@useobject{currentmarker}{}%
\end{pgfscope}%
\begin{pgfscope}%
\pgfsys@transformshift{2.859261in}{1.333223in}%
\pgfsys@useobject{currentmarker}{}%
\end{pgfscope}%
\begin{pgfscope}%
\pgfsys@transformshift{3.094687in}{1.226553in}%
\pgfsys@useobject{currentmarker}{}%
\end{pgfscope}%
\begin{pgfscope}%
\pgfsys@transformshift{3.330112in}{1.130827in}%
\pgfsys@useobject{currentmarker}{}%
\end{pgfscope}%
\begin{pgfscope}%
\pgfsys@transformshift{3.565538in}{1.084586in}%
\pgfsys@useobject{currentmarker}{}%
\end{pgfscope}%
\begin{pgfscope}%
\pgfsys@transformshift{3.800963in}{1.032238in}%
\pgfsys@useobject{currentmarker}{}%
\end{pgfscope}%
\begin{pgfscope}%
\pgfsys@transformshift{4.036389in}{0.967958in}%
\pgfsys@useobject{currentmarker}{}%
\end{pgfscope}%
\end{pgfscope}%
\begin{pgfscope}%
\pgfpathrectangle{\pgfqpoint{0.740433in}{0.566590in}}{\pgfqpoint{3.295956in}{1.828724in}}%
\pgfusepath{clip}%
\pgfsetrectcap%
\pgfsetroundjoin%
\pgfsetlinewidth{1.505625pt}%
\definecolor{currentstroke}{rgb}{0.850000,0.324000,0.098000}%
\pgfsetstrokecolor{currentstroke}%
\pgfsetdash{}{0pt}%
\pgfpathmoveto{\pgfqpoint{0.740433in}{1.919337in}}%
\pgfpathlineto{\pgfqpoint{0.975858in}{1.893555in}}%
\pgfpathlineto{\pgfqpoint{1.211283in}{1.843406in}}%
\pgfpathlineto{\pgfqpoint{1.446709in}{1.655577in}}%
\pgfpathlineto{\pgfqpoint{1.682134in}{1.660482in}}%
\pgfpathlineto{\pgfqpoint{1.917560in}{1.645187in}}%
\pgfpathlineto{\pgfqpoint{2.152985in}{1.628165in}}%
\pgfpathlineto{\pgfqpoint{2.388411in}{1.588352in}}%
\pgfpathlineto{\pgfqpoint{2.623836in}{1.509273in}}%
\pgfpathlineto{\pgfqpoint{2.859261in}{0.965051in}}%
\pgfpathlineto{\pgfqpoint{3.094687in}{0.909586in}}%
\pgfpathlineto{\pgfqpoint{3.330112in}{0.834838in}}%
\pgfpathlineto{\pgfqpoint{3.565538in}{0.784207in}}%
\pgfpathlineto{\pgfqpoint{3.800963in}{0.731538in}}%
\pgfpathlineto{\pgfqpoint{4.036389in}{0.669406in}}%
\pgfusepath{stroke}%
\end{pgfscope}%
\begin{pgfscope}%
\pgfpathrectangle{\pgfqpoint{0.740433in}{0.566590in}}{\pgfqpoint{3.295956in}{1.828724in}}%
\pgfusepath{clip}%
\pgfsetbuttcap%
\pgfsetroundjoin%
\definecolor{currentfill}{rgb}{0.850000,0.324000,0.098000}%
\pgfsetfillcolor{currentfill}%
\pgfsetlinewidth{1.003750pt}%
\definecolor{currentstroke}{rgb}{0.850000,0.324000,0.098000}%
\pgfsetstrokecolor{currentstroke}%
\pgfsetdash{}{0pt}%
\pgfsys@defobject{currentmarker}{\pgfqpoint{-0.041667in}{-0.041667in}}{\pgfqpoint{0.041667in}{0.041667in}}{%
\pgfpathmoveto{\pgfqpoint{-0.041667in}{0.000000in}}%
\pgfpathlineto{\pgfqpoint{0.041667in}{0.000000in}}%
\pgfpathmoveto{\pgfqpoint{0.000000in}{-0.041667in}}%
\pgfpathlineto{\pgfqpoint{0.000000in}{0.041667in}}%
\pgfusepath{stroke,fill}%
}%
\begin{pgfscope}%
\pgfsys@transformshift{0.740433in}{1.919337in}%
\pgfsys@useobject{currentmarker}{}%
\end{pgfscope}%
\begin{pgfscope}%
\pgfsys@transformshift{0.975858in}{1.893555in}%
\pgfsys@useobject{currentmarker}{}%
\end{pgfscope}%
\begin{pgfscope}%
\pgfsys@transformshift{1.211283in}{1.843406in}%
\pgfsys@useobject{currentmarker}{}%
\end{pgfscope}%
\begin{pgfscope}%
\pgfsys@transformshift{1.446709in}{1.655577in}%
\pgfsys@useobject{currentmarker}{}%
\end{pgfscope}%
\begin{pgfscope}%
\pgfsys@transformshift{1.682134in}{1.660482in}%
\pgfsys@useobject{currentmarker}{}%
\end{pgfscope}%
\begin{pgfscope}%
\pgfsys@transformshift{1.917560in}{1.645187in}%
\pgfsys@useobject{currentmarker}{}%
\end{pgfscope}%
\begin{pgfscope}%
\pgfsys@transformshift{2.152985in}{1.628165in}%
\pgfsys@useobject{currentmarker}{}%
\end{pgfscope}%
\begin{pgfscope}%
\pgfsys@transformshift{2.388411in}{1.588352in}%
\pgfsys@useobject{currentmarker}{}%
\end{pgfscope}%
\begin{pgfscope}%
\pgfsys@transformshift{2.623836in}{1.509273in}%
\pgfsys@useobject{currentmarker}{}%
\end{pgfscope}%
\begin{pgfscope}%
\pgfsys@transformshift{2.859261in}{0.965051in}%
\pgfsys@useobject{currentmarker}{}%
\end{pgfscope}%
\begin{pgfscope}%
\pgfsys@transformshift{3.094687in}{0.909586in}%
\pgfsys@useobject{currentmarker}{}%
\end{pgfscope}%
\begin{pgfscope}%
\pgfsys@transformshift{3.330112in}{0.834838in}%
\pgfsys@useobject{currentmarker}{}%
\end{pgfscope}%
\begin{pgfscope}%
\pgfsys@transformshift{3.565538in}{0.784207in}%
\pgfsys@useobject{currentmarker}{}%
\end{pgfscope}%
\begin{pgfscope}%
\pgfsys@transformshift{3.800963in}{0.731538in}%
\pgfsys@useobject{currentmarker}{}%
\end{pgfscope}%
\begin{pgfscope}%
\pgfsys@transformshift{4.036389in}{0.669406in}%
\pgfsys@useobject{currentmarker}{}%
\end{pgfscope}%
\end{pgfscope}%
\begin{pgfscope}%
\pgfpathrectangle{\pgfqpoint{0.740433in}{0.566590in}}{\pgfqpoint{3.295956in}{1.828724in}}%
\pgfusepath{clip}%
\pgfsetrectcap%
\pgfsetroundjoin%
\pgfsetlinewidth{1.505625pt}%
\definecolor{currentstroke}{rgb}{0.000000,0.500000,0.000000}%
\pgfsetstrokecolor{currentstroke}%
\pgfsetdash{}{0pt}%
\pgfpathmoveto{\pgfqpoint{0.740433in}{1.833887in}}%
\pgfpathlineto{\pgfqpoint{0.975858in}{1.712899in}}%
\pgfpathlineto{\pgfqpoint{1.211283in}{1.628231in}}%
\pgfpathlineto{\pgfqpoint{1.446709in}{1.588810in}}%
\pgfpathlineto{\pgfqpoint{1.682134in}{1.601266in}}%
\pgfpathlineto{\pgfqpoint{1.917560in}{1.584926in}}%
\pgfpathlineto{\pgfqpoint{2.152985in}{1.562168in}}%
\pgfpathlineto{\pgfqpoint{2.388411in}{1.502607in}}%
\pgfpathlineto{\pgfqpoint{2.623836in}{1.378256in}}%
\pgfpathlineto{\pgfqpoint{2.859261in}{0.954472in}}%
\pgfpathlineto{\pgfqpoint{3.094687in}{0.884161in}}%
\pgfpathlineto{\pgfqpoint{3.330112in}{0.819656in}}%
\pgfpathlineto{\pgfqpoint{3.565538in}{0.763090in}}%
\pgfpathlineto{\pgfqpoint{3.800963in}{0.712019in}}%
\pgfpathlineto{\pgfqpoint{4.036389in}{0.654841in}}%
\pgfusepath{stroke}%
\end{pgfscope}%
\begin{pgfscope}%
\pgfpathrectangle{\pgfqpoint{0.740433in}{0.566590in}}{\pgfqpoint{3.295956in}{1.828724in}}%
\pgfusepath{clip}%
\pgfsetbuttcap%
\pgfsetmiterjoin%
\definecolor{currentfill}{rgb}{0.000000,0.000000,0.000000}%
\pgfsetfillcolor{currentfill}%
\pgfsetfillopacity{0.000000}%
\pgfsetlinewidth{1.003750pt}%
\definecolor{currentstroke}{rgb}{0.000000,0.500000,0.000000}%
\pgfsetstrokecolor{currentstroke}%
\pgfsetdash{}{0pt}%
\pgfsys@defobject{currentmarker}{\pgfqpoint{-0.041667in}{-0.041667in}}{\pgfqpoint{0.041667in}{0.041667in}}{%
\pgfpathmoveto{\pgfqpoint{-0.041667in}{-0.041667in}}%
\pgfpathlineto{\pgfqpoint{0.041667in}{-0.041667in}}%
\pgfpathlineto{\pgfqpoint{0.041667in}{0.041667in}}%
\pgfpathlineto{\pgfqpoint{-0.041667in}{0.041667in}}%
\pgfpathclose%
\pgfusepath{stroke,fill}%
}%
\begin{pgfscope}%
\pgfsys@transformshift{0.740433in}{1.833887in}%
\pgfsys@useobject{currentmarker}{}%
\end{pgfscope}%
\begin{pgfscope}%
\pgfsys@transformshift{0.975858in}{1.712899in}%
\pgfsys@useobject{currentmarker}{}%
\end{pgfscope}%
\begin{pgfscope}%
\pgfsys@transformshift{1.211283in}{1.628231in}%
\pgfsys@useobject{currentmarker}{}%
\end{pgfscope}%
\begin{pgfscope}%
\pgfsys@transformshift{1.446709in}{1.588810in}%
\pgfsys@useobject{currentmarker}{}%
\end{pgfscope}%
\begin{pgfscope}%
\pgfsys@transformshift{1.682134in}{1.601266in}%
\pgfsys@useobject{currentmarker}{}%
\end{pgfscope}%
\begin{pgfscope}%
\pgfsys@transformshift{1.917560in}{1.584926in}%
\pgfsys@useobject{currentmarker}{}%
\end{pgfscope}%
\begin{pgfscope}%
\pgfsys@transformshift{2.152985in}{1.562168in}%
\pgfsys@useobject{currentmarker}{}%
\end{pgfscope}%
\begin{pgfscope}%
\pgfsys@transformshift{2.388411in}{1.502607in}%
\pgfsys@useobject{currentmarker}{}%
\end{pgfscope}%
\begin{pgfscope}%
\pgfsys@transformshift{2.623836in}{1.378256in}%
\pgfsys@useobject{currentmarker}{}%
\end{pgfscope}%
\begin{pgfscope}%
\pgfsys@transformshift{2.859261in}{0.954472in}%
\pgfsys@useobject{currentmarker}{}%
\end{pgfscope}%
\begin{pgfscope}%
\pgfsys@transformshift{3.094687in}{0.884161in}%
\pgfsys@useobject{currentmarker}{}%
\end{pgfscope}%
\begin{pgfscope}%
\pgfsys@transformshift{3.330112in}{0.819656in}%
\pgfsys@useobject{currentmarker}{}%
\end{pgfscope}%
\begin{pgfscope}%
\pgfsys@transformshift{3.565538in}{0.763090in}%
\pgfsys@useobject{currentmarker}{}%
\end{pgfscope}%
\begin{pgfscope}%
\pgfsys@transformshift{3.800963in}{0.712019in}%
\pgfsys@useobject{currentmarker}{}%
\end{pgfscope}%
\begin{pgfscope}%
\pgfsys@transformshift{4.036389in}{0.654841in}%
\pgfsys@useobject{currentmarker}{}%
\end{pgfscope}%
\end{pgfscope}%
\begin{pgfscope}%
\pgfpathrectangle{\pgfqpoint{0.740433in}{0.566590in}}{\pgfqpoint{3.295956in}{1.828724in}}%
\pgfusepath{clip}%
\pgfsetrectcap%
\pgfsetroundjoin%
\pgfsetlinewidth{1.505625pt}%
\definecolor{currentstroke}{rgb}{0.494000,0.184000,0.556000}%
\pgfsetstrokecolor{currentstroke}%
\pgfsetdash{}{0pt}%
\pgfpathmoveto{\pgfqpoint{0.740433in}{1.939516in}}%
\pgfpathlineto{\pgfqpoint{0.975858in}{1.897314in}}%
\pgfpathlineto{\pgfqpoint{1.211283in}{1.846289in}}%
\pgfpathlineto{\pgfqpoint{1.446709in}{1.556876in}}%
\pgfpathlineto{\pgfqpoint{1.682134in}{1.509752in}}%
\pgfpathlineto{\pgfqpoint{1.917560in}{1.490814in}}%
\pgfpathlineto{\pgfqpoint{2.152985in}{1.449920in}}%
\pgfpathlineto{\pgfqpoint{2.388411in}{1.369341in}}%
\pgfpathlineto{\pgfqpoint{2.623836in}{1.315208in}}%
\pgfpathlineto{\pgfqpoint{2.859261in}{1.216205in}}%
\pgfpathlineto{\pgfqpoint{3.094687in}{1.151923in}}%
\pgfpathlineto{\pgfqpoint{3.330112in}{1.073563in}}%
\pgfpathlineto{\pgfqpoint{3.565538in}{1.023536in}}%
\pgfpathlineto{\pgfqpoint{3.800963in}{0.964907in}}%
\pgfpathlineto{\pgfqpoint{4.036389in}{0.914817in}}%
\pgfusepath{stroke}%
\end{pgfscope}%
\begin{pgfscope}%
\pgfpathrectangle{\pgfqpoint{0.740433in}{0.566590in}}{\pgfqpoint{3.295956in}{1.828724in}}%
\pgfusepath{clip}%
\pgfsetbuttcap%
\pgfsetroundjoin%
\definecolor{currentfill}{rgb}{0.494000,0.184000,0.556000}%
\pgfsetfillcolor{currentfill}%
\pgfsetlinewidth{1.003750pt}%
\definecolor{currentstroke}{rgb}{0.494000,0.184000,0.556000}%
\pgfsetstrokecolor{currentstroke}%
\pgfsetdash{}{0pt}%
\pgfsys@defobject{currentmarker}{\pgfqpoint{-0.041667in}{-0.041667in}}{\pgfqpoint{0.041667in}{0.041667in}}{%
\pgfpathmoveto{\pgfqpoint{-0.041667in}{-0.041667in}}%
\pgfpathlineto{\pgfqpoint{0.041667in}{0.041667in}}%
\pgfpathmoveto{\pgfqpoint{-0.041667in}{0.041667in}}%
\pgfpathlineto{\pgfqpoint{0.041667in}{-0.041667in}}%
\pgfusepath{stroke,fill}%
}%
\begin{pgfscope}%
\pgfsys@transformshift{0.740433in}{1.939516in}%
\pgfsys@useobject{currentmarker}{}%
\end{pgfscope}%
\begin{pgfscope}%
\pgfsys@transformshift{0.975858in}{1.897314in}%
\pgfsys@useobject{currentmarker}{}%
\end{pgfscope}%
\begin{pgfscope}%
\pgfsys@transformshift{1.211283in}{1.846289in}%
\pgfsys@useobject{currentmarker}{}%
\end{pgfscope}%
\begin{pgfscope}%
\pgfsys@transformshift{1.446709in}{1.556876in}%
\pgfsys@useobject{currentmarker}{}%
\end{pgfscope}%
\begin{pgfscope}%
\pgfsys@transformshift{1.682134in}{1.509752in}%
\pgfsys@useobject{currentmarker}{}%
\end{pgfscope}%
\begin{pgfscope}%
\pgfsys@transformshift{1.917560in}{1.490814in}%
\pgfsys@useobject{currentmarker}{}%
\end{pgfscope}%
\begin{pgfscope}%
\pgfsys@transformshift{2.152985in}{1.449920in}%
\pgfsys@useobject{currentmarker}{}%
\end{pgfscope}%
\begin{pgfscope}%
\pgfsys@transformshift{2.388411in}{1.369341in}%
\pgfsys@useobject{currentmarker}{}%
\end{pgfscope}%
\begin{pgfscope}%
\pgfsys@transformshift{2.623836in}{1.315208in}%
\pgfsys@useobject{currentmarker}{}%
\end{pgfscope}%
\begin{pgfscope}%
\pgfsys@transformshift{2.859261in}{1.216205in}%
\pgfsys@useobject{currentmarker}{}%
\end{pgfscope}%
\begin{pgfscope}%
\pgfsys@transformshift{3.094687in}{1.151923in}%
\pgfsys@useobject{currentmarker}{}%
\end{pgfscope}%
\begin{pgfscope}%
\pgfsys@transformshift{3.330112in}{1.073563in}%
\pgfsys@useobject{currentmarker}{}%
\end{pgfscope}%
\begin{pgfscope}%
\pgfsys@transformshift{3.565538in}{1.023536in}%
\pgfsys@useobject{currentmarker}{}%
\end{pgfscope}%
\begin{pgfscope}%
\pgfsys@transformshift{3.800963in}{0.964907in}%
\pgfsys@useobject{currentmarker}{}%
\end{pgfscope}%
\begin{pgfscope}%
\pgfsys@transformshift{4.036389in}{0.914817in}%
\pgfsys@useobject{currentmarker}{}%
\end{pgfscope}%
\end{pgfscope}%
\begin{pgfscope}%
\pgfpathrectangle{\pgfqpoint{0.740433in}{0.566590in}}{\pgfqpoint{3.295956in}{1.828724in}}%
\pgfusepath{clip}%
\pgfsetrectcap%
\pgfsetroundjoin%
\pgfsetlinewidth{1.505625pt}%
\definecolor{currentstroke}{rgb}{0.635000,0.078000,0.184000}%
\pgfsetstrokecolor{currentstroke}%
\pgfsetdash{}{0pt}%
\pgfpathmoveto{\pgfqpoint{0.740433in}{2.017948in}}%
\pgfpathlineto{\pgfqpoint{0.975858in}{2.004515in}}%
\pgfpathlineto{\pgfqpoint{1.211283in}{1.984125in}}%
\pgfpathlineto{\pgfqpoint{1.446709in}{1.871636in}}%
\pgfpathlineto{\pgfqpoint{1.682134in}{1.619953in}}%
\pgfpathlineto{\pgfqpoint{1.917560in}{1.594156in}}%
\pgfpathlineto{\pgfqpoint{2.152985in}{1.549136in}}%
\pgfpathlineto{\pgfqpoint{2.388411in}{1.457868in}}%
\pgfpathlineto{\pgfqpoint{2.623836in}{1.344327in}}%
\pgfpathlineto{\pgfqpoint{2.859261in}{1.165434in}}%
\pgfpathlineto{\pgfqpoint{3.094687in}{1.087337in}}%
\pgfpathlineto{\pgfqpoint{3.330112in}{1.003302in}}%
\pgfpathlineto{\pgfqpoint{3.565538in}{0.957674in}}%
\pgfpathlineto{\pgfqpoint{3.800963in}{0.895644in}}%
\pgfpathlineto{\pgfqpoint{4.036389in}{0.848737in}}%
\pgfusepath{stroke}%
\end{pgfscope}%
\begin{pgfscope}%
\pgfpathrectangle{\pgfqpoint{0.740433in}{0.566590in}}{\pgfqpoint{3.295956in}{1.828724in}}%
\pgfusepath{clip}%
\pgfsetbuttcap%
\pgfsetmiterjoin%
\definecolor{currentfill}{rgb}{0.000000,0.000000,0.000000}%
\pgfsetfillcolor{currentfill}%
\pgfsetfillopacity{0.000000}%
\pgfsetlinewidth{1.003750pt}%
\definecolor{currentstroke}{rgb}{0.635000,0.078000,0.184000}%
\pgfsetstrokecolor{currentstroke}%
\pgfsetdash{}{0pt}%
\pgfsys@defobject{currentmarker}{\pgfqpoint{-0.035355in}{-0.058926in}}{\pgfqpoint{0.035355in}{0.058926in}}{%
\pgfpathmoveto{\pgfqpoint{-0.000000in}{-0.058926in}}%
\pgfpathlineto{\pgfqpoint{0.035355in}{0.000000in}}%
\pgfpathlineto{\pgfqpoint{0.000000in}{0.058926in}}%
\pgfpathlineto{\pgfqpoint{-0.035355in}{0.000000in}}%
\pgfpathclose%
\pgfusepath{stroke,fill}%
}%
\begin{pgfscope}%
\pgfsys@transformshift{0.740433in}{2.017948in}%
\pgfsys@useobject{currentmarker}{}%
\end{pgfscope}%
\begin{pgfscope}%
\pgfsys@transformshift{0.975858in}{2.004515in}%
\pgfsys@useobject{currentmarker}{}%
\end{pgfscope}%
\begin{pgfscope}%
\pgfsys@transformshift{1.211283in}{1.984125in}%
\pgfsys@useobject{currentmarker}{}%
\end{pgfscope}%
\begin{pgfscope}%
\pgfsys@transformshift{1.446709in}{1.871636in}%
\pgfsys@useobject{currentmarker}{}%
\end{pgfscope}%
\begin{pgfscope}%
\pgfsys@transformshift{1.682134in}{1.619953in}%
\pgfsys@useobject{currentmarker}{}%
\end{pgfscope}%
\begin{pgfscope}%
\pgfsys@transformshift{1.917560in}{1.594156in}%
\pgfsys@useobject{currentmarker}{}%
\end{pgfscope}%
\begin{pgfscope}%
\pgfsys@transformshift{2.152985in}{1.549136in}%
\pgfsys@useobject{currentmarker}{}%
\end{pgfscope}%
\begin{pgfscope}%
\pgfsys@transformshift{2.388411in}{1.457868in}%
\pgfsys@useobject{currentmarker}{}%
\end{pgfscope}%
\begin{pgfscope}%
\pgfsys@transformshift{2.623836in}{1.344327in}%
\pgfsys@useobject{currentmarker}{}%
\end{pgfscope}%
\begin{pgfscope}%
\pgfsys@transformshift{2.859261in}{1.165434in}%
\pgfsys@useobject{currentmarker}{}%
\end{pgfscope}%
\begin{pgfscope}%
\pgfsys@transformshift{3.094687in}{1.087337in}%
\pgfsys@useobject{currentmarker}{}%
\end{pgfscope}%
\begin{pgfscope}%
\pgfsys@transformshift{3.330112in}{1.003302in}%
\pgfsys@useobject{currentmarker}{}%
\end{pgfscope}%
\begin{pgfscope}%
\pgfsys@transformshift{3.565538in}{0.957674in}%
\pgfsys@useobject{currentmarker}{}%
\end{pgfscope}%
\begin{pgfscope}%
\pgfsys@transformshift{3.800963in}{0.895644in}%
\pgfsys@useobject{currentmarker}{}%
\end{pgfscope}%
\begin{pgfscope}%
\pgfsys@transformshift{4.036389in}{0.848737in}%
\pgfsys@useobject{currentmarker}{}%
\end{pgfscope}%
\end{pgfscope}%
\begin{pgfscope}%
\pgfsetrectcap%
\pgfsetmiterjoin%
\pgfsetlinewidth{0.803000pt}%
\definecolor{currentstroke}{rgb}{0.000000,0.000000,0.000000}%
\pgfsetstrokecolor{currentstroke}%
\pgfsetdash{}{0pt}%
\pgfpathmoveto{\pgfqpoint{0.740433in}{0.566590in}}%
\pgfpathlineto{\pgfqpoint{0.740433in}{2.395314in}}%
\pgfusepath{stroke}%
\end{pgfscope}%
\begin{pgfscope}%
\pgfsetrectcap%
\pgfsetmiterjoin%
\pgfsetlinewidth{0.803000pt}%
\definecolor{currentstroke}{rgb}{0.000000,0.000000,0.000000}%
\pgfsetstrokecolor{currentstroke}%
\pgfsetdash{}{0pt}%
\pgfpathmoveto{\pgfqpoint{4.036389in}{0.566590in}}%
\pgfpathlineto{\pgfqpoint{4.036389in}{2.395314in}}%
\pgfusepath{stroke}%
\end{pgfscope}%
\begin{pgfscope}%
\pgfsetrectcap%
\pgfsetmiterjoin%
\pgfsetlinewidth{0.803000pt}%
\definecolor{currentstroke}{rgb}{0.000000,0.000000,0.000000}%
\pgfsetstrokecolor{currentstroke}%
\pgfsetdash{}{0pt}%
\pgfpathmoveto{\pgfqpoint{0.740433in}{0.566590in}}%
\pgfpathlineto{\pgfqpoint{4.036389in}{0.566590in}}%
\pgfusepath{stroke}%
\end{pgfscope}%
\begin{pgfscope}%
\pgfsetrectcap%
\pgfsetmiterjoin%
\pgfsetlinewidth{0.803000pt}%
\definecolor{currentstroke}{rgb}{0.000000,0.000000,0.000000}%
\pgfsetstrokecolor{currentstroke}%
\pgfsetdash{}{0pt}%
\pgfpathmoveto{\pgfqpoint{0.740433in}{2.395314in}}%
\pgfpathlineto{\pgfqpoint{4.036389in}{2.395314in}}%
\pgfusepath{stroke}%
\end{pgfscope}%
\begin{pgfscope}%
\pgfsetbuttcap%
\pgfsetmiterjoin%
\definecolor{currentfill}{rgb}{1.000000,1.000000,1.000000}%
\pgfsetfillcolor{currentfill}%
\pgfsetfillopacity{0.800000}%
\pgfsetlinewidth{1.003750pt}%
\definecolor{currentstroke}{rgb}{0.800000,0.800000,0.800000}%
\pgfsetstrokecolor{currentstroke}%
\pgfsetstrokeopacity{0.800000}%
\pgfsetdash{}{0pt}%
\pgfpathmoveto{\pgfqpoint{0.827933in}{0.629090in}}%
\pgfpathlineto{\pgfqpoint{1.978111in}{0.629090in}}%
\pgfpathquadraticcurveto{\pgfqpoint{2.003111in}{0.629090in}}{\pgfqpoint{2.003111in}{0.654090in}}%
\pgfpathlineto{\pgfqpoint{2.003111in}{1.513088in}}%
\pgfpathquadraticcurveto{\pgfqpoint{2.003111in}{1.538088in}}{\pgfqpoint{1.978111in}{1.538088in}}%
\pgfpathlineto{\pgfqpoint{0.827933in}{1.538088in}}%
\pgfpathquadraticcurveto{\pgfqpoint{0.802933in}{1.538088in}}{\pgfqpoint{0.802933in}{1.513088in}}%
\pgfpathlineto{\pgfqpoint{0.802933in}{0.654090in}}%
\pgfpathquadraticcurveto{\pgfqpoint{0.802933in}{0.629090in}}{\pgfqpoint{0.827933in}{0.629090in}}%
\pgfpathclose%
\pgfusepath{stroke,fill}%
\end{pgfscope}%
\begin{pgfscope}%
\pgfsetbuttcap%
\pgfsetroundjoin%
\definecolor{currentfill}{rgb}{0.000000,0.000000,0.000000}%
\pgfsetfillcolor{currentfill}%
\pgfsetfillopacity{0.000000}%
\pgfsetlinewidth{1.003750pt}%
\definecolor{currentstroke}{rgb}{0.000000,0.447000,0.741000}%
\pgfsetstrokecolor{currentstroke}%
\pgfsetdash{}{0pt}%
\pgfsys@defobject{currentmarker}{\pgfqpoint{-0.041667in}{-0.041667in}}{\pgfqpoint{0.041667in}{0.041667in}}{%
\pgfpathmoveto{\pgfqpoint{0.000000in}{-0.041667in}}%
\pgfpathcurveto{\pgfqpoint{0.011050in}{-0.041667in}}{\pgfqpoint{0.021649in}{-0.037276in}}{\pgfqpoint{0.029463in}{-0.029463in}}%
\pgfpathcurveto{\pgfqpoint{0.037276in}{-0.021649in}}{\pgfqpoint{0.041667in}{-0.011050in}}{\pgfqpoint{0.041667in}{0.000000in}}%
\pgfpathcurveto{\pgfqpoint{0.041667in}{0.011050in}}{\pgfqpoint{0.037276in}{0.021649in}}{\pgfqpoint{0.029463in}{0.029463in}}%
\pgfpathcurveto{\pgfqpoint{0.021649in}{0.037276in}}{\pgfqpoint{0.011050in}{0.041667in}}{\pgfqpoint{0.000000in}{0.041667in}}%
\pgfpathcurveto{\pgfqpoint{-0.011050in}{0.041667in}}{\pgfqpoint{-0.021649in}{0.037276in}}{\pgfqpoint{-0.029463in}{0.029463in}}%
\pgfpathcurveto{\pgfqpoint{-0.037276in}{0.021649in}}{\pgfqpoint{-0.041667in}{0.011050in}}{\pgfqpoint{-0.041667in}{0.000000in}}%
\pgfpathcurveto{\pgfqpoint{-0.041667in}{-0.011050in}}{\pgfqpoint{-0.037276in}{-0.021649in}}{\pgfqpoint{-0.029463in}{-0.029463in}}%
\pgfpathcurveto{\pgfqpoint{-0.021649in}{-0.037276in}}{\pgfqpoint{-0.011050in}{-0.041667in}}{\pgfqpoint{0.000000in}{-0.041667in}}%
\pgfpathclose%
\pgfusepath{stroke,fill}%
}%
\begin{pgfscope}%
\pgfsys@transformshift{0.977933in}{1.444338in}%
\pgfsys@useobject{currentmarker}{}%
\end{pgfscope}%
\end{pgfscope}%
\begin{pgfscope}%
\definecolor{textcolor}{rgb}{0.000000,0.000000,0.000000}%
\pgfsetstrokecolor{textcolor}%
\pgfsetfillcolor{textcolor}%
\pgftext[x=1.202933in,y=1.400588in,left,base]{\color{textcolor}\rmfamily\fontsize{9.000000}{10.800000}\selectfont \(\displaystyle \nu_7 = \) -132.50}%
\end{pgfscope}%
\begin{pgfscope}%
\pgfsetbuttcap%
\pgfsetroundjoin%
\definecolor{currentfill}{rgb}{0.850000,0.324000,0.098000}%
\pgfsetfillcolor{currentfill}%
\pgfsetlinewidth{1.003750pt}%
\definecolor{currentstroke}{rgb}{0.850000,0.324000,0.098000}%
\pgfsetstrokecolor{currentstroke}%
\pgfsetdash{}{0pt}%
\pgfsys@defobject{currentmarker}{\pgfqpoint{-0.041667in}{-0.041667in}}{\pgfqpoint{0.041667in}{0.041667in}}{%
\pgfpathmoveto{\pgfqpoint{-0.041667in}{0.000000in}}%
\pgfpathlineto{\pgfqpoint{0.041667in}{0.000000in}}%
\pgfpathmoveto{\pgfqpoint{0.000000in}{-0.041667in}}%
\pgfpathlineto{\pgfqpoint{0.000000in}{0.041667in}}%
\pgfusepath{stroke,fill}%
}%
\begin{pgfscope}%
\pgfsys@transformshift{0.977933in}{1.270038in}%
\pgfsys@useobject{currentmarker}{}%
\end{pgfscope}%
\end{pgfscope}%
\begin{pgfscope}%
\definecolor{textcolor}{rgb}{0.000000,0.000000,0.000000}%
\pgfsetstrokecolor{textcolor}%
\pgfsetfillcolor{textcolor}%
\pgftext[x=1.202933in,y=1.226288in,left,base]{\color{textcolor}\rmfamily\fontsize{9.000000}{10.800000}\selectfont \(\displaystyle \nu_8 = \) -131.40}%
\end{pgfscope}%
\begin{pgfscope}%
\pgfsetbuttcap%
\pgfsetmiterjoin%
\definecolor{currentfill}{rgb}{0.000000,0.000000,0.000000}%
\pgfsetfillcolor{currentfill}%
\pgfsetfillopacity{0.000000}%
\pgfsetlinewidth{1.003750pt}%
\definecolor{currentstroke}{rgb}{0.000000,0.500000,0.000000}%
\pgfsetstrokecolor{currentstroke}%
\pgfsetdash{}{0pt}%
\pgfsys@defobject{currentmarker}{\pgfqpoint{-0.041667in}{-0.041667in}}{\pgfqpoint{0.041667in}{0.041667in}}{%
\pgfpathmoveto{\pgfqpoint{-0.041667in}{-0.041667in}}%
\pgfpathlineto{\pgfqpoint{0.041667in}{-0.041667in}}%
\pgfpathlineto{\pgfqpoint{0.041667in}{0.041667in}}%
\pgfpathlineto{\pgfqpoint{-0.041667in}{0.041667in}}%
\pgfpathclose%
\pgfusepath{stroke,fill}%
}%
\begin{pgfscope}%
\pgfsys@transformshift{0.977933in}{1.095739in}%
\pgfsys@useobject{currentmarker}{}%
\end{pgfscope}%
\end{pgfscope}%
\begin{pgfscope}%
\definecolor{textcolor}{rgb}{0.000000,0.000000,0.000000}%
\pgfsetstrokecolor{textcolor}%
\pgfsetfillcolor{textcolor}%
\pgftext[x=1.202933in,y=1.051989in,left,base]{\color{textcolor}\rmfamily\fontsize{9.000000}{10.800000}\selectfont \(\displaystyle \nu_9 = \) -130.01}%
\end{pgfscope}%
\begin{pgfscope}%
\pgfsetbuttcap%
\pgfsetroundjoin%
\definecolor{currentfill}{rgb}{0.494000,0.184000,0.556000}%
\pgfsetfillcolor{currentfill}%
\pgfsetlinewidth{1.003750pt}%
\definecolor{currentstroke}{rgb}{0.494000,0.184000,0.556000}%
\pgfsetstrokecolor{currentstroke}%
\pgfsetdash{}{0pt}%
\pgfsys@defobject{currentmarker}{\pgfqpoint{-0.041667in}{-0.041667in}}{\pgfqpoint{0.041667in}{0.041667in}}{%
\pgfpathmoveto{\pgfqpoint{-0.041667in}{-0.041667in}}%
\pgfpathlineto{\pgfqpoint{0.041667in}{0.041667in}}%
\pgfpathmoveto{\pgfqpoint{-0.041667in}{0.041667in}}%
\pgfpathlineto{\pgfqpoint{0.041667in}{-0.041667in}}%
\pgfusepath{stroke,fill}%
}%
\begin{pgfscope}%
\pgfsys@transformshift{0.977933in}{0.921439in}%
\pgfsys@useobject{currentmarker}{}%
\end{pgfscope}%
\end{pgfscope}%
\begin{pgfscope}%
\definecolor{textcolor}{rgb}{0.000000,0.000000,0.000000}%
\pgfsetstrokecolor{textcolor}%
\pgfsetfillcolor{textcolor}%
\pgftext[x=1.202933in,y=0.877689in,left,base]{\color{textcolor}\rmfamily\fontsize{9.000000}{10.800000}\selectfont \(\displaystyle \nu_{10} = \) -129.17}%
\end{pgfscope}%
\begin{pgfscope}%
\pgfsetbuttcap%
\pgfsetmiterjoin%
\definecolor{currentfill}{rgb}{0.000000,0.000000,0.000000}%
\pgfsetfillcolor{currentfill}%
\pgfsetfillopacity{0.000000}%
\pgfsetlinewidth{1.003750pt}%
\definecolor{currentstroke}{rgb}{0.635000,0.078000,0.184000}%
\pgfsetstrokecolor{currentstroke}%
\pgfsetdash{}{0pt}%
\pgfsys@defobject{currentmarker}{\pgfqpoint{-0.035355in}{-0.058926in}}{\pgfqpoint{0.035355in}{0.058926in}}{%
\pgfpathmoveto{\pgfqpoint{-0.000000in}{-0.058926in}}%
\pgfpathlineto{\pgfqpoint{0.035355in}{0.000000in}}%
\pgfpathlineto{\pgfqpoint{0.000000in}{0.058926in}}%
\pgfpathlineto{\pgfqpoint{-0.035355in}{0.000000in}}%
\pgfpathclose%
\pgfusepath{stroke,fill}%
}%
\begin{pgfscope}%
\pgfsys@transformshift{0.977933in}{0.747140in}%
\pgfsys@useobject{currentmarker}{}%
\end{pgfscope}%
\end{pgfscope}%
\begin{pgfscope}%
\definecolor{textcolor}{rgb}{0.000000,0.000000,0.000000}%
\pgfsetstrokecolor{textcolor}%
\pgfsetfillcolor{textcolor}%
\pgftext[x=1.202933in,y=0.703390in,left,base]{\color{textcolor}\rmfamily\fontsize{9.000000}{10.800000}\selectfont \(\displaystyle \nu_{11} = \) -128.09}%
\end{pgfscope}%
\end{pgfpicture}%
\makeatother%
\endgroup%
}
					\caption{Cluster II}
					\label{SubFig:Cluster_II_imag}
				\end{subfigure}
				\begin{subfigure}[h]{0.5\textwidth}
					\centering
					\resizebox{\linewidth}{!}{%% Creator: Matplotlib, PGF backend
%%
%% To include the figure in your LaTeX document, write
%%   \input{<filename>.pgf}
%%
%% Make sure the required packages are loaded in your preamble
%%   \usepackage{pgf}
%%
%% and, on pdftex
%%   \usepackage[utf8]{inputenc}\DeclareUnicodeCharacter{2212}{-}
%%
%% or, on luatex and xetex
%%   \usepackage{unicode-math}
%%
%% Figures using additional raster images can only be included by \input if
%% they are in the same directory as the main LaTeX file. For loading figures
%% from other directories you can use the `import` package
%%   \usepackage{import}
%%
%% and then include the figures with
%%   \import{<path to file>}{<filename>.pgf}
%%
%% Matplotlib used the following preamble
%%   \usepackage[utf8x]{inputenc}
%%   \usepackage[T1]{fontenc}
%%   \usepackage{amsmath,amssymb,amsfonts}
%%
\begingroup%
\makeatletter%
\begin{pgfpicture}%
\pgfpathrectangle{\pgfpointorigin}{\pgfqpoint{4.136389in}{2.495314in}}%
\pgfusepath{use as bounding box, clip}%
\begin{pgfscope}%
\pgfsetbuttcap%
\pgfsetmiterjoin%
\definecolor{currentfill}{rgb}{1.000000,1.000000,1.000000}%
\pgfsetfillcolor{currentfill}%
\pgfsetlinewidth{0.000000pt}%
\definecolor{currentstroke}{rgb}{1.000000,1.000000,1.000000}%
\pgfsetstrokecolor{currentstroke}%
\pgfsetdash{}{0pt}%
\pgfpathmoveto{\pgfqpoint{-0.000000in}{0.000000in}}%
\pgfpathlineto{\pgfqpoint{4.136389in}{0.000000in}}%
\pgfpathlineto{\pgfqpoint{4.136389in}{2.495314in}}%
\pgfpathlineto{\pgfqpoint{-0.000000in}{2.495314in}}%
\pgfpathclose%
\pgfusepath{fill}%
\end{pgfscope}%
\begin{pgfscope}%
\pgfsetbuttcap%
\pgfsetmiterjoin%
\definecolor{currentfill}{rgb}{1.000000,1.000000,1.000000}%
\pgfsetfillcolor{currentfill}%
\pgfsetlinewidth{0.000000pt}%
\definecolor{currentstroke}{rgb}{0.000000,0.000000,0.000000}%
\pgfsetstrokecolor{currentstroke}%
\pgfsetstrokeopacity{0.000000}%
\pgfsetdash{}{0pt}%
\pgfpathmoveto{\pgfqpoint{0.740433in}{0.566590in}}%
\pgfpathlineto{\pgfqpoint{4.036389in}{0.566590in}}%
\pgfpathlineto{\pgfqpoint{4.036389in}{2.395314in}}%
\pgfpathlineto{\pgfqpoint{0.740433in}{2.395314in}}%
\pgfpathclose%
\pgfusepath{fill}%
\end{pgfscope}%
\begin{pgfscope}%
\pgfpathrectangle{\pgfqpoint{0.740433in}{0.566590in}}{\pgfqpoint{3.295956in}{1.828724in}}%
\pgfusepath{clip}%
\pgfsetrectcap%
\pgfsetroundjoin%
\pgfsetlinewidth{0.803000pt}%
\definecolor{currentstroke}{rgb}{0.690196,0.690196,0.690196}%
\pgfsetstrokecolor{currentstroke}%
\pgfsetdash{}{0pt}%
\pgfpathmoveto{\pgfqpoint{0.740433in}{0.566590in}}%
\pgfpathlineto{\pgfqpoint{0.740433in}{2.395314in}}%
\pgfusepath{stroke}%
\end{pgfscope}%
\begin{pgfscope}%
\pgfsetbuttcap%
\pgfsetroundjoin%
\definecolor{currentfill}{rgb}{0.000000,0.000000,0.000000}%
\pgfsetfillcolor{currentfill}%
\pgfsetlinewidth{0.803000pt}%
\definecolor{currentstroke}{rgb}{0.000000,0.000000,0.000000}%
\pgfsetstrokecolor{currentstroke}%
\pgfsetdash{}{0pt}%
\pgfsys@defobject{currentmarker}{\pgfqpoint{0.000000in}{-0.048611in}}{\pgfqpoint{0.000000in}{0.000000in}}{%
\pgfpathmoveto{\pgfqpoint{0.000000in}{0.000000in}}%
\pgfpathlineto{\pgfqpoint{0.000000in}{-0.048611in}}%
\pgfusepath{stroke,fill}%
}%
\begin{pgfscope}%
\pgfsys@transformshift{0.740433in}{0.566590in}%
\pgfsys@useobject{currentmarker}{}%
\end{pgfscope}%
\end{pgfscope}%
\begin{pgfscope}%
\definecolor{textcolor}{rgb}{0.000000,0.000000,0.000000}%
\pgfsetstrokecolor{textcolor}%
\pgfsetfillcolor{textcolor}%
\pgftext[x=0.740433in,y=0.469368in,,top]{\color{textcolor}\rmfamily\fontsize{12.000000}{14.400000}\selectfont \(\displaystyle {-10}\)}%
\end{pgfscope}%
\begin{pgfscope}%
\pgfpathrectangle{\pgfqpoint{0.740433in}{0.566590in}}{\pgfqpoint{3.295956in}{1.828724in}}%
\pgfusepath{clip}%
\pgfsetrectcap%
\pgfsetroundjoin%
\pgfsetlinewidth{0.803000pt}%
\definecolor{currentstroke}{rgb}{0.690196,0.690196,0.690196}%
\pgfsetstrokecolor{currentstroke}%
\pgfsetdash{}{0pt}%
\pgfpathmoveto{\pgfqpoint{1.247503in}{0.566590in}}%
\pgfpathlineto{\pgfqpoint{1.247503in}{2.395314in}}%
\pgfusepath{stroke}%
\end{pgfscope}%
\begin{pgfscope}%
\pgfsetbuttcap%
\pgfsetroundjoin%
\definecolor{currentfill}{rgb}{0.000000,0.000000,0.000000}%
\pgfsetfillcolor{currentfill}%
\pgfsetlinewidth{0.803000pt}%
\definecolor{currentstroke}{rgb}{0.000000,0.000000,0.000000}%
\pgfsetstrokecolor{currentstroke}%
\pgfsetdash{}{0pt}%
\pgfsys@defobject{currentmarker}{\pgfqpoint{0.000000in}{-0.048611in}}{\pgfqpoint{0.000000in}{0.000000in}}{%
\pgfpathmoveto{\pgfqpoint{0.000000in}{0.000000in}}%
\pgfpathlineto{\pgfqpoint{0.000000in}{-0.048611in}}%
\pgfusepath{stroke,fill}%
}%
\begin{pgfscope}%
\pgfsys@transformshift{1.247503in}{0.566590in}%
\pgfsys@useobject{currentmarker}{}%
\end{pgfscope}%
\end{pgfscope}%
\begin{pgfscope}%
\definecolor{textcolor}{rgb}{0.000000,0.000000,0.000000}%
\pgfsetstrokecolor{textcolor}%
\pgfsetfillcolor{textcolor}%
\pgftext[x=1.247503in,y=0.469368in,,top]{\color{textcolor}\rmfamily\fontsize{12.000000}{14.400000}\selectfont \(\displaystyle {0}\)}%
\end{pgfscope}%
\begin{pgfscope}%
\pgfpathrectangle{\pgfqpoint{0.740433in}{0.566590in}}{\pgfqpoint{3.295956in}{1.828724in}}%
\pgfusepath{clip}%
\pgfsetrectcap%
\pgfsetroundjoin%
\pgfsetlinewidth{0.803000pt}%
\definecolor{currentstroke}{rgb}{0.690196,0.690196,0.690196}%
\pgfsetstrokecolor{currentstroke}%
\pgfsetdash{}{0pt}%
\pgfpathmoveto{\pgfqpoint{1.754573in}{0.566590in}}%
\pgfpathlineto{\pgfqpoint{1.754573in}{2.395314in}}%
\pgfusepath{stroke}%
\end{pgfscope}%
\begin{pgfscope}%
\pgfsetbuttcap%
\pgfsetroundjoin%
\definecolor{currentfill}{rgb}{0.000000,0.000000,0.000000}%
\pgfsetfillcolor{currentfill}%
\pgfsetlinewidth{0.803000pt}%
\definecolor{currentstroke}{rgb}{0.000000,0.000000,0.000000}%
\pgfsetstrokecolor{currentstroke}%
\pgfsetdash{}{0pt}%
\pgfsys@defobject{currentmarker}{\pgfqpoint{0.000000in}{-0.048611in}}{\pgfqpoint{0.000000in}{0.000000in}}{%
\pgfpathmoveto{\pgfqpoint{0.000000in}{0.000000in}}%
\pgfpathlineto{\pgfqpoint{0.000000in}{-0.048611in}}%
\pgfusepath{stroke,fill}%
}%
\begin{pgfscope}%
\pgfsys@transformshift{1.754573in}{0.566590in}%
\pgfsys@useobject{currentmarker}{}%
\end{pgfscope}%
\end{pgfscope}%
\begin{pgfscope}%
\definecolor{textcolor}{rgb}{0.000000,0.000000,0.000000}%
\pgfsetstrokecolor{textcolor}%
\pgfsetfillcolor{textcolor}%
\pgftext[x=1.754573in,y=0.469368in,,top]{\color{textcolor}\rmfamily\fontsize{12.000000}{14.400000}\selectfont \(\displaystyle {10}\)}%
\end{pgfscope}%
\begin{pgfscope}%
\pgfpathrectangle{\pgfqpoint{0.740433in}{0.566590in}}{\pgfqpoint{3.295956in}{1.828724in}}%
\pgfusepath{clip}%
\pgfsetrectcap%
\pgfsetroundjoin%
\pgfsetlinewidth{0.803000pt}%
\definecolor{currentstroke}{rgb}{0.690196,0.690196,0.690196}%
\pgfsetstrokecolor{currentstroke}%
\pgfsetdash{}{0pt}%
\pgfpathmoveto{\pgfqpoint{2.261643in}{0.566590in}}%
\pgfpathlineto{\pgfqpoint{2.261643in}{2.395314in}}%
\pgfusepath{stroke}%
\end{pgfscope}%
\begin{pgfscope}%
\pgfsetbuttcap%
\pgfsetroundjoin%
\definecolor{currentfill}{rgb}{0.000000,0.000000,0.000000}%
\pgfsetfillcolor{currentfill}%
\pgfsetlinewidth{0.803000pt}%
\definecolor{currentstroke}{rgb}{0.000000,0.000000,0.000000}%
\pgfsetstrokecolor{currentstroke}%
\pgfsetdash{}{0pt}%
\pgfsys@defobject{currentmarker}{\pgfqpoint{0.000000in}{-0.048611in}}{\pgfqpoint{0.000000in}{0.000000in}}{%
\pgfpathmoveto{\pgfqpoint{0.000000in}{0.000000in}}%
\pgfpathlineto{\pgfqpoint{0.000000in}{-0.048611in}}%
\pgfusepath{stroke,fill}%
}%
\begin{pgfscope}%
\pgfsys@transformshift{2.261643in}{0.566590in}%
\pgfsys@useobject{currentmarker}{}%
\end{pgfscope}%
\end{pgfscope}%
\begin{pgfscope}%
\definecolor{textcolor}{rgb}{0.000000,0.000000,0.000000}%
\pgfsetstrokecolor{textcolor}%
\pgfsetfillcolor{textcolor}%
\pgftext[x=2.261643in,y=0.469368in,,top]{\color{textcolor}\rmfamily\fontsize{12.000000}{14.400000}\selectfont \(\displaystyle {20}\)}%
\end{pgfscope}%
\begin{pgfscope}%
\pgfpathrectangle{\pgfqpoint{0.740433in}{0.566590in}}{\pgfqpoint{3.295956in}{1.828724in}}%
\pgfusepath{clip}%
\pgfsetrectcap%
\pgfsetroundjoin%
\pgfsetlinewidth{0.803000pt}%
\definecolor{currentstroke}{rgb}{0.690196,0.690196,0.690196}%
\pgfsetstrokecolor{currentstroke}%
\pgfsetdash{}{0pt}%
\pgfpathmoveto{\pgfqpoint{2.768713in}{0.566590in}}%
\pgfpathlineto{\pgfqpoint{2.768713in}{2.395314in}}%
\pgfusepath{stroke}%
\end{pgfscope}%
\begin{pgfscope}%
\pgfsetbuttcap%
\pgfsetroundjoin%
\definecolor{currentfill}{rgb}{0.000000,0.000000,0.000000}%
\pgfsetfillcolor{currentfill}%
\pgfsetlinewidth{0.803000pt}%
\definecolor{currentstroke}{rgb}{0.000000,0.000000,0.000000}%
\pgfsetstrokecolor{currentstroke}%
\pgfsetdash{}{0pt}%
\pgfsys@defobject{currentmarker}{\pgfqpoint{0.000000in}{-0.048611in}}{\pgfqpoint{0.000000in}{0.000000in}}{%
\pgfpathmoveto{\pgfqpoint{0.000000in}{0.000000in}}%
\pgfpathlineto{\pgfqpoint{0.000000in}{-0.048611in}}%
\pgfusepath{stroke,fill}%
}%
\begin{pgfscope}%
\pgfsys@transformshift{2.768713in}{0.566590in}%
\pgfsys@useobject{currentmarker}{}%
\end{pgfscope}%
\end{pgfscope}%
\begin{pgfscope}%
\definecolor{textcolor}{rgb}{0.000000,0.000000,0.000000}%
\pgfsetstrokecolor{textcolor}%
\pgfsetfillcolor{textcolor}%
\pgftext[x=2.768713in,y=0.469368in,,top]{\color{textcolor}\rmfamily\fontsize{12.000000}{14.400000}\selectfont \(\displaystyle {30}\)}%
\end{pgfscope}%
\begin{pgfscope}%
\pgfpathrectangle{\pgfqpoint{0.740433in}{0.566590in}}{\pgfqpoint{3.295956in}{1.828724in}}%
\pgfusepath{clip}%
\pgfsetrectcap%
\pgfsetroundjoin%
\pgfsetlinewidth{0.803000pt}%
\definecolor{currentstroke}{rgb}{0.690196,0.690196,0.690196}%
\pgfsetstrokecolor{currentstroke}%
\pgfsetdash{}{0pt}%
\pgfpathmoveto{\pgfqpoint{3.275783in}{0.566590in}}%
\pgfpathlineto{\pgfqpoint{3.275783in}{2.395314in}}%
\pgfusepath{stroke}%
\end{pgfscope}%
\begin{pgfscope}%
\pgfsetbuttcap%
\pgfsetroundjoin%
\definecolor{currentfill}{rgb}{0.000000,0.000000,0.000000}%
\pgfsetfillcolor{currentfill}%
\pgfsetlinewidth{0.803000pt}%
\definecolor{currentstroke}{rgb}{0.000000,0.000000,0.000000}%
\pgfsetstrokecolor{currentstroke}%
\pgfsetdash{}{0pt}%
\pgfsys@defobject{currentmarker}{\pgfqpoint{0.000000in}{-0.048611in}}{\pgfqpoint{0.000000in}{0.000000in}}{%
\pgfpathmoveto{\pgfqpoint{0.000000in}{0.000000in}}%
\pgfpathlineto{\pgfqpoint{0.000000in}{-0.048611in}}%
\pgfusepath{stroke,fill}%
}%
\begin{pgfscope}%
\pgfsys@transformshift{3.275783in}{0.566590in}%
\pgfsys@useobject{currentmarker}{}%
\end{pgfscope}%
\end{pgfscope}%
\begin{pgfscope}%
\definecolor{textcolor}{rgb}{0.000000,0.000000,0.000000}%
\pgfsetstrokecolor{textcolor}%
\pgfsetfillcolor{textcolor}%
\pgftext[x=3.275783in,y=0.469368in,,top]{\color{textcolor}\rmfamily\fontsize{12.000000}{14.400000}\selectfont \(\displaystyle {40}\)}%
\end{pgfscope}%
\begin{pgfscope}%
\pgfpathrectangle{\pgfqpoint{0.740433in}{0.566590in}}{\pgfqpoint{3.295956in}{1.828724in}}%
\pgfusepath{clip}%
\pgfsetrectcap%
\pgfsetroundjoin%
\pgfsetlinewidth{0.803000pt}%
\definecolor{currentstroke}{rgb}{0.690196,0.690196,0.690196}%
\pgfsetstrokecolor{currentstroke}%
\pgfsetdash{}{0pt}%
\pgfpathmoveto{\pgfqpoint{3.782853in}{0.566590in}}%
\pgfpathlineto{\pgfqpoint{3.782853in}{2.395314in}}%
\pgfusepath{stroke}%
\end{pgfscope}%
\begin{pgfscope}%
\pgfsetbuttcap%
\pgfsetroundjoin%
\definecolor{currentfill}{rgb}{0.000000,0.000000,0.000000}%
\pgfsetfillcolor{currentfill}%
\pgfsetlinewidth{0.803000pt}%
\definecolor{currentstroke}{rgb}{0.000000,0.000000,0.000000}%
\pgfsetstrokecolor{currentstroke}%
\pgfsetdash{}{0pt}%
\pgfsys@defobject{currentmarker}{\pgfqpoint{0.000000in}{-0.048611in}}{\pgfqpoint{0.000000in}{0.000000in}}{%
\pgfpathmoveto{\pgfqpoint{0.000000in}{0.000000in}}%
\pgfpathlineto{\pgfqpoint{0.000000in}{-0.048611in}}%
\pgfusepath{stroke,fill}%
}%
\begin{pgfscope}%
\pgfsys@transformshift{3.782853in}{0.566590in}%
\pgfsys@useobject{currentmarker}{}%
\end{pgfscope}%
\end{pgfscope}%
\begin{pgfscope}%
\definecolor{textcolor}{rgb}{0.000000,0.000000,0.000000}%
\pgfsetstrokecolor{textcolor}%
\pgfsetfillcolor{textcolor}%
\pgftext[x=3.782853in,y=0.469368in,,top]{\color{textcolor}\rmfamily\fontsize{12.000000}{14.400000}\selectfont \(\displaystyle {50}\)}%
\end{pgfscope}%
\begin{pgfscope}%
\definecolor{textcolor}{rgb}{0.000000,0.000000,0.000000}%
\pgfsetstrokecolor{textcolor}%
\pgfsetfillcolor{textcolor}%
\pgftext[x=2.388411in,y=0.266626in,,top]{\color{textcolor}\rmfamily\fontsize{12.000000}{14.400000}\selectfont SNR [dB]}%
\end{pgfscope}%
\begin{pgfscope}%
\pgfpathrectangle{\pgfqpoint{0.740433in}{0.566590in}}{\pgfqpoint{3.295956in}{1.828724in}}%
\pgfusepath{clip}%
\pgfsetrectcap%
\pgfsetroundjoin%
\pgfsetlinewidth{0.803000pt}%
\definecolor{currentstroke}{rgb}{0.690196,0.690196,0.690196}%
\pgfsetstrokecolor{currentstroke}%
\pgfsetdash{}{0pt}%
\pgfpathmoveto{\pgfqpoint{0.740433in}{0.752967in}}%
\pgfpathlineto{\pgfqpoint{4.036389in}{0.752967in}}%
\pgfusepath{stroke}%
\end{pgfscope}%
\begin{pgfscope}%
\pgfsetbuttcap%
\pgfsetroundjoin%
\definecolor{currentfill}{rgb}{0.000000,0.000000,0.000000}%
\pgfsetfillcolor{currentfill}%
\pgfsetlinewidth{0.803000pt}%
\definecolor{currentstroke}{rgb}{0.000000,0.000000,0.000000}%
\pgfsetstrokecolor{currentstroke}%
\pgfsetdash{}{0pt}%
\pgfsys@defobject{currentmarker}{\pgfqpoint{-0.048611in}{0.000000in}}{\pgfqpoint{-0.000000in}{0.000000in}}{%
\pgfpathmoveto{\pgfqpoint{-0.000000in}{0.000000in}}%
\pgfpathlineto{\pgfqpoint{-0.048611in}{0.000000in}}%
\pgfusepath{stroke,fill}%
}%
\begin{pgfscope}%
\pgfsys@transformshift{0.740433in}{0.752967in}%
\pgfsys@useobject{currentmarker}{}%
\end{pgfscope}%
\end{pgfscope}%
\begin{pgfscope}%
\definecolor{textcolor}{rgb}{0.000000,0.000000,0.000000}%
\pgfsetstrokecolor{textcolor}%
\pgfsetfillcolor{textcolor}%
\pgftext[x=0.322222in, y=0.695574in, left, base]{\color{textcolor}\rmfamily\fontsize{12.000000}{14.400000}\selectfont \(\displaystyle {10^{-4}}\)}%
\end{pgfscope}%
\begin{pgfscope}%
\pgfpathrectangle{\pgfqpoint{0.740433in}{0.566590in}}{\pgfqpoint{3.295956in}{1.828724in}}%
\pgfusepath{clip}%
\pgfsetrectcap%
\pgfsetroundjoin%
\pgfsetlinewidth{0.803000pt}%
\definecolor{currentstroke}{rgb}{0.690196,0.690196,0.690196}%
\pgfsetstrokecolor{currentstroke}%
\pgfsetdash{}{0pt}%
\pgfpathmoveto{\pgfqpoint{0.740433in}{1.236303in}}%
\pgfpathlineto{\pgfqpoint{4.036389in}{1.236303in}}%
\pgfusepath{stroke}%
\end{pgfscope}%
\begin{pgfscope}%
\pgfsetbuttcap%
\pgfsetroundjoin%
\definecolor{currentfill}{rgb}{0.000000,0.000000,0.000000}%
\pgfsetfillcolor{currentfill}%
\pgfsetlinewidth{0.803000pt}%
\definecolor{currentstroke}{rgb}{0.000000,0.000000,0.000000}%
\pgfsetstrokecolor{currentstroke}%
\pgfsetdash{}{0pt}%
\pgfsys@defobject{currentmarker}{\pgfqpoint{-0.048611in}{0.000000in}}{\pgfqpoint{-0.000000in}{0.000000in}}{%
\pgfpathmoveto{\pgfqpoint{-0.000000in}{0.000000in}}%
\pgfpathlineto{\pgfqpoint{-0.048611in}{0.000000in}}%
\pgfusepath{stroke,fill}%
}%
\begin{pgfscope}%
\pgfsys@transformshift{0.740433in}{1.236303in}%
\pgfsys@useobject{currentmarker}{}%
\end{pgfscope}%
\end{pgfscope}%
\begin{pgfscope}%
\definecolor{textcolor}{rgb}{0.000000,0.000000,0.000000}%
\pgfsetstrokecolor{textcolor}%
\pgfsetfillcolor{textcolor}%
\pgftext[x=0.322222in, y=1.178910in, left, base]{\color{textcolor}\rmfamily\fontsize{12.000000}{14.400000}\selectfont \(\displaystyle {10^{-2}}\)}%
\end{pgfscope}%
\begin{pgfscope}%
\pgfpathrectangle{\pgfqpoint{0.740433in}{0.566590in}}{\pgfqpoint{3.295956in}{1.828724in}}%
\pgfusepath{clip}%
\pgfsetrectcap%
\pgfsetroundjoin%
\pgfsetlinewidth{0.803000pt}%
\definecolor{currentstroke}{rgb}{0.690196,0.690196,0.690196}%
\pgfsetstrokecolor{currentstroke}%
\pgfsetdash{}{0pt}%
\pgfpathmoveto{\pgfqpoint{0.740433in}{1.719639in}}%
\pgfpathlineto{\pgfqpoint{4.036389in}{1.719639in}}%
\pgfusepath{stroke}%
\end{pgfscope}%
\begin{pgfscope}%
\pgfsetbuttcap%
\pgfsetroundjoin%
\definecolor{currentfill}{rgb}{0.000000,0.000000,0.000000}%
\pgfsetfillcolor{currentfill}%
\pgfsetlinewidth{0.803000pt}%
\definecolor{currentstroke}{rgb}{0.000000,0.000000,0.000000}%
\pgfsetstrokecolor{currentstroke}%
\pgfsetdash{}{0pt}%
\pgfsys@defobject{currentmarker}{\pgfqpoint{-0.048611in}{0.000000in}}{\pgfqpoint{-0.000000in}{0.000000in}}{%
\pgfpathmoveto{\pgfqpoint{-0.000000in}{0.000000in}}%
\pgfpathlineto{\pgfqpoint{-0.048611in}{0.000000in}}%
\pgfusepath{stroke,fill}%
}%
\begin{pgfscope}%
\pgfsys@transformshift{0.740433in}{1.719639in}%
\pgfsys@useobject{currentmarker}{}%
\end{pgfscope}%
\end{pgfscope}%
\begin{pgfscope}%
\definecolor{textcolor}{rgb}{0.000000,0.000000,0.000000}%
\pgfsetstrokecolor{textcolor}%
\pgfsetfillcolor{textcolor}%
\pgftext[x=0.414045in, y=1.662246in, left, base]{\color{textcolor}\rmfamily\fontsize{12.000000}{14.400000}\selectfont \(\displaystyle {10^{0}}\)}%
\end{pgfscope}%
\begin{pgfscope}%
\pgfpathrectangle{\pgfqpoint{0.740433in}{0.566590in}}{\pgfqpoint{3.295956in}{1.828724in}}%
\pgfusepath{clip}%
\pgfsetrectcap%
\pgfsetroundjoin%
\pgfsetlinewidth{0.803000pt}%
\definecolor{currentstroke}{rgb}{0.690196,0.690196,0.690196}%
\pgfsetstrokecolor{currentstroke}%
\pgfsetdash{}{0pt}%
\pgfpathmoveto{\pgfqpoint{0.740433in}{2.202975in}}%
\pgfpathlineto{\pgfqpoint{4.036389in}{2.202975in}}%
\pgfusepath{stroke}%
\end{pgfscope}%
\begin{pgfscope}%
\pgfsetbuttcap%
\pgfsetroundjoin%
\definecolor{currentfill}{rgb}{0.000000,0.000000,0.000000}%
\pgfsetfillcolor{currentfill}%
\pgfsetlinewidth{0.803000pt}%
\definecolor{currentstroke}{rgb}{0.000000,0.000000,0.000000}%
\pgfsetstrokecolor{currentstroke}%
\pgfsetdash{}{0pt}%
\pgfsys@defobject{currentmarker}{\pgfqpoint{-0.048611in}{0.000000in}}{\pgfqpoint{-0.000000in}{0.000000in}}{%
\pgfpathmoveto{\pgfqpoint{-0.000000in}{0.000000in}}%
\pgfpathlineto{\pgfqpoint{-0.048611in}{0.000000in}}%
\pgfusepath{stroke,fill}%
}%
\begin{pgfscope}%
\pgfsys@transformshift{0.740433in}{2.202975in}%
\pgfsys@useobject{currentmarker}{}%
\end{pgfscope}%
\end{pgfscope}%
\begin{pgfscope}%
\definecolor{textcolor}{rgb}{0.000000,0.000000,0.000000}%
\pgfsetstrokecolor{textcolor}%
\pgfsetfillcolor{textcolor}%
\pgftext[x=0.414045in, y=2.145582in, left, base]{\color{textcolor}\rmfamily\fontsize{12.000000}{14.400000}\selectfont \(\displaystyle {10^{2}}\)}%
\end{pgfscope}%
\begin{pgfscope}%
\definecolor{textcolor}{rgb}{0.000000,0.000000,0.000000}%
\pgfsetstrokecolor{textcolor}%
\pgfsetfillcolor{textcolor}%
\pgftext[x=0.266667in,y=1.480952in,,bottom,rotate=90.000000]{\color{textcolor}\rmfamily\fontsize{12.000000}{14.400000}\selectfont \(\displaystyle \hat{\sigma}_{\nu}(\mathrm{SNR})\)}%
\end{pgfscope}%
\begin{pgfscope}%
\pgfpathrectangle{\pgfqpoint{0.740433in}{0.566590in}}{\pgfqpoint{3.295956in}{1.828724in}}%
\pgfusepath{clip}%
\pgfsetbuttcap%
\pgfsetroundjoin%
\pgfsetlinewidth{1.505625pt}%
\definecolor{currentstroke}{rgb}{0.000000,0.447000,0.741000}%
\pgfsetstrokecolor{currentstroke}%
\pgfsetdash{{5.550000pt}{2.400000pt}}{0.000000pt}%
\pgfpathmoveto{\pgfqpoint{0.740433in}{2.284511in}}%
\pgfpathlineto{\pgfqpoint{0.837373in}{2.275245in}}%
\pgfpathlineto{\pgfqpoint{0.934312in}{2.273352in}}%
\pgfpathlineto{\pgfqpoint{1.031252in}{2.264873in}}%
\pgfpathlineto{\pgfqpoint{1.128192in}{2.267535in}}%
\pgfpathlineto{\pgfqpoint{1.225132in}{2.263825in}}%
\pgfpathlineto{\pgfqpoint{1.322072in}{2.255542in}}%
\pgfpathlineto{\pgfqpoint{1.419012in}{2.267895in}}%
\pgfpathlineto{\pgfqpoint{1.515952in}{2.267942in}}%
\pgfpathlineto{\pgfqpoint{1.612892in}{2.279980in}}%
\pgfpathlineto{\pgfqpoint{1.709831in}{2.277543in}}%
\pgfpathlineto{\pgfqpoint{1.806771in}{2.280968in}}%
\pgfpathlineto{\pgfqpoint{1.903711in}{2.272996in}}%
\pgfpathlineto{\pgfqpoint{2.000651in}{2.260214in}}%
\pgfpathlineto{\pgfqpoint{2.097591in}{2.233441in}}%
\pgfpathlineto{\pgfqpoint{2.194531in}{2.215837in}}%
\pgfpathlineto{\pgfqpoint{2.291471in}{2.158721in}}%
\pgfpathlineto{\pgfqpoint{2.388411in}{1.754234in}}%
\pgfpathlineto{\pgfqpoint{2.485350in}{1.705342in}}%
\pgfpathlineto{\pgfqpoint{2.582290in}{1.570694in}}%
\pgfpathlineto{\pgfqpoint{2.679230in}{1.199262in}}%
\pgfpathlineto{\pgfqpoint{2.776170in}{1.178314in}}%
\pgfpathlineto{\pgfqpoint{2.873110in}{1.151357in}}%
\pgfpathlineto{\pgfqpoint{2.970050in}{1.133581in}}%
\pgfpathlineto{\pgfqpoint{3.066990in}{1.113264in}}%
\pgfpathlineto{\pgfqpoint{3.163930in}{1.082472in}}%
\pgfpathlineto{\pgfqpoint{3.260870in}{1.061238in}}%
\pgfpathlineto{\pgfqpoint{3.357809in}{1.036983in}}%
\pgfpathlineto{\pgfqpoint{3.454749in}{1.021133in}}%
\pgfpathlineto{\pgfqpoint{3.551689in}{0.993372in}}%
\pgfpathlineto{\pgfqpoint{3.648629in}{0.970033in}}%
\pgfpathlineto{\pgfqpoint{3.745569in}{0.948771in}}%
\pgfpathlineto{\pgfqpoint{3.842509in}{0.924641in}}%
\pgfpathlineto{\pgfqpoint{3.939449in}{0.892658in}}%
\pgfpathlineto{\pgfqpoint{4.036389in}{0.883415in}}%
\pgfusepath{stroke}%
\end{pgfscope}%
\begin{pgfscope}%
\pgfpathrectangle{\pgfqpoint{0.740433in}{0.566590in}}{\pgfqpoint{3.295956in}{1.828724in}}%
\pgfusepath{clip}%
\pgfsetbuttcap%
\pgfsetroundjoin%
\definecolor{currentfill}{rgb}{0.000000,0.000000,0.000000}%
\pgfsetfillcolor{currentfill}%
\pgfsetfillopacity{0.000000}%
\pgfsetlinewidth{1.003750pt}%
\definecolor{currentstroke}{rgb}{0.000000,0.447000,0.741000}%
\pgfsetstrokecolor{currentstroke}%
\pgfsetdash{}{0pt}%
\pgfsys@defobject{currentmarker}{\pgfqpoint{-0.041667in}{-0.041667in}}{\pgfqpoint{0.041667in}{0.041667in}}{%
\pgfpathmoveto{\pgfqpoint{0.000000in}{-0.041667in}}%
\pgfpathcurveto{\pgfqpoint{0.011050in}{-0.041667in}}{\pgfqpoint{0.021649in}{-0.037276in}}{\pgfqpoint{0.029463in}{-0.029463in}}%
\pgfpathcurveto{\pgfqpoint{0.037276in}{-0.021649in}}{\pgfqpoint{0.041667in}{-0.011050in}}{\pgfqpoint{0.041667in}{0.000000in}}%
\pgfpathcurveto{\pgfqpoint{0.041667in}{0.011050in}}{\pgfqpoint{0.037276in}{0.021649in}}{\pgfqpoint{0.029463in}{0.029463in}}%
\pgfpathcurveto{\pgfqpoint{0.021649in}{0.037276in}}{\pgfqpoint{0.011050in}{0.041667in}}{\pgfqpoint{0.000000in}{0.041667in}}%
\pgfpathcurveto{\pgfqpoint{-0.011050in}{0.041667in}}{\pgfqpoint{-0.021649in}{0.037276in}}{\pgfqpoint{-0.029463in}{0.029463in}}%
\pgfpathcurveto{\pgfqpoint{-0.037276in}{0.021649in}}{\pgfqpoint{-0.041667in}{0.011050in}}{\pgfqpoint{-0.041667in}{0.000000in}}%
\pgfpathcurveto{\pgfqpoint{-0.041667in}{-0.011050in}}{\pgfqpoint{-0.037276in}{-0.021649in}}{\pgfqpoint{-0.029463in}{-0.029463in}}%
\pgfpathcurveto{\pgfqpoint{-0.021649in}{-0.037276in}}{\pgfqpoint{-0.011050in}{-0.041667in}}{\pgfqpoint{0.000000in}{-0.041667in}}%
\pgfpathclose%
\pgfusepath{stroke,fill}%
}%
\begin{pgfscope}%
\pgfsys@transformshift{0.740433in}{2.284511in}%
\pgfsys@useobject{currentmarker}{}%
\end{pgfscope}%
\begin{pgfscope}%
\pgfsys@transformshift{1.128192in}{2.267535in}%
\pgfsys@useobject{currentmarker}{}%
\end{pgfscope}%
\begin{pgfscope}%
\pgfsys@transformshift{1.515952in}{2.267942in}%
\pgfsys@useobject{currentmarker}{}%
\end{pgfscope}%
\begin{pgfscope}%
\pgfsys@transformshift{1.903711in}{2.272996in}%
\pgfsys@useobject{currentmarker}{}%
\end{pgfscope}%
\begin{pgfscope}%
\pgfsys@transformshift{2.291471in}{2.158721in}%
\pgfsys@useobject{currentmarker}{}%
\end{pgfscope}%
\begin{pgfscope}%
\pgfsys@transformshift{2.679230in}{1.199262in}%
\pgfsys@useobject{currentmarker}{}%
\end{pgfscope}%
\begin{pgfscope}%
\pgfsys@transformshift{3.066990in}{1.113264in}%
\pgfsys@useobject{currentmarker}{}%
\end{pgfscope}%
\begin{pgfscope}%
\pgfsys@transformshift{3.454749in}{1.021133in}%
\pgfsys@useobject{currentmarker}{}%
\end{pgfscope}%
\begin{pgfscope}%
\pgfsys@transformshift{3.842509in}{0.924641in}%
\pgfsys@useobject{currentmarker}{}%
\end{pgfscope}%
\end{pgfscope}%
\begin{pgfscope}%
\pgfpathrectangle{\pgfqpoint{0.740433in}{0.566590in}}{\pgfqpoint{3.295956in}{1.828724in}}%
\pgfusepath{clip}%
\pgfsetbuttcap%
\pgfsetroundjoin%
\pgfsetlinewidth{1.505625pt}%
\definecolor{currentstroke}{rgb}{0.850000,0.324000,0.098000}%
\pgfsetstrokecolor{currentstroke}%
\pgfsetdash{{5.550000pt}{2.400000pt}}{0.000000pt}%
\pgfpathmoveto{\pgfqpoint{0.740433in}{2.272060in}}%
\pgfpathlineto{\pgfqpoint{0.837373in}{2.261842in}}%
\pgfpathlineto{\pgfqpoint{0.934312in}{2.253142in}}%
\pgfpathlineto{\pgfqpoint{1.031252in}{2.237811in}}%
\pgfpathlineto{\pgfqpoint{1.128192in}{2.244753in}}%
\pgfpathlineto{\pgfqpoint{1.225132in}{2.239175in}}%
\pgfpathlineto{\pgfqpoint{1.322072in}{2.230191in}}%
\pgfpathlineto{\pgfqpoint{1.419012in}{2.237617in}}%
\pgfpathlineto{\pgfqpoint{1.515952in}{2.223013in}}%
\pgfpathlineto{\pgfqpoint{1.612892in}{2.242788in}}%
\pgfpathlineto{\pgfqpoint{1.709831in}{2.248087in}}%
\pgfpathlineto{\pgfqpoint{1.806771in}{2.248851in}}%
\pgfpathlineto{\pgfqpoint{1.903711in}{2.233270in}}%
\pgfpathlineto{\pgfqpoint{2.000651in}{2.197704in}}%
\pgfpathlineto{\pgfqpoint{2.097591in}{2.166093in}}%
\pgfpathlineto{\pgfqpoint{2.194531in}{2.186192in}}%
\pgfpathlineto{\pgfqpoint{2.291471in}{2.190500in}}%
\pgfpathlineto{\pgfqpoint{2.388411in}{2.168309in}}%
\pgfpathlineto{\pgfqpoint{2.485350in}{2.120401in}}%
\pgfpathlineto{\pgfqpoint{2.582290in}{1.988838in}}%
\pgfpathlineto{\pgfqpoint{2.679230in}{1.202169in}}%
\pgfpathlineto{\pgfqpoint{2.776170in}{1.178393in}}%
\pgfpathlineto{\pgfqpoint{2.873110in}{1.144137in}}%
\pgfpathlineto{\pgfqpoint{2.970050in}{1.131484in}}%
\pgfpathlineto{\pgfqpoint{3.066990in}{1.107622in}}%
\pgfpathlineto{\pgfqpoint{3.163930in}{1.091691in}}%
\pgfpathlineto{\pgfqpoint{3.260870in}{1.067292in}}%
\pgfpathlineto{\pgfqpoint{3.357809in}{1.023852in}}%
\pgfpathlineto{\pgfqpoint{3.454749in}{1.017746in}}%
\pgfpathlineto{\pgfqpoint{3.551689in}{0.985951in}}%
\pgfpathlineto{\pgfqpoint{3.648629in}{0.968977in}}%
\pgfpathlineto{\pgfqpoint{3.745569in}{0.952130in}}%
\pgfpathlineto{\pgfqpoint{3.842509in}{0.920346in}}%
\pgfpathlineto{\pgfqpoint{3.939449in}{0.897889in}}%
\pgfpathlineto{\pgfqpoint{4.036389in}{0.883754in}}%
\pgfusepath{stroke}%
\end{pgfscope}%
\begin{pgfscope}%
\pgfpathrectangle{\pgfqpoint{0.740433in}{0.566590in}}{\pgfqpoint{3.295956in}{1.828724in}}%
\pgfusepath{clip}%
\pgfsetbuttcap%
\pgfsetroundjoin%
\definecolor{currentfill}{rgb}{0.850000,0.324000,0.098000}%
\pgfsetfillcolor{currentfill}%
\pgfsetlinewidth{1.003750pt}%
\definecolor{currentstroke}{rgb}{0.850000,0.324000,0.098000}%
\pgfsetstrokecolor{currentstroke}%
\pgfsetdash{}{0pt}%
\pgfsys@defobject{currentmarker}{\pgfqpoint{-0.041667in}{-0.041667in}}{\pgfqpoint{0.041667in}{0.041667in}}{%
\pgfpathmoveto{\pgfqpoint{-0.041667in}{0.000000in}}%
\pgfpathlineto{\pgfqpoint{0.041667in}{0.000000in}}%
\pgfpathmoveto{\pgfqpoint{0.000000in}{-0.041667in}}%
\pgfpathlineto{\pgfqpoint{0.000000in}{0.041667in}}%
\pgfusepath{stroke,fill}%
}%
\begin{pgfscope}%
\pgfsys@transformshift{0.740433in}{2.272060in}%
\pgfsys@useobject{currentmarker}{}%
\end{pgfscope}%
\begin{pgfscope}%
\pgfsys@transformshift{1.031252in}{2.237811in}%
\pgfsys@useobject{currentmarker}{}%
\end{pgfscope}%
\begin{pgfscope}%
\pgfsys@transformshift{1.322072in}{2.230191in}%
\pgfsys@useobject{currentmarker}{}%
\end{pgfscope}%
\begin{pgfscope}%
\pgfsys@transformshift{1.612892in}{2.242788in}%
\pgfsys@useobject{currentmarker}{}%
\end{pgfscope}%
\begin{pgfscope}%
\pgfsys@transformshift{1.903711in}{2.233270in}%
\pgfsys@useobject{currentmarker}{}%
\end{pgfscope}%
\begin{pgfscope}%
\pgfsys@transformshift{2.194531in}{2.186192in}%
\pgfsys@useobject{currentmarker}{}%
\end{pgfscope}%
\begin{pgfscope}%
\pgfsys@transformshift{2.485350in}{2.120401in}%
\pgfsys@useobject{currentmarker}{}%
\end{pgfscope}%
\begin{pgfscope}%
\pgfsys@transformshift{2.776170in}{1.178393in}%
\pgfsys@useobject{currentmarker}{}%
\end{pgfscope}%
\begin{pgfscope}%
\pgfsys@transformshift{3.066990in}{1.107622in}%
\pgfsys@useobject{currentmarker}{}%
\end{pgfscope}%
\begin{pgfscope}%
\pgfsys@transformshift{3.357809in}{1.023852in}%
\pgfsys@useobject{currentmarker}{}%
\end{pgfscope}%
\begin{pgfscope}%
\pgfsys@transformshift{3.648629in}{0.968977in}%
\pgfsys@useobject{currentmarker}{}%
\end{pgfscope}%
\begin{pgfscope}%
\pgfsys@transformshift{3.939449in}{0.897889in}%
\pgfsys@useobject{currentmarker}{}%
\end{pgfscope}%
\end{pgfscope}%
\begin{pgfscope}%
\pgfpathrectangle{\pgfqpoint{0.740433in}{0.566590in}}{\pgfqpoint{3.295956in}{1.828724in}}%
\pgfusepath{clip}%
\pgfsetrectcap%
\pgfsetroundjoin%
\pgfsetlinewidth{1.505625pt}%
\definecolor{currentstroke}{rgb}{0.000000,0.447000,0.741000}%
\pgfsetstrokecolor{currentstroke}%
\pgfsetdash{}{0pt}%
\pgfpathmoveto{\pgfqpoint{0.740433in}{1.799141in}}%
\pgfpathlineto{\pgfqpoint{0.975858in}{1.650383in}}%
\pgfpathlineto{\pgfqpoint{1.211283in}{1.296532in}}%
\pgfpathlineto{\pgfqpoint{1.446709in}{1.228167in}}%
\pgfpathlineto{\pgfqpoint{1.682134in}{1.191692in}}%
\pgfpathlineto{\pgfqpoint{1.917560in}{1.121508in}}%
\pgfpathlineto{\pgfqpoint{2.152985in}{1.069054in}}%
\pgfpathlineto{\pgfqpoint{2.388411in}{1.009463in}}%
\pgfpathlineto{\pgfqpoint{2.623836in}{0.953698in}}%
\pgfpathlineto{\pgfqpoint{2.859261in}{0.898341in}}%
\pgfpathlineto{\pgfqpoint{3.094687in}{0.844195in}}%
\pgfpathlineto{\pgfqpoint{3.330112in}{0.794960in}}%
\pgfpathlineto{\pgfqpoint{3.565538in}{0.733032in}}%
\pgfpathlineto{\pgfqpoint{3.800963in}{0.683690in}}%
\pgfpathlineto{\pgfqpoint{4.036389in}{0.624011in}}%
\pgfusepath{stroke}%
\end{pgfscope}%
\begin{pgfscope}%
\pgfpathrectangle{\pgfqpoint{0.740433in}{0.566590in}}{\pgfqpoint{3.295956in}{1.828724in}}%
\pgfusepath{clip}%
\pgfsetbuttcap%
\pgfsetroundjoin%
\definecolor{currentfill}{rgb}{0.000000,0.000000,0.000000}%
\pgfsetfillcolor{currentfill}%
\pgfsetfillopacity{0.000000}%
\pgfsetlinewidth{1.003750pt}%
\definecolor{currentstroke}{rgb}{0.000000,0.447000,0.741000}%
\pgfsetstrokecolor{currentstroke}%
\pgfsetdash{}{0pt}%
\pgfsys@defobject{currentmarker}{\pgfqpoint{-0.041667in}{-0.041667in}}{\pgfqpoint{0.041667in}{0.041667in}}{%
\pgfpathmoveto{\pgfqpoint{0.000000in}{-0.041667in}}%
\pgfpathcurveto{\pgfqpoint{0.011050in}{-0.041667in}}{\pgfqpoint{0.021649in}{-0.037276in}}{\pgfqpoint{0.029463in}{-0.029463in}}%
\pgfpathcurveto{\pgfqpoint{0.037276in}{-0.021649in}}{\pgfqpoint{0.041667in}{-0.011050in}}{\pgfqpoint{0.041667in}{0.000000in}}%
\pgfpathcurveto{\pgfqpoint{0.041667in}{0.011050in}}{\pgfqpoint{0.037276in}{0.021649in}}{\pgfqpoint{0.029463in}{0.029463in}}%
\pgfpathcurveto{\pgfqpoint{0.021649in}{0.037276in}}{\pgfqpoint{0.011050in}{0.041667in}}{\pgfqpoint{0.000000in}{0.041667in}}%
\pgfpathcurveto{\pgfqpoint{-0.011050in}{0.041667in}}{\pgfqpoint{-0.021649in}{0.037276in}}{\pgfqpoint{-0.029463in}{0.029463in}}%
\pgfpathcurveto{\pgfqpoint{-0.037276in}{0.021649in}}{\pgfqpoint{-0.041667in}{0.011050in}}{\pgfqpoint{-0.041667in}{0.000000in}}%
\pgfpathcurveto{\pgfqpoint{-0.041667in}{-0.011050in}}{\pgfqpoint{-0.037276in}{-0.021649in}}{\pgfqpoint{-0.029463in}{-0.029463in}}%
\pgfpathcurveto{\pgfqpoint{-0.021649in}{-0.037276in}}{\pgfqpoint{-0.011050in}{-0.041667in}}{\pgfqpoint{0.000000in}{-0.041667in}}%
\pgfpathclose%
\pgfusepath{stroke,fill}%
}%
\begin{pgfscope}%
\pgfsys@transformshift{0.740433in}{1.799141in}%
\pgfsys@useobject{currentmarker}{}%
\end{pgfscope}%
\begin{pgfscope}%
\pgfsys@transformshift{0.975858in}{1.650383in}%
\pgfsys@useobject{currentmarker}{}%
\end{pgfscope}%
\begin{pgfscope}%
\pgfsys@transformshift{1.211283in}{1.296532in}%
\pgfsys@useobject{currentmarker}{}%
\end{pgfscope}%
\begin{pgfscope}%
\pgfsys@transformshift{1.446709in}{1.228167in}%
\pgfsys@useobject{currentmarker}{}%
\end{pgfscope}%
\begin{pgfscope}%
\pgfsys@transformshift{1.682134in}{1.191692in}%
\pgfsys@useobject{currentmarker}{}%
\end{pgfscope}%
\begin{pgfscope}%
\pgfsys@transformshift{1.917560in}{1.121508in}%
\pgfsys@useobject{currentmarker}{}%
\end{pgfscope}%
\begin{pgfscope}%
\pgfsys@transformshift{2.152985in}{1.069054in}%
\pgfsys@useobject{currentmarker}{}%
\end{pgfscope}%
\begin{pgfscope}%
\pgfsys@transformshift{2.388411in}{1.009463in}%
\pgfsys@useobject{currentmarker}{}%
\end{pgfscope}%
\begin{pgfscope}%
\pgfsys@transformshift{2.623836in}{0.953698in}%
\pgfsys@useobject{currentmarker}{}%
\end{pgfscope}%
\begin{pgfscope}%
\pgfsys@transformshift{2.859261in}{0.898341in}%
\pgfsys@useobject{currentmarker}{}%
\end{pgfscope}%
\begin{pgfscope}%
\pgfsys@transformshift{3.094687in}{0.844195in}%
\pgfsys@useobject{currentmarker}{}%
\end{pgfscope}%
\begin{pgfscope}%
\pgfsys@transformshift{3.330112in}{0.794960in}%
\pgfsys@useobject{currentmarker}{}%
\end{pgfscope}%
\begin{pgfscope}%
\pgfsys@transformshift{3.565538in}{0.733032in}%
\pgfsys@useobject{currentmarker}{}%
\end{pgfscope}%
\begin{pgfscope}%
\pgfsys@transformshift{3.800963in}{0.683690in}%
\pgfsys@useobject{currentmarker}{}%
\end{pgfscope}%
\begin{pgfscope}%
\pgfsys@transformshift{4.036389in}{0.624011in}%
\pgfsys@useobject{currentmarker}{}%
\end{pgfscope}%
\end{pgfscope}%
\begin{pgfscope}%
\pgfpathrectangle{\pgfqpoint{0.740433in}{0.566590in}}{\pgfqpoint{3.295956in}{1.828724in}}%
\pgfusepath{clip}%
\pgfsetrectcap%
\pgfsetroundjoin%
\pgfsetlinewidth{1.505625pt}%
\definecolor{currentstroke}{rgb}{0.850000,0.324000,0.098000}%
\pgfsetstrokecolor{currentstroke}%
\pgfsetdash{}{0pt}%
\pgfpathmoveto{\pgfqpoint{0.740433in}{1.782621in}}%
\pgfpathlineto{\pgfqpoint{0.975858in}{1.696595in}}%
\pgfpathlineto{\pgfqpoint{1.211283in}{1.288599in}}%
\pgfpathlineto{\pgfqpoint{1.446709in}{1.239171in}}%
\pgfpathlineto{\pgfqpoint{1.682134in}{1.189007in}}%
\pgfpathlineto{\pgfqpoint{1.917560in}{1.131828in}}%
\pgfpathlineto{\pgfqpoint{2.152985in}{1.070419in}}%
\pgfpathlineto{\pgfqpoint{2.388411in}{1.016802in}}%
\pgfpathlineto{\pgfqpoint{2.623836in}{0.959684in}}%
\pgfpathlineto{\pgfqpoint{2.859261in}{0.903718in}}%
\pgfpathlineto{\pgfqpoint{3.094687in}{0.835761in}}%
\pgfpathlineto{\pgfqpoint{3.330112in}{0.793200in}}%
\pgfpathlineto{\pgfqpoint{3.565538in}{0.737849in}}%
\pgfpathlineto{\pgfqpoint{3.800963in}{0.684283in}}%
\pgfpathlineto{\pgfqpoint{4.036389in}{0.621254in}}%
\pgfusepath{stroke}%
\end{pgfscope}%
\begin{pgfscope}%
\pgfpathrectangle{\pgfqpoint{0.740433in}{0.566590in}}{\pgfqpoint{3.295956in}{1.828724in}}%
\pgfusepath{clip}%
\pgfsetbuttcap%
\pgfsetroundjoin%
\definecolor{currentfill}{rgb}{0.850000,0.324000,0.098000}%
\pgfsetfillcolor{currentfill}%
\pgfsetlinewidth{1.003750pt}%
\definecolor{currentstroke}{rgb}{0.850000,0.324000,0.098000}%
\pgfsetstrokecolor{currentstroke}%
\pgfsetdash{}{0pt}%
\pgfsys@defobject{currentmarker}{\pgfqpoint{-0.041667in}{-0.041667in}}{\pgfqpoint{0.041667in}{0.041667in}}{%
\pgfpathmoveto{\pgfqpoint{-0.041667in}{0.000000in}}%
\pgfpathlineto{\pgfqpoint{0.041667in}{0.000000in}}%
\pgfpathmoveto{\pgfqpoint{0.000000in}{-0.041667in}}%
\pgfpathlineto{\pgfqpoint{0.000000in}{0.041667in}}%
\pgfusepath{stroke,fill}%
}%
\begin{pgfscope}%
\pgfsys@transformshift{0.740433in}{1.782621in}%
\pgfsys@useobject{currentmarker}{}%
\end{pgfscope}%
\begin{pgfscope}%
\pgfsys@transformshift{0.975858in}{1.696595in}%
\pgfsys@useobject{currentmarker}{}%
\end{pgfscope}%
\begin{pgfscope}%
\pgfsys@transformshift{1.211283in}{1.288599in}%
\pgfsys@useobject{currentmarker}{}%
\end{pgfscope}%
\begin{pgfscope}%
\pgfsys@transformshift{1.446709in}{1.239171in}%
\pgfsys@useobject{currentmarker}{}%
\end{pgfscope}%
\begin{pgfscope}%
\pgfsys@transformshift{1.682134in}{1.189007in}%
\pgfsys@useobject{currentmarker}{}%
\end{pgfscope}%
\begin{pgfscope}%
\pgfsys@transformshift{1.917560in}{1.131828in}%
\pgfsys@useobject{currentmarker}{}%
\end{pgfscope}%
\begin{pgfscope}%
\pgfsys@transformshift{2.152985in}{1.070419in}%
\pgfsys@useobject{currentmarker}{}%
\end{pgfscope}%
\begin{pgfscope}%
\pgfsys@transformshift{2.388411in}{1.016802in}%
\pgfsys@useobject{currentmarker}{}%
\end{pgfscope}%
\begin{pgfscope}%
\pgfsys@transformshift{2.623836in}{0.959684in}%
\pgfsys@useobject{currentmarker}{}%
\end{pgfscope}%
\begin{pgfscope}%
\pgfsys@transformshift{2.859261in}{0.903718in}%
\pgfsys@useobject{currentmarker}{}%
\end{pgfscope}%
\begin{pgfscope}%
\pgfsys@transformshift{3.094687in}{0.835761in}%
\pgfsys@useobject{currentmarker}{}%
\end{pgfscope}%
\begin{pgfscope}%
\pgfsys@transformshift{3.330112in}{0.793200in}%
\pgfsys@useobject{currentmarker}{}%
\end{pgfscope}%
\begin{pgfscope}%
\pgfsys@transformshift{3.565538in}{0.737849in}%
\pgfsys@useobject{currentmarker}{}%
\end{pgfscope}%
\begin{pgfscope}%
\pgfsys@transformshift{3.800963in}{0.684283in}%
\pgfsys@useobject{currentmarker}{}%
\end{pgfscope}%
\begin{pgfscope}%
\pgfsys@transformshift{4.036389in}{0.621254in}%
\pgfsys@useobject{currentmarker}{}%
\end{pgfscope}%
\end{pgfscope}%
\begin{pgfscope}%
\pgfsetrectcap%
\pgfsetmiterjoin%
\pgfsetlinewidth{0.803000pt}%
\definecolor{currentstroke}{rgb}{0.000000,0.000000,0.000000}%
\pgfsetstrokecolor{currentstroke}%
\pgfsetdash{}{0pt}%
\pgfpathmoveto{\pgfqpoint{0.740433in}{0.566590in}}%
\pgfpathlineto{\pgfqpoint{0.740433in}{2.395314in}}%
\pgfusepath{stroke}%
\end{pgfscope}%
\begin{pgfscope}%
\pgfsetrectcap%
\pgfsetmiterjoin%
\pgfsetlinewidth{0.803000pt}%
\definecolor{currentstroke}{rgb}{0.000000,0.000000,0.000000}%
\pgfsetstrokecolor{currentstroke}%
\pgfsetdash{}{0pt}%
\pgfpathmoveto{\pgfqpoint{4.036389in}{0.566590in}}%
\pgfpathlineto{\pgfqpoint{4.036389in}{2.395314in}}%
\pgfusepath{stroke}%
\end{pgfscope}%
\begin{pgfscope}%
\pgfsetrectcap%
\pgfsetmiterjoin%
\pgfsetlinewidth{0.803000pt}%
\definecolor{currentstroke}{rgb}{0.000000,0.000000,0.000000}%
\pgfsetstrokecolor{currentstroke}%
\pgfsetdash{}{0pt}%
\pgfpathmoveto{\pgfqpoint{0.740433in}{0.566590in}}%
\pgfpathlineto{\pgfqpoint{4.036389in}{0.566590in}}%
\pgfusepath{stroke}%
\end{pgfscope}%
\begin{pgfscope}%
\pgfsetrectcap%
\pgfsetmiterjoin%
\pgfsetlinewidth{0.803000pt}%
\definecolor{currentstroke}{rgb}{0.000000,0.000000,0.000000}%
\pgfsetstrokecolor{currentstroke}%
\pgfsetdash{}{0pt}%
\pgfpathmoveto{\pgfqpoint{0.740433in}{2.395314in}}%
\pgfpathlineto{\pgfqpoint{4.036389in}{2.395314in}}%
\pgfusepath{stroke}%
\end{pgfscope}%
\begin{pgfscope}%
\pgfsetbuttcap%
\pgfsetmiterjoin%
\definecolor{currentfill}{rgb}{1.000000,1.000000,1.000000}%
\pgfsetfillcolor{currentfill}%
\pgfsetfillopacity{0.800000}%
\pgfsetlinewidth{1.003750pt}%
\definecolor{currentstroke}{rgb}{0.800000,0.800000,0.800000}%
\pgfsetstrokecolor{currentstroke}%
\pgfsetstrokeopacity{0.800000}%
\pgfsetdash{}{0pt}%
\pgfpathmoveto{\pgfqpoint{2.905744in}{1.946715in}}%
\pgfpathlineto{\pgfqpoint{3.948889in}{1.946715in}}%
\pgfpathquadraticcurveto{\pgfqpoint{3.973889in}{1.946715in}}{\pgfqpoint{3.973889in}{1.971715in}}%
\pgfpathlineto{\pgfqpoint{3.973889in}{2.307814in}}%
\pgfpathquadraticcurveto{\pgfqpoint{3.973889in}{2.332814in}}{\pgfqpoint{3.948889in}{2.332814in}}%
\pgfpathlineto{\pgfqpoint{2.905744in}{2.332814in}}%
\pgfpathquadraticcurveto{\pgfqpoint{2.880744in}{2.332814in}}{\pgfqpoint{2.880744in}{2.307814in}}%
\pgfpathlineto{\pgfqpoint{2.880744in}{1.971715in}}%
\pgfpathquadraticcurveto{\pgfqpoint{2.880744in}{1.946715in}}{\pgfqpoint{2.905744in}{1.946715in}}%
\pgfpathclose%
\pgfusepath{stroke,fill}%
\end{pgfscope}%
\begin{pgfscope}%
\pgfsetbuttcap%
\pgfsetroundjoin%
\definecolor{currentfill}{rgb}{0.000000,0.000000,0.000000}%
\pgfsetfillcolor{currentfill}%
\pgfsetfillopacity{0.000000}%
\pgfsetlinewidth{1.003750pt}%
\definecolor{currentstroke}{rgb}{0.000000,0.447000,0.741000}%
\pgfsetstrokecolor{currentstroke}%
\pgfsetdash{}{0pt}%
\pgfsys@defobject{currentmarker}{\pgfqpoint{-0.041667in}{-0.041667in}}{\pgfqpoint{0.041667in}{0.041667in}}{%
\pgfpathmoveto{\pgfqpoint{0.000000in}{-0.041667in}}%
\pgfpathcurveto{\pgfqpoint{0.011050in}{-0.041667in}}{\pgfqpoint{0.021649in}{-0.037276in}}{\pgfqpoint{0.029463in}{-0.029463in}}%
\pgfpathcurveto{\pgfqpoint{0.037276in}{-0.021649in}}{\pgfqpoint{0.041667in}{-0.011050in}}{\pgfqpoint{0.041667in}{0.000000in}}%
\pgfpathcurveto{\pgfqpoint{0.041667in}{0.011050in}}{\pgfqpoint{0.037276in}{0.021649in}}{\pgfqpoint{0.029463in}{0.029463in}}%
\pgfpathcurveto{\pgfqpoint{0.021649in}{0.037276in}}{\pgfqpoint{0.011050in}{0.041667in}}{\pgfqpoint{0.000000in}{0.041667in}}%
\pgfpathcurveto{\pgfqpoint{-0.011050in}{0.041667in}}{\pgfqpoint{-0.021649in}{0.037276in}}{\pgfqpoint{-0.029463in}{0.029463in}}%
\pgfpathcurveto{\pgfqpoint{-0.037276in}{0.021649in}}{\pgfqpoint{-0.041667in}{0.011050in}}{\pgfqpoint{-0.041667in}{0.000000in}}%
\pgfpathcurveto{\pgfqpoint{-0.041667in}{-0.011050in}}{\pgfqpoint{-0.037276in}{-0.021649in}}{\pgfqpoint{-0.029463in}{-0.029463in}}%
\pgfpathcurveto{\pgfqpoint{-0.021649in}{-0.037276in}}{\pgfqpoint{-0.011050in}{-0.041667in}}{\pgfqpoint{0.000000in}{-0.041667in}}%
\pgfpathclose%
\pgfusepath{stroke,fill}%
}%
\begin{pgfscope}%
\pgfsys@transformshift{3.055744in}{2.239064in}%
\pgfsys@useobject{currentmarker}{}%
\end{pgfscope}%
\end{pgfscope}%
\begin{pgfscope}%
\definecolor{textcolor}{rgb}{0.000000,0.000000,0.000000}%
\pgfsetstrokecolor{textcolor}%
\pgfsetfillcolor{textcolor}%
\pgftext[x=3.280744in,y=2.195314in,left,base]{\color{textcolor}\rmfamily\fontsize{9.000000}{10.800000}\selectfont \(\displaystyle \nu_{12} = \) 14.10}%
\end{pgfscope}%
\begin{pgfscope}%
\pgfsetbuttcap%
\pgfsetroundjoin%
\definecolor{currentfill}{rgb}{0.850000,0.324000,0.098000}%
\pgfsetfillcolor{currentfill}%
\pgfsetlinewidth{1.003750pt}%
\definecolor{currentstroke}{rgb}{0.850000,0.324000,0.098000}%
\pgfsetstrokecolor{currentstroke}%
\pgfsetdash{}{0pt}%
\pgfsys@defobject{currentmarker}{\pgfqpoint{-0.041667in}{-0.041667in}}{\pgfqpoint{0.041667in}{0.041667in}}{%
\pgfpathmoveto{\pgfqpoint{-0.041667in}{0.000000in}}%
\pgfpathlineto{\pgfqpoint{0.041667in}{0.000000in}}%
\pgfpathmoveto{\pgfqpoint{0.000000in}{-0.041667in}}%
\pgfpathlineto{\pgfqpoint{0.000000in}{0.041667in}}%
\pgfusepath{stroke,fill}%
}%
\begin{pgfscope}%
\pgfsys@transformshift{3.055744in}{2.064765in}%
\pgfsys@useobject{currentmarker}{}%
\end{pgfscope}%
\end{pgfscope}%
\begin{pgfscope}%
\definecolor{textcolor}{rgb}{0.000000,0.000000,0.000000}%
\pgfsetstrokecolor{textcolor}%
\pgfsetfillcolor{textcolor}%
\pgftext[x=3.280744in,y=2.021015in,left,base]{\color{textcolor}\rmfamily\fontsize{9.000000}{10.800000}\selectfont \(\displaystyle \nu_{13} = \) 15.81}%
\end{pgfscope}%
\end{pgfpicture}%
\makeatother%
\endgroup%
}
					\caption{Cluster III}
					\label{SubFig:Cluster_III_imag}
				\end{subfigure}
				~
				\begin{subfigure}[h]{0.5\textwidth}
					\centering
					\resizebox{\linewidth}{!}{%% Creator: Matplotlib, PGF backend
%%
%% To include the figure in your LaTeX document, write
%%   \input{<filename>.pgf}
%%
%% Make sure the required packages are loaded in your preamble
%%   \usepackage{pgf}
%%
%% and, on pdftex
%%   \usepackage[utf8]{inputenc}\DeclareUnicodeCharacter{2212}{-}
%%
%% or, on luatex and xetex
%%   \usepackage{unicode-math}
%%
%% Figures using additional raster images can only be included by \input if
%% they are in the same directory as the main LaTeX file. For loading figures
%% from other directories you can use the `import` package
%%   \usepackage{import}
%%
%% and then include the figures with
%%   \import{<path to file>}{<filename>.pgf}
%%
%% Matplotlib used the following preamble
%%   \usepackage[utf8x]{inputenc}
%%   \usepackage[T1]{fontenc}
%%   \usepackage{amsmath,amssymb,amsfonts}
%%
\begingroup%
\makeatletter%
\begin{pgfpicture}%
\pgfpathrectangle{\pgfpointorigin}{\pgfqpoint{4.136389in}{2.495314in}}%
\pgfusepath{use as bounding box, clip}%
\begin{pgfscope}%
\pgfsetbuttcap%
\pgfsetmiterjoin%
\definecolor{currentfill}{rgb}{1.000000,1.000000,1.000000}%
\pgfsetfillcolor{currentfill}%
\pgfsetlinewidth{0.000000pt}%
\definecolor{currentstroke}{rgb}{1.000000,1.000000,1.000000}%
\pgfsetstrokecolor{currentstroke}%
\pgfsetdash{}{0pt}%
\pgfpathmoveto{\pgfqpoint{-0.000000in}{0.000000in}}%
\pgfpathlineto{\pgfqpoint{4.136389in}{0.000000in}}%
\pgfpathlineto{\pgfqpoint{4.136389in}{2.495314in}}%
\pgfpathlineto{\pgfqpoint{-0.000000in}{2.495314in}}%
\pgfpathclose%
\pgfusepath{fill}%
\end{pgfscope}%
\begin{pgfscope}%
\pgfsetbuttcap%
\pgfsetmiterjoin%
\definecolor{currentfill}{rgb}{1.000000,1.000000,1.000000}%
\pgfsetfillcolor{currentfill}%
\pgfsetlinewidth{0.000000pt}%
\definecolor{currentstroke}{rgb}{0.000000,0.000000,0.000000}%
\pgfsetstrokecolor{currentstroke}%
\pgfsetstrokeopacity{0.000000}%
\pgfsetdash{}{0pt}%
\pgfpathmoveto{\pgfqpoint{0.740433in}{0.566590in}}%
\pgfpathlineto{\pgfqpoint{4.036389in}{0.566590in}}%
\pgfpathlineto{\pgfqpoint{4.036389in}{2.395314in}}%
\pgfpathlineto{\pgfqpoint{0.740433in}{2.395314in}}%
\pgfpathclose%
\pgfusepath{fill}%
\end{pgfscope}%
\begin{pgfscope}%
\pgfpathrectangle{\pgfqpoint{0.740433in}{0.566590in}}{\pgfqpoint{3.295956in}{1.828724in}}%
\pgfusepath{clip}%
\pgfsetrectcap%
\pgfsetroundjoin%
\pgfsetlinewidth{0.803000pt}%
\definecolor{currentstroke}{rgb}{0.690196,0.690196,0.690196}%
\pgfsetstrokecolor{currentstroke}%
\pgfsetdash{}{0pt}%
\pgfpathmoveto{\pgfqpoint{0.740433in}{0.566590in}}%
\pgfpathlineto{\pgfqpoint{0.740433in}{2.395314in}}%
\pgfusepath{stroke}%
\end{pgfscope}%
\begin{pgfscope}%
\pgfsetbuttcap%
\pgfsetroundjoin%
\definecolor{currentfill}{rgb}{0.000000,0.000000,0.000000}%
\pgfsetfillcolor{currentfill}%
\pgfsetlinewidth{0.803000pt}%
\definecolor{currentstroke}{rgb}{0.000000,0.000000,0.000000}%
\pgfsetstrokecolor{currentstroke}%
\pgfsetdash{}{0pt}%
\pgfsys@defobject{currentmarker}{\pgfqpoint{0.000000in}{-0.048611in}}{\pgfqpoint{0.000000in}{0.000000in}}{%
\pgfpathmoveto{\pgfqpoint{0.000000in}{0.000000in}}%
\pgfpathlineto{\pgfqpoint{0.000000in}{-0.048611in}}%
\pgfusepath{stroke,fill}%
}%
\begin{pgfscope}%
\pgfsys@transformshift{0.740433in}{0.566590in}%
\pgfsys@useobject{currentmarker}{}%
\end{pgfscope}%
\end{pgfscope}%
\begin{pgfscope}%
\definecolor{textcolor}{rgb}{0.000000,0.000000,0.000000}%
\pgfsetstrokecolor{textcolor}%
\pgfsetfillcolor{textcolor}%
\pgftext[x=0.740433in,y=0.469368in,,top]{\color{textcolor}\rmfamily\fontsize{12.000000}{14.400000}\selectfont \(\displaystyle {-10}\)}%
\end{pgfscope}%
\begin{pgfscope}%
\pgfpathrectangle{\pgfqpoint{0.740433in}{0.566590in}}{\pgfqpoint{3.295956in}{1.828724in}}%
\pgfusepath{clip}%
\pgfsetrectcap%
\pgfsetroundjoin%
\pgfsetlinewidth{0.803000pt}%
\definecolor{currentstroke}{rgb}{0.690196,0.690196,0.690196}%
\pgfsetstrokecolor{currentstroke}%
\pgfsetdash{}{0pt}%
\pgfpathmoveto{\pgfqpoint{1.247503in}{0.566590in}}%
\pgfpathlineto{\pgfqpoint{1.247503in}{2.395314in}}%
\pgfusepath{stroke}%
\end{pgfscope}%
\begin{pgfscope}%
\pgfsetbuttcap%
\pgfsetroundjoin%
\definecolor{currentfill}{rgb}{0.000000,0.000000,0.000000}%
\pgfsetfillcolor{currentfill}%
\pgfsetlinewidth{0.803000pt}%
\definecolor{currentstroke}{rgb}{0.000000,0.000000,0.000000}%
\pgfsetstrokecolor{currentstroke}%
\pgfsetdash{}{0pt}%
\pgfsys@defobject{currentmarker}{\pgfqpoint{0.000000in}{-0.048611in}}{\pgfqpoint{0.000000in}{0.000000in}}{%
\pgfpathmoveto{\pgfqpoint{0.000000in}{0.000000in}}%
\pgfpathlineto{\pgfqpoint{0.000000in}{-0.048611in}}%
\pgfusepath{stroke,fill}%
}%
\begin{pgfscope}%
\pgfsys@transformshift{1.247503in}{0.566590in}%
\pgfsys@useobject{currentmarker}{}%
\end{pgfscope}%
\end{pgfscope}%
\begin{pgfscope}%
\definecolor{textcolor}{rgb}{0.000000,0.000000,0.000000}%
\pgfsetstrokecolor{textcolor}%
\pgfsetfillcolor{textcolor}%
\pgftext[x=1.247503in,y=0.469368in,,top]{\color{textcolor}\rmfamily\fontsize{12.000000}{14.400000}\selectfont \(\displaystyle {0}\)}%
\end{pgfscope}%
\begin{pgfscope}%
\pgfpathrectangle{\pgfqpoint{0.740433in}{0.566590in}}{\pgfqpoint{3.295956in}{1.828724in}}%
\pgfusepath{clip}%
\pgfsetrectcap%
\pgfsetroundjoin%
\pgfsetlinewidth{0.803000pt}%
\definecolor{currentstroke}{rgb}{0.690196,0.690196,0.690196}%
\pgfsetstrokecolor{currentstroke}%
\pgfsetdash{}{0pt}%
\pgfpathmoveto{\pgfqpoint{1.754573in}{0.566590in}}%
\pgfpathlineto{\pgfqpoint{1.754573in}{2.395314in}}%
\pgfusepath{stroke}%
\end{pgfscope}%
\begin{pgfscope}%
\pgfsetbuttcap%
\pgfsetroundjoin%
\definecolor{currentfill}{rgb}{0.000000,0.000000,0.000000}%
\pgfsetfillcolor{currentfill}%
\pgfsetlinewidth{0.803000pt}%
\definecolor{currentstroke}{rgb}{0.000000,0.000000,0.000000}%
\pgfsetstrokecolor{currentstroke}%
\pgfsetdash{}{0pt}%
\pgfsys@defobject{currentmarker}{\pgfqpoint{0.000000in}{-0.048611in}}{\pgfqpoint{0.000000in}{0.000000in}}{%
\pgfpathmoveto{\pgfqpoint{0.000000in}{0.000000in}}%
\pgfpathlineto{\pgfqpoint{0.000000in}{-0.048611in}}%
\pgfusepath{stroke,fill}%
}%
\begin{pgfscope}%
\pgfsys@transformshift{1.754573in}{0.566590in}%
\pgfsys@useobject{currentmarker}{}%
\end{pgfscope}%
\end{pgfscope}%
\begin{pgfscope}%
\definecolor{textcolor}{rgb}{0.000000,0.000000,0.000000}%
\pgfsetstrokecolor{textcolor}%
\pgfsetfillcolor{textcolor}%
\pgftext[x=1.754573in,y=0.469368in,,top]{\color{textcolor}\rmfamily\fontsize{12.000000}{14.400000}\selectfont \(\displaystyle {10}\)}%
\end{pgfscope}%
\begin{pgfscope}%
\pgfpathrectangle{\pgfqpoint{0.740433in}{0.566590in}}{\pgfqpoint{3.295956in}{1.828724in}}%
\pgfusepath{clip}%
\pgfsetrectcap%
\pgfsetroundjoin%
\pgfsetlinewidth{0.803000pt}%
\definecolor{currentstroke}{rgb}{0.690196,0.690196,0.690196}%
\pgfsetstrokecolor{currentstroke}%
\pgfsetdash{}{0pt}%
\pgfpathmoveto{\pgfqpoint{2.261643in}{0.566590in}}%
\pgfpathlineto{\pgfqpoint{2.261643in}{2.395314in}}%
\pgfusepath{stroke}%
\end{pgfscope}%
\begin{pgfscope}%
\pgfsetbuttcap%
\pgfsetroundjoin%
\definecolor{currentfill}{rgb}{0.000000,0.000000,0.000000}%
\pgfsetfillcolor{currentfill}%
\pgfsetlinewidth{0.803000pt}%
\definecolor{currentstroke}{rgb}{0.000000,0.000000,0.000000}%
\pgfsetstrokecolor{currentstroke}%
\pgfsetdash{}{0pt}%
\pgfsys@defobject{currentmarker}{\pgfqpoint{0.000000in}{-0.048611in}}{\pgfqpoint{0.000000in}{0.000000in}}{%
\pgfpathmoveto{\pgfqpoint{0.000000in}{0.000000in}}%
\pgfpathlineto{\pgfqpoint{0.000000in}{-0.048611in}}%
\pgfusepath{stroke,fill}%
}%
\begin{pgfscope}%
\pgfsys@transformshift{2.261643in}{0.566590in}%
\pgfsys@useobject{currentmarker}{}%
\end{pgfscope}%
\end{pgfscope}%
\begin{pgfscope}%
\definecolor{textcolor}{rgb}{0.000000,0.000000,0.000000}%
\pgfsetstrokecolor{textcolor}%
\pgfsetfillcolor{textcolor}%
\pgftext[x=2.261643in,y=0.469368in,,top]{\color{textcolor}\rmfamily\fontsize{12.000000}{14.400000}\selectfont \(\displaystyle {20}\)}%
\end{pgfscope}%
\begin{pgfscope}%
\pgfpathrectangle{\pgfqpoint{0.740433in}{0.566590in}}{\pgfqpoint{3.295956in}{1.828724in}}%
\pgfusepath{clip}%
\pgfsetrectcap%
\pgfsetroundjoin%
\pgfsetlinewidth{0.803000pt}%
\definecolor{currentstroke}{rgb}{0.690196,0.690196,0.690196}%
\pgfsetstrokecolor{currentstroke}%
\pgfsetdash{}{0pt}%
\pgfpathmoveto{\pgfqpoint{2.768713in}{0.566590in}}%
\pgfpathlineto{\pgfqpoint{2.768713in}{2.395314in}}%
\pgfusepath{stroke}%
\end{pgfscope}%
\begin{pgfscope}%
\pgfsetbuttcap%
\pgfsetroundjoin%
\definecolor{currentfill}{rgb}{0.000000,0.000000,0.000000}%
\pgfsetfillcolor{currentfill}%
\pgfsetlinewidth{0.803000pt}%
\definecolor{currentstroke}{rgb}{0.000000,0.000000,0.000000}%
\pgfsetstrokecolor{currentstroke}%
\pgfsetdash{}{0pt}%
\pgfsys@defobject{currentmarker}{\pgfqpoint{0.000000in}{-0.048611in}}{\pgfqpoint{0.000000in}{0.000000in}}{%
\pgfpathmoveto{\pgfqpoint{0.000000in}{0.000000in}}%
\pgfpathlineto{\pgfqpoint{0.000000in}{-0.048611in}}%
\pgfusepath{stroke,fill}%
}%
\begin{pgfscope}%
\pgfsys@transformshift{2.768713in}{0.566590in}%
\pgfsys@useobject{currentmarker}{}%
\end{pgfscope}%
\end{pgfscope}%
\begin{pgfscope}%
\definecolor{textcolor}{rgb}{0.000000,0.000000,0.000000}%
\pgfsetstrokecolor{textcolor}%
\pgfsetfillcolor{textcolor}%
\pgftext[x=2.768713in,y=0.469368in,,top]{\color{textcolor}\rmfamily\fontsize{12.000000}{14.400000}\selectfont \(\displaystyle {30}\)}%
\end{pgfscope}%
\begin{pgfscope}%
\pgfpathrectangle{\pgfqpoint{0.740433in}{0.566590in}}{\pgfqpoint{3.295956in}{1.828724in}}%
\pgfusepath{clip}%
\pgfsetrectcap%
\pgfsetroundjoin%
\pgfsetlinewidth{0.803000pt}%
\definecolor{currentstroke}{rgb}{0.690196,0.690196,0.690196}%
\pgfsetstrokecolor{currentstroke}%
\pgfsetdash{}{0pt}%
\pgfpathmoveto{\pgfqpoint{3.275783in}{0.566590in}}%
\pgfpathlineto{\pgfqpoint{3.275783in}{2.395314in}}%
\pgfusepath{stroke}%
\end{pgfscope}%
\begin{pgfscope}%
\pgfsetbuttcap%
\pgfsetroundjoin%
\definecolor{currentfill}{rgb}{0.000000,0.000000,0.000000}%
\pgfsetfillcolor{currentfill}%
\pgfsetlinewidth{0.803000pt}%
\definecolor{currentstroke}{rgb}{0.000000,0.000000,0.000000}%
\pgfsetstrokecolor{currentstroke}%
\pgfsetdash{}{0pt}%
\pgfsys@defobject{currentmarker}{\pgfqpoint{0.000000in}{-0.048611in}}{\pgfqpoint{0.000000in}{0.000000in}}{%
\pgfpathmoveto{\pgfqpoint{0.000000in}{0.000000in}}%
\pgfpathlineto{\pgfqpoint{0.000000in}{-0.048611in}}%
\pgfusepath{stroke,fill}%
}%
\begin{pgfscope}%
\pgfsys@transformshift{3.275783in}{0.566590in}%
\pgfsys@useobject{currentmarker}{}%
\end{pgfscope}%
\end{pgfscope}%
\begin{pgfscope}%
\definecolor{textcolor}{rgb}{0.000000,0.000000,0.000000}%
\pgfsetstrokecolor{textcolor}%
\pgfsetfillcolor{textcolor}%
\pgftext[x=3.275783in,y=0.469368in,,top]{\color{textcolor}\rmfamily\fontsize{12.000000}{14.400000}\selectfont \(\displaystyle {40}\)}%
\end{pgfscope}%
\begin{pgfscope}%
\pgfpathrectangle{\pgfqpoint{0.740433in}{0.566590in}}{\pgfqpoint{3.295956in}{1.828724in}}%
\pgfusepath{clip}%
\pgfsetrectcap%
\pgfsetroundjoin%
\pgfsetlinewidth{0.803000pt}%
\definecolor{currentstroke}{rgb}{0.690196,0.690196,0.690196}%
\pgfsetstrokecolor{currentstroke}%
\pgfsetdash{}{0pt}%
\pgfpathmoveto{\pgfqpoint{3.782853in}{0.566590in}}%
\pgfpathlineto{\pgfqpoint{3.782853in}{2.395314in}}%
\pgfusepath{stroke}%
\end{pgfscope}%
\begin{pgfscope}%
\pgfsetbuttcap%
\pgfsetroundjoin%
\definecolor{currentfill}{rgb}{0.000000,0.000000,0.000000}%
\pgfsetfillcolor{currentfill}%
\pgfsetlinewidth{0.803000pt}%
\definecolor{currentstroke}{rgb}{0.000000,0.000000,0.000000}%
\pgfsetstrokecolor{currentstroke}%
\pgfsetdash{}{0pt}%
\pgfsys@defobject{currentmarker}{\pgfqpoint{0.000000in}{-0.048611in}}{\pgfqpoint{0.000000in}{0.000000in}}{%
\pgfpathmoveto{\pgfqpoint{0.000000in}{0.000000in}}%
\pgfpathlineto{\pgfqpoint{0.000000in}{-0.048611in}}%
\pgfusepath{stroke,fill}%
}%
\begin{pgfscope}%
\pgfsys@transformshift{3.782853in}{0.566590in}%
\pgfsys@useobject{currentmarker}{}%
\end{pgfscope}%
\end{pgfscope}%
\begin{pgfscope}%
\definecolor{textcolor}{rgb}{0.000000,0.000000,0.000000}%
\pgfsetstrokecolor{textcolor}%
\pgfsetfillcolor{textcolor}%
\pgftext[x=3.782853in,y=0.469368in,,top]{\color{textcolor}\rmfamily\fontsize{12.000000}{14.400000}\selectfont \(\displaystyle {50}\)}%
\end{pgfscope}%
\begin{pgfscope}%
\definecolor{textcolor}{rgb}{0.000000,0.000000,0.000000}%
\pgfsetstrokecolor{textcolor}%
\pgfsetfillcolor{textcolor}%
\pgftext[x=2.388411in,y=0.266626in,,top]{\color{textcolor}\rmfamily\fontsize{12.000000}{14.400000}\selectfont SNR [dB]}%
\end{pgfscope}%
\begin{pgfscope}%
\pgfpathrectangle{\pgfqpoint{0.740433in}{0.566590in}}{\pgfqpoint{3.295956in}{1.828724in}}%
\pgfusepath{clip}%
\pgfsetrectcap%
\pgfsetroundjoin%
\pgfsetlinewidth{0.803000pt}%
\definecolor{currentstroke}{rgb}{0.690196,0.690196,0.690196}%
\pgfsetstrokecolor{currentstroke}%
\pgfsetdash{}{0pt}%
\pgfpathmoveto{\pgfqpoint{0.740433in}{0.566590in}}%
\pgfpathlineto{\pgfqpoint{4.036389in}{0.566590in}}%
\pgfusepath{stroke}%
\end{pgfscope}%
\begin{pgfscope}%
\pgfsetbuttcap%
\pgfsetroundjoin%
\definecolor{currentfill}{rgb}{0.000000,0.000000,0.000000}%
\pgfsetfillcolor{currentfill}%
\pgfsetlinewidth{0.803000pt}%
\definecolor{currentstroke}{rgb}{0.000000,0.000000,0.000000}%
\pgfsetstrokecolor{currentstroke}%
\pgfsetdash{}{0pt}%
\pgfsys@defobject{currentmarker}{\pgfqpoint{-0.048611in}{0.000000in}}{\pgfqpoint{-0.000000in}{0.000000in}}{%
\pgfpathmoveto{\pgfqpoint{-0.000000in}{0.000000in}}%
\pgfpathlineto{\pgfqpoint{-0.048611in}{0.000000in}}%
\pgfusepath{stroke,fill}%
}%
\begin{pgfscope}%
\pgfsys@transformshift{0.740433in}{0.566590in}%
\pgfsys@useobject{currentmarker}{}%
\end{pgfscope}%
\end{pgfscope}%
\begin{pgfscope}%
\definecolor{textcolor}{rgb}{0.000000,0.000000,0.000000}%
\pgfsetstrokecolor{textcolor}%
\pgfsetfillcolor{textcolor}%
\pgftext[x=0.322222in, y=0.509197in, left, base]{\color{textcolor}\rmfamily\fontsize{12.000000}{14.400000}\selectfont \(\displaystyle {10^{-4}}\)}%
\end{pgfscope}%
\begin{pgfscope}%
\pgfpathrectangle{\pgfqpoint{0.740433in}{0.566590in}}{\pgfqpoint{3.295956in}{1.828724in}}%
\pgfusepath{clip}%
\pgfsetrectcap%
\pgfsetroundjoin%
\pgfsetlinewidth{0.803000pt}%
\definecolor{currentstroke}{rgb}{0.690196,0.690196,0.690196}%
\pgfsetstrokecolor{currentstroke}%
\pgfsetdash{}{0pt}%
\pgfpathmoveto{\pgfqpoint{0.740433in}{1.104776in}}%
\pgfpathlineto{\pgfqpoint{4.036389in}{1.104776in}}%
\pgfusepath{stroke}%
\end{pgfscope}%
\begin{pgfscope}%
\pgfsetbuttcap%
\pgfsetroundjoin%
\definecolor{currentfill}{rgb}{0.000000,0.000000,0.000000}%
\pgfsetfillcolor{currentfill}%
\pgfsetlinewidth{0.803000pt}%
\definecolor{currentstroke}{rgb}{0.000000,0.000000,0.000000}%
\pgfsetstrokecolor{currentstroke}%
\pgfsetdash{}{0pt}%
\pgfsys@defobject{currentmarker}{\pgfqpoint{-0.048611in}{0.000000in}}{\pgfqpoint{-0.000000in}{0.000000in}}{%
\pgfpathmoveto{\pgfqpoint{-0.000000in}{0.000000in}}%
\pgfpathlineto{\pgfqpoint{-0.048611in}{0.000000in}}%
\pgfusepath{stroke,fill}%
}%
\begin{pgfscope}%
\pgfsys@transformshift{0.740433in}{1.104776in}%
\pgfsys@useobject{currentmarker}{}%
\end{pgfscope}%
\end{pgfscope}%
\begin{pgfscope}%
\definecolor{textcolor}{rgb}{0.000000,0.000000,0.000000}%
\pgfsetstrokecolor{textcolor}%
\pgfsetfillcolor{textcolor}%
\pgftext[x=0.322222in, y=1.047383in, left, base]{\color{textcolor}\rmfamily\fontsize{12.000000}{14.400000}\selectfont \(\displaystyle {10^{-2}}\)}%
\end{pgfscope}%
\begin{pgfscope}%
\pgfpathrectangle{\pgfqpoint{0.740433in}{0.566590in}}{\pgfqpoint{3.295956in}{1.828724in}}%
\pgfusepath{clip}%
\pgfsetrectcap%
\pgfsetroundjoin%
\pgfsetlinewidth{0.803000pt}%
\definecolor{currentstroke}{rgb}{0.690196,0.690196,0.690196}%
\pgfsetstrokecolor{currentstroke}%
\pgfsetdash{}{0pt}%
\pgfpathmoveto{\pgfqpoint{0.740433in}{1.642962in}}%
\pgfpathlineto{\pgfqpoint{4.036389in}{1.642962in}}%
\pgfusepath{stroke}%
\end{pgfscope}%
\begin{pgfscope}%
\pgfsetbuttcap%
\pgfsetroundjoin%
\definecolor{currentfill}{rgb}{0.000000,0.000000,0.000000}%
\pgfsetfillcolor{currentfill}%
\pgfsetlinewidth{0.803000pt}%
\definecolor{currentstroke}{rgb}{0.000000,0.000000,0.000000}%
\pgfsetstrokecolor{currentstroke}%
\pgfsetdash{}{0pt}%
\pgfsys@defobject{currentmarker}{\pgfqpoint{-0.048611in}{0.000000in}}{\pgfqpoint{-0.000000in}{0.000000in}}{%
\pgfpathmoveto{\pgfqpoint{-0.000000in}{0.000000in}}%
\pgfpathlineto{\pgfqpoint{-0.048611in}{0.000000in}}%
\pgfusepath{stroke,fill}%
}%
\begin{pgfscope}%
\pgfsys@transformshift{0.740433in}{1.642962in}%
\pgfsys@useobject{currentmarker}{}%
\end{pgfscope}%
\end{pgfscope}%
\begin{pgfscope}%
\definecolor{textcolor}{rgb}{0.000000,0.000000,0.000000}%
\pgfsetstrokecolor{textcolor}%
\pgfsetfillcolor{textcolor}%
\pgftext[x=0.414045in, y=1.585569in, left, base]{\color{textcolor}\rmfamily\fontsize{12.000000}{14.400000}\selectfont \(\displaystyle {10^{0}}\)}%
\end{pgfscope}%
\begin{pgfscope}%
\pgfpathrectangle{\pgfqpoint{0.740433in}{0.566590in}}{\pgfqpoint{3.295956in}{1.828724in}}%
\pgfusepath{clip}%
\pgfsetrectcap%
\pgfsetroundjoin%
\pgfsetlinewidth{0.803000pt}%
\definecolor{currentstroke}{rgb}{0.690196,0.690196,0.690196}%
\pgfsetstrokecolor{currentstroke}%
\pgfsetdash{}{0pt}%
\pgfpathmoveto{\pgfqpoint{0.740433in}{2.181148in}}%
\pgfpathlineto{\pgfqpoint{4.036389in}{2.181148in}}%
\pgfusepath{stroke}%
\end{pgfscope}%
\begin{pgfscope}%
\pgfsetbuttcap%
\pgfsetroundjoin%
\definecolor{currentfill}{rgb}{0.000000,0.000000,0.000000}%
\pgfsetfillcolor{currentfill}%
\pgfsetlinewidth{0.803000pt}%
\definecolor{currentstroke}{rgb}{0.000000,0.000000,0.000000}%
\pgfsetstrokecolor{currentstroke}%
\pgfsetdash{}{0pt}%
\pgfsys@defobject{currentmarker}{\pgfqpoint{-0.048611in}{0.000000in}}{\pgfqpoint{-0.000000in}{0.000000in}}{%
\pgfpathmoveto{\pgfqpoint{-0.000000in}{0.000000in}}%
\pgfpathlineto{\pgfqpoint{-0.048611in}{0.000000in}}%
\pgfusepath{stroke,fill}%
}%
\begin{pgfscope}%
\pgfsys@transformshift{0.740433in}{2.181148in}%
\pgfsys@useobject{currentmarker}{}%
\end{pgfscope}%
\end{pgfscope}%
\begin{pgfscope}%
\definecolor{textcolor}{rgb}{0.000000,0.000000,0.000000}%
\pgfsetstrokecolor{textcolor}%
\pgfsetfillcolor{textcolor}%
\pgftext[x=0.414045in, y=2.123755in, left, base]{\color{textcolor}\rmfamily\fontsize{12.000000}{14.400000}\selectfont \(\displaystyle {10^{2}}\)}%
\end{pgfscope}%
\begin{pgfscope}%
\definecolor{textcolor}{rgb}{0.000000,0.000000,0.000000}%
\pgfsetstrokecolor{textcolor}%
\pgfsetfillcolor{textcolor}%
\pgftext[x=0.266667in,y=1.480952in,,bottom,rotate=90.000000]{\color{textcolor}\rmfamily\fontsize{12.000000}{14.400000}\selectfont \(\displaystyle \hat{\sigma}_{\nu}(\mathrm{SNR})\)}%
\end{pgfscope}%
\begin{pgfscope}%
\pgfpathrectangle{\pgfqpoint{0.740433in}{0.566590in}}{\pgfqpoint{3.295956in}{1.828724in}}%
\pgfusepath{clip}%
\pgfsetbuttcap%
\pgfsetroundjoin%
\pgfsetlinewidth{1.505625pt}%
\definecolor{currentstroke}{rgb}{0.000000,0.447000,0.741000}%
\pgfsetstrokecolor{currentstroke}%
\pgfsetdash{{5.550000pt}{2.400000pt}}{0.000000pt}%
\pgfpathmoveto{\pgfqpoint{0.740433in}{2.278729in}}%
\pgfpathlineto{\pgfqpoint{0.837373in}{2.270934in}}%
\pgfpathlineto{\pgfqpoint{0.934312in}{2.266282in}}%
\pgfpathlineto{\pgfqpoint{1.031252in}{2.258334in}}%
\pgfpathlineto{\pgfqpoint{1.128192in}{2.268199in}}%
\pgfpathlineto{\pgfqpoint{1.225132in}{2.257250in}}%
\pgfpathlineto{\pgfqpoint{1.322072in}{2.249284in}}%
\pgfpathlineto{\pgfqpoint{1.419012in}{2.258670in}}%
\pgfpathlineto{\pgfqpoint{1.515952in}{2.254608in}}%
\pgfpathlineto{\pgfqpoint{1.612892in}{2.266364in}}%
\pgfpathlineto{\pgfqpoint{1.709831in}{2.258780in}}%
\pgfpathlineto{\pgfqpoint{1.806771in}{2.257161in}}%
\pgfpathlineto{\pgfqpoint{1.903711in}{2.249177in}}%
\pgfpathlineto{\pgfqpoint{2.000651in}{2.218106in}}%
\pgfpathlineto{\pgfqpoint{2.097591in}{2.193097in}}%
\pgfpathlineto{\pgfqpoint{2.194531in}{2.169948in}}%
\pgfpathlineto{\pgfqpoint{2.291471in}{2.097094in}}%
\pgfpathlineto{\pgfqpoint{2.388411in}{1.720955in}}%
\pgfpathlineto{\pgfqpoint{2.485350in}{1.663163in}}%
\pgfpathlineto{\pgfqpoint{2.582290in}{1.520896in}}%
\pgfpathlineto{\pgfqpoint{2.679230in}{1.294755in}}%
\pgfpathlineto{\pgfqpoint{2.776170in}{1.276039in}}%
\pgfpathlineto{\pgfqpoint{2.873110in}{1.260284in}}%
\pgfpathlineto{\pgfqpoint{2.970050in}{1.229646in}}%
\pgfpathlineto{\pgfqpoint{3.066990in}{1.199071in}}%
\pgfpathlineto{\pgfqpoint{3.163930in}{1.176279in}}%
\pgfpathlineto{\pgfqpoint{3.260870in}{1.150245in}}%
\pgfpathlineto{\pgfqpoint{3.357809in}{1.136269in}}%
\pgfpathlineto{\pgfqpoint{3.454749in}{1.108120in}}%
\pgfpathlineto{\pgfqpoint{3.551689in}{1.077315in}}%
\pgfpathlineto{\pgfqpoint{3.648629in}{1.045887in}}%
\pgfpathlineto{\pgfqpoint{3.745569in}{1.023587in}}%
\pgfpathlineto{\pgfqpoint{3.842509in}{0.996950in}}%
\pgfpathlineto{\pgfqpoint{3.939449in}{0.969639in}}%
\pgfpathlineto{\pgfqpoint{4.036389in}{0.944093in}}%
\pgfusepath{stroke}%
\end{pgfscope}%
\begin{pgfscope}%
\pgfpathrectangle{\pgfqpoint{0.740433in}{0.566590in}}{\pgfqpoint{3.295956in}{1.828724in}}%
\pgfusepath{clip}%
\pgfsetbuttcap%
\pgfsetroundjoin%
\definecolor{currentfill}{rgb}{0.000000,0.000000,0.000000}%
\pgfsetfillcolor{currentfill}%
\pgfsetfillopacity{0.000000}%
\pgfsetlinewidth{1.003750pt}%
\definecolor{currentstroke}{rgb}{0.000000,0.447000,0.741000}%
\pgfsetstrokecolor{currentstroke}%
\pgfsetdash{}{0pt}%
\pgfsys@defobject{currentmarker}{\pgfqpoint{-0.041667in}{-0.041667in}}{\pgfqpoint{0.041667in}{0.041667in}}{%
\pgfpathmoveto{\pgfqpoint{0.000000in}{-0.041667in}}%
\pgfpathcurveto{\pgfqpoint{0.011050in}{-0.041667in}}{\pgfqpoint{0.021649in}{-0.037276in}}{\pgfqpoint{0.029463in}{-0.029463in}}%
\pgfpathcurveto{\pgfqpoint{0.037276in}{-0.021649in}}{\pgfqpoint{0.041667in}{-0.011050in}}{\pgfqpoint{0.041667in}{0.000000in}}%
\pgfpathcurveto{\pgfqpoint{0.041667in}{0.011050in}}{\pgfqpoint{0.037276in}{0.021649in}}{\pgfqpoint{0.029463in}{0.029463in}}%
\pgfpathcurveto{\pgfqpoint{0.021649in}{0.037276in}}{\pgfqpoint{0.011050in}{0.041667in}}{\pgfqpoint{0.000000in}{0.041667in}}%
\pgfpathcurveto{\pgfqpoint{-0.011050in}{0.041667in}}{\pgfqpoint{-0.021649in}{0.037276in}}{\pgfqpoint{-0.029463in}{0.029463in}}%
\pgfpathcurveto{\pgfqpoint{-0.037276in}{0.021649in}}{\pgfqpoint{-0.041667in}{0.011050in}}{\pgfqpoint{-0.041667in}{0.000000in}}%
\pgfpathcurveto{\pgfqpoint{-0.041667in}{-0.011050in}}{\pgfqpoint{-0.037276in}{-0.021649in}}{\pgfqpoint{-0.029463in}{-0.029463in}}%
\pgfpathcurveto{\pgfqpoint{-0.021649in}{-0.037276in}}{\pgfqpoint{-0.011050in}{-0.041667in}}{\pgfqpoint{0.000000in}{-0.041667in}}%
\pgfpathclose%
\pgfusepath{stroke,fill}%
}%
\begin{pgfscope}%
\pgfsys@transformshift{0.740433in}{2.278729in}%
\pgfsys@useobject{currentmarker}{}%
\end{pgfscope}%
\begin{pgfscope}%
\pgfsys@transformshift{1.128192in}{2.268199in}%
\pgfsys@useobject{currentmarker}{}%
\end{pgfscope}%
\begin{pgfscope}%
\pgfsys@transformshift{1.515952in}{2.254608in}%
\pgfsys@useobject{currentmarker}{}%
\end{pgfscope}%
\begin{pgfscope}%
\pgfsys@transformshift{1.903711in}{2.249177in}%
\pgfsys@useobject{currentmarker}{}%
\end{pgfscope}%
\begin{pgfscope}%
\pgfsys@transformshift{2.291471in}{2.097094in}%
\pgfsys@useobject{currentmarker}{}%
\end{pgfscope}%
\begin{pgfscope}%
\pgfsys@transformshift{2.679230in}{1.294755in}%
\pgfsys@useobject{currentmarker}{}%
\end{pgfscope}%
\begin{pgfscope}%
\pgfsys@transformshift{3.066990in}{1.199071in}%
\pgfsys@useobject{currentmarker}{}%
\end{pgfscope}%
\begin{pgfscope}%
\pgfsys@transformshift{3.454749in}{1.108120in}%
\pgfsys@useobject{currentmarker}{}%
\end{pgfscope}%
\begin{pgfscope}%
\pgfsys@transformshift{3.842509in}{0.996950in}%
\pgfsys@useobject{currentmarker}{}%
\end{pgfscope}%
\end{pgfscope}%
\begin{pgfscope}%
\pgfpathrectangle{\pgfqpoint{0.740433in}{0.566590in}}{\pgfqpoint{3.295956in}{1.828724in}}%
\pgfusepath{clip}%
\pgfsetbuttcap%
\pgfsetroundjoin%
\pgfsetlinewidth{1.505625pt}%
\definecolor{currentstroke}{rgb}{0.850000,0.324000,0.098000}%
\pgfsetstrokecolor{currentstroke}%
\pgfsetdash{{5.550000pt}{2.400000pt}}{0.000000pt}%
\pgfpathmoveto{\pgfqpoint{0.740433in}{2.252427in}}%
\pgfpathlineto{\pgfqpoint{0.837373in}{2.240918in}}%
\pgfpathlineto{\pgfqpoint{0.934312in}{2.228718in}}%
\pgfpathlineto{\pgfqpoint{1.031252in}{2.227637in}}%
\pgfpathlineto{\pgfqpoint{1.128192in}{2.230973in}}%
\pgfpathlineto{\pgfqpoint{1.225132in}{2.211561in}}%
\pgfpathlineto{\pgfqpoint{1.322072in}{2.202842in}}%
\pgfpathlineto{\pgfqpoint{1.419012in}{2.215183in}}%
\pgfpathlineto{\pgfqpoint{1.515952in}{2.210312in}}%
\pgfpathlineto{\pgfqpoint{1.612892in}{2.222214in}}%
\pgfpathlineto{\pgfqpoint{1.709831in}{2.215589in}}%
\pgfpathlineto{\pgfqpoint{1.806771in}{2.210726in}}%
\pgfpathlineto{\pgfqpoint{1.903711in}{2.187317in}}%
\pgfpathlineto{\pgfqpoint{2.000651in}{2.150122in}}%
\pgfpathlineto{\pgfqpoint{2.097591in}{2.071303in}}%
\pgfpathlineto{\pgfqpoint{2.194531in}{1.782425in}}%
\pgfpathlineto{\pgfqpoint{2.291471in}{1.757441in}}%
\pgfpathlineto{\pgfqpoint{2.388411in}{1.703100in}}%
\pgfpathlineto{\pgfqpoint{2.485350in}{1.650559in}}%
\pgfpathlineto{\pgfqpoint{2.582290in}{1.514332in}}%
\pgfpathlineto{\pgfqpoint{2.679230in}{1.325860in}}%
\pgfpathlineto{\pgfqpoint{2.776170in}{1.302255in}}%
\pgfpathlineto{\pgfqpoint{2.873110in}{1.268664in}}%
\pgfpathlineto{\pgfqpoint{2.970050in}{1.241718in}}%
\pgfpathlineto{\pgfqpoint{3.066990in}{1.218245in}}%
\pgfpathlineto{\pgfqpoint{3.163930in}{1.193891in}}%
\pgfpathlineto{\pgfqpoint{3.260870in}{1.150561in}}%
\pgfpathlineto{\pgfqpoint{3.357809in}{1.145081in}}%
\pgfpathlineto{\pgfqpoint{3.454749in}{1.112306in}}%
\pgfpathlineto{\pgfqpoint{3.551689in}{1.094022in}}%
\pgfpathlineto{\pgfqpoint{3.648629in}{1.066822in}}%
\pgfpathlineto{\pgfqpoint{3.745569in}{1.037311in}}%
\pgfpathlineto{\pgfqpoint{3.842509in}{1.011909in}}%
\pgfpathlineto{\pgfqpoint{3.939449in}{0.977341in}}%
\pgfpathlineto{\pgfqpoint{4.036389in}{0.960632in}}%
\pgfusepath{stroke}%
\end{pgfscope}%
\begin{pgfscope}%
\pgfpathrectangle{\pgfqpoint{0.740433in}{0.566590in}}{\pgfqpoint{3.295956in}{1.828724in}}%
\pgfusepath{clip}%
\pgfsetbuttcap%
\pgfsetroundjoin%
\definecolor{currentfill}{rgb}{0.850000,0.324000,0.098000}%
\pgfsetfillcolor{currentfill}%
\pgfsetlinewidth{1.003750pt}%
\definecolor{currentstroke}{rgb}{0.850000,0.324000,0.098000}%
\pgfsetstrokecolor{currentstroke}%
\pgfsetdash{}{0pt}%
\pgfsys@defobject{currentmarker}{\pgfqpoint{-0.041667in}{-0.041667in}}{\pgfqpoint{0.041667in}{0.041667in}}{%
\pgfpathmoveto{\pgfqpoint{-0.041667in}{0.000000in}}%
\pgfpathlineto{\pgfqpoint{0.041667in}{0.000000in}}%
\pgfpathmoveto{\pgfqpoint{0.000000in}{-0.041667in}}%
\pgfpathlineto{\pgfqpoint{0.000000in}{0.041667in}}%
\pgfusepath{stroke,fill}%
}%
\begin{pgfscope}%
\pgfsys@transformshift{0.740433in}{2.252427in}%
\pgfsys@useobject{currentmarker}{}%
\end{pgfscope}%
\begin{pgfscope}%
\pgfsys@transformshift{1.031252in}{2.227637in}%
\pgfsys@useobject{currentmarker}{}%
\end{pgfscope}%
\begin{pgfscope}%
\pgfsys@transformshift{1.322072in}{2.202842in}%
\pgfsys@useobject{currentmarker}{}%
\end{pgfscope}%
\begin{pgfscope}%
\pgfsys@transformshift{1.612892in}{2.222214in}%
\pgfsys@useobject{currentmarker}{}%
\end{pgfscope}%
\begin{pgfscope}%
\pgfsys@transformshift{1.903711in}{2.187317in}%
\pgfsys@useobject{currentmarker}{}%
\end{pgfscope}%
\begin{pgfscope}%
\pgfsys@transformshift{2.194531in}{1.782425in}%
\pgfsys@useobject{currentmarker}{}%
\end{pgfscope}%
\begin{pgfscope}%
\pgfsys@transformshift{2.485350in}{1.650559in}%
\pgfsys@useobject{currentmarker}{}%
\end{pgfscope}%
\begin{pgfscope}%
\pgfsys@transformshift{2.776170in}{1.302255in}%
\pgfsys@useobject{currentmarker}{}%
\end{pgfscope}%
\begin{pgfscope}%
\pgfsys@transformshift{3.066990in}{1.218245in}%
\pgfsys@useobject{currentmarker}{}%
\end{pgfscope}%
\begin{pgfscope}%
\pgfsys@transformshift{3.357809in}{1.145081in}%
\pgfsys@useobject{currentmarker}{}%
\end{pgfscope}%
\begin{pgfscope}%
\pgfsys@transformshift{3.648629in}{1.066822in}%
\pgfsys@useobject{currentmarker}{}%
\end{pgfscope}%
\begin{pgfscope}%
\pgfsys@transformshift{3.939449in}{0.977341in}%
\pgfsys@useobject{currentmarker}{}%
\end{pgfscope}%
\end{pgfscope}%
\begin{pgfscope}%
\pgfpathrectangle{\pgfqpoint{0.740433in}{0.566590in}}{\pgfqpoint{3.295956in}{1.828724in}}%
\pgfusepath{clip}%
\pgfsetbuttcap%
\pgfsetroundjoin%
\pgfsetlinewidth{1.505625pt}%
\definecolor{currentstroke}{rgb}{0.000000,0.500000,0.000000}%
\pgfsetstrokecolor{currentstroke}%
\pgfsetdash{{5.550000pt}{2.400000pt}}{0.000000pt}%
\pgfpathmoveto{\pgfqpoint{0.740433in}{2.242253in}}%
\pgfpathlineto{\pgfqpoint{0.837373in}{2.229555in}}%
\pgfpathlineto{\pgfqpoint{0.934312in}{2.209567in}}%
\pgfpathlineto{\pgfqpoint{1.031252in}{2.207527in}}%
\pgfpathlineto{\pgfqpoint{1.128192in}{2.191833in}}%
\pgfpathlineto{\pgfqpoint{1.225132in}{2.179881in}}%
\pgfpathlineto{\pgfqpoint{1.322072in}{2.162183in}}%
\pgfpathlineto{\pgfqpoint{1.419012in}{2.182180in}}%
\pgfpathlineto{\pgfqpoint{1.515952in}{2.154754in}}%
\pgfpathlineto{\pgfqpoint{1.612892in}{2.177437in}}%
\pgfpathlineto{\pgfqpoint{1.709831in}{2.163220in}}%
\pgfpathlineto{\pgfqpoint{1.806771in}{2.164119in}}%
\pgfpathlineto{\pgfqpoint{1.903711in}{2.136884in}}%
\pgfpathlineto{\pgfqpoint{2.000651in}{2.034767in}}%
\pgfpathlineto{\pgfqpoint{2.097591in}{1.903485in}}%
\pgfpathlineto{\pgfqpoint{2.194531in}{1.827192in}}%
\pgfpathlineto{\pgfqpoint{2.291471in}{1.772619in}}%
\pgfpathlineto{\pgfqpoint{2.388411in}{1.658197in}}%
\pgfpathlineto{\pgfqpoint{2.485350in}{1.604677in}}%
\pgfpathlineto{\pgfqpoint{2.582290in}{1.472104in}}%
\pgfpathlineto{\pgfqpoint{2.679230in}{1.333816in}}%
\pgfpathlineto{\pgfqpoint{2.776170in}{1.320154in}}%
\pgfpathlineto{\pgfqpoint{2.873110in}{1.296374in}}%
\pgfpathlineto{\pgfqpoint{2.970050in}{1.249567in}}%
\pgfpathlineto{\pgfqpoint{3.066990in}{1.228030in}}%
\pgfpathlineto{\pgfqpoint{3.163930in}{1.211406in}}%
\pgfpathlineto{\pgfqpoint{3.260870in}{1.189751in}}%
\pgfpathlineto{\pgfqpoint{3.357809in}{1.153781in}}%
\pgfpathlineto{\pgfqpoint{3.454749in}{1.148717in}}%
\pgfpathlineto{\pgfqpoint{3.551689in}{1.113225in}}%
\pgfpathlineto{\pgfqpoint{3.648629in}{1.092849in}}%
\pgfpathlineto{\pgfqpoint{3.745569in}{1.052911in}}%
\pgfpathlineto{\pgfqpoint{3.842509in}{1.029725in}}%
\pgfpathlineto{\pgfqpoint{3.939449in}{1.012241in}}%
\pgfpathlineto{\pgfqpoint{4.036389in}{0.985998in}}%
\pgfusepath{stroke}%
\end{pgfscope}%
\begin{pgfscope}%
\pgfpathrectangle{\pgfqpoint{0.740433in}{0.566590in}}{\pgfqpoint{3.295956in}{1.828724in}}%
\pgfusepath{clip}%
\pgfsetbuttcap%
\pgfsetmiterjoin%
\definecolor{currentfill}{rgb}{0.000000,0.000000,0.000000}%
\pgfsetfillcolor{currentfill}%
\pgfsetfillopacity{0.000000}%
\pgfsetlinewidth{1.003750pt}%
\definecolor{currentstroke}{rgb}{0.000000,0.500000,0.000000}%
\pgfsetstrokecolor{currentstroke}%
\pgfsetdash{}{0pt}%
\pgfsys@defobject{currentmarker}{\pgfqpoint{-0.041667in}{-0.041667in}}{\pgfqpoint{0.041667in}{0.041667in}}{%
\pgfpathmoveto{\pgfqpoint{-0.041667in}{-0.041667in}}%
\pgfpathlineto{\pgfqpoint{0.041667in}{-0.041667in}}%
\pgfpathlineto{\pgfqpoint{0.041667in}{0.041667in}}%
\pgfpathlineto{\pgfqpoint{-0.041667in}{0.041667in}}%
\pgfpathclose%
\pgfusepath{stroke,fill}%
}%
\begin{pgfscope}%
\pgfsys@transformshift{0.740433in}{2.242253in}%
\pgfsys@useobject{currentmarker}{}%
\end{pgfscope}%
\begin{pgfscope}%
\pgfsys@transformshift{1.225132in}{2.179881in}%
\pgfsys@useobject{currentmarker}{}%
\end{pgfscope}%
\begin{pgfscope}%
\pgfsys@transformshift{1.709831in}{2.163220in}%
\pgfsys@useobject{currentmarker}{}%
\end{pgfscope}%
\begin{pgfscope}%
\pgfsys@transformshift{2.194531in}{1.827192in}%
\pgfsys@useobject{currentmarker}{}%
\end{pgfscope}%
\begin{pgfscope}%
\pgfsys@transformshift{2.679230in}{1.333816in}%
\pgfsys@useobject{currentmarker}{}%
\end{pgfscope}%
\begin{pgfscope}%
\pgfsys@transformshift{3.163930in}{1.211406in}%
\pgfsys@useobject{currentmarker}{}%
\end{pgfscope}%
\begin{pgfscope}%
\pgfsys@transformshift{3.648629in}{1.092849in}%
\pgfsys@useobject{currentmarker}{}%
\end{pgfscope}%
\end{pgfscope}%
\begin{pgfscope}%
\pgfpathrectangle{\pgfqpoint{0.740433in}{0.566590in}}{\pgfqpoint{3.295956in}{1.828724in}}%
\pgfusepath{clip}%
\pgfsetbuttcap%
\pgfsetroundjoin%
\pgfsetlinewidth{1.505625pt}%
\definecolor{currentstroke}{rgb}{0.494000,0.184000,0.556000}%
\pgfsetstrokecolor{currentstroke}%
\pgfsetdash{{5.550000pt}{2.400000pt}}{0.000000pt}%
\pgfpathmoveto{\pgfqpoint{0.740433in}{2.260685in}}%
\pgfpathlineto{\pgfqpoint{0.837373in}{2.248451in}}%
\pgfpathlineto{\pgfqpoint{0.934312in}{2.221331in}}%
\pgfpathlineto{\pgfqpoint{1.031252in}{2.213588in}}%
\pgfpathlineto{\pgfqpoint{1.128192in}{2.197606in}}%
\pgfpathlineto{\pgfqpoint{1.225132in}{2.188004in}}%
\pgfpathlineto{\pgfqpoint{1.322072in}{2.163931in}}%
\pgfpathlineto{\pgfqpoint{1.419012in}{2.168948in}}%
\pgfpathlineto{\pgfqpoint{1.515952in}{2.138353in}}%
\pgfpathlineto{\pgfqpoint{1.612892in}{2.133021in}}%
\pgfpathlineto{\pgfqpoint{1.709831in}{2.138182in}}%
\pgfpathlineto{\pgfqpoint{1.806771in}{2.100975in}}%
\pgfpathlineto{\pgfqpoint{1.903711in}{2.049234in}}%
\pgfpathlineto{\pgfqpoint{2.000651in}{2.029258in}}%
\pgfpathlineto{\pgfqpoint{2.097591in}{1.970031in}}%
\pgfpathlineto{\pgfqpoint{2.194531in}{1.942374in}}%
\pgfpathlineto{\pgfqpoint{2.291471in}{1.921766in}}%
\pgfpathlineto{\pgfqpoint{2.388411in}{1.878332in}}%
\pgfpathlineto{\pgfqpoint{2.485350in}{1.823715in}}%
\pgfpathlineto{\pgfqpoint{2.582290in}{1.620160in}}%
\pgfpathlineto{\pgfqpoint{2.679230in}{1.356881in}}%
\pgfpathlineto{\pgfqpoint{2.776170in}{1.333987in}}%
\pgfpathlineto{\pgfqpoint{2.873110in}{1.312042in}}%
\pgfpathlineto{\pgfqpoint{2.970050in}{1.283372in}}%
\pgfpathlineto{\pgfqpoint{3.066990in}{1.258491in}}%
\pgfpathlineto{\pgfqpoint{3.163930in}{1.229470in}}%
\pgfpathlineto{\pgfqpoint{3.260870in}{1.209182in}}%
\pgfpathlineto{\pgfqpoint{3.357809in}{1.182892in}}%
\pgfpathlineto{\pgfqpoint{3.454749in}{1.152283in}}%
\pgfpathlineto{\pgfqpoint{3.551689in}{1.115335in}}%
\pgfpathlineto{\pgfqpoint{3.648629in}{1.101574in}}%
\pgfpathlineto{\pgfqpoint{3.745569in}{1.065748in}}%
\pgfpathlineto{\pgfqpoint{3.842509in}{1.054959in}}%
\pgfpathlineto{\pgfqpoint{3.939449in}{1.024906in}}%
\pgfpathlineto{\pgfqpoint{4.036389in}{1.006297in}}%
\pgfusepath{stroke}%
\end{pgfscope}%
\begin{pgfscope}%
\pgfpathrectangle{\pgfqpoint{0.740433in}{0.566590in}}{\pgfqpoint{3.295956in}{1.828724in}}%
\pgfusepath{clip}%
\pgfsetbuttcap%
\pgfsetroundjoin%
\definecolor{currentfill}{rgb}{0.494000,0.184000,0.556000}%
\pgfsetfillcolor{currentfill}%
\pgfsetlinewidth{1.003750pt}%
\definecolor{currentstroke}{rgb}{0.494000,0.184000,0.556000}%
\pgfsetstrokecolor{currentstroke}%
\pgfsetdash{}{0pt}%
\pgfsys@defobject{currentmarker}{\pgfqpoint{-0.041667in}{-0.041667in}}{\pgfqpoint{0.041667in}{0.041667in}}{%
\pgfpathmoveto{\pgfqpoint{-0.041667in}{-0.041667in}}%
\pgfpathlineto{\pgfqpoint{0.041667in}{0.041667in}}%
\pgfpathmoveto{\pgfqpoint{-0.041667in}{0.041667in}}%
\pgfpathlineto{\pgfqpoint{0.041667in}{-0.041667in}}%
\pgfusepath{stroke,fill}%
}%
\begin{pgfscope}%
\pgfsys@transformshift{0.740433in}{2.260685in}%
\pgfsys@useobject{currentmarker}{}%
\end{pgfscope}%
\begin{pgfscope}%
\pgfsys@transformshift{1.128192in}{2.197606in}%
\pgfsys@useobject{currentmarker}{}%
\end{pgfscope}%
\begin{pgfscope}%
\pgfsys@transformshift{1.515952in}{2.138353in}%
\pgfsys@useobject{currentmarker}{}%
\end{pgfscope}%
\begin{pgfscope}%
\pgfsys@transformshift{1.903711in}{2.049234in}%
\pgfsys@useobject{currentmarker}{}%
\end{pgfscope}%
\begin{pgfscope}%
\pgfsys@transformshift{2.291471in}{1.921766in}%
\pgfsys@useobject{currentmarker}{}%
\end{pgfscope}%
\begin{pgfscope}%
\pgfsys@transformshift{2.679230in}{1.356881in}%
\pgfsys@useobject{currentmarker}{}%
\end{pgfscope}%
\begin{pgfscope}%
\pgfsys@transformshift{3.066990in}{1.258491in}%
\pgfsys@useobject{currentmarker}{}%
\end{pgfscope}%
\begin{pgfscope}%
\pgfsys@transformshift{3.454749in}{1.152283in}%
\pgfsys@useobject{currentmarker}{}%
\end{pgfscope}%
\begin{pgfscope}%
\pgfsys@transformshift{3.842509in}{1.054959in}%
\pgfsys@useobject{currentmarker}{}%
\end{pgfscope}%
\end{pgfscope}%
\begin{pgfscope}%
\pgfpathrectangle{\pgfqpoint{0.740433in}{0.566590in}}{\pgfqpoint{3.295956in}{1.828724in}}%
\pgfusepath{clip}%
\pgfsetrectcap%
\pgfsetroundjoin%
\pgfsetlinewidth{1.505625pt}%
\definecolor{currentstroke}{rgb}{0.000000,0.447000,0.741000}%
\pgfsetstrokecolor{currentstroke}%
\pgfsetdash{}{0pt}%
\pgfpathmoveto{\pgfqpoint{0.740433in}{1.772047in}}%
\pgfpathlineto{\pgfqpoint{0.975858in}{1.752086in}}%
\pgfpathlineto{\pgfqpoint{1.211283in}{1.741885in}}%
\pgfpathlineto{\pgfqpoint{1.446709in}{1.699580in}}%
\pgfpathlineto{\pgfqpoint{1.682134in}{1.634635in}}%
\pgfpathlineto{\pgfqpoint{1.917560in}{1.600236in}}%
\pgfpathlineto{\pgfqpoint{2.152985in}{1.566438in}}%
\pgfpathlineto{\pgfqpoint{2.388411in}{1.420590in}}%
\pgfpathlineto{\pgfqpoint{2.623836in}{1.176104in}}%
\pgfpathlineto{\pgfqpoint{2.859261in}{1.108570in}}%
\pgfpathlineto{\pgfqpoint{3.094687in}{1.050130in}}%
\pgfpathlineto{\pgfqpoint{3.330112in}{0.977124in}}%
\pgfpathlineto{\pgfqpoint{3.565538in}{0.913785in}}%
\pgfpathlineto{\pgfqpoint{3.800963in}{0.849039in}}%
\pgfpathlineto{\pgfqpoint{4.036389in}{0.785886in}}%
\pgfusepath{stroke}%
\end{pgfscope}%
\begin{pgfscope}%
\pgfpathrectangle{\pgfqpoint{0.740433in}{0.566590in}}{\pgfqpoint{3.295956in}{1.828724in}}%
\pgfusepath{clip}%
\pgfsetbuttcap%
\pgfsetroundjoin%
\definecolor{currentfill}{rgb}{0.000000,0.000000,0.000000}%
\pgfsetfillcolor{currentfill}%
\pgfsetfillopacity{0.000000}%
\pgfsetlinewidth{1.003750pt}%
\definecolor{currentstroke}{rgb}{0.000000,0.447000,0.741000}%
\pgfsetstrokecolor{currentstroke}%
\pgfsetdash{}{0pt}%
\pgfsys@defobject{currentmarker}{\pgfqpoint{-0.041667in}{-0.041667in}}{\pgfqpoint{0.041667in}{0.041667in}}{%
\pgfpathmoveto{\pgfqpoint{0.000000in}{-0.041667in}}%
\pgfpathcurveto{\pgfqpoint{0.011050in}{-0.041667in}}{\pgfqpoint{0.021649in}{-0.037276in}}{\pgfqpoint{0.029463in}{-0.029463in}}%
\pgfpathcurveto{\pgfqpoint{0.037276in}{-0.021649in}}{\pgfqpoint{0.041667in}{-0.011050in}}{\pgfqpoint{0.041667in}{0.000000in}}%
\pgfpathcurveto{\pgfqpoint{0.041667in}{0.011050in}}{\pgfqpoint{0.037276in}{0.021649in}}{\pgfqpoint{0.029463in}{0.029463in}}%
\pgfpathcurveto{\pgfqpoint{0.021649in}{0.037276in}}{\pgfqpoint{0.011050in}{0.041667in}}{\pgfqpoint{0.000000in}{0.041667in}}%
\pgfpathcurveto{\pgfqpoint{-0.011050in}{0.041667in}}{\pgfqpoint{-0.021649in}{0.037276in}}{\pgfqpoint{-0.029463in}{0.029463in}}%
\pgfpathcurveto{\pgfqpoint{-0.037276in}{0.021649in}}{\pgfqpoint{-0.041667in}{0.011050in}}{\pgfqpoint{-0.041667in}{0.000000in}}%
\pgfpathcurveto{\pgfqpoint{-0.041667in}{-0.011050in}}{\pgfqpoint{-0.037276in}{-0.021649in}}{\pgfqpoint{-0.029463in}{-0.029463in}}%
\pgfpathcurveto{\pgfqpoint{-0.021649in}{-0.037276in}}{\pgfqpoint{-0.011050in}{-0.041667in}}{\pgfqpoint{0.000000in}{-0.041667in}}%
\pgfpathclose%
\pgfusepath{stroke,fill}%
}%
\begin{pgfscope}%
\pgfsys@transformshift{0.740433in}{1.772047in}%
\pgfsys@useobject{currentmarker}{}%
\end{pgfscope}%
\begin{pgfscope}%
\pgfsys@transformshift{0.975858in}{1.752086in}%
\pgfsys@useobject{currentmarker}{}%
\end{pgfscope}%
\begin{pgfscope}%
\pgfsys@transformshift{1.211283in}{1.741885in}%
\pgfsys@useobject{currentmarker}{}%
\end{pgfscope}%
\begin{pgfscope}%
\pgfsys@transformshift{1.446709in}{1.699580in}%
\pgfsys@useobject{currentmarker}{}%
\end{pgfscope}%
\begin{pgfscope}%
\pgfsys@transformshift{1.682134in}{1.634635in}%
\pgfsys@useobject{currentmarker}{}%
\end{pgfscope}%
\begin{pgfscope}%
\pgfsys@transformshift{1.917560in}{1.600236in}%
\pgfsys@useobject{currentmarker}{}%
\end{pgfscope}%
\begin{pgfscope}%
\pgfsys@transformshift{2.152985in}{1.566438in}%
\pgfsys@useobject{currentmarker}{}%
\end{pgfscope}%
\begin{pgfscope}%
\pgfsys@transformshift{2.388411in}{1.420590in}%
\pgfsys@useobject{currentmarker}{}%
\end{pgfscope}%
\begin{pgfscope}%
\pgfsys@transformshift{2.623836in}{1.176104in}%
\pgfsys@useobject{currentmarker}{}%
\end{pgfscope}%
\begin{pgfscope}%
\pgfsys@transformshift{2.859261in}{1.108570in}%
\pgfsys@useobject{currentmarker}{}%
\end{pgfscope}%
\begin{pgfscope}%
\pgfsys@transformshift{3.094687in}{1.050130in}%
\pgfsys@useobject{currentmarker}{}%
\end{pgfscope}%
\begin{pgfscope}%
\pgfsys@transformshift{3.330112in}{0.977124in}%
\pgfsys@useobject{currentmarker}{}%
\end{pgfscope}%
\begin{pgfscope}%
\pgfsys@transformshift{3.565538in}{0.913785in}%
\pgfsys@useobject{currentmarker}{}%
\end{pgfscope}%
\begin{pgfscope}%
\pgfsys@transformshift{3.800963in}{0.849039in}%
\pgfsys@useobject{currentmarker}{}%
\end{pgfscope}%
\begin{pgfscope}%
\pgfsys@transformshift{4.036389in}{0.785886in}%
\pgfsys@useobject{currentmarker}{}%
\end{pgfscope}%
\end{pgfscope}%
\begin{pgfscope}%
\pgfpathrectangle{\pgfqpoint{0.740433in}{0.566590in}}{\pgfqpoint{3.295956in}{1.828724in}}%
\pgfusepath{clip}%
\pgfsetrectcap%
\pgfsetroundjoin%
\pgfsetlinewidth{1.505625pt}%
\definecolor{currentstroke}{rgb}{0.850000,0.324000,0.098000}%
\pgfsetstrokecolor{currentstroke}%
\pgfsetdash{}{0pt}%
\pgfpathmoveto{\pgfqpoint{0.740433in}{1.591784in}}%
\pgfpathlineto{\pgfqpoint{0.975858in}{1.589830in}}%
\pgfpathlineto{\pgfqpoint{1.211283in}{1.528176in}}%
\pgfpathlineto{\pgfqpoint{1.446709in}{1.462632in}}%
\pgfpathlineto{\pgfqpoint{1.682134in}{1.425726in}}%
\pgfpathlineto{\pgfqpoint{1.917560in}{1.377352in}}%
\pgfpathlineto{\pgfqpoint{2.152985in}{1.336995in}}%
\pgfpathlineto{\pgfqpoint{2.388411in}{1.222360in}}%
\pgfpathlineto{\pgfqpoint{2.623836in}{1.142660in}}%
\pgfpathlineto{\pgfqpoint{2.859261in}{1.084401in}}%
\pgfpathlineto{\pgfqpoint{3.094687in}{1.021534in}}%
\pgfpathlineto{\pgfqpoint{3.330112in}{0.948451in}}%
\pgfpathlineto{\pgfqpoint{3.565538in}{0.890106in}}%
\pgfpathlineto{\pgfqpoint{3.800963in}{0.828628in}}%
\pgfpathlineto{\pgfqpoint{4.036389in}{0.759752in}}%
\pgfusepath{stroke}%
\end{pgfscope}%
\begin{pgfscope}%
\pgfpathrectangle{\pgfqpoint{0.740433in}{0.566590in}}{\pgfqpoint{3.295956in}{1.828724in}}%
\pgfusepath{clip}%
\pgfsetbuttcap%
\pgfsetroundjoin%
\definecolor{currentfill}{rgb}{0.850000,0.324000,0.098000}%
\pgfsetfillcolor{currentfill}%
\pgfsetlinewidth{1.003750pt}%
\definecolor{currentstroke}{rgb}{0.850000,0.324000,0.098000}%
\pgfsetstrokecolor{currentstroke}%
\pgfsetdash{}{0pt}%
\pgfsys@defobject{currentmarker}{\pgfqpoint{-0.041667in}{-0.041667in}}{\pgfqpoint{0.041667in}{0.041667in}}{%
\pgfpathmoveto{\pgfqpoint{-0.041667in}{0.000000in}}%
\pgfpathlineto{\pgfqpoint{0.041667in}{0.000000in}}%
\pgfpathmoveto{\pgfqpoint{0.000000in}{-0.041667in}}%
\pgfpathlineto{\pgfqpoint{0.000000in}{0.041667in}}%
\pgfusepath{stroke,fill}%
}%
\begin{pgfscope}%
\pgfsys@transformshift{0.740433in}{1.591784in}%
\pgfsys@useobject{currentmarker}{}%
\end{pgfscope}%
\begin{pgfscope}%
\pgfsys@transformshift{0.975858in}{1.589830in}%
\pgfsys@useobject{currentmarker}{}%
\end{pgfscope}%
\begin{pgfscope}%
\pgfsys@transformshift{1.211283in}{1.528176in}%
\pgfsys@useobject{currentmarker}{}%
\end{pgfscope}%
\begin{pgfscope}%
\pgfsys@transformshift{1.446709in}{1.462632in}%
\pgfsys@useobject{currentmarker}{}%
\end{pgfscope}%
\begin{pgfscope}%
\pgfsys@transformshift{1.682134in}{1.425726in}%
\pgfsys@useobject{currentmarker}{}%
\end{pgfscope}%
\begin{pgfscope}%
\pgfsys@transformshift{1.917560in}{1.377352in}%
\pgfsys@useobject{currentmarker}{}%
\end{pgfscope}%
\begin{pgfscope}%
\pgfsys@transformshift{2.152985in}{1.336995in}%
\pgfsys@useobject{currentmarker}{}%
\end{pgfscope}%
\begin{pgfscope}%
\pgfsys@transformshift{2.388411in}{1.222360in}%
\pgfsys@useobject{currentmarker}{}%
\end{pgfscope}%
\begin{pgfscope}%
\pgfsys@transformshift{2.623836in}{1.142660in}%
\pgfsys@useobject{currentmarker}{}%
\end{pgfscope}%
\begin{pgfscope}%
\pgfsys@transformshift{2.859261in}{1.084401in}%
\pgfsys@useobject{currentmarker}{}%
\end{pgfscope}%
\begin{pgfscope}%
\pgfsys@transformshift{3.094687in}{1.021534in}%
\pgfsys@useobject{currentmarker}{}%
\end{pgfscope}%
\begin{pgfscope}%
\pgfsys@transformshift{3.330112in}{0.948451in}%
\pgfsys@useobject{currentmarker}{}%
\end{pgfscope}%
\begin{pgfscope}%
\pgfsys@transformshift{3.565538in}{0.890106in}%
\pgfsys@useobject{currentmarker}{}%
\end{pgfscope}%
\begin{pgfscope}%
\pgfsys@transformshift{3.800963in}{0.828628in}%
\pgfsys@useobject{currentmarker}{}%
\end{pgfscope}%
\begin{pgfscope}%
\pgfsys@transformshift{4.036389in}{0.759752in}%
\pgfsys@useobject{currentmarker}{}%
\end{pgfscope}%
\end{pgfscope}%
\begin{pgfscope}%
\pgfpathrectangle{\pgfqpoint{0.740433in}{0.566590in}}{\pgfqpoint{3.295956in}{1.828724in}}%
\pgfusepath{clip}%
\pgfsetrectcap%
\pgfsetroundjoin%
\pgfsetlinewidth{1.505625pt}%
\definecolor{currentstroke}{rgb}{0.000000,0.500000,0.000000}%
\pgfsetstrokecolor{currentstroke}%
\pgfsetdash{}{0pt}%
\pgfpathmoveto{\pgfqpoint{0.740433in}{1.650997in}}%
\pgfpathlineto{\pgfqpoint{0.975858in}{1.646074in}}%
\pgfpathlineto{\pgfqpoint{1.211283in}{1.618306in}}%
\pgfpathlineto{\pgfqpoint{1.446709in}{1.550898in}}%
\pgfpathlineto{\pgfqpoint{1.682134in}{1.442282in}}%
\pgfpathlineto{\pgfqpoint{1.917560in}{1.380612in}}%
\pgfpathlineto{\pgfqpoint{2.152985in}{1.293610in}}%
\pgfpathlineto{\pgfqpoint{2.388411in}{1.205527in}}%
\pgfpathlineto{\pgfqpoint{2.623836in}{1.124141in}}%
\pgfpathlineto{\pgfqpoint{2.859261in}{1.058815in}}%
\pgfpathlineto{\pgfqpoint{3.094687in}{0.995397in}}%
\pgfpathlineto{\pgfqpoint{3.330112in}{0.922399in}}%
\pgfpathlineto{\pgfqpoint{3.565538in}{0.864301in}}%
\pgfpathlineto{\pgfqpoint{3.800963in}{0.803083in}}%
\pgfpathlineto{\pgfqpoint{4.036389in}{0.733476in}}%
\pgfusepath{stroke}%
\end{pgfscope}%
\begin{pgfscope}%
\pgfpathrectangle{\pgfqpoint{0.740433in}{0.566590in}}{\pgfqpoint{3.295956in}{1.828724in}}%
\pgfusepath{clip}%
\pgfsetbuttcap%
\pgfsetmiterjoin%
\definecolor{currentfill}{rgb}{0.000000,0.000000,0.000000}%
\pgfsetfillcolor{currentfill}%
\pgfsetfillopacity{0.000000}%
\pgfsetlinewidth{1.003750pt}%
\definecolor{currentstroke}{rgb}{0.000000,0.500000,0.000000}%
\pgfsetstrokecolor{currentstroke}%
\pgfsetdash{}{0pt}%
\pgfsys@defobject{currentmarker}{\pgfqpoint{-0.041667in}{-0.041667in}}{\pgfqpoint{0.041667in}{0.041667in}}{%
\pgfpathmoveto{\pgfqpoint{-0.041667in}{-0.041667in}}%
\pgfpathlineto{\pgfqpoint{0.041667in}{-0.041667in}}%
\pgfpathlineto{\pgfqpoint{0.041667in}{0.041667in}}%
\pgfpathlineto{\pgfqpoint{-0.041667in}{0.041667in}}%
\pgfpathclose%
\pgfusepath{stroke,fill}%
}%
\begin{pgfscope}%
\pgfsys@transformshift{0.740433in}{1.650997in}%
\pgfsys@useobject{currentmarker}{}%
\end{pgfscope}%
\begin{pgfscope}%
\pgfsys@transformshift{0.975858in}{1.646074in}%
\pgfsys@useobject{currentmarker}{}%
\end{pgfscope}%
\begin{pgfscope}%
\pgfsys@transformshift{1.211283in}{1.618306in}%
\pgfsys@useobject{currentmarker}{}%
\end{pgfscope}%
\begin{pgfscope}%
\pgfsys@transformshift{1.446709in}{1.550898in}%
\pgfsys@useobject{currentmarker}{}%
\end{pgfscope}%
\begin{pgfscope}%
\pgfsys@transformshift{1.682134in}{1.442282in}%
\pgfsys@useobject{currentmarker}{}%
\end{pgfscope}%
\begin{pgfscope}%
\pgfsys@transformshift{1.917560in}{1.380612in}%
\pgfsys@useobject{currentmarker}{}%
\end{pgfscope}%
\begin{pgfscope}%
\pgfsys@transformshift{2.152985in}{1.293610in}%
\pgfsys@useobject{currentmarker}{}%
\end{pgfscope}%
\begin{pgfscope}%
\pgfsys@transformshift{2.388411in}{1.205527in}%
\pgfsys@useobject{currentmarker}{}%
\end{pgfscope}%
\begin{pgfscope}%
\pgfsys@transformshift{2.623836in}{1.124141in}%
\pgfsys@useobject{currentmarker}{}%
\end{pgfscope}%
\begin{pgfscope}%
\pgfsys@transformshift{2.859261in}{1.058815in}%
\pgfsys@useobject{currentmarker}{}%
\end{pgfscope}%
\begin{pgfscope}%
\pgfsys@transformshift{3.094687in}{0.995397in}%
\pgfsys@useobject{currentmarker}{}%
\end{pgfscope}%
\begin{pgfscope}%
\pgfsys@transformshift{3.330112in}{0.922399in}%
\pgfsys@useobject{currentmarker}{}%
\end{pgfscope}%
\begin{pgfscope}%
\pgfsys@transformshift{3.565538in}{0.864301in}%
\pgfsys@useobject{currentmarker}{}%
\end{pgfscope}%
\begin{pgfscope}%
\pgfsys@transformshift{3.800963in}{0.803083in}%
\pgfsys@useobject{currentmarker}{}%
\end{pgfscope}%
\begin{pgfscope}%
\pgfsys@transformshift{4.036389in}{0.733476in}%
\pgfsys@useobject{currentmarker}{}%
\end{pgfscope}%
\end{pgfscope}%
\begin{pgfscope}%
\pgfpathrectangle{\pgfqpoint{0.740433in}{0.566590in}}{\pgfqpoint{3.295956in}{1.828724in}}%
\pgfusepath{clip}%
\pgfsetrectcap%
\pgfsetroundjoin%
\pgfsetlinewidth{1.505625pt}%
\definecolor{currentstroke}{rgb}{0.494000,0.184000,0.556000}%
\pgfsetstrokecolor{currentstroke}%
\pgfsetdash{}{0pt}%
\pgfpathmoveto{\pgfqpoint{0.740433in}{1.714086in}}%
\pgfpathlineto{\pgfqpoint{0.975858in}{1.710118in}}%
\pgfpathlineto{\pgfqpoint{1.211283in}{1.690757in}}%
\pgfpathlineto{\pgfqpoint{1.446709in}{1.665582in}}%
\pgfpathlineto{\pgfqpoint{1.682134in}{1.569576in}}%
\pgfpathlineto{\pgfqpoint{1.917560in}{1.436667in}}%
\pgfpathlineto{\pgfqpoint{2.152985in}{1.359636in}}%
\pgfpathlineto{\pgfqpoint{2.388411in}{1.259759in}}%
\pgfpathlineto{\pgfqpoint{2.623836in}{1.178127in}}%
\pgfpathlineto{\pgfqpoint{2.859261in}{1.115093in}}%
\pgfpathlineto{\pgfqpoint{3.094687in}{1.048193in}}%
\pgfpathlineto{\pgfqpoint{3.330112in}{0.976055in}}%
\pgfpathlineto{\pgfqpoint{3.565538in}{0.915731in}}%
\pgfpathlineto{\pgfqpoint{3.800963in}{0.857466in}}%
\pgfpathlineto{\pgfqpoint{4.036389in}{0.791699in}}%
\pgfusepath{stroke}%
\end{pgfscope}%
\begin{pgfscope}%
\pgfpathrectangle{\pgfqpoint{0.740433in}{0.566590in}}{\pgfqpoint{3.295956in}{1.828724in}}%
\pgfusepath{clip}%
\pgfsetbuttcap%
\pgfsetroundjoin%
\definecolor{currentfill}{rgb}{0.494000,0.184000,0.556000}%
\pgfsetfillcolor{currentfill}%
\pgfsetlinewidth{1.003750pt}%
\definecolor{currentstroke}{rgb}{0.494000,0.184000,0.556000}%
\pgfsetstrokecolor{currentstroke}%
\pgfsetdash{}{0pt}%
\pgfsys@defobject{currentmarker}{\pgfqpoint{-0.041667in}{-0.041667in}}{\pgfqpoint{0.041667in}{0.041667in}}{%
\pgfpathmoveto{\pgfqpoint{-0.041667in}{-0.041667in}}%
\pgfpathlineto{\pgfqpoint{0.041667in}{0.041667in}}%
\pgfpathmoveto{\pgfqpoint{-0.041667in}{0.041667in}}%
\pgfpathlineto{\pgfqpoint{0.041667in}{-0.041667in}}%
\pgfusepath{stroke,fill}%
}%
\begin{pgfscope}%
\pgfsys@transformshift{0.740433in}{1.714086in}%
\pgfsys@useobject{currentmarker}{}%
\end{pgfscope}%
\begin{pgfscope}%
\pgfsys@transformshift{0.975858in}{1.710118in}%
\pgfsys@useobject{currentmarker}{}%
\end{pgfscope}%
\begin{pgfscope}%
\pgfsys@transformshift{1.211283in}{1.690757in}%
\pgfsys@useobject{currentmarker}{}%
\end{pgfscope}%
\begin{pgfscope}%
\pgfsys@transformshift{1.446709in}{1.665582in}%
\pgfsys@useobject{currentmarker}{}%
\end{pgfscope}%
\begin{pgfscope}%
\pgfsys@transformshift{1.682134in}{1.569576in}%
\pgfsys@useobject{currentmarker}{}%
\end{pgfscope}%
\begin{pgfscope}%
\pgfsys@transformshift{1.917560in}{1.436667in}%
\pgfsys@useobject{currentmarker}{}%
\end{pgfscope}%
\begin{pgfscope}%
\pgfsys@transformshift{2.152985in}{1.359636in}%
\pgfsys@useobject{currentmarker}{}%
\end{pgfscope}%
\begin{pgfscope}%
\pgfsys@transformshift{2.388411in}{1.259759in}%
\pgfsys@useobject{currentmarker}{}%
\end{pgfscope}%
\begin{pgfscope}%
\pgfsys@transformshift{2.623836in}{1.178127in}%
\pgfsys@useobject{currentmarker}{}%
\end{pgfscope}%
\begin{pgfscope}%
\pgfsys@transformshift{2.859261in}{1.115093in}%
\pgfsys@useobject{currentmarker}{}%
\end{pgfscope}%
\begin{pgfscope}%
\pgfsys@transformshift{3.094687in}{1.048193in}%
\pgfsys@useobject{currentmarker}{}%
\end{pgfscope}%
\begin{pgfscope}%
\pgfsys@transformshift{3.330112in}{0.976055in}%
\pgfsys@useobject{currentmarker}{}%
\end{pgfscope}%
\begin{pgfscope}%
\pgfsys@transformshift{3.565538in}{0.915731in}%
\pgfsys@useobject{currentmarker}{}%
\end{pgfscope}%
\begin{pgfscope}%
\pgfsys@transformshift{3.800963in}{0.857466in}%
\pgfsys@useobject{currentmarker}{}%
\end{pgfscope}%
\begin{pgfscope}%
\pgfsys@transformshift{4.036389in}{0.791699in}%
\pgfsys@useobject{currentmarker}{}%
\end{pgfscope}%
\end{pgfscope}%
\begin{pgfscope}%
\pgfsetrectcap%
\pgfsetmiterjoin%
\pgfsetlinewidth{0.803000pt}%
\definecolor{currentstroke}{rgb}{0.000000,0.000000,0.000000}%
\pgfsetstrokecolor{currentstroke}%
\pgfsetdash{}{0pt}%
\pgfpathmoveto{\pgfqpoint{0.740433in}{0.566590in}}%
\pgfpathlineto{\pgfqpoint{0.740433in}{2.395314in}}%
\pgfusepath{stroke}%
\end{pgfscope}%
\begin{pgfscope}%
\pgfsetrectcap%
\pgfsetmiterjoin%
\pgfsetlinewidth{0.803000pt}%
\definecolor{currentstroke}{rgb}{0.000000,0.000000,0.000000}%
\pgfsetstrokecolor{currentstroke}%
\pgfsetdash{}{0pt}%
\pgfpathmoveto{\pgfqpoint{4.036389in}{0.566590in}}%
\pgfpathlineto{\pgfqpoint{4.036389in}{2.395314in}}%
\pgfusepath{stroke}%
\end{pgfscope}%
\begin{pgfscope}%
\pgfsetrectcap%
\pgfsetmiterjoin%
\pgfsetlinewidth{0.803000pt}%
\definecolor{currentstroke}{rgb}{0.000000,0.000000,0.000000}%
\pgfsetstrokecolor{currentstroke}%
\pgfsetdash{}{0pt}%
\pgfpathmoveto{\pgfqpoint{0.740433in}{0.566590in}}%
\pgfpathlineto{\pgfqpoint{4.036389in}{0.566590in}}%
\pgfusepath{stroke}%
\end{pgfscope}%
\begin{pgfscope}%
\pgfsetrectcap%
\pgfsetmiterjoin%
\pgfsetlinewidth{0.803000pt}%
\definecolor{currentstroke}{rgb}{0.000000,0.000000,0.000000}%
\pgfsetstrokecolor{currentstroke}%
\pgfsetdash{}{0pt}%
\pgfpathmoveto{\pgfqpoint{0.740433in}{2.395314in}}%
\pgfpathlineto{\pgfqpoint{4.036389in}{2.395314in}}%
\pgfusepath{stroke}%
\end{pgfscope}%
\begin{pgfscope}%
\pgfsetbuttcap%
\pgfsetmiterjoin%
\definecolor{currentfill}{rgb}{1.000000,1.000000,1.000000}%
\pgfsetfillcolor{currentfill}%
\pgfsetfillopacity{0.800000}%
\pgfsetlinewidth{1.003750pt}%
\definecolor{currentstroke}{rgb}{0.800000,0.800000,0.800000}%
\pgfsetstrokecolor{currentstroke}%
\pgfsetstrokeopacity{0.800000}%
\pgfsetdash{}{0pt}%
\pgfpathmoveto{\pgfqpoint{2.841524in}{1.598116in}}%
\pgfpathlineto{\pgfqpoint{3.948889in}{1.598116in}}%
\pgfpathquadraticcurveto{\pgfqpoint{3.973889in}{1.598116in}}{\pgfqpoint{3.973889in}{1.623116in}}%
\pgfpathlineto{\pgfqpoint{3.973889in}{2.307814in}}%
\pgfpathquadraticcurveto{\pgfqpoint{3.973889in}{2.332814in}}{\pgfqpoint{3.948889in}{2.332814in}}%
\pgfpathlineto{\pgfqpoint{2.841524in}{2.332814in}}%
\pgfpathquadraticcurveto{\pgfqpoint{2.816524in}{2.332814in}}{\pgfqpoint{2.816524in}{2.307814in}}%
\pgfpathlineto{\pgfqpoint{2.816524in}{1.623116in}}%
\pgfpathquadraticcurveto{\pgfqpoint{2.816524in}{1.598116in}}{\pgfqpoint{2.841524in}{1.598116in}}%
\pgfpathclose%
\pgfusepath{stroke,fill}%
\end{pgfscope}%
\begin{pgfscope}%
\pgfsetbuttcap%
\pgfsetroundjoin%
\definecolor{currentfill}{rgb}{0.000000,0.000000,0.000000}%
\pgfsetfillcolor{currentfill}%
\pgfsetfillopacity{0.000000}%
\pgfsetlinewidth{1.003750pt}%
\definecolor{currentstroke}{rgb}{0.000000,0.447000,0.741000}%
\pgfsetstrokecolor{currentstroke}%
\pgfsetdash{}{0pt}%
\pgfsys@defobject{currentmarker}{\pgfqpoint{-0.041667in}{-0.041667in}}{\pgfqpoint{0.041667in}{0.041667in}}{%
\pgfpathmoveto{\pgfqpoint{0.000000in}{-0.041667in}}%
\pgfpathcurveto{\pgfqpoint{0.011050in}{-0.041667in}}{\pgfqpoint{0.021649in}{-0.037276in}}{\pgfqpoint{0.029463in}{-0.029463in}}%
\pgfpathcurveto{\pgfqpoint{0.037276in}{-0.021649in}}{\pgfqpoint{0.041667in}{-0.011050in}}{\pgfqpoint{0.041667in}{0.000000in}}%
\pgfpathcurveto{\pgfqpoint{0.041667in}{0.011050in}}{\pgfqpoint{0.037276in}{0.021649in}}{\pgfqpoint{0.029463in}{0.029463in}}%
\pgfpathcurveto{\pgfqpoint{0.021649in}{0.037276in}}{\pgfqpoint{0.011050in}{0.041667in}}{\pgfqpoint{0.000000in}{0.041667in}}%
\pgfpathcurveto{\pgfqpoint{-0.011050in}{0.041667in}}{\pgfqpoint{-0.021649in}{0.037276in}}{\pgfqpoint{-0.029463in}{0.029463in}}%
\pgfpathcurveto{\pgfqpoint{-0.037276in}{0.021649in}}{\pgfqpoint{-0.041667in}{0.011050in}}{\pgfqpoint{-0.041667in}{0.000000in}}%
\pgfpathcurveto{\pgfqpoint{-0.041667in}{-0.011050in}}{\pgfqpoint{-0.037276in}{-0.021649in}}{\pgfqpoint{-0.029463in}{-0.029463in}}%
\pgfpathcurveto{\pgfqpoint{-0.021649in}{-0.037276in}}{\pgfqpoint{-0.011050in}{-0.041667in}}{\pgfqpoint{0.000000in}{-0.041667in}}%
\pgfpathclose%
\pgfusepath{stroke,fill}%
}%
\begin{pgfscope}%
\pgfsys@transformshift{2.991524in}{2.239064in}%
\pgfsys@useobject{currentmarker}{}%
\end{pgfscope}%
\end{pgfscope}%
\begin{pgfscope}%
\definecolor{textcolor}{rgb}{0.000000,0.000000,0.000000}%
\pgfsetstrokecolor{textcolor}%
\pgfsetfillcolor{textcolor}%
\pgftext[x=3.216524in,y=2.195314in,left,base]{\color{textcolor}\rmfamily\fontsize{9.000000}{10.800000}\selectfont \(\displaystyle \nu_{14} = \) 107.70}%
\end{pgfscope}%
\begin{pgfscope}%
\pgfsetbuttcap%
\pgfsetroundjoin%
\definecolor{currentfill}{rgb}{0.850000,0.324000,0.098000}%
\pgfsetfillcolor{currentfill}%
\pgfsetlinewidth{1.003750pt}%
\definecolor{currentstroke}{rgb}{0.850000,0.324000,0.098000}%
\pgfsetstrokecolor{currentstroke}%
\pgfsetdash{}{0pt}%
\pgfsys@defobject{currentmarker}{\pgfqpoint{-0.041667in}{-0.041667in}}{\pgfqpoint{0.041667in}{0.041667in}}{%
\pgfpathmoveto{\pgfqpoint{-0.041667in}{0.000000in}}%
\pgfpathlineto{\pgfqpoint{0.041667in}{0.000000in}}%
\pgfpathmoveto{\pgfqpoint{0.000000in}{-0.041667in}}%
\pgfpathlineto{\pgfqpoint{0.000000in}{0.041667in}}%
\pgfusepath{stroke,fill}%
}%
\begin{pgfscope}%
\pgfsys@transformshift{2.991524in}{2.064765in}%
\pgfsys@useobject{currentmarker}{}%
\end{pgfscope}%
\end{pgfscope}%
\begin{pgfscope}%
\definecolor{textcolor}{rgb}{0.000000,0.000000,0.000000}%
\pgfsetstrokecolor{textcolor}%
\pgfsetfillcolor{textcolor}%
\pgftext[x=3.216524in,y=2.021015in,left,base]{\color{textcolor}\rmfamily\fontsize{9.000000}{10.800000}\selectfont \(\displaystyle \nu_{15} = \) 110.24}%
\end{pgfscope}%
\begin{pgfscope}%
\pgfsetbuttcap%
\pgfsetmiterjoin%
\definecolor{currentfill}{rgb}{0.000000,0.000000,0.000000}%
\pgfsetfillcolor{currentfill}%
\pgfsetfillopacity{0.000000}%
\pgfsetlinewidth{1.003750pt}%
\definecolor{currentstroke}{rgb}{0.000000,0.500000,0.000000}%
\pgfsetstrokecolor{currentstroke}%
\pgfsetdash{}{0pt}%
\pgfsys@defobject{currentmarker}{\pgfqpoint{-0.041667in}{-0.041667in}}{\pgfqpoint{0.041667in}{0.041667in}}{%
\pgfpathmoveto{\pgfqpoint{-0.041667in}{-0.041667in}}%
\pgfpathlineto{\pgfqpoint{0.041667in}{-0.041667in}}%
\pgfpathlineto{\pgfqpoint{0.041667in}{0.041667in}}%
\pgfpathlineto{\pgfqpoint{-0.041667in}{0.041667in}}%
\pgfpathclose%
\pgfusepath{stroke,fill}%
}%
\begin{pgfscope}%
\pgfsys@transformshift{2.991524in}{1.890465in}%
\pgfsys@useobject{currentmarker}{}%
\end{pgfscope}%
\end{pgfscope}%
\begin{pgfscope}%
\definecolor{textcolor}{rgb}{0.000000,0.000000,0.000000}%
\pgfsetstrokecolor{textcolor}%
\pgfsetfillcolor{textcolor}%
\pgftext[x=3.216524in,y=1.846715in,left,base]{\color{textcolor}\rmfamily\fontsize{9.000000}{10.800000}\selectfont \(\displaystyle \nu_{16} = \) 112.50}%
\end{pgfscope}%
\begin{pgfscope}%
\pgfsetbuttcap%
\pgfsetroundjoin%
\definecolor{currentfill}{rgb}{0.494000,0.184000,0.556000}%
\pgfsetfillcolor{currentfill}%
\pgfsetlinewidth{1.003750pt}%
\definecolor{currentstroke}{rgb}{0.494000,0.184000,0.556000}%
\pgfsetstrokecolor{currentstroke}%
\pgfsetdash{}{0pt}%
\pgfsys@defobject{currentmarker}{\pgfqpoint{-0.041667in}{-0.041667in}}{\pgfqpoint{0.041667in}{0.041667in}}{%
\pgfpathmoveto{\pgfqpoint{-0.041667in}{-0.041667in}}%
\pgfpathlineto{\pgfqpoint{0.041667in}{0.041667in}}%
\pgfpathmoveto{\pgfqpoint{-0.041667in}{0.041667in}}%
\pgfpathlineto{\pgfqpoint{0.041667in}{-0.041667in}}%
\pgfusepath{stroke,fill}%
}%
\begin{pgfscope}%
\pgfsys@transformshift{2.991524in}{1.716165in}%
\pgfsys@useobject{currentmarker}{}%
\end{pgfscope}%
\end{pgfscope}%
\begin{pgfscope}%
\definecolor{textcolor}{rgb}{0.000000,0.000000,0.000000}%
\pgfsetstrokecolor{textcolor}%
\pgfsetfillcolor{textcolor}%
\pgftext[x=3.216524in,y=1.672415in,left,base]{\color{textcolor}\rmfamily\fontsize{9.000000}{10.800000}\selectfont \(\displaystyle \nu_{17} = \) 114.00}%
\end{pgfscope}%
\end{pgfpicture}%
\makeatother%
\endgroup%
}
					\caption{Cluster IV $(a)$}
					\label{SubFig:Cluster_IV_imag}
				\end{subfigure}
				\begin{subfigure}[h]{0.5\textwidth}
					\centering
					\resizebox{\linewidth}{!}{%% Creator: Matplotlib, PGF backend
%%
%% To include the figure in your LaTeX document, write
%%   \input{<filename>.pgf}
%%
%% Make sure the required packages are loaded in your preamble
%%   \usepackage{pgf}
%%
%% and, on pdftex
%%   \usepackage[utf8]{inputenc}\DeclareUnicodeCharacter{2212}{-}
%%
%% or, on luatex and xetex
%%   \usepackage{unicode-math}
%%
%% Figures using additional raster images can only be included by \input if
%% they are in the same directory as the main LaTeX file. For loading figures
%% from other directories you can use the `import` package
%%   \usepackage{import}
%%
%% and then include the figures with
%%   \import{<path to file>}{<filename>.pgf}
%%
%% Matplotlib used the following preamble
%%   \usepackage[utf8x]{inputenc}
%%   \usepackage[T1]{fontenc}
%%   \usepackage{amsmath,amssymb,amsfonts}
%%
\begingroup%
\makeatletter%
\begin{pgfpicture}%
\pgfpathrectangle{\pgfpointorigin}{\pgfqpoint{4.136389in}{2.495314in}}%
\pgfusepath{use as bounding box, clip}%
\begin{pgfscope}%
\pgfsetbuttcap%
\pgfsetmiterjoin%
\definecolor{currentfill}{rgb}{1.000000,1.000000,1.000000}%
\pgfsetfillcolor{currentfill}%
\pgfsetlinewidth{0.000000pt}%
\definecolor{currentstroke}{rgb}{1.000000,1.000000,1.000000}%
\pgfsetstrokecolor{currentstroke}%
\pgfsetdash{}{0pt}%
\pgfpathmoveto{\pgfqpoint{-0.000000in}{0.000000in}}%
\pgfpathlineto{\pgfqpoint{4.136389in}{0.000000in}}%
\pgfpathlineto{\pgfqpoint{4.136389in}{2.495314in}}%
\pgfpathlineto{\pgfqpoint{-0.000000in}{2.495314in}}%
\pgfpathclose%
\pgfusepath{fill}%
\end{pgfscope}%
\begin{pgfscope}%
\pgfsetbuttcap%
\pgfsetmiterjoin%
\definecolor{currentfill}{rgb}{1.000000,1.000000,1.000000}%
\pgfsetfillcolor{currentfill}%
\pgfsetlinewidth{0.000000pt}%
\definecolor{currentstroke}{rgb}{0.000000,0.000000,0.000000}%
\pgfsetstrokecolor{currentstroke}%
\pgfsetstrokeopacity{0.000000}%
\pgfsetdash{}{0pt}%
\pgfpathmoveto{\pgfqpoint{0.740433in}{0.566590in}}%
\pgfpathlineto{\pgfqpoint{4.036389in}{0.566590in}}%
\pgfpathlineto{\pgfqpoint{4.036389in}{2.395314in}}%
\pgfpathlineto{\pgfqpoint{0.740433in}{2.395314in}}%
\pgfpathclose%
\pgfusepath{fill}%
\end{pgfscope}%
\begin{pgfscope}%
\pgfpathrectangle{\pgfqpoint{0.740433in}{0.566590in}}{\pgfqpoint{3.295956in}{1.828724in}}%
\pgfusepath{clip}%
\pgfsetrectcap%
\pgfsetroundjoin%
\pgfsetlinewidth{0.803000pt}%
\definecolor{currentstroke}{rgb}{0.690196,0.690196,0.690196}%
\pgfsetstrokecolor{currentstroke}%
\pgfsetdash{}{0pt}%
\pgfpathmoveto{\pgfqpoint{0.740433in}{0.566590in}}%
\pgfpathlineto{\pgfqpoint{0.740433in}{2.395314in}}%
\pgfusepath{stroke}%
\end{pgfscope}%
\begin{pgfscope}%
\pgfsetbuttcap%
\pgfsetroundjoin%
\definecolor{currentfill}{rgb}{0.000000,0.000000,0.000000}%
\pgfsetfillcolor{currentfill}%
\pgfsetlinewidth{0.803000pt}%
\definecolor{currentstroke}{rgb}{0.000000,0.000000,0.000000}%
\pgfsetstrokecolor{currentstroke}%
\pgfsetdash{}{0pt}%
\pgfsys@defobject{currentmarker}{\pgfqpoint{0.000000in}{-0.048611in}}{\pgfqpoint{0.000000in}{0.000000in}}{%
\pgfpathmoveto{\pgfqpoint{0.000000in}{0.000000in}}%
\pgfpathlineto{\pgfqpoint{0.000000in}{-0.048611in}}%
\pgfusepath{stroke,fill}%
}%
\begin{pgfscope}%
\pgfsys@transformshift{0.740433in}{0.566590in}%
\pgfsys@useobject{currentmarker}{}%
\end{pgfscope}%
\end{pgfscope}%
\begin{pgfscope}%
\definecolor{textcolor}{rgb}{0.000000,0.000000,0.000000}%
\pgfsetstrokecolor{textcolor}%
\pgfsetfillcolor{textcolor}%
\pgftext[x=0.740433in,y=0.469368in,,top]{\color{textcolor}\rmfamily\fontsize{12.000000}{14.400000}\selectfont \(\displaystyle {-10}\)}%
\end{pgfscope}%
\begin{pgfscope}%
\pgfpathrectangle{\pgfqpoint{0.740433in}{0.566590in}}{\pgfqpoint{3.295956in}{1.828724in}}%
\pgfusepath{clip}%
\pgfsetrectcap%
\pgfsetroundjoin%
\pgfsetlinewidth{0.803000pt}%
\definecolor{currentstroke}{rgb}{0.690196,0.690196,0.690196}%
\pgfsetstrokecolor{currentstroke}%
\pgfsetdash{}{0pt}%
\pgfpathmoveto{\pgfqpoint{1.247503in}{0.566590in}}%
\pgfpathlineto{\pgfqpoint{1.247503in}{2.395314in}}%
\pgfusepath{stroke}%
\end{pgfscope}%
\begin{pgfscope}%
\pgfsetbuttcap%
\pgfsetroundjoin%
\definecolor{currentfill}{rgb}{0.000000,0.000000,0.000000}%
\pgfsetfillcolor{currentfill}%
\pgfsetlinewidth{0.803000pt}%
\definecolor{currentstroke}{rgb}{0.000000,0.000000,0.000000}%
\pgfsetstrokecolor{currentstroke}%
\pgfsetdash{}{0pt}%
\pgfsys@defobject{currentmarker}{\pgfqpoint{0.000000in}{-0.048611in}}{\pgfqpoint{0.000000in}{0.000000in}}{%
\pgfpathmoveto{\pgfqpoint{0.000000in}{0.000000in}}%
\pgfpathlineto{\pgfqpoint{0.000000in}{-0.048611in}}%
\pgfusepath{stroke,fill}%
}%
\begin{pgfscope}%
\pgfsys@transformshift{1.247503in}{0.566590in}%
\pgfsys@useobject{currentmarker}{}%
\end{pgfscope}%
\end{pgfscope}%
\begin{pgfscope}%
\definecolor{textcolor}{rgb}{0.000000,0.000000,0.000000}%
\pgfsetstrokecolor{textcolor}%
\pgfsetfillcolor{textcolor}%
\pgftext[x=1.247503in,y=0.469368in,,top]{\color{textcolor}\rmfamily\fontsize{12.000000}{14.400000}\selectfont \(\displaystyle {0}\)}%
\end{pgfscope}%
\begin{pgfscope}%
\pgfpathrectangle{\pgfqpoint{0.740433in}{0.566590in}}{\pgfqpoint{3.295956in}{1.828724in}}%
\pgfusepath{clip}%
\pgfsetrectcap%
\pgfsetroundjoin%
\pgfsetlinewidth{0.803000pt}%
\definecolor{currentstroke}{rgb}{0.690196,0.690196,0.690196}%
\pgfsetstrokecolor{currentstroke}%
\pgfsetdash{}{0pt}%
\pgfpathmoveto{\pgfqpoint{1.754573in}{0.566590in}}%
\pgfpathlineto{\pgfqpoint{1.754573in}{2.395314in}}%
\pgfusepath{stroke}%
\end{pgfscope}%
\begin{pgfscope}%
\pgfsetbuttcap%
\pgfsetroundjoin%
\definecolor{currentfill}{rgb}{0.000000,0.000000,0.000000}%
\pgfsetfillcolor{currentfill}%
\pgfsetlinewidth{0.803000pt}%
\definecolor{currentstroke}{rgb}{0.000000,0.000000,0.000000}%
\pgfsetstrokecolor{currentstroke}%
\pgfsetdash{}{0pt}%
\pgfsys@defobject{currentmarker}{\pgfqpoint{0.000000in}{-0.048611in}}{\pgfqpoint{0.000000in}{0.000000in}}{%
\pgfpathmoveto{\pgfqpoint{0.000000in}{0.000000in}}%
\pgfpathlineto{\pgfqpoint{0.000000in}{-0.048611in}}%
\pgfusepath{stroke,fill}%
}%
\begin{pgfscope}%
\pgfsys@transformshift{1.754573in}{0.566590in}%
\pgfsys@useobject{currentmarker}{}%
\end{pgfscope}%
\end{pgfscope}%
\begin{pgfscope}%
\definecolor{textcolor}{rgb}{0.000000,0.000000,0.000000}%
\pgfsetstrokecolor{textcolor}%
\pgfsetfillcolor{textcolor}%
\pgftext[x=1.754573in,y=0.469368in,,top]{\color{textcolor}\rmfamily\fontsize{12.000000}{14.400000}\selectfont \(\displaystyle {10}\)}%
\end{pgfscope}%
\begin{pgfscope}%
\pgfpathrectangle{\pgfqpoint{0.740433in}{0.566590in}}{\pgfqpoint{3.295956in}{1.828724in}}%
\pgfusepath{clip}%
\pgfsetrectcap%
\pgfsetroundjoin%
\pgfsetlinewidth{0.803000pt}%
\definecolor{currentstroke}{rgb}{0.690196,0.690196,0.690196}%
\pgfsetstrokecolor{currentstroke}%
\pgfsetdash{}{0pt}%
\pgfpathmoveto{\pgfqpoint{2.261643in}{0.566590in}}%
\pgfpathlineto{\pgfqpoint{2.261643in}{2.395314in}}%
\pgfusepath{stroke}%
\end{pgfscope}%
\begin{pgfscope}%
\pgfsetbuttcap%
\pgfsetroundjoin%
\definecolor{currentfill}{rgb}{0.000000,0.000000,0.000000}%
\pgfsetfillcolor{currentfill}%
\pgfsetlinewidth{0.803000pt}%
\definecolor{currentstroke}{rgb}{0.000000,0.000000,0.000000}%
\pgfsetstrokecolor{currentstroke}%
\pgfsetdash{}{0pt}%
\pgfsys@defobject{currentmarker}{\pgfqpoint{0.000000in}{-0.048611in}}{\pgfqpoint{0.000000in}{0.000000in}}{%
\pgfpathmoveto{\pgfqpoint{0.000000in}{0.000000in}}%
\pgfpathlineto{\pgfqpoint{0.000000in}{-0.048611in}}%
\pgfusepath{stroke,fill}%
}%
\begin{pgfscope}%
\pgfsys@transformshift{2.261643in}{0.566590in}%
\pgfsys@useobject{currentmarker}{}%
\end{pgfscope}%
\end{pgfscope}%
\begin{pgfscope}%
\definecolor{textcolor}{rgb}{0.000000,0.000000,0.000000}%
\pgfsetstrokecolor{textcolor}%
\pgfsetfillcolor{textcolor}%
\pgftext[x=2.261643in,y=0.469368in,,top]{\color{textcolor}\rmfamily\fontsize{12.000000}{14.400000}\selectfont \(\displaystyle {20}\)}%
\end{pgfscope}%
\begin{pgfscope}%
\pgfpathrectangle{\pgfqpoint{0.740433in}{0.566590in}}{\pgfqpoint{3.295956in}{1.828724in}}%
\pgfusepath{clip}%
\pgfsetrectcap%
\pgfsetroundjoin%
\pgfsetlinewidth{0.803000pt}%
\definecolor{currentstroke}{rgb}{0.690196,0.690196,0.690196}%
\pgfsetstrokecolor{currentstroke}%
\pgfsetdash{}{0pt}%
\pgfpathmoveto{\pgfqpoint{2.768713in}{0.566590in}}%
\pgfpathlineto{\pgfqpoint{2.768713in}{2.395314in}}%
\pgfusepath{stroke}%
\end{pgfscope}%
\begin{pgfscope}%
\pgfsetbuttcap%
\pgfsetroundjoin%
\definecolor{currentfill}{rgb}{0.000000,0.000000,0.000000}%
\pgfsetfillcolor{currentfill}%
\pgfsetlinewidth{0.803000pt}%
\definecolor{currentstroke}{rgb}{0.000000,0.000000,0.000000}%
\pgfsetstrokecolor{currentstroke}%
\pgfsetdash{}{0pt}%
\pgfsys@defobject{currentmarker}{\pgfqpoint{0.000000in}{-0.048611in}}{\pgfqpoint{0.000000in}{0.000000in}}{%
\pgfpathmoveto{\pgfqpoint{0.000000in}{0.000000in}}%
\pgfpathlineto{\pgfqpoint{0.000000in}{-0.048611in}}%
\pgfusepath{stroke,fill}%
}%
\begin{pgfscope}%
\pgfsys@transformshift{2.768713in}{0.566590in}%
\pgfsys@useobject{currentmarker}{}%
\end{pgfscope}%
\end{pgfscope}%
\begin{pgfscope}%
\definecolor{textcolor}{rgb}{0.000000,0.000000,0.000000}%
\pgfsetstrokecolor{textcolor}%
\pgfsetfillcolor{textcolor}%
\pgftext[x=2.768713in,y=0.469368in,,top]{\color{textcolor}\rmfamily\fontsize{12.000000}{14.400000}\selectfont \(\displaystyle {30}\)}%
\end{pgfscope}%
\begin{pgfscope}%
\pgfpathrectangle{\pgfqpoint{0.740433in}{0.566590in}}{\pgfqpoint{3.295956in}{1.828724in}}%
\pgfusepath{clip}%
\pgfsetrectcap%
\pgfsetroundjoin%
\pgfsetlinewidth{0.803000pt}%
\definecolor{currentstroke}{rgb}{0.690196,0.690196,0.690196}%
\pgfsetstrokecolor{currentstroke}%
\pgfsetdash{}{0pt}%
\pgfpathmoveto{\pgfqpoint{3.275783in}{0.566590in}}%
\pgfpathlineto{\pgfqpoint{3.275783in}{2.395314in}}%
\pgfusepath{stroke}%
\end{pgfscope}%
\begin{pgfscope}%
\pgfsetbuttcap%
\pgfsetroundjoin%
\definecolor{currentfill}{rgb}{0.000000,0.000000,0.000000}%
\pgfsetfillcolor{currentfill}%
\pgfsetlinewidth{0.803000pt}%
\definecolor{currentstroke}{rgb}{0.000000,0.000000,0.000000}%
\pgfsetstrokecolor{currentstroke}%
\pgfsetdash{}{0pt}%
\pgfsys@defobject{currentmarker}{\pgfqpoint{0.000000in}{-0.048611in}}{\pgfqpoint{0.000000in}{0.000000in}}{%
\pgfpathmoveto{\pgfqpoint{0.000000in}{0.000000in}}%
\pgfpathlineto{\pgfqpoint{0.000000in}{-0.048611in}}%
\pgfusepath{stroke,fill}%
}%
\begin{pgfscope}%
\pgfsys@transformshift{3.275783in}{0.566590in}%
\pgfsys@useobject{currentmarker}{}%
\end{pgfscope}%
\end{pgfscope}%
\begin{pgfscope}%
\definecolor{textcolor}{rgb}{0.000000,0.000000,0.000000}%
\pgfsetstrokecolor{textcolor}%
\pgfsetfillcolor{textcolor}%
\pgftext[x=3.275783in,y=0.469368in,,top]{\color{textcolor}\rmfamily\fontsize{12.000000}{14.400000}\selectfont \(\displaystyle {40}\)}%
\end{pgfscope}%
\begin{pgfscope}%
\pgfpathrectangle{\pgfqpoint{0.740433in}{0.566590in}}{\pgfqpoint{3.295956in}{1.828724in}}%
\pgfusepath{clip}%
\pgfsetrectcap%
\pgfsetroundjoin%
\pgfsetlinewidth{0.803000pt}%
\definecolor{currentstroke}{rgb}{0.690196,0.690196,0.690196}%
\pgfsetstrokecolor{currentstroke}%
\pgfsetdash{}{0pt}%
\pgfpathmoveto{\pgfqpoint{3.782853in}{0.566590in}}%
\pgfpathlineto{\pgfqpoint{3.782853in}{2.395314in}}%
\pgfusepath{stroke}%
\end{pgfscope}%
\begin{pgfscope}%
\pgfsetbuttcap%
\pgfsetroundjoin%
\definecolor{currentfill}{rgb}{0.000000,0.000000,0.000000}%
\pgfsetfillcolor{currentfill}%
\pgfsetlinewidth{0.803000pt}%
\definecolor{currentstroke}{rgb}{0.000000,0.000000,0.000000}%
\pgfsetstrokecolor{currentstroke}%
\pgfsetdash{}{0pt}%
\pgfsys@defobject{currentmarker}{\pgfqpoint{0.000000in}{-0.048611in}}{\pgfqpoint{0.000000in}{0.000000in}}{%
\pgfpathmoveto{\pgfqpoint{0.000000in}{0.000000in}}%
\pgfpathlineto{\pgfqpoint{0.000000in}{-0.048611in}}%
\pgfusepath{stroke,fill}%
}%
\begin{pgfscope}%
\pgfsys@transformshift{3.782853in}{0.566590in}%
\pgfsys@useobject{currentmarker}{}%
\end{pgfscope}%
\end{pgfscope}%
\begin{pgfscope}%
\definecolor{textcolor}{rgb}{0.000000,0.000000,0.000000}%
\pgfsetstrokecolor{textcolor}%
\pgfsetfillcolor{textcolor}%
\pgftext[x=3.782853in,y=0.469368in,,top]{\color{textcolor}\rmfamily\fontsize{12.000000}{14.400000}\selectfont \(\displaystyle {50}\)}%
\end{pgfscope}%
\begin{pgfscope}%
\definecolor{textcolor}{rgb}{0.000000,0.000000,0.000000}%
\pgfsetstrokecolor{textcolor}%
\pgfsetfillcolor{textcolor}%
\pgftext[x=2.388411in,y=0.266626in,,top]{\color{textcolor}\rmfamily\fontsize{12.000000}{14.400000}\selectfont SNR [dB]}%
\end{pgfscope}%
\begin{pgfscope}%
\pgfpathrectangle{\pgfqpoint{0.740433in}{0.566590in}}{\pgfqpoint{3.295956in}{1.828724in}}%
\pgfusepath{clip}%
\pgfsetrectcap%
\pgfsetroundjoin%
\pgfsetlinewidth{0.803000pt}%
\definecolor{currentstroke}{rgb}{0.690196,0.690196,0.690196}%
\pgfsetstrokecolor{currentstroke}%
\pgfsetdash{}{0pt}%
\pgfpathmoveto{\pgfqpoint{0.740433in}{0.566590in}}%
\pgfpathlineto{\pgfqpoint{4.036389in}{0.566590in}}%
\pgfusepath{stroke}%
\end{pgfscope}%
\begin{pgfscope}%
\pgfsetbuttcap%
\pgfsetroundjoin%
\definecolor{currentfill}{rgb}{0.000000,0.000000,0.000000}%
\pgfsetfillcolor{currentfill}%
\pgfsetlinewidth{0.803000pt}%
\definecolor{currentstroke}{rgb}{0.000000,0.000000,0.000000}%
\pgfsetstrokecolor{currentstroke}%
\pgfsetdash{}{0pt}%
\pgfsys@defobject{currentmarker}{\pgfqpoint{-0.048611in}{0.000000in}}{\pgfqpoint{-0.000000in}{0.000000in}}{%
\pgfpathmoveto{\pgfqpoint{-0.000000in}{0.000000in}}%
\pgfpathlineto{\pgfqpoint{-0.048611in}{0.000000in}}%
\pgfusepath{stroke,fill}%
}%
\begin{pgfscope}%
\pgfsys@transformshift{0.740433in}{0.566590in}%
\pgfsys@useobject{currentmarker}{}%
\end{pgfscope}%
\end{pgfscope}%
\begin{pgfscope}%
\definecolor{textcolor}{rgb}{0.000000,0.000000,0.000000}%
\pgfsetstrokecolor{textcolor}%
\pgfsetfillcolor{textcolor}%
\pgftext[x=0.322222in, y=0.509197in, left, base]{\color{textcolor}\rmfamily\fontsize{12.000000}{14.400000}\selectfont \(\displaystyle {10^{-4}}\)}%
\end{pgfscope}%
\begin{pgfscope}%
\pgfpathrectangle{\pgfqpoint{0.740433in}{0.566590in}}{\pgfqpoint{3.295956in}{1.828724in}}%
\pgfusepath{clip}%
\pgfsetrectcap%
\pgfsetroundjoin%
\pgfsetlinewidth{0.803000pt}%
\definecolor{currentstroke}{rgb}{0.690196,0.690196,0.690196}%
\pgfsetstrokecolor{currentstroke}%
\pgfsetdash{}{0pt}%
\pgfpathmoveto{\pgfqpoint{0.740433in}{1.104776in}}%
\pgfpathlineto{\pgfqpoint{4.036389in}{1.104776in}}%
\pgfusepath{stroke}%
\end{pgfscope}%
\begin{pgfscope}%
\pgfsetbuttcap%
\pgfsetroundjoin%
\definecolor{currentfill}{rgb}{0.000000,0.000000,0.000000}%
\pgfsetfillcolor{currentfill}%
\pgfsetlinewidth{0.803000pt}%
\definecolor{currentstroke}{rgb}{0.000000,0.000000,0.000000}%
\pgfsetstrokecolor{currentstroke}%
\pgfsetdash{}{0pt}%
\pgfsys@defobject{currentmarker}{\pgfqpoint{-0.048611in}{0.000000in}}{\pgfqpoint{-0.000000in}{0.000000in}}{%
\pgfpathmoveto{\pgfqpoint{-0.000000in}{0.000000in}}%
\pgfpathlineto{\pgfqpoint{-0.048611in}{0.000000in}}%
\pgfusepath{stroke,fill}%
}%
\begin{pgfscope}%
\pgfsys@transformshift{0.740433in}{1.104776in}%
\pgfsys@useobject{currentmarker}{}%
\end{pgfscope}%
\end{pgfscope}%
\begin{pgfscope}%
\definecolor{textcolor}{rgb}{0.000000,0.000000,0.000000}%
\pgfsetstrokecolor{textcolor}%
\pgfsetfillcolor{textcolor}%
\pgftext[x=0.322222in, y=1.047383in, left, base]{\color{textcolor}\rmfamily\fontsize{12.000000}{14.400000}\selectfont \(\displaystyle {10^{-2}}\)}%
\end{pgfscope}%
\begin{pgfscope}%
\pgfpathrectangle{\pgfqpoint{0.740433in}{0.566590in}}{\pgfqpoint{3.295956in}{1.828724in}}%
\pgfusepath{clip}%
\pgfsetrectcap%
\pgfsetroundjoin%
\pgfsetlinewidth{0.803000pt}%
\definecolor{currentstroke}{rgb}{0.690196,0.690196,0.690196}%
\pgfsetstrokecolor{currentstroke}%
\pgfsetdash{}{0pt}%
\pgfpathmoveto{\pgfqpoint{0.740433in}{1.642962in}}%
\pgfpathlineto{\pgfqpoint{4.036389in}{1.642962in}}%
\pgfusepath{stroke}%
\end{pgfscope}%
\begin{pgfscope}%
\pgfsetbuttcap%
\pgfsetroundjoin%
\definecolor{currentfill}{rgb}{0.000000,0.000000,0.000000}%
\pgfsetfillcolor{currentfill}%
\pgfsetlinewidth{0.803000pt}%
\definecolor{currentstroke}{rgb}{0.000000,0.000000,0.000000}%
\pgfsetstrokecolor{currentstroke}%
\pgfsetdash{}{0pt}%
\pgfsys@defobject{currentmarker}{\pgfqpoint{-0.048611in}{0.000000in}}{\pgfqpoint{-0.000000in}{0.000000in}}{%
\pgfpathmoveto{\pgfqpoint{-0.000000in}{0.000000in}}%
\pgfpathlineto{\pgfqpoint{-0.048611in}{0.000000in}}%
\pgfusepath{stroke,fill}%
}%
\begin{pgfscope}%
\pgfsys@transformshift{0.740433in}{1.642962in}%
\pgfsys@useobject{currentmarker}{}%
\end{pgfscope}%
\end{pgfscope}%
\begin{pgfscope}%
\definecolor{textcolor}{rgb}{0.000000,0.000000,0.000000}%
\pgfsetstrokecolor{textcolor}%
\pgfsetfillcolor{textcolor}%
\pgftext[x=0.414045in, y=1.585569in, left, base]{\color{textcolor}\rmfamily\fontsize{12.000000}{14.400000}\selectfont \(\displaystyle {10^{0}}\)}%
\end{pgfscope}%
\begin{pgfscope}%
\pgfpathrectangle{\pgfqpoint{0.740433in}{0.566590in}}{\pgfqpoint{3.295956in}{1.828724in}}%
\pgfusepath{clip}%
\pgfsetrectcap%
\pgfsetroundjoin%
\pgfsetlinewidth{0.803000pt}%
\definecolor{currentstroke}{rgb}{0.690196,0.690196,0.690196}%
\pgfsetstrokecolor{currentstroke}%
\pgfsetdash{}{0pt}%
\pgfpathmoveto{\pgfqpoint{0.740433in}{2.181148in}}%
\pgfpathlineto{\pgfqpoint{4.036389in}{2.181148in}}%
\pgfusepath{stroke}%
\end{pgfscope}%
\begin{pgfscope}%
\pgfsetbuttcap%
\pgfsetroundjoin%
\definecolor{currentfill}{rgb}{0.000000,0.000000,0.000000}%
\pgfsetfillcolor{currentfill}%
\pgfsetlinewidth{0.803000pt}%
\definecolor{currentstroke}{rgb}{0.000000,0.000000,0.000000}%
\pgfsetstrokecolor{currentstroke}%
\pgfsetdash{}{0pt}%
\pgfsys@defobject{currentmarker}{\pgfqpoint{-0.048611in}{0.000000in}}{\pgfqpoint{-0.000000in}{0.000000in}}{%
\pgfpathmoveto{\pgfqpoint{-0.000000in}{0.000000in}}%
\pgfpathlineto{\pgfqpoint{-0.048611in}{0.000000in}}%
\pgfusepath{stroke,fill}%
}%
\begin{pgfscope}%
\pgfsys@transformshift{0.740433in}{2.181148in}%
\pgfsys@useobject{currentmarker}{}%
\end{pgfscope}%
\end{pgfscope}%
\begin{pgfscope}%
\definecolor{textcolor}{rgb}{0.000000,0.000000,0.000000}%
\pgfsetstrokecolor{textcolor}%
\pgfsetfillcolor{textcolor}%
\pgftext[x=0.414045in, y=2.123755in, left, base]{\color{textcolor}\rmfamily\fontsize{12.000000}{14.400000}\selectfont \(\displaystyle {10^{2}}\)}%
\end{pgfscope}%
\begin{pgfscope}%
\definecolor{textcolor}{rgb}{0.000000,0.000000,0.000000}%
\pgfsetstrokecolor{textcolor}%
\pgfsetfillcolor{textcolor}%
\pgftext[x=0.266667in,y=1.480952in,,bottom,rotate=90.000000]{\color{textcolor}\rmfamily\fontsize{12.000000}{14.400000}\selectfont \(\displaystyle \hat{\sigma}_{\nu}(\mathrm{SNR})\)}%
\end{pgfscope}%
\begin{pgfscope}%
\pgfpathrectangle{\pgfqpoint{0.740433in}{0.566590in}}{\pgfqpoint{3.295956in}{1.828724in}}%
\pgfusepath{clip}%
\pgfsetbuttcap%
\pgfsetroundjoin%
\pgfsetlinewidth{1.505625pt}%
\definecolor{currentstroke}{rgb}{0.000000,0.447000,0.741000}%
\pgfsetstrokecolor{currentstroke}%
\pgfsetdash{{5.550000pt}{2.400000pt}}{0.000000pt}%
\pgfpathmoveto{\pgfqpoint{0.740433in}{2.285578in}}%
\pgfpathlineto{\pgfqpoint{0.837373in}{2.277752in}}%
\pgfpathlineto{\pgfqpoint{0.934312in}{2.261621in}}%
\pgfpathlineto{\pgfqpoint{1.031252in}{2.254721in}}%
\pgfpathlineto{\pgfqpoint{1.128192in}{2.229665in}}%
\pgfpathlineto{\pgfqpoint{1.225132in}{2.240429in}}%
\pgfpathlineto{\pgfqpoint{1.322072in}{2.218923in}}%
\pgfpathlineto{\pgfqpoint{1.419012in}{2.244414in}}%
\pgfpathlineto{\pgfqpoint{1.515952in}{2.228750in}}%
\pgfpathlineto{\pgfqpoint{1.612892in}{2.200726in}}%
\pgfpathlineto{\pgfqpoint{1.709831in}{2.217154in}}%
\pgfpathlineto{\pgfqpoint{1.806771in}{2.137543in}}%
\pgfpathlineto{\pgfqpoint{1.903711in}{2.137009in}}%
\pgfpathlineto{\pgfqpoint{2.000651in}{2.118111in}}%
\pgfpathlineto{\pgfqpoint{2.097591in}{1.959659in}}%
\pgfpathlineto{\pgfqpoint{2.194531in}{1.735763in}}%
\pgfpathlineto{\pgfqpoint{2.291471in}{1.711008in}}%
\pgfpathlineto{\pgfqpoint{2.388411in}{1.664787in}}%
\pgfpathlineto{\pgfqpoint{2.485350in}{1.610085in}}%
\pgfpathlineto{\pgfqpoint{2.582290in}{1.116388in}}%
\pgfpathlineto{\pgfqpoint{2.679230in}{1.081982in}}%
\pgfpathlineto{\pgfqpoint{2.776170in}{1.070425in}}%
\pgfpathlineto{\pgfqpoint{2.873110in}{1.042250in}}%
\pgfpathlineto{\pgfqpoint{2.970050in}{1.014932in}}%
\pgfpathlineto{\pgfqpoint{3.066990in}{0.984216in}}%
\pgfpathlineto{\pgfqpoint{3.163930in}{0.961341in}}%
\pgfpathlineto{\pgfqpoint{3.260870in}{0.936783in}}%
\pgfpathlineto{\pgfqpoint{3.357809in}{0.924073in}}%
\pgfpathlineto{\pgfqpoint{3.454749in}{0.890010in}}%
\pgfpathlineto{\pgfqpoint{3.551689in}{0.860135in}}%
\pgfpathlineto{\pgfqpoint{3.648629in}{0.840242in}}%
\pgfpathlineto{\pgfqpoint{3.745569in}{0.814340in}}%
\pgfpathlineto{\pgfqpoint{3.842509in}{0.788295in}}%
\pgfpathlineto{\pgfqpoint{3.939449in}{0.765285in}}%
\pgfpathlineto{\pgfqpoint{4.036389in}{0.728155in}}%
\pgfusepath{stroke}%
\end{pgfscope}%
\begin{pgfscope}%
\pgfpathrectangle{\pgfqpoint{0.740433in}{0.566590in}}{\pgfqpoint{3.295956in}{1.828724in}}%
\pgfusepath{clip}%
\pgfsetbuttcap%
\pgfsetroundjoin%
\definecolor{currentfill}{rgb}{0.000000,0.000000,0.000000}%
\pgfsetfillcolor{currentfill}%
\pgfsetfillopacity{0.000000}%
\pgfsetlinewidth{1.003750pt}%
\definecolor{currentstroke}{rgb}{0.000000,0.447000,0.741000}%
\pgfsetstrokecolor{currentstroke}%
\pgfsetdash{}{0pt}%
\pgfsys@defobject{currentmarker}{\pgfqpoint{-0.041667in}{-0.041667in}}{\pgfqpoint{0.041667in}{0.041667in}}{%
\pgfpathmoveto{\pgfqpoint{0.000000in}{-0.041667in}}%
\pgfpathcurveto{\pgfqpoint{0.011050in}{-0.041667in}}{\pgfqpoint{0.021649in}{-0.037276in}}{\pgfqpoint{0.029463in}{-0.029463in}}%
\pgfpathcurveto{\pgfqpoint{0.037276in}{-0.021649in}}{\pgfqpoint{0.041667in}{-0.011050in}}{\pgfqpoint{0.041667in}{0.000000in}}%
\pgfpathcurveto{\pgfqpoint{0.041667in}{0.011050in}}{\pgfqpoint{0.037276in}{0.021649in}}{\pgfqpoint{0.029463in}{0.029463in}}%
\pgfpathcurveto{\pgfqpoint{0.021649in}{0.037276in}}{\pgfqpoint{0.011050in}{0.041667in}}{\pgfqpoint{0.000000in}{0.041667in}}%
\pgfpathcurveto{\pgfqpoint{-0.011050in}{0.041667in}}{\pgfqpoint{-0.021649in}{0.037276in}}{\pgfqpoint{-0.029463in}{0.029463in}}%
\pgfpathcurveto{\pgfqpoint{-0.037276in}{0.021649in}}{\pgfqpoint{-0.041667in}{0.011050in}}{\pgfqpoint{-0.041667in}{0.000000in}}%
\pgfpathcurveto{\pgfqpoint{-0.041667in}{-0.011050in}}{\pgfqpoint{-0.037276in}{-0.021649in}}{\pgfqpoint{-0.029463in}{-0.029463in}}%
\pgfpathcurveto{\pgfqpoint{-0.021649in}{-0.037276in}}{\pgfqpoint{-0.011050in}{-0.041667in}}{\pgfqpoint{0.000000in}{-0.041667in}}%
\pgfpathclose%
\pgfusepath{stroke,fill}%
}%
\begin{pgfscope}%
\pgfsys@transformshift{0.740433in}{2.285578in}%
\pgfsys@useobject{currentmarker}{}%
\end{pgfscope}%
\begin{pgfscope}%
\pgfsys@transformshift{1.128192in}{2.229665in}%
\pgfsys@useobject{currentmarker}{}%
\end{pgfscope}%
\begin{pgfscope}%
\pgfsys@transformshift{1.515952in}{2.228750in}%
\pgfsys@useobject{currentmarker}{}%
\end{pgfscope}%
\begin{pgfscope}%
\pgfsys@transformshift{1.903711in}{2.137009in}%
\pgfsys@useobject{currentmarker}{}%
\end{pgfscope}%
\begin{pgfscope}%
\pgfsys@transformshift{2.291471in}{1.711008in}%
\pgfsys@useobject{currentmarker}{}%
\end{pgfscope}%
\begin{pgfscope}%
\pgfsys@transformshift{2.679230in}{1.081982in}%
\pgfsys@useobject{currentmarker}{}%
\end{pgfscope}%
\begin{pgfscope}%
\pgfsys@transformshift{3.066990in}{0.984216in}%
\pgfsys@useobject{currentmarker}{}%
\end{pgfscope}%
\begin{pgfscope}%
\pgfsys@transformshift{3.454749in}{0.890010in}%
\pgfsys@useobject{currentmarker}{}%
\end{pgfscope}%
\begin{pgfscope}%
\pgfsys@transformshift{3.842509in}{0.788295in}%
\pgfsys@useobject{currentmarker}{}%
\end{pgfscope}%
\end{pgfscope}%
\begin{pgfscope}%
\pgfpathrectangle{\pgfqpoint{0.740433in}{0.566590in}}{\pgfqpoint{3.295956in}{1.828724in}}%
\pgfusepath{clip}%
\pgfsetbuttcap%
\pgfsetroundjoin%
\pgfsetlinewidth{1.505625pt}%
\definecolor{currentstroke}{rgb}{0.850000,0.324000,0.098000}%
\pgfsetstrokecolor{currentstroke}%
\pgfsetdash{{5.550000pt}{2.400000pt}}{0.000000pt}%
\pgfpathmoveto{\pgfqpoint{0.740433in}{2.314635in}}%
\pgfpathlineto{\pgfqpoint{0.837373in}{2.313674in}}%
\pgfpathlineto{\pgfqpoint{0.934312in}{2.302623in}}%
\pgfpathlineto{\pgfqpoint{1.031252in}{2.301678in}}%
\pgfpathlineto{\pgfqpoint{1.128192in}{2.288524in}}%
\pgfpathlineto{\pgfqpoint{1.225132in}{2.287717in}}%
\pgfpathlineto{\pgfqpoint{1.322072in}{2.291717in}}%
\pgfpathlineto{\pgfqpoint{1.419012in}{2.298674in}}%
\pgfpathlineto{\pgfqpoint{1.515952in}{2.286017in}}%
\pgfpathlineto{\pgfqpoint{1.612892in}{2.276068in}}%
\pgfpathlineto{\pgfqpoint{1.709831in}{2.278022in}}%
\pgfpathlineto{\pgfqpoint{1.806771in}{2.251478in}}%
\pgfpathlineto{\pgfqpoint{1.903711in}{2.239430in}}%
\pgfpathlineto{\pgfqpoint{2.000651in}{2.232191in}}%
\pgfpathlineto{\pgfqpoint{2.097591in}{2.184680in}}%
\pgfpathlineto{\pgfqpoint{2.194531in}{2.082845in}}%
\pgfpathlineto{\pgfqpoint{2.291471in}{2.049253in}}%
\pgfpathlineto{\pgfqpoint{2.388411in}{1.645040in}}%
\pgfpathlineto{\pgfqpoint{2.485350in}{1.590441in}}%
\pgfpathlineto{\pgfqpoint{2.582290in}{1.099000in}}%
\pgfpathlineto{\pgfqpoint{2.679230in}{1.068545in}}%
\pgfpathlineto{\pgfqpoint{2.776170in}{1.042839in}}%
\pgfpathlineto{\pgfqpoint{2.873110in}{1.015229in}}%
\pgfpathlineto{\pgfqpoint{2.970050in}{0.987073in}}%
\pgfpathlineto{\pgfqpoint{3.066990in}{0.956832in}}%
\pgfpathlineto{\pgfqpoint{3.163930in}{0.940333in}}%
\pgfpathlineto{\pgfqpoint{3.260870in}{0.918627in}}%
\pgfpathlineto{\pgfqpoint{3.357809in}{0.892209in}}%
\pgfpathlineto{\pgfqpoint{3.454749in}{0.859417in}}%
\pgfpathlineto{\pgfqpoint{3.551689in}{0.827284in}}%
\pgfpathlineto{\pgfqpoint{3.648629in}{0.823313in}}%
\pgfpathlineto{\pgfqpoint{3.745569in}{0.784214in}}%
\pgfpathlineto{\pgfqpoint{3.842509in}{0.760373in}}%
\pgfpathlineto{\pgfqpoint{3.939449in}{0.741153in}}%
\pgfpathlineto{\pgfqpoint{4.036389in}{0.708995in}}%
\pgfusepath{stroke}%
\end{pgfscope}%
\begin{pgfscope}%
\pgfpathrectangle{\pgfqpoint{0.740433in}{0.566590in}}{\pgfqpoint{3.295956in}{1.828724in}}%
\pgfusepath{clip}%
\pgfsetbuttcap%
\pgfsetroundjoin%
\definecolor{currentfill}{rgb}{0.850000,0.324000,0.098000}%
\pgfsetfillcolor{currentfill}%
\pgfsetlinewidth{1.003750pt}%
\definecolor{currentstroke}{rgb}{0.850000,0.324000,0.098000}%
\pgfsetstrokecolor{currentstroke}%
\pgfsetdash{}{0pt}%
\pgfsys@defobject{currentmarker}{\pgfqpoint{-0.041667in}{-0.041667in}}{\pgfqpoint{0.041667in}{0.041667in}}{%
\pgfpathmoveto{\pgfqpoint{-0.041667in}{0.000000in}}%
\pgfpathlineto{\pgfqpoint{0.041667in}{0.000000in}}%
\pgfpathmoveto{\pgfqpoint{0.000000in}{-0.041667in}}%
\pgfpathlineto{\pgfqpoint{0.000000in}{0.041667in}}%
\pgfusepath{stroke,fill}%
}%
\begin{pgfscope}%
\pgfsys@transformshift{0.740433in}{2.314635in}%
\pgfsys@useobject{currentmarker}{}%
\end{pgfscope}%
\begin{pgfscope}%
\pgfsys@transformshift{1.031252in}{2.301678in}%
\pgfsys@useobject{currentmarker}{}%
\end{pgfscope}%
\begin{pgfscope}%
\pgfsys@transformshift{1.322072in}{2.291717in}%
\pgfsys@useobject{currentmarker}{}%
\end{pgfscope}%
\begin{pgfscope}%
\pgfsys@transformshift{1.612892in}{2.276068in}%
\pgfsys@useobject{currentmarker}{}%
\end{pgfscope}%
\begin{pgfscope}%
\pgfsys@transformshift{1.903711in}{2.239430in}%
\pgfsys@useobject{currentmarker}{}%
\end{pgfscope}%
\begin{pgfscope}%
\pgfsys@transformshift{2.194531in}{2.082845in}%
\pgfsys@useobject{currentmarker}{}%
\end{pgfscope}%
\begin{pgfscope}%
\pgfsys@transformshift{2.485350in}{1.590441in}%
\pgfsys@useobject{currentmarker}{}%
\end{pgfscope}%
\begin{pgfscope}%
\pgfsys@transformshift{2.776170in}{1.042839in}%
\pgfsys@useobject{currentmarker}{}%
\end{pgfscope}%
\begin{pgfscope}%
\pgfsys@transformshift{3.066990in}{0.956832in}%
\pgfsys@useobject{currentmarker}{}%
\end{pgfscope}%
\begin{pgfscope}%
\pgfsys@transformshift{3.357809in}{0.892209in}%
\pgfsys@useobject{currentmarker}{}%
\end{pgfscope}%
\begin{pgfscope}%
\pgfsys@transformshift{3.648629in}{0.823313in}%
\pgfsys@useobject{currentmarker}{}%
\end{pgfscope}%
\begin{pgfscope}%
\pgfsys@transformshift{3.939449in}{0.741153in}%
\pgfsys@useobject{currentmarker}{}%
\end{pgfscope}%
\end{pgfscope}%
\begin{pgfscope}%
\pgfpathrectangle{\pgfqpoint{0.740433in}{0.566590in}}{\pgfqpoint{3.295956in}{1.828724in}}%
\pgfusepath{clip}%
\pgfsetbuttcap%
\pgfsetroundjoin%
\pgfsetlinewidth{1.505625pt}%
\definecolor{currentstroke}{rgb}{0.000000,0.500000,0.000000}%
\pgfsetstrokecolor{currentstroke}%
\pgfsetdash{{5.550000pt}{2.400000pt}}{0.000000pt}%
\pgfpathmoveto{\pgfqpoint{0.740433in}{2.344663in}}%
\pgfpathlineto{\pgfqpoint{0.837373in}{2.345530in}}%
\pgfpathlineto{\pgfqpoint{0.934312in}{2.341477in}}%
\pgfpathlineto{\pgfqpoint{1.031252in}{2.336653in}}%
\pgfpathlineto{\pgfqpoint{1.128192in}{2.330314in}}%
\pgfpathlineto{\pgfqpoint{1.225132in}{2.331603in}}%
\pgfpathlineto{\pgfqpoint{1.322072in}{2.344044in}}%
\pgfpathlineto{\pgfqpoint{1.419012in}{2.342969in}}%
\pgfpathlineto{\pgfqpoint{1.515952in}{2.343322in}}%
\pgfpathlineto{\pgfqpoint{1.612892in}{2.330138in}}%
\pgfpathlineto{\pgfqpoint{1.709831in}{2.334854in}}%
\pgfpathlineto{\pgfqpoint{1.806771in}{2.323933in}}%
\pgfpathlineto{\pgfqpoint{1.903711in}{2.326355in}}%
\pgfpathlineto{\pgfqpoint{2.000651in}{2.335024in}}%
\pgfpathlineto{\pgfqpoint{2.097591in}{2.317823in}}%
\pgfpathlineto{\pgfqpoint{2.194531in}{2.288219in}}%
\pgfpathlineto{\pgfqpoint{2.291471in}{2.274966in}}%
\pgfpathlineto{\pgfqpoint{2.388411in}{2.227364in}}%
\pgfpathlineto{\pgfqpoint{2.485350in}{2.178813in}}%
\pgfpathlineto{\pgfqpoint{2.582290in}{1.100723in}}%
\pgfpathlineto{\pgfqpoint{2.679230in}{1.076376in}}%
\pgfpathlineto{\pgfqpoint{2.776170in}{1.046779in}}%
\pgfpathlineto{\pgfqpoint{2.873110in}{1.012793in}}%
\pgfpathlineto{\pgfqpoint{2.970050in}{0.994583in}}%
\pgfpathlineto{\pgfqpoint{3.066990in}{0.961070in}}%
\pgfpathlineto{\pgfqpoint{3.163930in}{0.951636in}}%
\pgfpathlineto{\pgfqpoint{3.260870in}{0.915725in}}%
\pgfpathlineto{\pgfqpoint{3.357809in}{0.893123in}}%
\pgfpathlineto{\pgfqpoint{3.454749in}{0.862767in}}%
\pgfpathlineto{\pgfqpoint{3.551689in}{0.830661in}}%
\pgfpathlineto{\pgfqpoint{3.648629in}{0.821719in}}%
\pgfpathlineto{\pgfqpoint{3.745569in}{0.799555in}}%
\pgfpathlineto{\pgfqpoint{3.842509in}{0.762108in}}%
\pgfpathlineto{\pgfqpoint{3.939449in}{0.747116in}}%
\pgfpathlineto{\pgfqpoint{4.036389in}{0.710809in}}%
\pgfusepath{stroke}%
\end{pgfscope}%
\begin{pgfscope}%
\pgfpathrectangle{\pgfqpoint{0.740433in}{0.566590in}}{\pgfqpoint{3.295956in}{1.828724in}}%
\pgfusepath{clip}%
\pgfsetbuttcap%
\pgfsetmiterjoin%
\definecolor{currentfill}{rgb}{0.000000,0.000000,0.000000}%
\pgfsetfillcolor{currentfill}%
\pgfsetfillopacity{0.000000}%
\pgfsetlinewidth{1.003750pt}%
\definecolor{currentstroke}{rgb}{0.000000,0.500000,0.000000}%
\pgfsetstrokecolor{currentstroke}%
\pgfsetdash{}{0pt}%
\pgfsys@defobject{currentmarker}{\pgfqpoint{-0.041667in}{-0.041667in}}{\pgfqpoint{0.041667in}{0.041667in}}{%
\pgfpathmoveto{\pgfqpoint{-0.041667in}{-0.041667in}}%
\pgfpathlineto{\pgfqpoint{0.041667in}{-0.041667in}}%
\pgfpathlineto{\pgfqpoint{0.041667in}{0.041667in}}%
\pgfpathlineto{\pgfqpoint{-0.041667in}{0.041667in}}%
\pgfpathclose%
\pgfusepath{stroke,fill}%
}%
\begin{pgfscope}%
\pgfsys@transformshift{0.740433in}{2.344663in}%
\pgfsys@useobject{currentmarker}{}%
\end{pgfscope}%
\begin{pgfscope}%
\pgfsys@transformshift{1.225132in}{2.331603in}%
\pgfsys@useobject{currentmarker}{}%
\end{pgfscope}%
\begin{pgfscope}%
\pgfsys@transformshift{1.709831in}{2.334854in}%
\pgfsys@useobject{currentmarker}{}%
\end{pgfscope}%
\begin{pgfscope}%
\pgfsys@transformshift{2.194531in}{2.288219in}%
\pgfsys@useobject{currentmarker}{}%
\end{pgfscope}%
\begin{pgfscope}%
\pgfsys@transformshift{2.679230in}{1.076376in}%
\pgfsys@useobject{currentmarker}{}%
\end{pgfscope}%
\begin{pgfscope}%
\pgfsys@transformshift{3.163930in}{0.951636in}%
\pgfsys@useobject{currentmarker}{}%
\end{pgfscope}%
\begin{pgfscope}%
\pgfsys@transformshift{3.648629in}{0.821719in}%
\pgfsys@useobject{currentmarker}{}%
\end{pgfscope}%
\end{pgfscope}%
\begin{pgfscope}%
\pgfpathrectangle{\pgfqpoint{0.740433in}{0.566590in}}{\pgfqpoint{3.295956in}{1.828724in}}%
\pgfusepath{clip}%
\pgfsetrectcap%
\pgfsetroundjoin%
\pgfsetlinewidth{1.505625pt}%
\definecolor{currentstroke}{rgb}{0.000000,0.447000,0.741000}%
\pgfsetstrokecolor{currentstroke}%
\pgfsetdash{}{0pt}%
\pgfpathmoveto{\pgfqpoint{0.740433in}{1.771279in}}%
\pgfpathlineto{\pgfqpoint{0.975858in}{1.768525in}}%
\pgfpathlineto{\pgfqpoint{1.211283in}{1.757171in}}%
\pgfpathlineto{\pgfqpoint{1.446709in}{1.733266in}}%
\pgfpathlineto{\pgfqpoint{1.682134in}{1.652359in}}%
\pgfpathlineto{\pgfqpoint{1.917560in}{1.576557in}}%
\pgfpathlineto{\pgfqpoint{2.152985in}{1.517334in}}%
\pgfpathlineto{\pgfqpoint{2.388411in}{1.412251in}}%
\pgfpathlineto{\pgfqpoint{2.623836in}{1.020601in}}%
\pgfpathlineto{\pgfqpoint{2.859261in}{0.952321in}}%
\pgfpathlineto{\pgfqpoint{3.094687in}{0.885705in}}%
\pgfpathlineto{\pgfqpoint{3.330112in}{0.828928in}}%
\pgfpathlineto{\pgfqpoint{3.565538in}{0.766360in}}%
\pgfpathlineto{\pgfqpoint{3.800963in}{0.694691in}}%
\pgfpathlineto{\pgfqpoint{4.036389in}{0.643816in}}%
\pgfusepath{stroke}%
\end{pgfscope}%
\begin{pgfscope}%
\pgfpathrectangle{\pgfqpoint{0.740433in}{0.566590in}}{\pgfqpoint{3.295956in}{1.828724in}}%
\pgfusepath{clip}%
\pgfsetbuttcap%
\pgfsetroundjoin%
\definecolor{currentfill}{rgb}{0.000000,0.000000,0.000000}%
\pgfsetfillcolor{currentfill}%
\pgfsetfillopacity{0.000000}%
\pgfsetlinewidth{1.003750pt}%
\definecolor{currentstroke}{rgb}{0.000000,0.447000,0.741000}%
\pgfsetstrokecolor{currentstroke}%
\pgfsetdash{}{0pt}%
\pgfsys@defobject{currentmarker}{\pgfqpoint{-0.041667in}{-0.041667in}}{\pgfqpoint{0.041667in}{0.041667in}}{%
\pgfpathmoveto{\pgfqpoint{0.000000in}{-0.041667in}}%
\pgfpathcurveto{\pgfqpoint{0.011050in}{-0.041667in}}{\pgfqpoint{0.021649in}{-0.037276in}}{\pgfqpoint{0.029463in}{-0.029463in}}%
\pgfpathcurveto{\pgfqpoint{0.037276in}{-0.021649in}}{\pgfqpoint{0.041667in}{-0.011050in}}{\pgfqpoint{0.041667in}{0.000000in}}%
\pgfpathcurveto{\pgfqpoint{0.041667in}{0.011050in}}{\pgfqpoint{0.037276in}{0.021649in}}{\pgfqpoint{0.029463in}{0.029463in}}%
\pgfpathcurveto{\pgfqpoint{0.021649in}{0.037276in}}{\pgfqpoint{0.011050in}{0.041667in}}{\pgfqpoint{0.000000in}{0.041667in}}%
\pgfpathcurveto{\pgfqpoint{-0.011050in}{0.041667in}}{\pgfqpoint{-0.021649in}{0.037276in}}{\pgfqpoint{-0.029463in}{0.029463in}}%
\pgfpathcurveto{\pgfqpoint{-0.037276in}{0.021649in}}{\pgfqpoint{-0.041667in}{0.011050in}}{\pgfqpoint{-0.041667in}{0.000000in}}%
\pgfpathcurveto{\pgfqpoint{-0.041667in}{-0.011050in}}{\pgfqpoint{-0.037276in}{-0.021649in}}{\pgfqpoint{-0.029463in}{-0.029463in}}%
\pgfpathcurveto{\pgfqpoint{-0.021649in}{-0.037276in}}{\pgfqpoint{-0.011050in}{-0.041667in}}{\pgfqpoint{0.000000in}{-0.041667in}}%
\pgfpathclose%
\pgfusepath{stroke,fill}%
}%
\begin{pgfscope}%
\pgfsys@transformshift{0.740433in}{1.771279in}%
\pgfsys@useobject{currentmarker}{}%
\end{pgfscope}%
\begin{pgfscope}%
\pgfsys@transformshift{0.975858in}{1.768525in}%
\pgfsys@useobject{currentmarker}{}%
\end{pgfscope}%
\begin{pgfscope}%
\pgfsys@transformshift{1.211283in}{1.757171in}%
\pgfsys@useobject{currentmarker}{}%
\end{pgfscope}%
\begin{pgfscope}%
\pgfsys@transformshift{1.446709in}{1.733266in}%
\pgfsys@useobject{currentmarker}{}%
\end{pgfscope}%
\begin{pgfscope}%
\pgfsys@transformshift{1.682134in}{1.652359in}%
\pgfsys@useobject{currentmarker}{}%
\end{pgfscope}%
\begin{pgfscope}%
\pgfsys@transformshift{1.917560in}{1.576557in}%
\pgfsys@useobject{currentmarker}{}%
\end{pgfscope}%
\begin{pgfscope}%
\pgfsys@transformshift{2.152985in}{1.517334in}%
\pgfsys@useobject{currentmarker}{}%
\end{pgfscope}%
\begin{pgfscope}%
\pgfsys@transformshift{2.388411in}{1.412251in}%
\pgfsys@useobject{currentmarker}{}%
\end{pgfscope}%
\begin{pgfscope}%
\pgfsys@transformshift{2.623836in}{1.020601in}%
\pgfsys@useobject{currentmarker}{}%
\end{pgfscope}%
\begin{pgfscope}%
\pgfsys@transformshift{2.859261in}{0.952321in}%
\pgfsys@useobject{currentmarker}{}%
\end{pgfscope}%
\begin{pgfscope}%
\pgfsys@transformshift{3.094687in}{0.885705in}%
\pgfsys@useobject{currentmarker}{}%
\end{pgfscope}%
\begin{pgfscope}%
\pgfsys@transformshift{3.330112in}{0.828928in}%
\pgfsys@useobject{currentmarker}{}%
\end{pgfscope}%
\begin{pgfscope}%
\pgfsys@transformshift{3.565538in}{0.766360in}%
\pgfsys@useobject{currentmarker}{}%
\end{pgfscope}%
\begin{pgfscope}%
\pgfsys@transformshift{3.800963in}{0.694691in}%
\pgfsys@useobject{currentmarker}{}%
\end{pgfscope}%
\begin{pgfscope}%
\pgfsys@transformshift{4.036389in}{0.643816in}%
\pgfsys@useobject{currentmarker}{}%
\end{pgfscope}%
\end{pgfscope}%
\begin{pgfscope}%
\pgfpathrectangle{\pgfqpoint{0.740433in}{0.566590in}}{\pgfqpoint{3.295956in}{1.828724in}}%
\pgfusepath{clip}%
\pgfsetrectcap%
\pgfsetroundjoin%
\pgfsetlinewidth{1.505625pt}%
\definecolor{currentstroke}{rgb}{0.850000,0.324000,0.098000}%
\pgfsetstrokecolor{currentstroke}%
\pgfsetdash{}{0pt}%
\pgfpathmoveto{\pgfqpoint{0.740433in}{1.718188in}}%
\pgfpathlineto{\pgfqpoint{0.975858in}{1.701540in}}%
\pgfpathlineto{\pgfqpoint{1.211283in}{1.673331in}}%
\pgfpathlineto{\pgfqpoint{1.446709in}{1.597889in}}%
\pgfpathlineto{\pgfqpoint{1.682134in}{1.490460in}}%
\pgfpathlineto{\pgfqpoint{1.917560in}{1.360207in}}%
\pgfpathlineto{\pgfqpoint{2.152985in}{1.293613in}}%
\pgfpathlineto{\pgfqpoint{2.388411in}{1.185020in}}%
\pgfpathlineto{\pgfqpoint{2.623836in}{1.090951in}}%
\pgfpathlineto{\pgfqpoint{2.859261in}{1.017378in}}%
\pgfpathlineto{\pgfqpoint{3.094687in}{0.960586in}}%
\pgfpathlineto{\pgfqpoint{3.330112in}{0.891041in}}%
\pgfpathlineto{\pgfqpoint{3.565538in}{0.823545in}}%
\pgfpathlineto{\pgfqpoint{3.800963in}{0.760829in}}%
\pgfpathlineto{\pgfqpoint{4.036389in}{0.696981in}}%
\pgfusepath{stroke}%
\end{pgfscope}%
\begin{pgfscope}%
\pgfpathrectangle{\pgfqpoint{0.740433in}{0.566590in}}{\pgfqpoint{3.295956in}{1.828724in}}%
\pgfusepath{clip}%
\pgfsetbuttcap%
\pgfsetroundjoin%
\definecolor{currentfill}{rgb}{0.850000,0.324000,0.098000}%
\pgfsetfillcolor{currentfill}%
\pgfsetlinewidth{1.003750pt}%
\definecolor{currentstroke}{rgb}{0.850000,0.324000,0.098000}%
\pgfsetstrokecolor{currentstroke}%
\pgfsetdash{}{0pt}%
\pgfsys@defobject{currentmarker}{\pgfqpoint{-0.041667in}{-0.041667in}}{\pgfqpoint{0.041667in}{0.041667in}}{%
\pgfpathmoveto{\pgfqpoint{-0.041667in}{0.000000in}}%
\pgfpathlineto{\pgfqpoint{0.041667in}{0.000000in}}%
\pgfpathmoveto{\pgfqpoint{0.000000in}{-0.041667in}}%
\pgfpathlineto{\pgfqpoint{0.000000in}{0.041667in}}%
\pgfusepath{stroke,fill}%
}%
\begin{pgfscope}%
\pgfsys@transformshift{0.740433in}{1.718188in}%
\pgfsys@useobject{currentmarker}{}%
\end{pgfscope}%
\begin{pgfscope}%
\pgfsys@transformshift{0.975858in}{1.701540in}%
\pgfsys@useobject{currentmarker}{}%
\end{pgfscope}%
\begin{pgfscope}%
\pgfsys@transformshift{1.211283in}{1.673331in}%
\pgfsys@useobject{currentmarker}{}%
\end{pgfscope}%
\begin{pgfscope}%
\pgfsys@transformshift{1.446709in}{1.597889in}%
\pgfsys@useobject{currentmarker}{}%
\end{pgfscope}%
\begin{pgfscope}%
\pgfsys@transformshift{1.682134in}{1.490460in}%
\pgfsys@useobject{currentmarker}{}%
\end{pgfscope}%
\begin{pgfscope}%
\pgfsys@transformshift{1.917560in}{1.360207in}%
\pgfsys@useobject{currentmarker}{}%
\end{pgfscope}%
\begin{pgfscope}%
\pgfsys@transformshift{2.152985in}{1.293613in}%
\pgfsys@useobject{currentmarker}{}%
\end{pgfscope}%
\begin{pgfscope}%
\pgfsys@transformshift{2.388411in}{1.185020in}%
\pgfsys@useobject{currentmarker}{}%
\end{pgfscope}%
\begin{pgfscope}%
\pgfsys@transformshift{2.623836in}{1.090951in}%
\pgfsys@useobject{currentmarker}{}%
\end{pgfscope}%
\begin{pgfscope}%
\pgfsys@transformshift{2.859261in}{1.017378in}%
\pgfsys@useobject{currentmarker}{}%
\end{pgfscope}%
\begin{pgfscope}%
\pgfsys@transformshift{3.094687in}{0.960586in}%
\pgfsys@useobject{currentmarker}{}%
\end{pgfscope}%
\begin{pgfscope}%
\pgfsys@transformshift{3.330112in}{0.891041in}%
\pgfsys@useobject{currentmarker}{}%
\end{pgfscope}%
\begin{pgfscope}%
\pgfsys@transformshift{3.565538in}{0.823545in}%
\pgfsys@useobject{currentmarker}{}%
\end{pgfscope}%
\begin{pgfscope}%
\pgfsys@transformshift{3.800963in}{0.760829in}%
\pgfsys@useobject{currentmarker}{}%
\end{pgfscope}%
\begin{pgfscope}%
\pgfsys@transformshift{4.036389in}{0.696981in}%
\pgfsys@useobject{currentmarker}{}%
\end{pgfscope}%
\end{pgfscope}%
\begin{pgfscope}%
\pgfpathrectangle{\pgfqpoint{0.740433in}{0.566590in}}{\pgfqpoint{3.295956in}{1.828724in}}%
\pgfusepath{clip}%
\pgfsetrectcap%
\pgfsetroundjoin%
\pgfsetlinewidth{1.505625pt}%
\definecolor{currentstroke}{rgb}{0.000000,0.500000,0.000000}%
\pgfsetstrokecolor{currentstroke}%
\pgfsetdash{}{0pt}%
\pgfpathmoveto{\pgfqpoint{0.740433in}{1.653339in}}%
\pgfpathlineto{\pgfqpoint{0.975858in}{1.629762in}}%
\pgfpathlineto{\pgfqpoint{1.211283in}{1.599617in}}%
\pgfpathlineto{\pgfqpoint{1.446709in}{1.478609in}}%
\pgfpathlineto{\pgfqpoint{1.682134in}{1.419166in}}%
\pgfpathlineto{\pgfqpoint{1.917560in}{1.363022in}}%
\pgfpathlineto{\pgfqpoint{2.152985in}{1.303748in}}%
\pgfpathlineto{\pgfqpoint{2.388411in}{1.185428in}}%
\pgfpathlineto{\pgfqpoint{2.623836in}{1.086837in}}%
\pgfpathlineto{\pgfqpoint{2.859261in}{1.022575in}}%
\pgfpathlineto{\pgfqpoint{3.094687in}{0.960870in}}%
\pgfpathlineto{\pgfqpoint{3.330112in}{0.888343in}}%
\pgfpathlineto{\pgfqpoint{3.565538in}{0.827446in}}%
\pgfpathlineto{\pgfqpoint{3.800963in}{0.766103in}}%
\pgfpathlineto{\pgfqpoint{4.036389in}{0.697507in}}%
\pgfusepath{stroke}%
\end{pgfscope}%
\begin{pgfscope}%
\pgfpathrectangle{\pgfqpoint{0.740433in}{0.566590in}}{\pgfqpoint{3.295956in}{1.828724in}}%
\pgfusepath{clip}%
\pgfsetbuttcap%
\pgfsetmiterjoin%
\definecolor{currentfill}{rgb}{0.000000,0.000000,0.000000}%
\pgfsetfillcolor{currentfill}%
\pgfsetfillopacity{0.000000}%
\pgfsetlinewidth{1.003750pt}%
\definecolor{currentstroke}{rgb}{0.000000,0.500000,0.000000}%
\pgfsetstrokecolor{currentstroke}%
\pgfsetdash{}{0pt}%
\pgfsys@defobject{currentmarker}{\pgfqpoint{-0.041667in}{-0.041667in}}{\pgfqpoint{0.041667in}{0.041667in}}{%
\pgfpathmoveto{\pgfqpoint{-0.041667in}{-0.041667in}}%
\pgfpathlineto{\pgfqpoint{0.041667in}{-0.041667in}}%
\pgfpathlineto{\pgfqpoint{0.041667in}{0.041667in}}%
\pgfpathlineto{\pgfqpoint{-0.041667in}{0.041667in}}%
\pgfpathclose%
\pgfusepath{stroke,fill}%
}%
\begin{pgfscope}%
\pgfsys@transformshift{0.740433in}{1.653339in}%
\pgfsys@useobject{currentmarker}{}%
\end{pgfscope}%
\begin{pgfscope}%
\pgfsys@transformshift{0.975858in}{1.629762in}%
\pgfsys@useobject{currentmarker}{}%
\end{pgfscope}%
\begin{pgfscope}%
\pgfsys@transformshift{1.211283in}{1.599617in}%
\pgfsys@useobject{currentmarker}{}%
\end{pgfscope}%
\begin{pgfscope}%
\pgfsys@transformshift{1.446709in}{1.478609in}%
\pgfsys@useobject{currentmarker}{}%
\end{pgfscope}%
\begin{pgfscope}%
\pgfsys@transformshift{1.682134in}{1.419166in}%
\pgfsys@useobject{currentmarker}{}%
\end{pgfscope}%
\begin{pgfscope}%
\pgfsys@transformshift{1.917560in}{1.363022in}%
\pgfsys@useobject{currentmarker}{}%
\end{pgfscope}%
\begin{pgfscope}%
\pgfsys@transformshift{2.152985in}{1.303748in}%
\pgfsys@useobject{currentmarker}{}%
\end{pgfscope}%
\begin{pgfscope}%
\pgfsys@transformshift{2.388411in}{1.185428in}%
\pgfsys@useobject{currentmarker}{}%
\end{pgfscope}%
\begin{pgfscope}%
\pgfsys@transformshift{2.623836in}{1.086837in}%
\pgfsys@useobject{currentmarker}{}%
\end{pgfscope}%
\begin{pgfscope}%
\pgfsys@transformshift{2.859261in}{1.022575in}%
\pgfsys@useobject{currentmarker}{}%
\end{pgfscope}%
\begin{pgfscope}%
\pgfsys@transformshift{3.094687in}{0.960870in}%
\pgfsys@useobject{currentmarker}{}%
\end{pgfscope}%
\begin{pgfscope}%
\pgfsys@transformshift{3.330112in}{0.888343in}%
\pgfsys@useobject{currentmarker}{}%
\end{pgfscope}%
\begin{pgfscope}%
\pgfsys@transformshift{3.565538in}{0.827446in}%
\pgfsys@useobject{currentmarker}{}%
\end{pgfscope}%
\begin{pgfscope}%
\pgfsys@transformshift{3.800963in}{0.766103in}%
\pgfsys@useobject{currentmarker}{}%
\end{pgfscope}%
\begin{pgfscope}%
\pgfsys@transformshift{4.036389in}{0.697507in}%
\pgfsys@useobject{currentmarker}{}%
\end{pgfscope}%
\end{pgfscope}%
\begin{pgfscope}%
\pgfsetrectcap%
\pgfsetmiterjoin%
\pgfsetlinewidth{0.803000pt}%
\definecolor{currentstroke}{rgb}{0.000000,0.000000,0.000000}%
\pgfsetstrokecolor{currentstroke}%
\pgfsetdash{}{0pt}%
\pgfpathmoveto{\pgfqpoint{0.740433in}{0.566590in}}%
\pgfpathlineto{\pgfqpoint{0.740433in}{2.395314in}}%
\pgfusepath{stroke}%
\end{pgfscope}%
\begin{pgfscope}%
\pgfsetrectcap%
\pgfsetmiterjoin%
\pgfsetlinewidth{0.803000pt}%
\definecolor{currentstroke}{rgb}{0.000000,0.000000,0.000000}%
\pgfsetstrokecolor{currentstroke}%
\pgfsetdash{}{0pt}%
\pgfpathmoveto{\pgfqpoint{4.036389in}{0.566590in}}%
\pgfpathlineto{\pgfqpoint{4.036389in}{2.395314in}}%
\pgfusepath{stroke}%
\end{pgfscope}%
\begin{pgfscope}%
\pgfsetrectcap%
\pgfsetmiterjoin%
\pgfsetlinewidth{0.803000pt}%
\definecolor{currentstroke}{rgb}{0.000000,0.000000,0.000000}%
\pgfsetstrokecolor{currentstroke}%
\pgfsetdash{}{0pt}%
\pgfpathmoveto{\pgfqpoint{0.740433in}{0.566590in}}%
\pgfpathlineto{\pgfqpoint{4.036389in}{0.566590in}}%
\pgfusepath{stroke}%
\end{pgfscope}%
\begin{pgfscope}%
\pgfsetrectcap%
\pgfsetmiterjoin%
\pgfsetlinewidth{0.803000pt}%
\definecolor{currentstroke}{rgb}{0.000000,0.000000,0.000000}%
\pgfsetstrokecolor{currentstroke}%
\pgfsetdash{}{0pt}%
\pgfpathmoveto{\pgfqpoint{0.740433in}{2.395314in}}%
\pgfpathlineto{\pgfqpoint{4.036389in}{2.395314in}}%
\pgfusepath{stroke}%
\end{pgfscope}%
\begin{pgfscope}%
\pgfsetbuttcap%
\pgfsetmiterjoin%
\definecolor{currentfill}{rgb}{1.000000,1.000000,1.000000}%
\pgfsetfillcolor{currentfill}%
\pgfsetfillopacity{0.800000}%
\pgfsetlinewidth{1.003750pt}%
\definecolor{currentstroke}{rgb}{0.800000,0.800000,0.800000}%
\pgfsetstrokecolor{currentstroke}%
\pgfsetstrokeopacity{0.800000}%
\pgfsetdash{}{0pt}%
\pgfpathmoveto{\pgfqpoint{2.841524in}{1.772415in}}%
\pgfpathlineto{\pgfqpoint{3.948889in}{1.772415in}}%
\pgfpathquadraticcurveto{\pgfqpoint{3.973889in}{1.772415in}}{\pgfqpoint{3.973889in}{1.797415in}}%
\pgfpathlineto{\pgfqpoint{3.973889in}{2.307814in}}%
\pgfpathquadraticcurveto{\pgfqpoint{3.973889in}{2.332814in}}{\pgfqpoint{3.948889in}{2.332814in}}%
\pgfpathlineto{\pgfqpoint{2.841524in}{2.332814in}}%
\pgfpathquadraticcurveto{\pgfqpoint{2.816524in}{2.332814in}}{\pgfqpoint{2.816524in}{2.307814in}}%
\pgfpathlineto{\pgfqpoint{2.816524in}{1.797415in}}%
\pgfpathquadraticcurveto{\pgfqpoint{2.816524in}{1.772415in}}{\pgfqpoint{2.841524in}{1.772415in}}%
\pgfpathclose%
\pgfusepath{stroke,fill}%
\end{pgfscope}%
\begin{pgfscope}%
\pgfsetbuttcap%
\pgfsetroundjoin%
\definecolor{currentfill}{rgb}{0.000000,0.000000,0.000000}%
\pgfsetfillcolor{currentfill}%
\pgfsetfillopacity{0.000000}%
\pgfsetlinewidth{1.003750pt}%
\definecolor{currentstroke}{rgb}{0.000000,0.447000,0.741000}%
\pgfsetstrokecolor{currentstroke}%
\pgfsetdash{}{0pt}%
\pgfsys@defobject{currentmarker}{\pgfqpoint{-0.041667in}{-0.041667in}}{\pgfqpoint{0.041667in}{0.041667in}}{%
\pgfpathmoveto{\pgfqpoint{0.000000in}{-0.041667in}}%
\pgfpathcurveto{\pgfqpoint{0.011050in}{-0.041667in}}{\pgfqpoint{0.021649in}{-0.037276in}}{\pgfqpoint{0.029463in}{-0.029463in}}%
\pgfpathcurveto{\pgfqpoint{0.037276in}{-0.021649in}}{\pgfqpoint{0.041667in}{-0.011050in}}{\pgfqpoint{0.041667in}{0.000000in}}%
\pgfpathcurveto{\pgfqpoint{0.041667in}{0.011050in}}{\pgfqpoint{0.037276in}{0.021649in}}{\pgfqpoint{0.029463in}{0.029463in}}%
\pgfpathcurveto{\pgfqpoint{0.021649in}{0.037276in}}{\pgfqpoint{0.011050in}{0.041667in}}{\pgfqpoint{0.000000in}{0.041667in}}%
\pgfpathcurveto{\pgfqpoint{-0.011050in}{0.041667in}}{\pgfqpoint{-0.021649in}{0.037276in}}{\pgfqpoint{-0.029463in}{0.029463in}}%
\pgfpathcurveto{\pgfqpoint{-0.037276in}{0.021649in}}{\pgfqpoint{-0.041667in}{0.011050in}}{\pgfqpoint{-0.041667in}{0.000000in}}%
\pgfpathcurveto{\pgfqpoint{-0.041667in}{-0.011050in}}{\pgfqpoint{-0.037276in}{-0.021649in}}{\pgfqpoint{-0.029463in}{-0.029463in}}%
\pgfpathcurveto{\pgfqpoint{-0.021649in}{-0.037276in}}{\pgfqpoint{-0.011050in}{-0.041667in}}{\pgfqpoint{0.000000in}{-0.041667in}}%
\pgfpathclose%
\pgfusepath{stroke,fill}%
}%
\begin{pgfscope}%
\pgfsys@transformshift{2.991524in}{2.239064in}%
\pgfsys@useobject{currentmarker}{}%
\end{pgfscope}%
\end{pgfscope}%
\begin{pgfscope}%
\definecolor{textcolor}{rgb}{0.000000,0.000000,0.000000}%
\pgfsetstrokecolor{textcolor}%
\pgfsetfillcolor{textcolor}%
\pgftext[x=3.216524in,y=2.195314in,left,base]{\color{textcolor}\rmfamily\fontsize{9.000000}{10.800000}\selectfont \(\displaystyle \nu_{18} = \) 124.01}%
\end{pgfscope}%
\begin{pgfscope}%
\pgfsetbuttcap%
\pgfsetroundjoin%
\definecolor{currentfill}{rgb}{0.850000,0.324000,0.098000}%
\pgfsetfillcolor{currentfill}%
\pgfsetlinewidth{1.003750pt}%
\definecolor{currentstroke}{rgb}{0.850000,0.324000,0.098000}%
\pgfsetstrokecolor{currentstroke}%
\pgfsetdash{}{0pt}%
\pgfsys@defobject{currentmarker}{\pgfqpoint{-0.041667in}{-0.041667in}}{\pgfqpoint{0.041667in}{0.041667in}}{%
\pgfpathmoveto{\pgfqpoint{-0.041667in}{0.000000in}}%
\pgfpathlineto{\pgfqpoint{0.041667in}{0.000000in}}%
\pgfpathmoveto{\pgfqpoint{0.000000in}{-0.041667in}}%
\pgfpathlineto{\pgfqpoint{0.000000in}{0.041667in}}%
\pgfusepath{stroke,fill}%
}%
\begin{pgfscope}%
\pgfsys@transformshift{2.991524in}{2.064765in}%
\pgfsys@useobject{currentmarker}{}%
\end{pgfscope}%
\end{pgfscope}%
\begin{pgfscope}%
\definecolor{textcolor}{rgb}{0.000000,0.000000,0.000000}%
\pgfsetstrokecolor{textcolor}%
\pgfsetfillcolor{textcolor}%
\pgftext[x=3.216524in,y=2.021015in,left,base]{\color{textcolor}\rmfamily\fontsize{9.000000}{10.800000}\selectfont \(\displaystyle \nu_{19} = \) 125.62}%
\end{pgfscope}%
\begin{pgfscope}%
\pgfsetbuttcap%
\pgfsetmiterjoin%
\definecolor{currentfill}{rgb}{0.000000,0.000000,0.000000}%
\pgfsetfillcolor{currentfill}%
\pgfsetfillopacity{0.000000}%
\pgfsetlinewidth{1.003750pt}%
\definecolor{currentstroke}{rgb}{0.000000,0.500000,0.000000}%
\pgfsetstrokecolor{currentstroke}%
\pgfsetdash{}{0pt}%
\pgfsys@defobject{currentmarker}{\pgfqpoint{-0.041667in}{-0.041667in}}{\pgfqpoint{0.041667in}{0.041667in}}{%
\pgfpathmoveto{\pgfqpoint{-0.041667in}{-0.041667in}}%
\pgfpathlineto{\pgfqpoint{0.041667in}{-0.041667in}}%
\pgfpathlineto{\pgfqpoint{0.041667in}{0.041667in}}%
\pgfpathlineto{\pgfqpoint{-0.041667in}{0.041667in}}%
\pgfpathclose%
\pgfusepath{stroke,fill}%
}%
\begin{pgfscope}%
\pgfsys@transformshift{2.991524in}{1.890465in}%
\pgfsys@useobject{currentmarker}{}%
\end{pgfscope}%
\end{pgfscope}%
\begin{pgfscope}%
\definecolor{textcolor}{rgb}{0.000000,0.000000,0.000000}%
\pgfsetstrokecolor{textcolor}%
\pgfsetfillcolor{textcolor}%
\pgftext[x=3.216524in,y=1.846715in,left,base]{\color{textcolor}\rmfamily\fontsize{9.000000}{10.800000}\selectfont \(\displaystyle \nu_{20} = \) 126.98}%
\end{pgfscope}%
\end{pgfpicture}%
\makeatother%
\endgroup%
}
					\caption{Cluster IV $(b)$}
					\label{SubFig:Cluster_V_imag}
				\end{subfigure}
				\begin{subfigure}[h]{0.5\textwidth}
					\centering
					\resizebox{\linewidth}{!}{%% Creator: Matplotlib, PGF backend
%%
%% To include the figure in your LaTeX document, write
%%   \input{<filename>.pgf}
%%
%% Make sure the required packages are loaded in your preamble
%%   \usepackage{pgf}
%%
%% and, on pdftex
%%   \usepackage[utf8]{inputenc}\DeclareUnicodeCharacter{2212}{-}
%%
%% or, on luatex and xetex
%%   \usepackage{unicode-math}
%%
%% Figures using additional raster images can only be included by \input if
%% they are in the same directory as the main LaTeX file. For loading figures
%% from other directories you can use the `import` package
%%   \usepackage{import}
%%
%% and then include the figures with
%%   \import{<path to file>}{<filename>.pgf}
%%
%% Matplotlib used the following preamble
%%   \usepackage[utf8x]{inputenc}
%%   \usepackage[T1]{fontenc}
%%   \usepackage{amsmath,amssymb,amsfonts}
%%
\begingroup%
\makeatletter%
\begin{pgfpicture}%
\pgfpathrectangle{\pgfpointorigin}{\pgfqpoint{4.136389in}{2.495314in}}%
\pgfusepath{use as bounding box, clip}%
\begin{pgfscope}%
\pgfsetbuttcap%
\pgfsetmiterjoin%
\definecolor{currentfill}{rgb}{1.000000,1.000000,1.000000}%
\pgfsetfillcolor{currentfill}%
\pgfsetlinewidth{0.000000pt}%
\definecolor{currentstroke}{rgb}{1.000000,1.000000,1.000000}%
\pgfsetstrokecolor{currentstroke}%
\pgfsetdash{}{0pt}%
\pgfpathmoveto{\pgfqpoint{-0.000000in}{0.000000in}}%
\pgfpathlineto{\pgfqpoint{4.136389in}{0.000000in}}%
\pgfpathlineto{\pgfqpoint{4.136389in}{2.495314in}}%
\pgfpathlineto{\pgfqpoint{-0.000000in}{2.495314in}}%
\pgfpathclose%
\pgfusepath{fill}%
\end{pgfscope}%
\begin{pgfscope}%
\pgfsetbuttcap%
\pgfsetmiterjoin%
\definecolor{currentfill}{rgb}{1.000000,1.000000,1.000000}%
\pgfsetfillcolor{currentfill}%
\pgfsetlinewidth{0.000000pt}%
\definecolor{currentstroke}{rgb}{0.000000,0.000000,0.000000}%
\pgfsetstrokecolor{currentstroke}%
\pgfsetstrokeopacity{0.000000}%
\pgfsetdash{}{0pt}%
\pgfpathmoveto{\pgfqpoint{0.740433in}{0.566590in}}%
\pgfpathlineto{\pgfqpoint{4.036389in}{0.566590in}}%
\pgfpathlineto{\pgfqpoint{4.036389in}{2.395314in}}%
\pgfpathlineto{\pgfqpoint{0.740433in}{2.395314in}}%
\pgfpathclose%
\pgfusepath{fill}%
\end{pgfscope}%
\begin{pgfscope}%
\pgfpathrectangle{\pgfqpoint{0.740433in}{0.566590in}}{\pgfqpoint{3.295956in}{1.828724in}}%
\pgfusepath{clip}%
\pgfsetrectcap%
\pgfsetroundjoin%
\pgfsetlinewidth{0.803000pt}%
\definecolor{currentstroke}{rgb}{0.690196,0.690196,0.690196}%
\pgfsetstrokecolor{currentstroke}%
\pgfsetdash{}{0pt}%
\pgfpathmoveto{\pgfqpoint{0.740433in}{0.566590in}}%
\pgfpathlineto{\pgfqpoint{0.740433in}{2.395314in}}%
\pgfusepath{stroke}%
\end{pgfscope}%
\begin{pgfscope}%
\pgfsetbuttcap%
\pgfsetroundjoin%
\definecolor{currentfill}{rgb}{0.000000,0.000000,0.000000}%
\pgfsetfillcolor{currentfill}%
\pgfsetlinewidth{0.803000pt}%
\definecolor{currentstroke}{rgb}{0.000000,0.000000,0.000000}%
\pgfsetstrokecolor{currentstroke}%
\pgfsetdash{}{0pt}%
\pgfsys@defobject{currentmarker}{\pgfqpoint{0.000000in}{-0.048611in}}{\pgfqpoint{0.000000in}{0.000000in}}{%
\pgfpathmoveto{\pgfqpoint{0.000000in}{0.000000in}}%
\pgfpathlineto{\pgfqpoint{0.000000in}{-0.048611in}}%
\pgfusepath{stroke,fill}%
}%
\begin{pgfscope}%
\pgfsys@transformshift{0.740433in}{0.566590in}%
\pgfsys@useobject{currentmarker}{}%
\end{pgfscope}%
\end{pgfscope}%
\begin{pgfscope}%
\definecolor{textcolor}{rgb}{0.000000,0.000000,0.000000}%
\pgfsetstrokecolor{textcolor}%
\pgfsetfillcolor{textcolor}%
\pgftext[x=0.740433in,y=0.469368in,,top]{\color{textcolor}\rmfamily\fontsize{12.000000}{14.400000}\selectfont \(\displaystyle {-10}\)}%
\end{pgfscope}%
\begin{pgfscope}%
\pgfpathrectangle{\pgfqpoint{0.740433in}{0.566590in}}{\pgfqpoint{3.295956in}{1.828724in}}%
\pgfusepath{clip}%
\pgfsetrectcap%
\pgfsetroundjoin%
\pgfsetlinewidth{0.803000pt}%
\definecolor{currentstroke}{rgb}{0.690196,0.690196,0.690196}%
\pgfsetstrokecolor{currentstroke}%
\pgfsetdash{}{0pt}%
\pgfpathmoveto{\pgfqpoint{1.247503in}{0.566590in}}%
\pgfpathlineto{\pgfqpoint{1.247503in}{2.395314in}}%
\pgfusepath{stroke}%
\end{pgfscope}%
\begin{pgfscope}%
\pgfsetbuttcap%
\pgfsetroundjoin%
\definecolor{currentfill}{rgb}{0.000000,0.000000,0.000000}%
\pgfsetfillcolor{currentfill}%
\pgfsetlinewidth{0.803000pt}%
\definecolor{currentstroke}{rgb}{0.000000,0.000000,0.000000}%
\pgfsetstrokecolor{currentstroke}%
\pgfsetdash{}{0pt}%
\pgfsys@defobject{currentmarker}{\pgfqpoint{0.000000in}{-0.048611in}}{\pgfqpoint{0.000000in}{0.000000in}}{%
\pgfpathmoveto{\pgfqpoint{0.000000in}{0.000000in}}%
\pgfpathlineto{\pgfqpoint{0.000000in}{-0.048611in}}%
\pgfusepath{stroke,fill}%
}%
\begin{pgfscope}%
\pgfsys@transformshift{1.247503in}{0.566590in}%
\pgfsys@useobject{currentmarker}{}%
\end{pgfscope}%
\end{pgfscope}%
\begin{pgfscope}%
\definecolor{textcolor}{rgb}{0.000000,0.000000,0.000000}%
\pgfsetstrokecolor{textcolor}%
\pgfsetfillcolor{textcolor}%
\pgftext[x=1.247503in,y=0.469368in,,top]{\color{textcolor}\rmfamily\fontsize{12.000000}{14.400000}\selectfont \(\displaystyle {0}\)}%
\end{pgfscope}%
\begin{pgfscope}%
\pgfpathrectangle{\pgfqpoint{0.740433in}{0.566590in}}{\pgfqpoint{3.295956in}{1.828724in}}%
\pgfusepath{clip}%
\pgfsetrectcap%
\pgfsetroundjoin%
\pgfsetlinewidth{0.803000pt}%
\definecolor{currentstroke}{rgb}{0.690196,0.690196,0.690196}%
\pgfsetstrokecolor{currentstroke}%
\pgfsetdash{}{0pt}%
\pgfpathmoveto{\pgfqpoint{1.754573in}{0.566590in}}%
\pgfpathlineto{\pgfqpoint{1.754573in}{2.395314in}}%
\pgfusepath{stroke}%
\end{pgfscope}%
\begin{pgfscope}%
\pgfsetbuttcap%
\pgfsetroundjoin%
\definecolor{currentfill}{rgb}{0.000000,0.000000,0.000000}%
\pgfsetfillcolor{currentfill}%
\pgfsetlinewidth{0.803000pt}%
\definecolor{currentstroke}{rgb}{0.000000,0.000000,0.000000}%
\pgfsetstrokecolor{currentstroke}%
\pgfsetdash{}{0pt}%
\pgfsys@defobject{currentmarker}{\pgfqpoint{0.000000in}{-0.048611in}}{\pgfqpoint{0.000000in}{0.000000in}}{%
\pgfpathmoveto{\pgfqpoint{0.000000in}{0.000000in}}%
\pgfpathlineto{\pgfqpoint{0.000000in}{-0.048611in}}%
\pgfusepath{stroke,fill}%
}%
\begin{pgfscope}%
\pgfsys@transformshift{1.754573in}{0.566590in}%
\pgfsys@useobject{currentmarker}{}%
\end{pgfscope}%
\end{pgfscope}%
\begin{pgfscope}%
\definecolor{textcolor}{rgb}{0.000000,0.000000,0.000000}%
\pgfsetstrokecolor{textcolor}%
\pgfsetfillcolor{textcolor}%
\pgftext[x=1.754573in,y=0.469368in,,top]{\color{textcolor}\rmfamily\fontsize{12.000000}{14.400000}\selectfont \(\displaystyle {10}\)}%
\end{pgfscope}%
\begin{pgfscope}%
\pgfpathrectangle{\pgfqpoint{0.740433in}{0.566590in}}{\pgfqpoint{3.295956in}{1.828724in}}%
\pgfusepath{clip}%
\pgfsetrectcap%
\pgfsetroundjoin%
\pgfsetlinewidth{0.803000pt}%
\definecolor{currentstroke}{rgb}{0.690196,0.690196,0.690196}%
\pgfsetstrokecolor{currentstroke}%
\pgfsetdash{}{0pt}%
\pgfpathmoveto{\pgfqpoint{2.261643in}{0.566590in}}%
\pgfpathlineto{\pgfqpoint{2.261643in}{2.395314in}}%
\pgfusepath{stroke}%
\end{pgfscope}%
\begin{pgfscope}%
\pgfsetbuttcap%
\pgfsetroundjoin%
\definecolor{currentfill}{rgb}{0.000000,0.000000,0.000000}%
\pgfsetfillcolor{currentfill}%
\pgfsetlinewidth{0.803000pt}%
\definecolor{currentstroke}{rgb}{0.000000,0.000000,0.000000}%
\pgfsetstrokecolor{currentstroke}%
\pgfsetdash{}{0pt}%
\pgfsys@defobject{currentmarker}{\pgfqpoint{0.000000in}{-0.048611in}}{\pgfqpoint{0.000000in}{0.000000in}}{%
\pgfpathmoveto{\pgfqpoint{0.000000in}{0.000000in}}%
\pgfpathlineto{\pgfqpoint{0.000000in}{-0.048611in}}%
\pgfusepath{stroke,fill}%
}%
\begin{pgfscope}%
\pgfsys@transformshift{2.261643in}{0.566590in}%
\pgfsys@useobject{currentmarker}{}%
\end{pgfscope}%
\end{pgfscope}%
\begin{pgfscope}%
\definecolor{textcolor}{rgb}{0.000000,0.000000,0.000000}%
\pgfsetstrokecolor{textcolor}%
\pgfsetfillcolor{textcolor}%
\pgftext[x=2.261643in,y=0.469368in,,top]{\color{textcolor}\rmfamily\fontsize{12.000000}{14.400000}\selectfont \(\displaystyle {20}\)}%
\end{pgfscope}%
\begin{pgfscope}%
\pgfpathrectangle{\pgfqpoint{0.740433in}{0.566590in}}{\pgfqpoint{3.295956in}{1.828724in}}%
\pgfusepath{clip}%
\pgfsetrectcap%
\pgfsetroundjoin%
\pgfsetlinewidth{0.803000pt}%
\definecolor{currentstroke}{rgb}{0.690196,0.690196,0.690196}%
\pgfsetstrokecolor{currentstroke}%
\pgfsetdash{}{0pt}%
\pgfpathmoveto{\pgfqpoint{2.768713in}{0.566590in}}%
\pgfpathlineto{\pgfqpoint{2.768713in}{2.395314in}}%
\pgfusepath{stroke}%
\end{pgfscope}%
\begin{pgfscope}%
\pgfsetbuttcap%
\pgfsetroundjoin%
\definecolor{currentfill}{rgb}{0.000000,0.000000,0.000000}%
\pgfsetfillcolor{currentfill}%
\pgfsetlinewidth{0.803000pt}%
\definecolor{currentstroke}{rgb}{0.000000,0.000000,0.000000}%
\pgfsetstrokecolor{currentstroke}%
\pgfsetdash{}{0pt}%
\pgfsys@defobject{currentmarker}{\pgfqpoint{0.000000in}{-0.048611in}}{\pgfqpoint{0.000000in}{0.000000in}}{%
\pgfpathmoveto{\pgfqpoint{0.000000in}{0.000000in}}%
\pgfpathlineto{\pgfqpoint{0.000000in}{-0.048611in}}%
\pgfusepath{stroke,fill}%
}%
\begin{pgfscope}%
\pgfsys@transformshift{2.768713in}{0.566590in}%
\pgfsys@useobject{currentmarker}{}%
\end{pgfscope}%
\end{pgfscope}%
\begin{pgfscope}%
\definecolor{textcolor}{rgb}{0.000000,0.000000,0.000000}%
\pgfsetstrokecolor{textcolor}%
\pgfsetfillcolor{textcolor}%
\pgftext[x=2.768713in,y=0.469368in,,top]{\color{textcolor}\rmfamily\fontsize{12.000000}{14.400000}\selectfont \(\displaystyle {30}\)}%
\end{pgfscope}%
\begin{pgfscope}%
\pgfpathrectangle{\pgfqpoint{0.740433in}{0.566590in}}{\pgfqpoint{3.295956in}{1.828724in}}%
\pgfusepath{clip}%
\pgfsetrectcap%
\pgfsetroundjoin%
\pgfsetlinewidth{0.803000pt}%
\definecolor{currentstroke}{rgb}{0.690196,0.690196,0.690196}%
\pgfsetstrokecolor{currentstroke}%
\pgfsetdash{}{0pt}%
\pgfpathmoveto{\pgfqpoint{3.275783in}{0.566590in}}%
\pgfpathlineto{\pgfqpoint{3.275783in}{2.395314in}}%
\pgfusepath{stroke}%
\end{pgfscope}%
\begin{pgfscope}%
\pgfsetbuttcap%
\pgfsetroundjoin%
\definecolor{currentfill}{rgb}{0.000000,0.000000,0.000000}%
\pgfsetfillcolor{currentfill}%
\pgfsetlinewidth{0.803000pt}%
\definecolor{currentstroke}{rgb}{0.000000,0.000000,0.000000}%
\pgfsetstrokecolor{currentstroke}%
\pgfsetdash{}{0pt}%
\pgfsys@defobject{currentmarker}{\pgfqpoint{0.000000in}{-0.048611in}}{\pgfqpoint{0.000000in}{0.000000in}}{%
\pgfpathmoveto{\pgfqpoint{0.000000in}{0.000000in}}%
\pgfpathlineto{\pgfqpoint{0.000000in}{-0.048611in}}%
\pgfusepath{stroke,fill}%
}%
\begin{pgfscope}%
\pgfsys@transformshift{3.275783in}{0.566590in}%
\pgfsys@useobject{currentmarker}{}%
\end{pgfscope}%
\end{pgfscope}%
\begin{pgfscope}%
\definecolor{textcolor}{rgb}{0.000000,0.000000,0.000000}%
\pgfsetstrokecolor{textcolor}%
\pgfsetfillcolor{textcolor}%
\pgftext[x=3.275783in,y=0.469368in,,top]{\color{textcolor}\rmfamily\fontsize{12.000000}{14.400000}\selectfont \(\displaystyle {40}\)}%
\end{pgfscope}%
\begin{pgfscope}%
\pgfpathrectangle{\pgfqpoint{0.740433in}{0.566590in}}{\pgfqpoint{3.295956in}{1.828724in}}%
\pgfusepath{clip}%
\pgfsetrectcap%
\pgfsetroundjoin%
\pgfsetlinewidth{0.803000pt}%
\definecolor{currentstroke}{rgb}{0.690196,0.690196,0.690196}%
\pgfsetstrokecolor{currentstroke}%
\pgfsetdash{}{0pt}%
\pgfpathmoveto{\pgfqpoint{3.782853in}{0.566590in}}%
\pgfpathlineto{\pgfqpoint{3.782853in}{2.395314in}}%
\pgfusepath{stroke}%
\end{pgfscope}%
\begin{pgfscope}%
\pgfsetbuttcap%
\pgfsetroundjoin%
\definecolor{currentfill}{rgb}{0.000000,0.000000,0.000000}%
\pgfsetfillcolor{currentfill}%
\pgfsetlinewidth{0.803000pt}%
\definecolor{currentstroke}{rgb}{0.000000,0.000000,0.000000}%
\pgfsetstrokecolor{currentstroke}%
\pgfsetdash{}{0pt}%
\pgfsys@defobject{currentmarker}{\pgfqpoint{0.000000in}{-0.048611in}}{\pgfqpoint{0.000000in}{0.000000in}}{%
\pgfpathmoveto{\pgfqpoint{0.000000in}{0.000000in}}%
\pgfpathlineto{\pgfqpoint{0.000000in}{-0.048611in}}%
\pgfusepath{stroke,fill}%
}%
\begin{pgfscope}%
\pgfsys@transformshift{3.782853in}{0.566590in}%
\pgfsys@useobject{currentmarker}{}%
\end{pgfscope}%
\end{pgfscope}%
\begin{pgfscope}%
\definecolor{textcolor}{rgb}{0.000000,0.000000,0.000000}%
\pgfsetstrokecolor{textcolor}%
\pgfsetfillcolor{textcolor}%
\pgftext[x=3.782853in,y=0.469368in,,top]{\color{textcolor}\rmfamily\fontsize{12.000000}{14.400000}\selectfont \(\displaystyle {50}\)}%
\end{pgfscope}%
\begin{pgfscope}%
\definecolor{textcolor}{rgb}{0.000000,0.000000,0.000000}%
\pgfsetstrokecolor{textcolor}%
\pgfsetfillcolor{textcolor}%
\pgftext[x=2.388411in,y=0.266626in,,top]{\color{textcolor}\rmfamily\fontsize{12.000000}{14.400000}\selectfont SNR [dB]}%
\end{pgfscope}%
\begin{pgfscope}%
\pgfpathrectangle{\pgfqpoint{0.740433in}{0.566590in}}{\pgfqpoint{3.295956in}{1.828724in}}%
\pgfusepath{clip}%
\pgfsetrectcap%
\pgfsetroundjoin%
\pgfsetlinewidth{0.803000pt}%
\definecolor{currentstroke}{rgb}{0.690196,0.690196,0.690196}%
\pgfsetstrokecolor{currentstroke}%
\pgfsetdash{}{0pt}%
\pgfpathmoveto{\pgfqpoint{0.740433in}{0.752967in}}%
\pgfpathlineto{\pgfqpoint{4.036389in}{0.752967in}}%
\pgfusepath{stroke}%
\end{pgfscope}%
\begin{pgfscope}%
\pgfsetbuttcap%
\pgfsetroundjoin%
\definecolor{currentfill}{rgb}{0.000000,0.000000,0.000000}%
\pgfsetfillcolor{currentfill}%
\pgfsetlinewidth{0.803000pt}%
\definecolor{currentstroke}{rgb}{0.000000,0.000000,0.000000}%
\pgfsetstrokecolor{currentstroke}%
\pgfsetdash{}{0pt}%
\pgfsys@defobject{currentmarker}{\pgfqpoint{-0.048611in}{0.000000in}}{\pgfqpoint{-0.000000in}{0.000000in}}{%
\pgfpathmoveto{\pgfqpoint{-0.000000in}{0.000000in}}%
\pgfpathlineto{\pgfqpoint{-0.048611in}{0.000000in}}%
\pgfusepath{stroke,fill}%
}%
\begin{pgfscope}%
\pgfsys@transformshift{0.740433in}{0.752967in}%
\pgfsys@useobject{currentmarker}{}%
\end{pgfscope}%
\end{pgfscope}%
\begin{pgfscope}%
\definecolor{textcolor}{rgb}{0.000000,0.000000,0.000000}%
\pgfsetstrokecolor{textcolor}%
\pgfsetfillcolor{textcolor}%
\pgftext[x=0.322222in, y=0.695574in, left, base]{\color{textcolor}\rmfamily\fontsize{12.000000}{14.400000}\selectfont \(\displaystyle {10^{-4}}\)}%
\end{pgfscope}%
\begin{pgfscope}%
\pgfpathrectangle{\pgfqpoint{0.740433in}{0.566590in}}{\pgfqpoint{3.295956in}{1.828724in}}%
\pgfusepath{clip}%
\pgfsetrectcap%
\pgfsetroundjoin%
\pgfsetlinewidth{0.803000pt}%
\definecolor{currentstroke}{rgb}{0.690196,0.690196,0.690196}%
\pgfsetstrokecolor{currentstroke}%
\pgfsetdash{}{0pt}%
\pgfpathmoveto{\pgfqpoint{0.740433in}{1.236303in}}%
\pgfpathlineto{\pgfqpoint{4.036389in}{1.236303in}}%
\pgfusepath{stroke}%
\end{pgfscope}%
\begin{pgfscope}%
\pgfsetbuttcap%
\pgfsetroundjoin%
\definecolor{currentfill}{rgb}{0.000000,0.000000,0.000000}%
\pgfsetfillcolor{currentfill}%
\pgfsetlinewidth{0.803000pt}%
\definecolor{currentstroke}{rgb}{0.000000,0.000000,0.000000}%
\pgfsetstrokecolor{currentstroke}%
\pgfsetdash{}{0pt}%
\pgfsys@defobject{currentmarker}{\pgfqpoint{-0.048611in}{0.000000in}}{\pgfqpoint{-0.000000in}{0.000000in}}{%
\pgfpathmoveto{\pgfqpoint{-0.000000in}{0.000000in}}%
\pgfpathlineto{\pgfqpoint{-0.048611in}{0.000000in}}%
\pgfusepath{stroke,fill}%
}%
\begin{pgfscope}%
\pgfsys@transformshift{0.740433in}{1.236303in}%
\pgfsys@useobject{currentmarker}{}%
\end{pgfscope}%
\end{pgfscope}%
\begin{pgfscope}%
\definecolor{textcolor}{rgb}{0.000000,0.000000,0.000000}%
\pgfsetstrokecolor{textcolor}%
\pgfsetfillcolor{textcolor}%
\pgftext[x=0.322222in, y=1.178910in, left, base]{\color{textcolor}\rmfamily\fontsize{12.000000}{14.400000}\selectfont \(\displaystyle {10^{-2}}\)}%
\end{pgfscope}%
\begin{pgfscope}%
\pgfpathrectangle{\pgfqpoint{0.740433in}{0.566590in}}{\pgfqpoint{3.295956in}{1.828724in}}%
\pgfusepath{clip}%
\pgfsetrectcap%
\pgfsetroundjoin%
\pgfsetlinewidth{0.803000pt}%
\definecolor{currentstroke}{rgb}{0.690196,0.690196,0.690196}%
\pgfsetstrokecolor{currentstroke}%
\pgfsetdash{}{0pt}%
\pgfpathmoveto{\pgfqpoint{0.740433in}{1.719639in}}%
\pgfpathlineto{\pgfqpoint{4.036389in}{1.719639in}}%
\pgfusepath{stroke}%
\end{pgfscope}%
\begin{pgfscope}%
\pgfsetbuttcap%
\pgfsetroundjoin%
\definecolor{currentfill}{rgb}{0.000000,0.000000,0.000000}%
\pgfsetfillcolor{currentfill}%
\pgfsetlinewidth{0.803000pt}%
\definecolor{currentstroke}{rgb}{0.000000,0.000000,0.000000}%
\pgfsetstrokecolor{currentstroke}%
\pgfsetdash{}{0pt}%
\pgfsys@defobject{currentmarker}{\pgfqpoint{-0.048611in}{0.000000in}}{\pgfqpoint{-0.000000in}{0.000000in}}{%
\pgfpathmoveto{\pgfqpoint{-0.000000in}{0.000000in}}%
\pgfpathlineto{\pgfqpoint{-0.048611in}{0.000000in}}%
\pgfusepath{stroke,fill}%
}%
\begin{pgfscope}%
\pgfsys@transformshift{0.740433in}{1.719639in}%
\pgfsys@useobject{currentmarker}{}%
\end{pgfscope}%
\end{pgfscope}%
\begin{pgfscope}%
\definecolor{textcolor}{rgb}{0.000000,0.000000,0.000000}%
\pgfsetstrokecolor{textcolor}%
\pgfsetfillcolor{textcolor}%
\pgftext[x=0.414045in, y=1.662246in, left, base]{\color{textcolor}\rmfamily\fontsize{12.000000}{14.400000}\selectfont \(\displaystyle {10^{0}}\)}%
\end{pgfscope}%
\begin{pgfscope}%
\pgfpathrectangle{\pgfqpoint{0.740433in}{0.566590in}}{\pgfqpoint{3.295956in}{1.828724in}}%
\pgfusepath{clip}%
\pgfsetrectcap%
\pgfsetroundjoin%
\pgfsetlinewidth{0.803000pt}%
\definecolor{currentstroke}{rgb}{0.690196,0.690196,0.690196}%
\pgfsetstrokecolor{currentstroke}%
\pgfsetdash{}{0pt}%
\pgfpathmoveto{\pgfqpoint{0.740433in}{2.202975in}}%
\pgfpathlineto{\pgfqpoint{4.036389in}{2.202975in}}%
\pgfusepath{stroke}%
\end{pgfscope}%
\begin{pgfscope}%
\pgfsetbuttcap%
\pgfsetroundjoin%
\definecolor{currentfill}{rgb}{0.000000,0.000000,0.000000}%
\pgfsetfillcolor{currentfill}%
\pgfsetlinewidth{0.803000pt}%
\definecolor{currentstroke}{rgb}{0.000000,0.000000,0.000000}%
\pgfsetstrokecolor{currentstroke}%
\pgfsetdash{}{0pt}%
\pgfsys@defobject{currentmarker}{\pgfqpoint{-0.048611in}{0.000000in}}{\pgfqpoint{-0.000000in}{0.000000in}}{%
\pgfpathmoveto{\pgfqpoint{-0.000000in}{0.000000in}}%
\pgfpathlineto{\pgfqpoint{-0.048611in}{0.000000in}}%
\pgfusepath{stroke,fill}%
}%
\begin{pgfscope}%
\pgfsys@transformshift{0.740433in}{2.202975in}%
\pgfsys@useobject{currentmarker}{}%
\end{pgfscope}%
\end{pgfscope}%
\begin{pgfscope}%
\definecolor{textcolor}{rgb}{0.000000,0.000000,0.000000}%
\pgfsetstrokecolor{textcolor}%
\pgfsetfillcolor{textcolor}%
\pgftext[x=0.414045in, y=2.145582in, left, base]{\color{textcolor}\rmfamily\fontsize{12.000000}{14.400000}\selectfont \(\displaystyle {10^{2}}\)}%
\end{pgfscope}%
\begin{pgfscope}%
\definecolor{textcolor}{rgb}{0.000000,0.000000,0.000000}%
\pgfsetstrokecolor{textcolor}%
\pgfsetfillcolor{textcolor}%
\pgftext[x=0.266667in,y=1.480952in,,bottom,rotate=90.000000]{\color{textcolor}\rmfamily\fontsize{12.000000}{14.400000}\selectfont \(\displaystyle \hat{\sigma}_{\nu}(\mathrm{SNR})\)}%
\end{pgfscope}%
\begin{pgfscope}%
\pgfpathrectangle{\pgfqpoint{0.740433in}{0.566590in}}{\pgfqpoint{3.295956in}{1.828724in}}%
\pgfusepath{clip}%
\pgfsetbuttcap%
\pgfsetroundjoin%
\pgfsetlinewidth{1.505625pt}%
\definecolor{currentstroke}{rgb}{0.000000,0.447000,0.741000}%
\pgfsetstrokecolor{currentstroke}%
\pgfsetdash{{5.550000pt}{2.400000pt}}{0.000000pt}%
\pgfpathmoveto{\pgfqpoint{0.740433in}{2.284739in}}%
\pgfpathlineto{\pgfqpoint{0.837373in}{2.279929in}}%
\pgfpathlineto{\pgfqpoint{0.934312in}{2.279500in}}%
\pgfpathlineto{\pgfqpoint{1.031252in}{2.280645in}}%
\pgfpathlineto{\pgfqpoint{1.128192in}{2.299818in}}%
\pgfpathlineto{\pgfqpoint{1.225132in}{2.281210in}}%
\pgfpathlineto{\pgfqpoint{1.322072in}{2.267176in}}%
\pgfpathlineto{\pgfqpoint{1.419012in}{2.260309in}}%
\pgfpathlineto{\pgfqpoint{1.515952in}{2.262067in}}%
\pgfpathlineto{\pgfqpoint{1.612892in}{2.287335in}}%
\pgfpathlineto{\pgfqpoint{1.709831in}{2.273562in}}%
\pgfpathlineto{\pgfqpoint{1.806771in}{2.289929in}}%
\pgfpathlineto{\pgfqpoint{1.903711in}{2.286848in}}%
\pgfpathlineto{\pgfqpoint{2.000651in}{2.247583in}}%
\pgfpathlineto{\pgfqpoint{2.097591in}{2.201925in}}%
\pgfpathlineto{\pgfqpoint{2.194531in}{1.893274in}}%
\pgfpathlineto{\pgfqpoint{2.291471in}{1.681588in}}%
\pgfpathlineto{\pgfqpoint{2.388411in}{1.646087in}}%
\pgfpathlineto{\pgfqpoint{2.485350in}{1.597651in}}%
\pgfpathlineto{\pgfqpoint{2.582290in}{1.513144in}}%
\pgfpathlineto{\pgfqpoint{2.679230in}{1.489137in}}%
\pgfpathlineto{\pgfqpoint{2.776170in}{1.471546in}}%
\pgfpathlineto{\pgfqpoint{2.873110in}{1.435130in}}%
\pgfpathlineto{\pgfqpoint{2.970050in}{1.422188in}}%
\pgfpathlineto{\pgfqpoint{3.066990in}{1.403067in}}%
\pgfpathlineto{\pgfqpoint{3.163930in}{1.373501in}}%
\pgfpathlineto{\pgfqpoint{3.260870in}{1.353700in}}%
\pgfpathlineto{\pgfqpoint{3.357809in}{1.331705in}}%
\pgfpathlineto{\pgfqpoint{3.454749in}{1.306038in}}%
\pgfpathlineto{\pgfqpoint{3.551689in}{1.282392in}}%
\pgfpathlineto{\pgfqpoint{3.648629in}{1.259880in}}%
\pgfpathlineto{\pgfqpoint{3.745569in}{1.243226in}}%
\pgfpathlineto{\pgfqpoint{3.842509in}{1.204928in}}%
\pgfpathlineto{\pgfqpoint{3.939449in}{1.192994in}}%
\pgfpathlineto{\pgfqpoint{4.036389in}{1.162445in}}%
\pgfusepath{stroke}%
\end{pgfscope}%
\begin{pgfscope}%
\pgfpathrectangle{\pgfqpoint{0.740433in}{0.566590in}}{\pgfqpoint{3.295956in}{1.828724in}}%
\pgfusepath{clip}%
\pgfsetbuttcap%
\pgfsetroundjoin%
\definecolor{currentfill}{rgb}{0.000000,0.000000,0.000000}%
\pgfsetfillcolor{currentfill}%
\pgfsetfillopacity{0.000000}%
\pgfsetlinewidth{1.003750pt}%
\definecolor{currentstroke}{rgb}{0.000000,0.447000,0.741000}%
\pgfsetstrokecolor{currentstroke}%
\pgfsetdash{}{0pt}%
\pgfsys@defobject{currentmarker}{\pgfqpoint{-0.041667in}{-0.041667in}}{\pgfqpoint{0.041667in}{0.041667in}}{%
\pgfpathmoveto{\pgfqpoint{0.000000in}{-0.041667in}}%
\pgfpathcurveto{\pgfqpoint{0.011050in}{-0.041667in}}{\pgfqpoint{0.021649in}{-0.037276in}}{\pgfqpoint{0.029463in}{-0.029463in}}%
\pgfpathcurveto{\pgfqpoint{0.037276in}{-0.021649in}}{\pgfqpoint{0.041667in}{-0.011050in}}{\pgfqpoint{0.041667in}{0.000000in}}%
\pgfpathcurveto{\pgfqpoint{0.041667in}{0.011050in}}{\pgfqpoint{0.037276in}{0.021649in}}{\pgfqpoint{0.029463in}{0.029463in}}%
\pgfpathcurveto{\pgfqpoint{0.021649in}{0.037276in}}{\pgfqpoint{0.011050in}{0.041667in}}{\pgfqpoint{0.000000in}{0.041667in}}%
\pgfpathcurveto{\pgfqpoint{-0.011050in}{0.041667in}}{\pgfqpoint{-0.021649in}{0.037276in}}{\pgfqpoint{-0.029463in}{0.029463in}}%
\pgfpathcurveto{\pgfqpoint{-0.037276in}{0.021649in}}{\pgfqpoint{-0.041667in}{0.011050in}}{\pgfqpoint{-0.041667in}{0.000000in}}%
\pgfpathcurveto{\pgfqpoint{-0.041667in}{-0.011050in}}{\pgfqpoint{-0.037276in}{-0.021649in}}{\pgfqpoint{-0.029463in}{-0.029463in}}%
\pgfpathcurveto{\pgfqpoint{-0.021649in}{-0.037276in}}{\pgfqpoint{-0.011050in}{-0.041667in}}{\pgfqpoint{0.000000in}{-0.041667in}}%
\pgfpathclose%
\pgfusepath{stroke,fill}%
}%
\begin{pgfscope}%
\pgfsys@transformshift{0.740433in}{2.284739in}%
\pgfsys@useobject{currentmarker}{}%
\end{pgfscope}%
\begin{pgfscope}%
\pgfsys@transformshift{1.128192in}{2.299818in}%
\pgfsys@useobject{currentmarker}{}%
\end{pgfscope}%
\begin{pgfscope}%
\pgfsys@transformshift{1.515952in}{2.262067in}%
\pgfsys@useobject{currentmarker}{}%
\end{pgfscope}%
\begin{pgfscope}%
\pgfsys@transformshift{1.903711in}{2.286848in}%
\pgfsys@useobject{currentmarker}{}%
\end{pgfscope}%
\begin{pgfscope}%
\pgfsys@transformshift{2.291471in}{1.681588in}%
\pgfsys@useobject{currentmarker}{}%
\end{pgfscope}%
\begin{pgfscope}%
\pgfsys@transformshift{2.679230in}{1.489137in}%
\pgfsys@useobject{currentmarker}{}%
\end{pgfscope}%
\begin{pgfscope}%
\pgfsys@transformshift{3.066990in}{1.403067in}%
\pgfsys@useobject{currentmarker}{}%
\end{pgfscope}%
\begin{pgfscope}%
\pgfsys@transformshift{3.454749in}{1.306038in}%
\pgfsys@useobject{currentmarker}{}%
\end{pgfscope}%
\begin{pgfscope}%
\pgfsys@transformshift{3.842509in}{1.204928in}%
\pgfsys@useobject{currentmarker}{}%
\end{pgfscope}%
\end{pgfscope}%
\begin{pgfscope}%
\pgfpathrectangle{\pgfqpoint{0.740433in}{0.566590in}}{\pgfqpoint{3.295956in}{1.828724in}}%
\pgfusepath{clip}%
\pgfsetbuttcap%
\pgfsetroundjoin%
\pgfsetlinewidth{1.505625pt}%
\definecolor{currentstroke}{rgb}{0.850000,0.324000,0.098000}%
\pgfsetstrokecolor{currentstroke}%
\pgfsetdash{{5.550000pt}{2.400000pt}}{0.000000pt}%
\pgfpathmoveto{\pgfqpoint{0.740433in}{2.225442in}}%
\pgfpathlineto{\pgfqpoint{0.837373in}{2.228971in}}%
\pgfpathlineto{\pgfqpoint{0.934312in}{2.208509in}}%
\pgfpathlineto{\pgfqpoint{1.031252in}{2.192730in}}%
\pgfpathlineto{\pgfqpoint{1.128192in}{2.222308in}}%
\pgfpathlineto{\pgfqpoint{1.225132in}{2.181156in}}%
\pgfpathlineto{\pgfqpoint{1.322072in}{2.125703in}}%
\pgfpathlineto{\pgfqpoint{1.419012in}{1.859626in}}%
\pgfpathlineto{\pgfqpoint{1.515952in}{1.789079in}}%
\pgfpathlineto{\pgfqpoint{1.612892in}{1.793451in}}%
\pgfpathlineto{\pgfqpoint{1.709831in}{1.844187in}}%
\pgfpathlineto{\pgfqpoint{1.806771in}{1.759923in}}%
\pgfpathlineto{\pgfqpoint{1.903711in}{1.769781in}}%
\pgfpathlineto{\pgfqpoint{2.000651in}{1.759659in}}%
\pgfpathlineto{\pgfqpoint{2.097591in}{1.731837in}}%
\pgfpathlineto{\pgfqpoint{2.194531in}{1.672458in}}%
\pgfpathlineto{\pgfqpoint{2.291471in}{1.645357in}}%
\pgfpathlineto{\pgfqpoint{2.388411in}{1.617057in}}%
\pgfpathlineto{\pgfqpoint{2.485350in}{1.577140in}}%
\pgfpathlineto{\pgfqpoint{2.582290in}{1.536705in}}%
\pgfpathlineto{\pgfqpoint{2.679230in}{1.511008in}}%
\pgfpathlineto{\pgfqpoint{2.776170in}{1.480814in}}%
\pgfpathlineto{\pgfqpoint{2.873110in}{1.439894in}}%
\pgfpathlineto{\pgfqpoint{2.970050in}{1.440480in}}%
\pgfpathlineto{\pgfqpoint{3.066990in}{1.418497in}}%
\pgfpathlineto{\pgfqpoint{3.163930in}{1.388673in}}%
\pgfpathlineto{\pgfqpoint{3.260870in}{1.364911in}}%
\pgfpathlineto{\pgfqpoint{3.357809in}{1.341632in}}%
\pgfpathlineto{\pgfqpoint{3.454749in}{1.317176in}}%
\pgfpathlineto{\pgfqpoint{3.551689in}{1.292232in}}%
\pgfpathlineto{\pgfqpoint{3.648629in}{1.275948in}}%
\pgfpathlineto{\pgfqpoint{3.745569in}{1.256754in}}%
\pgfpathlineto{\pgfqpoint{3.842509in}{1.228719in}}%
\pgfpathlineto{\pgfqpoint{3.939449in}{1.203670in}}%
\pgfpathlineto{\pgfqpoint{4.036389in}{1.176496in}}%
\pgfusepath{stroke}%
\end{pgfscope}%
\begin{pgfscope}%
\pgfpathrectangle{\pgfqpoint{0.740433in}{0.566590in}}{\pgfqpoint{3.295956in}{1.828724in}}%
\pgfusepath{clip}%
\pgfsetbuttcap%
\pgfsetroundjoin%
\definecolor{currentfill}{rgb}{0.850000,0.324000,0.098000}%
\pgfsetfillcolor{currentfill}%
\pgfsetlinewidth{1.003750pt}%
\definecolor{currentstroke}{rgb}{0.850000,0.324000,0.098000}%
\pgfsetstrokecolor{currentstroke}%
\pgfsetdash{}{0pt}%
\pgfsys@defobject{currentmarker}{\pgfqpoint{-0.041667in}{-0.041667in}}{\pgfqpoint{0.041667in}{0.041667in}}{%
\pgfpathmoveto{\pgfqpoint{-0.041667in}{0.000000in}}%
\pgfpathlineto{\pgfqpoint{0.041667in}{0.000000in}}%
\pgfpathmoveto{\pgfqpoint{0.000000in}{-0.041667in}}%
\pgfpathlineto{\pgfqpoint{0.000000in}{0.041667in}}%
\pgfusepath{stroke,fill}%
}%
\begin{pgfscope}%
\pgfsys@transformshift{0.740433in}{2.225442in}%
\pgfsys@useobject{currentmarker}{}%
\end{pgfscope}%
\begin{pgfscope}%
\pgfsys@transformshift{1.031252in}{2.192730in}%
\pgfsys@useobject{currentmarker}{}%
\end{pgfscope}%
\begin{pgfscope}%
\pgfsys@transformshift{1.322072in}{2.125703in}%
\pgfsys@useobject{currentmarker}{}%
\end{pgfscope}%
\begin{pgfscope}%
\pgfsys@transformshift{1.612892in}{1.793451in}%
\pgfsys@useobject{currentmarker}{}%
\end{pgfscope}%
\begin{pgfscope}%
\pgfsys@transformshift{1.903711in}{1.769781in}%
\pgfsys@useobject{currentmarker}{}%
\end{pgfscope}%
\begin{pgfscope}%
\pgfsys@transformshift{2.194531in}{1.672458in}%
\pgfsys@useobject{currentmarker}{}%
\end{pgfscope}%
\begin{pgfscope}%
\pgfsys@transformshift{2.485350in}{1.577140in}%
\pgfsys@useobject{currentmarker}{}%
\end{pgfscope}%
\begin{pgfscope}%
\pgfsys@transformshift{2.776170in}{1.480814in}%
\pgfsys@useobject{currentmarker}{}%
\end{pgfscope}%
\begin{pgfscope}%
\pgfsys@transformshift{3.066990in}{1.418497in}%
\pgfsys@useobject{currentmarker}{}%
\end{pgfscope}%
\begin{pgfscope}%
\pgfsys@transformshift{3.357809in}{1.341632in}%
\pgfsys@useobject{currentmarker}{}%
\end{pgfscope}%
\begin{pgfscope}%
\pgfsys@transformshift{3.648629in}{1.275948in}%
\pgfsys@useobject{currentmarker}{}%
\end{pgfscope}%
\begin{pgfscope}%
\pgfsys@transformshift{3.939449in}{1.203670in}%
\pgfsys@useobject{currentmarker}{}%
\end{pgfscope}%
\end{pgfscope}%
\begin{pgfscope}%
\pgfpathrectangle{\pgfqpoint{0.740433in}{0.566590in}}{\pgfqpoint{3.295956in}{1.828724in}}%
\pgfusepath{clip}%
\pgfsetbuttcap%
\pgfsetroundjoin%
\pgfsetlinewidth{1.505625pt}%
\definecolor{currentstroke}{rgb}{0.000000,0.500000,0.000000}%
\pgfsetstrokecolor{currentstroke}%
\pgfsetdash{{5.550000pt}{2.400000pt}}{0.000000pt}%
\pgfpathmoveto{\pgfqpoint{0.740433in}{2.152105in}}%
\pgfpathlineto{\pgfqpoint{0.837373in}{2.064399in}}%
\pgfpathlineto{\pgfqpoint{0.934312in}{2.001246in}}%
\pgfpathlineto{\pgfqpoint{1.031252in}{1.939187in}}%
\pgfpathlineto{\pgfqpoint{1.128192in}{1.993665in}}%
\pgfpathlineto{\pgfqpoint{1.225132in}{1.868879in}}%
\pgfpathlineto{\pgfqpoint{1.322072in}{1.828384in}}%
\pgfpathlineto{\pgfqpoint{1.419012in}{1.937634in}}%
\pgfpathlineto{\pgfqpoint{1.515952in}{1.834362in}}%
\pgfpathlineto{\pgfqpoint{1.612892in}{1.921918in}}%
\pgfpathlineto{\pgfqpoint{1.709831in}{1.890682in}}%
\pgfpathlineto{\pgfqpoint{1.806771in}{1.804047in}}%
\pgfpathlineto{\pgfqpoint{1.903711in}{1.808794in}}%
\pgfpathlineto{\pgfqpoint{2.000651in}{1.798467in}}%
\pgfpathlineto{\pgfqpoint{2.097591in}{1.780706in}}%
\pgfpathlineto{\pgfqpoint{2.194531in}{1.712712in}}%
\pgfpathlineto{\pgfqpoint{2.291471in}{1.707999in}}%
\pgfpathlineto{\pgfqpoint{2.388411in}{1.665756in}}%
\pgfpathlineto{\pgfqpoint{2.485350in}{1.618956in}}%
\pgfpathlineto{\pgfqpoint{2.582290in}{1.506563in}}%
\pgfpathlineto{\pgfqpoint{2.679230in}{1.466137in}}%
\pgfpathlineto{\pgfqpoint{2.776170in}{1.452253in}}%
\pgfpathlineto{\pgfqpoint{2.873110in}{1.416955in}}%
\pgfpathlineto{\pgfqpoint{2.970050in}{1.402626in}}%
\pgfpathlineto{\pgfqpoint{3.066990in}{1.379782in}}%
\pgfpathlineto{\pgfqpoint{3.163930in}{1.358169in}}%
\pgfpathlineto{\pgfqpoint{3.260870in}{1.336572in}}%
\pgfpathlineto{\pgfqpoint{3.357809in}{1.310627in}}%
\pgfpathlineto{\pgfqpoint{3.454749in}{1.284413in}}%
\pgfpathlineto{\pgfqpoint{3.551689in}{1.261141in}}%
\pgfpathlineto{\pgfqpoint{3.648629in}{1.236022in}}%
\pgfpathlineto{\pgfqpoint{3.745569in}{1.227589in}}%
\pgfpathlineto{\pgfqpoint{3.842509in}{1.191681in}}%
\pgfpathlineto{\pgfqpoint{3.939449in}{1.177190in}}%
\pgfpathlineto{\pgfqpoint{4.036389in}{1.139870in}}%
\pgfusepath{stroke}%
\end{pgfscope}%
\begin{pgfscope}%
\pgfpathrectangle{\pgfqpoint{0.740433in}{0.566590in}}{\pgfqpoint{3.295956in}{1.828724in}}%
\pgfusepath{clip}%
\pgfsetbuttcap%
\pgfsetmiterjoin%
\definecolor{currentfill}{rgb}{0.000000,0.000000,0.000000}%
\pgfsetfillcolor{currentfill}%
\pgfsetfillopacity{0.000000}%
\pgfsetlinewidth{1.003750pt}%
\definecolor{currentstroke}{rgb}{0.000000,0.500000,0.000000}%
\pgfsetstrokecolor{currentstroke}%
\pgfsetdash{}{0pt}%
\pgfsys@defobject{currentmarker}{\pgfqpoint{-0.041667in}{-0.041667in}}{\pgfqpoint{0.041667in}{0.041667in}}{%
\pgfpathmoveto{\pgfqpoint{-0.041667in}{-0.041667in}}%
\pgfpathlineto{\pgfqpoint{0.041667in}{-0.041667in}}%
\pgfpathlineto{\pgfqpoint{0.041667in}{0.041667in}}%
\pgfpathlineto{\pgfqpoint{-0.041667in}{0.041667in}}%
\pgfpathclose%
\pgfusepath{stroke,fill}%
}%
\begin{pgfscope}%
\pgfsys@transformshift{0.740433in}{2.152105in}%
\pgfsys@useobject{currentmarker}{}%
\end{pgfscope}%
\begin{pgfscope}%
\pgfsys@transformshift{1.225132in}{1.868879in}%
\pgfsys@useobject{currentmarker}{}%
\end{pgfscope}%
\begin{pgfscope}%
\pgfsys@transformshift{1.709831in}{1.890682in}%
\pgfsys@useobject{currentmarker}{}%
\end{pgfscope}%
\begin{pgfscope}%
\pgfsys@transformshift{2.194531in}{1.712712in}%
\pgfsys@useobject{currentmarker}{}%
\end{pgfscope}%
\begin{pgfscope}%
\pgfsys@transformshift{2.679230in}{1.466137in}%
\pgfsys@useobject{currentmarker}{}%
\end{pgfscope}%
\begin{pgfscope}%
\pgfsys@transformshift{3.163930in}{1.358169in}%
\pgfsys@useobject{currentmarker}{}%
\end{pgfscope}%
\begin{pgfscope}%
\pgfsys@transformshift{3.648629in}{1.236022in}%
\pgfsys@useobject{currentmarker}{}%
\end{pgfscope}%
\end{pgfscope}%
\begin{pgfscope}%
\pgfpathrectangle{\pgfqpoint{0.740433in}{0.566590in}}{\pgfqpoint{3.295956in}{1.828724in}}%
\pgfusepath{clip}%
\pgfsetbuttcap%
\pgfsetroundjoin%
\pgfsetlinewidth{1.505625pt}%
\definecolor{currentstroke}{rgb}{0.494000,0.184000,0.556000}%
\pgfsetstrokecolor{currentstroke}%
\pgfsetdash{{5.550000pt}{2.400000pt}}{0.000000pt}%
\pgfpathmoveto{\pgfqpoint{0.740433in}{2.123969in}}%
\pgfpathlineto{\pgfqpoint{0.837373in}{2.101991in}}%
\pgfpathlineto{\pgfqpoint{0.934312in}{2.089708in}}%
\pgfpathlineto{\pgfqpoint{1.031252in}{2.064282in}}%
\pgfpathlineto{\pgfqpoint{1.128192in}{2.065140in}}%
\pgfpathlineto{\pgfqpoint{1.225132in}{2.044780in}}%
\pgfpathlineto{\pgfqpoint{1.322072in}{2.037254in}}%
\pgfpathlineto{\pgfqpoint{1.419012in}{2.058088in}}%
\pgfpathlineto{\pgfqpoint{1.515952in}{2.000345in}}%
\pgfpathlineto{\pgfqpoint{1.612892in}{2.033628in}}%
\pgfpathlineto{\pgfqpoint{1.709831in}{2.022167in}}%
\pgfpathlineto{\pgfqpoint{1.806771in}{2.000929in}}%
\pgfpathlineto{\pgfqpoint{1.903711in}{1.929168in}}%
\pgfpathlineto{\pgfqpoint{2.000651in}{1.918608in}}%
\pgfpathlineto{\pgfqpoint{2.097591in}{1.913928in}}%
\pgfpathlineto{\pgfqpoint{2.194531in}{1.681853in}}%
\pgfpathlineto{\pgfqpoint{2.291471in}{1.685454in}}%
\pgfpathlineto{\pgfqpoint{2.388411in}{1.644522in}}%
\pgfpathlineto{\pgfqpoint{2.485350in}{1.596712in}}%
\pgfpathlineto{\pgfqpoint{2.582290in}{1.363133in}}%
\pgfpathlineto{\pgfqpoint{2.679230in}{1.323715in}}%
\pgfpathlineto{\pgfqpoint{2.776170in}{1.305124in}}%
\pgfpathlineto{\pgfqpoint{2.873110in}{1.271353in}}%
\pgfpathlineto{\pgfqpoint{2.970050in}{1.256705in}}%
\pgfpathlineto{\pgfqpoint{3.066990in}{1.235083in}}%
\pgfpathlineto{\pgfqpoint{3.163930in}{1.216795in}}%
\pgfpathlineto{\pgfqpoint{3.260870in}{1.189737in}}%
\pgfpathlineto{\pgfqpoint{3.357809in}{1.160716in}}%
\pgfpathlineto{\pgfqpoint{3.454749in}{1.140473in}}%
\pgfpathlineto{\pgfqpoint{3.551689in}{1.115593in}}%
\pgfpathlineto{\pgfqpoint{3.648629in}{1.091046in}}%
\pgfpathlineto{\pgfqpoint{3.745569in}{1.082831in}}%
\pgfpathlineto{\pgfqpoint{3.842509in}{1.049038in}}%
\pgfpathlineto{\pgfqpoint{3.939449in}{1.030424in}}%
\pgfpathlineto{\pgfqpoint{4.036389in}{0.993759in}}%
\pgfusepath{stroke}%
\end{pgfscope}%
\begin{pgfscope}%
\pgfpathrectangle{\pgfqpoint{0.740433in}{0.566590in}}{\pgfqpoint{3.295956in}{1.828724in}}%
\pgfusepath{clip}%
\pgfsetbuttcap%
\pgfsetroundjoin%
\definecolor{currentfill}{rgb}{0.494000,0.184000,0.556000}%
\pgfsetfillcolor{currentfill}%
\pgfsetlinewidth{1.003750pt}%
\definecolor{currentstroke}{rgb}{0.494000,0.184000,0.556000}%
\pgfsetstrokecolor{currentstroke}%
\pgfsetdash{}{0pt}%
\pgfsys@defobject{currentmarker}{\pgfqpoint{-0.041667in}{-0.041667in}}{\pgfqpoint{0.041667in}{0.041667in}}{%
\pgfpathmoveto{\pgfqpoint{-0.041667in}{-0.041667in}}%
\pgfpathlineto{\pgfqpoint{0.041667in}{0.041667in}}%
\pgfpathmoveto{\pgfqpoint{-0.041667in}{0.041667in}}%
\pgfpathlineto{\pgfqpoint{0.041667in}{-0.041667in}}%
\pgfusepath{stroke,fill}%
}%
\begin{pgfscope}%
\pgfsys@transformshift{0.740433in}{2.123969in}%
\pgfsys@useobject{currentmarker}{}%
\end{pgfscope}%
\begin{pgfscope}%
\pgfsys@transformshift{1.128192in}{2.065140in}%
\pgfsys@useobject{currentmarker}{}%
\end{pgfscope}%
\begin{pgfscope}%
\pgfsys@transformshift{1.515952in}{2.000345in}%
\pgfsys@useobject{currentmarker}{}%
\end{pgfscope}%
\begin{pgfscope}%
\pgfsys@transformshift{1.903711in}{1.929168in}%
\pgfsys@useobject{currentmarker}{}%
\end{pgfscope}%
\begin{pgfscope}%
\pgfsys@transformshift{2.291471in}{1.685454in}%
\pgfsys@useobject{currentmarker}{}%
\end{pgfscope}%
\begin{pgfscope}%
\pgfsys@transformshift{2.679230in}{1.323715in}%
\pgfsys@useobject{currentmarker}{}%
\end{pgfscope}%
\begin{pgfscope}%
\pgfsys@transformshift{3.066990in}{1.235083in}%
\pgfsys@useobject{currentmarker}{}%
\end{pgfscope}%
\begin{pgfscope}%
\pgfsys@transformshift{3.454749in}{1.140473in}%
\pgfsys@useobject{currentmarker}{}%
\end{pgfscope}%
\begin{pgfscope}%
\pgfsys@transformshift{3.842509in}{1.049038in}%
\pgfsys@useobject{currentmarker}{}%
\end{pgfscope}%
\end{pgfscope}%
\begin{pgfscope}%
\pgfpathrectangle{\pgfqpoint{0.740433in}{0.566590in}}{\pgfqpoint{3.295956in}{1.828724in}}%
\pgfusepath{clip}%
\pgfsetbuttcap%
\pgfsetroundjoin%
\pgfsetlinewidth{1.505625pt}%
\definecolor{currentstroke}{rgb}{0.635000,0.078000,0.184000}%
\pgfsetstrokecolor{currentstroke}%
\pgfsetdash{{5.550000pt}{2.400000pt}}{0.000000pt}%
\pgfpathmoveto{\pgfqpoint{0.740433in}{2.185691in}}%
\pgfpathlineto{\pgfqpoint{0.837373in}{2.177276in}}%
\pgfpathlineto{\pgfqpoint{0.934312in}{2.165627in}}%
\pgfpathlineto{\pgfqpoint{1.031252in}{2.165672in}}%
\pgfpathlineto{\pgfqpoint{1.128192in}{2.157342in}}%
\pgfpathlineto{\pgfqpoint{1.225132in}{2.146998in}}%
\pgfpathlineto{\pgfqpoint{1.322072in}{2.156786in}}%
\pgfpathlineto{\pgfqpoint{1.419012in}{2.151034in}}%
\pgfpathlineto{\pgfqpoint{1.515952in}{2.154106in}}%
\pgfpathlineto{\pgfqpoint{1.612892in}{2.136131in}}%
\pgfpathlineto{\pgfqpoint{1.709831in}{2.148403in}}%
\pgfpathlineto{\pgfqpoint{1.806771in}{2.134614in}}%
\pgfpathlineto{\pgfqpoint{1.903711in}{2.108111in}}%
\pgfpathlineto{\pgfqpoint{2.000651in}{2.101729in}}%
\pgfpathlineto{\pgfqpoint{2.097591in}{2.086316in}}%
\pgfpathlineto{\pgfqpoint{2.194531in}{2.049054in}}%
\pgfpathlineto{\pgfqpoint{2.291471in}{2.049642in}}%
\pgfpathlineto{\pgfqpoint{2.388411in}{2.004327in}}%
\pgfpathlineto{\pgfqpoint{2.485350in}{1.956532in}}%
\pgfpathlineto{\pgfqpoint{2.582290in}{1.256153in}}%
\pgfpathlineto{\pgfqpoint{2.679230in}{1.220194in}}%
\pgfpathlineto{\pgfqpoint{2.776170in}{1.209430in}}%
\pgfpathlineto{\pgfqpoint{2.873110in}{1.177829in}}%
\pgfpathlineto{\pgfqpoint{2.970050in}{1.156579in}}%
\pgfpathlineto{\pgfqpoint{3.066990in}{1.130671in}}%
\pgfpathlineto{\pgfqpoint{3.163930in}{1.121715in}}%
\pgfpathlineto{\pgfqpoint{3.260870in}{1.091628in}}%
\pgfpathlineto{\pgfqpoint{3.357809in}{1.058015in}}%
\pgfpathlineto{\pgfqpoint{3.454749in}{1.039728in}}%
\pgfpathlineto{\pgfqpoint{3.551689in}{1.012552in}}%
\pgfpathlineto{\pgfqpoint{3.648629in}{0.985788in}}%
\pgfpathlineto{\pgfqpoint{3.745569in}{0.983317in}}%
\pgfpathlineto{\pgfqpoint{3.842509in}{0.950515in}}%
\pgfpathlineto{\pgfqpoint{3.939449in}{0.929808in}}%
\pgfpathlineto{\pgfqpoint{4.036389in}{0.891178in}}%
\pgfusepath{stroke}%
\end{pgfscope}%
\begin{pgfscope}%
\pgfpathrectangle{\pgfqpoint{0.740433in}{0.566590in}}{\pgfqpoint{3.295956in}{1.828724in}}%
\pgfusepath{clip}%
\pgfsetbuttcap%
\pgfsetmiterjoin%
\definecolor{currentfill}{rgb}{0.000000,0.000000,0.000000}%
\pgfsetfillcolor{currentfill}%
\pgfsetfillopacity{0.000000}%
\pgfsetlinewidth{1.003750pt}%
\definecolor{currentstroke}{rgb}{0.635000,0.078000,0.184000}%
\pgfsetstrokecolor{currentstroke}%
\pgfsetdash{}{0pt}%
\pgfsys@defobject{currentmarker}{\pgfqpoint{-0.035355in}{-0.058926in}}{\pgfqpoint{0.035355in}{0.058926in}}{%
\pgfpathmoveto{\pgfqpoint{-0.000000in}{-0.058926in}}%
\pgfpathlineto{\pgfqpoint{0.035355in}{0.000000in}}%
\pgfpathlineto{\pgfqpoint{0.000000in}{0.058926in}}%
\pgfpathlineto{\pgfqpoint{-0.035355in}{0.000000in}}%
\pgfpathclose%
\pgfusepath{stroke,fill}%
}%
\begin{pgfscope}%
\pgfsys@transformshift{0.740433in}{2.185691in}%
\pgfsys@useobject{currentmarker}{}%
\end{pgfscope}%
\begin{pgfscope}%
\pgfsys@transformshift{1.031252in}{2.165672in}%
\pgfsys@useobject{currentmarker}{}%
\end{pgfscope}%
\begin{pgfscope}%
\pgfsys@transformshift{1.322072in}{2.156786in}%
\pgfsys@useobject{currentmarker}{}%
\end{pgfscope}%
\begin{pgfscope}%
\pgfsys@transformshift{1.612892in}{2.136131in}%
\pgfsys@useobject{currentmarker}{}%
\end{pgfscope}%
\begin{pgfscope}%
\pgfsys@transformshift{1.903711in}{2.108111in}%
\pgfsys@useobject{currentmarker}{}%
\end{pgfscope}%
\begin{pgfscope}%
\pgfsys@transformshift{2.194531in}{2.049054in}%
\pgfsys@useobject{currentmarker}{}%
\end{pgfscope}%
\begin{pgfscope}%
\pgfsys@transformshift{2.485350in}{1.956532in}%
\pgfsys@useobject{currentmarker}{}%
\end{pgfscope}%
\begin{pgfscope}%
\pgfsys@transformshift{2.776170in}{1.209430in}%
\pgfsys@useobject{currentmarker}{}%
\end{pgfscope}%
\begin{pgfscope}%
\pgfsys@transformshift{3.066990in}{1.130671in}%
\pgfsys@useobject{currentmarker}{}%
\end{pgfscope}%
\begin{pgfscope}%
\pgfsys@transformshift{3.357809in}{1.058015in}%
\pgfsys@useobject{currentmarker}{}%
\end{pgfscope}%
\begin{pgfscope}%
\pgfsys@transformshift{3.648629in}{0.985788in}%
\pgfsys@useobject{currentmarker}{}%
\end{pgfscope}%
\begin{pgfscope}%
\pgfsys@transformshift{3.939449in}{0.929808in}%
\pgfsys@useobject{currentmarker}{}%
\end{pgfscope}%
\end{pgfscope}%
\begin{pgfscope}%
\pgfpathrectangle{\pgfqpoint{0.740433in}{0.566590in}}{\pgfqpoint{3.295956in}{1.828724in}}%
\pgfusepath{clip}%
\pgfsetrectcap%
\pgfsetroundjoin%
\pgfsetlinewidth{1.505625pt}%
\definecolor{currentstroke}{rgb}{0.000000,0.447000,0.741000}%
\pgfsetstrokecolor{currentstroke}%
\pgfsetdash{}{0pt}%
\pgfpathmoveto{\pgfqpoint{0.740433in}{1.778068in}}%
\pgfpathlineto{\pgfqpoint{0.975858in}{1.760151in}}%
\pgfpathlineto{\pgfqpoint{1.211283in}{1.737247in}}%
\pgfpathlineto{\pgfqpoint{1.446709in}{1.667216in}}%
\pgfpathlineto{\pgfqpoint{1.682134in}{1.528403in}}%
\pgfpathlineto{\pgfqpoint{1.917560in}{1.510995in}}%
\pgfpathlineto{\pgfqpoint{2.152985in}{1.448840in}}%
\pgfpathlineto{\pgfqpoint{2.388411in}{1.144386in}}%
\pgfpathlineto{\pgfqpoint{2.623836in}{1.089683in}}%
\pgfpathlineto{\pgfqpoint{2.859261in}{1.044645in}}%
\pgfpathlineto{\pgfqpoint{3.094687in}{0.988450in}}%
\pgfpathlineto{\pgfqpoint{3.330112in}{0.925228in}}%
\pgfpathlineto{\pgfqpoint{3.565538in}{0.866950in}}%
\pgfpathlineto{\pgfqpoint{3.800963in}{0.812230in}}%
\pgfpathlineto{\pgfqpoint{4.036389in}{0.753578in}}%
\pgfusepath{stroke}%
\end{pgfscope}%
\begin{pgfscope}%
\pgfpathrectangle{\pgfqpoint{0.740433in}{0.566590in}}{\pgfqpoint{3.295956in}{1.828724in}}%
\pgfusepath{clip}%
\pgfsetbuttcap%
\pgfsetroundjoin%
\definecolor{currentfill}{rgb}{0.000000,0.000000,0.000000}%
\pgfsetfillcolor{currentfill}%
\pgfsetfillopacity{0.000000}%
\pgfsetlinewidth{1.003750pt}%
\definecolor{currentstroke}{rgb}{0.000000,0.447000,0.741000}%
\pgfsetstrokecolor{currentstroke}%
\pgfsetdash{}{0pt}%
\pgfsys@defobject{currentmarker}{\pgfqpoint{-0.041667in}{-0.041667in}}{\pgfqpoint{0.041667in}{0.041667in}}{%
\pgfpathmoveto{\pgfqpoint{0.000000in}{-0.041667in}}%
\pgfpathcurveto{\pgfqpoint{0.011050in}{-0.041667in}}{\pgfqpoint{0.021649in}{-0.037276in}}{\pgfqpoint{0.029463in}{-0.029463in}}%
\pgfpathcurveto{\pgfqpoint{0.037276in}{-0.021649in}}{\pgfqpoint{0.041667in}{-0.011050in}}{\pgfqpoint{0.041667in}{0.000000in}}%
\pgfpathcurveto{\pgfqpoint{0.041667in}{0.011050in}}{\pgfqpoint{0.037276in}{0.021649in}}{\pgfqpoint{0.029463in}{0.029463in}}%
\pgfpathcurveto{\pgfqpoint{0.021649in}{0.037276in}}{\pgfqpoint{0.011050in}{0.041667in}}{\pgfqpoint{0.000000in}{0.041667in}}%
\pgfpathcurveto{\pgfqpoint{-0.011050in}{0.041667in}}{\pgfqpoint{-0.021649in}{0.037276in}}{\pgfqpoint{-0.029463in}{0.029463in}}%
\pgfpathcurveto{\pgfqpoint{-0.037276in}{0.021649in}}{\pgfqpoint{-0.041667in}{0.011050in}}{\pgfqpoint{-0.041667in}{0.000000in}}%
\pgfpathcurveto{\pgfqpoint{-0.041667in}{-0.011050in}}{\pgfqpoint{-0.037276in}{-0.021649in}}{\pgfqpoint{-0.029463in}{-0.029463in}}%
\pgfpathcurveto{\pgfqpoint{-0.021649in}{-0.037276in}}{\pgfqpoint{-0.011050in}{-0.041667in}}{\pgfqpoint{0.000000in}{-0.041667in}}%
\pgfpathclose%
\pgfusepath{stroke,fill}%
}%
\begin{pgfscope}%
\pgfsys@transformshift{0.740433in}{1.778068in}%
\pgfsys@useobject{currentmarker}{}%
\end{pgfscope}%
\begin{pgfscope}%
\pgfsys@transformshift{0.975858in}{1.760151in}%
\pgfsys@useobject{currentmarker}{}%
\end{pgfscope}%
\begin{pgfscope}%
\pgfsys@transformshift{1.211283in}{1.737247in}%
\pgfsys@useobject{currentmarker}{}%
\end{pgfscope}%
\begin{pgfscope}%
\pgfsys@transformshift{1.446709in}{1.667216in}%
\pgfsys@useobject{currentmarker}{}%
\end{pgfscope}%
\begin{pgfscope}%
\pgfsys@transformshift{1.682134in}{1.528403in}%
\pgfsys@useobject{currentmarker}{}%
\end{pgfscope}%
\begin{pgfscope}%
\pgfsys@transformshift{1.917560in}{1.510995in}%
\pgfsys@useobject{currentmarker}{}%
\end{pgfscope}%
\begin{pgfscope}%
\pgfsys@transformshift{2.152985in}{1.448840in}%
\pgfsys@useobject{currentmarker}{}%
\end{pgfscope}%
\begin{pgfscope}%
\pgfsys@transformshift{2.388411in}{1.144386in}%
\pgfsys@useobject{currentmarker}{}%
\end{pgfscope}%
\begin{pgfscope}%
\pgfsys@transformshift{2.623836in}{1.089683in}%
\pgfsys@useobject{currentmarker}{}%
\end{pgfscope}%
\begin{pgfscope}%
\pgfsys@transformshift{2.859261in}{1.044645in}%
\pgfsys@useobject{currentmarker}{}%
\end{pgfscope}%
\begin{pgfscope}%
\pgfsys@transformshift{3.094687in}{0.988450in}%
\pgfsys@useobject{currentmarker}{}%
\end{pgfscope}%
\begin{pgfscope}%
\pgfsys@transformshift{3.330112in}{0.925228in}%
\pgfsys@useobject{currentmarker}{}%
\end{pgfscope}%
\begin{pgfscope}%
\pgfsys@transformshift{3.565538in}{0.866950in}%
\pgfsys@useobject{currentmarker}{}%
\end{pgfscope}%
\begin{pgfscope}%
\pgfsys@transformshift{3.800963in}{0.812230in}%
\pgfsys@useobject{currentmarker}{}%
\end{pgfscope}%
\begin{pgfscope}%
\pgfsys@transformshift{4.036389in}{0.753578in}%
\pgfsys@useobject{currentmarker}{}%
\end{pgfscope}%
\end{pgfscope}%
\begin{pgfscope}%
\pgfpathrectangle{\pgfqpoint{0.740433in}{0.566590in}}{\pgfqpoint{3.295956in}{1.828724in}}%
\pgfusepath{clip}%
\pgfsetrectcap%
\pgfsetroundjoin%
\pgfsetlinewidth{1.505625pt}%
\definecolor{currentstroke}{rgb}{0.850000,0.324000,0.098000}%
\pgfsetstrokecolor{currentstroke}%
\pgfsetdash{}{0pt}%
\pgfpathmoveto{\pgfqpoint{0.740433in}{1.679809in}}%
\pgfpathlineto{\pgfqpoint{0.975858in}{1.599737in}}%
\pgfpathlineto{\pgfqpoint{1.211283in}{1.504484in}}%
\pgfpathlineto{\pgfqpoint{1.446709in}{1.481568in}}%
\pgfpathlineto{\pgfqpoint{1.682134in}{1.461425in}}%
\pgfpathlineto{\pgfqpoint{1.917560in}{1.439835in}}%
\pgfpathlineto{\pgfqpoint{2.152985in}{1.371778in}}%
\pgfpathlineto{\pgfqpoint{2.388411in}{1.189070in}}%
\pgfpathlineto{\pgfqpoint{2.623836in}{1.133366in}}%
\pgfpathlineto{\pgfqpoint{2.859261in}{1.088340in}}%
\pgfpathlineto{\pgfqpoint{3.094687in}{1.032379in}}%
\pgfpathlineto{\pgfqpoint{3.330112in}{0.970331in}}%
\pgfpathlineto{\pgfqpoint{3.565538in}{0.912363in}}%
\pgfpathlineto{\pgfqpoint{3.800963in}{0.856801in}}%
\pgfpathlineto{\pgfqpoint{4.036389in}{0.800111in}}%
\pgfusepath{stroke}%
\end{pgfscope}%
\begin{pgfscope}%
\pgfpathrectangle{\pgfqpoint{0.740433in}{0.566590in}}{\pgfqpoint{3.295956in}{1.828724in}}%
\pgfusepath{clip}%
\pgfsetbuttcap%
\pgfsetroundjoin%
\definecolor{currentfill}{rgb}{0.850000,0.324000,0.098000}%
\pgfsetfillcolor{currentfill}%
\pgfsetlinewidth{1.003750pt}%
\definecolor{currentstroke}{rgb}{0.850000,0.324000,0.098000}%
\pgfsetstrokecolor{currentstroke}%
\pgfsetdash{}{0pt}%
\pgfsys@defobject{currentmarker}{\pgfqpoint{-0.041667in}{-0.041667in}}{\pgfqpoint{0.041667in}{0.041667in}}{%
\pgfpathmoveto{\pgfqpoint{-0.041667in}{0.000000in}}%
\pgfpathlineto{\pgfqpoint{0.041667in}{0.000000in}}%
\pgfpathmoveto{\pgfqpoint{0.000000in}{-0.041667in}}%
\pgfpathlineto{\pgfqpoint{0.000000in}{0.041667in}}%
\pgfusepath{stroke,fill}%
}%
\begin{pgfscope}%
\pgfsys@transformshift{0.740433in}{1.679809in}%
\pgfsys@useobject{currentmarker}{}%
\end{pgfscope}%
\begin{pgfscope}%
\pgfsys@transformshift{0.975858in}{1.599737in}%
\pgfsys@useobject{currentmarker}{}%
\end{pgfscope}%
\begin{pgfscope}%
\pgfsys@transformshift{1.211283in}{1.504484in}%
\pgfsys@useobject{currentmarker}{}%
\end{pgfscope}%
\begin{pgfscope}%
\pgfsys@transformshift{1.446709in}{1.481568in}%
\pgfsys@useobject{currentmarker}{}%
\end{pgfscope}%
\begin{pgfscope}%
\pgfsys@transformshift{1.682134in}{1.461425in}%
\pgfsys@useobject{currentmarker}{}%
\end{pgfscope}%
\begin{pgfscope}%
\pgfsys@transformshift{1.917560in}{1.439835in}%
\pgfsys@useobject{currentmarker}{}%
\end{pgfscope}%
\begin{pgfscope}%
\pgfsys@transformshift{2.152985in}{1.371778in}%
\pgfsys@useobject{currentmarker}{}%
\end{pgfscope}%
\begin{pgfscope}%
\pgfsys@transformshift{2.388411in}{1.189070in}%
\pgfsys@useobject{currentmarker}{}%
\end{pgfscope}%
\begin{pgfscope}%
\pgfsys@transformshift{2.623836in}{1.133366in}%
\pgfsys@useobject{currentmarker}{}%
\end{pgfscope}%
\begin{pgfscope}%
\pgfsys@transformshift{2.859261in}{1.088340in}%
\pgfsys@useobject{currentmarker}{}%
\end{pgfscope}%
\begin{pgfscope}%
\pgfsys@transformshift{3.094687in}{1.032379in}%
\pgfsys@useobject{currentmarker}{}%
\end{pgfscope}%
\begin{pgfscope}%
\pgfsys@transformshift{3.330112in}{0.970331in}%
\pgfsys@useobject{currentmarker}{}%
\end{pgfscope}%
\begin{pgfscope}%
\pgfsys@transformshift{3.565538in}{0.912363in}%
\pgfsys@useobject{currentmarker}{}%
\end{pgfscope}%
\begin{pgfscope}%
\pgfsys@transformshift{3.800963in}{0.856801in}%
\pgfsys@useobject{currentmarker}{}%
\end{pgfscope}%
\begin{pgfscope}%
\pgfsys@transformshift{4.036389in}{0.800111in}%
\pgfsys@useobject{currentmarker}{}%
\end{pgfscope}%
\end{pgfscope}%
\begin{pgfscope}%
\pgfpathrectangle{\pgfqpoint{0.740433in}{0.566590in}}{\pgfqpoint{3.295956in}{1.828724in}}%
\pgfusepath{clip}%
\pgfsetrectcap%
\pgfsetroundjoin%
\pgfsetlinewidth{1.505625pt}%
\definecolor{currentstroke}{rgb}{0.000000,0.500000,0.000000}%
\pgfsetstrokecolor{currentstroke}%
\pgfsetdash{}{0pt}%
\pgfpathmoveto{\pgfqpoint{0.740433in}{1.576673in}}%
\pgfpathlineto{\pgfqpoint{0.975858in}{1.563058in}}%
\pgfpathlineto{\pgfqpoint{1.211283in}{1.547968in}}%
\pgfpathlineto{\pgfqpoint{1.446709in}{1.524330in}}%
\pgfpathlineto{\pgfqpoint{1.682134in}{1.510545in}}%
\pgfpathlineto{\pgfqpoint{1.917560in}{1.491459in}}%
\pgfpathlineto{\pgfqpoint{2.152985in}{1.434451in}}%
\pgfpathlineto{\pgfqpoint{2.388411in}{1.145733in}}%
\pgfpathlineto{\pgfqpoint{2.623836in}{1.089820in}}%
\pgfpathlineto{\pgfqpoint{2.859261in}{1.043314in}}%
\pgfpathlineto{\pgfqpoint{3.094687in}{0.990105in}}%
\pgfpathlineto{\pgfqpoint{3.330112in}{0.926796in}}%
\pgfpathlineto{\pgfqpoint{3.565538in}{0.865145in}}%
\pgfpathlineto{\pgfqpoint{3.800963in}{0.813388in}}%
\pgfpathlineto{\pgfqpoint{4.036389in}{0.752493in}}%
\pgfusepath{stroke}%
\end{pgfscope}%
\begin{pgfscope}%
\pgfpathrectangle{\pgfqpoint{0.740433in}{0.566590in}}{\pgfqpoint{3.295956in}{1.828724in}}%
\pgfusepath{clip}%
\pgfsetbuttcap%
\pgfsetmiterjoin%
\definecolor{currentfill}{rgb}{0.000000,0.000000,0.000000}%
\pgfsetfillcolor{currentfill}%
\pgfsetfillopacity{0.000000}%
\pgfsetlinewidth{1.003750pt}%
\definecolor{currentstroke}{rgb}{0.000000,0.500000,0.000000}%
\pgfsetstrokecolor{currentstroke}%
\pgfsetdash{}{0pt}%
\pgfsys@defobject{currentmarker}{\pgfqpoint{-0.041667in}{-0.041667in}}{\pgfqpoint{0.041667in}{0.041667in}}{%
\pgfpathmoveto{\pgfqpoint{-0.041667in}{-0.041667in}}%
\pgfpathlineto{\pgfqpoint{0.041667in}{-0.041667in}}%
\pgfpathlineto{\pgfqpoint{0.041667in}{0.041667in}}%
\pgfpathlineto{\pgfqpoint{-0.041667in}{0.041667in}}%
\pgfpathclose%
\pgfusepath{stroke,fill}%
}%
\begin{pgfscope}%
\pgfsys@transformshift{0.740433in}{1.576673in}%
\pgfsys@useobject{currentmarker}{}%
\end{pgfscope}%
\begin{pgfscope}%
\pgfsys@transformshift{0.975858in}{1.563058in}%
\pgfsys@useobject{currentmarker}{}%
\end{pgfscope}%
\begin{pgfscope}%
\pgfsys@transformshift{1.211283in}{1.547968in}%
\pgfsys@useobject{currentmarker}{}%
\end{pgfscope}%
\begin{pgfscope}%
\pgfsys@transformshift{1.446709in}{1.524330in}%
\pgfsys@useobject{currentmarker}{}%
\end{pgfscope}%
\begin{pgfscope}%
\pgfsys@transformshift{1.682134in}{1.510545in}%
\pgfsys@useobject{currentmarker}{}%
\end{pgfscope}%
\begin{pgfscope}%
\pgfsys@transformshift{1.917560in}{1.491459in}%
\pgfsys@useobject{currentmarker}{}%
\end{pgfscope}%
\begin{pgfscope}%
\pgfsys@transformshift{2.152985in}{1.434451in}%
\pgfsys@useobject{currentmarker}{}%
\end{pgfscope}%
\begin{pgfscope}%
\pgfsys@transformshift{2.388411in}{1.145733in}%
\pgfsys@useobject{currentmarker}{}%
\end{pgfscope}%
\begin{pgfscope}%
\pgfsys@transformshift{2.623836in}{1.089820in}%
\pgfsys@useobject{currentmarker}{}%
\end{pgfscope}%
\begin{pgfscope}%
\pgfsys@transformshift{2.859261in}{1.043314in}%
\pgfsys@useobject{currentmarker}{}%
\end{pgfscope}%
\begin{pgfscope}%
\pgfsys@transformshift{3.094687in}{0.990105in}%
\pgfsys@useobject{currentmarker}{}%
\end{pgfscope}%
\begin{pgfscope}%
\pgfsys@transformshift{3.330112in}{0.926796in}%
\pgfsys@useobject{currentmarker}{}%
\end{pgfscope}%
\begin{pgfscope}%
\pgfsys@transformshift{3.565538in}{0.865145in}%
\pgfsys@useobject{currentmarker}{}%
\end{pgfscope}%
\begin{pgfscope}%
\pgfsys@transformshift{3.800963in}{0.813388in}%
\pgfsys@useobject{currentmarker}{}%
\end{pgfscope}%
\begin{pgfscope}%
\pgfsys@transformshift{4.036389in}{0.752493in}%
\pgfsys@useobject{currentmarker}{}%
\end{pgfscope}%
\end{pgfscope}%
\begin{pgfscope}%
\pgfpathrectangle{\pgfqpoint{0.740433in}{0.566590in}}{\pgfqpoint{3.295956in}{1.828724in}}%
\pgfusepath{clip}%
\pgfsetrectcap%
\pgfsetroundjoin%
\pgfsetlinewidth{1.505625pt}%
\definecolor{currentstroke}{rgb}{0.494000,0.184000,0.556000}%
\pgfsetstrokecolor{currentstroke}%
\pgfsetdash{}{0pt}%
\pgfpathmoveto{\pgfqpoint{0.740433in}{1.610009in}}%
\pgfpathlineto{\pgfqpoint{0.975858in}{1.599455in}}%
\pgfpathlineto{\pgfqpoint{1.211283in}{1.548231in}}%
\pgfpathlineto{\pgfqpoint{1.446709in}{1.494431in}}%
\pgfpathlineto{\pgfqpoint{1.682134in}{1.483528in}}%
\pgfpathlineto{\pgfqpoint{1.917560in}{1.465692in}}%
\pgfpathlineto{\pgfqpoint{2.152985in}{1.407623in}}%
\pgfpathlineto{\pgfqpoint{2.388411in}{1.070150in}}%
\pgfpathlineto{\pgfqpoint{2.623836in}{1.021499in}}%
\pgfpathlineto{\pgfqpoint{2.859261in}{0.977924in}}%
\pgfpathlineto{\pgfqpoint{3.094687in}{0.924216in}}%
\pgfpathlineto{\pgfqpoint{3.330112in}{0.855198in}}%
\pgfpathlineto{\pgfqpoint{3.565538in}{0.798163in}}%
\pgfpathlineto{\pgfqpoint{3.800963in}{0.749107in}}%
\pgfpathlineto{\pgfqpoint{4.036389in}{0.687127in}}%
\pgfusepath{stroke}%
\end{pgfscope}%
\begin{pgfscope}%
\pgfpathrectangle{\pgfqpoint{0.740433in}{0.566590in}}{\pgfqpoint{3.295956in}{1.828724in}}%
\pgfusepath{clip}%
\pgfsetbuttcap%
\pgfsetroundjoin%
\definecolor{currentfill}{rgb}{0.494000,0.184000,0.556000}%
\pgfsetfillcolor{currentfill}%
\pgfsetlinewidth{1.003750pt}%
\definecolor{currentstroke}{rgb}{0.494000,0.184000,0.556000}%
\pgfsetstrokecolor{currentstroke}%
\pgfsetdash{}{0pt}%
\pgfsys@defobject{currentmarker}{\pgfqpoint{-0.041667in}{-0.041667in}}{\pgfqpoint{0.041667in}{0.041667in}}{%
\pgfpathmoveto{\pgfqpoint{-0.041667in}{-0.041667in}}%
\pgfpathlineto{\pgfqpoint{0.041667in}{0.041667in}}%
\pgfpathmoveto{\pgfqpoint{-0.041667in}{0.041667in}}%
\pgfpathlineto{\pgfqpoint{0.041667in}{-0.041667in}}%
\pgfusepath{stroke,fill}%
}%
\begin{pgfscope}%
\pgfsys@transformshift{0.740433in}{1.610009in}%
\pgfsys@useobject{currentmarker}{}%
\end{pgfscope}%
\begin{pgfscope}%
\pgfsys@transformshift{0.975858in}{1.599455in}%
\pgfsys@useobject{currentmarker}{}%
\end{pgfscope}%
\begin{pgfscope}%
\pgfsys@transformshift{1.211283in}{1.548231in}%
\pgfsys@useobject{currentmarker}{}%
\end{pgfscope}%
\begin{pgfscope}%
\pgfsys@transformshift{1.446709in}{1.494431in}%
\pgfsys@useobject{currentmarker}{}%
\end{pgfscope}%
\begin{pgfscope}%
\pgfsys@transformshift{1.682134in}{1.483528in}%
\pgfsys@useobject{currentmarker}{}%
\end{pgfscope}%
\begin{pgfscope}%
\pgfsys@transformshift{1.917560in}{1.465692in}%
\pgfsys@useobject{currentmarker}{}%
\end{pgfscope}%
\begin{pgfscope}%
\pgfsys@transformshift{2.152985in}{1.407623in}%
\pgfsys@useobject{currentmarker}{}%
\end{pgfscope}%
\begin{pgfscope}%
\pgfsys@transformshift{2.388411in}{1.070150in}%
\pgfsys@useobject{currentmarker}{}%
\end{pgfscope}%
\begin{pgfscope}%
\pgfsys@transformshift{2.623836in}{1.021499in}%
\pgfsys@useobject{currentmarker}{}%
\end{pgfscope}%
\begin{pgfscope}%
\pgfsys@transformshift{2.859261in}{0.977924in}%
\pgfsys@useobject{currentmarker}{}%
\end{pgfscope}%
\begin{pgfscope}%
\pgfsys@transformshift{3.094687in}{0.924216in}%
\pgfsys@useobject{currentmarker}{}%
\end{pgfscope}%
\begin{pgfscope}%
\pgfsys@transformshift{3.330112in}{0.855198in}%
\pgfsys@useobject{currentmarker}{}%
\end{pgfscope}%
\begin{pgfscope}%
\pgfsys@transformshift{3.565538in}{0.798163in}%
\pgfsys@useobject{currentmarker}{}%
\end{pgfscope}%
\begin{pgfscope}%
\pgfsys@transformshift{3.800963in}{0.749107in}%
\pgfsys@useobject{currentmarker}{}%
\end{pgfscope}%
\begin{pgfscope}%
\pgfsys@transformshift{4.036389in}{0.687127in}%
\pgfsys@useobject{currentmarker}{}%
\end{pgfscope}%
\end{pgfscope}%
\begin{pgfscope}%
\pgfpathrectangle{\pgfqpoint{0.740433in}{0.566590in}}{\pgfqpoint{3.295956in}{1.828724in}}%
\pgfusepath{clip}%
\pgfsetrectcap%
\pgfsetroundjoin%
\pgfsetlinewidth{1.505625pt}%
\definecolor{currentstroke}{rgb}{0.635000,0.078000,0.184000}%
\pgfsetstrokecolor{currentstroke}%
\pgfsetdash{}{0pt}%
\pgfpathmoveto{\pgfqpoint{0.740433in}{1.728716in}}%
\pgfpathlineto{\pgfqpoint{0.975858in}{1.720582in}}%
\pgfpathlineto{\pgfqpoint{1.211283in}{1.691357in}}%
\pgfpathlineto{\pgfqpoint{1.446709in}{1.655565in}}%
\pgfpathlineto{\pgfqpoint{1.682134in}{1.610922in}}%
\pgfpathlineto{\pgfqpoint{1.917560in}{1.587488in}}%
\pgfpathlineto{\pgfqpoint{2.152985in}{1.520567in}}%
\pgfpathlineto{\pgfqpoint{2.388411in}{1.363066in}}%
\pgfpathlineto{\pgfqpoint{2.623836in}{1.314744in}}%
\pgfpathlineto{\pgfqpoint{2.859261in}{1.259467in}}%
\pgfpathlineto{\pgfqpoint{3.094687in}{1.207009in}}%
\pgfpathlineto{\pgfqpoint{3.330112in}{1.145667in}}%
\pgfpathlineto{\pgfqpoint{3.565538in}{1.086750in}}%
\pgfpathlineto{\pgfqpoint{3.800963in}{1.037921in}}%
\pgfpathlineto{\pgfqpoint{4.036389in}{0.976660in}}%
\pgfusepath{stroke}%
\end{pgfscope}%
\begin{pgfscope}%
\pgfpathrectangle{\pgfqpoint{0.740433in}{0.566590in}}{\pgfqpoint{3.295956in}{1.828724in}}%
\pgfusepath{clip}%
\pgfsetbuttcap%
\pgfsetmiterjoin%
\definecolor{currentfill}{rgb}{0.000000,0.000000,0.000000}%
\pgfsetfillcolor{currentfill}%
\pgfsetfillopacity{0.000000}%
\pgfsetlinewidth{1.003750pt}%
\definecolor{currentstroke}{rgb}{0.635000,0.078000,0.184000}%
\pgfsetstrokecolor{currentstroke}%
\pgfsetdash{}{0pt}%
\pgfsys@defobject{currentmarker}{\pgfqpoint{-0.035355in}{-0.058926in}}{\pgfqpoint{0.035355in}{0.058926in}}{%
\pgfpathmoveto{\pgfqpoint{-0.000000in}{-0.058926in}}%
\pgfpathlineto{\pgfqpoint{0.035355in}{0.000000in}}%
\pgfpathlineto{\pgfqpoint{0.000000in}{0.058926in}}%
\pgfpathlineto{\pgfqpoint{-0.035355in}{0.000000in}}%
\pgfpathclose%
\pgfusepath{stroke,fill}%
}%
\begin{pgfscope}%
\pgfsys@transformshift{0.740433in}{1.728716in}%
\pgfsys@useobject{currentmarker}{}%
\end{pgfscope}%
\begin{pgfscope}%
\pgfsys@transformshift{0.975858in}{1.720582in}%
\pgfsys@useobject{currentmarker}{}%
\end{pgfscope}%
\begin{pgfscope}%
\pgfsys@transformshift{1.211283in}{1.691357in}%
\pgfsys@useobject{currentmarker}{}%
\end{pgfscope}%
\begin{pgfscope}%
\pgfsys@transformshift{1.446709in}{1.655565in}%
\pgfsys@useobject{currentmarker}{}%
\end{pgfscope}%
\begin{pgfscope}%
\pgfsys@transformshift{1.682134in}{1.610922in}%
\pgfsys@useobject{currentmarker}{}%
\end{pgfscope}%
\begin{pgfscope}%
\pgfsys@transformshift{1.917560in}{1.587488in}%
\pgfsys@useobject{currentmarker}{}%
\end{pgfscope}%
\begin{pgfscope}%
\pgfsys@transformshift{2.152985in}{1.520567in}%
\pgfsys@useobject{currentmarker}{}%
\end{pgfscope}%
\begin{pgfscope}%
\pgfsys@transformshift{2.388411in}{1.363066in}%
\pgfsys@useobject{currentmarker}{}%
\end{pgfscope}%
\begin{pgfscope}%
\pgfsys@transformshift{2.623836in}{1.314744in}%
\pgfsys@useobject{currentmarker}{}%
\end{pgfscope}%
\begin{pgfscope}%
\pgfsys@transformshift{2.859261in}{1.259467in}%
\pgfsys@useobject{currentmarker}{}%
\end{pgfscope}%
\begin{pgfscope}%
\pgfsys@transformshift{3.094687in}{1.207009in}%
\pgfsys@useobject{currentmarker}{}%
\end{pgfscope}%
\begin{pgfscope}%
\pgfsys@transformshift{3.330112in}{1.145667in}%
\pgfsys@useobject{currentmarker}{}%
\end{pgfscope}%
\begin{pgfscope}%
\pgfsys@transformshift{3.565538in}{1.086750in}%
\pgfsys@useobject{currentmarker}{}%
\end{pgfscope}%
\begin{pgfscope}%
\pgfsys@transformshift{3.800963in}{1.037921in}%
\pgfsys@useobject{currentmarker}{}%
\end{pgfscope}%
\begin{pgfscope}%
\pgfsys@transformshift{4.036389in}{0.976660in}%
\pgfsys@useobject{currentmarker}{}%
\end{pgfscope}%
\end{pgfscope}%
\begin{pgfscope}%
\pgfsetrectcap%
\pgfsetmiterjoin%
\pgfsetlinewidth{0.803000pt}%
\definecolor{currentstroke}{rgb}{0.000000,0.000000,0.000000}%
\pgfsetstrokecolor{currentstroke}%
\pgfsetdash{}{0pt}%
\pgfpathmoveto{\pgfqpoint{0.740433in}{0.566590in}}%
\pgfpathlineto{\pgfqpoint{0.740433in}{2.395314in}}%
\pgfusepath{stroke}%
\end{pgfscope}%
\begin{pgfscope}%
\pgfsetrectcap%
\pgfsetmiterjoin%
\pgfsetlinewidth{0.803000pt}%
\definecolor{currentstroke}{rgb}{0.000000,0.000000,0.000000}%
\pgfsetstrokecolor{currentstroke}%
\pgfsetdash{}{0pt}%
\pgfpathmoveto{\pgfqpoint{4.036389in}{0.566590in}}%
\pgfpathlineto{\pgfqpoint{4.036389in}{2.395314in}}%
\pgfusepath{stroke}%
\end{pgfscope}%
\begin{pgfscope}%
\pgfsetrectcap%
\pgfsetmiterjoin%
\pgfsetlinewidth{0.803000pt}%
\definecolor{currentstroke}{rgb}{0.000000,0.000000,0.000000}%
\pgfsetstrokecolor{currentstroke}%
\pgfsetdash{}{0pt}%
\pgfpathmoveto{\pgfqpoint{0.740433in}{0.566590in}}%
\pgfpathlineto{\pgfqpoint{4.036389in}{0.566590in}}%
\pgfusepath{stroke}%
\end{pgfscope}%
\begin{pgfscope}%
\pgfsetrectcap%
\pgfsetmiterjoin%
\pgfsetlinewidth{0.803000pt}%
\definecolor{currentstroke}{rgb}{0.000000,0.000000,0.000000}%
\pgfsetstrokecolor{currentstroke}%
\pgfsetdash{}{0pt}%
\pgfpathmoveto{\pgfqpoint{0.740433in}{2.395314in}}%
\pgfpathlineto{\pgfqpoint{4.036389in}{2.395314in}}%
\pgfusepath{stroke}%
\end{pgfscope}%
\begin{pgfscope}%
\pgfsetbuttcap%
\pgfsetmiterjoin%
\definecolor{currentfill}{rgb}{1.000000,1.000000,1.000000}%
\pgfsetfillcolor{currentfill}%
\pgfsetfillopacity{0.800000}%
\pgfsetlinewidth{1.003750pt}%
\definecolor{currentstroke}{rgb}{0.800000,0.800000,0.800000}%
\pgfsetstrokecolor{currentstroke}%
\pgfsetstrokeopacity{0.800000}%
\pgfsetdash{}{0pt}%
\pgfpathmoveto{\pgfqpoint{2.841524in}{1.423816in}}%
\pgfpathlineto{\pgfqpoint{3.948889in}{1.423816in}}%
\pgfpathquadraticcurveto{\pgfqpoint{3.973889in}{1.423816in}}{\pgfqpoint{3.973889in}{1.448816in}}%
\pgfpathlineto{\pgfqpoint{3.973889in}{2.307814in}}%
\pgfpathquadraticcurveto{\pgfqpoint{3.973889in}{2.332814in}}{\pgfqpoint{3.948889in}{2.332814in}}%
\pgfpathlineto{\pgfqpoint{2.841524in}{2.332814in}}%
\pgfpathquadraticcurveto{\pgfqpoint{2.816524in}{2.332814in}}{\pgfqpoint{2.816524in}{2.307814in}}%
\pgfpathlineto{\pgfqpoint{2.816524in}{1.448816in}}%
\pgfpathquadraticcurveto{\pgfqpoint{2.816524in}{1.423816in}}{\pgfqpoint{2.841524in}{1.423816in}}%
\pgfpathclose%
\pgfusepath{stroke,fill}%
\end{pgfscope}%
\begin{pgfscope}%
\pgfsetbuttcap%
\pgfsetroundjoin%
\definecolor{currentfill}{rgb}{0.000000,0.000000,0.000000}%
\pgfsetfillcolor{currentfill}%
\pgfsetfillopacity{0.000000}%
\pgfsetlinewidth{1.003750pt}%
\definecolor{currentstroke}{rgb}{0.000000,0.447000,0.741000}%
\pgfsetstrokecolor{currentstroke}%
\pgfsetdash{}{0pt}%
\pgfsys@defobject{currentmarker}{\pgfqpoint{-0.041667in}{-0.041667in}}{\pgfqpoint{0.041667in}{0.041667in}}{%
\pgfpathmoveto{\pgfqpoint{0.000000in}{-0.041667in}}%
\pgfpathcurveto{\pgfqpoint{0.011050in}{-0.041667in}}{\pgfqpoint{0.021649in}{-0.037276in}}{\pgfqpoint{0.029463in}{-0.029463in}}%
\pgfpathcurveto{\pgfqpoint{0.037276in}{-0.021649in}}{\pgfqpoint{0.041667in}{-0.011050in}}{\pgfqpoint{0.041667in}{0.000000in}}%
\pgfpathcurveto{\pgfqpoint{0.041667in}{0.011050in}}{\pgfqpoint{0.037276in}{0.021649in}}{\pgfqpoint{0.029463in}{0.029463in}}%
\pgfpathcurveto{\pgfqpoint{0.021649in}{0.037276in}}{\pgfqpoint{0.011050in}{0.041667in}}{\pgfqpoint{0.000000in}{0.041667in}}%
\pgfpathcurveto{\pgfqpoint{-0.011050in}{0.041667in}}{\pgfqpoint{-0.021649in}{0.037276in}}{\pgfqpoint{-0.029463in}{0.029463in}}%
\pgfpathcurveto{\pgfqpoint{-0.037276in}{0.021649in}}{\pgfqpoint{-0.041667in}{0.011050in}}{\pgfqpoint{-0.041667in}{0.000000in}}%
\pgfpathcurveto{\pgfqpoint{-0.041667in}{-0.011050in}}{\pgfqpoint{-0.037276in}{-0.021649in}}{\pgfqpoint{-0.029463in}{-0.029463in}}%
\pgfpathcurveto{\pgfqpoint{-0.021649in}{-0.037276in}}{\pgfqpoint{-0.011050in}{-0.041667in}}{\pgfqpoint{0.000000in}{-0.041667in}}%
\pgfpathclose%
\pgfusepath{stroke,fill}%
}%
\begin{pgfscope}%
\pgfsys@transformshift{2.991524in}{2.239064in}%
\pgfsys@useobject{currentmarker}{}%
\end{pgfscope}%
\end{pgfscope}%
\begin{pgfscope}%
\definecolor{textcolor}{rgb}{0.000000,0.000000,0.000000}%
\pgfsetstrokecolor{textcolor}%
\pgfsetfillcolor{textcolor}%
\pgftext[x=3.216524in,y=2.195314in,left,base]{\color{textcolor}\rmfamily\fontsize{9.000000}{10.800000}\selectfont \(\displaystyle \nu_{21} = \) 434.00}%
\end{pgfscope}%
\begin{pgfscope}%
\pgfsetbuttcap%
\pgfsetroundjoin%
\definecolor{currentfill}{rgb}{0.850000,0.324000,0.098000}%
\pgfsetfillcolor{currentfill}%
\pgfsetlinewidth{1.003750pt}%
\definecolor{currentstroke}{rgb}{0.850000,0.324000,0.098000}%
\pgfsetstrokecolor{currentstroke}%
\pgfsetdash{}{0pt}%
\pgfsys@defobject{currentmarker}{\pgfqpoint{-0.041667in}{-0.041667in}}{\pgfqpoint{0.041667in}{0.041667in}}{%
\pgfpathmoveto{\pgfqpoint{-0.041667in}{0.000000in}}%
\pgfpathlineto{\pgfqpoint{0.041667in}{0.000000in}}%
\pgfpathmoveto{\pgfqpoint{0.000000in}{-0.041667in}}%
\pgfpathlineto{\pgfqpoint{0.000000in}{0.041667in}}%
\pgfusepath{stroke,fill}%
}%
\begin{pgfscope}%
\pgfsys@transformshift{2.991524in}{2.064765in}%
\pgfsys@useobject{currentmarker}{}%
\end{pgfscope}%
\end{pgfscope}%
\begin{pgfscope}%
\definecolor{textcolor}{rgb}{0.000000,0.000000,0.000000}%
\pgfsetstrokecolor{textcolor}%
\pgfsetfillcolor{textcolor}%
\pgftext[x=3.216524in,y=2.021015in,left,base]{\color{textcolor}\rmfamily\fontsize{9.000000}{10.800000}\selectfont \(\displaystyle \nu_{22} = \) 435.38}%
\end{pgfscope}%
\begin{pgfscope}%
\pgfsetbuttcap%
\pgfsetmiterjoin%
\definecolor{currentfill}{rgb}{0.000000,0.000000,0.000000}%
\pgfsetfillcolor{currentfill}%
\pgfsetfillopacity{0.000000}%
\pgfsetlinewidth{1.003750pt}%
\definecolor{currentstroke}{rgb}{0.000000,0.500000,0.000000}%
\pgfsetstrokecolor{currentstroke}%
\pgfsetdash{}{0pt}%
\pgfsys@defobject{currentmarker}{\pgfqpoint{-0.041667in}{-0.041667in}}{\pgfqpoint{0.041667in}{0.041667in}}{%
\pgfpathmoveto{\pgfqpoint{-0.041667in}{-0.041667in}}%
\pgfpathlineto{\pgfqpoint{0.041667in}{-0.041667in}}%
\pgfpathlineto{\pgfqpoint{0.041667in}{0.041667in}}%
\pgfpathlineto{\pgfqpoint{-0.041667in}{0.041667in}}%
\pgfpathclose%
\pgfusepath{stroke,fill}%
}%
\begin{pgfscope}%
\pgfsys@transformshift{2.991524in}{1.890465in}%
\pgfsys@useobject{currentmarker}{}%
\end{pgfscope}%
\end{pgfscope}%
\begin{pgfscope}%
\definecolor{textcolor}{rgb}{0.000000,0.000000,0.000000}%
\pgfsetstrokecolor{textcolor}%
\pgfsetfillcolor{textcolor}%
\pgftext[x=3.216524in,y=1.846715in,left,base]{\color{textcolor}\rmfamily\fontsize{9.000000}{10.800000}\selectfont \(\displaystyle \nu_{23} = \) 436.19}%
\end{pgfscope}%
\begin{pgfscope}%
\pgfsetbuttcap%
\pgfsetroundjoin%
\definecolor{currentfill}{rgb}{0.494000,0.184000,0.556000}%
\pgfsetfillcolor{currentfill}%
\pgfsetlinewidth{1.003750pt}%
\definecolor{currentstroke}{rgb}{0.494000,0.184000,0.556000}%
\pgfsetstrokecolor{currentstroke}%
\pgfsetdash{}{0pt}%
\pgfsys@defobject{currentmarker}{\pgfqpoint{-0.041667in}{-0.041667in}}{\pgfqpoint{0.041667in}{0.041667in}}{%
\pgfpathmoveto{\pgfqpoint{-0.041667in}{-0.041667in}}%
\pgfpathlineto{\pgfqpoint{0.041667in}{0.041667in}}%
\pgfpathmoveto{\pgfqpoint{-0.041667in}{0.041667in}}%
\pgfpathlineto{\pgfqpoint{0.041667in}{-0.041667in}}%
\pgfusepath{stroke,fill}%
}%
\begin{pgfscope}%
\pgfsys@transformshift{2.991524in}{1.716165in}%
\pgfsys@useobject{currentmarker}{}%
\end{pgfscope}%
\end{pgfscope}%
\begin{pgfscope}%
\definecolor{textcolor}{rgb}{0.000000,0.000000,0.000000}%
\pgfsetstrokecolor{textcolor}%
\pgfsetfillcolor{textcolor}%
\pgftext[x=3.216524in,y=1.672415in,left,base]{\color{textcolor}\rmfamily\fontsize{9.000000}{10.800000}\selectfont \(\displaystyle \nu_{24} = \) 437.97}%
\end{pgfscope}%
\begin{pgfscope}%
\pgfsetbuttcap%
\pgfsetmiterjoin%
\definecolor{currentfill}{rgb}{0.000000,0.000000,0.000000}%
\pgfsetfillcolor{currentfill}%
\pgfsetfillopacity{0.000000}%
\pgfsetlinewidth{1.003750pt}%
\definecolor{currentstroke}{rgb}{0.635000,0.078000,0.184000}%
\pgfsetstrokecolor{currentstroke}%
\pgfsetdash{}{0pt}%
\pgfsys@defobject{currentmarker}{\pgfqpoint{-0.035355in}{-0.058926in}}{\pgfqpoint{0.035355in}{0.058926in}}{%
\pgfpathmoveto{\pgfqpoint{-0.000000in}{-0.058926in}}%
\pgfpathlineto{\pgfqpoint{0.035355in}{0.000000in}}%
\pgfpathlineto{\pgfqpoint{0.000000in}{0.058926in}}%
\pgfpathlineto{\pgfqpoint{-0.035355in}{0.000000in}}%
\pgfpathclose%
\pgfusepath{stroke,fill}%
}%
\begin{pgfscope}%
\pgfsys@transformshift{2.991524in}{1.541866in}%
\pgfsys@useobject{currentmarker}{}%
\end{pgfscope}%
\end{pgfscope}%
\begin{pgfscope}%
\definecolor{textcolor}{rgb}{0.000000,0.000000,0.000000}%
\pgfsetstrokecolor{textcolor}%
\pgfsetfillcolor{textcolor}%
\pgftext[x=3.216524in,y=1.498116in,left,base]{\color{textcolor}\rmfamily\fontsize{9.000000}{10.800000}\selectfont \(\displaystyle \nu_{25} = \) 439.51}%
\end{pgfscope}%
\end{pgfpicture}%
\makeatother%
\endgroup%
}
					\caption{Cluster V}
					\label{SubFig:Cluster_VI_imag}
				\end{subfigure}
				\caption{RMSE Frecuencias en función de SNR para los diferentes conjuntos. Lineas punteadas corresponden a \cite{Andersson2014}; lineas solidas corresponden a \emph{Shift-and-Zoom}.}
				\label{Fig:RMSE_Cluster_nu}
			\end{figure}
			\newpage
			\begin{figure}[h!]
				%\centering
				\begin{subfigure}[h]{0.5\textwidth}
					\centering
					\resizebox{\linewidth}{!}{%% Creator: Matplotlib, PGF backend
%%
%% To include the figure in your LaTeX document, write
%%   \input{<filename>.pgf}
%%
%% Make sure the required packages are loaded in your preamble
%%   \usepackage{pgf}
%%
%% and, on pdftex
%%   \usepackage[utf8]{inputenc}\DeclareUnicodeCharacter{2212}{-}
%%
%% or, on luatex and xetex
%%   \usepackage{unicode-math}
%%
%% Figures using additional raster images can only be included by \input if
%% they are in the same directory as the main LaTeX file. For loading figures
%% from other directories you can use the `import` package
%%   \usepackage{import}
%%
%% and then include the figures with
%%   \import{<path to file>}{<filename>.pgf}
%%
%% Matplotlib used the following preamble
%%   \usepackage[utf8x]{inputenc}
%%   \usepackage[T1]{fontenc}
%%   \usepackage{amsmath,amssymb,amsfonts}
%%
\begingroup%
\makeatletter%
\begin{pgfpicture}%
\pgfpathrectangle{\pgfpointorigin}{\pgfqpoint{4.136389in}{2.495314in}}%
\pgfusepath{use as bounding box, clip}%
\begin{pgfscope}%
\pgfsetbuttcap%
\pgfsetmiterjoin%
\definecolor{currentfill}{rgb}{1.000000,1.000000,1.000000}%
\pgfsetfillcolor{currentfill}%
\pgfsetlinewidth{0.000000pt}%
\definecolor{currentstroke}{rgb}{1.000000,1.000000,1.000000}%
\pgfsetstrokecolor{currentstroke}%
\pgfsetdash{}{0pt}%
\pgfpathmoveto{\pgfqpoint{0.000000in}{0.000000in}}%
\pgfpathlineto{\pgfqpoint{4.136389in}{0.000000in}}%
\pgfpathlineto{\pgfqpoint{4.136389in}{2.495314in}}%
\pgfpathlineto{\pgfqpoint{0.000000in}{2.495314in}}%
\pgfpathclose%
\pgfusepath{fill}%
\end{pgfscope}%
\begin{pgfscope}%
\pgfsetbuttcap%
\pgfsetmiterjoin%
\definecolor{currentfill}{rgb}{1.000000,1.000000,1.000000}%
\pgfsetfillcolor{currentfill}%
\pgfsetlinewidth{0.000000pt}%
\definecolor{currentstroke}{rgb}{0.000000,0.000000,0.000000}%
\pgfsetstrokecolor{currentstroke}%
\pgfsetstrokeopacity{0.000000}%
\pgfsetdash{}{0pt}%
\pgfpathmoveto{\pgfqpoint{0.745371in}{0.566590in}}%
\pgfpathlineto{\pgfqpoint{4.036389in}{0.566590in}}%
\pgfpathlineto{\pgfqpoint{4.036389in}{2.395314in}}%
\pgfpathlineto{\pgfqpoint{0.745371in}{2.395314in}}%
\pgfpathclose%
\pgfusepath{fill}%
\end{pgfscope}%
\begin{pgfscope}%
\pgfpathrectangle{\pgfqpoint{0.745371in}{0.566590in}}{\pgfqpoint{3.291018in}{1.828724in}}%
\pgfusepath{clip}%
\pgfsetrectcap%
\pgfsetroundjoin%
\pgfsetlinewidth{0.803000pt}%
\definecolor{currentstroke}{rgb}{0.690196,0.690196,0.690196}%
\pgfsetstrokecolor{currentstroke}%
\pgfsetdash{}{0pt}%
\pgfpathmoveto{\pgfqpoint{0.745371in}{0.566590in}}%
\pgfpathlineto{\pgfqpoint{0.745371in}{2.395314in}}%
\pgfusepath{stroke}%
\end{pgfscope}%
\begin{pgfscope}%
\pgfsetbuttcap%
\pgfsetroundjoin%
\definecolor{currentfill}{rgb}{0.000000,0.000000,0.000000}%
\pgfsetfillcolor{currentfill}%
\pgfsetlinewidth{0.803000pt}%
\definecolor{currentstroke}{rgb}{0.000000,0.000000,0.000000}%
\pgfsetstrokecolor{currentstroke}%
\pgfsetdash{}{0pt}%
\pgfsys@defobject{currentmarker}{\pgfqpoint{0.000000in}{-0.048611in}}{\pgfqpoint{0.000000in}{0.000000in}}{%
\pgfpathmoveto{\pgfqpoint{0.000000in}{0.000000in}}%
\pgfpathlineto{\pgfqpoint{0.000000in}{-0.048611in}}%
\pgfusepath{stroke,fill}%
}%
\begin{pgfscope}%
\pgfsys@transformshift{0.745371in}{0.566590in}%
\pgfsys@useobject{currentmarker}{}%
\end{pgfscope}%
\end{pgfscope}%
\begin{pgfscope}%
\definecolor{textcolor}{rgb}{0.000000,0.000000,0.000000}%
\pgfsetstrokecolor{textcolor}%
\pgfsetfillcolor{textcolor}%
\pgftext[x=0.745371in,y=0.469368in,,top]{\color{textcolor}\rmfamily\fontsize{12.000000}{14.400000}\selectfont \(\displaystyle {-10}\)}%
\end{pgfscope}%
\begin{pgfscope}%
\pgfpathrectangle{\pgfqpoint{0.745371in}{0.566590in}}{\pgfqpoint{3.291018in}{1.828724in}}%
\pgfusepath{clip}%
\pgfsetrectcap%
\pgfsetroundjoin%
\pgfsetlinewidth{0.803000pt}%
\definecolor{currentstroke}{rgb}{0.690196,0.690196,0.690196}%
\pgfsetstrokecolor{currentstroke}%
\pgfsetdash{}{0pt}%
\pgfpathmoveto{\pgfqpoint{1.251681in}{0.566590in}}%
\pgfpathlineto{\pgfqpoint{1.251681in}{2.395314in}}%
\pgfusepath{stroke}%
\end{pgfscope}%
\begin{pgfscope}%
\pgfsetbuttcap%
\pgfsetroundjoin%
\definecolor{currentfill}{rgb}{0.000000,0.000000,0.000000}%
\pgfsetfillcolor{currentfill}%
\pgfsetlinewidth{0.803000pt}%
\definecolor{currentstroke}{rgb}{0.000000,0.000000,0.000000}%
\pgfsetstrokecolor{currentstroke}%
\pgfsetdash{}{0pt}%
\pgfsys@defobject{currentmarker}{\pgfqpoint{0.000000in}{-0.048611in}}{\pgfqpoint{0.000000in}{0.000000in}}{%
\pgfpathmoveto{\pgfqpoint{0.000000in}{0.000000in}}%
\pgfpathlineto{\pgfqpoint{0.000000in}{-0.048611in}}%
\pgfusepath{stroke,fill}%
}%
\begin{pgfscope}%
\pgfsys@transformshift{1.251681in}{0.566590in}%
\pgfsys@useobject{currentmarker}{}%
\end{pgfscope}%
\end{pgfscope}%
\begin{pgfscope}%
\definecolor{textcolor}{rgb}{0.000000,0.000000,0.000000}%
\pgfsetstrokecolor{textcolor}%
\pgfsetfillcolor{textcolor}%
\pgftext[x=1.251681in,y=0.469368in,,top]{\color{textcolor}\rmfamily\fontsize{12.000000}{14.400000}\selectfont \(\displaystyle {0}\)}%
\end{pgfscope}%
\begin{pgfscope}%
\pgfpathrectangle{\pgfqpoint{0.745371in}{0.566590in}}{\pgfqpoint{3.291018in}{1.828724in}}%
\pgfusepath{clip}%
\pgfsetrectcap%
\pgfsetroundjoin%
\pgfsetlinewidth{0.803000pt}%
\definecolor{currentstroke}{rgb}{0.690196,0.690196,0.690196}%
\pgfsetstrokecolor{currentstroke}%
\pgfsetdash{}{0pt}%
\pgfpathmoveto{\pgfqpoint{1.757992in}{0.566590in}}%
\pgfpathlineto{\pgfqpoint{1.757992in}{2.395314in}}%
\pgfusepath{stroke}%
\end{pgfscope}%
\begin{pgfscope}%
\pgfsetbuttcap%
\pgfsetroundjoin%
\definecolor{currentfill}{rgb}{0.000000,0.000000,0.000000}%
\pgfsetfillcolor{currentfill}%
\pgfsetlinewidth{0.803000pt}%
\definecolor{currentstroke}{rgb}{0.000000,0.000000,0.000000}%
\pgfsetstrokecolor{currentstroke}%
\pgfsetdash{}{0pt}%
\pgfsys@defobject{currentmarker}{\pgfqpoint{0.000000in}{-0.048611in}}{\pgfqpoint{0.000000in}{0.000000in}}{%
\pgfpathmoveto{\pgfqpoint{0.000000in}{0.000000in}}%
\pgfpathlineto{\pgfqpoint{0.000000in}{-0.048611in}}%
\pgfusepath{stroke,fill}%
}%
\begin{pgfscope}%
\pgfsys@transformshift{1.757992in}{0.566590in}%
\pgfsys@useobject{currentmarker}{}%
\end{pgfscope}%
\end{pgfscope}%
\begin{pgfscope}%
\definecolor{textcolor}{rgb}{0.000000,0.000000,0.000000}%
\pgfsetstrokecolor{textcolor}%
\pgfsetfillcolor{textcolor}%
\pgftext[x=1.757992in,y=0.469368in,,top]{\color{textcolor}\rmfamily\fontsize{12.000000}{14.400000}\selectfont \(\displaystyle {10}\)}%
\end{pgfscope}%
\begin{pgfscope}%
\pgfpathrectangle{\pgfqpoint{0.745371in}{0.566590in}}{\pgfqpoint{3.291018in}{1.828724in}}%
\pgfusepath{clip}%
\pgfsetrectcap%
\pgfsetroundjoin%
\pgfsetlinewidth{0.803000pt}%
\definecolor{currentstroke}{rgb}{0.690196,0.690196,0.690196}%
\pgfsetstrokecolor{currentstroke}%
\pgfsetdash{}{0pt}%
\pgfpathmoveto{\pgfqpoint{2.264302in}{0.566590in}}%
\pgfpathlineto{\pgfqpoint{2.264302in}{2.395314in}}%
\pgfusepath{stroke}%
\end{pgfscope}%
\begin{pgfscope}%
\pgfsetbuttcap%
\pgfsetroundjoin%
\definecolor{currentfill}{rgb}{0.000000,0.000000,0.000000}%
\pgfsetfillcolor{currentfill}%
\pgfsetlinewidth{0.803000pt}%
\definecolor{currentstroke}{rgb}{0.000000,0.000000,0.000000}%
\pgfsetstrokecolor{currentstroke}%
\pgfsetdash{}{0pt}%
\pgfsys@defobject{currentmarker}{\pgfqpoint{0.000000in}{-0.048611in}}{\pgfqpoint{0.000000in}{0.000000in}}{%
\pgfpathmoveto{\pgfqpoint{0.000000in}{0.000000in}}%
\pgfpathlineto{\pgfqpoint{0.000000in}{-0.048611in}}%
\pgfusepath{stroke,fill}%
}%
\begin{pgfscope}%
\pgfsys@transformshift{2.264302in}{0.566590in}%
\pgfsys@useobject{currentmarker}{}%
\end{pgfscope}%
\end{pgfscope}%
\begin{pgfscope}%
\definecolor{textcolor}{rgb}{0.000000,0.000000,0.000000}%
\pgfsetstrokecolor{textcolor}%
\pgfsetfillcolor{textcolor}%
\pgftext[x=2.264302in,y=0.469368in,,top]{\color{textcolor}\rmfamily\fontsize{12.000000}{14.400000}\selectfont \(\displaystyle {20}\)}%
\end{pgfscope}%
\begin{pgfscope}%
\pgfpathrectangle{\pgfqpoint{0.745371in}{0.566590in}}{\pgfqpoint{3.291018in}{1.828724in}}%
\pgfusepath{clip}%
\pgfsetrectcap%
\pgfsetroundjoin%
\pgfsetlinewidth{0.803000pt}%
\definecolor{currentstroke}{rgb}{0.690196,0.690196,0.690196}%
\pgfsetstrokecolor{currentstroke}%
\pgfsetdash{}{0pt}%
\pgfpathmoveto{\pgfqpoint{2.770613in}{0.566590in}}%
\pgfpathlineto{\pgfqpoint{2.770613in}{2.395314in}}%
\pgfusepath{stroke}%
\end{pgfscope}%
\begin{pgfscope}%
\pgfsetbuttcap%
\pgfsetroundjoin%
\definecolor{currentfill}{rgb}{0.000000,0.000000,0.000000}%
\pgfsetfillcolor{currentfill}%
\pgfsetlinewidth{0.803000pt}%
\definecolor{currentstroke}{rgb}{0.000000,0.000000,0.000000}%
\pgfsetstrokecolor{currentstroke}%
\pgfsetdash{}{0pt}%
\pgfsys@defobject{currentmarker}{\pgfqpoint{0.000000in}{-0.048611in}}{\pgfqpoint{0.000000in}{0.000000in}}{%
\pgfpathmoveto{\pgfqpoint{0.000000in}{0.000000in}}%
\pgfpathlineto{\pgfqpoint{0.000000in}{-0.048611in}}%
\pgfusepath{stroke,fill}%
}%
\begin{pgfscope}%
\pgfsys@transformshift{2.770613in}{0.566590in}%
\pgfsys@useobject{currentmarker}{}%
\end{pgfscope}%
\end{pgfscope}%
\begin{pgfscope}%
\definecolor{textcolor}{rgb}{0.000000,0.000000,0.000000}%
\pgfsetstrokecolor{textcolor}%
\pgfsetfillcolor{textcolor}%
\pgftext[x=2.770613in,y=0.469368in,,top]{\color{textcolor}\rmfamily\fontsize{12.000000}{14.400000}\selectfont \(\displaystyle {30}\)}%
\end{pgfscope}%
\begin{pgfscope}%
\pgfpathrectangle{\pgfqpoint{0.745371in}{0.566590in}}{\pgfqpoint{3.291018in}{1.828724in}}%
\pgfusepath{clip}%
\pgfsetrectcap%
\pgfsetroundjoin%
\pgfsetlinewidth{0.803000pt}%
\definecolor{currentstroke}{rgb}{0.690196,0.690196,0.690196}%
\pgfsetstrokecolor{currentstroke}%
\pgfsetdash{}{0pt}%
\pgfpathmoveto{\pgfqpoint{3.276923in}{0.566590in}}%
\pgfpathlineto{\pgfqpoint{3.276923in}{2.395314in}}%
\pgfusepath{stroke}%
\end{pgfscope}%
\begin{pgfscope}%
\pgfsetbuttcap%
\pgfsetroundjoin%
\definecolor{currentfill}{rgb}{0.000000,0.000000,0.000000}%
\pgfsetfillcolor{currentfill}%
\pgfsetlinewidth{0.803000pt}%
\definecolor{currentstroke}{rgb}{0.000000,0.000000,0.000000}%
\pgfsetstrokecolor{currentstroke}%
\pgfsetdash{}{0pt}%
\pgfsys@defobject{currentmarker}{\pgfqpoint{0.000000in}{-0.048611in}}{\pgfqpoint{0.000000in}{0.000000in}}{%
\pgfpathmoveto{\pgfqpoint{0.000000in}{0.000000in}}%
\pgfpathlineto{\pgfqpoint{0.000000in}{-0.048611in}}%
\pgfusepath{stroke,fill}%
}%
\begin{pgfscope}%
\pgfsys@transformshift{3.276923in}{0.566590in}%
\pgfsys@useobject{currentmarker}{}%
\end{pgfscope}%
\end{pgfscope}%
\begin{pgfscope}%
\definecolor{textcolor}{rgb}{0.000000,0.000000,0.000000}%
\pgfsetstrokecolor{textcolor}%
\pgfsetfillcolor{textcolor}%
\pgftext[x=3.276923in,y=0.469368in,,top]{\color{textcolor}\rmfamily\fontsize{12.000000}{14.400000}\selectfont \(\displaystyle {40}\)}%
\end{pgfscope}%
\begin{pgfscope}%
\pgfpathrectangle{\pgfqpoint{0.745371in}{0.566590in}}{\pgfqpoint{3.291018in}{1.828724in}}%
\pgfusepath{clip}%
\pgfsetrectcap%
\pgfsetroundjoin%
\pgfsetlinewidth{0.803000pt}%
\definecolor{currentstroke}{rgb}{0.690196,0.690196,0.690196}%
\pgfsetstrokecolor{currentstroke}%
\pgfsetdash{}{0pt}%
\pgfpathmoveto{\pgfqpoint{3.783233in}{0.566590in}}%
\pgfpathlineto{\pgfqpoint{3.783233in}{2.395314in}}%
\pgfusepath{stroke}%
\end{pgfscope}%
\begin{pgfscope}%
\pgfsetbuttcap%
\pgfsetroundjoin%
\definecolor{currentfill}{rgb}{0.000000,0.000000,0.000000}%
\pgfsetfillcolor{currentfill}%
\pgfsetlinewidth{0.803000pt}%
\definecolor{currentstroke}{rgb}{0.000000,0.000000,0.000000}%
\pgfsetstrokecolor{currentstroke}%
\pgfsetdash{}{0pt}%
\pgfsys@defobject{currentmarker}{\pgfqpoint{0.000000in}{-0.048611in}}{\pgfqpoint{0.000000in}{0.000000in}}{%
\pgfpathmoveto{\pgfqpoint{0.000000in}{0.000000in}}%
\pgfpathlineto{\pgfqpoint{0.000000in}{-0.048611in}}%
\pgfusepath{stroke,fill}%
}%
\begin{pgfscope}%
\pgfsys@transformshift{3.783233in}{0.566590in}%
\pgfsys@useobject{currentmarker}{}%
\end{pgfscope}%
\end{pgfscope}%
\begin{pgfscope}%
\definecolor{textcolor}{rgb}{0.000000,0.000000,0.000000}%
\pgfsetstrokecolor{textcolor}%
\pgfsetfillcolor{textcolor}%
\pgftext[x=3.783233in,y=0.469368in,,top]{\color{textcolor}\rmfamily\fontsize{12.000000}{14.400000}\selectfont \(\displaystyle {50}\)}%
\end{pgfscope}%
\begin{pgfscope}%
\definecolor{textcolor}{rgb}{0.000000,0.000000,0.000000}%
\pgfsetstrokecolor{textcolor}%
\pgfsetfillcolor{textcolor}%
\pgftext[x=2.390880in,y=0.266626in,,top]{\color{textcolor}\rmfamily\fontsize{12.000000}{14.400000}\selectfont SNR [dB]}%
\end{pgfscope}%
\begin{pgfscope}%
\pgfpathrectangle{\pgfqpoint{0.745371in}{0.566590in}}{\pgfqpoint{3.291018in}{1.828724in}}%
\pgfusepath{clip}%
\pgfsetrectcap%
\pgfsetroundjoin%
\pgfsetlinewidth{0.803000pt}%
\definecolor{currentstroke}{rgb}{0.690196,0.690196,0.690196}%
\pgfsetstrokecolor{currentstroke}%
\pgfsetdash{}{0pt}%
\pgfpathmoveto{\pgfqpoint{0.745371in}{0.752967in}}%
\pgfpathlineto{\pgfqpoint{4.036389in}{0.752967in}}%
\pgfusepath{stroke}%
\end{pgfscope}%
\begin{pgfscope}%
\pgfsetbuttcap%
\pgfsetroundjoin%
\definecolor{currentfill}{rgb}{0.000000,0.000000,0.000000}%
\pgfsetfillcolor{currentfill}%
\pgfsetlinewidth{0.803000pt}%
\definecolor{currentstroke}{rgb}{0.000000,0.000000,0.000000}%
\pgfsetstrokecolor{currentstroke}%
\pgfsetdash{}{0pt}%
\pgfsys@defobject{currentmarker}{\pgfqpoint{-0.048611in}{0.000000in}}{\pgfqpoint{-0.000000in}{0.000000in}}{%
\pgfpathmoveto{\pgfqpoint{-0.000000in}{0.000000in}}%
\pgfpathlineto{\pgfqpoint{-0.048611in}{0.000000in}}%
\pgfusepath{stroke,fill}%
}%
\begin{pgfscope}%
\pgfsys@transformshift{0.745371in}{0.752967in}%
\pgfsys@useobject{currentmarker}{}%
\end{pgfscope}%
\end{pgfscope}%
\begin{pgfscope}%
\definecolor{textcolor}{rgb}{0.000000,0.000000,0.000000}%
\pgfsetstrokecolor{textcolor}%
\pgfsetfillcolor{textcolor}%
\pgftext[x=0.327160in, y=0.695574in, left, base]{\color{textcolor}\rmfamily\fontsize{12.000000}{14.400000}\selectfont \(\displaystyle {10^{-4}}\)}%
\end{pgfscope}%
\begin{pgfscope}%
\pgfpathrectangle{\pgfqpoint{0.745371in}{0.566590in}}{\pgfqpoint{3.291018in}{1.828724in}}%
\pgfusepath{clip}%
\pgfsetrectcap%
\pgfsetroundjoin%
\pgfsetlinewidth{0.803000pt}%
\definecolor{currentstroke}{rgb}{0.690196,0.690196,0.690196}%
\pgfsetstrokecolor{currentstroke}%
\pgfsetdash{}{0pt}%
\pgfpathmoveto{\pgfqpoint{0.745371in}{1.236303in}}%
\pgfpathlineto{\pgfqpoint{4.036389in}{1.236303in}}%
\pgfusepath{stroke}%
\end{pgfscope}%
\begin{pgfscope}%
\pgfsetbuttcap%
\pgfsetroundjoin%
\definecolor{currentfill}{rgb}{0.000000,0.000000,0.000000}%
\pgfsetfillcolor{currentfill}%
\pgfsetlinewidth{0.803000pt}%
\definecolor{currentstroke}{rgb}{0.000000,0.000000,0.000000}%
\pgfsetstrokecolor{currentstroke}%
\pgfsetdash{}{0pt}%
\pgfsys@defobject{currentmarker}{\pgfqpoint{-0.048611in}{0.000000in}}{\pgfqpoint{-0.000000in}{0.000000in}}{%
\pgfpathmoveto{\pgfqpoint{-0.000000in}{0.000000in}}%
\pgfpathlineto{\pgfqpoint{-0.048611in}{0.000000in}}%
\pgfusepath{stroke,fill}%
}%
\begin{pgfscope}%
\pgfsys@transformshift{0.745371in}{1.236303in}%
\pgfsys@useobject{currentmarker}{}%
\end{pgfscope}%
\end{pgfscope}%
\begin{pgfscope}%
\definecolor{textcolor}{rgb}{0.000000,0.000000,0.000000}%
\pgfsetstrokecolor{textcolor}%
\pgfsetfillcolor{textcolor}%
\pgftext[x=0.327160in, y=1.178910in, left, base]{\color{textcolor}\rmfamily\fontsize{12.000000}{14.400000}\selectfont \(\displaystyle {10^{-2}}\)}%
\end{pgfscope}%
\begin{pgfscope}%
\pgfpathrectangle{\pgfqpoint{0.745371in}{0.566590in}}{\pgfqpoint{3.291018in}{1.828724in}}%
\pgfusepath{clip}%
\pgfsetrectcap%
\pgfsetroundjoin%
\pgfsetlinewidth{0.803000pt}%
\definecolor{currentstroke}{rgb}{0.690196,0.690196,0.690196}%
\pgfsetstrokecolor{currentstroke}%
\pgfsetdash{}{0pt}%
\pgfpathmoveto{\pgfqpoint{0.745371in}{1.719639in}}%
\pgfpathlineto{\pgfqpoint{4.036389in}{1.719639in}}%
\pgfusepath{stroke}%
\end{pgfscope}%
\begin{pgfscope}%
\pgfsetbuttcap%
\pgfsetroundjoin%
\definecolor{currentfill}{rgb}{0.000000,0.000000,0.000000}%
\pgfsetfillcolor{currentfill}%
\pgfsetlinewidth{0.803000pt}%
\definecolor{currentstroke}{rgb}{0.000000,0.000000,0.000000}%
\pgfsetstrokecolor{currentstroke}%
\pgfsetdash{}{0pt}%
\pgfsys@defobject{currentmarker}{\pgfqpoint{-0.048611in}{0.000000in}}{\pgfqpoint{-0.000000in}{0.000000in}}{%
\pgfpathmoveto{\pgfqpoint{-0.000000in}{0.000000in}}%
\pgfpathlineto{\pgfqpoint{-0.048611in}{0.000000in}}%
\pgfusepath{stroke,fill}%
}%
\begin{pgfscope}%
\pgfsys@transformshift{0.745371in}{1.719639in}%
\pgfsys@useobject{currentmarker}{}%
\end{pgfscope}%
\end{pgfscope}%
\begin{pgfscope}%
\definecolor{textcolor}{rgb}{0.000000,0.000000,0.000000}%
\pgfsetstrokecolor{textcolor}%
\pgfsetfillcolor{textcolor}%
\pgftext[x=0.418983in, y=1.662246in, left, base]{\color{textcolor}\rmfamily\fontsize{12.000000}{14.400000}\selectfont \(\displaystyle {10^{0}}\)}%
\end{pgfscope}%
\begin{pgfscope}%
\pgfpathrectangle{\pgfqpoint{0.745371in}{0.566590in}}{\pgfqpoint{3.291018in}{1.828724in}}%
\pgfusepath{clip}%
\pgfsetrectcap%
\pgfsetroundjoin%
\pgfsetlinewidth{0.803000pt}%
\definecolor{currentstroke}{rgb}{0.690196,0.690196,0.690196}%
\pgfsetstrokecolor{currentstroke}%
\pgfsetdash{}{0pt}%
\pgfpathmoveto{\pgfqpoint{0.745371in}{2.202975in}}%
\pgfpathlineto{\pgfqpoint{4.036389in}{2.202975in}}%
\pgfusepath{stroke}%
\end{pgfscope}%
\begin{pgfscope}%
\pgfsetbuttcap%
\pgfsetroundjoin%
\definecolor{currentfill}{rgb}{0.000000,0.000000,0.000000}%
\pgfsetfillcolor{currentfill}%
\pgfsetlinewidth{0.803000pt}%
\definecolor{currentstroke}{rgb}{0.000000,0.000000,0.000000}%
\pgfsetstrokecolor{currentstroke}%
\pgfsetdash{}{0pt}%
\pgfsys@defobject{currentmarker}{\pgfqpoint{-0.048611in}{0.000000in}}{\pgfqpoint{-0.000000in}{0.000000in}}{%
\pgfpathmoveto{\pgfqpoint{-0.000000in}{0.000000in}}%
\pgfpathlineto{\pgfqpoint{-0.048611in}{0.000000in}}%
\pgfusepath{stroke,fill}%
}%
\begin{pgfscope}%
\pgfsys@transformshift{0.745371in}{2.202975in}%
\pgfsys@useobject{currentmarker}{}%
\end{pgfscope}%
\end{pgfscope}%
\begin{pgfscope}%
\definecolor{textcolor}{rgb}{0.000000,0.000000,0.000000}%
\pgfsetstrokecolor{textcolor}%
\pgfsetfillcolor{textcolor}%
\pgftext[x=0.418983in, y=2.145582in, left, base]{\color{textcolor}\rmfamily\fontsize{12.000000}{14.400000}\selectfont \(\displaystyle {10^{2}}\)}%
\end{pgfscope}%
\begin{pgfscope}%
\definecolor{textcolor}{rgb}{0.000000,0.000000,0.000000}%
\pgfsetstrokecolor{textcolor}%
\pgfsetfillcolor{textcolor}%
\pgftext[x=0.271605in,y=1.480952in,,bottom,rotate=90.000000]{\color{textcolor}\rmfamily\fontsize{12.000000}{14.400000}\selectfont \(\displaystyle \hat{\sigma}_{\gamma}(\mathrm{SNR})\)}%
\end{pgfscope}%
\begin{pgfscope}%
\pgfpathrectangle{\pgfqpoint{0.745371in}{0.566590in}}{\pgfqpoint{3.291018in}{1.828724in}}%
\pgfusepath{clip}%
\pgfsetbuttcap%
\pgfsetroundjoin%
\pgfsetlinewidth{1.505625pt}%
\definecolor{currentstroke}{rgb}{0.000000,0.447000,0.741000}%
\pgfsetstrokecolor{currentstroke}%
\pgfsetdash{{5.550000pt}{2.400000pt}}{0.000000pt}%
\pgfpathmoveto{\pgfqpoint{0.745371in}{2.209703in}}%
\pgfpathlineto{\pgfqpoint{0.842165in}{2.203163in}}%
\pgfpathlineto{\pgfqpoint{0.938960in}{2.196524in}}%
\pgfpathlineto{\pgfqpoint{1.035755in}{2.197644in}}%
\pgfpathlineto{\pgfqpoint{1.132549in}{2.145674in}}%
\pgfpathlineto{\pgfqpoint{1.229344in}{2.208284in}}%
\pgfpathlineto{\pgfqpoint{1.326139in}{2.141151in}}%
\pgfpathlineto{\pgfqpoint{1.422933in}{2.257209in}}%
\pgfpathlineto{\pgfqpoint{1.519728in}{2.116902in}}%
\pgfpathlineto{\pgfqpoint{1.616523in}{2.183150in}}%
\pgfpathlineto{\pgfqpoint{1.713317in}{2.185636in}}%
\pgfpathlineto{\pgfqpoint{1.810112in}{2.152132in}}%
\pgfpathlineto{\pgfqpoint{1.906906in}{2.170100in}}%
\pgfpathlineto{\pgfqpoint{2.003701in}{2.146883in}}%
\pgfpathlineto{\pgfqpoint{2.100496in}{2.187561in}}%
\pgfpathlineto{\pgfqpoint{2.197290in}{2.086885in}}%
\pgfpathlineto{\pgfqpoint{2.294085in}{2.103266in}}%
\pgfpathlineto{\pgfqpoint{2.390880in}{2.205393in}}%
\pgfpathlineto{\pgfqpoint{2.487674in}{1.825172in}}%
\pgfpathlineto{\pgfqpoint{2.584469in}{1.975549in}}%
\pgfpathlineto{\pgfqpoint{2.681264in}{1.469466in}}%
\pgfpathlineto{\pgfqpoint{2.778058in}{1.435106in}}%
\pgfpathlineto{\pgfqpoint{2.874853in}{1.413544in}}%
\pgfpathlineto{\pgfqpoint{2.971648in}{1.391267in}}%
\pgfpathlineto{\pgfqpoint{3.068442in}{1.382233in}}%
\pgfpathlineto{\pgfqpoint{3.165237in}{1.337189in}}%
\pgfpathlineto{\pgfqpoint{3.262031in}{1.321071in}}%
\pgfpathlineto{\pgfqpoint{3.358826in}{1.305499in}}%
\pgfpathlineto{\pgfqpoint{3.455621in}{1.272341in}}%
\pgfpathlineto{\pgfqpoint{3.552415in}{1.248021in}}%
\pgfpathlineto{\pgfqpoint{3.649210in}{1.234418in}}%
\pgfpathlineto{\pgfqpoint{3.746005in}{1.213667in}}%
\pgfpathlineto{\pgfqpoint{3.842799in}{1.193999in}}%
\pgfpathlineto{\pgfqpoint{3.939594in}{1.154685in}}%
\pgfpathlineto{\pgfqpoint{4.036389in}{1.140895in}}%
\pgfusepath{stroke}%
\end{pgfscope}%
\begin{pgfscope}%
\pgfpathrectangle{\pgfqpoint{0.745371in}{0.566590in}}{\pgfqpoint{3.291018in}{1.828724in}}%
\pgfusepath{clip}%
\pgfsetbuttcap%
\pgfsetroundjoin%
\definecolor{currentfill}{rgb}{0.000000,0.000000,0.000000}%
\pgfsetfillcolor{currentfill}%
\pgfsetfillopacity{0.000000}%
\pgfsetlinewidth{1.003750pt}%
\definecolor{currentstroke}{rgb}{0.000000,0.447000,0.741000}%
\pgfsetstrokecolor{currentstroke}%
\pgfsetdash{}{0pt}%
\pgfsys@defobject{currentmarker}{\pgfqpoint{-0.041667in}{-0.041667in}}{\pgfqpoint{0.041667in}{0.041667in}}{%
\pgfpathmoveto{\pgfqpoint{0.000000in}{-0.041667in}}%
\pgfpathcurveto{\pgfqpoint{0.011050in}{-0.041667in}}{\pgfqpoint{0.021649in}{-0.037276in}}{\pgfqpoint{0.029463in}{-0.029463in}}%
\pgfpathcurveto{\pgfqpoint{0.037276in}{-0.021649in}}{\pgfqpoint{0.041667in}{-0.011050in}}{\pgfqpoint{0.041667in}{0.000000in}}%
\pgfpathcurveto{\pgfqpoint{0.041667in}{0.011050in}}{\pgfqpoint{0.037276in}{0.021649in}}{\pgfqpoint{0.029463in}{0.029463in}}%
\pgfpathcurveto{\pgfqpoint{0.021649in}{0.037276in}}{\pgfqpoint{0.011050in}{0.041667in}}{\pgfqpoint{0.000000in}{0.041667in}}%
\pgfpathcurveto{\pgfqpoint{-0.011050in}{0.041667in}}{\pgfqpoint{-0.021649in}{0.037276in}}{\pgfqpoint{-0.029463in}{0.029463in}}%
\pgfpathcurveto{\pgfqpoint{-0.037276in}{0.021649in}}{\pgfqpoint{-0.041667in}{0.011050in}}{\pgfqpoint{-0.041667in}{0.000000in}}%
\pgfpathcurveto{\pgfqpoint{-0.041667in}{-0.011050in}}{\pgfqpoint{-0.037276in}{-0.021649in}}{\pgfqpoint{-0.029463in}{-0.029463in}}%
\pgfpathcurveto{\pgfqpoint{-0.021649in}{-0.037276in}}{\pgfqpoint{-0.011050in}{-0.041667in}}{\pgfqpoint{0.000000in}{-0.041667in}}%
\pgfpathclose%
\pgfusepath{stroke,fill}%
}%
\begin{pgfscope}%
\pgfsys@transformshift{0.745371in}{2.209703in}%
\pgfsys@useobject{currentmarker}{}%
\end{pgfscope}%
\begin{pgfscope}%
\pgfsys@transformshift{1.132549in}{2.145674in}%
\pgfsys@useobject{currentmarker}{}%
\end{pgfscope}%
\begin{pgfscope}%
\pgfsys@transformshift{1.519728in}{2.116902in}%
\pgfsys@useobject{currentmarker}{}%
\end{pgfscope}%
\begin{pgfscope}%
\pgfsys@transformshift{1.906906in}{2.170100in}%
\pgfsys@useobject{currentmarker}{}%
\end{pgfscope}%
\begin{pgfscope}%
\pgfsys@transformshift{2.294085in}{2.103266in}%
\pgfsys@useobject{currentmarker}{}%
\end{pgfscope}%
\begin{pgfscope}%
\pgfsys@transformshift{2.681264in}{1.469466in}%
\pgfsys@useobject{currentmarker}{}%
\end{pgfscope}%
\begin{pgfscope}%
\pgfsys@transformshift{3.068442in}{1.382233in}%
\pgfsys@useobject{currentmarker}{}%
\end{pgfscope}%
\begin{pgfscope}%
\pgfsys@transformshift{3.455621in}{1.272341in}%
\pgfsys@useobject{currentmarker}{}%
\end{pgfscope}%
\begin{pgfscope}%
\pgfsys@transformshift{3.842799in}{1.193999in}%
\pgfsys@useobject{currentmarker}{}%
\end{pgfscope}%
\end{pgfscope}%
\begin{pgfscope}%
\pgfpathrectangle{\pgfqpoint{0.745371in}{0.566590in}}{\pgfqpoint{3.291018in}{1.828724in}}%
\pgfusepath{clip}%
\pgfsetbuttcap%
\pgfsetroundjoin%
\pgfsetlinewidth{1.505625pt}%
\definecolor{currentstroke}{rgb}{0.850000,0.324000,0.098000}%
\pgfsetstrokecolor{currentstroke}%
\pgfsetdash{{5.550000pt}{2.400000pt}}{0.000000pt}%
\pgfpathmoveto{\pgfqpoint{0.745371in}{2.082874in}}%
\pgfpathlineto{\pgfqpoint{0.842165in}{2.175518in}}%
\pgfpathlineto{\pgfqpoint{0.938960in}{2.123930in}}%
\pgfpathlineto{\pgfqpoint{1.035755in}{2.132509in}}%
\pgfpathlineto{\pgfqpoint{1.132549in}{2.154191in}}%
\pgfpathlineto{\pgfqpoint{1.229344in}{2.085658in}}%
\pgfpathlineto{\pgfqpoint{1.326139in}{2.074413in}}%
\pgfpathlineto{\pgfqpoint{1.422933in}{2.080906in}}%
\pgfpathlineto{\pgfqpoint{1.519728in}{2.204557in}}%
\pgfpathlineto{\pgfqpoint{1.616523in}{2.021705in}}%
\pgfpathlineto{\pgfqpoint{1.713317in}{2.196661in}}%
\pgfpathlineto{\pgfqpoint{1.810112in}{2.208562in}}%
\pgfpathlineto{\pgfqpoint{1.906906in}{2.202708in}}%
\pgfpathlineto{\pgfqpoint{2.003701in}{2.097742in}}%
\pgfpathlineto{\pgfqpoint{2.100496in}{1.575735in}}%
\pgfpathlineto{\pgfqpoint{2.197290in}{1.509087in}}%
\pgfpathlineto{\pgfqpoint{2.294085in}{1.494628in}}%
\pgfpathlineto{\pgfqpoint{2.390880in}{1.371008in}}%
\pgfpathlineto{\pgfqpoint{2.487674in}{1.336477in}}%
\pgfpathlineto{\pgfqpoint{2.584469in}{1.322031in}}%
\pgfpathlineto{\pgfqpoint{2.681264in}{1.244955in}}%
\pgfpathlineto{\pgfqpoint{2.778058in}{1.236514in}}%
\pgfpathlineto{\pgfqpoint{2.874853in}{1.218775in}}%
\pgfpathlineto{\pgfqpoint{2.971648in}{1.198042in}}%
\pgfpathlineto{\pgfqpoint{3.068442in}{1.160975in}}%
\pgfpathlineto{\pgfqpoint{3.165237in}{1.142736in}}%
\pgfpathlineto{\pgfqpoint{3.262031in}{1.116686in}}%
\pgfpathlineto{\pgfqpoint{3.358826in}{1.095400in}}%
\pgfpathlineto{\pgfqpoint{3.455621in}{1.068647in}}%
\pgfpathlineto{\pgfqpoint{3.552415in}{1.058156in}}%
\pgfpathlineto{\pgfqpoint{3.649210in}{1.028851in}}%
\pgfpathlineto{\pgfqpoint{3.746005in}{1.007917in}}%
\pgfpathlineto{\pgfqpoint{3.842799in}{0.975865in}}%
\pgfpathlineto{\pgfqpoint{3.939594in}{0.966629in}}%
\pgfpathlineto{\pgfqpoint{4.036389in}{0.932288in}}%
\pgfusepath{stroke}%
\end{pgfscope}%
\begin{pgfscope}%
\pgfpathrectangle{\pgfqpoint{0.745371in}{0.566590in}}{\pgfqpoint{3.291018in}{1.828724in}}%
\pgfusepath{clip}%
\pgfsetbuttcap%
\pgfsetroundjoin%
\definecolor{currentfill}{rgb}{0.850000,0.324000,0.098000}%
\pgfsetfillcolor{currentfill}%
\pgfsetlinewidth{1.003750pt}%
\definecolor{currentstroke}{rgb}{0.850000,0.324000,0.098000}%
\pgfsetstrokecolor{currentstroke}%
\pgfsetdash{}{0pt}%
\pgfsys@defobject{currentmarker}{\pgfqpoint{-0.041667in}{-0.041667in}}{\pgfqpoint{0.041667in}{0.041667in}}{%
\pgfpathmoveto{\pgfqpoint{-0.041667in}{0.000000in}}%
\pgfpathlineto{\pgfqpoint{0.041667in}{0.000000in}}%
\pgfpathmoveto{\pgfqpoint{0.000000in}{-0.041667in}}%
\pgfpathlineto{\pgfqpoint{0.000000in}{0.041667in}}%
\pgfusepath{stroke,fill}%
}%
\begin{pgfscope}%
\pgfsys@transformshift{0.745371in}{2.082874in}%
\pgfsys@useobject{currentmarker}{}%
\end{pgfscope}%
\begin{pgfscope}%
\pgfsys@transformshift{1.035755in}{2.132509in}%
\pgfsys@useobject{currentmarker}{}%
\end{pgfscope}%
\begin{pgfscope}%
\pgfsys@transformshift{1.326139in}{2.074413in}%
\pgfsys@useobject{currentmarker}{}%
\end{pgfscope}%
\begin{pgfscope}%
\pgfsys@transformshift{1.616523in}{2.021705in}%
\pgfsys@useobject{currentmarker}{}%
\end{pgfscope}%
\begin{pgfscope}%
\pgfsys@transformshift{1.906906in}{2.202708in}%
\pgfsys@useobject{currentmarker}{}%
\end{pgfscope}%
\begin{pgfscope}%
\pgfsys@transformshift{2.197290in}{1.509087in}%
\pgfsys@useobject{currentmarker}{}%
\end{pgfscope}%
\begin{pgfscope}%
\pgfsys@transformshift{2.487674in}{1.336477in}%
\pgfsys@useobject{currentmarker}{}%
\end{pgfscope}%
\begin{pgfscope}%
\pgfsys@transformshift{2.778058in}{1.236514in}%
\pgfsys@useobject{currentmarker}{}%
\end{pgfscope}%
\begin{pgfscope}%
\pgfsys@transformshift{3.068442in}{1.160975in}%
\pgfsys@useobject{currentmarker}{}%
\end{pgfscope}%
\begin{pgfscope}%
\pgfsys@transformshift{3.358826in}{1.095400in}%
\pgfsys@useobject{currentmarker}{}%
\end{pgfscope}%
\begin{pgfscope}%
\pgfsys@transformshift{3.649210in}{1.028851in}%
\pgfsys@useobject{currentmarker}{}%
\end{pgfscope}%
\begin{pgfscope}%
\pgfsys@transformshift{3.939594in}{0.966629in}%
\pgfsys@useobject{currentmarker}{}%
\end{pgfscope}%
\end{pgfscope}%
\begin{pgfscope}%
\pgfpathrectangle{\pgfqpoint{0.745371in}{0.566590in}}{\pgfqpoint{3.291018in}{1.828724in}}%
\pgfusepath{clip}%
\pgfsetbuttcap%
\pgfsetroundjoin%
\pgfsetlinewidth{1.505625pt}%
\definecolor{currentstroke}{rgb}{0.000000,0.500000,0.000000}%
\pgfsetstrokecolor{currentstroke}%
\pgfsetdash{{5.550000pt}{2.400000pt}}{0.000000pt}%
\pgfpathmoveto{\pgfqpoint{0.745371in}{2.060492in}}%
\pgfpathlineto{\pgfqpoint{0.842165in}{2.170544in}}%
\pgfpathlineto{\pgfqpoint{0.938960in}{2.112675in}}%
\pgfpathlineto{\pgfqpoint{1.035755in}{2.119974in}}%
\pgfpathlineto{\pgfqpoint{1.132549in}{2.156712in}}%
\pgfpathlineto{\pgfqpoint{1.229344in}{2.133708in}}%
\pgfpathlineto{\pgfqpoint{1.326139in}{2.104013in}}%
\pgfpathlineto{\pgfqpoint{1.422933in}{2.072309in}}%
\pgfpathlineto{\pgfqpoint{1.519728in}{2.085761in}}%
\pgfpathlineto{\pgfqpoint{1.616523in}{2.024156in}}%
\pgfpathlineto{\pgfqpoint{1.713317in}{2.011635in}}%
\pgfpathlineto{\pgfqpoint{1.810112in}{1.979666in}}%
\pgfpathlineto{\pgfqpoint{1.906906in}{1.952445in}}%
\pgfpathlineto{\pgfqpoint{2.003701in}{2.134746in}}%
\pgfpathlineto{\pgfqpoint{2.100496in}{1.693791in}}%
\pgfpathlineto{\pgfqpoint{2.197290in}{1.889581in}}%
\pgfpathlineto{\pgfqpoint{2.294085in}{1.636919in}}%
\pgfpathlineto{\pgfqpoint{2.390880in}{1.623615in}}%
\pgfpathlineto{\pgfqpoint{2.487674in}{1.600384in}}%
\pgfpathlineto{\pgfqpoint{2.584469in}{1.577112in}}%
\pgfpathlineto{\pgfqpoint{2.681264in}{1.558493in}}%
\pgfpathlineto{\pgfqpoint{2.778058in}{1.528455in}}%
\pgfpathlineto{\pgfqpoint{2.874853in}{1.516612in}}%
\pgfpathlineto{\pgfqpoint{2.971648in}{1.500524in}}%
\pgfpathlineto{\pgfqpoint{3.068442in}{1.464012in}}%
\pgfpathlineto{\pgfqpoint{3.165237in}{1.446430in}}%
\pgfpathlineto{\pgfqpoint{3.262031in}{1.419564in}}%
\pgfpathlineto{\pgfqpoint{3.358826in}{1.390278in}}%
\pgfpathlineto{\pgfqpoint{3.455621in}{1.377649in}}%
\pgfpathlineto{\pgfqpoint{3.552415in}{1.349980in}}%
\pgfpathlineto{\pgfqpoint{3.649210in}{1.328091in}}%
\pgfpathlineto{\pgfqpoint{3.746005in}{1.303231in}}%
\pgfpathlineto{\pgfqpoint{3.842799in}{1.285685in}}%
\pgfpathlineto{\pgfqpoint{3.939594in}{1.249539in}}%
\pgfpathlineto{\pgfqpoint{4.036389in}{1.231991in}}%
\pgfusepath{stroke}%
\end{pgfscope}%
\begin{pgfscope}%
\pgfpathrectangle{\pgfqpoint{0.745371in}{0.566590in}}{\pgfqpoint{3.291018in}{1.828724in}}%
\pgfusepath{clip}%
\pgfsetbuttcap%
\pgfsetmiterjoin%
\definecolor{currentfill}{rgb}{0.000000,0.000000,0.000000}%
\pgfsetfillcolor{currentfill}%
\pgfsetfillopacity{0.000000}%
\pgfsetlinewidth{1.003750pt}%
\definecolor{currentstroke}{rgb}{0.000000,0.500000,0.000000}%
\pgfsetstrokecolor{currentstroke}%
\pgfsetdash{}{0pt}%
\pgfsys@defobject{currentmarker}{\pgfqpoint{-0.041667in}{-0.041667in}}{\pgfqpoint{0.041667in}{0.041667in}}{%
\pgfpathmoveto{\pgfqpoint{-0.041667in}{-0.041667in}}%
\pgfpathlineto{\pgfqpoint{0.041667in}{-0.041667in}}%
\pgfpathlineto{\pgfqpoint{0.041667in}{0.041667in}}%
\pgfpathlineto{\pgfqpoint{-0.041667in}{0.041667in}}%
\pgfpathclose%
\pgfusepath{stroke,fill}%
}%
\begin{pgfscope}%
\pgfsys@transformshift{0.745371in}{2.060492in}%
\pgfsys@useobject{currentmarker}{}%
\end{pgfscope}%
\begin{pgfscope}%
\pgfsys@transformshift{1.229344in}{2.133708in}%
\pgfsys@useobject{currentmarker}{}%
\end{pgfscope}%
\begin{pgfscope}%
\pgfsys@transformshift{1.713317in}{2.011635in}%
\pgfsys@useobject{currentmarker}{}%
\end{pgfscope}%
\begin{pgfscope}%
\pgfsys@transformshift{2.197290in}{1.889581in}%
\pgfsys@useobject{currentmarker}{}%
\end{pgfscope}%
\begin{pgfscope}%
\pgfsys@transformshift{2.681264in}{1.558493in}%
\pgfsys@useobject{currentmarker}{}%
\end{pgfscope}%
\begin{pgfscope}%
\pgfsys@transformshift{3.165237in}{1.446430in}%
\pgfsys@useobject{currentmarker}{}%
\end{pgfscope}%
\begin{pgfscope}%
\pgfsys@transformshift{3.649210in}{1.328091in}%
\pgfsys@useobject{currentmarker}{}%
\end{pgfscope}%
\end{pgfscope}%
\begin{pgfscope}%
\pgfpathrectangle{\pgfqpoint{0.745371in}{0.566590in}}{\pgfqpoint{3.291018in}{1.828724in}}%
\pgfusepath{clip}%
\pgfsetbuttcap%
\pgfsetroundjoin%
\pgfsetlinewidth{1.505625pt}%
\definecolor{currentstroke}{rgb}{0.494000,0.184000,0.556000}%
\pgfsetstrokecolor{currentstroke}%
\pgfsetdash{{5.550000pt}{2.400000pt}}{0.000000pt}%
\pgfpathmoveto{\pgfqpoint{0.745371in}{2.206886in}}%
\pgfpathlineto{\pgfqpoint{0.842165in}{2.161779in}}%
\pgfpathlineto{\pgfqpoint{0.938960in}{2.033980in}}%
\pgfpathlineto{\pgfqpoint{1.035755in}{2.275880in}}%
\pgfpathlineto{\pgfqpoint{1.132549in}{2.279531in}}%
\pgfpathlineto{\pgfqpoint{1.229344in}{1.894953in}}%
\pgfpathlineto{\pgfqpoint{1.326139in}{2.025397in}}%
\pgfpathlineto{\pgfqpoint{1.422933in}{2.033397in}}%
\pgfpathlineto{\pgfqpoint{1.519728in}{2.133105in}}%
\pgfpathlineto{\pgfqpoint{1.616523in}{1.869205in}}%
\pgfpathlineto{\pgfqpoint{1.713317in}{1.721554in}}%
\pgfpathlineto{\pgfqpoint{1.810112in}{2.014337in}}%
\pgfpathlineto{\pgfqpoint{1.906906in}{1.884210in}}%
\pgfpathlineto{\pgfqpoint{2.003701in}{1.677786in}}%
\pgfpathlineto{\pgfqpoint{2.100496in}{1.650693in}}%
\pgfpathlineto{\pgfqpoint{2.197290in}{1.635146in}}%
\pgfpathlineto{\pgfqpoint{2.294085in}{1.580422in}}%
\pgfpathlineto{\pgfqpoint{2.390880in}{1.557345in}}%
\pgfpathlineto{\pgfqpoint{2.487674in}{1.525212in}}%
\pgfpathlineto{\pgfqpoint{2.584469in}{1.502637in}}%
\pgfpathlineto{\pgfqpoint{2.681264in}{1.484930in}}%
\pgfpathlineto{\pgfqpoint{2.778058in}{1.454904in}}%
\pgfpathlineto{\pgfqpoint{2.874853in}{1.431181in}}%
\pgfpathlineto{\pgfqpoint{2.971648in}{1.415851in}}%
\pgfpathlineto{\pgfqpoint{3.068442in}{1.388430in}}%
\pgfpathlineto{\pgfqpoint{3.165237in}{1.363472in}}%
\pgfpathlineto{\pgfqpoint{3.262031in}{1.337670in}}%
\pgfpathlineto{\pgfqpoint{3.358826in}{1.313774in}}%
\pgfpathlineto{\pgfqpoint{3.455621in}{1.297762in}}%
\pgfpathlineto{\pgfqpoint{3.552415in}{1.269293in}}%
\pgfpathlineto{\pgfqpoint{3.649210in}{1.241041in}}%
\pgfpathlineto{\pgfqpoint{3.746005in}{1.220581in}}%
\pgfpathlineto{\pgfqpoint{3.842799in}{1.205588in}}%
\pgfpathlineto{\pgfqpoint{3.939594in}{1.172568in}}%
\pgfpathlineto{\pgfqpoint{4.036389in}{1.156404in}}%
\pgfusepath{stroke}%
\end{pgfscope}%
\begin{pgfscope}%
\pgfpathrectangle{\pgfqpoint{0.745371in}{0.566590in}}{\pgfqpoint{3.291018in}{1.828724in}}%
\pgfusepath{clip}%
\pgfsetbuttcap%
\pgfsetroundjoin%
\definecolor{currentfill}{rgb}{0.494000,0.184000,0.556000}%
\pgfsetfillcolor{currentfill}%
\pgfsetlinewidth{1.003750pt}%
\definecolor{currentstroke}{rgb}{0.494000,0.184000,0.556000}%
\pgfsetstrokecolor{currentstroke}%
\pgfsetdash{}{0pt}%
\pgfsys@defobject{currentmarker}{\pgfqpoint{-0.041667in}{-0.041667in}}{\pgfqpoint{0.041667in}{0.041667in}}{%
\pgfpathmoveto{\pgfqpoint{-0.041667in}{-0.041667in}}%
\pgfpathlineto{\pgfqpoint{0.041667in}{0.041667in}}%
\pgfpathmoveto{\pgfqpoint{-0.041667in}{0.041667in}}%
\pgfpathlineto{\pgfqpoint{0.041667in}{-0.041667in}}%
\pgfusepath{stroke,fill}%
}%
\begin{pgfscope}%
\pgfsys@transformshift{0.745371in}{2.206886in}%
\pgfsys@useobject{currentmarker}{}%
\end{pgfscope}%
\begin{pgfscope}%
\pgfsys@transformshift{1.132549in}{2.279531in}%
\pgfsys@useobject{currentmarker}{}%
\end{pgfscope}%
\begin{pgfscope}%
\pgfsys@transformshift{1.519728in}{2.133105in}%
\pgfsys@useobject{currentmarker}{}%
\end{pgfscope}%
\begin{pgfscope}%
\pgfsys@transformshift{1.906906in}{1.884210in}%
\pgfsys@useobject{currentmarker}{}%
\end{pgfscope}%
\begin{pgfscope}%
\pgfsys@transformshift{2.294085in}{1.580422in}%
\pgfsys@useobject{currentmarker}{}%
\end{pgfscope}%
\begin{pgfscope}%
\pgfsys@transformshift{2.681264in}{1.484930in}%
\pgfsys@useobject{currentmarker}{}%
\end{pgfscope}%
\begin{pgfscope}%
\pgfsys@transformshift{3.068442in}{1.388430in}%
\pgfsys@useobject{currentmarker}{}%
\end{pgfscope}%
\begin{pgfscope}%
\pgfsys@transformshift{3.455621in}{1.297762in}%
\pgfsys@useobject{currentmarker}{}%
\end{pgfscope}%
\begin{pgfscope}%
\pgfsys@transformshift{3.842799in}{1.205588in}%
\pgfsys@useobject{currentmarker}{}%
\end{pgfscope}%
\end{pgfscope}%
\begin{pgfscope}%
\pgfpathrectangle{\pgfqpoint{0.745371in}{0.566590in}}{\pgfqpoint{3.291018in}{1.828724in}}%
\pgfusepath{clip}%
\pgfsetbuttcap%
\pgfsetroundjoin%
\pgfsetlinewidth{1.505625pt}%
\definecolor{currentstroke}{rgb}{0.635000,0.078000,0.184000}%
\pgfsetstrokecolor{currentstroke}%
\pgfsetdash{{5.550000pt}{2.400000pt}}{0.000000pt}%
\pgfpathmoveto{\pgfqpoint{0.745371in}{2.146464in}}%
\pgfpathlineto{\pgfqpoint{0.842165in}{2.005701in}}%
\pgfpathlineto{\pgfqpoint{0.938960in}{2.140165in}}%
\pgfpathlineto{\pgfqpoint{1.035755in}{2.130870in}}%
\pgfpathlineto{\pgfqpoint{1.132549in}{2.118838in}}%
\pgfpathlineto{\pgfqpoint{1.229344in}{2.065602in}}%
\pgfpathlineto{\pgfqpoint{1.326139in}{2.037576in}}%
\pgfpathlineto{\pgfqpoint{1.422933in}{2.247687in}}%
\pgfpathlineto{\pgfqpoint{1.519728in}{2.104739in}}%
\pgfpathlineto{\pgfqpoint{1.616523in}{1.915718in}}%
\pgfpathlineto{\pgfqpoint{1.713317in}{2.030054in}}%
\pgfpathlineto{\pgfqpoint{1.810112in}{1.938079in}}%
\pgfpathlineto{\pgfqpoint{1.906906in}{1.808304in}}%
\pgfpathlineto{\pgfqpoint{2.003701in}{1.652677in}}%
\pgfpathlineto{\pgfqpoint{2.100496in}{1.595363in}}%
\pgfpathlineto{\pgfqpoint{2.197290in}{1.565522in}}%
\pgfpathlineto{\pgfqpoint{2.294085in}{1.527898in}}%
\pgfpathlineto{\pgfqpoint{2.390880in}{1.503321in}}%
\pgfpathlineto{\pgfqpoint{2.487674in}{1.470458in}}%
\pgfpathlineto{\pgfqpoint{2.584469in}{1.452030in}}%
\pgfpathlineto{\pgfqpoint{2.681264in}{1.433525in}}%
\pgfpathlineto{\pgfqpoint{2.778058in}{1.401687in}}%
\pgfpathlineto{\pgfqpoint{2.874853in}{1.380394in}}%
\pgfpathlineto{\pgfqpoint{2.971648in}{1.367520in}}%
\pgfpathlineto{\pgfqpoint{3.068442in}{1.331308in}}%
\pgfpathlineto{\pgfqpoint{3.165237in}{1.309162in}}%
\pgfpathlineto{\pgfqpoint{3.262031in}{1.289398in}}%
\pgfpathlineto{\pgfqpoint{3.358826in}{1.266068in}}%
\pgfpathlineto{\pgfqpoint{3.455621in}{1.244855in}}%
\pgfpathlineto{\pgfqpoint{3.552415in}{1.216173in}}%
\pgfpathlineto{\pgfqpoint{3.649210in}{1.188383in}}%
\pgfpathlineto{\pgfqpoint{3.746005in}{1.165337in}}%
\pgfpathlineto{\pgfqpoint{3.842799in}{1.154526in}}%
\pgfpathlineto{\pgfqpoint{3.939594in}{1.122216in}}%
\pgfpathlineto{\pgfqpoint{4.036389in}{1.107153in}}%
\pgfusepath{stroke}%
\end{pgfscope}%
\begin{pgfscope}%
\pgfpathrectangle{\pgfqpoint{0.745371in}{0.566590in}}{\pgfqpoint{3.291018in}{1.828724in}}%
\pgfusepath{clip}%
\pgfsetbuttcap%
\pgfsetmiterjoin%
\definecolor{currentfill}{rgb}{0.000000,0.000000,0.000000}%
\pgfsetfillcolor{currentfill}%
\pgfsetfillopacity{0.000000}%
\pgfsetlinewidth{1.003750pt}%
\definecolor{currentstroke}{rgb}{0.635000,0.078000,0.184000}%
\pgfsetstrokecolor{currentstroke}%
\pgfsetdash{}{0pt}%
\pgfsys@defobject{currentmarker}{\pgfqpoint{-0.035355in}{-0.058926in}}{\pgfqpoint{0.035355in}{0.058926in}}{%
\pgfpathmoveto{\pgfqpoint{-0.000000in}{-0.058926in}}%
\pgfpathlineto{\pgfqpoint{0.035355in}{0.000000in}}%
\pgfpathlineto{\pgfqpoint{0.000000in}{0.058926in}}%
\pgfpathlineto{\pgfqpoint{-0.035355in}{0.000000in}}%
\pgfpathclose%
\pgfusepath{stroke,fill}%
}%
\begin{pgfscope}%
\pgfsys@transformshift{0.745371in}{2.146464in}%
\pgfsys@useobject{currentmarker}{}%
\end{pgfscope}%
\begin{pgfscope}%
\pgfsys@transformshift{1.035755in}{2.130870in}%
\pgfsys@useobject{currentmarker}{}%
\end{pgfscope}%
\begin{pgfscope}%
\pgfsys@transformshift{1.326139in}{2.037576in}%
\pgfsys@useobject{currentmarker}{}%
\end{pgfscope}%
\begin{pgfscope}%
\pgfsys@transformshift{1.616523in}{1.915718in}%
\pgfsys@useobject{currentmarker}{}%
\end{pgfscope}%
\begin{pgfscope}%
\pgfsys@transformshift{1.906906in}{1.808304in}%
\pgfsys@useobject{currentmarker}{}%
\end{pgfscope}%
\begin{pgfscope}%
\pgfsys@transformshift{2.197290in}{1.565522in}%
\pgfsys@useobject{currentmarker}{}%
\end{pgfscope}%
\begin{pgfscope}%
\pgfsys@transformshift{2.487674in}{1.470458in}%
\pgfsys@useobject{currentmarker}{}%
\end{pgfscope}%
\begin{pgfscope}%
\pgfsys@transformshift{2.778058in}{1.401687in}%
\pgfsys@useobject{currentmarker}{}%
\end{pgfscope}%
\begin{pgfscope}%
\pgfsys@transformshift{3.068442in}{1.331308in}%
\pgfsys@useobject{currentmarker}{}%
\end{pgfscope}%
\begin{pgfscope}%
\pgfsys@transformshift{3.358826in}{1.266068in}%
\pgfsys@useobject{currentmarker}{}%
\end{pgfscope}%
\begin{pgfscope}%
\pgfsys@transformshift{3.649210in}{1.188383in}%
\pgfsys@useobject{currentmarker}{}%
\end{pgfscope}%
\begin{pgfscope}%
\pgfsys@transformshift{3.939594in}{1.122216in}%
\pgfsys@useobject{currentmarker}{}%
\end{pgfscope}%
\end{pgfscope}%
\begin{pgfscope}%
\pgfpathrectangle{\pgfqpoint{0.745371in}{0.566590in}}{\pgfqpoint{3.291018in}{1.828724in}}%
\pgfusepath{clip}%
\pgfsetbuttcap%
\pgfsetroundjoin%
\pgfsetlinewidth{1.505625pt}%
\definecolor{currentstroke}{rgb}{0.301000,0.745000,0.741000}%
\pgfsetstrokecolor{currentstroke}%
\pgfsetdash{{5.550000pt}{2.400000pt}}{0.000000pt}%
\pgfpathmoveto{\pgfqpoint{0.745371in}{2.136426in}}%
\pgfpathlineto{\pgfqpoint{0.842165in}{2.243298in}}%
\pgfpathlineto{\pgfqpoint{0.938960in}{2.179999in}}%
\pgfpathlineto{\pgfqpoint{1.035755in}{2.172626in}}%
\pgfpathlineto{\pgfqpoint{1.132549in}{2.052858in}}%
\pgfpathlineto{\pgfqpoint{1.229344in}{2.111822in}}%
\pgfpathlineto{\pgfqpoint{1.326139in}{2.169842in}}%
\pgfpathlineto{\pgfqpoint{1.422933in}{2.120794in}}%
\pgfpathlineto{\pgfqpoint{1.519728in}{2.134946in}}%
\pgfpathlineto{\pgfqpoint{1.616523in}{2.151722in}}%
\pgfpathlineto{\pgfqpoint{1.713317in}{2.115572in}}%
\pgfpathlineto{\pgfqpoint{1.810112in}{2.113989in}}%
\pgfpathlineto{\pgfqpoint{1.906906in}{2.027909in}}%
\pgfpathlineto{\pgfqpoint{2.003701in}{2.230074in}}%
\pgfpathlineto{\pgfqpoint{2.100496in}{1.614982in}}%
\pgfpathlineto{\pgfqpoint{2.197290in}{1.946496in}}%
\pgfpathlineto{\pgfqpoint{2.294085in}{1.565750in}}%
\pgfpathlineto{\pgfqpoint{2.390880in}{1.549423in}}%
\pgfpathlineto{\pgfqpoint{2.487674in}{1.528039in}}%
\pgfpathlineto{\pgfqpoint{2.584469in}{1.503746in}}%
\pgfpathlineto{\pgfqpoint{2.681264in}{1.478139in}}%
\pgfpathlineto{\pgfqpoint{2.778058in}{1.452270in}}%
\pgfpathlineto{\pgfqpoint{2.874853in}{1.424420in}}%
\pgfpathlineto{\pgfqpoint{2.971648in}{1.408782in}}%
\pgfpathlineto{\pgfqpoint{3.068442in}{1.380463in}}%
\pgfpathlineto{\pgfqpoint{3.165237in}{1.362797in}}%
\pgfpathlineto{\pgfqpoint{3.262031in}{1.346450in}}%
\pgfpathlineto{\pgfqpoint{3.358826in}{1.311453in}}%
\pgfpathlineto{\pgfqpoint{3.455621in}{1.291302in}}%
\pgfpathlineto{\pgfqpoint{3.552415in}{1.263479in}}%
\pgfpathlineto{\pgfqpoint{3.649210in}{1.245539in}}%
\pgfpathlineto{\pgfqpoint{3.746005in}{1.219590in}}%
\pgfpathlineto{\pgfqpoint{3.842799in}{1.199375in}}%
\pgfpathlineto{\pgfqpoint{3.939594in}{1.174750in}}%
\pgfpathlineto{\pgfqpoint{4.036389in}{1.148302in}}%
\pgfusepath{stroke}%
\end{pgfscope}%
\begin{pgfscope}%
\pgfpathrectangle{\pgfqpoint{0.745371in}{0.566590in}}{\pgfqpoint{3.291018in}{1.828724in}}%
\pgfusepath{clip}%
\pgfsetbuttcap%
\pgfsetmiterjoin%
\definecolor{currentfill}{rgb}{0.000000,0.000000,0.000000}%
\pgfsetfillcolor{currentfill}%
\pgfsetfillopacity{0.000000}%
\pgfsetlinewidth{1.003750pt}%
\definecolor{currentstroke}{rgb}{0.301000,0.745000,0.741000}%
\pgfsetstrokecolor{currentstroke}%
\pgfsetdash{}{0pt}%
\pgfsys@defobject{currentmarker}{\pgfqpoint{-0.041667in}{-0.041667in}}{\pgfqpoint{0.041667in}{0.041667in}}{%
\pgfpathmoveto{\pgfqpoint{0.000000in}{0.041667in}}%
\pgfpathlineto{\pgfqpoint{-0.041667in}{-0.041667in}}%
\pgfpathlineto{\pgfqpoint{0.041667in}{-0.041667in}}%
\pgfpathclose%
\pgfusepath{stroke,fill}%
}%
\begin{pgfscope}%
\pgfsys@transformshift{0.745371in}{2.136426in}%
\pgfsys@useobject{currentmarker}{}%
\end{pgfscope}%
\begin{pgfscope}%
\pgfsys@transformshift{1.229344in}{2.111822in}%
\pgfsys@useobject{currentmarker}{}%
\end{pgfscope}%
\begin{pgfscope}%
\pgfsys@transformshift{1.713317in}{2.115572in}%
\pgfsys@useobject{currentmarker}{}%
\end{pgfscope}%
\begin{pgfscope}%
\pgfsys@transformshift{2.197290in}{1.946496in}%
\pgfsys@useobject{currentmarker}{}%
\end{pgfscope}%
\begin{pgfscope}%
\pgfsys@transformshift{2.681264in}{1.478139in}%
\pgfsys@useobject{currentmarker}{}%
\end{pgfscope}%
\begin{pgfscope}%
\pgfsys@transformshift{3.165237in}{1.362797in}%
\pgfsys@useobject{currentmarker}{}%
\end{pgfscope}%
\begin{pgfscope}%
\pgfsys@transformshift{3.649210in}{1.245539in}%
\pgfsys@useobject{currentmarker}{}%
\end{pgfscope}%
\end{pgfscope}%
\begin{pgfscope}%
\pgfpathrectangle{\pgfqpoint{0.745371in}{0.566590in}}{\pgfqpoint{3.291018in}{1.828724in}}%
\pgfusepath{clip}%
\pgfsetrectcap%
\pgfsetroundjoin%
\pgfsetlinewidth{1.505625pt}%
\definecolor{currentstroke}{rgb}{0.000000,0.447000,0.741000}%
\pgfsetstrokecolor{currentstroke}%
\pgfsetdash{}{0pt}%
\pgfpathmoveto{\pgfqpoint{0.745371in}{1.731262in}}%
\pgfpathlineto{\pgfqpoint{0.980444in}{1.739197in}}%
\pgfpathlineto{\pgfqpoint{1.215516in}{1.762901in}}%
\pgfpathlineto{\pgfqpoint{1.450589in}{1.717869in}}%
\pgfpathlineto{\pgfqpoint{1.685662in}{1.547810in}}%
\pgfpathlineto{\pgfqpoint{1.920734in}{1.486984in}}%
\pgfpathlineto{\pgfqpoint{2.155807in}{1.434845in}}%
\pgfpathlineto{\pgfqpoint{2.390880in}{1.375957in}}%
\pgfpathlineto{\pgfqpoint{2.625952in}{1.320325in}}%
\pgfpathlineto{\pgfqpoint{2.861025in}{1.268755in}}%
\pgfpathlineto{\pgfqpoint{3.096098in}{1.210931in}}%
\pgfpathlineto{\pgfqpoint{3.331170in}{1.153113in}}%
\pgfpathlineto{\pgfqpoint{3.566243in}{1.098844in}}%
\pgfpathlineto{\pgfqpoint{3.801316in}{1.034413in}}%
\pgfpathlineto{\pgfqpoint{4.036389in}{0.983832in}}%
\pgfusepath{stroke}%
\end{pgfscope}%
\begin{pgfscope}%
\pgfpathrectangle{\pgfqpoint{0.745371in}{0.566590in}}{\pgfqpoint{3.291018in}{1.828724in}}%
\pgfusepath{clip}%
\pgfsetbuttcap%
\pgfsetroundjoin%
\definecolor{currentfill}{rgb}{0.000000,0.000000,0.000000}%
\pgfsetfillcolor{currentfill}%
\pgfsetfillopacity{0.000000}%
\pgfsetlinewidth{1.003750pt}%
\definecolor{currentstroke}{rgb}{0.000000,0.447000,0.741000}%
\pgfsetstrokecolor{currentstroke}%
\pgfsetdash{}{0pt}%
\pgfsys@defobject{currentmarker}{\pgfqpoint{-0.041667in}{-0.041667in}}{\pgfqpoint{0.041667in}{0.041667in}}{%
\pgfpathmoveto{\pgfqpoint{0.000000in}{-0.041667in}}%
\pgfpathcurveto{\pgfqpoint{0.011050in}{-0.041667in}}{\pgfqpoint{0.021649in}{-0.037276in}}{\pgfqpoint{0.029463in}{-0.029463in}}%
\pgfpathcurveto{\pgfqpoint{0.037276in}{-0.021649in}}{\pgfqpoint{0.041667in}{-0.011050in}}{\pgfqpoint{0.041667in}{0.000000in}}%
\pgfpathcurveto{\pgfqpoint{0.041667in}{0.011050in}}{\pgfqpoint{0.037276in}{0.021649in}}{\pgfqpoint{0.029463in}{0.029463in}}%
\pgfpathcurveto{\pgfqpoint{0.021649in}{0.037276in}}{\pgfqpoint{0.011050in}{0.041667in}}{\pgfqpoint{0.000000in}{0.041667in}}%
\pgfpathcurveto{\pgfqpoint{-0.011050in}{0.041667in}}{\pgfqpoint{-0.021649in}{0.037276in}}{\pgfqpoint{-0.029463in}{0.029463in}}%
\pgfpathcurveto{\pgfqpoint{-0.037276in}{0.021649in}}{\pgfqpoint{-0.041667in}{0.011050in}}{\pgfqpoint{-0.041667in}{0.000000in}}%
\pgfpathcurveto{\pgfqpoint{-0.041667in}{-0.011050in}}{\pgfqpoint{-0.037276in}{-0.021649in}}{\pgfqpoint{-0.029463in}{-0.029463in}}%
\pgfpathcurveto{\pgfqpoint{-0.021649in}{-0.037276in}}{\pgfqpoint{-0.011050in}{-0.041667in}}{\pgfqpoint{0.000000in}{-0.041667in}}%
\pgfpathclose%
\pgfusepath{stroke,fill}%
}%
\begin{pgfscope}%
\pgfsys@transformshift{0.745371in}{1.731262in}%
\pgfsys@useobject{currentmarker}{}%
\end{pgfscope}%
\begin{pgfscope}%
\pgfsys@transformshift{0.980444in}{1.739197in}%
\pgfsys@useobject{currentmarker}{}%
\end{pgfscope}%
\begin{pgfscope}%
\pgfsys@transformshift{1.215516in}{1.762901in}%
\pgfsys@useobject{currentmarker}{}%
\end{pgfscope}%
\begin{pgfscope}%
\pgfsys@transformshift{1.450589in}{1.717869in}%
\pgfsys@useobject{currentmarker}{}%
\end{pgfscope}%
\begin{pgfscope}%
\pgfsys@transformshift{1.685662in}{1.547810in}%
\pgfsys@useobject{currentmarker}{}%
\end{pgfscope}%
\begin{pgfscope}%
\pgfsys@transformshift{1.920734in}{1.486984in}%
\pgfsys@useobject{currentmarker}{}%
\end{pgfscope}%
\begin{pgfscope}%
\pgfsys@transformshift{2.155807in}{1.434845in}%
\pgfsys@useobject{currentmarker}{}%
\end{pgfscope}%
\begin{pgfscope}%
\pgfsys@transformshift{2.390880in}{1.375957in}%
\pgfsys@useobject{currentmarker}{}%
\end{pgfscope}%
\begin{pgfscope}%
\pgfsys@transformshift{2.625952in}{1.320325in}%
\pgfsys@useobject{currentmarker}{}%
\end{pgfscope}%
\begin{pgfscope}%
\pgfsys@transformshift{2.861025in}{1.268755in}%
\pgfsys@useobject{currentmarker}{}%
\end{pgfscope}%
\begin{pgfscope}%
\pgfsys@transformshift{3.096098in}{1.210931in}%
\pgfsys@useobject{currentmarker}{}%
\end{pgfscope}%
\begin{pgfscope}%
\pgfsys@transformshift{3.331170in}{1.153113in}%
\pgfsys@useobject{currentmarker}{}%
\end{pgfscope}%
\begin{pgfscope}%
\pgfsys@transformshift{3.566243in}{1.098844in}%
\pgfsys@useobject{currentmarker}{}%
\end{pgfscope}%
\begin{pgfscope}%
\pgfsys@transformshift{3.801316in}{1.034413in}%
\pgfsys@useobject{currentmarker}{}%
\end{pgfscope}%
\begin{pgfscope}%
\pgfsys@transformshift{4.036389in}{0.983832in}%
\pgfsys@useobject{currentmarker}{}%
\end{pgfscope}%
\end{pgfscope}%
\begin{pgfscope}%
\pgfpathrectangle{\pgfqpoint{0.745371in}{0.566590in}}{\pgfqpoint{3.291018in}{1.828724in}}%
\pgfusepath{clip}%
\pgfsetrectcap%
\pgfsetroundjoin%
\pgfsetlinewidth{1.505625pt}%
\definecolor{currentstroke}{rgb}{0.850000,0.324000,0.098000}%
\pgfsetstrokecolor{currentstroke}%
\pgfsetdash{}{0pt}%
\pgfpathmoveto{\pgfqpoint{0.745371in}{1.747351in}}%
\pgfpathlineto{\pgfqpoint{0.980444in}{1.760259in}}%
\pgfpathlineto{\pgfqpoint{1.215516in}{1.735807in}}%
\pgfpathlineto{\pgfqpoint{1.450589in}{1.637973in}}%
\pgfpathlineto{\pgfqpoint{1.685662in}{1.406562in}}%
\pgfpathlineto{\pgfqpoint{1.920734in}{1.271793in}}%
\pgfpathlineto{\pgfqpoint{2.155807in}{1.233370in}}%
\pgfpathlineto{\pgfqpoint{2.390880in}{1.162887in}}%
\pgfpathlineto{\pgfqpoint{2.625952in}{1.113989in}}%
\pgfpathlineto{\pgfqpoint{2.861025in}{1.062327in}}%
\pgfpathlineto{\pgfqpoint{3.096098in}{1.003135in}}%
\pgfpathlineto{\pgfqpoint{3.331170in}{0.950875in}}%
\pgfpathlineto{\pgfqpoint{3.566243in}{0.891892in}}%
\pgfpathlineto{\pgfqpoint{3.801316in}{0.837370in}}%
\pgfpathlineto{\pgfqpoint{4.036389in}{0.782456in}}%
\pgfusepath{stroke}%
\end{pgfscope}%
\begin{pgfscope}%
\pgfpathrectangle{\pgfqpoint{0.745371in}{0.566590in}}{\pgfqpoint{3.291018in}{1.828724in}}%
\pgfusepath{clip}%
\pgfsetbuttcap%
\pgfsetroundjoin%
\definecolor{currentfill}{rgb}{0.850000,0.324000,0.098000}%
\pgfsetfillcolor{currentfill}%
\pgfsetlinewidth{1.003750pt}%
\definecolor{currentstroke}{rgb}{0.850000,0.324000,0.098000}%
\pgfsetstrokecolor{currentstroke}%
\pgfsetdash{}{0pt}%
\pgfsys@defobject{currentmarker}{\pgfqpoint{-0.041667in}{-0.041667in}}{\pgfqpoint{0.041667in}{0.041667in}}{%
\pgfpathmoveto{\pgfqpoint{-0.041667in}{0.000000in}}%
\pgfpathlineto{\pgfqpoint{0.041667in}{0.000000in}}%
\pgfpathmoveto{\pgfqpoint{0.000000in}{-0.041667in}}%
\pgfpathlineto{\pgfqpoint{0.000000in}{0.041667in}}%
\pgfusepath{stroke,fill}%
}%
\begin{pgfscope}%
\pgfsys@transformshift{0.745371in}{1.747351in}%
\pgfsys@useobject{currentmarker}{}%
\end{pgfscope}%
\begin{pgfscope}%
\pgfsys@transformshift{0.980444in}{1.760259in}%
\pgfsys@useobject{currentmarker}{}%
\end{pgfscope}%
\begin{pgfscope}%
\pgfsys@transformshift{1.215516in}{1.735807in}%
\pgfsys@useobject{currentmarker}{}%
\end{pgfscope}%
\begin{pgfscope}%
\pgfsys@transformshift{1.450589in}{1.637973in}%
\pgfsys@useobject{currentmarker}{}%
\end{pgfscope}%
\begin{pgfscope}%
\pgfsys@transformshift{1.685662in}{1.406562in}%
\pgfsys@useobject{currentmarker}{}%
\end{pgfscope}%
\begin{pgfscope}%
\pgfsys@transformshift{1.920734in}{1.271793in}%
\pgfsys@useobject{currentmarker}{}%
\end{pgfscope}%
\begin{pgfscope}%
\pgfsys@transformshift{2.155807in}{1.233370in}%
\pgfsys@useobject{currentmarker}{}%
\end{pgfscope}%
\begin{pgfscope}%
\pgfsys@transformshift{2.390880in}{1.162887in}%
\pgfsys@useobject{currentmarker}{}%
\end{pgfscope}%
\begin{pgfscope}%
\pgfsys@transformshift{2.625952in}{1.113989in}%
\pgfsys@useobject{currentmarker}{}%
\end{pgfscope}%
\begin{pgfscope}%
\pgfsys@transformshift{2.861025in}{1.062327in}%
\pgfsys@useobject{currentmarker}{}%
\end{pgfscope}%
\begin{pgfscope}%
\pgfsys@transformshift{3.096098in}{1.003135in}%
\pgfsys@useobject{currentmarker}{}%
\end{pgfscope}%
\begin{pgfscope}%
\pgfsys@transformshift{3.331170in}{0.950875in}%
\pgfsys@useobject{currentmarker}{}%
\end{pgfscope}%
\begin{pgfscope}%
\pgfsys@transformshift{3.566243in}{0.891892in}%
\pgfsys@useobject{currentmarker}{}%
\end{pgfscope}%
\begin{pgfscope}%
\pgfsys@transformshift{3.801316in}{0.837370in}%
\pgfsys@useobject{currentmarker}{}%
\end{pgfscope}%
\begin{pgfscope}%
\pgfsys@transformshift{4.036389in}{0.782456in}%
\pgfsys@useobject{currentmarker}{}%
\end{pgfscope}%
\end{pgfscope}%
\begin{pgfscope}%
\pgfpathrectangle{\pgfqpoint{0.745371in}{0.566590in}}{\pgfqpoint{3.291018in}{1.828724in}}%
\pgfusepath{clip}%
\pgfsetrectcap%
\pgfsetroundjoin%
\pgfsetlinewidth{1.505625pt}%
\definecolor{currentstroke}{rgb}{0.000000,0.500000,0.000000}%
\pgfsetstrokecolor{currentstroke}%
\pgfsetdash{}{0pt}%
\pgfpathmoveto{\pgfqpoint{0.745371in}{1.796133in}}%
\pgfpathlineto{\pgfqpoint{0.980444in}{1.795634in}}%
\pgfpathlineto{\pgfqpoint{1.215516in}{1.753985in}}%
\pgfpathlineto{\pgfqpoint{1.450589in}{1.688013in}}%
\pgfpathlineto{\pgfqpoint{1.685662in}{1.542092in}}%
\pgfpathlineto{\pgfqpoint{1.920734in}{1.455545in}}%
\pgfpathlineto{\pgfqpoint{2.155807in}{1.402328in}}%
\pgfpathlineto{\pgfqpoint{2.390880in}{1.341884in}}%
\pgfpathlineto{\pgfqpoint{2.625952in}{1.287949in}}%
\pgfpathlineto{\pgfqpoint{2.861025in}{1.235600in}}%
\pgfpathlineto{\pgfqpoint{3.096098in}{1.165718in}}%
\pgfpathlineto{\pgfqpoint{3.331170in}{1.122845in}}%
\pgfpathlineto{\pgfqpoint{3.566243in}{1.061652in}}%
\pgfpathlineto{\pgfqpoint{3.801316in}{1.002906in}}%
\pgfpathlineto{\pgfqpoint{4.036389in}{0.947396in}}%
\pgfusepath{stroke}%
\end{pgfscope}%
\begin{pgfscope}%
\pgfpathrectangle{\pgfqpoint{0.745371in}{0.566590in}}{\pgfqpoint{3.291018in}{1.828724in}}%
\pgfusepath{clip}%
\pgfsetbuttcap%
\pgfsetmiterjoin%
\definecolor{currentfill}{rgb}{0.000000,0.000000,0.000000}%
\pgfsetfillcolor{currentfill}%
\pgfsetfillopacity{0.000000}%
\pgfsetlinewidth{1.003750pt}%
\definecolor{currentstroke}{rgb}{0.000000,0.500000,0.000000}%
\pgfsetstrokecolor{currentstroke}%
\pgfsetdash{}{0pt}%
\pgfsys@defobject{currentmarker}{\pgfqpoint{-0.041667in}{-0.041667in}}{\pgfqpoint{0.041667in}{0.041667in}}{%
\pgfpathmoveto{\pgfqpoint{-0.041667in}{-0.041667in}}%
\pgfpathlineto{\pgfqpoint{0.041667in}{-0.041667in}}%
\pgfpathlineto{\pgfqpoint{0.041667in}{0.041667in}}%
\pgfpathlineto{\pgfqpoint{-0.041667in}{0.041667in}}%
\pgfpathclose%
\pgfusepath{stroke,fill}%
}%
\begin{pgfscope}%
\pgfsys@transformshift{0.745371in}{1.796133in}%
\pgfsys@useobject{currentmarker}{}%
\end{pgfscope}%
\begin{pgfscope}%
\pgfsys@transformshift{0.980444in}{1.795634in}%
\pgfsys@useobject{currentmarker}{}%
\end{pgfscope}%
\begin{pgfscope}%
\pgfsys@transformshift{1.215516in}{1.753985in}%
\pgfsys@useobject{currentmarker}{}%
\end{pgfscope}%
\begin{pgfscope}%
\pgfsys@transformshift{1.450589in}{1.688013in}%
\pgfsys@useobject{currentmarker}{}%
\end{pgfscope}%
\begin{pgfscope}%
\pgfsys@transformshift{1.685662in}{1.542092in}%
\pgfsys@useobject{currentmarker}{}%
\end{pgfscope}%
\begin{pgfscope}%
\pgfsys@transformshift{1.920734in}{1.455545in}%
\pgfsys@useobject{currentmarker}{}%
\end{pgfscope}%
\begin{pgfscope}%
\pgfsys@transformshift{2.155807in}{1.402328in}%
\pgfsys@useobject{currentmarker}{}%
\end{pgfscope}%
\begin{pgfscope}%
\pgfsys@transformshift{2.390880in}{1.341884in}%
\pgfsys@useobject{currentmarker}{}%
\end{pgfscope}%
\begin{pgfscope}%
\pgfsys@transformshift{2.625952in}{1.287949in}%
\pgfsys@useobject{currentmarker}{}%
\end{pgfscope}%
\begin{pgfscope}%
\pgfsys@transformshift{2.861025in}{1.235600in}%
\pgfsys@useobject{currentmarker}{}%
\end{pgfscope}%
\begin{pgfscope}%
\pgfsys@transformshift{3.096098in}{1.165718in}%
\pgfsys@useobject{currentmarker}{}%
\end{pgfscope}%
\begin{pgfscope}%
\pgfsys@transformshift{3.331170in}{1.122845in}%
\pgfsys@useobject{currentmarker}{}%
\end{pgfscope}%
\begin{pgfscope}%
\pgfsys@transformshift{3.566243in}{1.061652in}%
\pgfsys@useobject{currentmarker}{}%
\end{pgfscope}%
\begin{pgfscope}%
\pgfsys@transformshift{3.801316in}{1.002906in}%
\pgfsys@useobject{currentmarker}{}%
\end{pgfscope}%
\begin{pgfscope}%
\pgfsys@transformshift{4.036389in}{0.947396in}%
\pgfsys@useobject{currentmarker}{}%
\end{pgfscope}%
\end{pgfscope}%
\begin{pgfscope}%
\pgfpathrectangle{\pgfqpoint{0.745371in}{0.566590in}}{\pgfqpoint{3.291018in}{1.828724in}}%
\pgfusepath{clip}%
\pgfsetrectcap%
\pgfsetroundjoin%
\pgfsetlinewidth{1.505625pt}%
\definecolor{currentstroke}{rgb}{0.494000,0.184000,0.556000}%
\pgfsetstrokecolor{currentstroke}%
\pgfsetdash{}{0pt}%
\pgfpathmoveto{\pgfqpoint{0.745371in}{1.689281in}}%
\pgfpathlineto{\pgfqpoint{0.980444in}{1.667892in}}%
\pgfpathlineto{\pgfqpoint{1.215516in}{1.716396in}}%
\pgfpathlineto{\pgfqpoint{1.450589in}{1.680783in}}%
\pgfpathlineto{\pgfqpoint{1.685662in}{1.484375in}}%
\pgfpathlineto{\pgfqpoint{1.920734in}{1.388913in}}%
\pgfpathlineto{\pgfqpoint{2.155807in}{1.340992in}}%
\pgfpathlineto{\pgfqpoint{2.390880in}{1.275897in}}%
\pgfpathlineto{\pgfqpoint{2.625952in}{1.219198in}}%
\pgfpathlineto{\pgfqpoint{2.861025in}{1.162855in}}%
\pgfpathlineto{\pgfqpoint{3.096098in}{1.106579in}}%
\pgfpathlineto{\pgfqpoint{3.331170in}{1.050092in}}%
\pgfpathlineto{\pgfqpoint{3.566243in}{0.993437in}}%
\pgfpathlineto{\pgfqpoint{3.801316in}{0.939557in}}%
\pgfpathlineto{\pgfqpoint{4.036389in}{0.879987in}}%
\pgfusepath{stroke}%
\end{pgfscope}%
\begin{pgfscope}%
\pgfpathrectangle{\pgfqpoint{0.745371in}{0.566590in}}{\pgfqpoint{3.291018in}{1.828724in}}%
\pgfusepath{clip}%
\pgfsetbuttcap%
\pgfsetroundjoin%
\definecolor{currentfill}{rgb}{0.494000,0.184000,0.556000}%
\pgfsetfillcolor{currentfill}%
\pgfsetlinewidth{1.003750pt}%
\definecolor{currentstroke}{rgb}{0.494000,0.184000,0.556000}%
\pgfsetstrokecolor{currentstroke}%
\pgfsetdash{}{0pt}%
\pgfsys@defobject{currentmarker}{\pgfqpoint{-0.041667in}{-0.041667in}}{\pgfqpoint{0.041667in}{0.041667in}}{%
\pgfpathmoveto{\pgfqpoint{-0.041667in}{-0.041667in}}%
\pgfpathlineto{\pgfqpoint{0.041667in}{0.041667in}}%
\pgfpathmoveto{\pgfqpoint{-0.041667in}{0.041667in}}%
\pgfpathlineto{\pgfqpoint{0.041667in}{-0.041667in}}%
\pgfusepath{stroke,fill}%
}%
\begin{pgfscope}%
\pgfsys@transformshift{0.745371in}{1.689281in}%
\pgfsys@useobject{currentmarker}{}%
\end{pgfscope}%
\begin{pgfscope}%
\pgfsys@transformshift{0.980444in}{1.667892in}%
\pgfsys@useobject{currentmarker}{}%
\end{pgfscope}%
\begin{pgfscope}%
\pgfsys@transformshift{1.215516in}{1.716396in}%
\pgfsys@useobject{currentmarker}{}%
\end{pgfscope}%
\begin{pgfscope}%
\pgfsys@transformshift{1.450589in}{1.680783in}%
\pgfsys@useobject{currentmarker}{}%
\end{pgfscope}%
\begin{pgfscope}%
\pgfsys@transformshift{1.685662in}{1.484375in}%
\pgfsys@useobject{currentmarker}{}%
\end{pgfscope}%
\begin{pgfscope}%
\pgfsys@transformshift{1.920734in}{1.388913in}%
\pgfsys@useobject{currentmarker}{}%
\end{pgfscope}%
\begin{pgfscope}%
\pgfsys@transformshift{2.155807in}{1.340992in}%
\pgfsys@useobject{currentmarker}{}%
\end{pgfscope}%
\begin{pgfscope}%
\pgfsys@transformshift{2.390880in}{1.275897in}%
\pgfsys@useobject{currentmarker}{}%
\end{pgfscope}%
\begin{pgfscope}%
\pgfsys@transformshift{2.625952in}{1.219198in}%
\pgfsys@useobject{currentmarker}{}%
\end{pgfscope}%
\begin{pgfscope}%
\pgfsys@transformshift{2.861025in}{1.162855in}%
\pgfsys@useobject{currentmarker}{}%
\end{pgfscope}%
\begin{pgfscope}%
\pgfsys@transformshift{3.096098in}{1.106579in}%
\pgfsys@useobject{currentmarker}{}%
\end{pgfscope}%
\begin{pgfscope}%
\pgfsys@transformshift{3.331170in}{1.050092in}%
\pgfsys@useobject{currentmarker}{}%
\end{pgfscope}%
\begin{pgfscope}%
\pgfsys@transformshift{3.566243in}{0.993437in}%
\pgfsys@useobject{currentmarker}{}%
\end{pgfscope}%
\begin{pgfscope}%
\pgfsys@transformshift{3.801316in}{0.939557in}%
\pgfsys@useobject{currentmarker}{}%
\end{pgfscope}%
\begin{pgfscope}%
\pgfsys@transformshift{4.036389in}{0.879987in}%
\pgfsys@useobject{currentmarker}{}%
\end{pgfscope}%
\end{pgfscope}%
\begin{pgfscope}%
\pgfpathrectangle{\pgfqpoint{0.745371in}{0.566590in}}{\pgfqpoint{3.291018in}{1.828724in}}%
\pgfusepath{clip}%
\pgfsetrectcap%
\pgfsetroundjoin%
\pgfsetlinewidth{1.505625pt}%
\definecolor{currentstroke}{rgb}{0.635000,0.078000,0.184000}%
\pgfsetstrokecolor{currentstroke}%
\pgfsetdash{}{0pt}%
\pgfpathmoveto{\pgfqpoint{0.745371in}{1.723313in}}%
\pgfpathlineto{\pgfqpoint{0.980444in}{1.750989in}}%
\pgfpathlineto{\pgfqpoint{1.215516in}{1.743331in}}%
\pgfpathlineto{\pgfqpoint{1.450589in}{1.640898in}}%
\pgfpathlineto{\pgfqpoint{1.685662in}{1.511134in}}%
\pgfpathlineto{\pgfqpoint{1.920734in}{1.307640in}}%
\pgfpathlineto{\pgfqpoint{2.155807in}{1.258126in}}%
\pgfpathlineto{\pgfqpoint{2.390880in}{1.193059in}}%
\pgfpathlineto{\pgfqpoint{2.625952in}{1.141166in}}%
\pgfpathlineto{\pgfqpoint{2.861025in}{1.082233in}}%
\pgfpathlineto{\pgfqpoint{3.096098in}{1.023904in}}%
\pgfpathlineto{\pgfqpoint{3.331170in}{0.963297in}}%
\pgfpathlineto{\pgfqpoint{3.566243in}{0.908393in}}%
\pgfpathlineto{\pgfqpoint{3.801316in}{0.848803in}}%
\pgfpathlineto{\pgfqpoint{4.036389in}{0.806641in}}%
\pgfusepath{stroke}%
\end{pgfscope}%
\begin{pgfscope}%
\pgfpathrectangle{\pgfqpoint{0.745371in}{0.566590in}}{\pgfqpoint{3.291018in}{1.828724in}}%
\pgfusepath{clip}%
\pgfsetbuttcap%
\pgfsetmiterjoin%
\definecolor{currentfill}{rgb}{0.000000,0.000000,0.000000}%
\pgfsetfillcolor{currentfill}%
\pgfsetfillopacity{0.000000}%
\pgfsetlinewidth{1.003750pt}%
\definecolor{currentstroke}{rgb}{0.635000,0.078000,0.184000}%
\pgfsetstrokecolor{currentstroke}%
\pgfsetdash{}{0pt}%
\pgfsys@defobject{currentmarker}{\pgfqpoint{-0.035355in}{-0.058926in}}{\pgfqpoint{0.035355in}{0.058926in}}{%
\pgfpathmoveto{\pgfqpoint{-0.000000in}{-0.058926in}}%
\pgfpathlineto{\pgfqpoint{0.035355in}{0.000000in}}%
\pgfpathlineto{\pgfqpoint{0.000000in}{0.058926in}}%
\pgfpathlineto{\pgfqpoint{-0.035355in}{0.000000in}}%
\pgfpathclose%
\pgfusepath{stroke,fill}%
}%
\begin{pgfscope}%
\pgfsys@transformshift{0.745371in}{1.723313in}%
\pgfsys@useobject{currentmarker}{}%
\end{pgfscope}%
\begin{pgfscope}%
\pgfsys@transformshift{0.980444in}{1.750989in}%
\pgfsys@useobject{currentmarker}{}%
\end{pgfscope}%
\begin{pgfscope}%
\pgfsys@transformshift{1.215516in}{1.743331in}%
\pgfsys@useobject{currentmarker}{}%
\end{pgfscope}%
\begin{pgfscope}%
\pgfsys@transformshift{1.450589in}{1.640898in}%
\pgfsys@useobject{currentmarker}{}%
\end{pgfscope}%
\begin{pgfscope}%
\pgfsys@transformshift{1.685662in}{1.511134in}%
\pgfsys@useobject{currentmarker}{}%
\end{pgfscope}%
\begin{pgfscope}%
\pgfsys@transformshift{1.920734in}{1.307640in}%
\pgfsys@useobject{currentmarker}{}%
\end{pgfscope}%
\begin{pgfscope}%
\pgfsys@transformshift{2.155807in}{1.258126in}%
\pgfsys@useobject{currentmarker}{}%
\end{pgfscope}%
\begin{pgfscope}%
\pgfsys@transformshift{2.390880in}{1.193059in}%
\pgfsys@useobject{currentmarker}{}%
\end{pgfscope}%
\begin{pgfscope}%
\pgfsys@transformshift{2.625952in}{1.141166in}%
\pgfsys@useobject{currentmarker}{}%
\end{pgfscope}%
\begin{pgfscope}%
\pgfsys@transformshift{2.861025in}{1.082233in}%
\pgfsys@useobject{currentmarker}{}%
\end{pgfscope}%
\begin{pgfscope}%
\pgfsys@transformshift{3.096098in}{1.023904in}%
\pgfsys@useobject{currentmarker}{}%
\end{pgfscope}%
\begin{pgfscope}%
\pgfsys@transformshift{3.331170in}{0.963297in}%
\pgfsys@useobject{currentmarker}{}%
\end{pgfscope}%
\begin{pgfscope}%
\pgfsys@transformshift{3.566243in}{0.908393in}%
\pgfsys@useobject{currentmarker}{}%
\end{pgfscope}%
\begin{pgfscope}%
\pgfsys@transformshift{3.801316in}{0.848803in}%
\pgfsys@useobject{currentmarker}{}%
\end{pgfscope}%
\begin{pgfscope}%
\pgfsys@transformshift{4.036389in}{0.806641in}%
\pgfsys@useobject{currentmarker}{}%
\end{pgfscope}%
\end{pgfscope}%
\begin{pgfscope}%
\pgfpathrectangle{\pgfqpoint{0.745371in}{0.566590in}}{\pgfqpoint{3.291018in}{1.828724in}}%
\pgfusepath{clip}%
\pgfsetrectcap%
\pgfsetroundjoin%
\pgfsetlinewidth{1.505625pt}%
\definecolor{currentstroke}{rgb}{0.301000,0.745000,0.741000}%
\pgfsetstrokecolor{currentstroke}%
\pgfsetdash{}{0pt}%
\pgfpathmoveto{\pgfqpoint{0.745371in}{1.798229in}}%
\pgfpathlineto{\pgfqpoint{0.980444in}{1.752612in}}%
\pgfpathlineto{\pgfqpoint{1.215516in}{1.770790in}}%
\pgfpathlineto{\pgfqpoint{1.450589in}{1.717601in}}%
\pgfpathlineto{\pgfqpoint{1.685662in}{1.582168in}}%
\pgfpathlineto{\pgfqpoint{1.920734in}{1.531857in}}%
\pgfpathlineto{\pgfqpoint{2.155807in}{1.452126in}}%
\pgfpathlineto{\pgfqpoint{2.390880in}{1.390902in}}%
\pgfpathlineto{\pgfqpoint{2.625952in}{1.331715in}}%
\pgfpathlineto{\pgfqpoint{2.861025in}{1.277998in}}%
\pgfpathlineto{\pgfqpoint{3.096098in}{1.217515in}}%
\pgfpathlineto{\pgfqpoint{3.331170in}{1.155148in}}%
\pgfpathlineto{\pgfqpoint{3.566243in}{1.107097in}}%
\pgfpathlineto{\pgfqpoint{3.801316in}{1.044418in}}%
\pgfpathlineto{\pgfqpoint{4.036389in}{0.994591in}}%
\pgfusepath{stroke}%
\end{pgfscope}%
\begin{pgfscope}%
\pgfpathrectangle{\pgfqpoint{0.745371in}{0.566590in}}{\pgfqpoint{3.291018in}{1.828724in}}%
\pgfusepath{clip}%
\pgfsetbuttcap%
\pgfsetmiterjoin%
\definecolor{currentfill}{rgb}{0.000000,0.000000,0.000000}%
\pgfsetfillcolor{currentfill}%
\pgfsetfillopacity{0.000000}%
\pgfsetlinewidth{1.003750pt}%
\definecolor{currentstroke}{rgb}{0.301000,0.745000,0.741000}%
\pgfsetstrokecolor{currentstroke}%
\pgfsetdash{}{0pt}%
\pgfsys@defobject{currentmarker}{\pgfqpoint{-0.041667in}{-0.041667in}}{\pgfqpoint{0.041667in}{0.041667in}}{%
\pgfpathmoveto{\pgfqpoint{0.000000in}{0.041667in}}%
\pgfpathlineto{\pgfqpoint{-0.041667in}{-0.041667in}}%
\pgfpathlineto{\pgfqpoint{0.041667in}{-0.041667in}}%
\pgfpathclose%
\pgfusepath{stroke,fill}%
}%
\begin{pgfscope}%
\pgfsys@transformshift{0.745371in}{1.798229in}%
\pgfsys@useobject{currentmarker}{}%
\end{pgfscope}%
\begin{pgfscope}%
\pgfsys@transformshift{0.980444in}{1.752612in}%
\pgfsys@useobject{currentmarker}{}%
\end{pgfscope}%
\begin{pgfscope}%
\pgfsys@transformshift{1.215516in}{1.770790in}%
\pgfsys@useobject{currentmarker}{}%
\end{pgfscope}%
\begin{pgfscope}%
\pgfsys@transformshift{1.450589in}{1.717601in}%
\pgfsys@useobject{currentmarker}{}%
\end{pgfscope}%
\begin{pgfscope}%
\pgfsys@transformshift{1.685662in}{1.582168in}%
\pgfsys@useobject{currentmarker}{}%
\end{pgfscope}%
\begin{pgfscope}%
\pgfsys@transformshift{1.920734in}{1.531857in}%
\pgfsys@useobject{currentmarker}{}%
\end{pgfscope}%
\begin{pgfscope}%
\pgfsys@transformshift{2.155807in}{1.452126in}%
\pgfsys@useobject{currentmarker}{}%
\end{pgfscope}%
\begin{pgfscope}%
\pgfsys@transformshift{2.390880in}{1.390902in}%
\pgfsys@useobject{currentmarker}{}%
\end{pgfscope}%
\begin{pgfscope}%
\pgfsys@transformshift{2.625952in}{1.331715in}%
\pgfsys@useobject{currentmarker}{}%
\end{pgfscope}%
\begin{pgfscope}%
\pgfsys@transformshift{2.861025in}{1.277998in}%
\pgfsys@useobject{currentmarker}{}%
\end{pgfscope}%
\begin{pgfscope}%
\pgfsys@transformshift{3.096098in}{1.217515in}%
\pgfsys@useobject{currentmarker}{}%
\end{pgfscope}%
\begin{pgfscope}%
\pgfsys@transformshift{3.331170in}{1.155148in}%
\pgfsys@useobject{currentmarker}{}%
\end{pgfscope}%
\begin{pgfscope}%
\pgfsys@transformshift{3.566243in}{1.107097in}%
\pgfsys@useobject{currentmarker}{}%
\end{pgfscope}%
\begin{pgfscope}%
\pgfsys@transformshift{3.801316in}{1.044418in}%
\pgfsys@useobject{currentmarker}{}%
\end{pgfscope}%
\begin{pgfscope}%
\pgfsys@transformshift{4.036389in}{0.994591in}%
\pgfsys@useobject{currentmarker}{}%
\end{pgfscope}%
\end{pgfscope}%
\begin{pgfscope}%
\pgfsetrectcap%
\pgfsetmiterjoin%
\pgfsetlinewidth{0.803000pt}%
\definecolor{currentstroke}{rgb}{0.000000,0.000000,0.000000}%
\pgfsetstrokecolor{currentstroke}%
\pgfsetdash{}{0pt}%
\pgfpathmoveto{\pgfqpoint{0.745371in}{0.566590in}}%
\pgfpathlineto{\pgfqpoint{0.745371in}{2.395314in}}%
\pgfusepath{stroke}%
\end{pgfscope}%
\begin{pgfscope}%
\pgfsetrectcap%
\pgfsetmiterjoin%
\pgfsetlinewidth{0.803000pt}%
\definecolor{currentstroke}{rgb}{0.000000,0.000000,0.000000}%
\pgfsetstrokecolor{currentstroke}%
\pgfsetdash{}{0pt}%
\pgfpathmoveto{\pgfqpoint{4.036389in}{0.566590in}}%
\pgfpathlineto{\pgfqpoint{4.036389in}{2.395314in}}%
\pgfusepath{stroke}%
\end{pgfscope}%
\begin{pgfscope}%
\pgfsetrectcap%
\pgfsetmiterjoin%
\pgfsetlinewidth{0.803000pt}%
\definecolor{currentstroke}{rgb}{0.000000,0.000000,0.000000}%
\pgfsetstrokecolor{currentstroke}%
\pgfsetdash{}{0pt}%
\pgfpathmoveto{\pgfqpoint{0.745371in}{0.566590in}}%
\pgfpathlineto{\pgfqpoint{4.036389in}{0.566590in}}%
\pgfusepath{stroke}%
\end{pgfscope}%
\begin{pgfscope}%
\pgfsetrectcap%
\pgfsetmiterjoin%
\pgfsetlinewidth{0.803000pt}%
\definecolor{currentstroke}{rgb}{0.000000,0.000000,0.000000}%
\pgfsetstrokecolor{currentstroke}%
\pgfsetdash{}{0pt}%
\pgfpathmoveto{\pgfqpoint{0.745371in}{2.395314in}}%
\pgfpathlineto{\pgfqpoint{4.036389in}{2.395314in}}%
\pgfusepath{stroke}%
\end{pgfscope}%
\begin{pgfscope}%
\pgfsetbuttcap%
\pgfsetmiterjoin%
\definecolor{currentfill}{rgb}{1.000000,1.000000,1.000000}%
\pgfsetfillcolor{currentfill}%
\pgfsetfillopacity{0.800000}%
\pgfsetlinewidth{1.003750pt}%
\definecolor{currentstroke}{rgb}{0.800000,0.800000,0.800000}%
\pgfsetstrokecolor{currentstroke}%
\pgfsetstrokeopacity{0.800000}%
\pgfsetdash{}{0pt}%
\pgfpathmoveto{\pgfqpoint{0.832871in}{0.629090in}}%
\pgfpathlineto{\pgfqpoint{1.806671in}{0.629090in}}%
\pgfpathquadraticcurveto{\pgfqpoint{1.831671in}{0.629090in}}{\pgfqpoint{1.831671in}{0.654090in}}%
\pgfpathlineto{\pgfqpoint{1.831671in}{1.687423in}}%
\pgfpathquadraticcurveto{\pgfqpoint{1.831671in}{1.712423in}}{\pgfqpoint{1.806671in}{1.712423in}}%
\pgfpathlineto{\pgfqpoint{0.832871in}{1.712423in}}%
\pgfpathquadraticcurveto{\pgfqpoint{0.807871in}{1.712423in}}{\pgfqpoint{0.807871in}{1.687423in}}%
\pgfpathlineto{\pgfqpoint{0.807871in}{0.654090in}}%
\pgfpathquadraticcurveto{\pgfqpoint{0.807871in}{0.629090in}}{\pgfqpoint{0.832871in}{0.629090in}}%
\pgfpathclose%
\pgfusepath{stroke,fill}%
\end{pgfscope}%
\begin{pgfscope}%
\pgfsetbuttcap%
\pgfsetroundjoin%
\definecolor{currentfill}{rgb}{0.000000,0.000000,0.000000}%
\pgfsetfillcolor{currentfill}%
\pgfsetfillopacity{0.000000}%
\pgfsetlinewidth{1.003750pt}%
\definecolor{currentstroke}{rgb}{0.000000,0.447000,0.741000}%
\pgfsetstrokecolor{currentstroke}%
\pgfsetdash{}{0pt}%
\pgfsys@defobject{currentmarker}{\pgfqpoint{-0.041667in}{-0.041667in}}{\pgfqpoint{0.041667in}{0.041667in}}{%
\pgfpathmoveto{\pgfqpoint{0.000000in}{-0.041667in}}%
\pgfpathcurveto{\pgfqpoint{0.011050in}{-0.041667in}}{\pgfqpoint{0.021649in}{-0.037276in}}{\pgfqpoint{0.029463in}{-0.029463in}}%
\pgfpathcurveto{\pgfqpoint{0.037276in}{-0.021649in}}{\pgfqpoint{0.041667in}{-0.011050in}}{\pgfqpoint{0.041667in}{0.000000in}}%
\pgfpathcurveto{\pgfqpoint{0.041667in}{0.011050in}}{\pgfqpoint{0.037276in}{0.021649in}}{\pgfqpoint{0.029463in}{0.029463in}}%
\pgfpathcurveto{\pgfqpoint{0.021649in}{0.037276in}}{\pgfqpoint{0.011050in}{0.041667in}}{\pgfqpoint{0.000000in}{0.041667in}}%
\pgfpathcurveto{\pgfqpoint{-0.011050in}{0.041667in}}{\pgfqpoint{-0.021649in}{0.037276in}}{\pgfqpoint{-0.029463in}{0.029463in}}%
\pgfpathcurveto{\pgfqpoint{-0.037276in}{0.021649in}}{\pgfqpoint{-0.041667in}{0.011050in}}{\pgfqpoint{-0.041667in}{0.000000in}}%
\pgfpathcurveto{\pgfqpoint{-0.041667in}{-0.011050in}}{\pgfqpoint{-0.037276in}{-0.021649in}}{\pgfqpoint{-0.029463in}{-0.029463in}}%
\pgfpathcurveto{\pgfqpoint{-0.021649in}{-0.037276in}}{\pgfqpoint{-0.011050in}{-0.041667in}}{\pgfqpoint{0.000000in}{-0.041667in}}%
\pgfpathclose%
\pgfusepath{stroke,fill}%
}%
\begin{pgfscope}%
\pgfsys@transformshift{0.982871in}{1.618673in}%
\pgfsys@useobject{currentmarker}{}%
\end{pgfscope}%
\end{pgfscope}%
\begin{pgfscope}%
\definecolor{textcolor}{rgb}{0.000000,0.000000,0.000000}%
\pgfsetstrokecolor{textcolor}%
\pgfsetfillcolor{textcolor}%
\pgftext[x=1.207871in,y=1.574923in,left,base]{\color{textcolor}\rmfamily\fontsize{9.000000}{10.800000}\selectfont \(\displaystyle \gamma_1 = \) -0.07}%
\end{pgfscope}%
\begin{pgfscope}%
\pgfsetbuttcap%
\pgfsetroundjoin%
\definecolor{currentfill}{rgb}{0.850000,0.324000,0.098000}%
\pgfsetfillcolor{currentfill}%
\pgfsetlinewidth{1.003750pt}%
\definecolor{currentstroke}{rgb}{0.850000,0.324000,0.098000}%
\pgfsetstrokecolor{currentstroke}%
\pgfsetdash{}{0pt}%
\pgfsys@defobject{currentmarker}{\pgfqpoint{-0.041667in}{-0.041667in}}{\pgfqpoint{0.041667in}{0.041667in}}{%
\pgfpathmoveto{\pgfqpoint{-0.041667in}{0.000000in}}%
\pgfpathlineto{\pgfqpoint{0.041667in}{0.000000in}}%
\pgfpathmoveto{\pgfqpoint{0.000000in}{-0.041667in}}%
\pgfpathlineto{\pgfqpoint{0.000000in}{0.041667in}}%
\pgfusepath{stroke,fill}%
}%
\begin{pgfscope}%
\pgfsys@transformshift{0.982871in}{1.444368in}%
\pgfsys@useobject{currentmarker}{}%
\end{pgfscope}%
\end{pgfscope}%
\begin{pgfscope}%
\definecolor{textcolor}{rgb}{0.000000,0.000000,0.000000}%
\pgfsetstrokecolor{textcolor}%
\pgfsetfillcolor{textcolor}%
\pgftext[x=1.207871in,y=1.400618in,left,base]{\color{textcolor}\rmfamily\fontsize{9.000000}{10.800000}\selectfont \(\displaystyle \gamma_2 = \) -0.13}%
\end{pgfscope}%
\begin{pgfscope}%
\pgfsetbuttcap%
\pgfsetmiterjoin%
\definecolor{currentfill}{rgb}{0.000000,0.000000,0.000000}%
\pgfsetfillcolor{currentfill}%
\pgfsetfillopacity{0.000000}%
\pgfsetlinewidth{1.003750pt}%
\definecolor{currentstroke}{rgb}{0.000000,0.500000,0.000000}%
\pgfsetstrokecolor{currentstroke}%
\pgfsetdash{}{0pt}%
\pgfsys@defobject{currentmarker}{\pgfqpoint{-0.041667in}{-0.041667in}}{\pgfqpoint{0.041667in}{0.041667in}}{%
\pgfpathmoveto{\pgfqpoint{-0.041667in}{-0.041667in}}%
\pgfpathlineto{\pgfqpoint{0.041667in}{-0.041667in}}%
\pgfpathlineto{\pgfqpoint{0.041667in}{0.041667in}}%
\pgfpathlineto{\pgfqpoint{-0.041667in}{0.041667in}}%
\pgfpathclose%
\pgfusepath{stroke,fill}%
}%
\begin{pgfscope}%
\pgfsys@transformshift{0.982871in}{1.270062in}%
\pgfsys@useobject{currentmarker}{}%
\end{pgfscope}%
\end{pgfscope}%
\begin{pgfscope}%
\definecolor{textcolor}{rgb}{0.000000,0.000000,0.000000}%
\pgfsetstrokecolor{textcolor}%
\pgfsetfillcolor{textcolor}%
\pgftext[x=1.207871in,y=1.226312in,left,base]{\color{textcolor}\rmfamily\fontsize{9.000000}{10.800000}\selectfont \(\displaystyle \gamma_3 = \) -0.10}%
\end{pgfscope}%
\begin{pgfscope}%
\pgfsetbuttcap%
\pgfsetroundjoin%
\definecolor{currentfill}{rgb}{0.494000,0.184000,0.556000}%
\pgfsetfillcolor{currentfill}%
\pgfsetlinewidth{1.003750pt}%
\definecolor{currentstroke}{rgb}{0.494000,0.184000,0.556000}%
\pgfsetstrokecolor{currentstroke}%
\pgfsetdash{}{0pt}%
\pgfsys@defobject{currentmarker}{\pgfqpoint{-0.041667in}{-0.041667in}}{\pgfqpoint{0.041667in}{0.041667in}}{%
\pgfpathmoveto{\pgfqpoint{-0.041667in}{-0.041667in}}%
\pgfpathlineto{\pgfqpoint{0.041667in}{0.041667in}}%
\pgfpathmoveto{\pgfqpoint{-0.041667in}{0.041667in}}%
\pgfpathlineto{\pgfqpoint{0.041667in}{-0.041667in}}%
\pgfusepath{stroke,fill}%
}%
\begin{pgfscope}%
\pgfsys@transformshift{0.982871in}{1.095757in}%
\pgfsys@useobject{currentmarker}{}%
\end{pgfscope}%
\end{pgfscope}%
\begin{pgfscope}%
\definecolor{textcolor}{rgb}{0.000000,0.000000,0.000000}%
\pgfsetstrokecolor{textcolor}%
\pgfsetfillcolor{textcolor}%
\pgftext[x=1.207871in,y=1.052007in,left,base]{\color{textcolor}\rmfamily\fontsize{9.000000}{10.800000}\selectfont \(\displaystyle \gamma_4 = \) -0.11}%
\end{pgfscope}%
\begin{pgfscope}%
\pgfsetbuttcap%
\pgfsetmiterjoin%
\definecolor{currentfill}{rgb}{0.000000,0.000000,0.000000}%
\pgfsetfillcolor{currentfill}%
\pgfsetfillopacity{0.000000}%
\pgfsetlinewidth{1.003750pt}%
\definecolor{currentstroke}{rgb}{0.635000,0.078000,0.184000}%
\pgfsetstrokecolor{currentstroke}%
\pgfsetdash{}{0pt}%
\pgfsys@defobject{currentmarker}{\pgfqpoint{-0.035355in}{-0.058926in}}{\pgfqpoint{0.035355in}{0.058926in}}{%
\pgfpathmoveto{\pgfqpoint{-0.000000in}{-0.058926in}}%
\pgfpathlineto{\pgfqpoint{0.035355in}{0.000000in}}%
\pgfpathlineto{\pgfqpoint{0.000000in}{0.058926in}}%
\pgfpathlineto{\pgfqpoint{-0.035355in}{0.000000in}}%
\pgfpathclose%
\pgfusepath{stroke,fill}%
}%
\begin{pgfscope}%
\pgfsys@transformshift{0.982871in}{0.921451in}%
\pgfsys@useobject{currentmarker}{}%
\end{pgfscope}%
\end{pgfscope}%
\begin{pgfscope}%
\definecolor{textcolor}{rgb}{0.000000,0.000000,0.000000}%
\pgfsetstrokecolor{textcolor}%
\pgfsetfillcolor{textcolor}%
\pgftext[x=1.207871in,y=0.877701in,left,base]{\color{textcolor}\rmfamily\fontsize{9.000000}{10.800000}\selectfont \(\displaystyle \gamma_5 = \) -0.12}%
\end{pgfscope}%
\begin{pgfscope}%
\pgfsetbuttcap%
\pgfsetmiterjoin%
\definecolor{currentfill}{rgb}{0.000000,0.000000,0.000000}%
\pgfsetfillcolor{currentfill}%
\pgfsetfillopacity{0.000000}%
\pgfsetlinewidth{1.003750pt}%
\definecolor{currentstroke}{rgb}{0.301000,0.745000,0.741000}%
\pgfsetstrokecolor{currentstroke}%
\pgfsetdash{}{0pt}%
\pgfsys@defobject{currentmarker}{\pgfqpoint{-0.041667in}{-0.041667in}}{\pgfqpoint{0.041667in}{0.041667in}}{%
\pgfpathmoveto{\pgfqpoint{0.000000in}{0.041667in}}%
\pgfpathlineto{\pgfqpoint{-0.041667in}{-0.041667in}}%
\pgfpathlineto{\pgfqpoint{0.041667in}{-0.041667in}}%
\pgfpathclose%
\pgfusepath{stroke,fill}%
}%
\begin{pgfscope}%
\pgfsys@transformshift{0.982871in}{0.747146in}%
\pgfsys@useobject{currentmarker}{}%
\end{pgfscope}%
\end{pgfscope}%
\begin{pgfscope}%
\definecolor{textcolor}{rgb}{0.000000,0.000000,0.000000}%
\pgfsetstrokecolor{textcolor}%
\pgfsetfillcolor{textcolor}%
\pgftext[x=1.207871in,y=0.703396in,left,base]{\color{textcolor}\rmfamily\fontsize{9.000000}{10.800000}\selectfont \(\displaystyle \gamma_6 = \) -0.08}%
\end{pgfscope}%
\end{pgfpicture}%
\makeatother%
\endgroup%
}
					\caption{Cluster I}
					\label{SubFig:Cluster_I_real}
				\end{subfigure}
				\begin{subfigure}[h]{0.5\textwidth}
					\centering
					\resizebox{\linewidth}{!}{%% Creator: Matplotlib, PGF backend
%%
%% To include the figure in your LaTeX document, write
%%   \input{<filename>.pgf}
%%
%% Make sure the required packages are loaded in your preamble
%%   \usepackage{pgf}
%%
%% and, on pdftex
%%   \usepackage[utf8]{inputenc}\DeclareUnicodeCharacter{2212}{-}
%%
%% or, on luatex and xetex
%%   \usepackage{unicode-math}
%%
%% Figures using additional raster images can only be included by \input if
%% they are in the same directory as the main LaTeX file. For loading figures
%% from other directories you can use the `import` package
%%   \usepackage{import}
%%
%% and then include the figures with
%%   \import{<path to file>}{<filename>.pgf}
%%
%% Matplotlib used the following preamble
%%   \usepackage[utf8x]{inputenc}
%%   \usepackage[T1]{fontenc}
%%   \usepackage{amsmath,amssymb,amsfonts}
%%
\begingroup%
\makeatletter%
\begin{pgfpicture}%
\pgfpathrectangle{\pgfpointorigin}{\pgfqpoint{4.136389in}{2.495314in}}%
\pgfusepath{use as bounding box, clip}%
\begin{pgfscope}%
\pgfsetbuttcap%
\pgfsetmiterjoin%
\definecolor{currentfill}{rgb}{1.000000,1.000000,1.000000}%
\pgfsetfillcolor{currentfill}%
\pgfsetlinewidth{0.000000pt}%
\definecolor{currentstroke}{rgb}{1.000000,1.000000,1.000000}%
\pgfsetstrokecolor{currentstroke}%
\pgfsetdash{}{0pt}%
\pgfpathmoveto{\pgfqpoint{0.000000in}{0.000000in}}%
\pgfpathlineto{\pgfqpoint{4.136389in}{0.000000in}}%
\pgfpathlineto{\pgfqpoint{4.136389in}{2.495314in}}%
\pgfpathlineto{\pgfqpoint{0.000000in}{2.495314in}}%
\pgfpathclose%
\pgfusepath{fill}%
\end{pgfscope}%
\begin{pgfscope}%
\pgfsetbuttcap%
\pgfsetmiterjoin%
\definecolor{currentfill}{rgb}{1.000000,1.000000,1.000000}%
\pgfsetfillcolor{currentfill}%
\pgfsetlinewidth{0.000000pt}%
\definecolor{currentstroke}{rgb}{0.000000,0.000000,0.000000}%
\pgfsetstrokecolor{currentstroke}%
\pgfsetstrokeopacity{0.000000}%
\pgfsetdash{}{0pt}%
\pgfpathmoveto{\pgfqpoint{0.745371in}{0.566590in}}%
\pgfpathlineto{\pgfqpoint{4.036389in}{0.566590in}}%
\pgfpathlineto{\pgfqpoint{4.036389in}{2.395314in}}%
\pgfpathlineto{\pgfqpoint{0.745371in}{2.395314in}}%
\pgfpathclose%
\pgfusepath{fill}%
\end{pgfscope}%
\begin{pgfscope}%
\pgfpathrectangle{\pgfqpoint{0.745371in}{0.566590in}}{\pgfqpoint{3.291018in}{1.828724in}}%
\pgfusepath{clip}%
\pgfsetrectcap%
\pgfsetroundjoin%
\pgfsetlinewidth{0.803000pt}%
\definecolor{currentstroke}{rgb}{0.690196,0.690196,0.690196}%
\pgfsetstrokecolor{currentstroke}%
\pgfsetdash{}{0pt}%
\pgfpathmoveto{\pgfqpoint{0.745371in}{0.566590in}}%
\pgfpathlineto{\pgfqpoint{0.745371in}{2.395314in}}%
\pgfusepath{stroke}%
\end{pgfscope}%
\begin{pgfscope}%
\pgfsetbuttcap%
\pgfsetroundjoin%
\definecolor{currentfill}{rgb}{0.000000,0.000000,0.000000}%
\pgfsetfillcolor{currentfill}%
\pgfsetlinewidth{0.803000pt}%
\definecolor{currentstroke}{rgb}{0.000000,0.000000,0.000000}%
\pgfsetstrokecolor{currentstroke}%
\pgfsetdash{}{0pt}%
\pgfsys@defobject{currentmarker}{\pgfqpoint{0.000000in}{-0.048611in}}{\pgfqpoint{0.000000in}{0.000000in}}{%
\pgfpathmoveto{\pgfqpoint{0.000000in}{0.000000in}}%
\pgfpathlineto{\pgfqpoint{0.000000in}{-0.048611in}}%
\pgfusepath{stroke,fill}%
}%
\begin{pgfscope}%
\pgfsys@transformshift{0.745371in}{0.566590in}%
\pgfsys@useobject{currentmarker}{}%
\end{pgfscope}%
\end{pgfscope}%
\begin{pgfscope}%
\definecolor{textcolor}{rgb}{0.000000,0.000000,0.000000}%
\pgfsetstrokecolor{textcolor}%
\pgfsetfillcolor{textcolor}%
\pgftext[x=0.745371in,y=0.469368in,,top]{\color{textcolor}\rmfamily\fontsize{12.000000}{14.400000}\selectfont \(\displaystyle {-10}\)}%
\end{pgfscope}%
\begin{pgfscope}%
\pgfpathrectangle{\pgfqpoint{0.745371in}{0.566590in}}{\pgfqpoint{3.291018in}{1.828724in}}%
\pgfusepath{clip}%
\pgfsetrectcap%
\pgfsetroundjoin%
\pgfsetlinewidth{0.803000pt}%
\definecolor{currentstroke}{rgb}{0.690196,0.690196,0.690196}%
\pgfsetstrokecolor{currentstroke}%
\pgfsetdash{}{0pt}%
\pgfpathmoveto{\pgfqpoint{1.251681in}{0.566590in}}%
\pgfpathlineto{\pgfqpoint{1.251681in}{2.395314in}}%
\pgfusepath{stroke}%
\end{pgfscope}%
\begin{pgfscope}%
\pgfsetbuttcap%
\pgfsetroundjoin%
\definecolor{currentfill}{rgb}{0.000000,0.000000,0.000000}%
\pgfsetfillcolor{currentfill}%
\pgfsetlinewidth{0.803000pt}%
\definecolor{currentstroke}{rgb}{0.000000,0.000000,0.000000}%
\pgfsetstrokecolor{currentstroke}%
\pgfsetdash{}{0pt}%
\pgfsys@defobject{currentmarker}{\pgfqpoint{0.000000in}{-0.048611in}}{\pgfqpoint{0.000000in}{0.000000in}}{%
\pgfpathmoveto{\pgfqpoint{0.000000in}{0.000000in}}%
\pgfpathlineto{\pgfqpoint{0.000000in}{-0.048611in}}%
\pgfusepath{stroke,fill}%
}%
\begin{pgfscope}%
\pgfsys@transformshift{1.251681in}{0.566590in}%
\pgfsys@useobject{currentmarker}{}%
\end{pgfscope}%
\end{pgfscope}%
\begin{pgfscope}%
\definecolor{textcolor}{rgb}{0.000000,0.000000,0.000000}%
\pgfsetstrokecolor{textcolor}%
\pgfsetfillcolor{textcolor}%
\pgftext[x=1.251681in,y=0.469368in,,top]{\color{textcolor}\rmfamily\fontsize{12.000000}{14.400000}\selectfont \(\displaystyle {0}\)}%
\end{pgfscope}%
\begin{pgfscope}%
\pgfpathrectangle{\pgfqpoint{0.745371in}{0.566590in}}{\pgfqpoint{3.291018in}{1.828724in}}%
\pgfusepath{clip}%
\pgfsetrectcap%
\pgfsetroundjoin%
\pgfsetlinewidth{0.803000pt}%
\definecolor{currentstroke}{rgb}{0.690196,0.690196,0.690196}%
\pgfsetstrokecolor{currentstroke}%
\pgfsetdash{}{0pt}%
\pgfpathmoveto{\pgfqpoint{1.757992in}{0.566590in}}%
\pgfpathlineto{\pgfqpoint{1.757992in}{2.395314in}}%
\pgfusepath{stroke}%
\end{pgfscope}%
\begin{pgfscope}%
\pgfsetbuttcap%
\pgfsetroundjoin%
\definecolor{currentfill}{rgb}{0.000000,0.000000,0.000000}%
\pgfsetfillcolor{currentfill}%
\pgfsetlinewidth{0.803000pt}%
\definecolor{currentstroke}{rgb}{0.000000,0.000000,0.000000}%
\pgfsetstrokecolor{currentstroke}%
\pgfsetdash{}{0pt}%
\pgfsys@defobject{currentmarker}{\pgfqpoint{0.000000in}{-0.048611in}}{\pgfqpoint{0.000000in}{0.000000in}}{%
\pgfpathmoveto{\pgfqpoint{0.000000in}{0.000000in}}%
\pgfpathlineto{\pgfqpoint{0.000000in}{-0.048611in}}%
\pgfusepath{stroke,fill}%
}%
\begin{pgfscope}%
\pgfsys@transformshift{1.757992in}{0.566590in}%
\pgfsys@useobject{currentmarker}{}%
\end{pgfscope}%
\end{pgfscope}%
\begin{pgfscope}%
\definecolor{textcolor}{rgb}{0.000000,0.000000,0.000000}%
\pgfsetstrokecolor{textcolor}%
\pgfsetfillcolor{textcolor}%
\pgftext[x=1.757992in,y=0.469368in,,top]{\color{textcolor}\rmfamily\fontsize{12.000000}{14.400000}\selectfont \(\displaystyle {10}\)}%
\end{pgfscope}%
\begin{pgfscope}%
\pgfpathrectangle{\pgfqpoint{0.745371in}{0.566590in}}{\pgfqpoint{3.291018in}{1.828724in}}%
\pgfusepath{clip}%
\pgfsetrectcap%
\pgfsetroundjoin%
\pgfsetlinewidth{0.803000pt}%
\definecolor{currentstroke}{rgb}{0.690196,0.690196,0.690196}%
\pgfsetstrokecolor{currentstroke}%
\pgfsetdash{}{0pt}%
\pgfpathmoveto{\pgfqpoint{2.264302in}{0.566590in}}%
\pgfpathlineto{\pgfqpoint{2.264302in}{2.395314in}}%
\pgfusepath{stroke}%
\end{pgfscope}%
\begin{pgfscope}%
\pgfsetbuttcap%
\pgfsetroundjoin%
\definecolor{currentfill}{rgb}{0.000000,0.000000,0.000000}%
\pgfsetfillcolor{currentfill}%
\pgfsetlinewidth{0.803000pt}%
\definecolor{currentstroke}{rgb}{0.000000,0.000000,0.000000}%
\pgfsetstrokecolor{currentstroke}%
\pgfsetdash{}{0pt}%
\pgfsys@defobject{currentmarker}{\pgfqpoint{0.000000in}{-0.048611in}}{\pgfqpoint{0.000000in}{0.000000in}}{%
\pgfpathmoveto{\pgfqpoint{0.000000in}{0.000000in}}%
\pgfpathlineto{\pgfqpoint{0.000000in}{-0.048611in}}%
\pgfusepath{stroke,fill}%
}%
\begin{pgfscope}%
\pgfsys@transformshift{2.264302in}{0.566590in}%
\pgfsys@useobject{currentmarker}{}%
\end{pgfscope}%
\end{pgfscope}%
\begin{pgfscope}%
\definecolor{textcolor}{rgb}{0.000000,0.000000,0.000000}%
\pgfsetstrokecolor{textcolor}%
\pgfsetfillcolor{textcolor}%
\pgftext[x=2.264302in,y=0.469368in,,top]{\color{textcolor}\rmfamily\fontsize{12.000000}{14.400000}\selectfont \(\displaystyle {20}\)}%
\end{pgfscope}%
\begin{pgfscope}%
\pgfpathrectangle{\pgfqpoint{0.745371in}{0.566590in}}{\pgfqpoint{3.291018in}{1.828724in}}%
\pgfusepath{clip}%
\pgfsetrectcap%
\pgfsetroundjoin%
\pgfsetlinewidth{0.803000pt}%
\definecolor{currentstroke}{rgb}{0.690196,0.690196,0.690196}%
\pgfsetstrokecolor{currentstroke}%
\pgfsetdash{}{0pt}%
\pgfpathmoveto{\pgfqpoint{2.770613in}{0.566590in}}%
\pgfpathlineto{\pgfqpoint{2.770613in}{2.395314in}}%
\pgfusepath{stroke}%
\end{pgfscope}%
\begin{pgfscope}%
\pgfsetbuttcap%
\pgfsetroundjoin%
\definecolor{currentfill}{rgb}{0.000000,0.000000,0.000000}%
\pgfsetfillcolor{currentfill}%
\pgfsetlinewidth{0.803000pt}%
\definecolor{currentstroke}{rgb}{0.000000,0.000000,0.000000}%
\pgfsetstrokecolor{currentstroke}%
\pgfsetdash{}{0pt}%
\pgfsys@defobject{currentmarker}{\pgfqpoint{0.000000in}{-0.048611in}}{\pgfqpoint{0.000000in}{0.000000in}}{%
\pgfpathmoveto{\pgfqpoint{0.000000in}{0.000000in}}%
\pgfpathlineto{\pgfqpoint{0.000000in}{-0.048611in}}%
\pgfusepath{stroke,fill}%
}%
\begin{pgfscope}%
\pgfsys@transformshift{2.770613in}{0.566590in}%
\pgfsys@useobject{currentmarker}{}%
\end{pgfscope}%
\end{pgfscope}%
\begin{pgfscope}%
\definecolor{textcolor}{rgb}{0.000000,0.000000,0.000000}%
\pgfsetstrokecolor{textcolor}%
\pgfsetfillcolor{textcolor}%
\pgftext[x=2.770613in,y=0.469368in,,top]{\color{textcolor}\rmfamily\fontsize{12.000000}{14.400000}\selectfont \(\displaystyle {30}\)}%
\end{pgfscope}%
\begin{pgfscope}%
\pgfpathrectangle{\pgfqpoint{0.745371in}{0.566590in}}{\pgfqpoint{3.291018in}{1.828724in}}%
\pgfusepath{clip}%
\pgfsetrectcap%
\pgfsetroundjoin%
\pgfsetlinewidth{0.803000pt}%
\definecolor{currentstroke}{rgb}{0.690196,0.690196,0.690196}%
\pgfsetstrokecolor{currentstroke}%
\pgfsetdash{}{0pt}%
\pgfpathmoveto{\pgfqpoint{3.276923in}{0.566590in}}%
\pgfpathlineto{\pgfqpoint{3.276923in}{2.395314in}}%
\pgfusepath{stroke}%
\end{pgfscope}%
\begin{pgfscope}%
\pgfsetbuttcap%
\pgfsetroundjoin%
\definecolor{currentfill}{rgb}{0.000000,0.000000,0.000000}%
\pgfsetfillcolor{currentfill}%
\pgfsetlinewidth{0.803000pt}%
\definecolor{currentstroke}{rgb}{0.000000,0.000000,0.000000}%
\pgfsetstrokecolor{currentstroke}%
\pgfsetdash{}{0pt}%
\pgfsys@defobject{currentmarker}{\pgfqpoint{0.000000in}{-0.048611in}}{\pgfqpoint{0.000000in}{0.000000in}}{%
\pgfpathmoveto{\pgfqpoint{0.000000in}{0.000000in}}%
\pgfpathlineto{\pgfqpoint{0.000000in}{-0.048611in}}%
\pgfusepath{stroke,fill}%
}%
\begin{pgfscope}%
\pgfsys@transformshift{3.276923in}{0.566590in}%
\pgfsys@useobject{currentmarker}{}%
\end{pgfscope}%
\end{pgfscope}%
\begin{pgfscope}%
\definecolor{textcolor}{rgb}{0.000000,0.000000,0.000000}%
\pgfsetstrokecolor{textcolor}%
\pgfsetfillcolor{textcolor}%
\pgftext[x=3.276923in,y=0.469368in,,top]{\color{textcolor}\rmfamily\fontsize{12.000000}{14.400000}\selectfont \(\displaystyle {40}\)}%
\end{pgfscope}%
\begin{pgfscope}%
\pgfpathrectangle{\pgfqpoint{0.745371in}{0.566590in}}{\pgfqpoint{3.291018in}{1.828724in}}%
\pgfusepath{clip}%
\pgfsetrectcap%
\pgfsetroundjoin%
\pgfsetlinewidth{0.803000pt}%
\definecolor{currentstroke}{rgb}{0.690196,0.690196,0.690196}%
\pgfsetstrokecolor{currentstroke}%
\pgfsetdash{}{0pt}%
\pgfpathmoveto{\pgfqpoint{3.783233in}{0.566590in}}%
\pgfpathlineto{\pgfqpoint{3.783233in}{2.395314in}}%
\pgfusepath{stroke}%
\end{pgfscope}%
\begin{pgfscope}%
\pgfsetbuttcap%
\pgfsetroundjoin%
\definecolor{currentfill}{rgb}{0.000000,0.000000,0.000000}%
\pgfsetfillcolor{currentfill}%
\pgfsetlinewidth{0.803000pt}%
\definecolor{currentstroke}{rgb}{0.000000,0.000000,0.000000}%
\pgfsetstrokecolor{currentstroke}%
\pgfsetdash{}{0pt}%
\pgfsys@defobject{currentmarker}{\pgfqpoint{0.000000in}{-0.048611in}}{\pgfqpoint{0.000000in}{0.000000in}}{%
\pgfpathmoveto{\pgfqpoint{0.000000in}{0.000000in}}%
\pgfpathlineto{\pgfqpoint{0.000000in}{-0.048611in}}%
\pgfusepath{stroke,fill}%
}%
\begin{pgfscope}%
\pgfsys@transformshift{3.783233in}{0.566590in}%
\pgfsys@useobject{currentmarker}{}%
\end{pgfscope}%
\end{pgfscope}%
\begin{pgfscope}%
\definecolor{textcolor}{rgb}{0.000000,0.000000,0.000000}%
\pgfsetstrokecolor{textcolor}%
\pgfsetfillcolor{textcolor}%
\pgftext[x=3.783233in,y=0.469368in,,top]{\color{textcolor}\rmfamily\fontsize{12.000000}{14.400000}\selectfont \(\displaystyle {50}\)}%
\end{pgfscope}%
\begin{pgfscope}%
\definecolor{textcolor}{rgb}{0.000000,0.000000,0.000000}%
\pgfsetstrokecolor{textcolor}%
\pgfsetfillcolor{textcolor}%
\pgftext[x=2.390880in,y=0.266626in,,top]{\color{textcolor}\rmfamily\fontsize{12.000000}{14.400000}\selectfont SNR [dB]}%
\end{pgfscope}%
\begin{pgfscope}%
\pgfpathrectangle{\pgfqpoint{0.745371in}{0.566590in}}{\pgfqpoint{3.291018in}{1.828724in}}%
\pgfusepath{clip}%
\pgfsetrectcap%
\pgfsetroundjoin%
\pgfsetlinewidth{0.803000pt}%
\definecolor{currentstroke}{rgb}{0.690196,0.690196,0.690196}%
\pgfsetstrokecolor{currentstroke}%
\pgfsetdash{}{0pt}%
\pgfpathmoveto{\pgfqpoint{0.745371in}{0.566590in}}%
\pgfpathlineto{\pgfqpoint{4.036389in}{0.566590in}}%
\pgfusepath{stroke}%
\end{pgfscope}%
\begin{pgfscope}%
\pgfsetbuttcap%
\pgfsetroundjoin%
\definecolor{currentfill}{rgb}{0.000000,0.000000,0.000000}%
\pgfsetfillcolor{currentfill}%
\pgfsetlinewidth{0.803000pt}%
\definecolor{currentstroke}{rgb}{0.000000,0.000000,0.000000}%
\pgfsetstrokecolor{currentstroke}%
\pgfsetdash{}{0pt}%
\pgfsys@defobject{currentmarker}{\pgfqpoint{-0.048611in}{0.000000in}}{\pgfqpoint{-0.000000in}{0.000000in}}{%
\pgfpathmoveto{\pgfqpoint{-0.000000in}{0.000000in}}%
\pgfpathlineto{\pgfqpoint{-0.048611in}{0.000000in}}%
\pgfusepath{stroke,fill}%
}%
\begin{pgfscope}%
\pgfsys@transformshift{0.745371in}{0.566590in}%
\pgfsys@useobject{currentmarker}{}%
\end{pgfscope}%
\end{pgfscope}%
\begin{pgfscope}%
\definecolor{textcolor}{rgb}{0.000000,0.000000,0.000000}%
\pgfsetstrokecolor{textcolor}%
\pgfsetfillcolor{textcolor}%
\pgftext[x=0.327160in, y=0.509197in, left, base]{\color{textcolor}\rmfamily\fontsize{12.000000}{14.400000}\selectfont \(\displaystyle {10^{-4}}\)}%
\end{pgfscope}%
\begin{pgfscope}%
\pgfpathrectangle{\pgfqpoint{0.745371in}{0.566590in}}{\pgfqpoint{3.291018in}{1.828724in}}%
\pgfusepath{clip}%
\pgfsetrectcap%
\pgfsetroundjoin%
\pgfsetlinewidth{0.803000pt}%
\definecolor{currentstroke}{rgb}{0.690196,0.690196,0.690196}%
\pgfsetstrokecolor{currentstroke}%
\pgfsetdash{}{0pt}%
\pgfpathmoveto{\pgfqpoint{0.745371in}{1.104776in}}%
\pgfpathlineto{\pgfqpoint{4.036389in}{1.104776in}}%
\pgfusepath{stroke}%
\end{pgfscope}%
\begin{pgfscope}%
\pgfsetbuttcap%
\pgfsetroundjoin%
\definecolor{currentfill}{rgb}{0.000000,0.000000,0.000000}%
\pgfsetfillcolor{currentfill}%
\pgfsetlinewidth{0.803000pt}%
\definecolor{currentstroke}{rgb}{0.000000,0.000000,0.000000}%
\pgfsetstrokecolor{currentstroke}%
\pgfsetdash{}{0pt}%
\pgfsys@defobject{currentmarker}{\pgfqpoint{-0.048611in}{0.000000in}}{\pgfqpoint{-0.000000in}{0.000000in}}{%
\pgfpathmoveto{\pgfqpoint{-0.000000in}{0.000000in}}%
\pgfpathlineto{\pgfqpoint{-0.048611in}{0.000000in}}%
\pgfusepath{stroke,fill}%
}%
\begin{pgfscope}%
\pgfsys@transformshift{0.745371in}{1.104776in}%
\pgfsys@useobject{currentmarker}{}%
\end{pgfscope}%
\end{pgfscope}%
\begin{pgfscope}%
\definecolor{textcolor}{rgb}{0.000000,0.000000,0.000000}%
\pgfsetstrokecolor{textcolor}%
\pgfsetfillcolor{textcolor}%
\pgftext[x=0.327160in, y=1.047383in, left, base]{\color{textcolor}\rmfamily\fontsize{12.000000}{14.400000}\selectfont \(\displaystyle {10^{-2}}\)}%
\end{pgfscope}%
\begin{pgfscope}%
\pgfpathrectangle{\pgfqpoint{0.745371in}{0.566590in}}{\pgfqpoint{3.291018in}{1.828724in}}%
\pgfusepath{clip}%
\pgfsetrectcap%
\pgfsetroundjoin%
\pgfsetlinewidth{0.803000pt}%
\definecolor{currentstroke}{rgb}{0.690196,0.690196,0.690196}%
\pgfsetstrokecolor{currentstroke}%
\pgfsetdash{}{0pt}%
\pgfpathmoveto{\pgfqpoint{0.745371in}{1.642962in}}%
\pgfpathlineto{\pgfqpoint{4.036389in}{1.642962in}}%
\pgfusepath{stroke}%
\end{pgfscope}%
\begin{pgfscope}%
\pgfsetbuttcap%
\pgfsetroundjoin%
\definecolor{currentfill}{rgb}{0.000000,0.000000,0.000000}%
\pgfsetfillcolor{currentfill}%
\pgfsetlinewidth{0.803000pt}%
\definecolor{currentstroke}{rgb}{0.000000,0.000000,0.000000}%
\pgfsetstrokecolor{currentstroke}%
\pgfsetdash{}{0pt}%
\pgfsys@defobject{currentmarker}{\pgfqpoint{-0.048611in}{0.000000in}}{\pgfqpoint{-0.000000in}{0.000000in}}{%
\pgfpathmoveto{\pgfqpoint{-0.000000in}{0.000000in}}%
\pgfpathlineto{\pgfqpoint{-0.048611in}{0.000000in}}%
\pgfusepath{stroke,fill}%
}%
\begin{pgfscope}%
\pgfsys@transformshift{0.745371in}{1.642962in}%
\pgfsys@useobject{currentmarker}{}%
\end{pgfscope}%
\end{pgfscope}%
\begin{pgfscope}%
\definecolor{textcolor}{rgb}{0.000000,0.000000,0.000000}%
\pgfsetstrokecolor{textcolor}%
\pgfsetfillcolor{textcolor}%
\pgftext[x=0.418983in, y=1.585569in, left, base]{\color{textcolor}\rmfamily\fontsize{12.000000}{14.400000}\selectfont \(\displaystyle {10^{0}}\)}%
\end{pgfscope}%
\begin{pgfscope}%
\pgfpathrectangle{\pgfqpoint{0.745371in}{0.566590in}}{\pgfqpoint{3.291018in}{1.828724in}}%
\pgfusepath{clip}%
\pgfsetrectcap%
\pgfsetroundjoin%
\pgfsetlinewidth{0.803000pt}%
\definecolor{currentstroke}{rgb}{0.690196,0.690196,0.690196}%
\pgfsetstrokecolor{currentstroke}%
\pgfsetdash{}{0pt}%
\pgfpathmoveto{\pgfqpoint{0.745371in}{2.181148in}}%
\pgfpathlineto{\pgfqpoint{4.036389in}{2.181148in}}%
\pgfusepath{stroke}%
\end{pgfscope}%
\begin{pgfscope}%
\pgfsetbuttcap%
\pgfsetroundjoin%
\definecolor{currentfill}{rgb}{0.000000,0.000000,0.000000}%
\pgfsetfillcolor{currentfill}%
\pgfsetlinewidth{0.803000pt}%
\definecolor{currentstroke}{rgb}{0.000000,0.000000,0.000000}%
\pgfsetstrokecolor{currentstroke}%
\pgfsetdash{}{0pt}%
\pgfsys@defobject{currentmarker}{\pgfqpoint{-0.048611in}{0.000000in}}{\pgfqpoint{-0.000000in}{0.000000in}}{%
\pgfpathmoveto{\pgfqpoint{-0.000000in}{0.000000in}}%
\pgfpathlineto{\pgfqpoint{-0.048611in}{0.000000in}}%
\pgfusepath{stroke,fill}%
}%
\begin{pgfscope}%
\pgfsys@transformshift{0.745371in}{2.181148in}%
\pgfsys@useobject{currentmarker}{}%
\end{pgfscope}%
\end{pgfscope}%
\begin{pgfscope}%
\definecolor{textcolor}{rgb}{0.000000,0.000000,0.000000}%
\pgfsetstrokecolor{textcolor}%
\pgfsetfillcolor{textcolor}%
\pgftext[x=0.418983in, y=2.123755in, left, base]{\color{textcolor}\rmfamily\fontsize{12.000000}{14.400000}\selectfont \(\displaystyle {10^{2}}\)}%
\end{pgfscope}%
\begin{pgfscope}%
\definecolor{textcolor}{rgb}{0.000000,0.000000,0.000000}%
\pgfsetstrokecolor{textcolor}%
\pgfsetfillcolor{textcolor}%
\pgftext[x=0.271605in,y=1.480952in,,bottom,rotate=90.000000]{\color{textcolor}\rmfamily\fontsize{12.000000}{14.400000}\selectfont \(\displaystyle \hat{\sigma}_{\gamma}(\mathrm{SNR})\)}%
\end{pgfscope}%
\begin{pgfscope}%
\pgfpathrectangle{\pgfqpoint{0.745371in}{0.566590in}}{\pgfqpoint{3.291018in}{1.828724in}}%
\pgfusepath{clip}%
\pgfsetbuttcap%
\pgfsetroundjoin%
\pgfsetlinewidth{1.505625pt}%
\definecolor{currentstroke}{rgb}{0.000000,0.447000,0.741000}%
\pgfsetstrokecolor{currentstroke}%
\pgfsetdash{{5.550000pt}{2.400000pt}}{0.000000pt}%
\pgfpathmoveto{\pgfqpoint{0.745371in}{2.118996in}}%
\pgfpathlineto{\pgfqpoint{0.842165in}{2.126927in}}%
\pgfpathlineto{\pgfqpoint{0.938960in}{2.088497in}}%
\pgfpathlineto{\pgfqpoint{1.035755in}{2.105650in}}%
\pgfpathlineto{\pgfqpoint{1.132549in}{1.915595in}}%
\pgfpathlineto{\pgfqpoint{1.229344in}{2.037469in}}%
\pgfpathlineto{\pgfqpoint{1.326139in}{2.170005in}}%
\pgfpathlineto{\pgfqpoint{1.422933in}{2.050137in}}%
\pgfpathlineto{\pgfqpoint{1.519728in}{2.193652in}}%
\pgfpathlineto{\pgfqpoint{1.616523in}{2.133348in}}%
\pgfpathlineto{\pgfqpoint{1.713317in}{2.034835in}}%
\pgfpathlineto{\pgfqpoint{1.810112in}{2.058042in}}%
\pgfpathlineto{\pgfqpoint{1.906906in}{2.145693in}}%
\pgfpathlineto{\pgfqpoint{2.003701in}{2.157236in}}%
\pgfpathlineto{\pgfqpoint{2.100496in}{2.035021in}}%
\pgfpathlineto{\pgfqpoint{2.197290in}{2.180160in}}%
\pgfpathlineto{\pgfqpoint{2.294085in}{2.174203in}}%
\pgfpathlineto{\pgfqpoint{2.390880in}{2.265396in}}%
\pgfpathlineto{\pgfqpoint{2.487674in}{2.097029in}}%
\pgfpathlineto{\pgfqpoint{2.584469in}{1.879322in}}%
\pgfpathlineto{\pgfqpoint{2.681264in}{1.453496in}}%
\pgfpathlineto{\pgfqpoint{2.778058in}{1.411379in}}%
\pgfpathlineto{\pgfqpoint{2.874853in}{1.385953in}}%
\pgfpathlineto{\pgfqpoint{2.971648in}{1.372399in}}%
\pgfpathlineto{\pgfqpoint{3.068442in}{1.341645in}}%
\pgfpathlineto{\pgfqpoint{3.165237in}{1.314428in}}%
\pgfpathlineto{\pgfqpoint{3.262031in}{1.281422in}}%
\pgfpathlineto{\pgfqpoint{3.358826in}{1.251486in}}%
\pgfpathlineto{\pgfqpoint{3.455621in}{1.230261in}}%
\pgfpathlineto{\pgfqpoint{3.552415in}{1.201759in}}%
\pgfpathlineto{\pgfqpoint{3.649210in}{1.177942in}}%
\pgfpathlineto{\pgfqpoint{3.746005in}{1.149517in}}%
\pgfpathlineto{\pgfqpoint{3.842799in}{1.127551in}}%
\pgfpathlineto{\pgfqpoint{3.939594in}{1.092429in}}%
\pgfpathlineto{\pgfqpoint{4.036389in}{1.084716in}}%
\pgfusepath{stroke}%
\end{pgfscope}%
\begin{pgfscope}%
\pgfpathrectangle{\pgfqpoint{0.745371in}{0.566590in}}{\pgfqpoint{3.291018in}{1.828724in}}%
\pgfusepath{clip}%
\pgfsetbuttcap%
\pgfsetroundjoin%
\definecolor{currentfill}{rgb}{0.000000,0.000000,0.000000}%
\pgfsetfillcolor{currentfill}%
\pgfsetfillopacity{0.000000}%
\pgfsetlinewidth{1.003750pt}%
\definecolor{currentstroke}{rgb}{0.000000,0.447000,0.741000}%
\pgfsetstrokecolor{currentstroke}%
\pgfsetdash{}{0pt}%
\pgfsys@defobject{currentmarker}{\pgfqpoint{-0.041667in}{-0.041667in}}{\pgfqpoint{0.041667in}{0.041667in}}{%
\pgfpathmoveto{\pgfqpoint{0.000000in}{-0.041667in}}%
\pgfpathcurveto{\pgfqpoint{0.011050in}{-0.041667in}}{\pgfqpoint{0.021649in}{-0.037276in}}{\pgfqpoint{0.029463in}{-0.029463in}}%
\pgfpathcurveto{\pgfqpoint{0.037276in}{-0.021649in}}{\pgfqpoint{0.041667in}{-0.011050in}}{\pgfqpoint{0.041667in}{0.000000in}}%
\pgfpathcurveto{\pgfqpoint{0.041667in}{0.011050in}}{\pgfqpoint{0.037276in}{0.021649in}}{\pgfqpoint{0.029463in}{0.029463in}}%
\pgfpathcurveto{\pgfqpoint{0.021649in}{0.037276in}}{\pgfqpoint{0.011050in}{0.041667in}}{\pgfqpoint{0.000000in}{0.041667in}}%
\pgfpathcurveto{\pgfqpoint{-0.011050in}{0.041667in}}{\pgfqpoint{-0.021649in}{0.037276in}}{\pgfqpoint{-0.029463in}{0.029463in}}%
\pgfpathcurveto{\pgfqpoint{-0.037276in}{0.021649in}}{\pgfqpoint{-0.041667in}{0.011050in}}{\pgfqpoint{-0.041667in}{0.000000in}}%
\pgfpathcurveto{\pgfqpoint{-0.041667in}{-0.011050in}}{\pgfqpoint{-0.037276in}{-0.021649in}}{\pgfqpoint{-0.029463in}{-0.029463in}}%
\pgfpathcurveto{\pgfqpoint{-0.021649in}{-0.037276in}}{\pgfqpoint{-0.011050in}{-0.041667in}}{\pgfqpoint{0.000000in}{-0.041667in}}%
\pgfpathclose%
\pgfusepath{stroke,fill}%
}%
\begin{pgfscope}%
\pgfsys@transformshift{0.745371in}{2.118996in}%
\pgfsys@useobject{currentmarker}{}%
\end{pgfscope}%
\begin{pgfscope}%
\pgfsys@transformshift{1.132549in}{1.915595in}%
\pgfsys@useobject{currentmarker}{}%
\end{pgfscope}%
\begin{pgfscope}%
\pgfsys@transformshift{1.519728in}{2.193652in}%
\pgfsys@useobject{currentmarker}{}%
\end{pgfscope}%
\begin{pgfscope}%
\pgfsys@transformshift{1.906906in}{2.145693in}%
\pgfsys@useobject{currentmarker}{}%
\end{pgfscope}%
\begin{pgfscope}%
\pgfsys@transformshift{2.294085in}{2.174203in}%
\pgfsys@useobject{currentmarker}{}%
\end{pgfscope}%
\begin{pgfscope}%
\pgfsys@transformshift{2.681264in}{1.453496in}%
\pgfsys@useobject{currentmarker}{}%
\end{pgfscope}%
\begin{pgfscope}%
\pgfsys@transformshift{3.068442in}{1.341645in}%
\pgfsys@useobject{currentmarker}{}%
\end{pgfscope}%
\begin{pgfscope}%
\pgfsys@transformshift{3.455621in}{1.230261in}%
\pgfsys@useobject{currentmarker}{}%
\end{pgfscope}%
\begin{pgfscope}%
\pgfsys@transformshift{3.842799in}{1.127551in}%
\pgfsys@useobject{currentmarker}{}%
\end{pgfscope}%
\end{pgfscope}%
\begin{pgfscope}%
\pgfpathrectangle{\pgfqpoint{0.745371in}{0.566590in}}{\pgfqpoint{3.291018in}{1.828724in}}%
\pgfusepath{clip}%
\pgfsetbuttcap%
\pgfsetroundjoin%
\pgfsetlinewidth{1.505625pt}%
\definecolor{currentstroke}{rgb}{0.850000,0.324000,0.098000}%
\pgfsetstrokecolor{currentstroke}%
\pgfsetdash{{5.550000pt}{2.400000pt}}{0.000000pt}%
\pgfpathmoveto{\pgfqpoint{0.745371in}{2.093932in}}%
\pgfpathlineto{\pgfqpoint{0.842165in}{2.294592in}}%
\pgfpathlineto{\pgfqpoint{0.938960in}{2.186823in}}%
\pgfpathlineto{\pgfqpoint{1.035755in}{2.086713in}}%
\pgfpathlineto{\pgfqpoint{1.132549in}{2.062205in}}%
\pgfpathlineto{\pgfqpoint{1.229344in}{2.115865in}}%
\pgfpathlineto{\pgfqpoint{1.326139in}{2.100899in}}%
\pgfpathlineto{\pgfqpoint{1.422933in}{2.111167in}}%
\pgfpathlineto{\pgfqpoint{1.519728in}{2.123758in}}%
\pgfpathlineto{\pgfqpoint{1.616523in}{2.086421in}}%
\pgfpathlineto{\pgfqpoint{1.713317in}{2.104730in}}%
\pgfpathlineto{\pgfqpoint{1.810112in}{2.169190in}}%
\pgfpathlineto{\pgfqpoint{1.906906in}{2.113699in}}%
\pgfpathlineto{\pgfqpoint{2.003701in}{2.105332in}}%
\pgfpathlineto{\pgfqpoint{2.100496in}{2.013479in}}%
\pgfpathlineto{\pgfqpoint{2.197290in}{1.975826in}}%
\pgfpathlineto{\pgfqpoint{2.294085in}{1.954775in}}%
\pgfpathlineto{\pgfqpoint{2.390880in}{1.825975in}}%
\pgfpathlineto{\pgfqpoint{2.487674in}{1.870722in}}%
\pgfpathlineto{\pgfqpoint{2.584469in}{1.707089in}}%
\pgfpathlineto{\pgfqpoint{2.681264in}{1.640415in}}%
\pgfpathlineto{\pgfqpoint{2.778058in}{1.596707in}}%
\pgfpathlineto{\pgfqpoint{2.874853in}{1.568649in}}%
\pgfpathlineto{\pgfqpoint{2.971648in}{1.553031in}}%
\pgfpathlineto{\pgfqpoint{3.068442in}{1.521568in}}%
\pgfpathlineto{\pgfqpoint{3.165237in}{1.493285in}}%
\pgfpathlineto{\pgfqpoint{3.262031in}{1.467009in}}%
\pgfpathlineto{\pgfqpoint{3.358826in}{1.433272in}}%
\pgfpathlineto{\pgfqpoint{3.455621in}{1.413615in}}%
\pgfpathlineto{\pgfqpoint{3.552415in}{1.380637in}}%
\pgfpathlineto{\pgfqpoint{3.649210in}{1.360770in}}%
\pgfpathlineto{\pgfqpoint{3.746005in}{1.330011in}}%
\pgfpathlineto{\pgfqpoint{3.842799in}{1.307936in}}%
\pgfpathlineto{\pgfqpoint{3.939594in}{1.273826in}}%
\pgfpathlineto{\pgfqpoint{4.036389in}{1.265053in}}%
\pgfusepath{stroke}%
\end{pgfscope}%
\begin{pgfscope}%
\pgfpathrectangle{\pgfqpoint{0.745371in}{0.566590in}}{\pgfqpoint{3.291018in}{1.828724in}}%
\pgfusepath{clip}%
\pgfsetbuttcap%
\pgfsetroundjoin%
\definecolor{currentfill}{rgb}{0.850000,0.324000,0.098000}%
\pgfsetfillcolor{currentfill}%
\pgfsetlinewidth{1.003750pt}%
\definecolor{currentstroke}{rgb}{0.850000,0.324000,0.098000}%
\pgfsetstrokecolor{currentstroke}%
\pgfsetdash{}{0pt}%
\pgfsys@defobject{currentmarker}{\pgfqpoint{-0.041667in}{-0.041667in}}{\pgfqpoint{0.041667in}{0.041667in}}{%
\pgfpathmoveto{\pgfqpoint{-0.041667in}{0.000000in}}%
\pgfpathlineto{\pgfqpoint{0.041667in}{0.000000in}}%
\pgfpathmoveto{\pgfqpoint{0.000000in}{-0.041667in}}%
\pgfpathlineto{\pgfqpoint{0.000000in}{0.041667in}}%
\pgfusepath{stroke,fill}%
}%
\begin{pgfscope}%
\pgfsys@transformshift{0.745371in}{2.093932in}%
\pgfsys@useobject{currentmarker}{}%
\end{pgfscope}%
\begin{pgfscope}%
\pgfsys@transformshift{1.035755in}{2.086713in}%
\pgfsys@useobject{currentmarker}{}%
\end{pgfscope}%
\begin{pgfscope}%
\pgfsys@transformshift{1.326139in}{2.100899in}%
\pgfsys@useobject{currentmarker}{}%
\end{pgfscope}%
\begin{pgfscope}%
\pgfsys@transformshift{1.616523in}{2.086421in}%
\pgfsys@useobject{currentmarker}{}%
\end{pgfscope}%
\begin{pgfscope}%
\pgfsys@transformshift{1.906906in}{2.113699in}%
\pgfsys@useobject{currentmarker}{}%
\end{pgfscope}%
\begin{pgfscope}%
\pgfsys@transformshift{2.197290in}{1.975826in}%
\pgfsys@useobject{currentmarker}{}%
\end{pgfscope}%
\begin{pgfscope}%
\pgfsys@transformshift{2.487674in}{1.870722in}%
\pgfsys@useobject{currentmarker}{}%
\end{pgfscope}%
\begin{pgfscope}%
\pgfsys@transformshift{2.778058in}{1.596707in}%
\pgfsys@useobject{currentmarker}{}%
\end{pgfscope}%
\begin{pgfscope}%
\pgfsys@transformshift{3.068442in}{1.521568in}%
\pgfsys@useobject{currentmarker}{}%
\end{pgfscope}%
\begin{pgfscope}%
\pgfsys@transformshift{3.358826in}{1.433272in}%
\pgfsys@useobject{currentmarker}{}%
\end{pgfscope}%
\begin{pgfscope}%
\pgfsys@transformshift{3.649210in}{1.360770in}%
\pgfsys@useobject{currentmarker}{}%
\end{pgfscope}%
\begin{pgfscope}%
\pgfsys@transformshift{3.939594in}{1.273826in}%
\pgfsys@useobject{currentmarker}{}%
\end{pgfscope}%
\end{pgfscope}%
\begin{pgfscope}%
\pgfpathrectangle{\pgfqpoint{0.745371in}{0.566590in}}{\pgfqpoint{3.291018in}{1.828724in}}%
\pgfusepath{clip}%
\pgfsetbuttcap%
\pgfsetroundjoin%
\pgfsetlinewidth{1.505625pt}%
\definecolor{currentstroke}{rgb}{0.000000,0.500000,0.000000}%
\pgfsetstrokecolor{currentstroke}%
\pgfsetdash{{5.550000pt}{2.400000pt}}{0.000000pt}%
\pgfpathmoveto{\pgfqpoint{0.745371in}{2.136733in}}%
\pgfpathlineto{\pgfqpoint{0.842165in}{2.042535in}}%
\pgfpathlineto{\pgfqpoint{0.938960in}{2.051188in}}%
\pgfpathlineto{\pgfqpoint{1.035755in}{2.117828in}}%
\pgfpathlineto{\pgfqpoint{1.132549in}{2.112114in}}%
\pgfpathlineto{\pgfqpoint{1.229344in}{1.896425in}}%
\pgfpathlineto{\pgfqpoint{1.326139in}{2.097333in}}%
\pgfpathlineto{\pgfqpoint{1.422933in}{1.964871in}}%
\pgfpathlineto{\pgfqpoint{1.519728in}{1.998572in}}%
\pgfpathlineto{\pgfqpoint{1.616523in}{1.608828in}}%
\pgfpathlineto{\pgfqpoint{1.713317in}{1.996647in}}%
\pgfpathlineto{\pgfqpoint{1.810112in}{1.876708in}}%
\pgfpathlineto{\pgfqpoint{1.906906in}{2.031694in}}%
\pgfpathlineto{\pgfqpoint{2.003701in}{2.011122in}}%
\pgfpathlineto{\pgfqpoint{2.100496in}{1.848444in}}%
\pgfpathlineto{\pgfqpoint{2.197290in}{1.904691in}}%
\pgfpathlineto{\pgfqpoint{2.294085in}{1.964307in}}%
\pgfpathlineto{\pgfqpoint{2.390880in}{1.920672in}}%
\pgfpathlineto{\pgfqpoint{2.487674in}{1.873142in}}%
\pgfpathlineto{\pgfqpoint{2.584469in}{1.841769in}}%
\pgfpathlineto{\pgfqpoint{2.681264in}{1.717774in}}%
\pgfpathlineto{\pgfqpoint{2.778058in}{1.669369in}}%
\pgfpathlineto{\pgfqpoint{2.874853in}{1.643345in}}%
\pgfpathlineto{\pgfqpoint{2.971648in}{1.602858in}}%
\pgfpathlineto{\pgfqpoint{3.068442in}{1.570475in}}%
\pgfpathlineto{\pgfqpoint{3.165237in}{1.544009in}}%
\pgfpathlineto{\pgfqpoint{3.262031in}{1.528499in}}%
\pgfpathlineto{\pgfqpoint{3.358826in}{1.490464in}}%
\pgfpathlineto{\pgfqpoint{3.455621in}{1.466690in}}%
\pgfpathlineto{\pgfqpoint{3.552415in}{1.436056in}}%
\pgfpathlineto{\pgfqpoint{3.649210in}{1.413974in}}%
\pgfpathlineto{\pgfqpoint{3.746005in}{1.386833in}}%
\pgfpathlineto{\pgfqpoint{3.842799in}{1.364675in}}%
\pgfpathlineto{\pgfqpoint{3.939594in}{1.331023in}}%
\pgfpathlineto{\pgfqpoint{4.036389in}{1.314113in}}%
\pgfusepath{stroke}%
\end{pgfscope}%
\begin{pgfscope}%
\pgfpathrectangle{\pgfqpoint{0.745371in}{0.566590in}}{\pgfqpoint{3.291018in}{1.828724in}}%
\pgfusepath{clip}%
\pgfsetbuttcap%
\pgfsetmiterjoin%
\definecolor{currentfill}{rgb}{0.000000,0.000000,0.000000}%
\pgfsetfillcolor{currentfill}%
\pgfsetfillopacity{0.000000}%
\pgfsetlinewidth{1.003750pt}%
\definecolor{currentstroke}{rgb}{0.000000,0.500000,0.000000}%
\pgfsetstrokecolor{currentstroke}%
\pgfsetdash{}{0pt}%
\pgfsys@defobject{currentmarker}{\pgfqpoint{-0.041667in}{-0.041667in}}{\pgfqpoint{0.041667in}{0.041667in}}{%
\pgfpathmoveto{\pgfqpoint{-0.041667in}{-0.041667in}}%
\pgfpathlineto{\pgfqpoint{0.041667in}{-0.041667in}}%
\pgfpathlineto{\pgfqpoint{0.041667in}{0.041667in}}%
\pgfpathlineto{\pgfqpoint{-0.041667in}{0.041667in}}%
\pgfpathclose%
\pgfusepath{stroke,fill}%
}%
\begin{pgfscope}%
\pgfsys@transformshift{0.745371in}{2.136733in}%
\pgfsys@useobject{currentmarker}{}%
\end{pgfscope}%
\begin{pgfscope}%
\pgfsys@transformshift{1.229344in}{1.896425in}%
\pgfsys@useobject{currentmarker}{}%
\end{pgfscope}%
\begin{pgfscope}%
\pgfsys@transformshift{1.713317in}{1.996647in}%
\pgfsys@useobject{currentmarker}{}%
\end{pgfscope}%
\begin{pgfscope}%
\pgfsys@transformshift{2.197290in}{1.904691in}%
\pgfsys@useobject{currentmarker}{}%
\end{pgfscope}%
\begin{pgfscope}%
\pgfsys@transformshift{2.681264in}{1.717774in}%
\pgfsys@useobject{currentmarker}{}%
\end{pgfscope}%
\begin{pgfscope}%
\pgfsys@transformshift{3.165237in}{1.544009in}%
\pgfsys@useobject{currentmarker}{}%
\end{pgfscope}%
\begin{pgfscope}%
\pgfsys@transformshift{3.649210in}{1.413974in}%
\pgfsys@useobject{currentmarker}{}%
\end{pgfscope}%
\end{pgfscope}%
\begin{pgfscope}%
\pgfpathrectangle{\pgfqpoint{0.745371in}{0.566590in}}{\pgfqpoint{3.291018in}{1.828724in}}%
\pgfusepath{clip}%
\pgfsetbuttcap%
\pgfsetroundjoin%
\pgfsetlinewidth{1.505625pt}%
\definecolor{currentstroke}{rgb}{0.494000,0.184000,0.556000}%
\pgfsetstrokecolor{currentstroke}%
\pgfsetdash{{5.550000pt}{2.400000pt}}{0.000000pt}%
\pgfpathmoveto{\pgfqpoint{0.745371in}{2.070896in}}%
\pgfpathlineto{\pgfqpoint{0.842165in}{2.038404in}}%
\pgfpathlineto{\pgfqpoint{0.938960in}{2.191744in}}%
\pgfpathlineto{\pgfqpoint{1.035755in}{2.064434in}}%
\pgfpathlineto{\pgfqpoint{1.132549in}{2.135646in}}%
\pgfpathlineto{\pgfqpoint{1.229344in}{2.121493in}}%
\pgfpathlineto{\pgfqpoint{1.326139in}{2.039371in}}%
\pgfpathlineto{\pgfqpoint{1.422933in}{2.148947in}}%
\pgfpathlineto{\pgfqpoint{1.519728in}{2.042451in}}%
\pgfpathlineto{\pgfqpoint{1.616523in}{2.164672in}}%
\pgfpathlineto{\pgfqpoint{1.713317in}{1.863389in}}%
\pgfpathlineto{\pgfqpoint{1.810112in}{1.566804in}}%
\pgfpathlineto{\pgfqpoint{1.906906in}{1.969294in}}%
\pgfpathlineto{\pgfqpoint{2.003701in}{1.563816in}}%
\pgfpathlineto{\pgfqpoint{2.100496in}{1.707461in}}%
\pgfpathlineto{\pgfqpoint{2.197290in}{1.940847in}}%
\pgfpathlineto{\pgfqpoint{2.294085in}{1.861838in}}%
\pgfpathlineto{\pgfqpoint{2.390880in}{1.883257in}}%
\pgfpathlineto{\pgfqpoint{2.487674in}{1.870146in}}%
\pgfpathlineto{\pgfqpoint{2.584469in}{1.821000in}}%
\pgfpathlineto{\pgfqpoint{2.681264in}{1.720112in}}%
\pgfpathlineto{\pgfqpoint{2.778058in}{1.676076in}}%
\pgfpathlineto{\pgfqpoint{2.874853in}{1.653593in}}%
\pgfpathlineto{\pgfqpoint{2.971648in}{1.632221in}}%
\pgfpathlineto{\pgfqpoint{3.068442in}{1.592892in}}%
\pgfpathlineto{\pgfqpoint{3.165237in}{1.559279in}}%
\pgfpathlineto{\pgfqpoint{3.262031in}{1.547493in}}%
\pgfpathlineto{\pgfqpoint{3.358826in}{1.514430in}}%
\pgfpathlineto{\pgfqpoint{3.455621in}{1.494872in}}%
\pgfpathlineto{\pgfqpoint{3.552415in}{1.459327in}}%
\pgfpathlineto{\pgfqpoint{3.649210in}{1.438653in}}%
\pgfpathlineto{\pgfqpoint{3.746005in}{1.409618in}}%
\pgfpathlineto{\pgfqpoint{3.842799in}{1.386376in}}%
\pgfpathlineto{\pgfqpoint{3.939594in}{1.353384in}}%
\pgfpathlineto{\pgfqpoint{4.036389in}{1.338204in}}%
\pgfusepath{stroke}%
\end{pgfscope}%
\begin{pgfscope}%
\pgfpathrectangle{\pgfqpoint{0.745371in}{0.566590in}}{\pgfqpoint{3.291018in}{1.828724in}}%
\pgfusepath{clip}%
\pgfsetbuttcap%
\pgfsetroundjoin%
\definecolor{currentfill}{rgb}{0.494000,0.184000,0.556000}%
\pgfsetfillcolor{currentfill}%
\pgfsetlinewidth{1.003750pt}%
\definecolor{currentstroke}{rgb}{0.494000,0.184000,0.556000}%
\pgfsetstrokecolor{currentstroke}%
\pgfsetdash{}{0pt}%
\pgfsys@defobject{currentmarker}{\pgfqpoint{-0.041667in}{-0.041667in}}{\pgfqpoint{0.041667in}{0.041667in}}{%
\pgfpathmoveto{\pgfqpoint{-0.041667in}{-0.041667in}}%
\pgfpathlineto{\pgfqpoint{0.041667in}{0.041667in}}%
\pgfpathmoveto{\pgfqpoint{-0.041667in}{0.041667in}}%
\pgfpathlineto{\pgfqpoint{0.041667in}{-0.041667in}}%
\pgfusepath{stroke,fill}%
}%
\begin{pgfscope}%
\pgfsys@transformshift{0.745371in}{2.070896in}%
\pgfsys@useobject{currentmarker}{}%
\end{pgfscope}%
\begin{pgfscope}%
\pgfsys@transformshift{1.132549in}{2.135646in}%
\pgfsys@useobject{currentmarker}{}%
\end{pgfscope}%
\begin{pgfscope}%
\pgfsys@transformshift{1.519728in}{2.042451in}%
\pgfsys@useobject{currentmarker}{}%
\end{pgfscope}%
\begin{pgfscope}%
\pgfsys@transformshift{1.906906in}{1.969294in}%
\pgfsys@useobject{currentmarker}{}%
\end{pgfscope}%
\begin{pgfscope}%
\pgfsys@transformshift{2.294085in}{1.861838in}%
\pgfsys@useobject{currentmarker}{}%
\end{pgfscope}%
\begin{pgfscope}%
\pgfsys@transformshift{2.681264in}{1.720112in}%
\pgfsys@useobject{currentmarker}{}%
\end{pgfscope}%
\begin{pgfscope}%
\pgfsys@transformshift{3.068442in}{1.592892in}%
\pgfsys@useobject{currentmarker}{}%
\end{pgfscope}%
\begin{pgfscope}%
\pgfsys@transformshift{3.455621in}{1.494872in}%
\pgfsys@useobject{currentmarker}{}%
\end{pgfscope}%
\begin{pgfscope}%
\pgfsys@transformshift{3.842799in}{1.386376in}%
\pgfsys@useobject{currentmarker}{}%
\end{pgfscope}%
\end{pgfscope}%
\begin{pgfscope}%
\pgfpathrectangle{\pgfqpoint{0.745371in}{0.566590in}}{\pgfqpoint{3.291018in}{1.828724in}}%
\pgfusepath{clip}%
\pgfsetbuttcap%
\pgfsetroundjoin%
\pgfsetlinewidth{1.505625pt}%
\definecolor{currentstroke}{rgb}{0.635000,0.078000,0.184000}%
\pgfsetstrokecolor{currentstroke}%
\pgfsetdash{{5.550000pt}{2.400000pt}}{0.000000pt}%
\pgfpathmoveto{\pgfqpoint{0.745371in}{2.161298in}}%
\pgfpathlineto{\pgfqpoint{0.842165in}{2.132011in}}%
\pgfpathlineto{\pgfqpoint{0.938960in}{2.106056in}}%
\pgfpathlineto{\pgfqpoint{1.035755in}{2.061145in}}%
\pgfpathlineto{\pgfqpoint{1.132549in}{1.982879in}}%
\pgfpathlineto{\pgfqpoint{1.229344in}{2.171044in}}%
\pgfpathlineto{\pgfqpoint{1.326139in}{2.065144in}}%
\pgfpathlineto{\pgfqpoint{1.422933in}{2.218639in}}%
\pgfpathlineto{\pgfqpoint{1.519728in}{2.104412in}}%
\pgfpathlineto{\pgfqpoint{1.616523in}{2.183074in}}%
\pgfpathlineto{\pgfqpoint{1.713317in}{2.163511in}}%
\pgfpathlineto{\pgfqpoint{1.810112in}{1.918235in}}%
\pgfpathlineto{\pgfqpoint{1.906906in}{2.116124in}}%
\pgfpathlineto{\pgfqpoint{2.003701in}{1.773643in}}%
\pgfpathlineto{\pgfqpoint{2.100496in}{2.112794in}}%
\pgfpathlineto{\pgfqpoint{2.197290in}{2.188737in}}%
\pgfpathlineto{\pgfqpoint{2.294085in}{2.102690in}}%
\pgfpathlineto{\pgfqpoint{2.390880in}{2.115241in}}%
\pgfpathlineto{\pgfqpoint{2.487674in}{1.932915in}}%
\pgfpathlineto{\pgfqpoint{2.584469in}{1.635101in}}%
\pgfpathlineto{\pgfqpoint{2.681264in}{1.507356in}}%
\pgfpathlineto{\pgfqpoint{2.778058in}{1.476553in}}%
\pgfpathlineto{\pgfqpoint{2.874853in}{1.459713in}}%
\pgfpathlineto{\pgfqpoint{2.971648in}{1.444484in}}%
\pgfpathlineto{\pgfqpoint{3.068442in}{1.389359in}}%
\pgfpathlineto{\pgfqpoint{3.165237in}{1.360218in}}%
\pgfpathlineto{\pgfqpoint{3.262031in}{1.339418in}}%
\pgfpathlineto{\pgfqpoint{3.358826in}{1.317155in}}%
\pgfpathlineto{\pgfqpoint{3.455621in}{1.288226in}}%
\pgfpathlineto{\pgfqpoint{3.552415in}{1.263899in}}%
\pgfpathlineto{\pgfqpoint{3.649210in}{1.234335in}}%
\pgfpathlineto{\pgfqpoint{3.746005in}{1.213242in}}%
\pgfpathlineto{\pgfqpoint{3.842799in}{1.187471in}}%
\pgfpathlineto{\pgfqpoint{3.939594in}{1.157260in}}%
\pgfpathlineto{\pgfqpoint{4.036389in}{1.130714in}}%
\pgfusepath{stroke}%
\end{pgfscope}%
\begin{pgfscope}%
\pgfpathrectangle{\pgfqpoint{0.745371in}{0.566590in}}{\pgfqpoint{3.291018in}{1.828724in}}%
\pgfusepath{clip}%
\pgfsetbuttcap%
\pgfsetmiterjoin%
\definecolor{currentfill}{rgb}{0.000000,0.000000,0.000000}%
\pgfsetfillcolor{currentfill}%
\pgfsetfillopacity{0.000000}%
\pgfsetlinewidth{1.003750pt}%
\definecolor{currentstroke}{rgb}{0.635000,0.078000,0.184000}%
\pgfsetstrokecolor{currentstroke}%
\pgfsetdash{}{0pt}%
\pgfsys@defobject{currentmarker}{\pgfqpoint{-0.035355in}{-0.058926in}}{\pgfqpoint{0.035355in}{0.058926in}}{%
\pgfpathmoveto{\pgfqpoint{-0.000000in}{-0.058926in}}%
\pgfpathlineto{\pgfqpoint{0.035355in}{0.000000in}}%
\pgfpathlineto{\pgfqpoint{0.000000in}{0.058926in}}%
\pgfpathlineto{\pgfqpoint{-0.035355in}{0.000000in}}%
\pgfpathclose%
\pgfusepath{stroke,fill}%
}%
\begin{pgfscope}%
\pgfsys@transformshift{0.745371in}{2.161298in}%
\pgfsys@useobject{currentmarker}{}%
\end{pgfscope}%
\begin{pgfscope}%
\pgfsys@transformshift{1.035755in}{2.061145in}%
\pgfsys@useobject{currentmarker}{}%
\end{pgfscope}%
\begin{pgfscope}%
\pgfsys@transformshift{1.326139in}{2.065144in}%
\pgfsys@useobject{currentmarker}{}%
\end{pgfscope}%
\begin{pgfscope}%
\pgfsys@transformshift{1.616523in}{2.183074in}%
\pgfsys@useobject{currentmarker}{}%
\end{pgfscope}%
\begin{pgfscope}%
\pgfsys@transformshift{1.906906in}{2.116124in}%
\pgfsys@useobject{currentmarker}{}%
\end{pgfscope}%
\begin{pgfscope}%
\pgfsys@transformshift{2.197290in}{2.188737in}%
\pgfsys@useobject{currentmarker}{}%
\end{pgfscope}%
\begin{pgfscope}%
\pgfsys@transformshift{2.487674in}{1.932915in}%
\pgfsys@useobject{currentmarker}{}%
\end{pgfscope}%
\begin{pgfscope}%
\pgfsys@transformshift{2.778058in}{1.476553in}%
\pgfsys@useobject{currentmarker}{}%
\end{pgfscope}%
\begin{pgfscope}%
\pgfsys@transformshift{3.068442in}{1.389359in}%
\pgfsys@useobject{currentmarker}{}%
\end{pgfscope}%
\begin{pgfscope}%
\pgfsys@transformshift{3.358826in}{1.317155in}%
\pgfsys@useobject{currentmarker}{}%
\end{pgfscope}%
\begin{pgfscope}%
\pgfsys@transformshift{3.649210in}{1.234335in}%
\pgfsys@useobject{currentmarker}{}%
\end{pgfscope}%
\begin{pgfscope}%
\pgfsys@transformshift{3.939594in}{1.157260in}%
\pgfsys@useobject{currentmarker}{}%
\end{pgfscope}%
\end{pgfscope}%
\begin{pgfscope}%
\pgfpathrectangle{\pgfqpoint{0.745371in}{0.566590in}}{\pgfqpoint{3.291018in}{1.828724in}}%
\pgfusepath{clip}%
\pgfsetrectcap%
\pgfsetroundjoin%
\pgfsetlinewidth{1.505625pt}%
\definecolor{currentstroke}{rgb}{0.000000,0.447000,0.741000}%
\pgfsetstrokecolor{currentstroke}%
\pgfsetdash{}{0pt}%
\pgfpathmoveto{\pgfqpoint{0.745371in}{1.822961in}}%
\pgfpathlineto{\pgfqpoint{0.980444in}{1.850296in}}%
\pgfpathlineto{\pgfqpoint{1.215516in}{1.840663in}}%
\pgfpathlineto{\pgfqpoint{1.450589in}{1.781040in}}%
\pgfpathlineto{\pgfqpoint{1.685662in}{1.764971in}}%
\pgfpathlineto{\pgfqpoint{1.920734in}{1.750709in}}%
\pgfpathlineto{\pgfqpoint{2.155807in}{1.734232in}}%
\pgfpathlineto{\pgfqpoint{2.390880in}{1.692303in}}%
\pgfpathlineto{\pgfqpoint{2.625952in}{1.653552in}}%
\pgfpathlineto{\pgfqpoint{2.861025in}{1.460190in}}%
\pgfpathlineto{\pgfqpoint{3.096098in}{1.354707in}}%
\pgfpathlineto{\pgfqpoint{3.331170in}{1.260308in}}%
\pgfpathlineto{\pgfqpoint{3.566243in}{1.212515in}}%
\pgfpathlineto{\pgfqpoint{3.801316in}{1.141214in}}%
\pgfpathlineto{\pgfqpoint{4.036389in}{1.079466in}}%
\pgfusepath{stroke}%
\end{pgfscope}%
\begin{pgfscope}%
\pgfpathrectangle{\pgfqpoint{0.745371in}{0.566590in}}{\pgfqpoint{3.291018in}{1.828724in}}%
\pgfusepath{clip}%
\pgfsetbuttcap%
\pgfsetroundjoin%
\definecolor{currentfill}{rgb}{0.000000,0.000000,0.000000}%
\pgfsetfillcolor{currentfill}%
\pgfsetfillopacity{0.000000}%
\pgfsetlinewidth{1.003750pt}%
\definecolor{currentstroke}{rgb}{0.000000,0.447000,0.741000}%
\pgfsetstrokecolor{currentstroke}%
\pgfsetdash{}{0pt}%
\pgfsys@defobject{currentmarker}{\pgfqpoint{-0.041667in}{-0.041667in}}{\pgfqpoint{0.041667in}{0.041667in}}{%
\pgfpathmoveto{\pgfqpoint{0.000000in}{-0.041667in}}%
\pgfpathcurveto{\pgfqpoint{0.011050in}{-0.041667in}}{\pgfqpoint{0.021649in}{-0.037276in}}{\pgfqpoint{0.029463in}{-0.029463in}}%
\pgfpathcurveto{\pgfqpoint{0.037276in}{-0.021649in}}{\pgfqpoint{0.041667in}{-0.011050in}}{\pgfqpoint{0.041667in}{0.000000in}}%
\pgfpathcurveto{\pgfqpoint{0.041667in}{0.011050in}}{\pgfqpoint{0.037276in}{0.021649in}}{\pgfqpoint{0.029463in}{0.029463in}}%
\pgfpathcurveto{\pgfqpoint{0.021649in}{0.037276in}}{\pgfqpoint{0.011050in}{0.041667in}}{\pgfqpoint{0.000000in}{0.041667in}}%
\pgfpathcurveto{\pgfqpoint{-0.011050in}{0.041667in}}{\pgfqpoint{-0.021649in}{0.037276in}}{\pgfqpoint{-0.029463in}{0.029463in}}%
\pgfpathcurveto{\pgfqpoint{-0.037276in}{0.021649in}}{\pgfqpoint{-0.041667in}{0.011050in}}{\pgfqpoint{-0.041667in}{0.000000in}}%
\pgfpathcurveto{\pgfqpoint{-0.041667in}{-0.011050in}}{\pgfqpoint{-0.037276in}{-0.021649in}}{\pgfqpoint{-0.029463in}{-0.029463in}}%
\pgfpathcurveto{\pgfqpoint{-0.021649in}{-0.037276in}}{\pgfqpoint{-0.011050in}{-0.041667in}}{\pgfqpoint{0.000000in}{-0.041667in}}%
\pgfpathclose%
\pgfusepath{stroke,fill}%
}%
\begin{pgfscope}%
\pgfsys@transformshift{0.745371in}{1.822961in}%
\pgfsys@useobject{currentmarker}{}%
\end{pgfscope}%
\begin{pgfscope}%
\pgfsys@transformshift{0.980444in}{1.850296in}%
\pgfsys@useobject{currentmarker}{}%
\end{pgfscope}%
\begin{pgfscope}%
\pgfsys@transformshift{1.215516in}{1.840663in}%
\pgfsys@useobject{currentmarker}{}%
\end{pgfscope}%
\begin{pgfscope}%
\pgfsys@transformshift{1.450589in}{1.781040in}%
\pgfsys@useobject{currentmarker}{}%
\end{pgfscope}%
\begin{pgfscope}%
\pgfsys@transformshift{1.685662in}{1.764971in}%
\pgfsys@useobject{currentmarker}{}%
\end{pgfscope}%
\begin{pgfscope}%
\pgfsys@transformshift{1.920734in}{1.750709in}%
\pgfsys@useobject{currentmarker}{}%
\end{pgfscope}%
\begin{pgfscope}%
\pgfsys@transformshift{2.155807in}{1.734232in}%
\pgfsys@useobject{currentmarker}{}%
\end{pgfscope}%
\begin{pgfscope}%
\pgfsys@transformshift{2.390880in}{1.692303in}%
\pgfsys@useobject{currentmarker}{}%
\end{pgfscope}%
\begin{pgfscope}%
\pgfsys@transformshift{2.625952in}{1.653552in}%
\pgfsys@useobject{currentmarker}{}%
\end{pgfscope}%
\begin{pgfscope}%
\pgfsys@transformshift{2.861025in}{1.460190in}%
\pgfsys@useobject{currentmarker}{}%
\end{pgfscope}%
\begin{pgfscope}%
\pgfsys@transformshift{3.096098in}{1.354707in}%
\pgfsys@useobject{currentmarker}{}%
\end{pgfscope}%
\begin{pgfscope}%
\pgfsys@transformshift{3.331170in}{1.260308in}%
\pgfsys@useobject{currentmarker}{}%
\end{pgfscope}%
\begin{pgfscope}%
\pgfsys@transformshift{3.566243in}{1.212515in}%
\pgfsys@useobject{currentmarker}{}%
\end{pgfscope}%
\begin{pgfscope}%
\pgfsys@transformshift{3.801316in}{1.141214in}%
\pgfsys@useobject{currentmarker}{}%
\end{pgfscope}%
\begin{pgfscope}%
\pgfsys@transformshift{4.036389in}{1.079466in}%
\pgfsys@useobject{currentmarker}{}%
\end{pgfscope}%
\end{pgfscope}%
\begin{pgfscope}%
\pgfpathrectangle{\pgfqpoint{0.745371in}{0.566590in}}{\pgfqpoint{3.291018in}{1.828724in}}%
\pgfusepath{clip}%
\pgfsetrectcap%
\pgfsetroundjoin%
\pgfsetlinewidth{1.505625pt}%
\definecolor{currentstroke}{rgb}{0.850000,0.324000,0.098000}%
\pgfsetstrokecolor{currentstroke}%
\pgfsetdash{}{0pt}%
\pgfpathmoveto{\pgfqpoint{0.745371in}{1.842224in}}%
\pgfpathlineto{\pgfqpoint{0.980444in}{1.717060in}}%
\pgfpathlineto{\pgfqpoint{1.215516in}{1.666320in}}%
\pgfpathlineto{\pgfqpoint{1.450589in}{1.764526in}}%
\pgfpathlineto{\pgfqpoint{1.685662in}{1.733357in}}%
\pgfpathlineto{\pgfqpoint{1.920734in}{1.727144in}}%
\pgfpathlineto{\pgfqpoint{2.155807in}{1.733940in}}%
\pgfpathlineto{\pgfqpoint{2.390880in}{1.690184in}}%
\pgfpathlineto{\pgfqpoint{2.625952in}{1.682811in}}%
\pgfpathlineto{\pgfqpoint{2.861025in}{1.086879in}}%
\pgfpathlineto{\pgfqpoint{3.096098in}{0.992660in}}%
\pgfpathlineto{\pgfqpoint{3.331170in}{0.909718in}}%
\pgfpathlineto{\pgfqpoint{3.566243in}{0.855707in}}%
\pgfpathlineto{\pgfqpoint{3.801316in}{0.787734in}}%
\pgfpathlineto{\pgfqpoint{4.036389in}{0.719463in}}%
\pgfusepath{stroke}%
\end{pgfscope}%
\begin{pgfscope}%
\pgfpathrectangle{\pgfqpoint{0.745371in}{0.566590in}}{\pgfqpoint{3.291018in}{1.828724in}}%
\pgfusepath{clip}%
\pgfsetbuttcap%
\pgfsetroundjoin%
\definecolor{currentfill}{rgb}{0.850000,0.324000,0.098000}%
\pgfsetfillcolor{currentfill}%
\pgfsetlinewidth{1.003750pt}%
\definecolor{currentstroke}{rgb}{0.850000,0.324000,0.098000}%
\pgfsetstrokecolor{currentstroke}%
\pgfsetdash{}{0pt}%
\pgfsys@defobject{currentmarker}{\pgfqpoint{-0.041667in}{-0.041667in}}{\pgfqpoint{0.041667in}{0.041667in}}{%
\pgfpathmoveto{\pgfqpoint{-0.041667in}{0.000000in}}%
\pgfpathlineto{\pgfqpoint{0.041667in}{0.000000in}}%
\pgfpathmoveto{\pgfqpoint{0.000000in}{-0.041667in}}%
\pgfpathlineto{\pgfqpoint{0.000000in}{0.041667in}}%
\pgfusepath{stroke,fill}%
}%
\begin{pgfscope}%
\pgfsys@transformshift{0.745371in}{1.842224in}%
\pgfsys@useobject{currentmarker}{}%
\end{pgfscope}%
\begin{pgfscope}%
\pgfsys@transformshift{0.980444in}{1.717060in}%
\pgfsys@useobject{currentmarker}{}%
\end{pgfscope}%
\begin{pgfscope}%
\pgfsys@transformshift{1.215516in}{1.666320in}%
\pgfsys@useobject{currentmarker}{}%
\end{pgfscope}%
\begin{pgfscope}%
\pgfsys@transformshift{1.450589in}{1.764526in}%
\pgfsys@useobject{currentmarker}{}%
\end{pgfscope}%
\begin{pgfscope}%
\pgfsys@transformshift{1.685662in}{1.733357in}%
\pgfsys@useobject{currentmarker}{}%
\end{pgfscope}%
\begin{pgfscope}%
\pgfsys@transformshift{1.920734in}{1.727144in}%
\pgfsys@useobject{currentmarker}{}%
\end{pgfscope}%
\begin{pgfscope}%
\pgfsys@transformshift{2.155807in}{1.733940in}%
\pgfsys@useobject{currentmarker}{}%
\end{pgfscope}%
\begin{pgfscope}%
\pgfsys@transformshift{2.390880in}{1.690184in}%
\pgfsys@useobject{currentmarker}{}%
\end{pgfscope}%
\begin{pgfscope}%
\pgfsys@transformshift{2.625952in}{1.682811in}%
\pgfsys@useobject{currentmarker}{}%
\end{pgfscope}%
\begin{pgfscope}%
\pgfsys@transformshift{2.861025in}{1.086879in}%
\pgfsys@useobject{currentmarker}{}%
\end{pgfscope}%
\begin{pgfscope}%
\pgfsys@transformshift{3.096098in}{0.992660in}%
\pgfsys@useobject{currentmarker}{}%
\end{pgfscope}%
\begin{pgfscope}%
\pgfsys@transformshift{3.331170in}{0.909718in}%
\pgfsys@useobject{currentmarker}{}%
\end{pgfscope}%
\begin{pgfscope}%
\pgfsys@transformshift{3.566243in}{0.855707in}%
\pgfsys@useobject{currentmarker}{}%
\end{pgfscope}%
\begin{pgfscope}%
\pgfsys@transformshift{3.801316in}{0.787734in}%
\pgfsys@useobject{currentmarker}{}%
\end{pgfscope}%
\begin{pgfscope}%
\pgfsys@transformshift{4.036389in}{0.719463in}%
\pgfsys@useobject{currentmarker}{}%
\end{pgfscope}%
\end{pgfscope}%
\begin{pgfscope}%
\pgfpathrectangle{\pgfqpoint{0.745371in}{0.566590in}}{\pgfqpoint{3.291018in}{1.828724in}}%
\pgfusepath{clip}%
\pgfsetrectcap%
\pgfsetroundjoin%
\pgfsetlinewidth{1.505625pt}%
\definecolor{currentstroke}{rgb}{0.000000,0.500000,0.000000}%
\pgfsetstrokecolor{currentstroke}%
\pgfsetdash{}{0pt}%
\pgfpathmoveto{\pgfqpoint{0.745371in}{1.713598in}}%
\pgfpathlineto{\pgfqpoint{0.980444in}{1.630881in}}%
\pgfpathlineto{\pgfqpoint{1.215516in}{1.685392in}}%
\pgfpathlineto{\pgfqpoint{1.450589in}{1.672099in}}%
\pgfpathlineto{\pgfqpoint{1.685662in}{1.678011in}}%
\pgfpathlineto{\pgfqpoint{1.920734in}{1.664489in}}%
\pgfpathlineto{\pgfqpoint{2.155807in}{1.659602in}}%
\pgfpathlineto{\pgfqpoint{2.390880in}{1.680010in}}%
\pgfpathlineto{\pgfqpoint{2.625952in}{1.593418in}}%
\pgfpathlineto{\pgfqpoint{2.861025in}{1.025276in}}%
\pgfpathlineto{\pgfqpoint{3.096098in}{0.931685in}}%
\pgfpathlineto{\pgfqpoint{3.331170in}{0.854079in}}%
\pgfpathlineto{\pgfqpoint{3.566243in}{0.797669in}}%
\pgfpathlineto{\pgfqpoint{3.801316in}{0.733874in}}%
\pgfpathlineto{\pgfqpoint{4.036389in}{0.671259in}}%
\pgfusepath{stroke}%
\end{pgfscope}%
\begin{pgfscope}%
\pgfpathrectangle{\pgfqpoint{0.745371in}{0.566590in}}{\pgfqpoint{3.291018in}{1.828724in}}%
\pgfusepath{clip}%
\pgfsetbuttcap%
\pgfsetmiterjoin%
\definecolor{currentfill}{rgb}{0.000000,0.000000,0.000000}%
\pgfsetfillcolor{currentfill}%
\pgfsetfillopacity{0.000000}%
\pgfsetlinewidth{1.003750pt}%
\definecolor{currentstroke}{rgb}{0.000000,0.500000,0.000000}%
\pgfsetstrokecolor{currentstroke}%
\pgfsetdash{}{0pt}%
\pgfsys@defobject{currentmarker}{\pgfqpoint{-0.041667in}{-0.041667in}}{\pgfqpoint{0.041667in}{0.041667in}}{%
\pgfpathmoveto{\pgfqpoint{-0.041667in}{-0.041667in}}%
\pgfpathlineto{\pgfqpoint{0.041667in}{-0.041667in}}%
\pgfpathlineto{\pgfqpoint{0.041667in}{0.041667in}}%
\pgfpathlineto{\pgfqpoint{-0.041667in}{0.041667in}}%
\pgfpathclose%
\pgfusepath{stroke,fill}%
}%
\begin{pgfscope}%
\pgfsys@transformshift{0.745371in}{1.713598in}%
\pgfsys@useobject{currentmarker}{}%
\end{pgfscope}%
\begin{pgfscope}%
\pgfsys@transformshift{0.980444in}{1.630881in}%
\pgfsys@useobject{currentmarker}{}%
\end{pgfscope}%
\begin{pgfscope}%
\pgfsys@transformshift{1.215516in}{1.685392in}%
\pgfsys@useobject{currentmarker}{}%
\end{pgfscope}%
\begin{pgfscope}%
\pgfsys@transformshift{1.450589in}{1.672099in}%
\pgfsys@useobject{currentmarker}{}%
\end{pgfscope}%
\begin{pgfscope}%
\pgfsys@transformshift{1.685662in}{1.678011in}%
\pgfsys@useobject{currentmarker}{}%
\end{pgfscope}%
\begin{pgfscope}%
\pgfsys@transformshift{1.920734in}{1.664489in}%
\pgfsys@useobject{currentmarker}{}%
\end{pgfscope}%
\begin{pgfscope}%
\pgfsys@transformshift{2.155807in}{1.659602in}%
\pgfsys@useobject{currentmarker}{}%
\end{pgfscope}%
\begin{pgfscope}%
\pgfsys@transformshift{2.390880in}{1.680010in}%
\pgfsys@useobject{currentmarker}{}%
\end{pgfscope}%
\begin{pgfscope}%
\pgfsys@transformshift{2.625952in}{1.593418in}%
\pgfsys@useobject{currentmarker}{}%
\end{pgfscope}%
\begin{pgfscope}%
\pgfsys@transformshift{2.861025in}{1.025276in}%
\pgfsys@useobject{currentmarker}{}%
\end{pgfscope}%
\begin{pgfscope}%
\pgfsys@transformshift{3.096098in}{0.931685in}%
\pgfsys@useobject{currentmarker}{}%
\end{pgfscope}%
\begin{pgfscope}%
\pgfsys@transformshift{3.331170in}{0.854079in}%
\pgfsys@useobject{currentmarker}{}%
\end{pgfscope}%
\begin{pgfscope}%
\pgfsys@transformshift{3.566243in}{0.797669in}%
\pgfsys@useobject{currentmarker}{}%
\end{pgfscope}%
\begin{pgfscope}%
\pgfsys@transformshift{3.801316in}{0.733874in}%
\pgfsys@useobject{currentmarker}{}%
\end{pgfscope}%
\begin{pgfscope}%
\pgfsys@transformshift{4.036389in}{0.671259in}%
\pgfsys@useobject{currentmarker}{}%
\end{pgfscope}%
\end{pgfscope}%
\begin{pgfscope}%
\pgfpathrectangle{\pgfqpoint{0.745371in}{0.566590in}}{\pgfqpoint{3.291018in}{1.828724in}}%
\pgfusepath{clip}%
\pgfsetrectcap%
\pgfsetroundjoin%
\pgfsetlinewidth{1.505625pt}%
\definecolor{currentstroke}{rgb}{0.494000,0.184000,0.556000}%
\pgfsetstrokecolor{currentstroke}%
\pgfsetdash{}{0pt}%
\pgfpathmoveto{\pgfqpoint{0.745371in}{1.762630in}}%
\pgfpathlineto{\pgfqpoint{0.980444in}{1.771151in}}%
\pgfpathlineto{\pgfqpoint{1.215516in}{1.742352in}}%
\pgfpathlineto{\pgfqpoint{1.450589in}{1.715738in}}%
\pgfpathlineto{\pgfqpoint{1.685662in}{1.722065in}}%
\pgfpathlineto{\pgfqpoint{1.920734in}{1.714305in}}%
\pgfpathlineto{\pgfqpoint{2.155807in}{1.688673in}}%
\pgfpathlineto{\pgfqpoint{2.390880in}{1.617964in}}%
\pgfpathlineto{\pgfqpoint{2.625952in}{1.453920in}}%
\pgfpathlineto{\pgfqpoint{2.861025in}{1.283660in}}%
\pgfpathlineto{\pgfqpoint{3.096098in}{1.195412in}}%
\pgfpathlineto{\pgfqpoint{3.331170in}{1.103231in}}%
\pgfpathlineto{\pgfqpoint{3.566243in}{1.057231in}}%
\pgfpathlineto{\pgfqpoint{3.801316in}{0.986745in}}%
\pgfpathlineto{\pgfqpoint{4.036389in}{0.923357in}}%
\pgfusepath{stroke}%
\end{pgfscope}%
\begin{pgfscope}%
\pgfpathrectangle{\pgfqpoint{0.745371in}{0.566590in}}{\pgfqpoint{3.291018in}{1.828724in}}%
\pgfusepath{clip}%
\pgfsetbuttcap%
\pgfsetroundjoin%
\definecolor{currentfill}{rgb}{0.494000,0.184000,0.556000}%
\pgfsetfillcolor{currentfill}%
\pgfsetlinewidth{1.003750pt}%
\definecolor{currentstroke}{rgb}{0.494000,0.184000,0.556000}%
\pgfsetstrokecolor{currentstroke}%
\pgfsetdash{}{0pt}%
\pgfsys@defobject{currentmarker}{\pgfqpoint{-0.041667in}{-0.041667in}}{\pgfqpoint{0.041667in}{0.041667in}}{%
\pgfpathmoveto{\pgfqpoint{-0.041667in}{-0.041667in}}%
\pgfpathlineto{\pgfqpoint{0.041667in}{0.041667in}}%
\pgfpathmoveto{\pgfqpoint{-0.041667in}{0.041667in}}%
\pgfpathlineto{\pgfqpoint{0.041667in}{-0.041667in}}%
\pgfusepath{stroke,fill}%
}%
\begin{pgfscope}%
\pgfsys@transformshift{0.745371in}{1.762630in}%
\pgfsys@useobject{currentmarker}{}%
\end{pgfscope}%
\begin{pgfscope}%
\pgfsys@transformshift{0.980444in}{1.771151in}%
\pgfsys@useobject{currentmarker}{}%
\end{pgfscope}%
\begin{pgfscope}%
\pgfsys@transformshift{1.215516in}{1.742352in}%
\pgfsys@useobject{currentmarker}{}%
\end{pgfscope}%
\begin{pgfscope}%
\pgfsys@transformshift{1.450589in}{1.715738in}%
\pgfsys@useobject{currentmarker}{}%
\end{pgfscope}%
\begin{pgfscope}%
\pgfsys@transformshift{1.685662in}{1.722065in}%
\pgfsys@useobject{currentmarker}{}%
\end{pgfscope}%
\begin{pgfscope}%
\pgfsys@transformshift{1.920734in}{1.714305in}%
\pgfsys@useobject{currentmarker}{}%
\end{pgfscope}%
\begin{pgfscope}%
\pgfsys@transformshift{2.155807in}{1.688673in}%
\pgfsys@useobject{currentmarker}{}%
\end{pgfscope}%
\begin{pgfscope}%
\pgfsys@transformshift{2.390880in}{1.617964in}%
\pgfsys@useobject{currentmarker}{}%
\end{pgfscope}%
\begin{pgfscope}%
\pgfsys@transformshift{2.625952in}{1.453920in}%
\pgfsys@useobject{currentmarker}{}%
\end{pgfscope}%
\begin{pgfscope}%
\pgfsys@transformshift{2.861025in}{1.283660in}%
\pgfsys@useobject{currentmarker}{}%
\end{pgfscope}%
\begin{pgfscope}%
\pgfsys@transformshift{3.096098in}{1.195412in}%
\pgfsys@useobject{currentmarker}{}%
\end{pgfscope}%
\begin{pgfscope}%
\pgfsys@transformshift{3.331170in}{1.103231in}%
\pgfsys@useobject{currentmarker}{}%
\end{pgfscope}%
\begin{pgfscope}%
\pgfsys@transformshift{3.566243in}{1.057231in}%
\pgfsys@useobject{currentmarker}{}%
\end{pgfscope}%
\begin{pgfscope}%
\pgfsys@transformshift{3.801316in}{0.986745in}%
\pgfsys@useobject{currentmarker}{}%
\end{pgfscope}%
\begin{pgfscope}%
\pgfsys@transformshift{4.036389in}{0.923357in}%
\pgfsys@useobject{currentmarker}{}%
\end{pgfscope}%
\end{pgfscope}%
\begin{pgfscope}%
\pgfpathrectangle{\pgfqpoint{0.745371in}{0.566590in}}{\pgfqpoint{3.291018in}{1.828724in}}%
\pgfusepath{clip}%
\pgfsetrectcap%
\pgfsetroundjoin%
\pgfsetlinewidth{1.505625pt}%
\definecolor{currentstroke}{rgb}{0.635000,0.078000,0.184000}%
\pgfsetstrokecolor{currentstroke}%
\pgfsetdash{}{0pt}%
\pgfpathmoveto{\pgfqpoint{0.745371in}{1.811412in}}%
\pgfpathlineto{\pgfqpoint{0.980444in}{1.803444in}}%
\pgfpathlineto{\pgfqpoint{1.215516in}{1.714731in}}%
\pgfpathlineto{\pgfqpoint{1.450589in}{1.574658in}}%
\pgfpathlineto{\pgfqpoint{1.685662in}{1.624212in}}%
\pgfpathlineto{\pgfqpoint{1.920734in}{1.621122in}}%
\pgfpathlineto{\pgfqpoint{2.155807in}{1.597420in}}%
\pgfpathlineto{\pgfqpoint{2.390880in}{1.519686in}}%
\pgfpathlineto{\pgfqpoint{2.625952in}{1.375963in}}%
\pgfpathlineto{\pgfqpoint{2.861025in}{1.144193in}}%
\pgfpathlineto{\pgfqpoint{3.096098in}{1.059378in}}%
\pgfpathlineto{\pgfqpoint{3.331170in}{0.982245in}}%
\pgfpathlineto{\pgfqpoint{3.566243in}{0.929035in}}%
\pgfpathlineto{\pgfqpoint{3.801316in}{0.858803in}}%
\pgfpathlineto{\pgfqpoint{4.036389in}{0.797143in}}%
\pgfusepath{stroke}%
\end{pgfscope}%
\begin{pgfscope}%
\pgfpathrectangle{\pgfqpoint{0.745371in}{0.566590in}}{\pgfqpoint{3.291018in}{1.828724in}}%
\pgfusepath{clip}%
\pgfsetbuttcap%
\pgfsetmiterjoin%
\definecolor{currentfill}{rgb}{0.000000,0.000000,0.000000}%
\pgfsetfillcolor{currentfill}%
\pgfsetfillopacity{0.000000}%
\pgfsetlinewidth{1.003750pt}%
\definecolor{currentstroke}{rgb}{0.635000,0.078000,0.184000}%
\pgfsetstrokecolor{currentstroke}%
\pgfsetdash{}{0pt}%
\pgfsys@defobject{currentmarker}{\pgfqpoint{-0.035355in}{-0.058926in}}{\pgfqpoint{0.035355in}{0.058926in}}{%
\pgfpathmoveto{\pgfqpoint{-0.000000in}{-0.058926in}}%
\pgfpathlineto{\pgfqpoint{0.035355in}{0.000000in}}%
\pgfpathlineto{\pgfqpoint{0.000000in}{0.058926in}}%
\pgfpathlineto{\pgfqpoint{-0.035355in}{0.000000in}}%
\pgfpathclose%
\pgfusepath{stroke,fill}%
}%
\begin{pgfscope}%
\pgfsys@transformshift{0.745371in}{1.811412in}%
\pgfsys@useobject{currentmarker}{}%
\end{pgfscope}%
\begin{pgfscope}%
\pgfsys@transformshift{0.980444in}{1.803444in}%
\pgfsys@useobject{currentmarker}{}%
\end{pgfscope}%
\begin{pgfscope}%
\pgfsys@transformshift{1.215516in}{1.714731in}%
\pgfsys@useobject{currentmarker}{}%
\end{pgfscope}%
\begin{pgfscope}%
\pgfsys@transformshift{1.450589in}{1.574658in}%
\pgfsys@useobject{currentmarker}{}%
\end{pgfscope}%
\begin{pgfscope}%
\pgfsys@transformshift{1.685662in}{1.624212in}%
\pgfsys@useobject{currentmarker}{}%
\end{pgfscope}%
\begin{pgfscope}%
\pgfsys@transformshift{1.920734in}{1.621122in}%
\pgfsys@useobject{currentmarker}{}%
\end{pgfscope}%
\begin{pgfscope}%
\pgfsys@transformshift{2.155807in}{1.597420in}%
\pgfsys@useobject{currentmarker}{}%
\end{pgfscope}%
\begin{pgfscope}%
\pgfsys@transformshift{2.390880in}{1.519686in}%
\pgfsys@useobject{currentmarker}{}%
\end{pgfscope}%
\begin{pgfscope}%
\pgfsys@transformshift{2.625952in}{1.375963in}%
\pgfsys@useobject{currentmarker}{}%
\end{pgfscope}%
\begin{pgfscope}%
\pgfsys@transformshift{2.861025in}{1.144193in}%
\pgfsys@useobject{currentmarker}{}%
\end{pgfscope}%
\begin{pgfscope}%
\pgfsys@transformshift{3.096098in}{1.059378in}%
\pgfsys@useobject{currentmarker}{}%
\end{pgfscope}%
\begin{pgfscope}%
\pgfsys@transformshift{3.331170in}{0.982245in}%
\pgfsys@useobject{currentmarker}{}%
\end{pgfscope}%
\begin{pgfscope}%
\pgfsys@transformshift{3.566243in}{0.929035in}%
\pgfsys@useobject{currentmarker}{}%
\end{pgfscope}%
\begin{pgfscope}%
\pgfsys@transformshift{3.801316in}{0.858803in}%
\pgfsys@useobject{currentmarker}{}%
\end{pgfscope}%
\begin{pgfscope}%
\pgfsys@transformshift{4.036389in}{0.797143in}%
\pgfsys@useobject{currentmarker}{}%
\end{pgfscope}%
\end{pgfscope}%
\begin{pgfscope}%
\pgfsetrectcap%
\pgfsetmiterjoin%
\pgfsetlinewidth{0.803000pt}%
\definecolor{currentstroke}{rgb}{0.000000,0.000000,0.000000}%
\pgfsetstrokecolor{currentstroke}%
\pgfsetdash{}{0pt}%
\pgfpathmoveto{\pgfqpoint{0.745371in}{0.566590in}}%
\pgfpathlineto{\pgfqpoint{0.745371in}{2.395314in}}%
\pgfusepath{stroke}%
\end{pgfscope}%
\begin{pgfscope}%
\pgfsetrectcap%
\pgfsetmiterjoin%
\pgfsetlinewidth{0.803000pt}%
\definecolor{currentstroke}{rgb}{0.000000,0.000000,0.000000}%
\pgfsetstrokecolor{currentstroke}%
\pgfsetdash{}{0pt}%
\pgfpathmoveto{\pgfqpoint{4.036389in}{0.566590in}}%
\pgfpathlineto{\pgfqpoint{4.036389in}{2.395314in}}%
\pgfusepath{stroke}%
\end{pgfscope}%
\begin{pgfscope}%
\pgfsetrectcap%
\pgfsetmiterjoin%
\pgfsetlinewidth{0.803000pt}%
\definecolor{currentstroke}{rgb}{0.000000,0.000000,0.000000}%
\pgfsetstrokecolor{currentstroke}%
\pgfsetdash{}{0pt}%
\pgfpathmoveto{\pgfqpoint{0.745371in}{0.566590in}}%
\pgfpathlineto{\pgfqpoint{4.036389in}{0.566590in}}%
\pgfusepath{stroke}%
\end{pgfscope}%
\begin{pgfscope}%
\pgfsetrectcap%
\pgfsetmiterjoin%
\pgfsetlinewidth{0.803000pt}%
\definecolor{currentstroke}{rgb}{0.000000,0.000000,0.000000}%
\pgfsetstrokecolor{currentstroke}%
\pgfsetdash{}{0pt}%
\pgfpathmoveto{\pgfqpoint{0.745371in}{2.395314in}}%
\pgfpathlineto{\pgfqpoint{4.036389in}{2.395314in}}%
\pgfusepath{stroke}%
\end{pgfscope}%
\begin{pgfscope}%
\pgfsetbuttcap%
\pgfsetmiterjoin%
\definecolor{currentfill}{rgb}{1.000000,1.000000,1.000000}%
\pgfsetfillcolor{currentfill}%
\pgfsetfillopacity{0.800000}%
\pgfsetlinewidth{1.003750pt}%
\definecolor{currentstroke}{rgb}{0.800000,0.800000,0.800000}%
\pgfsetstrokecolor{currentstroke}%
\pgfsetstrokeopacity{0.800000}%
\pgfsetdash{}{0pt}%
\pgfpathmoveto{\pgfqpoint{0.832871in}{0.629090in}}%
\pgfpathlineto{\pgfqpoint{1.857596in}{0.629090in}}%
\pgfpathquadraticcurveto{\pgfqpoint{1.882596in}{0.629090in}}{\pgfqpoint{1.882596in}{0.654090in}}%
\pgfpathlineto{\pgfqpoint{1.882596in}{1.513118in}}%
\pgfpathquadraticcurveto{\pgfqpoint{1.882596in}{1.538118in}}{\pgfqpoint{1.857596in}{1.538118in}}%
\pgfpathlineto{\pgfqpoint{0.832871in}{1.538118in}}%
\pgfpathquadraticcurveto{\pgfqpoint{0.807871in}{1.538118in}}{\pgfqpoint{0.807871in}{1.513118in}}%
\pgfpathlineto{\pgfqpoint{0.807871in}{0.654090in}}%
\pgfpathquadraticcurveto{\pgfqpoint{0.807871in}{0.629090in}}{\pgfqpoint{0.832871in}{0.629090in}}%
\pgfpathclose%
\pgfusepath{stroke,fill}%
\end{pgfscope}%
\begin{pgfscope}%
\pgfsetbuttcap%
\pgfsetroundjoin%
\definecolor{currentfill}{rgb}{0.000000,0.000000,0.000000}%
\pgfsetfillcolor{currentfill}%
\pgfsetfillopacity{0.000000}%
\pgfsetlinewidth{1.003750pt}%
\definecolor{currentstroke}{rgb}{0.000000,0.447000,0.741000}%
\pgfsetstrokecolor{currentstroke}%
\pgfsetdash{}{0pt}%
\pgfsys@defobject{currentmarker}{\pgfqpoint{-0.041667in}{-0.041667in}}{\pgfqpoint{0.041667in}{0.041667in}}{%
\pgfpathmoveto{\pgfqpoint{0.000000in}{-0.041667in}}%
\pgfpathcurveto{\pgfqpoint{0.011050in}{-0.041667in}}{\pgfqpoint{0.021649in}{-0.037276in}}{\pgfqpoint{0.029463in}{-0.029463in}}%
\pgfpathcurveto{\pgfqpoint{0.037276in}{-0.021649in}}{\pgfqpoint{0.041667in}{-0.011050in}}{\pgfqpoint{0.041667in}{0.000000in}}%
\pgfpathcurveto{\pgfqpoint{0.041667in}{0.011050in}}{\pgfqpoint{0.037276in}{0.021649in}}{\pgfqpoint{0.029463in}{0.029463in}}%
\pgfpathcurveto{\pgfqpoint{0.021649in}{0.037276in}}{\pgfqpoint{0.011050in}{0.041667in}}{\pgfqpoint{0.000000in}{0.041667in}}%
\pgfpathcurveto{\pgfqpoint{-0.011050in}{0.041667in}}{\pgfqpoint{-0.021649in}{0.037276in}}{\pgfqpoint{-0.029463in}{0.029463in}}%
\pgfpathcurveto{\pgfqpoint{-0.037276in}{0.021649in}}{\pgfqpoint{-0.041667in}{0.011050in}}{\pgfqpoint{-0.041667in}{0.000000in}}%
\pgfpathcurveto{\pgfqpoint{-0.041667in}{-0.011050in}}{\pgfqpoint{-0.037276in}{-0.021649in}}{\pgfqpoint{-0.029463in}{-0.029463in}}%
\pgfpathcurveto{\pgfqpoint{-0.021649in}{-0.037276in}}{\pgfqpoint{-0.011050in}{-0.041667in}}{\pgfqpoint{0.000000in}{-0.041667in}}%
\pgfpathclose%
\pgfusepath{stroke,fill}%
}%
\begin{pgfscope}%
\pgfsys@transformshift{0.982871in}{1.444368in}%
\pgfsys@useobject{currentmarker}{}%
\end{pgfscope}%
\end{pgfscope}%
\begin{pgfscope}%
\definecolor{textcolor}{rgb}{0.000000,0.000000,0.000000}%
\pgfsetstrokecolor{textcolor}%
\pgfsetfillcolor{textcolor}%
\pgftext[x=1.207871in,y=1.400618in,left,base]{\color{textcolor}\rmfamily\fontsize{9.000000}{10.800000}\selectfont \(\displaystyle \gamma_7 = \) -0.11}%
\end{pgfscope}%
\begin{pgfscope}%
\pgfsetbuttcap%
\pgfsetroundjoin%
\definecolor{currentfill}{rgb}{0.850000,0.324000,0.098000}%
\pgfsetfillcolor{currentfill}%
\pgfsetlinewidth{1.003750pt}%
\definecolor{currentstroke}{rgb}{0.850000,0.324000,0.098000}%
\pgfsetstrokecolor{currentstroke}%
\pgfsetdash{}{0pt}%
\pgfsys@defobject{currentmarker}{\pgfqpoint{-0.041667in}{-0.041667in}}{\pgfqpoint{0.041667in}{0.041667in}}{%
\pgfpathmoveto{\pgfqpoint{-0.041667in}{0.000000in}}%
\pgfpathlineto{\pgfqpoint{0.041667in}{0.000000in}}%
\pgfpathmoveto{\pgfqpoint{0.000000in}{-0.041667in}}%
\pgfpathlineto{\pgfqpoint{0.000000in}{0.041667in}}%
\pgfusepath{stroke,fill}%
}%
\begin{pgfscope}%
\pgfsys@transformshift{0.982871in}{1.270062in}%
\pgfsys@useobject{currentmarker}{}%
\end{pgfscope}%
\end{pgfscope}%
\begin{pgfscope}%
\definecolor{textcolor}{rgb}{0.000000,0.000000,0.000000}%
\pgfsetstrokecolor{textcolor}%
\pgfsetfillcolor{textcolor}%
\pgftext[x=1.207871in,y=1.226312in,left,base]{\color{textcolor}\rmfamily\fontsize{9.000000}{10.800000}\selectfont \(\displaystyle \gamma_8 = \) -0.13}%
\end{pgfscope}%
\begin{pgfscope}%
\pgfsetbuttcap%
\pgfsetmiterjoin%
\definecolor{currentfill}{rgb}{0.000000,0.000000,0.000000}%
\pgfsetfillcolor{currentfill}%
\pgfsetfillopacity{0.000000}%
\pgfsetlinewidth{1.003750pt}%
\definecolor{currentstroke}{rgb}{0.000000,0.500000,0.000000}%
\pgfsetstrokecolor{currentstroke}%
\pgfsetdash{}{0pt}%
\pgfsys@defobject{currentmarker}{\pgfqpoint{-0.041667in}{-0.041667in}}{\pgfqpoint{0.041667in}{0.041667in}}{%
\pgfpathmoveto{\pgfqpoint{-0.041667in}{-0.041667in}}%
\pgfpathlineto{\pgfqpoint{0.041667in}{-0.041667in}}%
\pgfpathlineto{\pgfqpoint{0.041667in}{0.041667in}}%
\pgfpathlineto{\pgfqpoint{-0.041667in}{0.041667in}}%
\pgfpathclose%
\pgfusepath{stroke,fill}%
}%
\begin{pgfscope}%
\pgfsys@transformshift{0.982871in}{1.095757in}%
\pgfsys@useobject{currentmarker}{}%
\end{pgfscope}%
\end{pgfscope}%
\begin{pgfscope}%
\definecolor{textcolor}{rgb}{0.000000,0.000000,0.000000}%
\pgfsetstrokecolor{textcolor}%
\pgfsetfillcolor{textcolor}%
\pgftext[x=1.207871in,y=1.052007in,left,base]{\color{textcolor}\rmfamily\fontsize{9.000000}{10.800000}\selectfont \(\displaystyle \gamma_9 = \) -0.20}%
\end{pgfscope}%
\begin{pgfscope}%
\pgfsetbuttcap%
\pgfsetroundjoin%
\definecolor{currentfill}{rgb}{0.494000,0.184000,0.556000}%
\pgfsetfillcolor{currentfill}%
\pgfsetlinewidth{1.003750pt}%
\definecolor{currentstroke}{rgb}{0.494000,0.184000,0.556000}%
\pgfsetstrokecolor{currentstroke}%
\pgfsetdash{}{0pt}%
\pgfsys@defobject{currentmarker}{\pgfqpoint{-0.041667in}{-0.041667in}}{\pgfqpoint{0.041667in}{0.041667in}}{%
\pgfpathmoveto{\pgfqpoint{-0.041667in}{-0.041667in}}%
\pgfpathlineto{\pgfqpoint{0.041667in}{0.041667in}}%
\pgfpathmoveto{\pgfqpoint{-0.041667in}{0.041667in}}%
\pgfpathlineto{\pgfqpoint{0.041667in}{-0.041667in}}%
\pgfusepath{stroke,fill}%
}%
\begin{pgfscope}%
\pgfsys@transformshift{0.982871in}{0.921451in}%
\pgfsys@useobject{currentmarker}{}%
\end{pgfscope}%
\end{pgfscope}%
\begin{pgfscope}%
\definecolor{textcolor}{rgb}{0.000000,0.000000,0.000000}%
\pgfsetstrokecolor{textcolor}%
\pgfsetfillcolor{textcolor}%
\pgftext[x=1.207871in,y=0.877701in,left,base]{\color{textcolor}\rmfamily\fontsize{9.000000}{10.800000}\selectfont \(\displaystyle \gamma_{10} = \) -0.16}%
\end{pgfscope}%
\begin{pgfscope}%
\pgfsetbuttcap%
\pgfsetmiterjoin%
\definecolor{currentfill}{rgb}{0.000000,0.000000,0.000000}%
\pgfsetfillcolor{currentfill}%
\pgfsetfillopacity{0.000000}%
\pgfsetlinewidth{1.003750pt}%
\definecolor{currentstroke}{rgb}{0.635000,0.078000,0.184000}%
\pgfsetstrokecolor{currentstroke}%
\pgfsetdash{}{0pt}%
\pgfsys@defobject{currentmarker}{\pgfqpoint{-0.035355in}{-0.058926in}}{\pgfqpoint{0.035355in}{0.058926in}}{%
\pgfpathmoveto{\pgfqpoint{-0.000000in}{-0.058926in}}%
\pgfpathlineto{\pgfqpoint{0.035355in}{0.000000in}}%
\pgfpathlineto{\pgfqpoint{0.000000in}{0.058926in}}%
\pgfpathlineto{\pgfqpoint{-0.035355in}{0.000000in}}%
\pgfpathclose%
\pgfusepath{stroke,fill}%
}%
\begin{pgfscope}%
\pgfsys@transformshift{0.982871in}{0.747146in}%
\pgfsys@useobject{currentmarker}{}%
\end{pgfscope}%
\end{pgfscope}%
\begin{pgfscope}%
\definecolor{textcolor}{rgb}{0.000000,0.000000,0.000000}%
\pgfsetstrokecolor{textcolor}%
\pgfsetfillcolor{textcolor}%
\pgftext[x=1.207871in,y=0.703396in,left,base]{\color{textcolor}\rmfamily\fontsize{9.000000}{10.800000}\selectfont \(\displaystyle \gamma_{11} = \) -0.19}%
\end{pgfscope}%
\end{pgfpicture}%
\makeatother%
\endgroup%
}
					\caption{Cluster II}
					\label{SubFig:Cluster_II_real}
				\end{subfigure}
				\begin{subfigure}[h]{0.5\textwidth}
					\centering
					\resizebox{\linewidth}{!}{%% Creator: Matplotlib, PGF backend
%%
%% To include the figure in your LaTeX document, write
%%   \input{<filename>.pgf}
%%
%% Make sure the required packages are loaded in your preamble
%%   \usepackage{pgf}
%%
%% and, on pdftex
%%   \usepackage[utf8]{inputenc}\DeclareUnicodeCharacter{2212}{-}
%%
%% or, on luatex and xetex
%%   \usepackage{unicode-math}
%%
%% Figures using additional raster images can only be included by \input if
%% they are in the same directory as the main LaTeX file. For loading figures
%% from other directories you can use the `import` package
%%   \usepackage{import}
%%
%% and then include the figures with
%%   \import{<path to file>}{<filename>.pgf}
%%
%% Matplotlib used the following preamble
%%   \usepackage[utf8x]{inputenc}
%%   \usepackage[T1]{fontenc}
%%   \usepackage{amsmath,amssymb,amsfonts}
%%
\begingroup%
\makeatletter%
\begin{pgfpicture}%
\pgfpathrectangle{\pgfpointorigin}{\pgfqpoint{4.136389in}{2.495314in}}%
\pgfusepath{use as bounding box, clip}%
\begin{pgfscope}%
\pgfsetbuttcap%
\pgfsetmiterjoin%
\definecolor{currentfill}{rgb}{1.000000,1.000000,1.000000}%
\pgfsetfillcolor{currentfill}%
\pgfsetlinewidth{0.000000pt}%
\definecolor{currentstroke}{rgb}{1.000000,1.000000,1.000000}%
\pgfsetstrokecolor{currentstroke}%
\pgfsetdash{}{0pt}%
\pgfpathmoveto{\pgfqpoint{0.000000in}{0.000000in}}%
\pgfpathlineto{\pgfqpoint{4.136389in}{0.000000in}}%
\pgfpathlineto{\pgfqpoint{4.136389in}{2.495314in}}%
\pgfpathlineto{\pgfqpoint{0.000000in}{2.495314in}}%
\pgfpathclose%
\pgfusepath{fill}%
\end{pgfscope}%
\begin{pgfscope}%
\pgfsetbuttcap%
\pgfsetmiterjoin%
\definecolor{currentfill}{rgb}{1.000000,1.000000,1.000000}%
\pgfsetfillcolor{currentfill}%
\pgfsetlinewidth{0.000000pt}%
\definecolor{currentstroke}{rgb}{0.000000,0.000000,0.000000}%
\pgfsetstrokecolor{currentstroke}%
\pgfsetstrokeopacity{0.000000}%
\pgfsetdash{}{0pt}%
\pgfpathmoveto{\pgfqpoint{0.745371in}{0.566590in}}%
\pgfpathlineto{\pgfqpoint{4.036389in}{0.566590in}}%
\pgfpathlineto{\pgfqpoint{4.036389in}{2.395314in}}%
\pgfpathlineto{\pgfqpoint{0.745371in}{2.395314in}}%
\pgfpathclose%
\pgfusepath{fill}%
\end{pgfscope}%
\begin{pgfscope}%
\pgfpathrectangle{\pgfqpoint{0.745371in}{0.566590in}}{\pgfqpoint{3.291018in}{1.828724in}}%
\pgfusepath{clip}%
\pgfsetrectcap%
\pgfsetroundjoin%
\pgfsetlinewidth{0.803000pt}%
\definecolor{currentstroke}{rgb}{0.690196,0.690196,0.690196}%
\pgfsetstrokecolor{currentstroke}%
\pgfsetdash{}{0pt}%
\pgfpathmoveto{\pgfqpoint{0.745371in}{0.566590in}}%
\pgfpathlineto{\pgfqpoint{0.745371in}{2.395314in}}%
\pgfusepath{stroke}%
\end{pgfscope}%
\begin{pgfscope}%
\pgfsetbuttcap%
\pgfsetroundjoin%
\definecolor{currentfill}{rgb}{0.000000,0.000000,0.000000}%
\pgfsetfillcolor{currentfill}%
\pgfsetlinewidth{0.803000pt}%
\definecolor{currentstroke}{rgb}{0.000000,0.000000,0.000000}%
\pgfsetstrokecolor{currentstroke}%
\pgfsetdash{}{0pt}%
\pgfsys@defobject{currentmarker}{\pgfqpoint{0.000000in}{-0.048611in}}{\pgfqpoint{0.000000in}{0.000000in}}{%
\pgfpathmoveto{\pgfqpoint{0.000000in}{0.000000in}}%
\pgfpathlineto{\pgfqpoint{0.000000in}{-0.048611in}}%
\pgfusepath{stroke,fill}%
}%
\begin{pgfscope}%
\pgfsys@transformshift{0.745371in}{0.566590in}%
\pgfsys@useobject{currentmarker}{}%
\end{pgfscope}%
\end{pgfscope}%
\begin{pgfscope}%
\definecolor{textcolor}{rgb}{0.000000,0.000000,0.000000}%
\pgfsetstrokecolor{textcolor}%
\pgfsetfillcolor{textcolor}%
\pgftext[x=0.745371in,y=0.469368in,,top]{\color{textcolor}\rmfamily\fontsize{12.000000}{14.400000}\selectfont \(\displaystyle {-10}\)}%
\end{pgfscope}%
\begin{pgfscope}%
\pgfpathrectangle{\pgfqpoint{0.745371in}{0.566590in}}{\pgfqpoint{3.291018in}{1.828724in}}%
\pgfusepath{clip}%
\pgfsetrectcap%
\pgfsetroundjoin%
\pgfsetlinewidth{0.803000pt}%
\definecolor{currentstroke}{rgb}{0.690196,0.690196,0.690196}%
\pgfsetstrokecolor{currentstroke}%
\pgfsetdash{}{0pt}%
\pgfpathmoveto{\pgfqpoint{1.251681in}{0.566590in}}%
\pgfpathlineto{\pgfqpoint{1.251681in}{2.395314in}}%
\pgfusepath{stroke}%
\end{pgfscope}%
\begin{pgfscope}%
\pgfsetbuttcap%
\pgfsetroundjoin%
\definecolor{currentfill}{rgb}{0.000000,0.000000,0.000000}%
\pgfsetfillcolor{currentfill}%
\pgfsetlinewidth{0.803000pt}%
\definecolor{currentstroke}{rgb}{0.000000,0.000000,0.000000}%
\pgfsetstrokecolor{currentstroke}%
\pgfsetdash{}{0pt}%
\pgfsys@defobject{currentmarker}{\pgfqpoint{0.000000in}{-0.048611in}}{\pgfqpoint{0.000000in}{0.000000in}}{%
\pgfpathmoveto{\pgfqpoint{0.000000in}{0.000000in}}%
\pgfpathlineto{\pgfqpoint{0.000000in}{-0.048611in}}%
\pgfusepath{stroke,fill}%
}%
\begin{pgfscope}%
\pgfsys@transformshift{1.251681in}{0.566590in}%
\pgfsys@useobject{currentmarker}{}%
\end{pgfscope}%
\end{pgfscope}%
\begin{pgfscope}%
\definecolor{textcolor}{rgb}{0.000000,0.000000,0.000000}%
\pgfsetstrokecolor{textcolor}%
\pgfsetfillcolor{textcolor}%
\pgftext[x=1.251681in,y=0.469368in,,top]{\color{textcolor}\rmfamily\fontsize{12.000000}{14.400000}\selectfont \(\displaystyle {0}\)}%
\end{pgfscope}%
\begin{pgfscope}%
\pgfpathrectangle{\pgfqpoint{0.745371in}{0.566590in}}{\pgfqpoint{3.291018in}{1.828724in}}%
\pgfusepath{clip}%
\pgfsetrectcap%
\pgfsetroundjoin%
\pgfsetlinewidth{0.803000pt}%
\definecolor{currentstroke}{rgb}{0.690196,0.690196,0.690196}%
\pgfsetstrokecolor{currentstroke}%
\pgfsetdash{}{0pt}%
\pgfpathmoveto{\pgfqpoint{1.757992in}{0.566590in}}%
\pgfpathlineto{\pgfqpoint{1.757992in}{2.395314in}}%
\pgfusepath{stroke}%
\end{pgfscope}%
\begin{pgfscope}%
\pgfsetbuttcap%
\pgfsetroundjoin%
\definecolor{currentfill}{rgb}{0.000000,0.000000,0.000000}%
\pgfsetfillcolor{currentfill}%
\pgfsetlinewidth{0.803000pt}%
\definecolor{currentstroke}{rgb}{0.000000,0.000000,0.000000}%
\pgfsetstrokecolor{currentstroke}%
\pgfsetdash{}{0pt}%
\pgfsys@defobject{currentmarker}{\pgfqpoint{0.000000in}{-0.048611in}}{\pgfqpoint{0.000000in}{0.000000in}}{%
\pgfpathmoveto{\pgfqpoint{0.000000in}{0.000000in}}%
\pgfpathlineto{\pgfqpoint{0.000000in}{-0.048611in}}%
\pgfusepath{stroke,fill}%
}%
\begin{pgfscope}%
\pgfsys@transformshift{1.757992in}{0.566590in}%
\pgfsys@useobject{currentmarker}{}%
\end{pgfscope}%
\end{pgfscope}%
\begin{pgfscope}%
\definecolor{textcolor}{rgb}{0.000000,0.000000,0.000000}%
\pgfsetstrokecolor{textcolor}%
\pgfsetfillcolor{textcolor}%
\pgftext[x=1.757992in,y=0.469368in,,top]{\color{textcolor}\rmfamily\fontsize{12.000000}{14.400000}\selectfont \(\displaystyle {10}\)}%
\end{pgfscope}%
\begin{pgfscope}%
\pgfpathrectangle{\pgfqpoint{0.745371in}{0.566590in}}{\pgfqpoint{3.291018in}{1.828724in}}%
\pgfusepath{clip}%
\pgfsetrectcap%
\pgfsetroundjoin%
\pgfsetlinewidth{0.803000pt}%
\definecolor{currentstroke}{rgb}{0.690196,0.690196,0.690196}%
\pgfsetstrokecolor{currentstroke}%
\pgfsetdash{}{0pt}%
\pgfpathmoveto{\pgfqpoint{2.264302in}{0.566590in}}%
\pgfpathlineto{\pgfqpoint{2.264302in}{2.395314in}}%
\pgfusepath{stroke}%
\end{pgfscope}%
\begin{pgfscope}%
\pgfsetbuttcap%
\pgfsetroundjoin%
\definecolor{currentfill}{rgb}{0.000000,0.000000,0.000000}%
\pgfsetfillcolor{currentfill}%
\pgfsetlinewidth{0.803000pt}%
\definecolor{currentstroke}{rgb}{0.000000,0.000000,0.000000}%
\pgfsetstrokecolor{currentstroke}%
\pgfsetdash{}{0pt}%
\pgfsys@defobject{currentmarker}{\pgfqpoint{0.000000in}{-0.048611in}}{\pgfqpoint{0.000000in}{0.000000in}}{%
\pgfpathmoveto{\pgfqpoint{0.000000in}{0.000000in}}%
\pgfpathlineto{\pgfqpoint{0.000000in}{-0.048611in}}%
\pgfusepath{stroke,fill}%
}%
\begin{pgfscope}%
\pgfsys@transformshift{2.264302in}{0.566590in}%
\pgfsys@useobject{currentmarker}{}%
\end{pgfscope}%
\end{pgfscope}%
\begin{pgfscope}%
\definecolor{textcolor}{rgb}{0.000000,0.000000,0.000000}%
\pgfsetstrokecolor{textcolor}%
\pgfsetfillcolor{textcolor}%
\pgftext[x=2.264302in,y=0.469368in,,top]{\color{textcolor}\rmfamily\fontsize{12.000000}{14.400000}\selectfont \(\displaystyle {20}\)}%
\end{pgfscope}%
\begin{pgfscope}%
\pgfpathrectangle{\pgfqpoint{0.745371in}{0.566590in}}{\pgfqpoint{3.291018in}{1.828724in}}%
\pgfusepath{clip}%
\pgfsetrectcap%
\pgfsetroundjoin%
\pgfsetlinewidth{0.803000pt}%
\definecolor{currentstroke}{rgb}{0.690196,0.690196,0.690196}%
\pgfsetstrokecolor{currentstroke}%
\pgfsetdash{}{0pt}%
\pgfpathmoveto{\pgfqpoint{2.770613in}{0.566590in}}%
\pgfpathlineto{\pgfqpoint{2.770613in}{2.395314in}}%
\pgfusepath{stroke}%
\end{pgfscope}%
\begin{pgfscope}%
\pgfsetbuttcap%
\pgfsetroundjoin%
\definecolor{currentfill}{rgb}{0.000000,0.000000,0.000000}%
\pgfsetfillcolor{currentfill}%
\pgfsetlinewidth{0.803000pt}%
\definecolor{currentstroke}{rgb}{0.000000,0.000000,0.000000}%
\pgfsetstrokecolor{currentstroke}%
\pgfsetdash{}{0pt}%
\pgfsys@defobject{currentmarker}{\pgfqpoint{0.000000in}{-0.048611in}}{\pgfqpoint{0.000000in}{0.000000in}}{%
\pgfpathmoveto{\pgfqpoint{0.000000in}{0.000000in}}%
\pgfpathlineto{\pgfqpoint{0.000000in}{-0.048611in}}%
\pgfusepath{stroke,fill}%
}%
\begin{pgfscope}%
\pgfsys@transformshift{2.770613in}{0.566590in}%
\pgfsys@useobject{currentmarker}{}%
\end{pgfscope}%
\end{pgfscope}%
\begin{pgfscope}%
\definecolor{textcolor}{rgb}{0.000000,0.000000,0.000000}%
\pgfsetstrokecolor{textcolor}%
\pgfsetfillcolor{textcolor}%
\pgftext[x=2.770613in,y=0.469368in,,top]{\color{textcolor}\rmfamily\fontsize{12.000000}{14.400000}\selectfont \(\displaystyle {30}\)}%
\end{pgfscope}%
\begin{pgfscope}%
\pgfpathrectangle{\pgfqpoint{0.745371in}{0.566590in}}{\pgfqpoint{3.291018in}{1.828724in}}%
\pgfusepath{clip}%
\pgfsetrectcap%
\pgfsetroundjoin%
\pgfsetlinewidth{0.803000pt}%
\definecolor{currentstroke}{rgb}{0.690196,0.690196,0.690196}%
\pgfsetstrokecolor{currentstroke}%
\pgfsetdash{}{0pt}%
\pgfpathmoveto{\pgfqpoint{3.276923in}{0.566590in}}%
\pgfpathlineto{\pgfqpoint{3.276923in}{2.395314in}}%
\pgfusepath{stroke}%
\end{pgfscope}%
\begin{pgfscope}%
\pgfsetbuttcap%
\pgfsetroundjoin%
\definecolor{currentfill}{rgb}{0.000000,0.000000,0.000000}%
\pgfsetfillcolor{currentfill}%
\pgfsetlinewidth{0.803000pt}%
\definecolor{currentstroke}{rgb}{0.000000,0.000000,0.000000}%
\pgfsetstrokecolor{currentstroke}%
\pgfsetdash{}{0pt}%
\pgfsys@defobject{currentmarker}{\pgfqpoint{0.000000in}{-0.048611in}}{\pgfqpoint{0.000000in}{0.000000in}}{%
\pgfpathmoveto{\pgfqpoint{0.000000in}{0.000000in}}%
\pgfpathlineto{\pgfqpoint{0.000000in}{-0.048611in}}%
\pgfusepath{stroke,fill}%
}%
\begin{pgfscope}%
\pgfsys@transformshift{3.276923in}{0.566590in}%
\pgfsys@useobject{currentmarker}{}%
\end{pgfscope}%
\end{pgfscope}%
\begin{pgfscope}%
\definecolor{textcolor}{rgb}{0.000000,0.000000,0.000000}%
\pgfsetstrokecolor{textcolor}%
\pgfsetfillcolor{textcolor}%
\pgftext[x=3.276923in,y=0.469368in,,top]{\color{textcolor}\rmfamily\fontsize{12.000000}{14.400000}\selectfont \(\displaystyle {40}\)}%
\end{pgfscope}%
\begin{pgfscope}%
\pgfpathrectangle{\pgfqpoint{0.745371in}{0.566590in}}{\pgfqpoint{3.291018in}{1.828724in}}%
\pgfusepath{clip}%
\pgfsetrectcap%
\pgfsetroundjoin%
\pgfsetlinewidth{0.803000pt}%
\definecolor{currentstroke}{rgb}{0.690196,0.690196,0.690196}%
\pgfsetstrokecolor{currentstroke}%
\pgfsetdash{}{0pt}%
\pgfpathmoveto{\pgfqpoint{3.783233in}{0.566590in}}%
\pgfpathlineto{\pgfqpoint{3.783233in}{2.395314in}}%
\pgfusepath{stroke}%
\end{pgfscope}%
\begin{pgfscope}%
\pgfsetbuttcap%
\pgfsetroundjoin%
\definecolor{currentfill}{rgb}{0.000000,0.000000,0.000000}%
\pgfsetfillcolor{currentfill}%
\pgfsetlinewidth{0.803000pt}%
\definecolor{currentstroke}{rgb}{0.000000,0.000000,0.000000}%
\pgfsetstrokecolor{currentstroke}%
\pgfsetdash{}{0pt}%
\pgfsys@defobject{currentmarker}{\pgfqpoint{0.000000in}{-0.048611in}}{\pgfqpoint{0.000000in}{0.000000in}}{%
\pgfpathmoveto{\pgfqpoint{0.000000in}{0.000000in}}%
\pgfpathlineto{\pgfqpoint{0.000000in}{-0.048611in}}%
\pgfusepath{stroke,fill}%
}%
\begin{pgfscope}%
\pgfsys@transformshift{3.783233in}{0.566590in}%
\pgfsys@useobject{currentmarker}{}%
\end{pgfscope}%
\end{pgfscope}%
\begin{pgfscope}%
\definecolor{textcolor}{rgb}{0.000000,0.000000,0.000000}%
\pgfsetstrokecolor{textcolor}%
\pgfsetfillcolor{textcolor}%
\pgftext[x=3.783233in,y=0.469368in,,top]{\color{textcolor}\rmfamily\fontsize{12.000000}{14.400000}\selectfont \(\displaystyle {50}\)}%
\end{pgfscope}%
\begin{pgfscope}%
\definecolor{textcolor}{rgb}{0.000000,0.000000,0.000000}%
\pgfsetstrokecolor{textcolor}%
\pgfsetfillcolor{textcolor}%
\pgftext[x=2.390880in,y=0.266626in,,top]{\color{textcolor}\rmfamily\fontsize{12.000000}{14.400000}\selectfont SNR [dB]}%
\end{pgfscope}%
\begin{pgfscope}%
\pgfpathrectangle{\pgfqpoint{0.745371in}{0.566590in}}{\pgfqpoint{3.291018in}{1.828724in}}%
\pgfusepath{clip}%
\pgfsetrectcap%
\pgfsetroundjoin%
\pgfsetlinewidth{0.803000pt}%
\definecolor{currentstroke}{rgb}{0.690196,0.690196,0.690196}%
\pgfsetstrokecolor{currentstroke}%
\pgfsetdash{}{0pt}%
\pgfpathmoveto{\pgfqpoint{0.745371in}{0.752967in}}%
\pgfpathlineto{\pgfqpoint{4.036389in}{0.752967in}}%
\pgfusepath{stroke}%
\end{pgfscope}%
\begin{pgfscope}%
\pgfsetbuttcap%
\pgfsetroundjoin%
\definecolor{currentfill}{rgb}{0.000000,0.000000,0.000000}%
\pgfsetfillcolor{currentfill}%
\pgfsetlinewidth{0.803000pt}%
\definecolor{currentstroke}{rgb}{0.000000,0.000000,0.000000}%
\pgfsetstrokecolor{currentstroke}%
\pgfsetdash{}{0pt}%
\pgfsys@defobject{currentmarker}{\pgfqpoint{-0.048611in}{0.000000in}}{\pgfqpoint{-0.000000in}{0.000000in}}{%
\pgfpathmoveto{\pgfqpoint{-0.000000in}{0.000000in}}%
\pgfpathlineto{\pgfqpoint{-0.048611in}{0.000000in}}%
\pgfusepath{stroke,fill}%
}%
\begin{pgfscope}%
\pgfsys@transformshift{0.745371in}{0.752967in}%
\pgfsys@useobject{currentmarker}{}%
\end{pgfscope}%
\end{pgfscope}%
\begin{pgfscope}%
\definecolor{textcolor}{rgb}{0.000000,0.000000,0.000000}%
\pgfsetstrokecolor{textcolor}%
\pgfsetfillcolor{textcolor}%
\pgftext[x=0.327160in, y=0.695574in, left, base]{\color{textcolor}\rmfamily\fontsize{12.000000}{14.400000}\selectfont \(\displaystyle {10^{-4}}\)}%
\end{pgfscope}%
\begin{pgfscope}%
\pgfpathrectangle{\pgfqpoint{0.745371in}{0.566590in}}{\pgfqpoint{3.291018in}{1.828724in}}%
\pgfusepath{clip}%
\pgfsetrectcap%
\pgfsetroundjoin%
\pgfsetlinewidth{0.803000pt}%
\definecolor{currentstroke}{rgb}{0.690196,0.690196,0.690196}%
\pgfsetstrokecolor{currentstroke}%
\pgfsetdash{}{0pt}%
\pgfpathmoveto{\pgfqpoint{0.745371in}{1.236303in}}%
\pgfpathlineto{\pgfqpoint{4.036389in}{1.236303in}}%
\pgfusepath{stroke}%
\end{pgfscope}%
\begin{pgfscope}%
\pgfsetbuttcap%
\pgfsetroundjoin%
\definecolor{currentfill}{rgb}{0.000000,0.000000,0.000000}%
\pgfsetfillcolor{currentfill}%
\pgfsetlinewidth{0.803000pt}%
\definecolor{currentstroke}{rgb}{0.000000,0.000000,0.000000}%
\pgfsetstrokecolor{currentstroke}%
\pgfsetdash{}{0pt}%
\pgfsys@defobject{currentmarker}{\pgfqpoint{-0.048611in}{0.000000in}}{\pgfqpoint{-0.000000in}{0.000000in}}{%
\pgfpathmoveto{\pgfqpoint{-0.000000in}{0.000000in}}%
\pgfpathlineto{\pgfqpoint{-0.048611in}{0.000000in}}%
\pgfusepath{stroke,fill}%
}%
\begin{pgfscope}%
\pgfsys@transformshift{0.745371in}{1.236303in}%
\pgfsys@useobject{currentmarker}{}%
\end{pgfscope}%
\end{pgfscope}%
\begin{pgfscope}%
\definecolor{textcolor}{rgb}{0.000000,0.000000,0.000000}%
\pgfsetstrokecolor{textcolor}%
\pgfsetfillcolor{textcolor}%
\pgftext[x=0.327160in, y=1.178910in, left, base]{\color{textcolor}\rmfamily\fontsize{12.000000}{14.400000}\selectfont \(\displaystyle {10^{-2}}\)}%
\end{pgfscope}%
\begin{pgfscope}%
\pgfpathrectangle{\pgfqpoint{0.745371in}{0.566590in}}{\pgfqpoint{3.291018in}{1.828724in}}%
\pgfusepath{clip}%
\pgfsetrectcap%
\pgfsetroundjoin%
\pgfsetlinewidth{0.803000pt}%
\definecolor{currentstroke}{rgb}{0.690196,0.690196,0.690196}%
\pgfsetstrokecolor{currentstroke}%
\pgfsetdash{}{0pt}%
\pgfpathmoveto{\pgfqpoint{0.745371in}{1.719639in}}%
\pgfpathlineto{\pgfqpoint{4.036389in}{1.719639in}}%
\pgfusepath{stroke}%
\end{pgfscope}%
\begin{pgfscope}%
\pgfsetbuttcap%
\pgfsetroundjoin%
\definecolor{currentfill}{rgb}{0.000000,0.000000,0.000000}%
\pgfsetfillcolor{currentfill}%
\pgfsetlinewidth{0.803000pt}%
\definecolor{currentstroke}{rgb}{0.000000,0.000000,0.000000}%
\pgfsetstrokecolor{currentstroke}%
\pgfsetdash{}{0pt}%
\pgfsys@defobject{currentmarker}{\pgfqpoint{-0.048611in}{0.000000in}}{\pgfqpoint{-0.000000in}{0.000000in}}{%
\pgfpathmoveto{\pgfqpoint{-0.000000in}{0.000000in}}%
\pgfpathlineto{\pgfqpoint{-0.048611in}{0.000000in}}%
\pgfusepath{stroke,fill}%
}%
\begin{pgfscope}%
\pgfsys@transformshift{0.745371in}{1.719639in}%
\pgfsys@useobject{currentmarker}{}%
\end{pgfscope}%
\end{pgfscope}%
\begin{pgfscope}%
\definecolor{textcolor}{rgb}{0.000000,0.000000,0.000000}%
\pgfsetstrokecolor{textcolor}%
\pgfsetfillcolor{textcolor}%
\pgftext[x=0.418983in, y=1.662246in, left, base]{\color{textcolor}\rmfamily\fontsize{12.000000}{14.400000}\selectfont \(\displaystyle {10^{0}}\)}%
\end{pgfscope}%
\begin{pgfscope}%
\pgfpathrectangle{\pgfqpoint{0.745371in}{0.566590in}}{\pgfqpoint{3.291018in}{1.828724in}}%
\pgfusepath{clip}%
\pgfsetrectcap%
\pgfsetroundjoin%
\pgfsetlinewidth{0.803000pt}%
\definecolor{currentstroke}{rgb}{0.690196,0.690196,0.690196}%
\pgfsetstrokecolor{currentstroke}%
\pgfsetdash{}{0pt}%
\pgfpathmoveto{\pgfqpoint{0.745371in}{2.202975in}}%
\pgfpathlineto{\pgfqpoint{4.036389in}{2.202975in}}%
\pgfusepath{stroke}%
\end{pgfscope}%
\begin{pgfscope}%
\pgfsetbuttcap%
\pgfsetroundjoin%
\definecolor{currentfill}{rgb}{0.000000,0.000000,0.000000}%
\pgfsetfillcolor{currentfill}%
\pgfsetlinewidth{0.803000pt}%
\definecolor{currentstroke}{rgb}{0.000000,0.000000,0.000000}%
\pgfsetstrokecolor{currentstroke}%
\pgfsetdash{}{0pt}%
\pgfsys@defobject{currentmarker}{\pgfqpoint{-0.048611in}{0.000000in}}{\pgfqpoint{-0.000000in}{0.000000in}}{%
\pgfpathmoveto{\pgfqpoint{-0.000000in}{0.000000in}}%
\pgfpathlineto{\pgfqpoint{-0.048611in}{0.000000in}}%
\pgfusepath{stroke,fill}%
}%
\begin{pgfscope}%
\pgfsys@transformshift{0.745371in}{2.202975in}%
\pgfsys@useobject{currentmarker}{}%
\end{pgfscope}%
\end{pgfscope}%
\begin{pgfscope}%
\definecolor{textcolor}{rgb}{0.000000,0.000000,0.000000}%
\pgfsetstrokecolor{textcolor}%
\pgfsetfillcolor{textcolor}%
\pgftext[x=0.418983in, y=2.145582in, left, base]{\color{textcolor}\rmfamily\fontsize{12.000000}{14.400000}\selectfont \(\displaystyle {10^{2}}\)}%
\end{pgfscope}%
\begin{pgfscope}%
\definecolor{textcolor}{rgb}{0.000000,0.000000,0.000000}%
\pgfsetstrokecolor{textcolor}%
\pgfsetfillcolor{textcolor}%
\pgftext[x=0.271605in,y=1.480952in,,bottom,rotate=90.000000]{\color{textcolor}\rmfamily\fontsize{12.000000}{14.400000}\selectfont \(\displaystyle \hat{\sigma}_{\gamma}(\mathrm{SNR})\)}%
\end{pgfscope}%
\begin{pgfscope}%
\pgfpathrectangle{\pgfqpoint{0.745371in}{0.566590in}}{\pgfqpoint{3.291018in}{1.828724in}}%
\pgfusepath{clip}%
\pgfsetbuttcap%
\pgfsetroundjoin%
\pgfsetlinewidth{1.505625pt}%
\definecolor{currentstroke}{rgb}{0.000000,0.447000,0.741000}%
\pgfsetstrokecolor{currentstroke}%
\pgfsetdash{{5.550000pt}{2.400000pt}}{0.000000pt}%
\pgfpathmoveto{\pgfqpoint{0.745371in}{2.068223in}}%
\pgfpathlineto{\pgfqpoint{0.842165in}{2.261859in}}%
\pgfpathlineto{\pgfqpoint{0.938960in}{2.108185in}}%
\pgfpathlineto{\pgfqpoint{1.035755in}{2.204288in}}%
\pgfpathlineto{\pgfqpoint{1.132549in}{2.121481in}}%
\pgfpathlineto{\pgfqpoint{1.229344in}{1.723831in}}%
\pgfpathlineto{\pgfqpoint{1.326139in}{2.191650in}}%
\pgfpathlineto{\pgfqpoint{1.422933in}{2.150292in}}%
\pgfpathlineto{\pgfqpoint{1.519728in}{2.109153in}}%
\pgfpathlineto{\pgfqpoint{1.616523in}{2.144903in}}%
\pgfpathlineto{\pgfqpoint{1.713317in}{2.053628in}}%
\pgfpathlineto{\pgfqpoint{1.810112in}{1.973476in}}%
\pgfpathlineto{\pgfqpoint{1.906906in}{2.199782in}}%
\pgfpathlineto{\pgfqpoint{2.003701in}{2.098222in}}%
\pgfpathlineto{\pgfqpoint{2.100496in}{2.046084in}}%
\pgfpathlineto{\pgfqpoint{2.197290in}{2.179100in}}%
\pgfpathlineto{\pgfqpoint{2.294085in}{2.119309in}}%
\pgfpathlineto{\pgfqpoint{2.390880in}{1.393603in}}%
\pgfpathlineto{\pgfqpoint{2.487674in}{1.373057in}}%
\pgfpathlineto{\pgfqpoint{2.584469in}{1.342088in}}%
\pgfpathlineto{\pgfqpoint{2.681264in}{1.312543in}}%
\pgfpathlineto{\pgfqpoint{2.778058in}{1.288331in}}%
\pgfpathlineto{\pgfqpoint{2.874853in}{1.272097in}}%
\pgfpathlineto{\pgfqpoint{2.971648in}{1.260697in}}%
\pgfpathlineto{\pgfqpoint{3.068442in}{1.226860in}}%
\pgfpathlineto{\pgfqpoint{3.165237in}{1.206089in}}%
\pgfpathlineto{\pgfqpoint{3.262031in}{1.187618in}}%
\pgfpathlineto{\pgfqpoint{3.358826in}{1.159789in}}%
\pgfpathlineto{\pgfqpoint{3.455621in}{1.133985in}}%
\pgfpathlineto{\pgfqpoint{3.552415in}{1.114500in}}%
\pgfpathlineto{\pgfqpoint{3.649210in}{1.083234in}}%
\pgfpathlineto{\pgfqpoint{3.746005in}{1.054775in}}%
\pgfpathlineto{\pgfqpoint{3.842799in}{1.045662in}}%
\pgfpathlineto{\pgfqpoint{3.939594in}{1.001062in}}%
\pgfpathlineto{\pgfqpoint{4.036389in}{0.998476in}}%
\pgfusepath{stroke}%
\end{pgfscope}%
\begin{pgfscope}%
\pgfpathrectangle{\pgfqpoint{0.745371in}{0.566590in}}{\pgfqpoint{3.291018in}{1.828724in}}%
\pgfusepath{clip}%
\pgfsetbuttcap%
\pgfsetroundjoin%
\definecolor{currentfill}{rgb}{0.000000,0.000000,0.000000}%
\pgfsetfillcolor{currentfill}%
\pgfsetfillopacity{0.000000}%
\pgfsetlinewidth{1.003750pt}%
\definecolor{currentstroke}{rgb}{0.000000,0.447000,0.741000}%
\pgfsetstrokecolor{currentstroke}%
\pgfsetdash{}{0pt}%
\pgfsys@defobject{currentmarker}{\pgfqpoint{-0.041667in}{-0.041667in}}{\pgfqpoint{0.041667in}{0.041667in}}{%
\pgfpathmoveto{\pgfqpoint{0.000000in}{-0.041667in}}%
\pgfpathcurveto{\pgfqpoint{0.011050in}{-0.041667in}}{\pgfqpoint{0.021649in}{-0.037276in}}{\pgfqpoint{0.029463in}{-0.029463in}}%
\pgfpathcurveto{\pgfqpoint{0.037276in}{-0.021649in}}{\pgfqpoint{0.041667in}{-0.011050in}}{\pgfqpoint{0.041667in}{0.000000in}}%
\pgfpathcurveto{\pgfqpoint{0.041667in}{0.011050in}}{\pgfqpoint{0.037276in}{0.021649in}}{\pgfqpoint{0.029463in}{0.029463in}}%
\pgfpathcurveto{\pgfqpoint{0.021649in}{0.037276in}}{\pgfqpoint{0.011050in}{0.041667in}}{\pgfqpoint{0.000000in}{0.041667in}}%
\pgfpathcurveto{\pgfqpoint{-0.011050in}{0.041667in}}{\pgfqpoint{-0.021649in}{0.037276in}}{\pgfqpoint{-0.029463in}{0.029463in}}%
\pgfpathcurveto{\pgfqpoint{-0.037276in}{0.021649in}}{\pgfqpoint{-0.041667in}{0.011050in}}{\pgfqpoint{-0.041667in}{0.000000in}}%
\pgfpathcurveto{\pgfqpoint{-0.041667in}{-0.011050in}}{\pgfqpoint{-0.037276in}{-0.021649in}}{\pgfqpoint{-0.029463in}{-0.029463in}}%
\pgfpathcurveto{\pgfqpoint{-0.021649in}{-0.037276in}}{\pgfqpoint{-0.011050in}{-0.041667in}}{\pgfqpoint{0.000000in}{-0.041667in}}%
\pgfpathclose%
\pgfusepath{stroke,fill}%
}%
\begin{pgfscope}%
\pgfsys@transformshift{0.745371in}{2.068223in}%
\pgfsys@useobject{currentmarker}{}%
\end{pgfscope}%
\begin{pgfscope}%
\pgfsys@transformshift{1.132549in}{2.121481in}%
\pgfsys@useobject{currentmarker}{}%
\end{pgfscope}%
\begin{pgfscope}%
\pgfsys@transformshift{1.519728in}{2.109153in}%
\pgfsys@useobject{currentmarker}{}%
\end{pgfscope}%
\begin{pgfscope}%
\pgfsys@transformshift{1.906906in}{2.199782in}%
\pgfsys@useobject{currentmarker}{}%
\end{pgfscope}%
\begin{pgfscope}%
\pgfsys@transformshift{2.294085in}{2.119309in}%
\pgfsys@useobject{currentmarker}{}%
\end{pgfscope}%
\begin{pgfscope}%
\pgfsys@transformshift{2.681264in}{1.312543in}%
\pgfsys@useobject{currentmarker}{}%
\end{pgfscope}%
\begin{pgfscope}%
\pgfsys@transformshift{3.068442in}{1.226860in}%
\pgfsys@useobject{currentmarker}{}%
\end{pgfscope}%
\begin{pgfscope}%
\pgfsys@transformshift{3.455621in}{1.133985in}%
\pgfsys@useobject{currentmarker}{}%
\end{pgfscope}%
\begin{pgfscope}%
\pgfsys@transformshift{3.842799in}{1.045662in}%
\pgfsys@useobject{currentmarker}{}%
\end{pgfscope}%
\end{pgfscope}%
\begin{pgfscope}%
\pgfpathrectangle{\pgfqpoint{0.745371in}{0.566590in}}{\pgfqpoint{3.291018in}{1.828724in}}%
\pgfusepath{clip}%
\pgfsetbuttcap%
\pgfsetroundjoin%
\pgfsetlinewidth{1.505625pt}%
\definecolor{currentstroke}{rgb}{0.850000,0.324000,0.098000}%
\pgfsetstrokecolor{currentstroke}%
\pgfsetdash{{5.550000pt}{2.400000pt}}{0.000000pt}%
\pgfpathmoveto{\pgfqpoint{0.745371in}{2.109308in}}%
\pgfpathlineto{\pgfqpoint{0.842165in}{2.189375in}}%
\pgfpathlineto{\pgfqpoint{0.938960in}{2.127957in}}%
\pgfpathlineto{\pgfqpoint{1.035755in}{2.040954in}}%
\pgfpathlineto{\pgfqpoint{1.132549in}{1.991257in}}%
\pgfpathlineto{\pgfqpoint{1.229344in}{2.192613in}}%
\pgfpathlineto{\pgfqpoint{1.326139in}{2.154470in}}%
\pgfpathlineto{\pgfqpoint{1.422933in}{2.054354in}}%
\pgfpathlineto{\pgfqpoint{1.519728in}{2.072356in}}%
\pgfpathlineto{\pgfqpoint{1.616523in}{2.164981in}}%
\pgfpathlineto{\pgfqpoint{1.713317in}{2.086621in}}%
\pgfpathlineto{\pgfqpoint{1.810112in}{2.165420in}}%
\pgfpathlineto{\pgfqpoint{1.906906in}{2.011733in}}%
\pgfpathlineto{\pgfqpoint{2.003701in}{1.923484in}}%
\pgfpathlineto{\pgfqpoint{2.100496in}{2.010622in}}%
\pgfpathlineto{\pgfqpoint{2.197290in}{2.214670in}}%
\pgfpathlineto{\pgfqpoint{2.294085in}{2.116621in}}%
\pgfpathlineto{\pgfqpoint{2.390880in}{2.046015in}}%
\pgfpathlineto{\pgfqpoint{2.487674in}{1.449907in}}%
\pgfpathlineto{\pgfqpoint{2.584469in}{1.337627in}}%
\pgfpathlineto{\pgfqpoint{2.681264in}{1.326942in}}%
\pgfpathlineto{\pgfqpoint{2.778058in}{1.299416in}}%
\pgfpathlineto{\pgfqpoint{2.874853in}{1.274112in}}%
\pgfpathlineto{\pgfqpoint{2.971648in}{1.252285in}}%
\pgfpathlineto{\pgfqpoint{3.068442in}{1.226884in}}%
\pgfpathlineto{\pgfqpoint{3.165237in}{1.207498in}}%
\pgfpathlineto{\pgfqpoint{3.262031in}{1.174296in}}%
\pgfpathlineto{\pgfqpoint{3.358826in}{1.155795in}}%
\pgfpathlineto{\pgfqpoint{3.455621in}{1.134398in}}%
\pgfpathlineto{\pgfqpoint{3.552415in}{1.109286in}}%
\pgfpathlineto{\pgfqpoint{3.649210in}{1.093566in}}%
\pgfpathlineto{\pgfqpoint{3.746005in}{1.062354in}}%
\pgfpathlineto{\pgfqpoint{3.842799in}{1.043365in}}%
\pgfpathlineto{\pgfqpoint{3.939594in}{1.014412in}}%
\pgfpathlineto{\pgfqpoint{4.036389in}{1.000309in}}%
\pgfusepath{stroke}%
\end{pgfscope}%
\begin{pgfscope}%
\pgfpathrectangle{\pgfqpoint{0.745371in}{0.566590in}}{\pgfqpoint{3.291018in}{1.828724in}}%
\pgfusepath{clip}%
\pgfsetbuttcap%
\pgfsetroundjoin%
\definecolor{currentfill}{rgb}{0.850000,0.324000,0.098000}%
\pgfsetfillcolor{currentfill}%
\pgfsetlinewidth{1.003750pt}%
\definecolor{currentstroke}{rgb}{0.850000,0.324000,0.098000}%
\pgfsetstrokecolor{currentstroke}%
\pgfsetdash{}{0pt}%
\pgfsys@defobject{currentmarker}{\pgfqpoint{-0.041667in}{-0.041667in}}{\pgfqpoint{0.041667in}{0.041667in}}{%
\pgfpathmoveto{\pgfqpoint{-0.041667in}{0.000000in}}%
\pgfpathlineto{\pgfqpoint{0.041667in}{0.000000in}}%
\pgfpathmoveto{\pgfqpoint{0.000000in}{-0.041667in}}%
\pgfpathlineto{\pgfqpoint{0.000000in}{0.041667in}}%
\pgfusepath{stroke,fill}%
}%
\begin{pgfscope}%
\pgfsys@transformshift{0.745371in}{2.109308in}%
\pgfsys@useobject{currentmarker}{}%
\end{pgfscope}%
\begin{pgfscope}%
\pgfsys@transformshift{1.035755in}{2.040954in}%
\pgfsys@useobject{currentmarker}{}%
\end{pgfscope}%
\begin{pgfscope}%
\pgfsys@transformshift{1.326139in}{2.154470in}%
\pgfsys@useobject{currentmarker}{}%
\end{pgfscope}%
\begin{pgfscope}%
\pgfsys@transformshift{1.616523in}{2.164981in}%
\pgfsys@useobject{currentmarker}{}%
\end{pgfscope}%
\begin{pgfscope}%
\pgfsys@transformshift{1.906906in}{2.011733in}%
\pgfsys@useobject{currentmarker}{}%
\end{pgfscope}%
\begin{pgfscope}%
\pgfsys@transformshift{2.197290in}{2.214670in}%
\pgfsys@useobject{currentmarker}{}%
\end{pgfscope}%
\begin{pgfscope}%
\pgfsys@transformshift{2.487674in}{1.449907in}%
\pgfsys@useobject{currentmarker}{}%
\end{pgfscope}%
\begin{pgfscope}%
\pgfsys@transformshift{2.778058in}{1.299416in}%
\pgfsys@useobject{currentmarker}{}%
\end{pgfscope}%
\begin{pgfscope}%
\pgfsys@transformshift{3.068442in}{1.226884in}%
\pgfsys@useobject{currentmarker}{}%
\end{pgfscope}%
\begin{pgfscope}%
\pgfsys@transformshift{3.358826in}{1.155795in}%
\pgfsys@useobject{currentmarker}{}%
\end{pgfscope}%
\begin{pgfscope}%
\pgfsys@transformshift{3.649210in}{1.093566in}%
\pgfsys@useobject{currentmarker}{}%
\end{pgfscope}%
\begin{pgfscope}%
\pgfsys@transformshift{3.939594in}{1.014412in}%
\pgfsys@useobject{currentmarker}{}%
\end{pgfscope}%
\end{pgfscope}%
\begin{pgfscope}%
\pgfpathrectangle{\pgfqpoint{0.745371in}{0.566590in}}{\pgfqpoint{3.291018in}{1.828724in}}%
\pgfusepath{clip}%
\pgfsetrectcap%
\pgfsetroundjoin%
\pgfsetlinewidth{1.505625pt}%
\definecolor{currentstroke}{rgb}{0.000000,0.447000,0.741000}%
\pgfsetstrokecolor{currentstroke}%
\pgfsetdash{}{0pt}%
\pgfpathmoveto{\pgfqpoint{0.745371in}{1.716428in}}%
\pgfpathlineto{\pgfqpoint{0.980444in}{1.659780in}}%
\pgfpathlineto{\pgfqpoint{1.215516in}{1.494915in}}%
\pgfpathlineto{\pgfqpoint{1.450589in}{1.436295in}}%
\pgfpathlineto{\pgfqpoint{1.685662in}{1.377846in}}%
\pgfpathlineto{\pgfqpoint{1.920734in}{1.311034in}}%
\pgfpathlineto{\pgfqpoint{2.155807in}{1.263758in}}%
\pgfpathlineto{\pgfqpoint{2.390880in}{1.199849in}}%
\pgfpathlineto{\pgfqpoint{2.625952in}{1.136349in}}%
\pgfpathlineto{\pgfqpoint{2.861025in}{1.083629in}}%
\pgfpathlineto{\pgfqpoint{3.096098in}{1.019544in}}%
\pgfpathlineto{\pgfqpoint{3.331170in}{0.973527in}}%
\pgfpathlineto{\pgfqpoint{3.566243in}{0.923458in}}%
\pgfpathlineto{\pgfqpoint{3.801316in}{0.866983in}}%
\pgfpathlineto{\pgfqpoint{4.036389in}{0.803000in}}%
\pgfusepath{stroke}%
\end{pgfscope}%
\begin{pgfscope}%
\pgfpathrectangle{\pgfqpoint{0.745371in}{0.566590in}}{\pgfqpoint{3.291018in}{1.828724in}}%
\pgfusepath{clip}%
\pgfsetbuttcap%
\pgfsetroundjoin%
\definecolor{currentfill}{rgb}{0.000000,0.000000,0.000000}%
\pgfsetfillcolor{currentfill}%
\pgfsetfillopacity{0.000000}%
\pgfsetlinewidth{1.003750pt}%
\definecolor{currentstroke}{rgb}{0.000000,0.447000,0.741000}%
\pgfsetstrokecolor{currentstroke}%
\pgfsetdash{}{0pt}%
\pgfsys@defobject{currentmarker}{\pgfqpoint{-0.041667in}{-0.041667in}}{\pgfqpoint{0.041667in}{0.041667in}}{%
\pgfpathmoveto{\pgfqpoint{0.000000in}{-0.041667in}}%
\pgfpathcurveto{\pgfqpoint{0.011050in}{-0.041667in}}{\pgfqpoint{0.021649in}{-0.037276in}}{\pgfqpoint{0.029463in}{-0.029463in}}%
\pgfpathcurveto{\pgfqpoint{0.037276in}{-0.021649in}}{\pgfqpoint{0.041667in}{-0.011050in}}{\pgfqpoint{0.041667in}{0.000000in}}%
\pgfpathcurveto{\pgfqpoint{0.041667in}{0.011050in}}{\pgfqpoint{0.037276in}{0.021649in}}{\pgfqpoint{0.029463in}{0.029463in}}%
\pgfpathcurveto{\pgfqpoint{0.021649in}{0.037276in}}{\pgfqpoint{0.011050in}{0.041667in}}{\pgfqpoint{0.000000in}{0.041667in}}%
\pgfpathcurveto{\pgfqpoint{-0.011050in}{0.041667in}}{\pgfqpoint{-0.021649in}{0.037276in}}{\pgfqpoint{-0.029463in}{0.029463in}}%
\pgfpathcurveto{\pgfqpoint{-0.037276in}{0.021649in}}{\pgfqpoint{-0.041667in}{0.011050in}}{\pgfqpoint{-0.041667in}{0.000000in}}%
\pgfpathcurveto{\pgfqpoint{-0.041667in}{-0.011050in}}{\pgfqpoint{-0.037276in}{-0.021649in}}{\pgfqpoint{-0.029463in}{-0.029463in}}%
\pgfpathcurveto{\pgfqpoint{-0.021649in}{-0.037276in}}{\pgfqpoint{-0.011050in}{-0.041667in}}{\pgfqpoint{0.000000in}{-0.041667in}}%
\pgfpathclose%
\pgfusepath{stroke,fill}%
}%
\begin{pgfscope}%
\pgfsys@transformshift{0.745371in}{1.716428in}%
\pgfsys@useobject{currentmarker}{}%
\end{pgfscope}%
\begin{pgfscope}%
\pgfsys@transformshift{0.980444in}{1.659780in}%
\pgfsys@useobject{currentmarker}{}%
\end{pgfscope}%
\begin{pgfscope}%
\pgfsys@transformshift{1.215516in}{1.494915in}%
\pgfsys@useobject{currentmarker}{}%
\end{pgfscope}%
\begin{pgfscope}%
\pgfsys@transformshift{1.450589in}{1.436295in}%
\pgfsys@useobject{currentmarker}{}%
\end{pgfscope}%
\begin{pgfscope}%
\pgfsys@transformshift{1.685662in}{1.377846in}%
\pgfsys@useobject{currentmarker}{}%
\end{pgfscope}%
\begin{pgfscope}%
\pgfsys@transformshift{1.920734in}{1.311034in}%
\pgfsys@useobject{currentmarker}{}%
\end{pgfscope}%
\begin{pgfscope}%
\pgfsys@transformshift{2.155807in}{1.263758in}%
\pgfsys@useobject{currentmarker}{}%
\end{pgfscope}%
\begin{pgfscope}%
\pgfsys@transformshift{2.390880in}{1.199849in}%
\pgfsys@useobject{currentmarker}{}%
\end{pgfscope}%
\begin{pgfscope}%
\pgfsys@transformshift{2.625952in}{1.136349in}%
\pgfsys@useobject{currentmarker}{}%
\end{pgfscope}%
\begin{pgfscope}%
\pgfsys@transformshift{2.861025in}{1.083629in}%
\pgfsys@useobject{currentmarker}{}%
\end{pgfscope}%
\begin{pgfscope}%
\pgfsys@transformshift{3.096098in}{1.019544in}%
\pgfsys@useobject{currentmarker}{}%
\end{pgfscope}%
\begin{pgfscope}%
\pgfsys@transformshift{3.331170in}{0.973527in}%
\pgfsys@useobject{currentmarker}{}%
\end{pgfscope}%
\begin{pgfscope}%
\pgfsys@transformshift{3.566243in}{0.923458in}%
\pgfsys@useobject{currentmarker}{}%
\end{pgfscope}%
\begin{pgfscope}%
\pgfsys@transformshift{3.801316in}{0.866983in}%
\pgfsys@useobject{currentmarker}{}%
\end{pgfscope}%
\begin{pgfscope}%
\pgfsys@transformshift{4.036389in}{0.803000in}%
\pgfsys@useobject{currentmarker}{}%
\end{pgfscope}%
\end{pgfscope}%
\begin{pgfscope}%
\pgfpathrectangle{\pgfqpoint{0.745371in}{0.566590in}}{\pgfqpoint{3.291018in}{1.828724in}}%
\pgfusepath{clip}%
\pgfsetrectcap%
\pgfsetroundjoin%
\pgfsetlinewidth{1.505625pt}%
\definecolor{currentstroke}{rgb}{0.850000,0.324000,0.098000}%
\pgfsetstrokecolor{currentstroke}%
\pgfsetdash{}{0pt}%
\pgfpathmoveto{\pgfqpoint{0.745371in}{1.773225in}}%
\pgfpathlineto{\pgfqpoint{0.980444in}{1.583077in}}%
\pgfpathlineto{\pgfqpoint{1.215516in}{1.492612in}}%
\pgfpathlineto{\pgfqpoint{1.450589in}{1.417106in}}%
\pgfpathlineto{\pgfqpoint{1.685662in}{1.372424in}}%
\pgfpathlineto{\pgfqpoint{1.920734in}{1.319852in}}%
\pgfpathlineto{\pgfqpoint{2.155807in}{1.254766in}}%
\pgfpathlineto{\pgfqpoint{2.390880in}{1.193039in}}%
\pgfpathlineto{\pgfqpoint{2.625952in}{1.147221in}}%
\pgfpathlineto{\pgfqpoint{2.861025in}{1.090672in}}%
\pgfpathlineto{\pgfqpoint{3.096098in}{1.032391in}}%
\pgfpathlineto{\pgfqpoint{3.331170in}{0.967330in}}%
\pgfpathlineto{\pgfqpoint{3.566243in}{0.921884in}}%
\pgfpathlineto{\pgfqpoint{3.801316in}{0.869433in}}%
\pgfpathlineto{\pgfqpoint{4.036389in}{0.813738in}}%
\pgfusepath{stroke}%
\end{pgfscope}%
\begin{pgfscope}%
\pgfpathrectangle{\pgfqpoint{0.745371in}{0.566590in}}{\pgfqpoint{3.291018in}{1.828724in}}%
\pgfusepath{clip}%
\pgfsetbuttcap%
\pgfsetroundjoin%
\definecolor{currentfill}{rgb}{0.850000,0.324000,0.098000}%
\pgfsetfillcolor{currentfill}%
\pgfsetlinewidth{1.003750pt}%
\definecolor{currentstroke}{rgb}{0.850000,0.324000,0.098000}%
\pgfsetstrokecolor{currentstroke}%
\pgfsetdash{}{0pt}%
\pgfsys@defobject{currentmarker}{\pgfqpoint{-0.041667in}{-0.041667in}}{\pgfqpoint{0.041667in}{0.041667in}}{%
\pgfpathmoveto{\pgfqpoint{-0.041667in}{0.000000in}}%
\pgfpathlineto{\pgfqpoint{0.041667in}{0.000000in}}%
\pgfpathmoveto{\pgfqpoint{0.000000in}{-0.041667in}}%
\pgfpathlineto{\pgfqpoint{0.000000in}{0.041667in}}%
\pgfusepath{stroke,fill}%
}%
\begin{pgfscope}%
\pgfsys@transformshift{0.745371in}{1.773225in}%
\pgfsys@useobject{currentmarker}{}%
\end{pgfscope}%
\begin{pgfscope}%
\pgfsys@transformshift{0.980444in}{1.583077in}%
\pgfsys@useobject{currentmarker}{}%
\end{pgfscope}%
\begin{pgfscope}%
\pgfsys@transformshift{1.215516in}{1.492612in}%
\pgfsys@useobject{currentmarker}{}%
\end{pgfscope}%
\begin{pgfscope}%
\pgfsys@transformshift{1.450589in}{1.417106in}%
\pgfsys@useobject{currentmarker}{}%
\end{pgfscope}%
\begin{pgfscope}%
\pgfsys@transformshift{1.685662in}{1.372424in}%
\pgfsys@useobject{currentmarker}{}%
\end{pgfscope}%
\begin{pgfscope}%
\pgfsys@transformshift{1.920734in}{1.319852in}%
\pgfsys@useobject{currentmarker}{}%
\end{pgfscope}%
\begin{pgfscope}%
\pgfsys@transformshift{2.155807in}{1.254766in}%
\pgfsys@useobject{currentmarker}{}%
\end{pgfscope}%
\begin{pgfscope}%
\pgfsys@transformshift{2.390880in}{1.193039in}%
\pgfsys@useobject{currentmarker}{}%
\end{pgfscope}%
\begin{pgfscope}%
\pgfsys@transformshift{2.625952in}{1.147221in}%
\pgfsys@useobject{currentmarker}{}%
\end{pgfscope}%
\begin{pgfscope}%
\pgfsys@transformshift{2.861025in}{1.090672in}%
\pgfsys@useobject{currentmarker}{}%
\end{pgfscope}%
\begin{pgfscope}%
\pgfsys@transformshift{3.096098in}{1.032391in}%
\pgfsys@useobject{currentmarker}{}%
\end{pgfscope}%
\begin{pgfscope}%
\pgfsys@transformshift{3.331170in}{0.967330in}%
\pgfsys@useobject{currentmarker}{}%
\end{pgfscope}%
\begin{pgfscope}%
\pgfsys@transformshift{3.566243in}{0.921884in}%
\pgfsys@useobject{currentmarker}{}%
\end{pgfscope}%
\begin{pgfscope}%
\pgfsys@transformshift{3.801316in}{0.869433in}%
\pgfsys@useobject{currentmarker}{}%
\end{pgfscope}%
\begin{pgfscope}%
\pgfsys@transformshift{4.036389in}{0.813738in}%
\pgfsys@useobject{currentmarker}{}%
\end{pgfscope}%
\end{pgfscope}%
\begin{pgfscope}%
\pgfsetrectcap%
\pgfsetmiterjoin%
\pgfsetlinewidth{0.803000pt}%
\definecolor{currentstroke}{rgb}{0.000000,0.000000,0.000000}%
\pgfsetstrokecolor{currentstroke}%
\pgfsetdash{}{0pt}%
\pgfpathmoveto{\pgfqpoint{0.745371in}{0.566590in}}%
\pgfpathlineto{\pgfqpoint{0.745371in}{2.395314in}}%
\pgfusepath{stroke}%
\end{pgfscope}%
\begin{pgfscope}%
\pgfsetrectcap%
\pgfsetmiterjoin%
\pgfsetlinewidth{0.803000pt}%
\definecolor{currentstroke}{rgb}{0.000000,0.000000,0.000000}%
\pgfsetstrokecolor{currentstroke}%
\pgfsetdash{}{0pt}%
\pgfpathmoveto{\pgfqpoint{4.036389in}{0.566590in}}%
\pgfpathlineto{\pgfqpoint{4.036389in}{2.395314in}}%
\pgfusepath{stroke}%
\end{pgfscope}%
\begin{pgfscope}%
\pgfsetrectcap%
\pgfsetmiterjoin%
\pgfsetlinewidth{0.803000pt}%
\definecolor{currentstroke}{rgb}{0.000000,0.000000,0.000000}%
\pgfsetstrokecolor{currentstroke}%
\pgfsetdash{}{0pt}%
\pgfpathmoveto{\pgfqpoint{0.745371in}{0.566590in}}%
\pgfpathlineto{\pgfqpoint{4.036389in}{0.566590in}}%
\pgfusepath{stroke}%
\end{pgfscope}%
\begin{pgfscope}%
\pgfsetrectcap%
\pgfsetmiterjoin%
\pgfsetlinewidth{0.803000pt}%
\definecolor{currentstroke}{rgb}{0.000000,0.000000,0.000000}%
\pgfsetstrokecolor{currentstroke}%
\pgfsetdash{}{0pt}%
\pgfpathmoveto{\pgfqpoint{0.745371in}{2.395314in}}%
\pgfpathlineto{\pgfqpoint{4.036389in}{2.395314in}}%
\pgfusepath{stroke}%
\end{pgfscope}%
\begin{pgfscope}%
\pgfsetbuttcap%
\pgfsetmiterjoin%
\definecolor{currentfill}{rgb}{1.000000,1.000000,1.000000}%
\pgfsetfillcolor{currentfill}%
\pgfsetfillopacity{0.800000}%
\pgfsetlinewidth{1.003750pt}%
\definecolor{currentstroke}{rgb}{0.800000,0.800000,0.800000}%
\pgfsetstrokecolor{currentstroke}%
\pgfsetstrokeopacity{0.800000}%
\pgfsetdash{}{0pt}%
\pgfpathmoveto{\pgfqpoint{2.924163in}{1.946703in}}%
\pgfpathlineto{\pgfqpoint{3.948889in}{1.946703in}}%
\pgfpathquadraticcurveto{\pgfqpoint{3.973889in}{1.946703in}}{\pgfqpoint{3.973889in}{1.971703in}}%
\pgfpathlineto{\pgfqpoint{3.973889in}{2.307814in}}%
\pgfpathquadraticcurveto{\pgfqpoint{3.973889in}{2.332814in}}{\pgfqpoint{3.948889in}{2.332814in}}%
\pgfpathlineto{\pgfqpoint{2.924163in}{2.332814in}}%
\pgfpathquadraticcurveto{\pgfqpoint{2.899163in}{2.332814in}}{\pgfqpoint{2.899163in}{2.307814in}}%
\pgfpathlineto{\pgfqpoint{2.899163in}{1.971703in}}%
\pgfpathquadraticcurveto{\pgfqpoint{2.899163in}{1.946703in}}{\pgfqpoint{2.924163in}{1.946703in}}%
\pgfpathclose%
\pgfusepath{stroke,fill}%
\end{pgfscope}%
\begin{pgfscope}%
\pgfsetbuttcap%
\pgfsetroundjoin%
\definecolor{currentfill}{rgb}{0.000000,0.000000,0.000000}%
\pgfsetfillcolor{currentfill}%
\pgfsetfillopacity{0.000000}%
\pgfsetlinewidth{1.003750pt}%
\definecolor{currentstroke}{rgb}{0.000000,0.447000,0.741000}%
\pgfsetstrokecolor{currentstroke}%
\pgfsetdash{}{0pt}%
\pgfsys@defobject{currentmarker}{\pgfqpoint{-0.041667in}{-0.041667in}}{\pgfqpoint{0.041667in}{0.041667in}}{%
\pgfpathmoveto{\pgfqpoint{0.000000in}{-0.041667in}}%
\pgfpathcurveto{\pgfqpoint{0.011050in}{-0.041667in}}{\pgfqpoint{0.021649in}{-0.037276in}}{\pgfqpoint{0.029463in}{-0.029463in}}%
\pgfpathcurveto{\pgfqpoint{0.037276in}{-0.021649in}}{\pgfqpoint{0.041667in}{-0.011050in}}{\pgfqpoint{0.041667in}{0.000000in}}%
\pgfpathcurveto{\pgfqpoint{0.041667in}{0.011050in}}{\pgfqpoint{0.037276in}{0.021649in}}{\pgfqpoint{0.029463in}{0.029463in}}%
\pgfpathcurveto{\pgfqpoint{0.021649in}{0.037276in}}{\pgfqpoint{0.011050in}{0.041667in}}{\pgfqpoint{0.000000in}{0.041667in}}%
\pgfpathcurveto{\pgfqpoint{-0.011050in}{0.041667in}}{\pgfqpoint{-0.021649in}{0.037276in}}{\pgfqpoint{-0.029463in}{0.029463in}}%
\pgfpathcurveto{\pgfqpoint{-0.037276in}{0.021649in}}{\pgfqpoint{-0.041667in}{0.011050in}}{\pgfqpoint{-0.041667in}{0.000000in}}%
\pgfpathcurveto{\pgfqpoint{-0.041667in}{-0.011050in}}{\pgfqpoint{-0.037276in}{-0.021649in}}{\pgfqpoint{-0.029463in}{-0.029463in}}%
\pgfpathcurveto{\pgfqpoint{-0.021649in}{-0.037276in}}{\pgfqpoint{-0.011050in}{-0.041667in}}{\pgfqpoint{0.000000in}{-0.041667in}}%
\pgfpathclose%
\pgfusepath{stroke,fill}%
}%
\begin{pgfscope}%
\pgfsys@transformshift{3.074163in}{2.239064in}%
\pgfsys@useobject{currentmarker}{}%
\end{pgfscope}%
\end{pgfscope}%
\begin{pgfscope}%
\definecolor{textcolor}{rgb}{0.000000,0.000000,0.000000}%
\pgfsetstrokecolor{textcolor}%
\pgfsetfillcolor{textcolor}%
\pgftext[x=3.299163in,y=2.195314in,left,base]{\color{textcolor}\rmfamily\fontsize{9.000000}{10.800000}\selectfont \(\displaystyle \gamma_{12} = \) -0.10}%
\end{pgfscope}%
\begin{pgfscope}%
\pgfsetbuttcap%
\pgfsetroundjoin%
\definecolor{currentfill}{rgb}{0.850000,0.324000,0.098000}%
\pgfsetfillcolor{currentfill}%
\pgfsetlinewidth{1.003750pt}%
\definecolor{currentstroke}{rgb}{0.850000,0.324000,0.098000}%
\pgfsetstrokecolor{currentstroke}%
\pgfsetdash{}{0pt}%
\pgfsys@defobject{currentmarker}{\pgfqpoint{-0.041667in}{-0.041667in}}{\pgfqpoint{0.041667in}{0.041667in}}{%
\pgfpathmoveto{\pgfqpoint{-0.041667in}{0.000000in}}%
\pgfpathlineto{\pgfqpoint{0.041667in}{0.000000in}}%
\pgfpathmoveto{\pgfqpoint{0.000000in}{-0.041667in}}%
\pgfpathlineto{\pgfqpoint{0.000000in}{0.041667in}}%
\pgfusepath{stroke,fill}%
}%
\begin{pgfscope}%
\pgfsys@transformshift{3.074163in}{2.064759in}%
\pgfsys@useobject{currentmarker}{}%
\end{pgfscope}%
\end{pgfscope}%
\begin{pgfscope}%
\definecolor{textcolor}{rgb}{0.000000,0.000000,0.000000}%
\pgfsetstrokecolor{textcolor}%
\pgfsetfillcolor{textcolor}%
\pgftext[x=3.299163in,y=2.021009in,left,base]{\color{textcolor}\rmfamily\fontsize{9.000000}{10.800000}\selectfont \(\displaystyle \gamma_{13} = \) -0.13}%
\end{pgfscope}%
\end{pgfpicture}%
\makeatother%
\endgroup%
}
					\caption{Cluster III}
					\label{SubFig:Cluster_III_real}
				\end{subfigure}
				~
				\begin{subfigure}[h]{0.5\textwidth}
					\centering
					\resizebox{\linewidth}{!}{%% Creator: Matplotlib, PGF backend
%%
%% To include the figure in your LaTeX document, write
%%   \input{<filename>.pgf}
%%
%% Make sure the required packages are loaded in your preamble
%%   \usepackage{pgf}
%%
%% and, on pdftex
%%   \usepackage[utf8]{inputenc}\DeclareUnicodeCharacter{2212}{-}
%%
%% or, on luatex and xetex
%%   \usepackage{unicode-math}
%%
%% Figures using additional raster images can only be included by \input if
%% they are in the same directory as the main LaTeX file. For loading figures
%% from other directories you can use the `import` package
%%   \usepackage{import}
%%
%% and then include the figures with
%%   \import{<path to file>}{<filename>.pgf}
%%
%% Matplotlib used the following preamble
%%   \usepackage[utf8x]{inputenc}
%%   \usepackage[T1]{fontenc}
%%   \usepackage{amsmath,amssymb,amsfonts}
%%
\begingroup%
\makeatletter%
\begin{pgfpicture}%
\pgfpathrectangle{\pgfpointorigin}{\pgfqpoint{4.136389in}{2.495314in}}%
\pgfusepath{use as bounding box, clip}%
\begin{pgfscope}%
\pgfsetbuttcap%
\pgfsetmiterjoin%
\definecolor{currentfill}{rgb}{1.000000,1.000000,1.000000}%
\pgfsetfillcolor{currentfill}%
\pgfsetlinewidth{0.000000pt}%
\definecolor{currentstroke}{rgb}{1.000000,1.000000,1.000000}%
\pgfsetstrokecolor{currentstroke}%
\pgfsetdash{}{0pt}%
\pgfpathmoveto{\pgfqpoint{0.000000in}{0.000000in}}%
\pgfpathlineto{\pgfqpoint{4.136389in}{0.000000in}}%
\pgfpathlineto{\pgfqpoint{4.136389in}{2.495314in}}%
\pgfpathlineto{\pgfqpoint{0.000000in}{2.495314in}}%
\pgfpathclose%
\pgfusepath{fill}%
\end{pgfscope}%
\begin{pgfscope}%
\pgfsetbuttcap%
\pgfsetmiterjoin%
\definecolor{currentfill}{rgb}{1.000000,1.000000,1.000000}%
\pgfsetfillcolor{currentfill}%
\pgfsetlinewidth{0.000000pt}%
\definecolor{currentstroke}{rgb}{0.000000,0.000000,0.000000}%
\pgfsetstrokecolor{currentstroke}%
\pgfsetstrokeopacity{0.000000}%
\pgfsetdash{}{0pt}%
\pgfpathmoveto{\pgfqpoint{0.745371in}{0.566590in}}%
\pgfpathlineto{\pgfqpoint{4.036389in}{0.566590in}}%
\pgfpathlineto{\pgfqpoint{4.036389in}{2.395314in}}%
\pgfpathlineto{\pgfqpoint{0.745371in}{2.395314in}}%
\pgfpathclose%
\pgfusepath{fill}%
\end{pgfscope}%
\begin{pgfscope}%
\pgfpathrectangle{\pgfqpoint{0.745371in}{0.566590in}}{\pgfqpoint{3.291018in}{1.828724in}}%
\pgfusepath{clip}%
\pgfsetrectcap%
\pgfsetroundjoin%
\pgfsetlinewidth{0.803000pt}%
\definecolor{currentstroke}{rgb}{0.690196,0.690196,0.690196}%
\pgfsetstrokecolor{currentstroke}%
\pgfsetdash{}{0pt}%
\pgfpathmoveto{\pgfqpoint{0.745371in}{0.566590in}}%
\pgfpathlineto{\pgfqpoint{0.745371in}{2.395314in}}%
\pgfusepath{stroke}%
\end{pgfscope}%
\begin{pgfscope}%
\pgfsetbuttcap%
\pgfsetroundjoin%
\definecolor{currentfill}{rgb}{0.000000,0.000000,0.000000}%
\pgfsetfillcolor{currentfill}%
\pgfsetlinewidth{0.803000pt}%
\definecolor{currentstroke}{rgb}{0.000000,0.000000,0.000000}%
\pgfsetstrokecolor{currentstroke}%
\pgfsetdash{}{0pt}%
\pgfsys@defobject{currentmarker}{\pgfqpoint{0.000000in}{-0.048611in}}{\pgfqpoint{0.000000in}{0.000000in}}{%
\pgfpathmoveto{\pgfqpoint{0.000000in}{0.000000in}}%
\pgfpathlineto{\pgfqpoint{0.000000in}{-0.048611in}}%
\pgfusepath{stroke,fill}%
}%
\begin{pgfscope}%
\pgfsys@transformshift{0.745371in}{0.566590in}%
\pgfsys@useobject{currentmarker}{}%
\end{pgfscope}%
\end{pgfscope}%
\begin{pgfscope}%
\definecolor{textcolor}{rgb}{0.000000,0.000000,0.000000}%
\pgfsetstrokecolor{textcolor}%
\pgfsetfillcolor{textcolor}%
\pgftext[x=0.745371in,y=0.469368in,,top]{\color{textcolor}\rmfamily\fontsize{12.000000}{14.400000}\selectfont \(\displaystyle {-10}\)}%
\end{pgfscope}%
\begin{pgfscope}%
\pgfpathrectangle{\pgfqpoint{0.745371in}{0.566590in}}{\pgfqpoint{3.291018in}{1.828724in}}%
\pgfusepath{clip}%
\pgfsetrectcap%
\pgfsetroundjoin%
\pgfsetlinewidth{0.803000pt}%
\definecolor{currentstroke}{rgb}{0.690196,0.690196,0.690196}%
\pgfsetstrokecolor{currentstroke}%
\pgfsetdash{}{0pt}%
\pgfpathmoveto{\pgfqpoint{1.251681in}{0.566590in}}%
\pgfpathlineto{\pgfqpoint{1.251681in}{2.395314in}}%
\pgfusepath{stroke}%
\end{pgfscope}%
\begin{pgfscope}%
\pgfsetbuttcap%
\pgfsetroundjoin%
\definecolor{currentfill}{rgb}{0.000000,0.000000,0.000000}%
\pgfsetfillcolor{currentfill}%
\pgfsetlinewidth{0.803000pt}%
\definecolor{currentstroke}{rgb}{0.000000,0.000000,0.000000}%
\pgfsetstrokecolor{currentstroke}%
\pgfsetdash{}{0pt}%
\pgfsys@defobject{currentmarker}{\pgfqpoint{0.000000in}{-0.048611in}}{\pgfqpoint{0.000000in}{0.000000in}}{%
\pgfpathmoveto{\pgfqpoint{0.000000in}{0.000000in}}%
\pgfpathlineto{\pgfqpoint{0.000000in}{-0.048611in}}%
\pgfusepath{stroke,fill}%
}%
\begin{pgfscope}%
\pgfsys@transformshift{1.251681in}{0.566590in}%
\pgfsys@useobject{currentmarker}{}%
\end{pgfscope}%
\end{pgfscope}%
\begin{pgfscope}%
\definecolor{textcolor}{rgb}{0.000000,0.000000,0.000000}%
\pgfsetstrokecolor{textcolor}%
\pgfsetfillcolor{textcolor}%
\pgftext[x=1.251681in,y=0.469368in,,top]{\color{textcolor}\rmfamily\fontsize{12.000000}{14.400000}\selectfont \(\displaystyle {0}\)}%
\end{pgfscope}%
\begin{pgfscope}%
\pgfpathrectangle{\pgfqpoint{0.745371in}{0.566590in}}{\pgfqpoint{3.291018in}{1.828724in}}%
\pgfusepath{clip}%
\pgfsetrectcap%
\pgfsetroundjoin%
\pgfsetlinewidth{0.803000pt}%
\definecolor{currentstroke}{rgb}{0.690196,0.690196,0.690196}%
\pgfsetstrokecolor{currentstroke}%
\pgfsetdash{}{0pt}%
\pgfpathmoveto{\pgfqpoint{1.757992in}{0.566590in}}%
\pgfpathlineto{\pgfqpoint{1.757992in}{2.395314in}}%
\pgfusepath{stroke}%
\end{pgfscope}%
\begin{pgfscope}%
\pgfsetbuttcap%
\pgfsetroundjoin%
\definecolor{currentfill}{rgb}{0.000000,0.000000,0.000000}%
\pgfsetfillcolor{currentfill}%
\pgfsetlinewidth{0.803000pt}%
\definecolor{currentstroke}{rgb}{0.000000,0.000000,0.000000}%
\pgfsetstrokecolor{currentstroke}%
\pgfsetdash{}{0pt}%
\pgfsys@defobject{currentmarker}{\pgfqpoint{0.000000in}{-0.048611in}}{\pgfqpoint{0.000000in}{0.000000in}}{%
\pgfpathmoveto{\pgfqpoint{0.000000in}{0.000000in}}%
\pgfpathlineto{\pgfqpoint{0.000000in}{-0.048611in}}%
\pgfusepath{stroke,fill}%
}%
\begin{pgfscope}%
\pgfsys@transformshift{1.757992in}{0.566590in}%
\pgfsys@useobject{currentmarker}{}%
\end{pgfscope}%
\end{pgfscope}%
\begin{pgfscope}%
\definecolor{textcolor}{rgb}{0.000000,0.000000,0.000000}%
\pgfsetstrokecolor{textcolor}%
\pgfsetfillcolor{textcolor}%
\pgftext[x=1.757992in,y=0.469368in,,top]{\color{textcolor}\rmfamily\fontsize{12.000000}{14.400000}\selectfont \(\displaystyle {10}\)}%
\end{pgfscope}%
\begin{pgfscope}%
\pgfpathrectangle{\pgfqpoint{0.745371in}{0.566590in}}{\pgfqpoint{3.291018in}{1.828724in}}%
\pgfusepath{clip}%
\pgfsetrectcap%
\pgfsetroundjoin%
\pgfsetlinewidth{0.803000pt}%
\definecolor{currentstroke}{rgb}{0.690196,0.690196,0.690196}%
\pgfsetstrokecolor{currentstroke}%
\pgfsetdash{}{0pt}%
\pgfpathmoveto{\pgfqpoint{2.264302in}{0.566590in}}%
\pgfpathlineto{\pgfqpoint{2.264302in}{2.395314in}}%
\pgfusepath{stroke}%
\end{pgfscope}%
\begin{pgfscope}%
\pgfsetbuttcap%
\pgfsetroundjoin%
\definecolor{currentfill}{rgb}{0.000000,0.000000,0.000000}%
\pgfsetfillcolor{currentfill}%
\pgfsetlinewidth{0.803000pt}%
\definecolor{currentstroke}{rgb}{0.000000,0.000000,0.000000}%
\pgfsetstrokecolor{currentstroke}%
\pgfsetdash{}{0pt}%
\pgfsys@defobject{currentmarker}{\pgfqpoint{0.000000in}{-0.048611in}}{\pgfqpoint{0.000000in}{0.000000in}}{%
\pgfpathmoveto{\pgfqpoint{0.000000in}{0.000000in}}%
\pgfpathlineto{\pgfqpoint{0.000000in}{-0.048611in}}%
\pgfusepath{stroke,fill}%
}%
\begin{pgfscope}%
\pgfsys@transformshift{2.264302in}{0.566590in}%
\pgfsys@useobject{currentmarker}{}%
\end{pgfscope}%
\end{pgfscope}%
\begin{pgfscope}%
\definecolor{textcolor}{rgb}{0.000000,0.000000,0.000000}%
\pgfsetstrokecolor{textcolor}%
\pgfsetfillcolor{textcolor}%
\pgftext[x=2.264302in,y=0.469368in,,top]{\color{textcolor}\rmfamily\fontsize{12.000000}{14.400000}\selectfont \(\displaystyle {20}\)}%
\end{pgfscope}%
\begin{pgfscope}%
\pgfpathrectangle{\pgfqpoint{0.745371in}{0.566590in}}{\pgfqpoint{3.291018in}{1.828724in}}%
\pgfusepath{clip}%
\pgfsetrectcap%
\pgfsetroundjoin%
\pgfsetlinewidth{0.803000pt}%
\definecolor{currentstroke}{rgb}{0.690196,0.690196,0.690196}%
\pgfsetstrokecolor{currentstroke}%
\pgfsetdash{}{0pt}%
\pgfpathmoveto{\pgfqpoint{2.770613in}{0.566590in}}%
\pgfpathlineto{\pgfqpoint{2.770613in}{2.395314in}}%
\pgfusepath{stroke}%
\end{pgfscope}%
\begin{pgfscope}%
\pgfsetbuttcap%
\pgfsetroundjoin%
\definecolor{currentfill}{rgb}{0.000000,0.000000,0.000000}%
\pgfsetfillcolor{currentfill}%
\pgfsetlinewidth{0.803000pt}%
\definecolor{currentstroke}{rgb}{0.000000,0.000000,0.000000}%
\pgfsetstrokecolor{currentstroke}%
\pgfsetdash{}{0pt}%
\pgfsys@defobject{currentmarker}{\pgfqpoint{0.000000in}{-0.048611in}}{\pgfqpoint{0.000000in}{0.000000in}}{%
\pgfpathmoveto{\pgfqpoint{0.000000in}{0.000000in}}%
\pgfpathlineto{\pgfqpoint{0.000000in}{-0.048611in}}%
\pgfusepath{stroke,fill}%
}%
\begin{pgfscope}%
\pgfsys@transformshift{2.770613in}{0.566590in}%
\pgfsys@useobject{currentmarker}{}%
\end{pgfscope}%
\end{pgfscope}%
\begin{pgfscope}%
\definecolor{textcolor}{rgb}{0.000000,0.000000,0.000000}%
\pgfsetstrokecolor{textcolor}%
\pgfsetfillcolor{textcolor}%
\pgftext[x=2.770613in,y=0.469368in,,top]{\color{textcolor}\rmfamily\fontsize{12.000000}{14.400000}\selectfont \(\displaystyle {30}\)}%
\end{pgfscope}%
\begin{pgfscope}%
\pgfpathrectangle{\pgfqpoint{0.745371in}{0.566590in}}{\pgfqpoint{3.291018in}{1.828724in}}%
\pgfusepath{clip}%
\pgfsetrectcap%
\pgfsetroundjoin%
\pgfsetlinewidth{0.803000pt}%
\definecolor{currentstroke}{rgb}{0.690196,0.690196,0.690196}%
\pgfsetstrokecolor{currentstroke}%
\pgfsetdash{}{0pt}%
\pgfpathmoveto{\pgfqpoint{3.276923in}{0.566590in}}%
\pgfpathlineto{\pgfqpoint{3.276923in}{2.395314in}}%
\pgfusepath{stroke}%
\end{pgfscope}%
\begin{pgfscope}%
\pgfsetbuttcap%
\pgfsetroundjoin%
\definecolor{currentfill}{rgb}{0.000000,0.000000,0.000000}%
\pgfsetfillcolor{currentfill}%
\pgfsetlinewidth{0.803000pt}%
\definecolor{currentstroke}{rgb}{0.000000,0.000000,0.000000}%
\pgfsetstrokecolor{currentstroke}%
\pgfsetdash{}{0pt}%
\pgfsys@defobject{currentmarker}{\pgfqpoint{0.000000in}{-0.048611in}}{\pgfqpoint{0.000000in}{0.000000in}}{%
\pgfpathmoveto{\pgfqpoint{0.000000in}{0.000000in}}%
\pgfpathlineto{\pgfqpoint{0.000000in}{-0.048611in}}%
\pgfusepath{stroke,fill}%
}%
\begin{pgfscope}%
\pgfsys@transformshift{3.276923in}{0.566590in}%
\pgfsys@useobject{currentmarker}{}%
\end{pgfscope}%
\end{pgfscope}%
\begin{pgfscope}%
\definecolor{textcolor}{rgb}{0.000000,0.000000,0.000000}%
\pgfsetstrokecolor{textcolor}%
\pgfsetfillcolor{textcolor}%
\pgftext[x=3.276923in,y=0.469368in,,top]{\color{textcolor}\rmfamily\fontsize{12.000000}{14.400000}\selectfont \(\displaystyle {40}\)}%
\end{pgfscope}%
\begin{pgfscope}%
\pgfpathrectangle{\pgfqpoint{0.745371in}{0.566590in}}{\pgfqpoint{3.291018in}{1.828724in}}%
\pgfusepath{clip}%
\pgfsetrectcap%
\pgfsetroundjoin%
\pgfsetlinewidth{0.803000pt}%
\definecolor{currentstroke}{rgb}{0.690196,0.690196,0.690196}%
\pgfsetstrokecolor{currentstroke}%
\pgfsetdash{}{0pt}%
\pgfpathmoveto{\pgfqpoint{3.783233in}{0.566590in}}%
\pgfpathlineto{\pgfqpoint{3.783233in}{2.395314in}}%
\pgfusepath{stroke}%
\end{pgfscope}%
\begin{pgfscope}%
\pgfsetbuttcap%
\pgfsetroundjoin%
\definecolor{currentfill}{rgb}{0.000000,0.000000,0.000000}%
\pgfsetfillcolor{currentfill}%
\pgfsetlinewidth{0.803000pt}%
\definecolor{currentstroke}{rgb}{0.000000,0.000000,0.000000}%
\pgfsetstrokecolor{currentstroke}%
\pgfsetdash{}{0pt}%
\pgfsys@defobject{currentmarker}{\pgfqpoint{0.000000in}{-0.048611in}}{\pgfqpoint{0.000000in}{0.000000in}}{%
\pgfpathmoveto{\pgfqpoint{0.000000in}{0.000000in}}%
\pgfpathlineto{\pgfqpoint{0.000000in}{-0.048611in}}%
\pgfusepath{stroke,fill}%
}%
\begin{pgfscope}%
\pgfsys@transformshift{3.783233in}{0.566590in}%
\pgfsys@useobject{currentmarker}{}%
\end{pgfscope}%
\end{pgfscope}%
\begin{pgfscope}%
\definecolor{textcolor}{rgb}{0.000000,0.000000,0.000000}%
\pgfsetstrokecolor{textcolor}%
\pgfsetfillcolor{textcolor}%
\pgftext[x=3.783233in,y=0.469368in,,top]{\color{textcolor}\rmfamily\fontsize{12.000000}{14.400000}\selectfont \(\displaystyle {50}\)}%
\end{pgfscope}%
\begin{pgfscope}%
\definecolor{textcolor}{rgb}{0.000000,0.000000,0.000000}%
\pgfsetstrokecolor{textcolor}%
\pgfsetfillcolor{textcolor}%
\pgftext[x=2.390880in,y=0.266626in,,top]{\color{textcolor}\rmfamily\fontsize{12.000000}{14.400000}\selectfont SNR [dB]}%
\end{pgfscope}%
\begin{pgfscope}%
\pgfpathrectangle{\pgfqpoint{0.745371in}{0.566590in}}{\pgfqpoint{3.291018in}{1.828724in}}%
\pgfusepath{clip}%
\pgfsetrectcap%
\pgfsetroundjoin%
\pgfsetlinewidth{0.803000pt}%
\definecolor{currentstroke}{rgb}{0.690196,0.690196,0.690196}%
\pgfsetstrokecolor{currentstroke}%
\pgfsetdash{}{0pt}%
\pgfpathmoveto{\pgfqpoint{0.745371in}{0.566590in}}%
\pgfpathlineto{\pgfqpoint{4.036389in}{0.566590in}}%
\pgfusepath{stroke}%
\end{pgfscope}%
\begin{pgfscope}%
\pgfsetbuttcap%
\pgfsetroundjoin%
\definecolor{currentfill}{rgb}{0.000000,0.000000,0.000000}%
\pgfsetfillcolor{currentfill}%
\pgfsetlinewidth{0.803000pt}%
\definecolor{currentstroke}{rgb}{0.000000,0.000000,0.000000}%
\pgfsetstrokecolor{currentstroke}%
\pgfsetdash{}{0pt}%
\pgfsys@defobject{currentmarker}{\pgfqpoint{-0.048611in}{0.000000in}}{\pgfqpoint{-0.000000in}{0.000000in}}{%
\pgfpathmoveto{\pgfqpoint{-0.000000in}{0.000000in}}%
\pgfpathlineto{\pgfqpoint{-0.048611in}{0.000000in}}%
\pgfusepath{stroke,fill}%
}%
\begin{pgfscope}%
\pgfsys@transformshift{0.745371in}{0.566590in}%
\pgfsys@useobject{currentmarker}{}%
\end{pgfscope}%
\end{pgfscope}%
\begin{pgfscope}%
\definecolor{textcolor}{rgb}{0.000000,0.000000,0.000000}%
\pgfsetstrokecolor{textcolor}%
\pgfsetfillcolor{textcolor}%
\pgftext[x=0.327160in, y=0.509197in, left, base]{\color{textcolor}\rmfamily\fontsize{12.000000}{14.400000}\selectfont \(\displaystyle {10^{-4}}\)}%
\end{pgfscope}%
\begin{pgfscope}%
\pgfpathrectangle{\pgfqpoint{0.745371in}{0.566590in}}{\pgfqpoint{3.291018in}{1.828724in}}%
\pgfusepath{clip}%
\pgfsetrectcap%
\pgfsetroundjoin%
\pgfsetlinewidth{0.803000pt}%
\definecolor{currentstroke}{rgb}{0.690196,0.690196,0.690196}%
\pgfsetstrokecolor{currentstroke}%
\pgfsetdash{}{0pt}%
\pgfpathmoveto{\pgfqpoint{0.745371in}{1.104776in}}%
\pgfpathlineto{\pgfqpoint{4.036389in}{1.104776in}}%
\pgfusepath{stroke}%
\end{pgfscope}%
\begin{pgfscope}%
\pgfsetbuttcap%
\pgfsetroundjoin%
\definecolor{currentfill}{rgb}{0.000000,0.000000,0.000000}%
\pgfsetfillcolor{currentfill}%
\pgfsetlinewidth{0.803000pt}%
\definecolor{currentstroke}{rgb}{0.000000,0.000000,0.000000}%
\pgfsetstrokecolor{currentstroke}%
\pgfsetdash{}{0pt}%
\pgfsys@defobject{currentmarker}{\pgfqpoint{-0.048611in}{0.000000in}}{\pgfqpoint{-0.000000in}{0.000000in}}{%
\pgfpathmoveto{\pgfqpoint{-0.000000in}{0.000000in}}%
\pgfpathlineto{\pgfqpoint{-0.048611in}{0.000000in}}%
\pgfusepath{stroke,fill}%
}%
\begin{pgfscope}%
\pgfsys@transformshift{0.745371in}{1.104776in}%
\pgfsys@useobject{currentmarker}{}%
\end{pgfscope}%
\end{pgfscope}%
\begin{pgfscope}%
\definecolor{textcolor}{rgb}{0.000000,0.000000,0.000000}%
\pgfsetstrokecolor{textcolor}%
\pgfsetfillcolor{textcolor}%
\pgftext[x=0.327160in, y=1.047383in, left, base]{\color{textcolor}\rmfamily\fontsize{12.000000}{14.400000}\selectfont \(\displaystyle {10^{-2}}\)}%
\end{pgfscope}%
\begin{pgfscope}%
\pgfpathrectangle{\pgfqpoint{0.745371in}{0.566590in}}{\pgfqpoint{3.291018in}{1.828724in}}%
\pgfusepath{clip}%
\pgfsetrectcap%
\pgfsetroundjoin%
\pgfsetlinewidth{0.803000pt}%
\definecolor{currentstroke}{rgb}{0.690196,0.690196,0.690196}%
\pgfsetstrokecolor{currentstroke}%
\pgfsetdash{}{0pt}%
\pgfpathmoveto{\pgfqpoint{0.745371in}{1.642962in}}%
\pgfpathlineto{\pgfqpoint{4.036389in}{1.642962in}}%
\pgfusepath{stroke}%
\end{pgfscope}%
\begin{pgfscope}%
\pgfsetbuttcap%
\pgfsetroundjoin%
\definecolor{currentfill}{rgb}{0.000000,0.000000,0.000000}%
\pgfsetfillcolor{currentfill}%
\pgfsetlinewidth{0.803000pt}%
\definecolor{currentstroke}{rgb}{0.000000,0.000000,0.000000}%
\pgfsetstrokecolor{currentstroke}%
\pgfsetdash{}{0pt}%
\pgfsys@defobject{currentmarker}{\pgfqpoint{-0.048611in}{0.000000in}}{\pgfqpoint{-0.000000in}{0.000000in}}{%
\pgfpathmoveto{\pgfqpoint{-0.000000in}{0.000000in}}%
\pgfpathlineto{\pgfqpoint{-0.048611in}{0.000000in}}%
\pgfusepath{stroke,fill}%
}%
\begin{pgfscope}%
\pgfsys@transformshift{0.745371in}{1.642962in}%
\pgfsys@useobject{currentmarker}{}%
\end{pgfscope}%
\end{pgfscope}%
\begin{pgfscope}%
\definecolor{textcolor}{rgb}{0.000000,0.000000,0.000000}%
\pgfsetstrokecolor{textcolor}%
\pgfsetfillcolor{textcolor}%
\pgftext[x=0.418983in, y=1.585569in, left, base]{\color{textcolor}\rmfamily\fontsize{12.000000}{14.400000}\selectfont \(\displaystyle {10^{0}}\)}%
\end{pgfscope}%
\begin{pgfscope}%
\pgfpathrectangle{\pgfqpoint{0.745371in}{0.566590in}}{\pgfqpoint{3.291018in}{1.828724in}}%
\pgfusepath{clip}%
\pgfsetrectcap%
\pgfsetroundjoin%
\pgfsetlinewidth{0.803000pt}%
\definecolor{currentstroke}{rgb}{0.690196,0.690196,0.690196}%
\pgfsetstrokecolor{currentstroke}%
\pgfsetdash{}{0pt}%
\pgfpathmoveto{\pgfqpoint{0.745371in}{2.181148in}}%
\pgfpathlineto{\pgfqpoint{4.036389in}{2.181148in}}%
\pgfusepath{stroke}%
\end{pgfscope}%
\begin{pgfscope}%
\pgfsetbuttcap%
\pgfsetroundjoin%
\definecolor{currentfill}{rgb}{0.000000,0.000000,0.000000}%
\pgfsetfillcolor{currentfill}%
\pgfsetlinewidth{0.803000pt}%
\definecolor{currentstroke}{rgb}{0.000000,0.000000,0.000000}%
\pgfsetstrokecolor{currentstroke}%
\pgfsetdash{}{0pt}%
\pgfsys@defobject{currentmarker}{\pgfqpoint{-0.048611in}{0.000000in}}{\pgfqpoint{-0.000000in}{0.000000in}}{%
\pgfpathmoveto{\pgfqpoint{-0.000000in}{0.000000in}}%
\pgfpathlineto{\pgfqpoint{-0.048611in}{0.000000in}}%
\pgfusepath{stroke,fill}%
}%
\begin{pgfscope}%
\pgfsys@transformshift{0.745371in}{2.181148in}%
\pgfsys@useobject{currentmarker}{}%
\end{pgfscope}%
\end{pgfscope}%
\begin{pgfscope}%
\definecolor{textcolor}{rgb}{0.000000,0.000000,0.000000}%
\pgfsetstrokecolor{textcolor}%
\pgfsetfillcolor{textcolor}%
\pgftext[x=0.418983in, y=2.123755in, left, base]{\color{textcolor}\rmfamily\fontsize{12.000000}{14.400000}\selectfont \(\displaystyle {10^{2}}\)}%
\end{pgfscope}%
\begin{pgfscope}%
\definecolor{textcolor}{rgb}{0.000000,0.000000,0.000000}%
\pgfsetstrokecolor{textcolor}%
\pgfsetfillcolor{textcolor}%
\pgftext[x=0.271605in,y=1.480952in,,bottom,rotate=90.000000]{\color{textcolor}\rmfamily\fontsize{12.000000}{14.400000}\selectfont \(\displaystyle \hat{\sigma}_{\gamma}(\mathrm{SNR})\)}%
\end{pgfscope}%
\begin{pgfscope}%
\pgfpathrectangle{\pgfqpoint{0.745371in}{0.566590in}}{\pgfqpoint{3.291018in}{1.828724in}}%
\pgfusepath{clip}%
\pgfsetbuttcap%
\pgfsetroundjoin%
\pgfsetlinewidth{1.505625pt}%
\definecolor{currentstroke}{rgb}{0.000000,0.447000,0.741000}%
\pgfsetstrokecolor{currentstroke}%
\pgfsetdash{{5.550000pt}{2.400000pt}}{0.000000pt}%
\pgfpathmoveto{\pgfqpoint{0.745371in}{2.148959in}}%
\pgfpathlineto{\pgfqpoint{0.842165in}{2.126522in}}%
\pgfpathlineto{\pgfqpoint{0.938960in}{2.087289in}}%
\pgfpathlineto{\pgfqpoint{1.035755in}{2.048158in}}%
\pgfpathlineto{\pgfqpoint{1.132549in}{2.166939in}}%
\pgfpathlineto{\pgfqpoint{1.229344in}{2.165274in}}%
\pgfpathlineto{\pgfqpoint{1.326139in}{2.122786in}}%
\pgfpathlineto{\pgfqpoint{1.422933in}{2.309155in}}%
\pgfpathlineto{\pgfqpoint{1.519728in}{2.084047in}}%
\pgfpathlineto{\pgfqpoint{1.616523in}{2.130138in}}%
\pgfpathlineto{\pgfqpoint{1.713317in}{2.108244in}}%
\pgfpathlineto{\pgfqpoint{1.810112in}{2.148157in}}%
\pgfpathlineto{\pgfqpoint{1.906906in}{2.113293in}}%
\pgfpathlineto{\pgfqpoint{2.003701in}{2.016854in}}%
\pgfpathlineto{\pgfqpoint{2.100496in}{1.730998in}}%
\pgfpathlineto{\pgfqpoint{2.197290in}{1.881740in}}%
\pgfpathlineto{\pgfqpoint{2.294085in}{1.833853in}}%
\pgfpathlineto{\pgfqpoint{2.390880in}{1.532674in}}%
\pgfpathlineto{\pgfqpoint{2.487674in}{1.498529in}}%
\pgfpathlineto{\pgfqpoint{2.584469in}{1.462641in}}%
\pgfpathlineto{\pgfqpoint{2.681264in}{1.430521in}}%
\pgfpathlineto{\pgfqpoint{2.778058in}{1.397688in}}%
\pgfpathlineto{\pgfqpoint{2.874853in}{1.391349in}}%
\pgfpathlineto{\pgfqpoint{2.971648in}{1.364475in}}%
\pgfpathlineto{\pgfqpoint{3.068442in}{1.336572in}}%
\pgfpathlineto{\pgfqpoint{3.165237in}{1.302981in}}%
\pgfpathlineto{\pgfqpoint{3.262031in}{1.284388in}}%
\pgfpathlineto{\pgfqpoint{3.358826in}{1.264449in}}%
\pgfpathlineto{\pgfqpoint{3.455621in}{1.238199in}}%
\pgfpathlineto{\pgfqpoint{3.552415in}{1.202495in}}%
\pgfpathlineto{\pgfqpoint{3.649210in}{1.179446in}}%
\pgfpathlineto{\pgfqpoint{3.746005in}{1.143465in}}%
\pgfpathlineto{\pgfqpoint{3.842799in}{1.124281in}}%
\pgfpathlineto{\pgfqpoint{3.939594in}{1.104376in}}%
\pgfpathlineto{\pgfqpoint{4.036389in}{1.079329in}}%
\pgfusepath{stroke}%
\end{pgfscope}%
\begin{pgfscope}%
\pgfpathrectangle{\pgfqpoint{0.745371in}{0.566590in}}{\pgfqpoint{3.291018in}{1.828724in}}%
\pgfusepath{clip}%
\pgfsetbuttcap%
\pgfsetroundjoin%
\definecolor{currentfill}{rgb}{0.000000,0.000000,0.000000}%
\pgfsetfillcolor{currentfill}%
\pgfsetfillopacity{0.000000}%
\pgfsetlinewidth{1.003750pt}%
\definecolor{currentstroke}{rgb}{0.000000,0.447000,0.741000}%
\pgfsetstrokecolor{currentstroke}%
\pgfsetdash{}{0pt}%
\pgfsys@defobject{currentmarker}{\pgfqpoint{-0.041667in}{-0.041667in}}{\pgfqpoint{0.041667in}{0.041667in}}{%
\pgfpathmoveto{\pgfqpoint{0.000000in}{-0.041667in}}%
\pgfpathcurveto{\pgfqpoint{0.011050in}{-0.041667in}}{\pgfqpoint{0.021649in}{-0.037276in}}{\pgfqpoint{0.029463in}{-0.029463in}}%
\pgfpathcurveto{\pgfqpoint{0.037276in}{-0.021649in}}{\pgfqpoint{0.041667in}{-0.011050in}}{\pgfqpoint{0.041667in}{0.000000in}}%
\pgfpathcurveto{\pgfqpoint{0.041667in}{0.011050in}}{\pgfqpoint{0.037276in}{0.021649in}}{\pgfqpoint{0.029463in}{0.029463in}}%
\pgfpathcurveto{\pgfqpoint{0.021649in}{0.037276in}}{\pgfqpoint{0.011050in}{0.041667in}}{\pgfqpoint{0.000000in}{0.041667in}}%
\pgfpathcurveto{\pgfqpoint{-0.011050in}{0.041667in}}{\pgfqpoint{-0.021649in}{0.037276in}}{\pgfqpoint{-0.029463in}{0.029463in}}%
\pgfpathcurveto{\pgfqpoint{-0.037276in}{0.021649in}}{\pgfqpoint{-0.041667in}{0.011050in}}{\pgfqpoint{-0.041667in}{0.000000in}}%
\pgfpathcurveto{\pgfqpoint{-0.041667in}{-0.011050in}}{\pgfqpoint{-0.037276in}{-0.021649in}}{\pgfqpoint{-0.029463in}{-0.029463in}}%
\pgfpathcurveto{\pgfqpoint{-0.021649in}{-0.037276in}}{\pgfqpoint{-0.011050in}{-0.041667in}}{\pgfqpoint{0.000000in}{-0.041667in}}%
\pgfpathclose%
\pgfusepath{stroke,fill}%
}%
\begin{pgfscope}%
\pgfsys@transformshift{0.745371in}{2.148959in}%
\pgfsys@useobject{currentmarker}{}%
\end{pgfscope}%
\begin{pgfscope}%
\pgfsys@transformshift{1.132549in}{2.166939in}%
\pgfsys@useobject{currentmarker}{}%
\end{pgfscope}%
\begin{pgfscope}%
\pgfsys@transformshift{1.519728in}{2.084047in}%
\pgfsys@useobject{currentmarker}{}%
\end{pgfscope}%
\begin{pgfscope}%
\pgfsys@transformshift{1.906906in}{2.113293in}%
\pgfsys@useobject{currentmarker}{}%
\end{pgfscope}%
\begin{pgfscope}%
\pgfsys@transformshift{2.294085in}{1.833853in}%
\pgfsys@useobject{currentmarker}{}%
\end{pgfscope}%
\begin{pgfscope}%
\pgfsys@transformshift{2.681264in}{1.430521in}%
\pgfsys@useobject{currentmarker}{}%
\end{pgfscope}%
\begin{pgfscope}%
\pgfsys@transformshift{3.068442in}{1.336572in}%
\pgfsys@useobject{currentmarker}{}%
\end{pgfscope}%
\begin{pgfscope}%
\pgfsys@transformshift{3.455621in}{1.238199in}%
\pgfsys@useobject{currentmarker}{}%
\end{pgfscope}%
\begin{pgfscope}%
\pgfsys@transformshift{3.842799in}{1.124281in}%
\pgfsys@useobject{currentmarker}{}%
\end{pgfscope}%
\end{pgfscope}%
\begin{pgfscope}%
\pgfpathrectangle{\pgfqpoint{0.745371in}{0.566590in}}{\pgfqpoint{3.291018in}{1.828724in}}%
\pgfusepath{clip}%
\pgfsetbuttcap%
\pgfsetroundjoin%
\pgfsetlinewidth{1.505625pt}%
\definecolor{currentstroke}{rgb}{0.850000,0.324000,0.098000}%
\pgfsetstrokecolor{currentstroke}%
\pgfsetdash{{5.550000pt}{2.400000pt}}{0.000000pt}%
\pgfpathmoveto{\pgfqpoint{0.745371in}{2.158977in}}%
\pgfpathlineto{\pgfqpoint{0.842165in}{2.183138in}}%
\pgfpathlineto{\pgfqpoint{0.938960in}{2.204174in}}%
\pgfpathlineto{\pgfqpoint{1.035755in}{2.021071in}}%
\pgfpathlineto{\pgfqpoint{1.132549in}{2.081334in}}%
\pgfpathlineto{\pgfqpoint{1.229344in}{2.138819in}}%
\pgfpathlineto{\pgfqpoint{1.326139in}{2.055210in}}%
\pgfpathlineto{\pgfqpoint{1.422933in}{1.708076in}}%
\pgfpathlineto{\pgfqpoint{1.519728in}{2.061225in}}%
\pgfpathlineto{\pgfqpoint{1.616523in}{2.120118in}}%
\pgfpathlineto{\pgfqpoint{1.713317in}{2.082769in}}%
\pgfpathlineto{\pgfqpoint{1.810112in}{2.096300in}}%
\pgfpathlineto{\pgfqpoint{1.906906in}{2.079035in}}%
\pgfpathlineto{\pgfqpoint{2.003701in}{1.797098in}}%
\pgfpathlineto{\pgfqpoint{2.100496in}{1.837684in}}%
\pgfpathlineto{\pgfqpoint{2.197290in}{1.773700in}}%
\pgfpathlineto{\pgfqpoint{2.294085in}{1.577723in}}%
\pgfpathlineto{\pgfqpoint{2.390880in}{1.903076in}}%
\pgfpathlineto{\pgfqpoint{2.487674in}{1.510842in}}%
\pgfpathlineto{\pgfqpoint{2.584469in}{1.487130in}}%
\pgfpathlineto{\pgfqpoint{2.681264in}{1.456972in}}%
\pgfpathlineto{\pgfqpoint{2.778058in}{1.430383in}}%
\pgfpathlineto{\pgfqpoint{2.874853in}{1.407552in}}%
\pgfpathlineto{\pgfqpoint{2.971648in}{1.379238in}}%
\pgfpathlineto{\pgfqpoint{3.068442in}{1.342076in}}%
\pgfpathlineto{\pgfqpoint{3.165237in}{1.329332in}}%
\pgfpathlineto{\pgfqpoint{3.262031in}{1.302256in}}%
\pgfpathlineto{\pgfqpoint{3.358826in}{1.274926in}}%
\pgfpathlineto{\pgfqpoint{3.455621in}{1.246442in}}%
\pgfpathlineto{\pgfqpoint{3.552415in}{1.213186in}}%
\pgfpathlineto{\pgfqpoint{3.649210in}{1.184966in}}%
\pgfpathlineto{\pgfqpoint{3.746005in}{1.170665in}}%
\pgfpathlineto{\pgfqpoint{3.842799in}{1.141095in}}%
\pgfpathlineto{\pgfqpoint{3.939594in}{1.117714in}}%
\pgfpathlineto{\pgfqpoint{4.036389in}{1.090007in}}%
\pgfusepath{stroke}%
\end{pgfscope}%
\begin{pgfscope}%
\pgfpathrectangle{\pgfqpoint{0.745371in}{0.566590in}}{\pgfqpoint{3.291018in}{1.828724in}}%
\pgfusepath{clip}%
\pgfsetbuttcap%
\pgfsetroundjoin%
\definecolor{currentfill}{rgb}{0.850000,0.324000,0.098000}%
\pgfsetfillcolor{currentfill}%
\pgfsetlinewidth{1.003750pt}%
\definecolor{currentstroke}{rgb}{0.850000,0.324000,0.098000}%
\pgfsetstrokecolor{currentstroke}%
\pgfsetdash{}{0pt}%
\pgfsys@defobject{currentmarker}{\pgfqpoint{-0.041667in}{-0.041667in}}{\pgfqpoint{0.041667in}{0.041667in}}{%
\pgfpathmoveto{\pgfqpoint{-0.041667in}{0.000000in}}%
\pgfpathlineto{\pgfqpoint{0.041667in}{0.000000in}}%
\pgfpathmoveto{\pgfqpoint{0.000000in}{-0.041667in}}%
\pgfpathlineto{\pgfqpoint{0.000000in}{0.041667in}}%
\pgfusepath{stroke,fill}%
}%
\begin{pgfscope}%
\pgfsys@transformshift{0.745371in}{2.158977in}%
\pgfsys@useobject{currentmarker}{}%
\end{pgfscope}%
\begin{pgfscope}%
\pgfsys@transformshift{1.035755in}{2.021071in}%
\pgfsys@useobject{currentmarker}{}%
\end{pgfscope}%
\begin{pgfscope}%
\pgfsys@transformshift{1.326139in}{2.055210in}%
\pgfsys@useobject{currentmarker}{}%
\end{pgfscope}%
\begin{pgfscope}%
\pgfsys@transformshift{1.616523in}{2.120118in}%
\pgfsys@useobject{currentmarker}{}%
\end{pgfscope}%
\begin{pgfscope}%
\pgfsys@transformshift{1.906906in}{2.079035in}%
\pgfsys@useobject{currentmarker}{}%
\end{pgfscope}%
\begin{pgfscope}%
\pgfsys@transformshift{2.197290in}{1.773700in}%
\pgfsys@useobject{currentmarker}{}%
\end{pgfscope}%
\begin{pgfscope}%
\pgfsys@transformshift{2.487674in}{1.510842in}%
\pgfsys@useobject{currentmarker}{}%
\end{pgfscope}%
\begin{pgfscope}%
\pgfsys@transformshift{2.778058in}{1.430383in}%
\pgfsys@useobject{currentmarker}{}%
\end{pgfscope}%
\begin{pgfscope}%
\pgfsys@transformshift{3.068442in}{1.342076in}%
\pgfsys@useobject{currentmarker}{}%
\end{pgfscope}%
\begin{pgfscope}%
\pgfsys@transformshift{3.358826in}{1.274926in}%
\pgfsys@useobject{currentmarker}{}%
\end{pgfscope}%
\begin{pgfscope}%
\pgfsys@transformshift{3.649210in}{1.184966in}%
\pgfsys@useobject{currentmarker}{}%
\end{pgfscope}%
\begin{pgfscope}%
\pgfsys@transformshift{3.939594in}{1.117714in}%
\pgfsys@useobject{currentmarker}{}%
\end{pgfscope}%
\end{pgfscope}%
\begin{pgfscope}%
\pgfpathrectangle{\pgfqpoint{0.745371in}{0.566590in}}{\pgfqpoint{3.291018in}{1.828724in}}%
\pgfusepath{clip}%
\pgfsetbuttcap%
\pgfsetroundjoin%
\pgfsetlinewidth{1.505625pt}%
\definecolor{currentstroke}{rgb}{0.000000,0.500000,0.000000}%
\pgfsetstrokecolor{currentstroke}%
\pgfsetdash{{5.550000pt}{2.400000pt}}{0.000000pt}%
\pgfpathmoveto{\pgfqpoint{0.745371in}{2.185040in}}%
\pgfpathlineto{\pgfqpoint{0.842165in}{2.130214in}}%
\pgfpathlineto{\pgfqpoint{0.938960in}{2.131357in}}%
\pgfpathlineto{\pgfqpoint{1.035755in}{2.099486in}}%
\pgfpathlineto{\pgfqpoint{1.132549in}{1.989192in}}%
\pgfpathlineto{\pgfqpoint{1.229344in}{2.047857in}}%
\pgfpathlineto{\pgfqpoint{1.326139in}{2.173237in}}%
\pgfpathlineto{\pgfqpoint{1.422933in}{2.123522in}}%
\pgfpathlineto{\pgfqpoint{1.519728in}{2.036179in}}%
\pgfpathlineto{\pgfqpoint{1.616523in}{1.976973in}}%
\pgfpathlineto{\pgfqpoint{1.713317in}{1.799501in}}%
\pgfpathlineto{\pgfqpoint{1.810112in}{2.027404in}}%
\pgfpathlineto{\pgfqpoint{1.906906in}{1.922041in}}%
\pgfpathlineto{\pgfqpoint{2.003701in}{1.677871in}}%
\pgfpathlineto{\pgfqpoint{2.100496in}{1.881230in}}%
\pgfpathlineto{\pgfqpoint{2.197290in}{1.924045in}}%
\pgfpathlineto{\pgfqpoint{2.294085in}{1.789728in}}%
\pgfpathlineto{\pgfqpoint{2.390880in}{1.554052in}}%
\pgfpathlineto{\pgfqpoint{2.487674in}{1.525155in}}%
\pgfpathlineto{\pgfqpoint{2.584469in}{1.509506in}}%
\pgfpathlineto{\pgfqpoint{2.681264in}{1.471901in}}%
\pgfpathlineto{\pgfqpoint{2.778058in}{1.450249in}}%
\pgfpathlineto{\pgfqpoint{2.874853in}{1.416980in}}%
\pgfpathlineto{\pgfqpoint{2.971648in}{1.394359in}}%
\pgfpathlineto{\pgfqpoint{3.068442in}{1.376116in}}%
\pgfpathlineto{\pgfqpoint{3.165237in}{1.346797in}}%
\pgfpathlineto{\pgfqpoint{3.262031in}{1.323649in}}%
\pgfpathlineto{\pgfqpoint{3.358826in}{1.299400in}}%
\pgfpathlineto{\pgfqpoint{3.455621in}{1.263270in}}%
\pgfpathlineto{\pgfqpoint{3.552415in}{1.234619in}}%
\pgfpathlineto{\pgfqpoint{3.649210in}{1.221533in}}%
\pgfpathlineto{\pgfqpoint{3.746005in}{1.183826in}}%
\pgfpathlineto{\pgfqpoint{3.842799in}{1.163353in}}%
\pgfpathlineto{\pgfqpoint{3.939594in}{1.151294in}}%
\pgfpathlineto{\pgfqpoint{4.036389in}{1.115325in}}%
\pgfusepath{stroke}%
\end{pgfscope}%
\begin{pgfscope}%
\pgfpathrectangle{\pgfqpoint{0.745371in}{0.566590in}}{\pgfqpoint{3.291018in}{1.828724in}}%
\pgfusepath{clip}%
\pgfsetbuttcap%
\pgfsetmiterjoin%
\definecolor{currentfill}{rgb}{0.000000,0.000000,0.000000}%
\pgfsetfillcolor{currentfill}%
\pgfsetfillopacity{0.000000}%
\pgfsetlinewidth{1.003750pt}%
\definecolor{currentstroke}{rgb}{0.000000,0.500000,0.000000}%
\pgfsetstrokecolor{currentstroke}%
\pgfsetdash{}{0pt}%
\pgfsys@defobject{currentmarker}{\pgfqpoint{-0.041667in}{-0.041667in}}{\pgfqpoint{0.041667in}{0.041667in}}{%
\pgfpathmoveto{\pgfqpoint{-0.041667in}{-0.041667in}}%
\pgfpathlineto{\pgfqpoint{0.041667in}{-0.041667in}}%
\pgfpathlineto{\pgfqpoint{0.041667in}{0.041667in}}%
\pgfpathlineto{\pgfqpoint{-0.041667in}{0.041667in}}%
\pgfpathclose%
\pgfusepath{stroke,fill}%
}%
\begin{pgfscope}%
\pgfsys@transformshift{0.745371in}{2.185040in}%
\pgfsys@useobject{currentmarker}{}%
\end{pgfscope}%
\begin{pgfscope}%
\pgfsys@transformshift{1.229344in}{2.047857in}%
\pgfsys@useobject{currentmarker}{}%
\end{pgfscope}%
\begin{pgfscope}%
\pgfsys@transformshift{1.713317in}{1.799501in}%
\pgfsys@useobject{currentmarker}{}%
\end{pgfscope}%
\begin{pgfscope}%
\pgfsys@transformshift{2.197290in}{1.924045in}%
\pgfsys@useobject{currentmarker}{}%
\end{pgfscope}%
\begin{pgfscope}%
\pgfsys@transformshift{2.681264in}{1.471901in}%
\pgfsys@useobject{currentmarker}{}%
\end{pgfscope}%
\begin{pgfscope}%
\pgfsys@transformshift{3.165237in}{1.346797in}%
\pgfsys@useobject{currentmarker}{}%
\end{pgfscope}%
\begin{pgfscope}%
\pgfsys@transformshift{3.649210in}{1.221533in}%
\pgfsys@useobject{currentmarker}{}%
\end{pgfscope}%
\end{pgfscope}%
\begin{pgfscope}%
\pgfpathrectangle{\pgfqpoint{0.745371in}{0.566590in}}{\pgfqpoint{3.291018in}{1.828724in}}%
\pgfusepath{clip}%
\pgfsetbuttcap%
\pgfsetroundjoin%
\pgfsetlinewidth{1.505625pt}%
\definecolor{currentstroke}{rgb}{0.494000,0.184000,0.556000}%
\pgfsetstrokecolor{currentstroke}%
\pgfsetdash{{5.550000pt}{2.400000pt}}{0.000000pt}%
\pgfpathmoveto{\pgfqpoint{0.745371in}{2.066155in}}%
\pgfpathlineto{\pgfqpoint{0.842165in}{2.158524in}}%
\pgfpathlineto{\pgfqpoint{0.938960in}{2.141054in}}%
\pgfpathlineto{\pgfqpoint{1.035755in}{2.002089in}}%
\pgfpathlineto{\pgfqpoint{1.132549in}{1.897527in}}%
\pgfpathlineto{\pgfqpoint{1.229344in}{2.204240in}}%
\pgfpathlineto{\pgfqpoint{1.326139in}{1.930709in}}%
\pgfpathlineto{\pgfqpoint{1.422933in}{2.104688in}}%
\pgfpathlineto{\pgfqpoint{1.519728in}{1.874024in}}%
\pgfpathlineto{\pgfqpoint{1.616523in}{2.011325in}}%
\pgfpathlineto{\pgfqpoint{1.713317in}{2.032114in}}%
\pgfpathlineto{\pgfqpoint{1.810112in}{2.108256in}}%
\pgfpathlineto{\pgfqpoint{1.906906in}{1.705256in}}%
\pgfpathlineto{\pgfqpoint{2.003701in}{1.572317in}}%
\pgfpathlineto{\pgfqpoint{2.100496in}{1.844817in}}%
\pgfpathlineto{\pgfqpoint{2.197290in}{1.772590in}}%
\pgfpathlineto{\pgfqpoint{2.294085in}{1.821606in}}%
\pgfpathlineto{\pgfqpoint{2.390880in}{1.566980in}}%
\pgfpathlineto{\pgfqpoint{2.487674in}{1.539608in}}%
\pgfpathlineto{\pgfqpoint{2.584469in}{1.523731in}}%
\pgfpathlineto{\pgfqpoint{2.681264in}{1.495752in}}%
\pgfpathlineto{\pgfqpoint{2.778058in}{1.464471in}}%
\pgfpathlineto{\pgfqpoint{2.874853in}{1.435078in}}%
\pgfpathlineto{\pgfqpoint{2.971648in}{1.413746in}}%
\pgfpathlineto{\pgfqpoint{3.068442in}{1.388091in}}%
\pgfpathlineto{\pgfqpoint{3.165237in}{1.357977in}}%
\pgfpathlineto{\pgfqpoint{3.262031in}{1.342442in}}%
\pgfpathlineto{\pgfqpoint{3.358826in}{1.308597in}}%
\pgfpathlineto{\pgfqpoint{3.455621in}{1.278121in}}%
\pgfpathlineto{\pgfqpoint{3.552415in}{1.264045in}}%
\pgfpathlineto{\pgfqpoint{3.649210in}{1.242961in}}%
\pgfpathlineto{\pgfqpoint{3.746005in}{1.214563in}}%
\pgfpathlineto{\pgfqpoint{3.842799in}{1.180710in}}%
\pgfpathlineto{\pgfqpoint{3.939594in}{1.162490in}}%
\pgfpathlineto{\pgfqpoint{4.036389in}{1.124083in}}%
\pgfusepath{stroke}%
\end{pgfscope}%
\begin{pgfscope}%
\pgfpathrectangle{\pgfqpoint{0.745371in}{0.566590in}}{\pgfqpoint{3.291018in}{1.828724in}}%
\pgfusepath{clip}%
\pgfsetbuttcap%
\pgfsetroundjoin%
\definecolor{currentfill}{rgb}{0.494000,0.184000,0.556000}%
\pgfsetfillcolor{currentfill}%
\pgfsetlinewidth{1.003750pt}%
\definecolor{currentstroke}{rgb}{0.494000,0.184000,0.556000}%
\pgfsetstrokecolor{currentstroke}%
\pgfsetdash{}{0pt}%
\pgfsys@defobject{currentmarker}{\pgfqpoint{-0.041667in}{-0.041667in}}{\pgfqpoint{0.041667in}{0.041667in}}{%
\pgfpathmoveto{\pgfqpoint{-0.041667in}{-0.041667in}}%
\pgfpathlineto{\pgfqpoint{0.041667in}{0.041667in}}%
\pgfpathmoveto{\pgfqpoint{-0.041667in}{0.041667in}}%
\pgfpathlineto{\pgfqpoint{0.041667in}{-0.041667in}}%
\pgfusepath{stroke,fill}%
}%
\begin{pgfscope}%
\pgfsys@transformshift{0.745371in}{2.066155in}%
\pgfsys@useobject{currentmarker}{}%
\end{pgfscope}%
\begin{pgfscope}%
\pgfsys@transformshift{1.132549in}{1.897527in}%
\pgfsys@useobject{currentmarker}{}%
\end{pgfscope}%
\begin{pgfscope}%
\pgfsys@transformshift{1.519728in}{1.874024in}%
\pgfsys@useobject{currentmarker}{}%
\end{pgfscope}%
\begin{pgfscope}%
\pgfsys@transformshift{1.906906in}{1.705256in}%
\pgfsys@useobject{currentmarker}{}%
\end{pgfscope}%
\begin{pgfscope}%
\pgfsys@transformshift{2.294085in}{1.821606in}%
\pgfsys@useobject{currentmarker}{}%
\end{pgfscope}%
\begin{pgfscope}%
\pgfsys@transformshift{2.681264in}{1.495752in}%
\pgfsys@useobject{currentmarker}{}%
\end{pgfscope}%
\begin{pgfscope}%
\pgfsys@transformshift{3.068442in}{1.388091in}%
\pgfsys@useobject{currentmarker}{}%
\end{pgfscope}%
\begin{pgfscope}%
\pgfsys@transformshift{3.455621in}{1.278121in}%
\pgfsys@useobject{currentmarker}{}%
\end{pgfscope}%
\begin{pgfscope}%
\pgfsys@transformshift{3.842799in}{1.180710in}%
\pgfsys@useobject{currentmarker}{}%
\end{pgfscope}%
\end{pgfscope}%
\begin{pgfscope}%
\pgfpathrectangle{\pgfqpoint{0.745371in}{0.566590in}}{\pgfqpoint{3.291018in}{1.828724in}}%
\pgfusepath{clip}%
\pgfsetrectcap%
\pgfsetroundjoin%
\pgfsetlinewidth{1.505625pt}%
\definecolor{currentstroke}{rgb}{0.000000,0.447000,0.741000}%
\pgfsetstrokecolor{currentstroke}%
\pgfsetdash{}{0pt}%
\pgfpathmoveto{\pgfqpoint{0.745371in}{1.683717in}}%
\pgfpathlineto{\pgfqpoint{0.980444in}{1.639416in}}%
\pgfpathlineto{\pgfqpoint{1.215516in}{1.676422in}}%
\pgfpathlineto{\pgfqpoint{1.450589in}{1.590100in}}%
\pgfpathlineto{\pgfqpoint{1.685662in}{1.574291in}}%
\pgfpathlineto{\pgfqpoint{1.920734in}{1.517607in}}%
\pgfpathlineto{\pgfqpoint{2.155807in}{1.433591in}}%
\pgfpathlineto{\pgfqpoint{2.390880in}{1.318945in}}%
\pgfpathlineto{\pgfqpoint{2.625952in}{1.237181in}}%
\pgfpathlineto{\pgfqpoint{2.861025in}{1.169937in}}%
\pgfpathlineto{\pgfqpoint{3.096098in}{1.106336in}}%
\pgfpathlineto{\pgfqpoint{3.331170in}{1.038201in}}%
\pgfpathlineto{\pgfqpoint{3.566243in}{0.970451in}}%
\pgfpathlineto{\pgfqpoint{3.801316in}{0.909507in}}%
\pgfpathlineto{\pgfqpoint{4.036389in}{0.845858in}}%
\pgfusepath{stroke}%
\end{pgfscope}%
\begin{pgfscope}%
\pgfpathrectangle{\pgfqpoint{0.745371in}{0.566590in}}{\pgfqpoint{3.291018in}{1.828724in}}%
\pgfusepath{clip}%
\pgfsetbuttcap%
\pgfsetroundjoin%
\definecolor{currentfill}{rgb}{0.000000,0.000000,0.000000}%
\pgfsetfillcolor{currentfill}%
\pgfsetfillopacity{0.000000}%
\pgfsetlinewidth{1.003750pt}%
\definecolor{currentstroke}{rgb}{0.000000,0.447000,0.741000}%
\pgfsetstrokecolor{currentstroke}%
\pgfsetdash{}{0pt}%
\pgfsys@defobject{currentmarker}{\pgfqpoint{-0.041667in}{-0.041667in}}{\pgfqpoint{0.041667in}{0.041667in}}{%
\pgfpathmoveto{\pgfqpoint{0.000000in}{-0.041667in}}%
\pgfpathcurveto{\pgfqpoint{0.011050in}{-0.041667in}}{\pgfqpoint{0.021649in}{-0.037276in}}{\pgfqpoint{0.029463in}{-0.029463in}}%
\pgfpathcurveto{\pgfqpoint{0.037276in}{-0.021649in}}{\pgfqpoint{0.041667in}{-0.011050in}}{\pgfqpoint{0.041667in}{0.000000in}}%
\pgfpathcurveto{\pgfqpoint{0.041667in}{0.011050in}}{\pgfqpoint{0.037276in}{0.021649in}}{\pgfqpoint{0.029463in}{0.029463in}}%
\pgfpathcurveto{\pgfqpoint{0.021649in}{0.037276in}}{\pgfqpoint{0.011050in}{0.041667in}}{\pgfqpoint{0.000000in}{0.041667in}}%
\pgfpathcurveto{\pgfqpoint{-0.011050in}{0.041667in}}{\pgfqpoint{-0.021649in}{0.037276in}}{\pgfqpoint{-0.029463in}{0.029463in}}%
\pgfpathcurveto{\pgfqpoint{-0.037276in}{0.021649in}}{\pgfqpoint{-0.041667in}{0.011050in}}{\pgfqpoint{-0.041667in}{0.000000in}}%
\pgfpathcurveto{\pgfqpoint{-0.041667in}{-0.011050in}}{\pgfqpoint{-0.037276in}{-0.021649in}}{\pgfqpoint{-0.029463in}{-0.029463in}}%
\pgfpathcurveto{\pgfqpoint{-0.021649in}{-0.037276in}}{\pgfqpoint{-0.011050in}{-0.041667in}}{\pgfqpoint{0.000000in}{-0.041667in}}%
\pgfpathclose%
\pgfusepath{stroke,fill}%
}%
\begin{pgfscope}%
\pgfsys@transformshift{0.745371in}{1.683717in}%
\pgfsys@useobject{currentmarker}{}%
\end{pgfscope}%
\begin{pgfscope}%
\pgfsys@transformshift{0.980444in}{1.639416in}%
\pgfsys@useobject{currentmarker}{}%
\end{pgfscope}%
\begin{pgfscope}%
\pgfsys@transformshift{1.215516in}{1.676422in}%
\pgfsys@useobject{currentmarker}{}%
\end{pgfscope}%
\begin{pgfscope}%
\pgfsys@transformshift{1.450589in}{1.590100in}%
\pgfsys@useobject{currentmarker}{}%
\end{pgfscope}%
\begin{pgfscope}%
\pgfsys@transformshift{1.685662in}{1.574291in}%
\pgfsys@useobject{currentmarker}{}%
\end{pgfscope}%
\begin{pgfscope}%
\pgfsys@transformshift{1.920734in}{1.517607in}%
\pgfsys@useobject{currentmarker}{}%
\end{pgfscope}%
\begin{pgfscope}%
\pgfsys@transformshift{2.155807in}{1.433591in}%
\pgfsys@useobject{currentmarker}{}%
\end{pgfscope}%
\begin{pgfscope}%
\pgfsys@transformshift{2.390880in}{1.318945in}%
\pgfsys@useobject{currentmarker}{}%
\end{pgfscope}%
\begin{pgfscope}%
\pgfsys@transformshift{2.625952in}{1.237181in}%
\pgfsys@useobject{currentmarker}{}%
\end{pgfscope}%
\begin{pgfscope}%
\pgfsys@transformshift{2.861025in}{1.169937in}%
\pgfsys@useobject{currentmarker}{}%
\end{pgfscope}%
\begin{pgfscope}%
\pgfsys@transformshift{3.096098in}{1.106336in}%
\pgfsys@useobject{currentmarker}{}%
\end{pgfscope}%
\begin{pgfscope}%
\pgfsys@transformshift{3.331170in}{1.038201in}%
\pgfsys@useobject{currentmarker}{}%
\end{pgfscope}%
\begin{pgfscope}%
\pgfsys@transformshift{3.566243in}{0.970451in}%
\pgfsys@useobject{currentmarker}{}%
\end{pgfscope}%
\begin{pgfscope}%
\pgfsys@transformshift{3.801316in}{0.909507in}%
\pgfsys@useobject{currentmarker}{}%
\end{pgfscope}%
\begin{pgfscope}%
\pgfsys@transformshift{4.036389in}{0.845858in}%
\pgfsys@useobject{currentmarker}{}%
\end{pgfscope}%
\end{pgfscope}%
\begin{pgfscope}%
\pgfpathrectangle{\pgfqpoint{0.745371in}{0.566590in}}{\pgfqpoint{3.291018in}{1.828724in}}%
\pgfusepath{clip}%
\pgfsetrectcap%
\pgfsetroundjoin%
\pgfsetlinewidth{1.505625pt}%
\definecolor{currentstroke}{rgb}{0.850000,0.324000,0.098000}%
\pgfsetstrokecolor{currentstroke}%
\pgfsetdash{}{0pt}%
\pgfpathmoveto{\pgfqpoint{0.745371in}{1.583051in}}%
\pgfpathlineto{\pgfqpoint{0.980444in}{1.636614in}}%
\pgfpathlineto{\pgfqpoint{1.215516in}{1.565052in}}%
\pgfpathlineto{\pgfqpoint{1.450589in}{1.551644in}}%
\pgfpathlineto{\pgfqpoint{1.685662in}{1.530324in}}%
\pgfpathlineto{\pgfqpoint{1.920734in}{1.509225in}}%
\pgfpathlineto{\pgfqpoint{2.155807in}{1.445092in}}%
\pgfpathlineto{\pgfqpoint{2.390880in}{1.371590in}}%
\pgfpathlineto{\pgfqpoint{2.625952in}{1.309113in}}%
\pgfpathlineto{\pgfqpoint{2.861025in}{1.240479in}}%
\pgfpathlineto{\pgfqpoint{3.096098in}{1.180437in}}%
\pgfpathlineto{\pgfqpoint{3.331170in}{1.114322in}}%
\pgfpathlineto{\pgfqpoint{3.566243in}{1.048648in}}%
\pgfpathlineto{\pgfqpoint{3.801316in}{0.986050in}}%
\pgfpathlineto{\pgfqpoint{4.036389in}{0.920851in}}%
\pgfusepath{stroke}%
\end{pgfscope}%
\begin{pgfscope}%
\pgfpathrectangle{\pgfqpoint{0.745371in}{0.566590in}}{\pgfqpoint{3.291018in}{1.828724in}}%
\pgfusepath{clip}%
\pgfsetbuttcap%
\pgfsetroundjoin%
\definecolor{currentfill}{rgb}{0.850000,0.324000,0.098000}%
\pgfsetfillcolor{currentfill}%
\pgfsetlinewidth{1.003750pt}%
\definecolor{currentstroke}{rgb}{0.850000,0.324000,0.098000}%
\pgfsetstrokecolor{currentstroke}%
\pgfsetdash{}{0pt}%
\pgfsys@defobject{currentmarker}{\pgfqpoint{-0.041667in}{-0.041667in}}{\pgfqpoint{0.041667in}{0.041667in}}{%
\pgfpathmoveto{\pgfqpoint{-0.041667in}{0.000000in}}%
\pgfpathlineto{\pgfqpoint{0.041667in}{0.000000in}}%
\pgfpathmoveto{\pgfqpoint{0.000000in}{-0.041667in}}%
\pgfpathlineto{\pgfqpoint{0.000000in}{0.041667in}}%
\pgfusepath{stroke,fill}%
}%
\begin{pgfscope}%
\pgfsys@transformshift{0.745371in}{1.583051in}%
\pgfsys@useobject{currentmarker}{}%
\end{pgfscope}%
\begin{pgfscope}%
\pgfsys@transformshift{0.980444in}{1.636614in}%
\pgfsys@useobject{currentmarker}{}%
\end{pgfscope}%
\begin{pgfscope}%
\pgfsys@transformshift{1.215516in}{1.565052in}%
\pgfsys@useobject{currentmarker}{}%
\end{pgfscope}%
\begin{pgfscope}%
\pgfsys@transformshift{1.450589in}{1.551644in}%
\pgfsys@useobject{currentmarker}{}%
\end{pgfscope}%
\begin{pgfscope}%
\pgfsys@transformshift{1.685662in}{1.530324in}%
\pgfsys@useobject{currentmarker}{}%
\end{pgfscope}%
\begin{pgfscope}%
\pgfsys@transformshift{1.920734in}{1.509225in}%
\pgfsys@useobject{currentmarker}{}%
\end{pgfscope}%
\begin{pgfscope}%
\pgfsys@transformshift{2.155807in}{1.445092in}%
\pgfsys@useobject{currentmarker}{}%
\end{pgfscope}%
\begin{pgfscope}%
\pgfsys@transformshift{2.390880in}{1.371590in}%
\pgfsys@useobject{currentmarker}{}%
\end{pgfscope}%
\begin{pgfscope}%
\pgfsys@transformshift{2.625952in}{1.309113in}%
\pgfsys@useobject{currentmarker}{}%
\end{pgfscope}%
\begin{pgfscope}%
\pgfsys@transformshift{2.861025in}{1.240479in}%
\pgfsys@useobject{currentmarker}{}%
\end{pgfscope}%
\begin{pgfscope}%
\pgfsys@transformshift{3.096098in}{1.180437in}%
\pgfsys@useobject{currentmarker}{}%
\end{pgfscope}%
\begin{pgfscope}%
\pgfsys@transformshift{3.331170in}{1.114322in}%
\pgfsys@useobject{currentmarker}{}%
\end{pgfscope}%
\begin{pgfscope}%
\pgfsys@transformshift{3.566243in}{1.048648in}%
\pgfsys@useobject{currentmarker}{}%
\end{pgfscope}%
\begin{pgfscope}%
\pgfsys@transformshift{3.801316in}{0.986050in}%
\pgfsys@useobject{currentmarker}{}%
\end{pgfscope}%
\begin{pgfscope}%
\pgfsys@transformshift{4.036389in}{0.920851in}%
\pgfsys@useobject{currentmarker}{}%
\end{pgfscope}%
\end{pgfscope}%
\begin{pgfscope}%
\pgfpathrectangle{\pgfqpoint{0.745371in}{0.566590in}}{\pgfqpoint{3.291018in}{1.828724in}}%
\pgfusepath{clip}%
\pgfsetrectcap%
\pgfsetroundjoin%
\pgfsetlinewidth{1.505625pt}%
\definecolor{currentstroke}{rgb}{0.000000,0.500000,0.000000}%
\pgfsetstrokecolor{currentstroke}%
\pgfsetdash{}{0pt}%
\pgfpathmoveto{\pgfqpoint{0.745371in}{1.620441in}}%
\pgfpathlineto{\pgfqpoint{0.980444in}{1.611015in}}%
\pgfpathlineto{\pgfqpoint{1.215516in}{1.594443in}}%
\pgfpathlineto{\pgfqpoint{1.450589in}{1.584333in}}%
\pgfpathlineto{\pgfqpoint{1.685662in}{1.532068in}}%
\pgfpathlineto{\pgfqpoint{1.920734in}{1.499427in}}%
\pgfpathlineto{\pgfqpoint{2.155807in}{1.447785in}}%
\pgfpathlineto{\pgfqpoint{2.390880in}{1.349414in}}%
\pgfpathlineto{\pgfqpoint{2.625952in}{1.261314in}}%
\pgfpathlineto{\pgfqpoint{2.861025in}{1.191881in}}%
\pgfpathlineto{\pgfqpoint{3.096098in}{1.130347in}}%
\pgfpathlineto{\pgfqpoint{3.331170in}{1.062941in}}%
\pgfpathlineto{\pgfqpoint{3.566243in}{1.001500in}}%
\pgfpathlineto{\pgfqpoint{3.801316in}{0.938335in}}%
\pgfpathlineto{\pgfqpoint{4.036389in}{0.869892in}}%
\pgfusepath{stroke}%
\end{pgfscope}%
\begin{pgfscope}%
\pgfpathrectangle{\pgfqpoint{0.745371in}{0.566590in}}{\pgfqpoint{3.291018in}{1.828724in}}%
\pgfusepath{clip}%
\pgfsetbuttcap%
\pgfsetmiterjoin%
\definecolor{currentfill}{rgb}{0.000000,0.000000,0.000000}%
\pgfsetfillcolor{currentfill}%
\pgfsetfillopacity{0.000000}%
\pgfsetlinewidth{1.003750pt}%
\definecolor{currentstroke}{rgb}{0.000000,0.500000,0.000000}%
\pgfsetstrokecolor{currentstroke}%
\pgfsetdash{}{0pt}%
\pgfsys@defobject{currentmarker}{\pgfqpoint{-0.041667in}{-0.041667in}}{\pgfqpoint{0.041667in}{0.041667in}}{%
\pgfpathmoveto{\pgfqpoint{-0.041667in}{-0.041667in}}%
\pgfpathlineto{\pgfqpoint{0.041667in}{-0.041667in}}%
\pgfpathlineto{\pgfqpoint{0.041667in}{0.041667in}}%
\pgfpathlineto{\pgfqpoint{-0.041667in}{0.041667in}}%
\pgfpathclose%
\pgfusepath{stroke,fill}%
}%
\begin{pgfscope}%
\pgfsys@transformshift{0.745371in}{1.620441in}%
\pgfsys@useobject{currentmarker}{}%
\end{pgfscope}%
\begin{pgfscope}%
\pgfsys@transformshift{0.980444in}{1.611015in}%
\pgfsys@useobject{currentmarker}{}%
\end{pgfscope}%
\begin{pgfscope}%
\pgfsys@transformshift{1.215516in}{1.594443in}%
\pgfsys@useobject{currentmarker}{}%
\end{pgfscope}%
\begin{pgfscope}%
\pgfsys@transformshift{1.450589in}{1.584333in}%
\pgfsys@useobject{currentmarker}{}%
\end{pgfscope}%
\begin{pgfscope}%
\pgfsys@transformshift{1.685662in}{1.532068in}%
\pgfsys@useobject{currentmarker}{}%
\end{pgfscope}%
\begin{pgfscope}%
\pgfsys@transformshift{1.920734in}{1.499427in}%
\pgfsys@useobject{currentmarker}{}%
\end{pgfscope}%
\begin{pgfscope}%
\pgfsys@transformshift{2.155807in}{1.447785in}%
\pgfsys@useobject{currentmarker}{}%
\end{pgfscope}%
\begin{pgfscope}%
\pgfsys@transformshift{2.390880in}{1.349414in}%
\pgfsys@useobject{currentmarker}{}%
\end{pgfscope}%
\begin{pgfscope}%
\pgfsys@transformshift{2.625952in}{1.261314in}%
\pgfsys@useobject{currentmarker}{}%
\end{pgfscope}%
\begin{pgfscope}%
\pgfsys@transformshift{2.861025in}{1.191881in}%
\pgfsys@useobject{currentmarker}{}%
\end{pgfscope}%
\begin{pgfscope}%
\pgfsys@transformshift{3.096098in}{1.130347in}%
\pgfsys@useobject{currentmarker}{}%
\end{pgfscope}%
\begin{pgfscope}%
\pgfsys@transformshift{3.331170in}{1.062941in}%
\pgfsys@useobject{currentmarker}{}%
\end{pgfscope}%
\begin{pgfscope}%
\pgfsys@transformshift{3.566243in}{1.001500in}%
\pgfsys@useobject{currentmarker}{}%
\end{pgfscope}%
\begin{pgfscope}%
\pgfsys@transformshift{3.801316in}{0.938335in}%
\pgfsys@useobject{currentmarker}{}%
\end{pgfscope}%
\begin{pgfscope}%
\pgfsys@transformshift{4.036389in}{0.869892in}%
\pgfsys@useobject{currentmarker}{}%
\end{pgfscope}%
\end{pgfscope}%
\begin{pgfscope}%
\pgfpathrectangle{\pgfqpoint{0.745371in}{0.566590in}}{\pgfqpoint{3.291018in}{1.828724in}}%
\pgfusepath{clip}%
\pgfsetrectcap%
\pgfsetroundjoin%
\pgfsetlinewidth{1.505625pt}%
\definecolor{currentstroke}{rgb}{0.494000,0.184000,0.556000}%
\pgfsetstrokecolor{currentstroke}%
\pgfsetdash{}{0pt}%
\pgfpathmoveto{\pgfqpoint{0.745371in}{1.641245in}}%
\pgfpathlineto{\pgfqpoint{0.980444in}{1.632200in}}%
\pgfpathlineto{\pgfqpoint{1.215516in}{1.616953in}}%
\pgfpathlineto{\pgfqpoint{1.450589in}{1.604366in}}%
\pgfpathlineto{\pgfqpoint{1.685662in}{1.554592in}}%
\pgfpathlineto{\pgfqpoint{1.920734in}{1.432637in}}%
\pgfpathlineto{\pgfqpoint{2.155807in}{1.396914in}}%
\pgfpathlineto{\pgfqpoint{2.390880in}{1.277538in}}%
\pgfpathlineto{\pgfqpoint{2.625952in}{1.190244in}}%
\pgfpathlineto{\pgfqpoint{2.861025in}{1.124754in}}%
\pgfpathlineto{\pgfqpoint{3.096098in}{1.055297in}}%
\pgfpathlineto{\pgfqpoint{3.331170in}{1.001967in}}%
\pgfpathlineto{\pgfqpoint{3.566243in}{0.932252in}}%
\pgfpathlineto{\pgfqpoint{3.801316in}{0.870656in}}%
\pgfpathlineto{\pgfqpoint{4.036389in}{0.807514in}}%
\pgfusepath{stroke}%
\end{pgfscope}%
\begin{pgfscope}%
\pgfpathrectangle{\pgfqpoint{0.745371in}{0.566590in}}{\pgfqpoint{3.291018in}{1.828724in}}%
\pgfusepath{clip}%
\pgfsetbuttcap%
\pgfsetroundjoin%
\definecolor{currentfill}{rgb}{0.494000,0.184000,0.556000}%
\pgfsetfillcolor{currentfill}%
\pgfsetlinewidth{1.003750pt}%
\definecolor{currentstroke}{rgb}{0.494000,0.184000,0.556000}%
\pgfsetstrokecolor{currentstroke}%
\pgfsetdash{}{0pt}%
\pgfsys@defobject{currentmarker}{\pgfqpoint{-0.041667in}{-0.041667in}}{\pgfqpoint{0.041667in}{0.041667in}}{%
\pgfpathmoveto{\pgfqpoint{-0.041667in}{-0.041667in}}%
\pgfpathlineto{\pgfqpoint{0.041667in}{0.041667in}}%
\pgfpathmoveto{\pgfqpoint{-0.041667in}{0.041667in}}%
\pgfpathlineto{\pgfqpoint{0.041667in}{-0.041667in}}%
\pgfusepath{stroke,fill}%
}%
\begin{pgfscope}%
\pgfsys@transformshift{0.745371in}{1.641245in}%
\pgfsys@useobject{currentmarker}{}%
\end{pgfscope}%
\begin{pgfscope}%
\pgfsys@transformshift{0.980444in}{1.632200in}%
\pgfsys@useobject{currentmarker}{}%
\end{pgfscope}%
\begin{pgfscope}%
\pgfsys@transformshift{1.215516in}{1.616953in}%
\pgfsys@useobject{currentmarker}{}%
\end{pgfscope}%
\begin{pgfscope}%
\pgfsys@transformshift{1.450589in}{1.604366in}%
\pgfsys@useobject{currentmarker}{}%
\end{pgfscope}%
\begin{pgfscope}%
\pgfsys@transformshift{1.685662in}{1.554592in}%
\pgfsys@useobject{currentmarker}{}%
\end{pgfscope}%
\begin{pgfscope}%
\pgfsys@transformshift{1.920734in}{1.432637in}%
\pgfsys@useobject{currentmarker}{}%
\end{pgfscope}%
\begin{pgfscope}%
\pgfsys@transformshift{2.155807in}{1.396914in}%
\pgfsys@useobject{currentmarker}{}%
\end{pgfscope}%
\begin{pgfscope}%
\pgfsys@transformshift{2.390880in}{1.277538in}%
\pgfsys@useobject{currentmarker}{}%
\end{pgfscope}%
\begin{pgfscope}%
\pgfsys@transformshift{2.625952in}{1.190244in}%
\pgfsys@useobject{currentmarker}{}%
\end{pgfscope}%
\begin{pgfscope}%
\pgfsys@transformshift{2.861025in}{1.124754in}%
\pgfsys@useobject{currentmarker}{}%
\end{pgfscope}%
\begin{pgfscope}%
\pgfsys@transformshift{3.096098in}{1.055297in}%
\pgfsys@useobject{currentmarker}{}%
\end{pgfscope}%
\begin{pgfscope}%
\pgfsys@transformshift{3.331170in}{1.001967in}%
\pgfsys@useobject{currentmarker}{}%
\end{pgfscope}%
\begin{pgfscope}%
\pgfsys@transformshift{3.566243in}{0.932252in}%
\pgfsys@useobject{currentmarker}{}%
\end{pgfscope}%
\begin{pgfscope}%
\pgfsys@transformshift{3.801316in}{0.870656in}%
\pgfsys@useobject{currentmarker}{}%
\end{pgfscope}%
\begin{pgfscope}%
\pgfsys@transformshift{4.036389in}{0.807514in}%
\pgfsys@useobject{currentmarker}{}%
\end{pgfscope}%
\end{pgfscope}%
\begin{pgfscope}%
\pgfsetrectcap%
\pgfsetmiterjoin%
\pgfsetlinewidth{0.803000pt}%
\definecolor{currentstroke}{rgb}{0.000000,0.000000,0.000000}%
\pgfsetstrokecolor{currentstroke}%
\pgfsetdash{}{0pt}%
\pgfpathmoveto{\pgfqpoint{0.745371in}{0.566590in}}%
\pgfpathlineto{\pgfqpoint{0.745371in}{2.395314in}}%
\pgfusepath{stroke}%
\end{pgfscope}%
\begin{pgfscope}%
\pgfsetrectcap%
\pgfsetmiterjoin%
\pgfsetlinewidth{0.803000pt}%
\definecolor{currentstroke}{rgb}{0.000000,0.000000,0.000000}%
\pgfsetstrokecolor{currentstroke}%
\pgfsetdash{}{0pt}%
\pgfpathmoveto{\pgfqpoint{4.036389in}{0.566590in}}%
\pgfpathlineto{\pgfqpoint{4.036389in}{2.395314in}}%
\pgfusepath{stroke}%
\end{pgfscope}%
\begin{pgfscope}%
\pgfsetrectcap%
\pgfsetmiterjoin%
\pgfsetlinewidth{0.803000pt}%
\definecolor{currentstroke}{rgb}{0.000000,0.000000,0.000000}%
\pgfsetstrokecolor{currentstroke}%
\pgfsetdash{}{0pt}%
\pgfpathmoveto{\pgfqpoint{0.745371in}{0.566590in}}%
\pgfpathlineto{\pgfqpoint{4.036389in}{0.566590in}}%
\pgfusepath{stroke}%
\end{pgfscope}%
\begin{pgfscope}%
\pgfsetrectcap%
\pgfsetmiterjoin%
\pgfsetlinewidth{0.803000pt}%
\definecolor{currentstroke}{rgb}{0.000000,0.000000,0.000000}%
\pgfsetstrokecolor{currentstroke}%
\pgfsetdash{}{0pt}%
\pgfpathmoveto{\pgfqpoint{0.745371in}{2.395314in}}%
\pgfpathlineto{\pgfqpoint{4.036389in}{2.395314in}}%
\pgfusepath{stroke}%
\end{pgfscope}%
\begin{pgfscope}%
\pgfsetbuttcap%
\pgfsetmiterjoin%
\definecolor{currentfill}{rgb}{1.000000,1.000000,1.000000}%
\pgfsetfillcolor{currentfill}%
\pgfsetfillopacity{0.800000}%
\pgfsetlinewidth{1.003750pt}%
\definecolor{currentstroke}{rgb}{0.800000,0.800000,0.800000}%
\pgfsetstrokecolor{currentstroke}%
\pgfsetstrokeopacity{0.800000}%
\pgfsetdash{}{0pt}%
\pgfpathmoveto{\pgfqpoint{2.924163in}{1.598092in}}%
\pgfpathlineto{\pgfqpoint{3.948889in}{1.598092in}}%
\pgfpathquadraticcurveto{\pgfqpoint{3.973889in}{1.598092in}}{\pgfqpoint{3.973889in}{1.623092in}}%
\pgfpathlineto{\pgfqpoint{3.973889in}{2.307814in}}%
\pgfpathquadraticcurveto{\pgfqpoint{3.973889in}{2.332814in}}{\pgfqpoint{3.948889in}{2.332814in}}%
\pgfpathlineto{\pgfqpoint{2.924163in}{2.332814in}}%
\pgfpathquadraticcurveto{\pgfqpoint{2.899163in}{2.332814in}}{\pgfqpoint{2.899163in}{2.307814in}}%
\pgfpathlineto{\pgfqpoint{2.899163in}{1.623092in}}%
\pgfpathquadraticcurveto{\pgfqpoint{2.899163in}{1.598092in}}{\pgfqpoint{2.924163in}{1.598092in}}%
\pgfpathclose%
\pgfusepath{stroke,fill}%
\end{pgfscope}%
\begin{pgfscope}%
\pgfsetbuttcap%
\pgfsetroundjoin%
\definecolor{currentfill}{rgb}{0.000000,0.000000,0.000000}%
\pgfsetfillcolor{currentfill}%
\pgfsetfillopacity{0.000000}%
\pgfsetlinewidth{1.003750pt}%
\definecolor{currentstroke}{rgb}{0.000000,0.447000,0.741000}%
\pgfsetstrokecolor{currentstroke}%
\pgfsetdash{}{0pt}%
\pgfsys@defobject{currentmarker}{\pgfqpoint{-0.041667in}{-0.041667in}}{\pgfqpoint{0.041667in}{0.041667in}}{%
\pgfpathmoveto{\pgfqpoint{0.000000in}{-0.041667in}}%
\pgfpathcurveto{\pgfqpoint{0.011050in}{-0.041667in}}{\pgfqpoint{0.021649in}{-0.037276in}}{\pgfqpoint{0.029463in}{-0.029463in}}%
\pgfpathcurveto{\pgfqpoint{0.037276in}{-0.021649in}}{\pgfqpoint{0.041667in}{-0.011050in}}{\pgfqpoint{0.041667in}{0.000000in}}%
\pgfpathcurveto{\pgfqpoint{0.041667in}{0.011050in}}{\pgfqpoint{0.037276in}{0.021649in}}{\pgfqpoint{0.029463in}{0.029463in}}%
\pgfpathcurveto{\pgfqpoint{0.021649in}{0.037276in}}{\pgfqpoint{0.011050in}{0.041667in}}{\pgfqpoint{0.000000in}{0.041667in}}%
\pgfpathcurveto{\pgfqpoint{-0.011050in}{0.041667in}}{\pgfqpoint{-0.021649in}{0.037276in}}{\pgfqpoint{-0.029463in}{0.029463in}}%
\pgfpathcurveto{\pgfqpoint{-0.037276in}{0.021649in}}{\pgfqpoint{-0.041667in}{0.011050in}}{\pgfqpoint{-0.041667in}{0.000000in}}%
\pgfpathcurveto{\pgfqpoint{-0.041667in}{-0.011050in}}{\pgfqpoint{-0.037276in}{-0.021649in}}{\pgfqpoint{-0.029463in}{-0.029463in}}%
\pgfpathcurveto{\pgfqpoint{-0.021649in}{-0.037276in}}{\pgfqpoint{-0.011050in}{-0.041667in}}{\pgfqpoint{0.000000in}{-0.041667in}}%
\pgfpathclose%
\pgfusepath{stroke,fill}%
}%
\begin{pgfscope}%
\pgfsys@transformshift{3.074163in}{2.239064in}%
\pgfsys@useobject{currentmarker}{}%
\end{pgfscope}%
\end{pgfscope}%
\begin{pgfscope}%
\definecolor{textcolor}{rgb}{0.000000,0.000000,0.000000}%
\pgfsetstrokecolor{textcolor}%
\pgfsetfillcolor{textcolor}%
\pgftext[x=3.299163in,y=2.195314in,left,base]{\color{textcolor}\rmfamily\fontsize{9.000000}{10.800000}\selectfont \(\displaystyle \gamma_{14} = \) -0.08}%
\end{pgfscope}%
\begin{pgfscope}%
\pgfsetbuttcap%
\pgfsetroundjoin%
\definecolor{currentfill}{rgb}{0.850000,0.324000,0.098000}%
\pgfsetfillcolor{currentfill}%
\pgfsetlinewidth{1.003750pt}%
\definecolor{currentstroke}{rgb}{0.850000,0.324000,0.098000}%
\pgfsetstrokecolor{currentstroke}%
\pgfsetdash{}{0pt}%
\pgfsys@defobject{currentmarker}{\pgfqpoint{-0.041667in}{-0.041667in}}{\pgfqpoint{0.041667in}{0.041667in}}{%
\pgfpathmoveto{\pgfqpoint{-0.041667in}{0.000000in}}%
\pgfpathlineto{\pgfqpoint{0.041667in}{0.000000in}}%
\pgfpathmoveto{\pgfqpoint{0.000000in}{-0.041667in}}%
\pgfpathlineto{\pgfqpoint{0.000000in}{0.041667in}}%
\pgfusepath{stroke,fill}%
}%
\begin{pgfscope}%
\pgfsys@transformshift{3.074163in}{2.064759in}%
\pgfsys@useobject{currentmarker}{}%
\end{pgfscope}%
\end{pgfscope}%
\begin{pgfscope}%
\definecolor{textcolor}{rgb}{0.000000,0.000000,0.000000}%
\pgfsetstrokecolor{textcolor}%
\pgfsetfillcolor{textcolor}%
\pgftext[x=3.299163in,y=2.021009in,left,base]{\color{textcolor}\rmfamily\fontsize{9.000000}{10.800000}\selectfont \(\displaystyle \gamma_{15} = \) -0.09}%
\end{pgfscope}%
\begin{pgfscope}%
\pgfsetbuttcap%
\pgfsetmiterjoin%
\definecolor{currentfill}{rgb}{0.000000,0.000000,0.000000}%
\pgfsetfillcolor{currentfill}%
\pgfsetfillopacity{0.000000}%
\pgfsetlinewidth{1.003750pt}%
\definecolor{currentstroke}{rgb}{0.000000,0.500000,0.000000}%
\pgfsetstrokecolor{currentstroke}%
\pgfsetdash{}{0pt}%
\pgfsys@defobject{currentmarker}{\pgfqpoint{-0.041667in}{-0.041667in}}{\pgfqpoint{0.041667in}{0.041667in}}{%
\pgfpathmoveto{\pgfqpoint{-0.041667in}{-0.041667in}}%
\pgfpathlineto{\pgfqpoint{0.041667in}{-0.041667in}}%
\pgfpathlineto{\pgfqpoint{0.041667in}{0.041667in}}%
\pgfpathlineto{\pgfqpoint{-0.041667in}{0.041667in}}%
\pgfpathclose%
\pgfusepath{stroke,fill}%
}%
\begin{pgfscope}%
\pgfsys@transformshift{3.074163in}{1.890453in}%
\pgfsys@useobject{currentmarker}{}%
\end{pgfscope}%
\end{pgfscope}%
\begin{pgfscope}%
\definecolor{textcolor}{rgb}{0.000000,0.000000,0.000000}%
\pgfsetstrokecolor{textcolor}%
\pgfsetfillcolor{textcolor}%
\pgftext[x=3.299163in,y=1.846703in,left,base]{\color{textcolor}\rmfamily\fontsize{9.000000}{10.800000}\selectfont \(\displaystyle \gamma_{16} = \) -0.10}%
\end{pgfscope}%
\begin{pgfscope}%
\pgfsetbuttcap%
\pgfsetroundjoin%
\definecolor{currentfill}{rgb}{0.494000,0.184000,0.556000}%
\pgfsetfillcolor{currentfill}%
\pgfsetlinewidth{1.003750pt}%
\definecolor{currentstroke}{rgb}{0.494000,0.184000,0.556000}%
\pgfsetstrokecolor{currentstroke}%
\pgfsetdash{}{0pt}%
\pgfsys@defobject{currentmarker}{\pgfqpoint{-0.041667in}{-0.041667in}}{\pgfqpoint{0.041667in}{0.041667in}}{%
\pgfpathmoveto{\pgfqpoint{-0.041667in}{-0.041667in}}%
\pgfpathlineto{\pgfqpoint{0.041667in}{0.041667in}}%
\pgfpathmoveto{\pgfqpoint{-0.041667in}{0.041667in}}%
\pgfpathlineto{\pgfqpoint{0.041667in}{-0.041667in}}%
\pgfusepath{stroke,fill}%
}%
\begin{pgfscope}%
\pgfsys@transformshift{3.074163in}{1.716147in}%
\pgfsys@useobject{currentmarker}{}%
\end{pgfscope}%
\end{pgfscope}%
\begin{pgfscope}%
\definecolor{textcolor}{rgb}{0.000000,0.000000,0.000000}%
\pgfsetstrokecolor{textcolor}%
\pgfsetfillcolor{textcolor}%
\pgftext[x=3.299163in,y=1.672397in,left,base]{\color{textcolor}\rmfamily\fontsize{9.000000}{10.800000}\selectfont \(\displaystyle \gamma_{17} = \) -0.08}%
\end{pgfscope}%
\end{pgfpicture}%
\makeatother%
\endgroup%
}
					\caption{Cluster IV $(a)$}
					\label{SubFig:Cluster_IV_real}
				\end{subfigure}
				\begin{subfigure}[h]{0.5\textwidth}
					\centering
					\resizebox{\linewidth}{!}{%% Creator: Matplotlib, PGF backend
%%
%% To include the figure in your LaTeX document, write
%%   \input{<filename>.pgf}
%%
%% Make sure the required packages are loaded in your preamble
%%   \usepackage{pgf}
%%
%% and, on pdftex
%%   \usepackage[utf8]{inputenc}\DeclareUnicodeCharacter{2212}{-}
%%
%% or, on luatex and xetex
%%   \usepackage{unicode-math}
%%
%% Figures using additional raster images can only be included by \input if
%% they are in the same directory as the main LaTeX file. For loading figures
%% from other directories you can use the `import` package
%%   \usepackage{import}
%%
%% and then include the figures with
%%   \import{<path to file>}{<filename>.pgf}
%%
%% Matplotlib used the following preamble
%%   \usepackage[utf8x]{inputenc}
%%   \usepackage[T1]{fontenc}
%%   \usepackage{amsmath,amssymb,amsfonts}
%%
\begingroup%
\makeatletter%
\begin{pgfpicture}%
\pgfpathrectangle{\pgfpointorigin}{\pgfqpoint{4.136389in}{2.495314in}}%
\pgfusepath{use as bounding box, clip}%
\begin{pgfscope}%
\pgfsetbuttcap%
\pgfsetmiterjoin%
\definecolor{currentfill}{rgb}{1.000000,1.000000,1.000000}%
\pgfsetfillcolor{currentfill}%
\pgfsetlinewidth{0.000000pt}%
\definecolor{currentstroke}{rgb}{1.000000,1.000000,1.000000}%
\pgfsetstrokecolor{currentstroke}%
\pgfsetdash{}{0pt}%
\pgfpathmoveto{\pgfqpoint{0.000000in}{0.000000in}}%
\pgfpathlineto{\pgfqpoint{4.136389in}{0.000000in}}%
\pgfpathlineto{\pgfqpoint{4.136389in}{2.495314in}}%
\pgfpathlineto{\pgfqpoint{0.000000in}{2.495314in}}%
\pgfpathclose%
\pgfusepath{fill}%
\end{pgfscope}%
\begin{pgfscope}%
\pgfsetbuttcap%
\pgfsetmiterjoin%
\definecolor{currentfill}{rgb}{1.000000,1.000000,1.000000}%
\pgfsetfillcolor{currentfill}%
\pgfsetlinewidth{0.000000pt}%
\definecolor{currentstroke}{rgb}{0.000000,0.000000,0.000000}%
\pgfsetstrokecolor{currentstroke}%
\pgfsetstrokeopacity{0.000000}%
\pgfsetdash{}{0pt}%
\pgfpathmoveto{\pgfqpoint{0.745371in}{0.566590in}}%
\pgfpathlineto{\pgfqpoint{4.036389in}{0.566590in}}%
\pgfpathlineto{\pgfqpoint{4.036389in}{2.395314in}}%
\pgfpathlineto{\pgfqpoint{0.745371in}{2.395314in}}%
\pgfpathclose%
\pgfusepath{fill}%
\end{pgfscope}%
\begin{pgfscope}%
\pgfpathrectangle{\pgfqpoint{0.745371in}{0.566590in}}{\pgfqpoint{3.291018in}{1.828724in}}%
\pgfusepath{clip}%
\pgfsetrectcap%
\pgfsetroundjoin%
\pgfsetlinewidth{0.803000pt}%
\definecolor{currentstroke}{rgb}{0.690196,0.690196,0.690196}%
\pgfsetstrokecolor{currentstroke}%
\pgfsetdash{}{0pt}%
\pgfpathmoveto{\pgfqpoint{0.745371in}{0.566590in}}%
\pgfpathlineto{\pgfqpoint{0.745371in}{2.395314in}}%
\pgfusepath{stroke}%
\end{pgfscope}%
\begin{pgfscope}%
\pgfsetbuttcap%
\pgfsetroundjoin%
\definecolor{currentfill}{rgb}{0.000000,0.000000,0.000000}%
\pgfsetfillcolor{currentfill}%
\pgfsetlinewidth{0.803000pt}%
\definecolor{currentstroke}{rgb}{0.000000,0.000000,0.000000}%
\pgfsetstrokecolor{currentstroke}%
\pgfsetdash{}{0pt}%
\pgfsys@defobject{currentmarker}{\pgfqpoint{0.000000in}{-0.048611in}}{\pgfqpoint{0.000000in}{0.000000in}}{%
\pgfpathmoveto{\pgfqpoint{0.000000in}{0.000000in}}%
\pgfpathlineto{\pgfqpoint{0.000000in}{-0.048611in}}%
\pgfusepath{stroke,fill}%
}%
\begin{pgfscope}%
\pgfsys@transformshift{0.745371in}{0.566590in}%
\pgfsys@useobject{currentmarker}{}%
\end{pgfscope}%
\end{pgfscope}%
\begin{pgfscope}%
\definecolor{textcolor}{rgb}{0.000000,0.000000,0.000000}%
\pgfsetstrokecolor{textcolor}%
\pgfsetfillcolor{textcolor}%
\pgftext[x=0.745371in,y=0.469368in,,top]{\color{textcolor}\rmfamily\fontsize{12.000000}{14.400000}\selectfont \(\displaystyle {-10}\)}%
\end{pgfscope}%
\begin{pgfscope}%
\pgfpathrectangle{\pgfqpoint{0.745371in}{0.566590in}}{\pgfqpoint{3.291018in}{1.828724in}}%
\pgfusepath{clip}%
\pgfsetrectcap%
\pgfsetroundjoin%
\pgfsetlinewidth{0.803000pt}%
\definecolor{currentstroke}{rgb}{0.690196,0.690196,0.690196}%
\pgfsetstrokecolor{currentstroke}%
\pgfsetdash{}{0pt}%
\pgfpathmoveto{\pgfqpoint{1.251681in}{0.566590in}}%
\pgfpathlineto{\pgfqpoint{1.251681in}{2.395314in}}%
\pgfusepath{stroke}%
\end{pgfscope}%
\begin{pgfscope}%
\pgfsetbuttcap%
\pgfsetroundjoin%
\definecolor{currentfill}{rgb}{0.000000,0.000000,0.000000}%
\pgfsetfillcolor{currentfill}%
\pgfsetlinewidth{0.803000pt}%
\definecolor{currentstroke}{rgb}{0.000000,0.000000,0.000000}%
\pgfsetstrokecolor{currentstroke}%
\pgfsetdash{}{0pt}%
\pgfsys@defobject{currentmarker}{\pgfqpoint{0.000000in}{-0.048611in}}{\pgfqpoint{0.000000in}{0.000000in}}{%
\pgfpathmoveto{\pgfqpoint{0.000000in}{0.000000in}}%
\pgfpathlineto{\pgfqpoint{0.000000in}{-0.048611in}}%
\pgfusepath{stroke,fill}%
}%
\begin{pgfscope}%
\pgfsys@transformshift{1.251681in}{0.566590in}%
\pgfsys@useobject{currentmarker}{}%
\end{pgfscope}%
\end{pgfscope}%
\begin{pgfscope}%
\definecolor{textcolor}{rgb}{0.000000,0.000000,0.000000}%
\pgfsetstrokecolor{textcolor}%
\pgfsetfillcolor{textcolor}%
\pgftext[x=1.251681in,y=0.469368in,,top]{\color{textcolor}\rmfamily\fontsize{12.000000}{14.400000}\selectfont \(\displaystyle {0}\)}%
\end{pgfscope}%
\begin{pgfscope}%
\pgfpathrectangle{\pgfqpoint{0.745371in}{0.566590in}}{\pgfqpoint{3.291018in}{1.828724in}}%
\pgfusepath{clip}%
\pgfsetrectcap%
\pgfsetroundjoin%
\pgfsetlinewidth{0.803000pt}%
\definecolor{currentstroke}{rgb}{0.690196,0.690196,0.690196}%
\pgfsetstrokecolor{currentstroke}%
\pgfsetdash{}{0pt}%
\pgfpathmoveto{\pgfqpoint{1.757992in}{0.566590in}}%
\pgfpathlineto{\pgfqpoint{1.757992in}{2.395314in}}%
\pgfusepath{stroke}%
\end{pgfscope}%
\begin{pgfscope}%
\pgfsetbuttcap%
\pgfsetroundjoin%
\definecolor{currentfill}{rgb}{0.000000,0.000000,0.000000}%
\pgfsetfillcolor{currentfill}%
\pgfsetlinewidth{0.803000pt}%
\definecolor{currentstroke}{rgb}{0.000000,0.000000,0.000000}%
\pgfsetstrokecolor{currentstroke}%
\pgfsetdash{}{0pt}%
\pgfsys@defobject{currentmarker}{\pgfqpoint{0.000000in}{-0.048611in}}{\pgfqpoint{0.000000in}{0.000000in}}{%
\pgfpathmoveto{\pgfqpoint{0.000000in}{0.000000in}}%
\pgfpathlineto{\pgfqpoint{0.000000in}{-0.048611in}}%
\pgfusepath{stroke,fill}%
}%
\begin{pgfscope}%
\pgfsys@transformshift{1.757992in}{0.566590in}%
\pgfsys@useobject{currentmarker}{}%
\end{pgfscope}%
\end{pgfscope}%
\begin{pgfscope}%
\definecolor{textcolor}{rgb}{0.000000,0.000000,0.000000}%
\pgfsetstrokecolor{textcolor}%
\pgfsetfillcolor{textcolor}%
\pgftext[x=1.757992in,y=0.469368in,,top]{\color{textcolor}\rmfamily\fontsize{12.000000}{14.400000}\selectfont \(\displaystyle {10}\)}%
\end{pgfscope}%
\begin{pgfscope}%
\pgfpathrectangle{\pgfqpoint{0.745371in}{0.566590in}}{\pgfqpoint{3.291018in}{1.828724in}}%
\pgfusepath{clip}%
\pgfsetrectcap%
\pgfsetroundjoin%
\pgfsetlinewidth{0.803000pt}%
\definecolor{currentstroke}{rgb}{0.690196,0.690196,0.690196}%
\pgfsetstrokecolor{currentstroke}%
\pgfsetdash{}{0pt}%
\pgfpathmoveto{\pgfqpoint{2.264302in}{0.566590in}}%
\pgfpathlineto{\pgfqpoint{2.264302in}{2.395314in}}%
\pgfusepath{stroke}%
\end{pgfscope}%
\begin{pgfscope}%
\pgfsetbuttcap%
\pgfsetroundjoin%
\definecolor{currentfill}{rgb}{0.000000,0.000000,0.000000}%
\pgfsetfillcolor{currentfill}%
\pgfsetlinewidth{0.803000pt}%
\definecolor{currentstroke}{rgb}{0.000000,0.000000,0.000000}%
\pgfsetstrokecolor{currentstroke}%
\pgfsetdash{}{0pt}%
\pgfsys@defobject{currentmarker}{\pgfqpoint{0.000000in}{-0.048611in}}{\pgfqpoint{0.000000in}{0.000000in}}{%
\pgfpathmoveto{\pgfqpoint{0.000000in}{0.000000in}}%
\pgfpathlineto{\pgfqpoint{0.000000in}{-0.048611in}}%
\pgfusepath{stroke,fill}%
}%
\begin{pgfscope}%
\pgfsys@transformshift{2.264302in}{0.566590in}%
\pgfsys@useobject{currentmarker}{}%
\end{pgfscope}%
\end{pgfscope}%
\begin{pgfscope}%
\definecolor{textcolor}{rgb}{0.000000,0.000000,0.000000}%
\pgfsetstrokecolor{textcolor}%
\pgfsetfillcolor{textcolor}%
\pgftext[x=2.264302in,y=0.469368in,,top]{\color{textcolor}\rmfamily\fontsize{12.000000}{14.400000}\selectfont \(\displaystyle {20}\)}%
\end{pgfscope}%
\begin{pgfscope}%
\pgfpathrectangle{\pgfqpoint{0.745371in}{0.566590in}}{\pgfqpoint{3.291018in}{1.828724in}}%
\pgfusepath{clip}%
\pgfsetrectcap%
\pgfsetroundjoin%
\pgfsetlinewidth{0.803000pt}%
\definecolor{currentstroke}{rgb}{0.690196,0.690196,0.690196}%
\pgfsetstrokecolor{currentstroke}%
\pgfsetdash{}{0pt}%
\pgfpathmoveto{\pgfqpoint{2.770613in}{0.566590in}}%
\pgfpathlineto{\pgfqpoint{2.770613in}{2.395314in}}%
\pgfusepath{stroke}%
\end{pgfscope}%
\begin{pgfscope}%
\pgfsetbuttcap%
\pgfsetroundjoin%
\definecolor{currentfill}{rgb}{0.000000,0.000000,0.000000}%
\pgfsetfillcolor{currentfill}%
\pgfsetlinewidth{0.803000pt}%
\definecolor{currentstroke}{rgb}{0.000000,0.000000,0.000000}%
\pgfsetstrokecolor{currentstroke}%
\pgfsetdash{}{0pt}%
\pgfsys@defobject{currentmarker}{\pgfqpoint{0.000000in}{-0.048611in}}{\pgfqpoint{0.000000in}{0.000000in}}{%
\pgfpathmoveto{\pgfqpoint{0.000000in}{0.000000in}}%
\pgfpathlineto{\pgfqpoint{0.000000in}{-0.048611in}}%
\pgfusepath{stroke,fill}%
}%
\begin{pgfscope}%
\pgfsys@transformshift{2.770613in}{0.566590in}%
\pgfsys@useobject{currentmarker}{}%
\end{pgfscope}%
\end{pgfscope}%
\begin{pgfscope}%
\definecolor{textcolor}{rgb}{0.000000,0.000000,0.000000}%
\pgfsetstrokecolor{textcolor}%
\pgfsetfillcolor{textcolor}%
\pgftext[x=2.770613in,y=0.469368in,,top]{\color{textcolor}\rmfamily\fontsize{12.000000}{14.400000}\selectfont \(\displaystyle {30}\)}%
\end{pgfscope}%
\begin{pgfscope}%
\pgfpathrectangle{\pgfqpoint{0.745371in}{0.566590in}}{\pgfqpoint{3.291018in}{1.828724in}}%
\pgfusepath{clip}%
\pgfsetrectcap%
\pgfsetroundjoin%
\pgfsetlinewidth{0.803000pt}%
\definecolor{currentstroke}{rgb}{0.690196,0.690196,0.690196}%
\pgfsetstrokecolor{currentstroke}%
\pgfsetdash{}{0pt}%
\pgfpathmoveto{\pgfqpoint{3.276923in}{0.566590in}}%
\pgfpathlineto{\pgfqpoint{3.276923in}{2.395314in}}%
\pgfusepath{stroke}%
\end{pgfscope}%
\begin{pgfscope}%
\pgfsetbuttcap%
\pgfsetroundjoin%
\definecolor{currentfill}{rgb}{0.000000,0.000000,0.000000}%
\pgfsetfillcolor{currentfill}%
\pgfsetlinewidth{0.803000pt}%
\definecolor{currentstroke}{rgb}{0.000000,0.000000,0.000000}%
\pgfsetstrokecolor{currentstroke}%
\pgfsetdash{}{0pt}%
\pgfsys@defobject{currentmarker}{\pgfqpoint{0.000000in}{-0.048611in}}{\pgfqpoint{0.000000in}{0.000000in}}{%
\pgfpathmoveto{\pgfqpoint{0.000000in}{0.000000in}}%
\pgfpathlineto{\pgfqpoint{0.000000in}{-0.048611in}}%
\pgfusepath{stroke,fill}%
}%
\begin{pgfscope}%
\pgfsys@transformshift{3.276923in}{0.566590in}%
\pgfsys@useobject{currentmarker}{}%
\end{pgfscope}%
\end{pgfscope}%
\begin{pgfscope}%
\definecolor{textcolor}{rgb}{0.000000,0.000000,0.000000}%
\pgfsetstrokecolor{textcolor}%
\pgfsetfillcolor{textcolor}%
\pgftext[x=3.276923in,y=0.469368in,,top]{\color{textcolor}\rmfamily\fontsize{12.000000}{14.400000}\selectfont \(\displaystyle {40}\)}%
\end{pgfscope}%
\begin{pgfscope}%
\pgfpathrectangle{\pgfqpoint{0.745371in}{0.566590in}}{\pgfqpoint{3.291018in}{1.828724in}}%
\pgfusepath{clip}%
\pgfsetrectcap%
\pgfsetroundjoin%
\pgfsetlinewidth{0.803000pt}%
\definecolor{currentstroke}{rgb}{0.690196,0.690196,0.690196}%
\pgfsetstrokecolor{currentstroke}%
\pgfsetdash{}{0pt}%
\pgfpathmoveto{\pgfqpoint{3.783233in}{0.566590in}}%
\pgfpathlineto{\pgfqpoint{3.783233in}{2.395314in}}%
\pgfusepath{stroke}%
\end{pgfscope}%
\begin{pgfscope}%
\pgfsetbuttcap%
\pgfsetroundjoin%
\definecolor{currentfill}{rgb}{0.000000,0.000000,0.000000}%
\pgfsetfillcolor{currentfill}%
\pgfsetlinewidth{0.803000pt}%
\definecolor{currentstroke}{rgb}{0.000000,0.000000,0.000000}%
\pgfsetstrokecolor{currentstroke}%
\pgfsetdash{}{0pt}%
\pgfsys@defobject{currentmarker}{\pgfqpoint{0.000000in}{-0.048611in}}{\pgfqpoint{0.000000in}{0.000000in}}{%
\pgfpathmoveto{\pgfqpoint{0.000000in}{0.000000in}}%
\pgfpathlineto{\pgfqpoint{0.000000in}{-0.048611in}}%
\pgfusepath{stroke,fill}%
}%
\begin{pgfscope}%
\pgfsys@transformshift{3.783233in}{0.566590in}%
\pgfsys@useobject{currentmarker}{}%
\end{pgfscope}%
\end{pgfscope}%
\begin{pgfscope}%
\definecolor{textcolor}{rgb}{0.000000,0.000000,0.000000}%
\pgfsetstrokecolor{textcolor}%
\pgfsetfillcolor{textcolor}%
\pgftext[x=3.783233in,y=0.469368in,,top]{\color{textcolor}\rmfamily\fontsize{12.000000}{14.400000}\selectfont \(\displaystyle {50}\)}%
\end{pgfscope}%
\begin{pgfscope}%
\definecolor{textcolor}{rgb}{0.000000,0.000000,0.000000}%
\pgfsetstrokecolor{textcolor}%
\pgfsetfillcolor{textcolor}%
\pgftext[x=2.390880in,y=0.266626in,,top]{\color{textcolor}\rmfamily\fontsize{12.000000}{14.400000}\selectfont SNR [dB]}%
\end{pgfscope}%
\begin{pgfscope}%
\pgfpathrectangle{\pgfqpoint{0.745371in}{0.566590in}}{\pgfqpoint{3.291018in}{1.828724in}}%
\pgfusepath{clip}%
\pgfsetrectcap%
\pgfsetroundjoin%
\pgfsetlinewidth{0.803000pt}%
\definecolor{currentstroke}{rgb}{0.690196,0.690196,0.690196}%
\pgfsetstrokecolor{currentstroke}%
\pgfsetdash{}{0pt}%
\pgfpathmoveto{\pgfqpoint{0.745371in}{0.566590in}}%
\pgfpathlineto{\pgfqpoint{4.036389in}{0.566590in}}%
\pgfusepath{stroke}%
\end{pgfscope}%
\begin{pgfscope}%
\pgfsetbuttcap%
\pgfsetroundjoin%
\definecolor{currentfill}{rgb}{0.000000,0.000000,0.000000}%
\pgfsetfillcolor{currentfill}%
\pgfsetlinewidth{0.803000pt}%
\definecolor{currentstroke}{rgb}{0.000000,0.000000,0.000000}%
\pgfsetstrokecolor{currentstroke}%
\pgfsetdash{}{0pt}%
\pgfsys@defobject{currentmarker}{\pgfqpoint{-0.048611in}{0.000000in}}{\pgfqpoint{-0.000000in}{0.000000in}}{%
\pgfpathmoveto{\pgfqpoint{-0.000000in}{0.000000in}}%
\pgfpathlineto{\pgfqpoint{-0.048611in}{0.000000in}}%
\pgfusepath{stroke,fill}%
}%
\begin{pgfscope}%
\pgfsys@transformshift{0.745371in}{0.566590in}%
\pgfsys@useobject{currentmarker}{}%
\end{pgfscope}%
\end{pgfscope}%
\begin{pgfscope}%
\definecolor{textcolor}{rgb}{0.000000,0.000000,0.000000}%
\pgfsetstrokecolor{textcolor}%
\pgfsetfillcolor{textcolor}%
\pgftext[x=0.327160in, y=0.509197in, left, base]{\color{textcolor}\rmfamily\fontsize{12.000000}{14.400000}\selectfont \(\displaystyle {10^{-4}}\)}%
\end{pgfscope}%
\begin{pgfscope}%
\pgfpathrectangle{\pgfqpoint{0.745371in}{0.566590in}}{\pgfqpoint{3.291018in}{1.828724in}}%
\pgfusepath{clip}%
\pgfsetrectcap%
\pgfsetroundjoin%
\pgfsetlinewidth{0.803000pt}%
\definecolor{currentstroke}{rgb}{0.690196,0.690196,0.690196}%
\pgfsetstrokecolor{currentstroke}%
\pgfsetdash{}{0pt}%
\pgfpathmoveto{\pgfqpoint{0.745371in}{1.104776in}}%
\pgfpathlineto{\pgfqpoint{4.036389in}{1.104776in}}%
\pgfusepath{stroke}%
\end{pgfscope}%
\begin{pgfscope}%
\pgfsetbuttcap%
\pgfsetroundjoin%
\definecolor{currentfill}{rgb}{0.000000,0.000000,0.000000}%
\pgfsetfillcolor{currentfill}%
\pgfsetlinewidth{0.803000pt}%
\definecolor{currentstroke}{rgb}{0.000000,0.000000,0.000000}%
\pgfsetstrokecolor{currentstroke}%
\pgfsetdash{}{0pt}%
\pgfsys@defobject{currentmarker}{\pgfqpoint{-0.048611in}{0.000000in}}{\pgfqpoint{-0.000000in}{0.000000in}}{%
\pgfpathmoveto{\pgfqpoint{-0.000000in}{0.000000in}}%
\pgfpathlineto{\pgfqpoint{-0.048611in}{0.000000in}}%
\pgfusepath{stroke,fill}%
}%
\begin{pgfscope}%
\pgfsys@transformshift{0.745371in}{1.104776in}%
\pgfsys@useobject{currentmarker}{}%
\end{pgfscope}%
\end{pgfscope}%
\begin{pgfscope}%
\definecolor{textcolor}{rgb}{0.000000,0.000000,0.000000}%
\pgfsetstrokecolor{textcolor}%
\pgfsetfillcolor{textcolor}%
\pgftext[x=0.327160in, y=1.047383in, left, base]{\color{textcolor}\rmfamily\fontsize{12.000000}{14.400000}\selectfont \(\displaystyle {10^{-2}}\)}%
\end{pgfscope}%
\begin{pgfscope}%
\pgfpathrectangle{\pgfqpoint{0.745371in}{0.566590in}}{\pgfqpoint{3.291018in}{1.828724in}}%
\pgfusepath{clip}%
\pgfsetrectcap%
\pgfsetroundjoin%
\pgfsetlinewidth{0.803000pt}%
\definecolor{currentstroke}{rgb}{0.690196,0.690196,0.690196}%
\pgfsetstrokecolor{currentstroke}%
\pgfsetdash{}{0pt}%
\pgfpathmoveto{\pgfqpoint{0.745371in}{1.642962in}}%
\pgfpathlineto{\pgfqpoint{4.036389in}{1.642962in}}%
\pgfusepath{stroke}%
\end{pgfscope}%
\begin{pgfscope}%
\pgfsetbuttcap%
\pgfsetroundjoin%
\definecolor{currentfill}{rgb}{0.000000,0.000000,0.000000}%
\pgfsetfillcolor{currentfill}%
\pgfsetlinewidth{0.803000pt}%
\definecolor{currentstroke}{rgb}{0.000000,0.000000,0.000000}%
\pgfsetstrokecolor{currentstroke}%
\pgfsetdash{}{0pt}%
\pgfsys@defobject{currentmarker}{\pgfqpoint{-0.048611in}{0.000000in}}{\pgfqpoint{-0.000000in}{0.000000in}}{%
\pgfpathmoveto{\pgfqpoint{-0.000000in}{0.000000in}}%
\pgfpathlineto{\pgfqpoint{-0.048611in}{0.000000in}}%
\pgfusepath{stroke,fill}%
}%
\begin{pgfscope}%
\pgfsys@transformshift{0.745371in}{1.642962in}%
\pgfsys@useobject{currentmarker}{}%
\end{pgfscope}%
\end{pgfscope}%
\begin{pgfscope}%
\definecolor{textcolor}{rgb}{0.000000,0.000000,0.000000}%
\pgfsetstrokecolor{textcolor}%
\pgfsetfillcolor{textcolor}%
\pgftext[x=0.418983in, y=1.585569in, left, base]{\color{textcolor}\rmfamily\fontsize{12.000000}{14.400000}\selectfont \(\displaystyle {10^{0}}\)}%
\end{pgfscope}%
\begin{pgfscope}%
\pgfpathrectangle{\pgfqpoint{0.745371in}{0.566590in}}{\pgfqpoint{3.291018in}{1.828724in}}%
\pgfusepath{clip}%
\pgfsetrectcap%
\pgfsetroundjoin%
\pgfsetlinewidth{0.803000pt}%
\definecolor{currentstroke}{rgb}{0.690196,0.690196,0.690196}%
\pgfsetstrokecolor{currentstroke}%
\pgfsetdash{}{0pt}%
\pgfpathmoveto{\pgfqpoint{0.745371in}{2.181148in}}%
\pgfpathlineto{\pgfqpoint{4.036389in}{2.181148in}}%
\pgfusepath{stroke}%
\end{pgfscope}%
\begin{pgfscope}%
\pgfsetbuttcap%
\pgfsetroundjoin%
\definecolor{currentfill}{rgb}{0.000000,0.000000,0.000000}%
\pgfsetfillcolor{currentfill}%
\pgfsetlinewidth{0.803000pt}%
\definecolor{currentstroke}{rgb}{0.000000,0.000000,0.000000}%
\pgfsetstrokecolor{currentstroke}%
\pgfsetdash{}{0pt}%
\pgfsys@defobject{currentmarker}{\pgfqpoint{-0.048611in}{0.000000in}}{\pgfqpoint{-0.000000in}{0.000000in}}{%
\pgfpathmoveto{\pgfqpoint{-0.000000in}{0.000000in}}%
\pgfpathlineto{\pgfqpoint{-0.048611in}{0.000000in}}%
\pgfusepath{stroke,fill}%
}%
\begin{pgfscope}%
\pgfsys@transformshift{0.745371in}{2.181148in}%
\pgfsys@useobject{currentmarker}{}%
\end{pgfscope}%
\end{pgfscope}%
\begin{pgfscope}%
\definecolor{textcolor}{rgb}{0.000000,0.000000,0.000000}%
\pgfsetstrokecolor{textcolor}%
\pgfsetfillcolor{textcolor}%
\pgftext[x=0.418983in, y=2.123755in, left, base]{\color{textcolor}\rmfamily\fontsize{12.000000}{14.400000}\selectfont \(\displaystyle {10^{2}}\)}%
\end{pgfscope}%
\begin{pgfscope}%
\definecolor{textcolor}{rgb}{0.000000,0.000000,0.000000}%
\pgfsetstrokecolor{textcolor}%
\pgfsetfillcolor{textcolor}%
\pgftext[x=0.271605in,y=1.480952in,,bottom,rotate=90.000000]{\color{textcolor}\rmfamily\fontsize{12.000000}{14.400000}\selectfont \(\displaystyle \hat{\sigma}_{\gamma}(\mathrm{SNR})\)}%
\end{pgfscope}%
\begin{pgfscope}%
\pgfpathrectangle{\pgfqpoint{0.745371in}{0.566590in}}{\pgfqpoint{3.291018in}{1.828724in}}%
\pgfusepath{clip}%
\pgfsetbuttcap%
\pgfsetroundjoin%
\pgfsetlinewidth{1.505625pt}%
\definecolor{currentstroke}{rgb}{0.000000,0.447000,0.741000}%
\pgfsetstrokecolor{currentstroke}%
\pgfsetdash{{5.550000pt}{2.400000pt}}{0.000000pt}%
\pgfpathmoveto{\pgfqpoint{0.745371in}{2.172746in}}%
\pgfpathlineto{\pgfqpoint{0.842165in}{2.043858in}}%
\pgfpathlineto{\pgfqpoint{0.938960in}{2.100058in}}%
\pgfpathlineto{\pgfqpoint{1.035755in}{2.187876in}}%
\pgfpathlineto{\pgfqpoint{1.132549in}{2.129671in}}%
\pgfpathlineto{\pgfqpoint{1.229344in}{2.134557in}}%
\pgfpathlineto{\pgfqpoint{1.326139in}{2.111126in}}%
\pgfpathlineto{\pgfqpoint{1.422933in}{2.105359in}}%
\pgfpathlineto{\pgfqpoint{1.519728in}{2.056614in}}%
\pgfpathlineto{\pgfqpoint{1.616523in}{2.015304in}}%
\pgfpathlineto{\pgfqpoint{1.713317in}{2.105616in}}%
\pgfpathlineto{\pgfqpoint{1.810112in}{2.002493in}}%
\pgfpathlineto{\pgfqpoint{1.906906in}{1.804594in}}%
\pgfpathlineto{\pgfqpoint{2.003701in}{1.938920in}}%
\pgfpathlineto{\pgfqpoint{2.100496in}{1.830008in}}%
\pgfpathlineto{\pgfqpoint{2.197290in}{1.371395in}}%
\pgfpathlineto{\pgfqpoint{2.294085in}{1.340510in}}%
\pgfpathlineto{\pgfqpoint{2.390880in}{1.313063in}}%
\pgfpathlineto{\pgfqpoint{2.487674in}{1.282823in}}%
\pgfpathlineto{\pgfqpoint{2.584469in}{1.263945in}}%
\pgfpathlineto{\pgfqpoint{2.681264in}{1.234257in}}%
\pgfpathlineto{\pgfqpoint{2.778058in}{1.197397in}}%
\pgfpathlineto{\pgfqpoint{2.874853in}{1.178201in}}%
\pgfpathlineto{\pgfqpoint{2.971648in}{1.153277in}}%
\pgfpathlineto{\pgfqpoint{3.068442in}{1.133692in}}%
\pgfpathlineto{\pgfqpoint{3.165237in}{1.105830in}}%
\pgfpathlineto{\pgfqpoint{3.262031in}{1.082394in}}%
\pgfpathlineto{\pgfqpoint{3.358826in}{1.055853in}}%
\pgfpathlineto{\pgfqpoint{3.455621in}{1.022321in}}%
\pgfpathlineto{\pgfqpoint{3.552415in}{0.983555in}}%
\pgfpathlineto{\pgfqpoint{3.649210in}{0.979384in}}%
\pgfpathlineto{\pgfqpoint{3.746005in}{0.949284in}}%
\pgfpathlineto{\pgfqpoint{3.842799in}{0.912530in}}%
\pgfpathlineto{\pgfqpoint{3.939594in}{0.892180in}}%
\pgfpathlineto{\pgfqpoint{4.036389in}{0.862110in}}%
\pgfusepath{stroke}%
\end{pgfscope}%
\begin{pgfscope}%
\pgfpathrectangle{\pgfqpoint{0.745371in}{0.566590in}}{\pgfqpoint{3.291018in}{1.828724in}}%
\pgfusepath{clip}%
\pgfsetbuttcap%
\pgfsetroundjoin%
\definecolor{currentfill}{rgb}{0.000000,0.000000,0.000000}%
\pgfsetfillcolor{currentfill}%
\pgfsetfillopacity{0.000000}%
\pgfsetlinewidth{1.003750pt}%
\definecolor{currentstroke}{rgb}{0.000000,0.447000,0.741000}%
\pgfsetstrokecolor{currentstroke}%
\pgfsetdash{}{0pt}%
\pgfsys@defobject{currentmarker}{\pgfqpoint{-0.041667in}{-0.041667in}}{\pgfqpoint{0.041667in}{0.041667in}}{%
\pgfpathmoveto{\pgfqpoint{0.000000in}{-0.041667in}}%
\pgfpathcurveto{\pgfqpoint{0.011050in}{-0.041667in}}{\pgfqpoint{0.021649in}{-0.037276in}}{\pgfqpoint{0.029463in}{-0.029463in}}%
\pgfpathcurveto{\pgfqpoint{0.037276in}{-0.021649in}}{\pgfqpoint{0.041667in}{-0.011050in}}{\pgfqpoint{0.041667in}{0.000000in}}%
\pgfpathcurveto{\pgfqpoint{0.041667in}{0.011050in}}{\pgfqpoint{0.037276in}{0.021649in}}{\pgfqpoint{0.029463in}{0.029463in}}%
\pgfpathcurveto{\pgfqpoint{0.021649in}{0.037276in}}{\pgfqpoint{0.011050in}{0.041667in}}{\pgfqpoint{0.000000in}{0.041667in}}%
\pgfpathcurveto{\pgfqpoint{-0.011050in}{0.041667in}}{\pgfqpoint{-0.021649in}{0.037276in}}{\pgfqpoint{-0.029463in}{0.029463in}}%
\pgfpathcurveto{\pgfqpoint{-0.037276in}{0.021649in}}{\pgfqpoint{-0.041667in}{0.011050in}}{\pgfqpoint{-0.041667in}{0.000000in}}%
\pgfpathcurveto{\pgfqpoint{-0.041667in}{-0.011050in}}{\pgfqpoint{-0.037276in}{-0.021649in}}{\pgfqpoint{-0.029463in}{-0.029463in}}%
\pgfpathcurveto{\pgfqpoint{-0.021649in}{-0.037276in}}{\pgfqpoint{-0.011050in}{-0.041667in}}{\pgfqpoint{0.000000in}{-0.041667in}}%
\pgfpathclose%
\pgfusepath{stroke,fill}%
}%
\begin{pgfscope}%
\pgfsys@transformshift{0.745371in}{2.172746in}%
\pgfsys@useobject{currentmarker}{}%
\end{pgfscope}%
\begin{pgfscope}%
\pgfsys@transformshift{1.132549in}{2.129671in}%
\pgfsys@useobject{currentmarker}{}%
\end{pgfscope}%
\begin{pgfscope}%
\pgfsys@transformshift{1.519728in}{2.056614in}%
\pgfsys@useobject{currentmarker}{}%
\end{pgfscope}%
\begin{pgfscope}%
\pgfsys@transformshift{1.906906in}{1.804594in}%
\pgfsys@useobject{currentmarker}{}%
\end{pgfscope}%
\begin{pgfscope}%
\pgfsys@transformshift{2.294085in}{1.340510in}%
\pgfsys@useobject{currentmarker}{}%
\end{pgfscope}%
\begin{pgfscope}%
\pgfsys@transformshift{2.681264in}{1.234257in}%
\pgfsys@useobject{currentmarker}{}%
\end{pgfscope}%
\begin{pgfscope}%
\pgfsys@transformshift{3.068442in}{1.133692in}%
\pgfsys@useobject{currentmarker}{}%
\end{pgfscope}%
\begin{pgfscope}%
\pgfsys@transformshift{3.455621in}{1.022321in}%
\pgfsys@useobject{currentmarker}{}%
\end{pgfscope}%
\begin{pgfscope}%
\pgfsys@transformshift{3.842799in}{0.912530in}%
\pgfsys@useobject{currentmarker}{}%
\end{pgfscope}%
\end{pgfscope}%
\begin{pgfscope}%
\pgfpathrectangle{\pgfqpoint{0.745371in}{0.566590in}}{\pgfqpoint{3.291018in}{1.828724in}}%
\pgfusepath{clip}%
\pgfsetbuttcap%
\pgfsetroundjoin%
\pgfsetlinewidth{1.505625pt}%
\definecolor{currentstroke}{rgb}{0.850000,0.324000,0.098000}%
\pgfsetstrokecolor{currentstroke}%
\pgfsetdash{{5.550000pt}{2.400000pt}}{0.000000pt}%
\pgfpathmoveto{\pgfqpoint{0.745371in}{2.187485in}}%
\pgfpathlineto{\pgfqpoint{0.842165in}{2.167123in}}%
\pgfpathlineto{\pgfqpoint{0.938960in}{2.186004in}}%
\pgfpathlineto{\pgfqpoint{1.035755in}{2.130349in}}%
\pgfpathlineto{\pgfqpoint{1.132549in}{2.005566in}}%
\pgfpathlineto{\pgfqpoint{1.229344in}{2.075770in}}%
\pgfpathlineto{\pgfqpoint{1.326139in}{2.133950in}}%
\pgfpathlineto{\pgfqpoint{1.422933in}{2.104479in}}%
\pgfpathlineto{\pgfqpoint{1.519728in}{2.206191in}}%
\pgfpathlineto{\pgfqpoint{1.616523in}{2.157737in}}%
\pgfpathlineto{\pgfqpoint{1.713317in}{2.161657in}}%
\pgfpathlineto{\pgfqpoint{1.810112in}{2.098043in}}%
\pgfpathlineto{\pgfqpoint{1.906906in}{1.980613in}}%
\pgfpathlineto{\pgfqpoint{2.003701in}{2.044185in}}%
\pgfpathlineto{\pgfqpoint{2.100496in}{1.814532in}}%
\pgfpathlineto{\pgfqpoint{2.197290in}{1.988487in}}%
\pgfpathlineto{\pgfqpoint{2.294085in}{1.950148in}}%
\pgfpathlineto{\pgfqpoint{2.390880in}{1.296343in}}%
\pgfpathlineto{\pgfqpoint{2.487674in}{1.254895in}}%
\pgfpathlineto{\pgfqpoint{2.584469in}{1.227671in}}%
\pgfpathlineto{\pgfqpoint{2.681264in}{1.207572in}}%
\pgfpathlineto{\pgfqpoint{2.778058in}{1.172869in}}%
\pgfpathlineto{\pgfqpoint{2.874853in}{1.159572in}}%
\pgfpathlineto{\pgfqpoint{2.971648in}{1.128137in}}%
\pgfpathlineto{\pgfqpoint{3.068442in}{1.101459in}}%
\pgfpathlineto{\pgfqpoint{3.165237in}{1.081919in}}%
\pgfpathlineto{\pgfqpoint{3.262031in}{1.056778in}}%
\pgfpathlineto{\pgfqpoint{3.358826in}{1.028236in}}%
\pgfpathlineto{\pgfqpoint{3.455621in}{0.997167in}}%
\pgfpathlineto{\pgfqpoint{3.552415in}{0.973775in}}%
\pgfpathlineto{\pgfqpoint{3.649210in}{0.947875in}}%
\pgfpathlineto{\pgfqpoint{3.746005in}{0.925837in}}%
\pgfpathlineto{\pgfqpoint{3.842799in}{0.891401in}}%
\pgfpathlineto{\pgfqpoint{3.939594in}{0.864745in}}%
\pgfpathlineto{\pgfqpoint{4.036389in}{0.840741in}}%
\pgfusepath{stroke}%
\end{pgfscope}%
\begin{pgfscope}%
\pgfpathrectangle{\pgfqpoint{0.745371in}{0.566590in}}{\pgfqpoint{3.291018in}{1.828724in}}%
\pgfusepath{clip}%
\pgfsetbuttcap%
\pgfsetroundjoin%
\definecolor{currentfill}{rgb}{0.850000,0.324000,0.098000}%
\pgfsetfillcolor{currentfill}%
\pgfsetlinewidth{1.003750pt}%
\definecolor{currentstroke}{rgb}{0.850000,0.324000,0.098000}%
\pgfsetstrokecolor{currentstroke}%
\pgfsetdash{}{0pt}%
\pgfsys@defobject{currentmarker}{\pgfqpoint{-0.041667in}{-0.041667in}}{\pgfqpoint{0.041667in}{0.041667in}}{%
\pgfpathmoveto{\pgfqpoint{-0.041667in}{0.000000in}}%
\pgfpathlineto{\pgfqpoint{0.041667in}{0.000000in}}%
\pgfpathmoveto{\pgfqpoint{0.000000in}{-0.041667in}}%
\pgfpathlineto{\pgfqpoint{0.000000in}{0.041667in}}%
\pgfusepath{stroke,fill}%
}%
\begin{pgfscope}%
\pgfsys@transformshift{0.745371in}{2.187485in}%
\pgfsys@useobject{currentmarker}{}%
\end{pgfscope}%
\begin{pgfscope}%
\pgfsys@transformshift{1.035755in}{2.130349in}%
\pgfsys@useobject{currentmarker}{}%
\end{pgfscope}%
\begin{pgfscope}%
\pgfsys@transformshift{1.326139in}{2.133950in}%
\pgfsys@useobject{currentmarker}{}%
\end{pgfscope}%
\begin{pgfscope}%
\pgfsys@transformshift{1.616523in}{2.157737in}%
\pgfsys@useobject{currentmarker}{}%
\end{pgfscope}%
\begin{pgfscope}%
\pgfsys@transformshift{1.906906in}{1.980613in}%
\pgfsys@useobject{currentmarker}{}%
\end{pgfscope}%
\begin{pgfscope}%
\pgfsys@transformshift{2.197290in}{1.988487in}%
\pgfsys@useobject{currentmarker}{}%
\end{pgfscope}%
\begin{pgfscope}%
\pgfsys@transformshift{2.487674in}{1.254895in}%
\pgfsys@useobject{currentmarker}{}%
\end{pgfscope}%
\begin{pgfscope}%
\pgfsys@transformshift{2.778058in}{1.172869in}%
\pgfsys@useobject{currentmarker}{}%
\end{pgfscope}%
\begin{pgfscope}%
\pgfsys@transformshift{3.068442in}{1.101459in}%
\pgfsys@useobject{currentmarker}{}%
\end{pgfscope}%
\begin{pgfscope}%
\pgfsys@transformshift{3.358826in}{1.028236in}%
\pgfsys@useobject{currentmarker}{}%
\end{pgfscope}%
\begin{pgfscope}%
\pgfsys@transformshift{3.649210in}{0.947875in}%
\pgfsys@useobject{currentmarker}{}%
\end{pgfscope}%
\begin{pgfscope}%
\pgfsys@transformshift{3.939594in}{0.864745in}%
\pgfsys@useobject{currentmarker}{}%
\end{pgfscope}%
\end{pgfscope}%
\begin{pgfscope}%
\pgfpathrectangle{\pgfqpoint{0.745371in}{0.566590in}}{\pgfqpoint{3.291018in}{1.828724in}}%
\pgfusepath{clip}%
\pgfsetbuttcap%
\pgfsetroundjoin%
\pgfsetlinewidth{1.505625pt}%
\definecolor{currentstroke}{rgb}{0.000000,0.500000,0.000000}%
\pgfsetstrokecolor{currentstroke}%
\pgfsetdash{{5.550000pt}{2.400000pt}}{0.000000pt}%
\pgfpathmoveto{\pgfqpoint{0.745371in}{2.228857in}}%
\pgfpathlineto{\pgfqpoint{0.842165in}{2.167570in}}%
\pgfpathlineto{\pgfqpoint{0.938960in}{2.220367in}}%
\pgfpathlineto{\pgfqpoint{1.035755in}{2.127624in}}%
\pgfpathlineto{\pgfqpoint{1.132549in}{2.207345in}}%
\pgfpathlineto{\pgfqpoint{1.229344in}{2.121237in}}%
\pgfpathlineto{\pgfqpoint{1.326139in}{2.142098in}}%
\pgfpathlineto{\pgfqpoint{1.422933in}{2.129430in}}%
\pgfpathlineto{\pgfqpoint{1.519728in}{2.281118in}}%
\pgfpathlineto{\pgfqpoint{1.616523in}{2.176793in}}%
\pgfpathlineto{\pgfqpoint{1.713317in}{2.141651in}}%
\pgfpathlineto{\pgfqpoint{1.810112in}{2.119920in}}%
\pgfpathlineto{\pgfqpoint{1.906906in}{2.065937in}}%
\pgfpathlineto{\pgfqpoint{2.003701in}{2.182489in}}%
\pgfpathlineto{\pgfqpoint{2.100496in}{2.075071in}}%
\pgfpathlineto{\pgfqpoint{2.197290in}{2.100040in}}%
\pgfpathlineto{\pgfqpoint{2.294085in}{2.164919in}}%
\pgfpathlineto{\pgfqpoint{2.390880in}{2.122234in}}%
\pgfpathlineto{\pgfqpoint{2.487674in}{2.123928in}}%
\pgfpathlineto{\pgfqpoint{2.584469in}{1.236580in}}%
\pgfpathlineto{\pgfqpoint{2.681264in}{1.212176in}}%
\pgfpathlineto{\pgfqpoint{2.778058in}{1.183171in}}%
\pgfpathlineto{\pgfqpoint{2.874853in}{1.162771in}}%
\pgfpathlineto{\pgfqpoint{2.971648in}{1.130029in}}%
\pgfpathlineto{\pgfqpoint{3.068442in}{1.113660in}}%
\pgfpathlineto{\pgfqpoint{3.165237in}{1.079365in}}%
\pgfpathlineto{\pgfqpoint{3.262031in}{1.057064in}}%
\pgfpathlineto{\pgfqpoint{3.358826in}{1.033923in}}%
\pgfpathlineto{\pgfqpoint{3.455621in}{0.996622in}}%
\pgfpathlineto{\pgfqpoint{3.552415in}{0.975367in}}%
\pgfpathlineto{\pgfqpoint{3.649210in}{0.958322in}}%
\pgfpathlineto{\pgfqpoint{3.746005in}{0.927574in}}%
\pgfpathlineto{\pgfqpoint{3.842799in}{0.891166in}}%
\pgfpathlineto{\pgfqpoint{3.939594in}{0.875491in}}%
\pgfpathlineto{\pgfqpoint{4.036389in}{0.844578in}}%
\pgfusepath{stroke}%
\end{pgfscope}%
\begin{pgfscope}%
\pgfpathrectangle{\pgfqpoint{0.745371in}{0.566590in}}{\pgfqpoint{3.291018in}{1.828724in}}%
\pgfusepath{clip}%
\pgfsetbuttcap%
\pgfsetmiterjoin%
\definecolor{currentfill}{rgb}{0.000000,0.000000,0.000000}%
\pgfsetfillcolor{currentfill}%
\pgfsetfillopacity{0.000000}%
\pgfsetlinewidth{1.003750pt}%
\definecolor{currentstroke}{rgb}{0.000000,0.500000,0.000000}%
\pgfsetstrokecolor{currentstroke}%
\pgfsetdash{}{0pt}%
\pgfsys@defobject{currentmarker}{\pgfqpoint{-0.041667in}{-0.041667in}}{\pgfqpoint{0.041667in}{0.041667in}}{%
\pgfpathmoveto{\pgfqpoint{-0.041667in}{-0.041667in}}%
\pgfpathlineto{\pgfqpoint{0.041667in}{-0.041667in}}%
\pgfpathlineto{\pgfqpoint{0.041667in}{0.041667in}}%
\pgfpathlineto{\pgfqpoint{-0.041667in}{0.041667in}}%
\pgfpathclose%
\pgfusepath{stroke,fill}%
}%
\begin{pgfscope}%
\pgfsys@transformshift{0.745371in}{2.228857in}%
\pgfsys@useobject{currentmarker}{}%
\end{pgfscope}%
\begin{pgfscope}%
\pgfsys@transformshift{1.229344in}{2.121237in}%
\pgfsys@useobject{currentmarker}{}%
\end{pgfscope}%
\begin{pgfscope}%
\pgfsys@transformshift{1.713317in}{2.141651in}%
\pgfsys@useobject{currentmarker}{}%
\end{pgfscope}%
\begin{pgfscope}%
\pgfsys@transformshift{2.197290in}{2.100040in}%
\pgfsys@useobject{currentmarker}{}%
\end{pgfscope}%
\begin{pgfscope}%
\pgfsys@transformshift{2.681264in}{1.212176in}%
\pgfsys@useobject{currentmarker}{}%
\end{pgfscope}%
\begin{pgfscope}%
\pgfsys@transformshift{3.165237in}{1.079365in}%
\pgfsys@useobject{currentmarker}{}%
\end{pgfscope}%
\begin{pgfscope}%
\pgfsys@transformshift{3.649210in}{0.958322in}%
\pgfsys@useobject{currentmarker}{}%
\end{pgfscope}%
\end{pgfscope}%
\begin{pgfscope}%
\pgfpathrectangle{\pgfqpoint{0.745371in}{0.566590in}}{\pgfqpoint{3.291018in}{1.828724in}}%
\pgfusepath{clip}%
\pgfsetrectcap%
\pgfsetroundjoin%
\pgfsetlinewidth{1.505625pt}%
\definecolor{currentstroke}{rgb}{0.000000,0.447000,0.741000}%
\pgfsetstrokecolor{currentstroke}%
\pgfsetdash{}{0pt}%
\pgfpathmoveto{\pgfqpoint{0.745371in}{1.577861in}}%
\pgfpathlineto{\pgfqpoint{0.980444in}{1.542293in}}%
\pgfpathlineto{\pgfqpoint{1.215516in}{1.586528in}}%
\pgfpathlineto{\pgfqpoint{1.450589in}{1.541884in}}%
\pgfpathlineto{\pgfqpoint{1.685662in}{1.518517in}}%
\pgfpathlineto{\pgfqpoint{1.920734in}{1.434871in}}%
\pgfpathlineto{\pgfqpoint{2.155807in}{1.357921in}}%
\pgfpathlineto{\pgfqpoint{2.390880in}{1.288002in}}%
\pgfpathlineto{\pgfqpoint{2.625952in}{1.121088in}}%
\pgfpathlineto{\pgfqpoint{2.861025in}{1.053775in}}%
\pgfpathlineto{\pgfqpoint{3.096098in}{0.998324in}}%
\pgfpathlineto{\pgfqpoint{3.331170in}{0.934550in}}%
\pgfpathlineto{\pgfqpoint{3.566243in}{0.873697in}}%
\pgfpathlineto{\pgfqpoint{3.801316in}{0.809517in}}%
\pgfpathlineto{\pgfqpoint{4.036389in}{0.746802in}}%
\pgfusepath{stroke}%
\end{pgfscope}%
\begin{pgfscope}%
\pgfpathrectangle{\pgfqpoint{0.745371in}{0.566590in}}{\pgfqpoint{3.291018in}{1.828724in}}%
\pgfusepath{clip}%
\pgfsetbuttcap%
\pgfsetroundjoin%
\definecolor{currentfill}{rgb}{0.000000,0.000000,0.000000}%
\pgfsetfillcolor{currentfill}%
\pgfsetfillopacity{0.000000}%
\pgfsetlinewidth{1.003750pt}%
\definecolor{currentstroke}{rgb}{0.000000,0.447000,0.741000}%
\pgfsetstrokecolor{currentstroke}%
\pgfsetdash{}{0pt}%
\pgfsys@defobject{currentmarker}{\pgfqpoint{-0.041667in}{-0.041667in}}{\pgfqpoint{0.041667in}{0.041667in}}{%
\pgfpathmoveto{\pgfqpoint{0.000000in}{-0.041667in}}%
\pgfpathcurveto{\pgfqpoint{0.011050in}{-0.041667in}}{\pgfqpoint{0.021649in}{-0.037276in}}{\pgfqpoint{0.029463in}{-0.029463in}}%
\pgfpathcurveto{\pgfqpoint{0.037276in}{-0.021649in}}{\pgfqpoint{0.041667in}{-0.011050in}}{\pgfqpoint{0.041667in}{0.000000in}}%
\pgfpathcurveto{\pgfqpoint{0.041667in}{0.011050in}}{\pgfqpoint{0.037276in}{0.021649in}}{\pgfqpoint{0.029463in}{0.029463in}}%
\pgfpathcurveto{\pgfqpoint{0.021649in}{0.037276in}}{\pgfqpoint{0.011050in}{0.041667in}}{\pgfqpoint{0.000000in}{0.041667in}}%
\pgfpathcurveto{\pgfqpoint{-0.011050in}{0.041667in}}{\pgfqpoint{-0.021649in}{0.037276in}}{\pgfqpoint{-0.029463in}{0.029463in}}%
\pgfpathcurveto{\pgfqpoint{-0.037276in}{0.021649in}}{\pgfqpoint{-0.041667in}{0.011050in}}{\pgfqpoint{-0.041667in}{0.000000in}}%
\pgfpathcurveto{\pgfqpoint{-0.041667in}{-0.011050in}}{\pgfqpoint{-0.037276in}{-0.021649in}}{\pgfqpoint{-0.029463in}{-0.029463in}}%
\pgfpathcurveto{\pgfqpoint{-0.021649in}{-0.037276in}}{\pgfqpoint{-0.011050in}{-0.041667in}}{\pgfqpoint{0.000000in}{-0.041667in}}%
\pgfpathclose%
\pgfusepath{stroke,fill}%
}%
\begin{pgfscope}%
\pgfsys@transformshift{0.745371in}{1.577861in}%
\pgfsys@useobject{currentmarker}{}%
\end{pgfscope}%
\begin{pgfscope}%
\pgfsys@transformshift{0.980444in}{1.542293in}%
\pgfsys@useobject{currentmarker}{}%
\end{pgfscope}%
\begin{pgfscope}%
\pgfsys@transformshift{1.215516in}{1.586528in}%
\pgfsys@useobject{currentmarker}{}%
\end{pgfscope}%
\begin{pgfscope}%
\pgfsys@transformshift{1.450589in}{1.541884in}%
\pgfsys@useobject{currentmarker}{}%
\end{pgfscope}%
\begin{pgfscope}%
\pgfsys@transformshift{1.685662in}{1.518517in}%
\pgfsys@useobject{currentmarker}{}%
\end{pgfscope}%
\begin{pgfscope}%
\pgfsys@transformshift{1.920734in}{1.434871in}%
\pgfsys@useobject{currentmarker}{}%
\end{pgfscope}%
\begin{pgfscope}%
\pgfsys@transformshift{2.155807in}{1.357921in}%
\pgfsys@useobject{currentmarker}{}%
\end{pgfscope}%
\begin{pgfscope}%
\pgfsys@transformshift{2.390880in}{1.288002in}%
\pgfsys@useobject{currentmarker}{}%
\end{pgfscope}%
\begin{pgfscope}%
\pgfsys@transformshift{2.625952in}{1.121088in}%
\pgfsys@useobject{currentmarker}{}%
\end{pgfscope}%
\begin{pgfscope}%
\pgfsys@transformshift{2.861025in}{1.053775in}%
\pgfsys@useobject{currentmarker}{}%
\end{pgfscope}%
\begin{pgfscope}%
\pgfsys@transformshift{3.096098in}{0.998324in}%
\pgfsys@useobject{currentmarker}{}%
\end{pgfscope}%
\begin{pgfscope}%
\pgfsys@transformshift{3.331170in}{0.934550in}%
\pgfsys@useobject{currentmarker}{}%
\end{pgfscope}%
\begin{pgfscope}%
\pgfsys@transformshift{3.566243in}{0.873697in}%
\pgfsys@useobject{currentmarker}{}%
\end{pgfscope}%
\begin{pgfscope}%
\pgfsys@transformshift{3.801316in}{0.809517in}%
\pgfsys@useobject{currentmarker}{}%
\end{pgfscope}%
\begin{pgfscope}%
\pgfsys@transformshift{4.036389in}{0.746802in}%
\pgfsys@useobject{currentmarker}{}%
\end{pgfscope}%
\end{pgfscope}%
\begin{pgfscope}%
\pgfpathrectangle{\pgfqpoint{0.745371in}{0.566590in}}{\pgfqpoint{3.291018in}{1.828724in}}%
\pgfusepath{clip}%
\pgfsetrectcap%
\pgfsetroundjoin%
\pgfsetlinewidth{1.505625pt}%
\definecolor{currentstroke}{rgb}{0.850000,0.324000,0.098000}%
\pgfsetstrokecolor{currentstroke}%
\pgfsetdash{}{0pt}%
\pgfpathmoveto{\pgfqpoint{0.745371in}{1.622483in}}%
\pgfpathlineto{\pgfqpoint{0.980444in}{1.632138in}}%
\pgfpathlineto{\pgfqpoint{1.215516in}{1.575247in}}%
\pgfpathlineto{\pgfqpoint{1.450589in}{1.565600in}}%
\pgfpathlineto{\pgfqpoint{1.685662in}{1.519073in}}%
\pgfpathlineto{\pgfqpoint{1.920734in}{1.466067in}}%
\pgfpathlineto{\pgfqpoint{2.155807in}{1.409376in}}%
\pgfpathlineto{\pgfqpoint{2.390880in}{1.325807in}}%
\pgfpathlineto{\pgfqpoint{2.625952in}{1.262762in}}%
\pgfpathlineto{\pgfqpoint{2.861025in}{1.200037in}}%
\pgfpathlineto{\pgfqpoint{3.096098in}{1.139939in}}%
\pgfpathlineto{\pgfqpoint{3.331170in}{1.064348in}}%
\pgfpathlineto{\pgfqpoint{3.566243in}{1.002869in}}%
\pgfpathlineto{\pgfqpoint{3.801316in}{0.944879in}}%
\pgfpathlineto{\pgfqpoint{4.036389in}{0.875595in}}%
\pgfusepath{stroke}%
\end{pgfscope}%
\begin{pgfscope}%
\pgfpathrectangle{\pgfqpoint{0.745371in}{0.566590in}}{\pgfqpoint{3.291018in}{1.828724in}}%
\pgfusepath{clip}%
\pgfsetbuttcap%
\pgfsetroundjoin%
\definecolor{currentfill}{rgb}{0.850000,0.324000,0.098000}%
\pgfsetfillcolor{currentfill}%
\pgfsetlinewidth{1.003750pt}%
\definecolor{currentstroke}{rgb}{0.850000,0.324000,0.098000}%
\pgfsetstrokecolor{currentstroke}%
\pgfsetdash{}{0pt}%
\pgfsys@defobject{currentmarker}{\pgfqpoint{-0.041667in}{-0.041667in}}{\pgfqpoint{0.041667in}{0.041667in}}{%
\pgfpathmoveto{\pgfqpoint{-0.041667in}{0.000000in}}%
\pgfpathlineto{\pgfqpoint{0.041667in}{0.000000in}}%
\pgfpathmoveto{\pgfqpoint{0.000000in}{-0.041667in}}%
\pgfpathlineto{\pgfqpoint{0.000000in}{0.041667in}}%
\pgfusepath{stroke,fill}%
}%
\begin{pgfscope}%
\pgfsys@transformshift{0.745371in}{1.622483in}%
\pgfsys@useobject{currentmarker}{}%
\end{pgfscope}%
\begin{pgfscope}%
\pgfsys@transformshift{0.980444in}{1.632138in}%
\pgfsys@useobject{currentmarker}{}%
\end{pgfscope}%
\begin{pgfscope}%
\pgfsys@transformshift{1.215516in}{1.575247in}%
\pgfsys@useobject{currentmarker}{}%
\end{pgfscope}%
\begin{pgfscope}%
\pgfsys@transformshift{1.450589in}{1.565600in}%
\pgfsys@useobject{currentmarker}{}%
\end{pgfscope}%
\begin{pgfscope}%
\pgfsys@transformshift{1.685662in}{1.519073in}%
\pgfsys@useobject{currentmarker}{}%
\end{pgfscope}%
\begin{pgfscope}%
\pgfsys@transformshift{1.920734in}{1.466067in}%
\pgfsys@useobject{currentmarker}{}%
\end{pgfscope}%
\begin{pgfscope}%
\pgfsys@transformshift{2.155807in}{1.409376in}%
\pgfsys@useobject{currentmarker}{}%
\end{pgfscope}%
\begin{pgfscope}%
\pgfsys@transformshift{2.390880in}{1.325807in}%
\pgfsys@useobject{currentmarker}{}%
\end{pgfscope}%
\begin{pgfscope}%
\pgfsys@transformshift{2.625952in}{1.262762in}%
\pgfsys@useobject{currentmarker}{}%
\end{pgfscope}%
\begin{pgfscope}%
\pgfsys@transformshift{2.861025in}{1.200037in}%
\pgfsys@useobject{currentmarker}{}%
\end{pgfscope}%
\begin{pgfscope}%
\pgfsys@transformshift{3.096098in}{1.139939in}%
\pgfsys@useobject{currentmarker}{}%
\end{pgfscope}%
\begin{pgfscope}%
\pgfsys@transformshift{3.331170in}{1.064348in}%
\pgfsys@useobject{currentmarker}{}%
\end{pgfscope}%
\begin{pgfscope}%
\pgfsys@transformshift{3.566243in}{1.002869in}%
\pgfsys@useobject{currentmarker}{}%
\end{pgfscope}%
\begin{pgfscope}%
\pgfsys@transformshift{3.801316in}{0.944879in}%
\pgfsys@useobject{currentmarker}{}%
\end{pgfscope}%
\begin{pgfscope}%
\pgfsys@transformshift{4.036389in}{0.875595in}%
\pgfsys@useobject{currentmarker}{}%
\end{pgfscope}%
\end{pgfscope}%
\begin{pgfscope}%
\pgfpathrectangle{\pgfqpoint{0.745371in}{0.566590in}}{\pgfqpoint{3.291018in}{1.828724in}}%
\pgfusepath{clip}%
\pgfsetrectcap%
\pgfsetroundjoin%
\pgfsetlinewidth{1.505625pt}%
\definecolor{currentstroke}{rgb}{0.000000,0.500000,0.000000}%
\pgfsetstrokecolor{currentstroke}%
\pgfsetdash{}{0pt}%
\pgfpathmoveto{\pgfqpoint{0.745371in}{1.618262in}}%
\pgfpathlineto{\pgfqpoint{0.980444in}{1.591530in}}%
\pgfpathlineto{\pgfqpoint{1.215516in}{1.587616in}}%
\pgfpathlineto{\pgfqpoint{1.450589in}{1.561303in}}%
\pgfpathlineto{\pgfqpoint{1.685662in}{1.487831in}}%
\pgfpathlineto{\pgfqpoint{1.920734in}{1.444072in}}%
\pgfpathlineto{\pgfqpoint{2.155807in}{1.361283in}}%
\pgfpathlineto{\pgfqpoint{2.390880in}{1.227270in}}%
\pgfpathlineto{\pgfqpoint{2.625952in}{1.132035in}}%
\pgfpathlineto{\pgfqpoint{2.861025in}{1.064979in}}%
\pgfpathlineto{\pgfqpoint{3.096098in}{0.999601in}}%
\pgfpathlineto{\pgfqpoint{3.331170in}{0.944169in}}%
\pgfpathlineto{\pgfqpoint{3.566243in}{0.871475in}}%
\pgfpathlineto{\pgfqpoint{3.801316in}{0.812169in}}%
\pgfpathlineto{\pgfqpoint{4.036389in}{0.749116in}}%
\pgfusepath{stroke}%
\end{pgfscope}%
\begin{pgfscope}%
\pgfpathrectangle{\pgfqpoint{0.745371in}{0.566590in}}{\pgfqpoint{3.291018in}{1.828724in}}%
\pgfusepath{clip}%
\pgfsetbuttcap%
\pgfsetmiterjoin%
\definecolor{currentfill}{rgb}{0.000000,0.000000,0.000000}%
\pgfsetfillcolor{currentfill}%
\pgfsetfillopacity{0.000000}%
\pgfsetlinewidth{1.003750pt}%
\definecolor{currentstroke}{rgb}{0.000000,0.500000,0.000000}%
\pgfsetstrokecolor{currentstroke}%
\pgfsetdash{}{0pt}%
\pgfsys@defobject{currentmarker}{\pgfqpoint{-0.041667in}{-0.041667in}}{\pgfqpoint{0.041667in}{0.041667in}}{%
\pgfpathmoveto{\pgfqpoint{-0.041667in}{-0.041667in}}%
\pgfpathlineto{\pgfqpoint{0.041667in}{-0.041667in}}%
\pgfpathlineto{\pgfqpoint{0.041667in}{0.041667in}}%
\pgfpathlineto{\pgfqpoint{-0.041667in}{0.041667in}}%
\pgfpathclose%
\pgfusepath{stroke,fill}%
}%
\begin{pgfscope}%
\pgfsys@transformshift{0.745371in}{1.618262in}%
\pgfsys@useobject{currentmarker}{}%
\end{pgfscope}%
\begin{pgfscope}%
\pgfsys@transformshift{0.980444in}{1.591530in}%
\pgfsys@useobject{currentmarker}{}%
\end{pgfscope}%
\begin{pgfscope}%
\pgfsys@transformshift{1.215516in}{1.587616in}%
\pgfsys@useobject{currentmarker}{}%
\end{pgfscope}%
\begin{pgfscope}%
\pgfsys@transformshift{1.450589in}{1.561303in}%
\pgfsys@useobject{currentmarker}{}%
\end{pgfscope}%
\begin{pgfscope}%
\pgfsys@transformshift{1.685662in}{1.487831in}%
\pgfsys@useobject{currentmarker}{}%
\end{pgfscope}%
\begin{pgfscope}%
\pgfsys@transformshift{1.920734in}{1.444072in}%
\pgfsys@useobject{currentmarker}{}%
\end{pgfscope}%
\begin{pgfscope}%
\pgfsys@transformshift{2.155807in}{1.361283in}%
\pgfsys@useobject{currentmarker}{}%
\end{pgfscope}%
\begin{pgfscope}%
\pgfsys@transformshift{2.390880in}{1.227270in}%
\pgfsys@useobject{currentmarker}{}%
\end{pgfscope}%
\begin{pgfscope}%
\pgfsys@transformshift{2.625952in}{1.132035in}%
\pgfsys@useobject{currentmarker}{}%
\end{pgfscope}%
\begin{pgfscope}%
\pgfsys@transformshift{2.861025in}{1.064979in}%
\pgfsys@useobject{currentmarker}{}%
\end{pgfscope}%
\begin{pgfscope}%
\pgfsys@transformshift{3.096098in}{0.999601in}%
\pgfsys@useobject{currentmarker}{}%
\end{pgfscope}%
\begin{pgfscope}%
\pgfsys@transformshift{3.331170in}{0.944169in}%
\pgfsys@useobject{currentmarker}{}%
\end{pgfscope}%
\begin{pgfscope}%
\pgfsys@transformshift{3.566243in}{0.871475in}%
\pgfsys@useobject{currentmarker}{}%
\end{pgfscope}%
\begin{pgfscope}%
\pgfsys@transformshift{3.801316in}{0.812169in}%
\pgfsys@useobject{currentmarker}{}%
\end{pgfscope}%
\begin{pgfscope}%
\pgfsys@transformshift{4.036389in}{0.749116in}%
\pgfsys@useobject{currentmarker}{}%
\end{pgfscope}%
\end{pgfscope}%
\begin{pgfscope}%
\pgfsetrectcap%
\pgfsetmiterjoin%
\pgfsetlinewidth{0.803000pt}%
\definecolor{currentstroke}{rgb}{0.000000,0.000000,0.000000}%
\pgfsetstrokecolor{currentstroke}%
\pgfsetdash{}{0pt}%
\pgfpathmoveto{\pgfqpoint{0.745371in}{0.566590in}}%
\pgfpathlineto{\pgfqpoint{0.745371in}{2.395314in}}%
\pgfusepath{stroke}%
\end{pgfscope}%
\begin{pgfscope}%
\pgfsetrectcap%
\pgfsetmiterjoin%
\pgfsetlinewidth{0.803000pt}%
\definecolor{currentstroke}{rgb}{0.000000,0.000000,0.000000}%
\pgfsetstrokecolor{currentstroke}%
\pgfsetdash{}{0pt}%
\pgfpathmoveto{\pgfqpoint{4.036389in}{0.566590in}}%
\pgfpathlineto{\pgfqpoint{4.036389in}{2.395314in}}%
\pgfusepath{stroke}%
\end{pgfscope}%
\begin{pgfscope}%
\pgfsetrectcap%
\pgfsetmiterjoin%
\pgfsetlinewidth{0.803000pt}%
\definecolor{currentstroke}{rgb}{0.000000,0.000000,0.000000}%
\pgfsetstrokecolor{currentstroke}%
\pgfsetdash{}{0pt}%
\pgfpathmoveto{\pgfqpoint{0.745371in}{0.566590in}}%
\pgfpathlineto{\pgfqpoint{4.036389in}{0.566590in}}%
\pgfusepath{stroke}%
\end{pgfscope}%
\begin{pgfscope}%
\pgfsetrectcap%
\pgfsetmiterjoin%
\pgfsetlinewidth{0.803000pt}%
\definecolor{currentstroke}{rgb}{0.000000,0.000000,0.000000}%
\pgfsetstrokecolor{currentstroke}%
\pgfsetdash{}{0pt}%
\pgfpathmoveto{\pgfqpoint{0.745371in}{2.395314in}}%
\pgfpathlineto{\pgfqpoint{4.036389in}{2.395314in}}%
\pgfusepath{stroke}%
\end{pgfscope}%
\begin{pgfscope}%
\pgfsetbuttcap%
\pgfsetmiterjoin%
\definecolor{currentfill}{rgb}{1.000000,1.000000,1.000000}%
\pgfsetfillcolor{currentfill}%
\pgfsetfillopacity{0.800000}%
\pgfsetlinewidth{1.003750pt}%
\definecolor{currentstroke}{rgb}{0.800000,0.800000,0.800000}%
\pgfsetstrokecolor{currentstroke}%
\pgfsetstrokeopacity{0.800000}%
\pgfsetdash{}{0pt}%
\pgfpathmoveto{\pgfqpoint{2.924163in}{1.772397in}}%
\pgfpathlineto{\pgfqpoint{3.948889in}{1.772397in}}%
\pgfpathquadraticcurveto{\pgfqpoint{3.973889in}{1.772397in}}{\pgfqpoint{3.973889in}{1.797397in}}%
\pgfpathlineto{\pgfqpoint{3.973889in}{2.307814in}}%
\pgfpathquadraticcurveto{\pgfqpoint{3.973889in}{2.332814in}}{\pgfqpoint{3.948889in}{2.332814in}}%
\pgfpathlineto{\pgfqpoint{2.924163in}{2.332814in}}%
\pgfpathquadraticcurveto{\pgfqpoint{2.899163in}{2.332814in}}{\pgfqpoint{2.899163in}{2.307814in}}%
\pgfpathlineto{\pgfqpoint{2.899163in}{1.797397in}}%
\pgfpathquadraticcurveto{\pgfqpoint{2.899163in}{1.772397in}}{\pgfqpoint{2.924163in}{1.772397in}}%
\pgfpathclose%
\pgfusepath{stroke,fill}%
\end{pgfscope}%
\begin{pgfscope}%
\pgfsetbuttcap%
\pgfsetroundjoin%
\definecolor{currentfill}{rgb}{0.000000,0.000000,0.000000}%
\pgfsetfillcolor{currentfill}%
\pgfsetfillopacity{0.000000}%
\pgfsetlinewidth{1.003750pt}%
\definecolor{currentstroke}{rgb}{0.000000,0.447000,0.741000}%
\pgfsetstrokecolor{currentstroke}%
\pgfsetdash{}{0pt}%
\pgfsys@defobject{currentmarker}{\pgfqpoint{-0.041667in}{-0.041667in}}{\pgfqpoint{0.041667in}{0.041667in}}{%
\pgfpathmoveto{\pgfqpoint{0.000000in}{-0.041667in}}%
\pgfpathcurveto{\pgfqpoint{0.011050in}{-0.041667in}}{\pgfqpoint{0.021649in}{-0.037276in}}{\pgfqpoint{0.029463in}{-0.029463in}}%
\pgfpathcurveto{\pgfqpoint{0.037276in}{-0.021649in}}{\pgfqpoint{0.041667in}{-0.011050in}}{\pgfqpoint{0.041667in}{0.000000in}}%
\pgfpathcurveto{\pgfqpoint{0.041667in}{0.011050in}}{\pgfqpoint{0.037276in}{0.021649in}}{\pgfqpoint{0.029463in}{0.029463in}}%
\pgfpathcurveto{\pgfqpoint{0.021649in}{0.037276in}}{\pgfqpoint{0.011050in}{0.041667in}}{\pgfqpoint{0.000000in}{0.041667in}}%
\pgfpathcurveto{\pgfqpoint{-0.011050in}{0.041667in}}{\pgfqpoint{-0.021649in}{0.037276in}}{\pgfqpoint{-0.029463in}{0.029463in}}%
\pgfpathcurveto{\pgfqpoint{-0.037276in}{0.021649in}}{\pgfqpoint{-0.041667in}{0.011050in}}{\pgfqpoint{-0.041667in}{0.000000in}}%
\pgfpathcurveto{\pgfqpoint{-0.041667in}{-0.011050in}}{\pgfqpoint{-0.037276in}{-0.021649in}}{\pgfqpoint{-0.029463in}{-0.029463in}}%
\pgfpathcurveto{\pgfqpoint{-0.021649in}{-0.037276in}}{\pgfqpoint{-0.011050in}{-0.041667in}}{\pgfqpoint{0.000000in}{-0.041667in}}%
\pgfpathclose%
\pgfusepath{stroke,fill}%
}%
\begin{pgfscope}%
\pgfsys@transformshift{3.074163in}{2.239064in}%
\pgfsys@useobject{currentmarker}{}%
\end{pgfscope}%
\end{pgfscope}%
\begin{pgfscope}%
\definecolor{textcolor}{rgb}{0.000000,0.000000,0.000000}%
\pgfsetstrokecolor{textcolor}%
\pgfsetfillcolor{textcolor}%
\pgftext[x=3.299163in,y=2.195314in,left,base]{\color{textcolor}\rmfamily\fontsize{9.000000}{10.800000}\selectfont \(\displaystyle \gamma_{18} = \) -0.21}%
\end{pgfscope}%
\begin{pgfscope}%
\pgfsetbuttcap%
\pgfsetroundjoin%
\definecolor{currentfill}{rgb}{0.850000,0.324000,0.098000}%
\pgfsetfillcolor{currentfill}%
\pgfsetlinewidth{1.003750pt}%
\definecolor{currentstroke}{rgb}{0.850000,0.324000,0.098000}%
\pgfsetstrokecolor{currentstroke}%
\pgfsetdash{}{0pt}%
\pgfsys@defobject{currentmarker}{\pgfqpoint{-0.041667in}{-0.041667in}}{\pgfqpoint{0.041667in}{0.041667in}}{%
\pgfpathmoveto{\pgfqpoint{-0.041667in}{0.000000in}}%
\pgfpathlineto{\pgfqpoint{0.041667in}{0.000000in}}%
\pgfpathmoveto{\pgfqpoint{0.000000in}{-0.041667in}}%
\pgfpathlineto{\pgfqpoint{0.000000in}{0.041667in}}%
\pgfusepath{stroke,fill}%
}%
\begin{pgfscope}%
\pgfsys@transformshift{3.074163in}{2.064759in}%
\pgfsys@useobject{currentmarker}{}%
\end{pgfscope}%
\end{pgfscope}%
\begin{pgfscope}%
\definecolor{textcolor}{rgb}{0.000000,0.000000,0.000000}%
\pgfsetstrokecolor{textcolor}%
\pgfsetfillcolor{textcolor}%
\pgftext[x=3.299163in,y=2.021009in,left,base]{\color{textcolor}\rmfamily\fontsize{9.000000}{10.800000}\selectfont \(\displaystyle \gamma_{19} = \) -0.15}%
\end{pgfscope}%
\begin{pgfscope}%
\pgfsetbuttcap%
\pgfsetmiterjoin%
\definecolor{currentfill}{rgb}{0.000000,0.000000,0.000000}%
\pgfsetfillcolor{currentfill}%
\pgfsetfillopacity{0.000000}%
\pgfsetlinewidth{1.003750pt}%
\definecolor{currentstroke}{rgb}{0.000000,0.500000,0.000000}%
\pgfsetstrokecolor{currentstroke}%
\pgfsetdash{}{0pt}%
\pgfsys@defobject{currentmarker}{\pgfqpoint{-0.041667in}{-0.041667in}}{\pgfqpoint{0.041667in}{0.041667in}}{%
\pgfpathmoveto{\pgfqpoint{-0.041667in}{-0.041667in}}%
\pgfpathlineto{\pgfqpoint{0.041667in}{-0.041667in}}%
\pgfpathlineto{\pgfqpoint{0.041667in}{0.041667in}}%
\pgfpathlineto{\pgfqpoint{-0.041667in}{0.041667in}}%
\pgfpathclose%
\pgfusepath{stroke,fill}%
}%
\begin{pgfscope}%
\pgfsys@transformshift{3.074163in}{1.890453in}%
\pgfsys@useobject{currentmarker}{}%
\end{pgfscope}%
\end{pgfscope}%
\begin{pgfscope}%
\definecolor{textcolor}{rgb}{0.000000,0.000000,0.000000}%
\pgfsetstrokecolor{textcolor}%
\pgfsetfillcolor{textcolor}%
\pgftext[x=3.299163in,y=1.846703in,left,base]{\color{textcolor}\rmfamily\fontsize{9.000000}{10.800000}\selectfont \(\displaystyle \gamma_{20} = \) -0.17}%
\end{pgfscope}%
\end{pgfpicture}%
\makeatother%
\endgroup%
}
					\caption{Cluster IV $(b)$}
					\label{SubFig:Cluster_V_real}
				\end{subfigure}
				\begin{subfigure}[h]{0.5\textwidth}
					\centering
					\resizebox{\linewidth}{!}{%% Creator: Matplotlib, PGF backend
%%
%% To include the figure in your LaTeX document, write
%%   \input{<filename>.pgf}
%%
%% Make sure the required packages are loaded in your preamble
%%   \usepackage{pgf}
%%
%% and, on pdftex
%%   \usepackage[utf8]{inputenc}\DeclareUnicodeCharacter{2212}{-}
%%
%% or, on luatex and xetex
%%   \usepackage{unicode-math}
%%
%% Figures using additional raster images can only be included by \input if
%% they are in the same directory as the main LaTeX file. For loading figures
%% from other directories you can use the `import` package
%%   \usepackage{import}
%%
%% and then include the figures with
%%   \import{<path to file>}{<filename>.pgf}
%%
%% Matplotlib used the following preamble
%%   \usepackage[utf8x]{inputenc}
%%   \usepackage[T1]{fontenc}
%%   \usepackage{amsmath,amssymb,amsfonts}
%%
\begingroup%
\makeatletter%
\begin{pgfpicture}%
\pgfpathrectangle{\pgfpointorigin}{\pgfqpoint{4.136389in}{2.495314in}}%
\pgfusepath{use as bounding box, clip}%
\begin{pgfscope}%
\pgfsetbuttcap%
\pgfsetmiterjoin%
\definecolor{currentfill}{rgb}{1.000000,1.000000,1.000000}%
\pgfsetfillcolor{currentfill}%
\pgfsetlinewidth{0.000000pt}%
\definecolor{currentstroke}{rgb}{1.000000,1.000000,1.000000}%
\pgfsetstrokecolor{currentstroke}%
\pgfsetdash{}{0pt}%
\pgfpathmoveto{\pgfqpoint{0.000000in}{0.000000in}}%
\pgfpathlineto{\pgfqpoint{4.136389in}{0.000000in}}%
\pgfpathlineto{\pgfqpoint{4.136389in}{2.495314in}}%
\pgfpathlineto{\pgfqpoint{0.000000in}{2.495314in}}%
\pgfpathclose%
\pgfusepath{fill}%
\end{pgfscope}%
\begin{pgfscope}%
\pgfsetbuttcap%
\pgfsetmiterjoin%
\definecolor{currentfill}{rgb}{1.000000,1.000000,1.000000}%
\pgfsetfillcolor{currentfill}%
\pgfsetlinewidth{0.000000pt}%
\definecolor{currentstroke}{rgb}{0.000000,0.000000,0.000000}%
\pgfsetstrokecolor{currentstroke}%
\pgfsetstrokeopacity{0.000000}%
\pgfsetdash{}{0pt}%
\pgfpathmoveto{\pgfqpoint{0.745371in}{0.566590in}}%
\pgfpathlineto{\pgfqpoint{4.036389in}{0.566590in}}%
\pgfpathlineto{\pgfqpoint{4.036389in}{2.395314in}}%
\pgfpathlineto{\pgfqpoint{0.745371in}{2.395314in}}%
\pgfpathclose%
\pgfusepath{fill}%
\end{pgfscope}%
\begin{pgfscope}%
\pgfpathrectangle{\pgfqpoint{0.745371in}{0.566590in}}{\pgfqpoint{3.291018in}{1.828724in}}%
\pgfusepath{clip}%
\pgfsetrectcap%
\pgfsetroundjoin%
\pgfsetlinewidth{0.803000pt}%
\definecolor{currentstroke}{rgb}{0.690196,0.690196,0.690196}%
\pgfsetstrokecolor{currentstroke}%
\pgfsetdash{}{0pt}%
\pgfpathmoveto{\pgfqpoint{0.745371in}{0.566590in}}%
\pgfpathlineto{\pgfqpoint{0.745371in}{2.395314in}}%
\pgfusepath{stroke}%
\end{pgfscope}%
\begin{pgfscope}%
\pgfsetbuttcap%
\pgfsetroundjoin%
\definecolor{currentfill}{rgb}{0.000000,0.000000,0.000000}%
\pgfsetfillcolor{currentfill}%
\pgfsetlinewidth{0.803000pt}%
\definecolor{currentstroke}{rgb}{0.000000,0.000000,0.000000}%
\pgfsetstrokecolor{currentstroke}%
\pgfsetdash{}{0pt}%
\pgfsys@defobject{currentmarker}{\pgfqpoint{0.000000in}{-0.048611in}}{\pgfqpoint{0.000000in}{0.000000in}}{%
\pgfpathmoveto{\pgfqpoint{0.000000in}{0.000000in}}%
\pgfpathlineto{\pgfqpoint{0.000000in}{-0.048611in}}%
\pgfusepath{stroke,fill}%
}%
\begin{pgfscope}%
\pgfsys@transformshift{0.745371in}{0.566590in}%
\pgfsys@useobject{currentmarker}{}%
\end{pgfscope}%
\end{pgfscope}%
\begin{pgfscope}%
\definecolor{textcolor}{rgb}{0.000000,0.000000,0.000000}%
\pgfsetstrokecolor{textcolor}%
\pgfsetfillcolor{textcolor}%
\pgftext[x=0.745371in,y=0.469368in,,top]{\color{textcolor}\rmfamily\fontsize{12.000000}{14.400000}\selectfont \(\displaystyle {-10}\)}%
\end{pgfscope}%
\begin{pgfscope}%
\pgfpathrectangle{\pgfqpoint{0.745371in}{0.566590in}}{\pgfqpoint{3.291018in}{1.828724in}}%
\pgfusepath{clip}%
\pgfsetrectcap%
\pgfsetroundjoin%
\pgfsetlinewidth{0.803000pt}%
\definecolor{currentstroke}{rgb}{0.690196,0.690196,0.690196}%
\pgfsetstrokecolor{currentstroke}%
\pgfsetdash{}{0pt}%
\pgfpathmoveto{\pgfqpoint{1.251681in}{0.566590in}}%
\pgfpathlineto{\pgfqpoint{1.251681in}{2.395314in}}%
\pgfusepath{stroke}%
\end{pgfscope}%
\begin{pgfscope}%
\pgfsetbuttcap%
\pgfsetroundjoin%
\definecolor{currentfill}{rgb}{0.000000,0.000000,0.000000}%
\pgfsetfillcolor{currentfill}%
\pgfsetlinewidth{0.803000pt}%
\definecolor{currentstroke}{rgb}{0.000000,0.000000,0.000000}%
\pgfsetstrokecolor{currentstroke}%
\pgfsetdash{}{0pt}%
\pgfsys@defobject{currentmarker}{\pgfqpoint{0.000000in}{-0.048611in}}{\pgfqpoint{0.000000in}{0.000000in}}{%
\pgfpathmoveto{\pgfqpoint{0.000000in}{0.000000in}}%
\pgfpathlineto{\pgfqpoint{0.000000in}{-0.048611in}}%
\pgfusepath{stroke,fill}%
}%
\begin{pgfscope}%
\pgfsys@transformshift{1.251681in}{0.566590in}%
\pgfsys@useobject{currentmarker}{}%
\end{pgfscope}%
\end{pgfscope}%
\begin{pgfscope}%
\definecolor{textcolor}{rgb}{0.000000,0.000000,0.000000}%
\pgfsetstrokecolor{textcolor}%
\pgfsetfillcolor{textcolor}%
\pgftext[x=1.251681in,y=0.469368in,,top]{\color{textcolor}\rmfamily\fontsize{12.000000}{14.400000}\selectfont \(\displaystyle {0}\)}%
\end{pgfscope}%
\begin{pgfscope}%
\pgfpathrectangle{\pgfqpoint{0.745371in}{0.566590in}}{\pgfqpoint{3.291018in}{1.828724in}}%
\pgfusepath{clip}%
\pgfsetrectcap%
\pgfsetroundjoin%
\pgfsetlinewidth{0.803000pt}%
\definecolor{currentstroke}{rgb}{0.690196,0.690196,0.690196}%
\pgfsetstrokecolor{currentstroke}%
\pgfsetdash{}{0pt}%
\pgfpathmoveto{\pgfqpoint{1.757992in}{0.566590in}}%
\pgfpathlineto{\pgfqpoint{1.757992in}{2.395314in}}%
\pgfusepath{stroke}%
\end{pgfscope}%
\begin{pgfscope}%
\pgfsetbuttcap%
\pgfsetroundjoin%
\definecolor{currentfill}{rgb}{0.000000,0.000000,0.000000}%
\pgfsetfillcolor{currentfill}%
\pgfsetlinewidth{0.803000pt}%
\definecolor{currentstroke}{rgb}{0.000000,0.000000,0.000000}%
\pgfsetstrokecolor{currentstroke}%
\pgfsetdash{}{0pt}%
\pgfsys@defobject{currentmarker}{\pgfqpoint{0.000000in}{-0.048611in}}{\pgfqpoint{0.000000in}{0.000000in}}{%
\pgfpathmoveto{\pgfqpoint{0.000000in}{0.000000in}}%
\pgfpathlineto{\pgfqpoint{0.000000in}{-0.048611in}}%
\pgfusepath{stroke,fill}%
}%
\begin{pgfscope}%
\pgfsys@transformshift{1.757992in}{0.566590in}%
\pgfsys@useobject{currentmarker}{}%
\end{pgfscope}%
\end{pgfscope}%
\begin{pgfscope}%
\definecolor{textcolor}{rgb}{0.000000,0.000000,0.000000}%
\pgfsetstrokecolor{textcolor}%
\pgfsetfillcolor{textcolor}%
\pgftext[x=1.757992in,y=0.469368in,,top]{\color{textcolor}\rmfamily\fontsize{12.000000}{14.400000}\selectfont \(\displaystyle {10}\)}%
\end{pgfscope}%
\begin{pgfscope}%
\pgfpathrectangle{\pgfqpoint{0.745371in}{0.566590in}}{\pgfqpoint{3.291018in}{1.828724in}}%
\pgfusepath{clip}%
\pgfsetrectcap%
\pgfsetroundjoin%
\pgfsetlinewidth{0.803000pt}%
\definecolor{currentstroke}{rgb}{0.690196,0.690196,0.690196}%
\pgfsetstrokecolor{currentstroke}%
\pgfsetdash{}{0pt}%
\pgfpathmoveto{\pgfqpoint{2.264302in}{0.566590in}}%
\pgfpathlineto{\pgfqpoint{2.264302in}{2.395314in}}%
\pgfusepath{stroke}%
\end{pgfscope}%
\begin{pgfscope}%
\pgfsetbuttcap%
\pgfsetroundjoin%
\definecolor{currentfill}{rgb}{0.000000,0.000000,0.000000}%
\pgfsetfillcolor{currentfill}%
\pgfsetlinewidth{0.803000pt}%
\definecolor{currentstroke}{rgb}{0.000000,0.000000,0.000000}%
\pgfsetstrokecolor{currentstroke}%
\pgfsetdash{}{0pt}%
\pgfsys@defobject{currentmarker}{\pgfqpoint{0.000000in}{-0.048611in}}{\pgfqpoint{0.000000in}{0.000000in}}{%
\pgfpathmoveto{\pgfqpoint{0.000000in}{0.000000in}}%
\pgfpathlineto{\pgfqpoint{0.000000in}{-0.048611in}}%
\pgfusepath{stroke,fill}%
}%
\begin{pgfscope}%
\pgfsys@transformshift{2.264302in}{0.566590in}%
\pgfsys@useobject{currentmarker}{}%
\end{pgfscope}%
\end{pgfscope}%
\begin{pgfscope}%
\definecolor{textcolor}{rgb}{0.000000,0.000000,0.000000}%
\pgfsetstrokecolor{textcolor}%
\pgfsetfillcolor{textcolor}%
\pgftext[x=2.264302in,y=0.469368in,,top]{\color{textcolor}\rmfamily\fontsize{12.000000}{14.400000}\selectfont \(\displaystyle {20}\)}%
\end{pgfscope}%
\begin{pgfscope}%
\pgfpathrectangle{\pgfqpoint{0.745371in}{0.566590in}}{\pgfqpoint{3.291018in}{1.828724in}}%
\pgfusepath{clip}%
\pgfsetrectcap%
\pgfsetroundjoin%
\pgfsetlinewidth{0.803000pt}%
\definecolor{currentstroke}{rgb}{0.690196,0.690196,0.690196}%
\pgfsetstrokecolor{currentstroke}%
\pgfsetdash{}{0pt}%
\pgfpathmoveto{\pgfqpoint{2.770613in}{0.566590in}}%
\pgfpathlineto{\pgfqpoint{2.770613in}{2.395314in}}%
\pgfusepath{stroke}%
\end{pgfscope}%
\begin{pgfscope}%
\pgfsetbuttcap%
\pgfsetroundjoin%
\definecolor{currentfill}{rgb}{0.000000,0.000000,0.000000}%
\pgfsetfillcolor{currentfill}%
\pgfsetlinewidth{0.803000pt}%
\definecolor{currentstroke}{rgb}{0.000000,0.000000,0.000000}%
\pgfsetstrokecolor{currentstroke}%
\pgfsetdash{}{0pt}%
\pgfsys@defobject{currentmarker}{\pgfqpoint{0.000000in}{-0.048611in}}{\pgfqpoint{0.000000in}{0.000000in}}{%
\pgfpathmoveto{\pgfqpoint{0.000000in}{0.000000in}}%
\pgfpathlineto{\pgfqpoint{0.000000in}{-0.048611in}}%
\pgfusepath{stroke,fill}%
}%
\begin{pgfscope}%
\pgfsys@transformshift{2.770613in}{0.566590in}%
\pgfsys@useobject{currentmarker}{}%
\end{pgfscope}%
\end{pgfscope}%
\begin{pgfscope}%
\definecolor{textcolor}{rgb}{0.000000,0.000000,0.000000}%
\pgfsetstrokecolor{textcolor}%
\pgfsetfillcolor{textcolor}%
\pgftext[x=2.770613in,y=0.469368in,,top]{\color{textcolor}\rmfamily\fontsize{12.000000}{14.400000}\selectfont \(\displaystyle {30}\)}%
\end{pgfscope}%
\begin{pgfscope}%
\pgfpathrectangle{\pgfqpoint{0.745371in}{0.566590in}}{\pgfqpoint{3.291018in}{1.828724in}}%
\pgfusepath{clip}%
\pgfsetrectcap%
\pgfsetroundjoin%
\pgfsetlinewidth{0.803000pt}%
\definecolor{currentstroke}{rgb}{0.690196,0.690196,0.690196}%
\pgfsetstrokecolor{currentstroke}%
\pgfsetdash{}{0pt}%
\pgfpathmoveto{\pgfqpoint{3.276923in}{0.566590in}}%
\pgfpathlineto{\pgfqpoint{3.276923in}{2.395314in}}%
\pgfusepath{stroke}%
\end{pgfscope}%
\begin{pgfscope}%
\pgfsetbuttcap%
\pgfsetroundjoin%
\definecolor{currentfill}{rgb}{0.000000,0.000000,0.000000}%
\pgfsetfillcolor{currentfill}%
\pgfsetlinewidth{0.803000pt}%
\definecolor{currentstroke}{rgb}{0.000000,0.000000,0.000000}%
\pgfsetstrokecolor{currentstroke}%
\pgfsetdash{}{0pt}%
\pgfsys@defobject{currentmarker}{\pgfqpoint{0.000000in}{-0.048611in}}{\pgfqpoint{0.000000in}{0.000000in}}{%
\pgfpathmoveto{\pgfqpoint{0.000000in}{0.000000in}}%
\pgfpathlineto{\pgfqpoint{0.000000in}{-0.048611in}}%
\pgfusepath{stroke,fill}%
}%
\begin{pgfscope}%
\pgfsys@transformshift{3.276923in}{0.566590in}%
\pgfsys@useobject{currentmarker}{}%
\end{pgfscope}%
\end{pgfscope}%
\begin{pgfscope}%
\definecolor{textcolor}{rgb}{0.000000,0.000000,0.000000}%
\pgfsetstrokecolor{textcolor}%
\pgfsetfillcolor{textcolor}%
\pgftext[x=3.276923in,y=0.469368in,,top]{\color{textcolor}\rmfamily\fontsize{12.000000}{14.400000}\selectfont \(\displaystyle {40}\)}%
\end{pgfscope}%
\begin{pgfscope}%
\pgfpathrectangle{\pgfqpoint{0.745371in}{0.566590in}}{\pgfqpoint{3.291018in}{1.828724in}}%
\pgfusepath{clip}%
\pgfsetrectcap%
\pgfsetroundjoin%
\pgfsetlinewidth{0.803000pt}%
\definecolor{currentstroke}{rgb}{0.690196,0.690196,0.690196}%
\pgfsetstrokecolor{currentstroke}%
\pgfsetdash{}{0pt}%
\pgfpathmoveto{\pgfqpoint{3.783233in}{0.566590in}}%
\pgfpathlineto{\pgfqpoint{3.783233in}{2.395314in}}%
\pgfusepath{stroke}%
\end{pgfscope}%
\begin{pgfscope}%
\pgfsetbuttcap%
\pgfsetroundjoin%
\definecolor{currentfill}{rgb}{0.000000,0.000000,0.000000}%
\pgfsetfillcolor{currentfill}%
\pgfsetlinewidth{0.803000pt}%
\definecolor{currentstroke}{rgb}{0.000000,0.000000,0.000000}%
\pgfsetstrokecolor{currentstroke}%
\pgfsetdash{}{0pt}%
\pgfsys@defobject{currentmarker}{\pgfqpoint{0.000000in}{-0.048611in}}{\pgfqpoint{0.000000in}{0.000000in}}{%
\pgfpathmoveto{\pgfqpoint{0.000000in}{0.000000in}}%
\pgfpathlineto{\pgfqpoint{0.000000in}{-0.048611in}}%
\pgfusepath{stroke,fill}%
}%
\begin{pgfscope}%
\pgfsys@transformshift{3.783233in}{0.566590in}%
\pgfsys@useobject{currentmarker}{}%
\end{pgfscope}%
\end{pgfscope}%
\begin{pgfscope}%
\definecolor{textcolor}{rgb}{0.000000,0.000000,0.000000}%
\pgfsetstrokecolor{textcolor}%
\pgfsetfillcolor{textcolor}%
\pgftext[x=3.783233in,y=0.469368in,,top]{\color{textcolor}\rmfamily\fontsize{12.000000}{14.400000}\selectfont \(\displaystyle {50}\)}%
\end{pgfscope}%
\begin{pgfscope}%
\definecolor{textcolor}{rgb}{0.000000,0.000000,0.000000}%
\pgfsetstrokecolor{textcolor}%
\pgfsetfillcolor{textcolor}%
\pgftext[x=2.390880in,y=0.266626in,,top]{\color{textcolor}\rmfamily\fontsize{12.000000}{14.400000}\selectfont SNR [dB]}%
\end{pgfscope}%
\begin{pgfscope}%
\pgfpathrectangle{\pgfqpoint{0.745371in}{0.566590in}}{\pgfqpoint{3.291018in}{1.828724in}}%
\pgfusepath{clip}%
\pgfsetrectcap%
\pgfsetroundjoin%
\pgfsetlinewidth{0.803000pt}%
\definecolor{currentstroke}{rgb}{0.690196,0.690196,0.690196}%
\pgfsetstrokecolor{currentstroke}%
\pgfsetdash{}{0pt}%
\pgfpathmoveto{\pgfqpoint{0.745371in}{0.752967in}}%
\pgfpathlineto{\pgfqpoint{4.036389in}{0.752967in}}%
\pgfusepath{stroke}%
\end{pgfscope}%
\begin{pgfscope}%
\pgfsetbuttcap%
\pgfsetroundjoin%
\definecolor{currentfill}{rgb}{0.000000,0.000000,0.000000}%
\pgfsetfillcolor{currentfill}%
\pgfsetlinewidth{0.803000pt}%
\definecolor{currentstroke}{rgb}{0.000000,0.000000,0.000000}%
\pgfsetstrokecolor{currentstroke}%
\pgfsetdash{}{0pt}%
\pgfsys@defobject{currentmarker}{\pgfqpoint{-0.048611in}{0.000000in}}{\pgfqpoint{-0.000000in}{0.000000in}}{%
\pgfpathmoveto{\pgfqpoint{-0.000000in}{0.000000in}}%
\pgfpathlineto{\pgfqpoint{-0.048611in}{0.000000in}}%
\pgfusepath{stroke,fill}%
}%
\begin{pgfscope}%
\pgfsys@transformshift{0.745371in}{0.752967in}%
\pgfsys@useobject{currentmarker}{}%
\end{pgfscope}%
\end{pgfscope}%
\begin{pgfscope}%
\definecolor{textcolor}{rgb}{0.000000,0.000000,0.000000}%
\pgfsetstrokecolor{textcolor}%
\pgfsetfillcolor{textcolor}%
\pgftext[x=0.327160in, y=0.695574in, left, base]{\color{textcolor}\rmfamily\fontsize{12.000000}{14.400000}\selectfont \(\displaystyle {10^{-4}}\)}%
\end{pgfscope}%
\begin{pgfscope}%
\pgfpathrectangle{\pgfqpoint{0.745371in}{0.566590in}}{\pgfqpoint{3.291018in}{1.828724in}}%
\pgfusepath{clip}%
\pgfsetrectcap%
\pgfsetroundjoin%
\pgfsetlinewidth{0.803000pt}%
\definecolor{currentstroke}{rgb}{0.690196,0.690196,0.690196}%
\pgfsetstrokecolor{currentstroke}%
\pgfsetdash{}{0pt}%
\pgfpathmoveto{\pgfqpoint{0.745371in}{1.236303in}}%
\pgfpathlineto{\pgfqpoint{4.036389in}{1.236303in}}%
\pgfusepath{stroke}%
\end{pgfscope}%
\begin{pgfscope}%
\pgfsetbuttcap%
\pgfsetroundjoin%
\definecolor{currentfill}{rgb}{0.000000,0.000000,0.000000}%
\pgfsetfillcolor{currentfill}%
\pgfsetlinewidth{0.803000pt}%
\definecolor{currentstroke}{rgb}{0.000000,0.000000,0.000000}%
\pgfsetstrokecolor{currentstroke}%
\pgfsetdash{}{0pt}%
\pgfsys@defobject{currentmarker}{\pgfqpoint{-0.048611in}{0.000000in}}{\pgfqpoint{-0.000000in}{0.000000in}}{%
\pgfpathmoveto{\pgfqpoint{-0.000000in}{0.000000in}}%
\pgfpathlineto{\pgfqpoint{-0.048611in}{0.000000in}}%
\pgfusepath{stroke,fill}%
}%
\begin{pgfscope}%
\pgfsys@transformshift{0.745371in}{1.236303in}%
\pgfsys@useobject{currentmarker}{}%
\end{pgfscope}%
\end{pgfscope}%
\begin{pgfscope}%
\definecolor{textcolor}{rgb}{0.000000,0.000000,0.000000}%
\pgfsetstrokecolor{textcolor}%
\pgfsetfillcolor{textcolor}%
\pgftext[x=0.327160in, y=1.178910in, left, base]{\color{textcolor}\rmfamily\fontsize{12.000000}{14.400000}\selectfont \(\displaystyle {10^{-2}}\)}%
\end{pgfscope}%
\begin{pgfscope}%
\pgfpathrectangle{\pgfqpoint{0.745371in}{0.566590in}}{\pgfqpoint{3.291018in}{1.828724in}}%
\pgfusepath{clip}%
\pgfsetrectcap%
\pgfsetroundjoin%
\pgfsetlinewidth{0.803000pt}%
\definecolor{currentstroke}{rgb}{0.690196,0.690196,0.690196}%
\pgfsetstrokecolor{currentstroke}%
\pgfsetdash{}{0pt}%
\pgfpathmoveto{\pgfqpoint{0.745371in}{1.719639in}}%
\pgfpathlineto{\pgfqpoint{4.036389in}{1.719639in}}%
\pgfusepath{stroke}%
\end{pgfscope}%
\begin{pgfscope}%
\pgfsetbuttcap%
\pgfsetroundjoin%
\definecolor{currentfill}{rgb}{0.000000,0.000000,0.000000}%
\pgfsetfillcolor{currentfill}%
\pgfsetlinewidth{0.803000pt}%
\definecolor{currentstroke}{rgb}{0.000000,0.000000,0.000000}%
\pgfsetstrokecolor{currentstroke}%
\pgfsetdash{}{0pt}%
\pgfsys@defobject{currentmarker}{\pgfqpoint{-0.048611in}{0.000000in}}{\pgfqpoint{-0.000000in}{0.000000in}}{%
\pgfpathmoveto{\pgfqpoint{-0.000000in}{0.000000in}}%
\pgfpathlineto{\pgfqpoint{-0.048611in}{0.000000in}}%
\pgfusepath{stroke,fill}%
}%
\begin{pgfscope}%
\pgfsys@transformshift{0.745371in}{1.719639in}%
\pgfsys@useobject{currentmarker}{}%
\end{pgfscope}%
\end{pgfscope}%
\begin{pgfscope}%
\definecolor{textcolor}{rgb}{0.000000,0.000000,0.000000}%
\pgfsetstrokecolor{textcolor}%
\pgfsetfillcolor{textcolor}%
\pgftext[x=0.418983in, y=1.662246in, left, base]{\color{textcolor}\rmfamily\fontsize{12.000000}{14.400000}\selectfont \(\displaystyle {10^{0}}\)}%
\end{pgfscope}%
\begin{pgfscope}%
\pgfpathrectangle{\pgfqpoint{0.745371in}{0.566590in}}{\pgfqpoint{3.291018in}{1.828724in}}%
\pgfusepath{clip}%
\pgfsetrectcap%
\pgfsetroundjoin%
\pgfsetlinewidth{0.803000pt}%
\definecolor{currentstroke}{rgb}{0.690196,0.690196,0.690196}%
\pgfsetstrokecolor{currentstroke}%
\pgfsetdash{}{0pt}%
\pgfpathmoveto{\pgfqpoint{0.745371in}{2.202975in}}%
\pgfpathlineto{\pgfqpoint{4.036389in}{2.202975in}}%
\pgfusepath{stroke}%
\end{pgfscope}%
\begin{pgfscope}%
\pgfsetbuttcap%
\pgfsetroundjoin%
\definecolor{currentfill}{rgb}{0.000000,0.000000,0.000000}%
\pgfsetfillcolor{currentfill}%
\pgfsetlinewidth{0.803000pt}%
\definecolor{currentstroke}{rgb}{0.000000,0.000000,0.000000}%
\pgfsetstrokecolor{currentstroke}%
\pgfsetdash{}{0pt}%
\pgfsys@defobject{currentmarker}{\pgfqpoint{-0.048611in}{0.000000in}}{\pgfqpoint{-0.000000in}{0.000000in}}{%
\pgfpathmoveto{\pgfqpoint{-0.000000in}{0.000000in}}%
\pgfpathlineto{\pgfqpoint{-0.048611in}{0.000000in}}%
\pgfusepath{stroke,fill}%
}%
\begin{pgfscope}%
\pgfsys@transformshift{0.745371in}{2.202975in}%
\pgfsys@useobject{currentmarker}{}%
\end{pgfscope}%
\end{pgfscope}%
\begin{pgfscope}%
\definecolor{textcolor}{rgb}{0.000000,0.000000,0.000000}%
\pgfsetstrokecolor{textcolor}%
\pgfsetfillcolor{textcolor}%
\pgftext[x=0.418983in, y=2.145582in, left, base]{\color{textcolor}\rmfamily\fontsize{12.000000}{14.400000}\selectfont \(\displaystyle {10^{2}}\)}%
\end{pgfscope}%
\begin{pgfscope}%
\definecolor{textcolor}{rgb}{0.000000,0.000000,0.000000}%
\pgfsetstrokecolor{textcolor}%
\pgfsetfillcolor{textcolor}%
\pgftext[x=0.271605in,y=1.480952in,,bottom,rotate=90.000000]{\color{textcolor}\rmfamily\fontsize{12.000000}{14.400000}\selectfont \(\displaystyle \hat{\sigma}_{\gamma}(\mathrm{SNR})\)}%
\end{pgfscope}%
\begin{pgfscope}%
\pgfpathrectangle{\pgfqpoint{0.745371in}{0.566590in}}{\pgfqpoint{3.291018in}{1.828724in}}%
\pgfusepath{clip}%
\pgfsetbuttcap%
\pgfsetroundjoin%
\pgfsetlinewidth{1.505625pt}%
\definecolor{currentstroke}{rgb}{0.000000,0.447000,0.741000}%
\pgfsetstrokecolor{currentstroke}%
\pgfsetdash{{5.550000pt}{2.400000pt}}{0.000000pt}%
\pgfpathmoveto{\pgfqpoint{0.745371in}{2.027169in}}%
\pgfpathlineto{\pgfqpoint{0.842165in}{2.134812in}}%
\pgfpathlineto{\pgfqpoint{0.938960in}{2.015455in}}%
\pgfpathlineto{\pgfqpoint{1.035755in}{2.163238in}}%
\pgfpathlineto{\pgfqpoint{1.132549in}{2.093779in}}%
\pgfpathlineto{\pgfqpoint{1.229344in}{2.173471in}}%
\pgfpathlineto{\pgfqpoint{1.326139in}{2.189963in}}%
\pgfpathlineto{\pgfqpoint{1.422933in}{2.204604in}}%
\pgfpathlineto{\pgfqpoint{1.519728in}{2.165104in}}%
\pgfpathlineto{\pgfqpoint{1.616523in}{2.139085in}}%
\pgfpathlineto{\pgfqpoint{1.713317in}{2.139725in}}%
\pgfpathlineto{\pgfqpoint{1.810112in}{2.151871in}}%
\pgfpathlineto{\pgfqpoint{1.906906in}{2.248759in}}%
\pgfpathlineto{\pgfqpoint{2.003701in}{2.153015in}}%
\pgfpathlineto{\pgfqpoint{2.100496in}{1.937141in}}%
\pgfpathlineto{\pgfqpoint{2.197290in}{1.753214in}}%
\pgfpathlineto{\pgfqpoint{2.294085in}{1.717696in}}%
\pgfpathlineto{\pgfqpoint{2.390880in}{1.690860in}}%
\pgfpathlineto{\pgfqpoint{2.487674in}{1.660642in}}%
\pgfpathlineto{\pgfqpoint{2.584469in}{1.629161in}}%
\pgfpathlineto{\pgfqpoint{2.681264in}{1.617389in}}%
\pgfpathlineto{\pgfqpoint{2.778058in}{1.585587in}}%
\pgfpathlineto{\pgfqpoint{2.874853in}{1.564265in}}%
\pgfpathlineto{\pgfqpoint{2.971648in}{1.553815in}}%
\pgfpathlineto{\pgfqpoint{3.068442in}{1.514846in}}%
\pgfpathlineto{\pgfqpoint{3.165237in}{1.477165in}}%
\pgfpathlineto{\pgfqpoint{3.262031in}{1.478088in}}%
\pgfpathlineto{\pgfqpoint{3.358826in}{1.445391in}}%
\pgfpathlineto{\pgfqpoint{3.455621in}{1.420830in}}%
\pgfpathlineto{\pgfqpoint{3.552415in}{1.403315in}}%
\pgfpathlineto{\pgfqpoint{3.649210in}{1.386745in}}%
\pgfpathlineto{\pgfqpoint{3.746005in}{1.348536in}}%
\pgfpathlineto{\pgfqpoint{3.842799in}{1.335670in}}%
\pgfpathlineto{\pgfqpoint{3.939594in}{1.304370in}}%
\pgfpathlineto{\pgfqpoint{4.036389in}{1.283963in}}%
\pgfusepath{stroke}%
\end{pgfscope}%
\begin{pgfscope}%
\pgfpathrectangle{\pgfqpoint{0.745371in}{0.566590in}}{\pgfqpoint{3.291018in}{1.828724in}}%
\pgfusepath{clip}%
\pgfsetbuttcap%
\pgfsetroundjoin%
\definecolor{currentfill}{rgb}{0.000000,0.000000,0.000000}%
\pgfsetfillcolor{currentfill}%
\pgfsetfillopacity{0.000000}%
\pgfsetlinewidth{1.003750pt}%
\definecolor{currentstroke}{rgb}{0.000000,0.447000,0.741000}%
\pgfsetstrokecolor{currentstroke}%
\pgfsetdash{}{0pt}%
\pgfsys@defobject{currentmarker}{\pgfqpoint{-0.041667in}{-0.041667in}}{\pgfqpoint{0.041667in}{0.041667in}}{%
\pgfpathmoveto{\pgfqpoint{0.000000in}{-0.041667in}}%
\pgfpathcurveto{\pgfqpoint{0.011050in}{-0.041667in}}{\pgfqpoint{0.021649in}{-0.037276in}}{\pgfqpoint{0.029463in}{-0.029463in}}%
\pgfpathcurveto{\pgfqpoint{0.037276in}{-0.021649in}}{\pgfqpoint{0.041667in}{-0.011050in}}{\pgfqpoint{0.041667in}{0.000000in}}%
\pgfpathcurveto{\pgfqpoint{0.041667in}{0.011050in}}{\pgfqpoint{0.037276in}{0.021649in}}{\pgfqpoint{0.029463in}{0.029463in}}%
\pgfpathcurveto{\pgfqpoint{0.021649in}{0.037276in}}{\pgfqpoint{0.011050in}{0.041667in}}{\pgfqpoint{0.000000in}{0.041667in}}%
\pgfpathcurveto{\pgfqpoint{-0.011050in}{0.041667in}}{\pgfqpoint{-0.021649in}{0.037276in}}{\pgfqpoint{-0.029463in}{0.029463in}}%
\pgfpathcurveto{\pgfqpoint{-0.037276in}{0.021649in}}{\pgfqpoint{-0.041667in}{0.011050in}}{\pgfqpoint{-0.041667in}{0.000000in}}%
\pgfpathcurveto{\pgfqpoint{-0.041667in}{-0.011050in}}{\pgfqpoint{-0.037276in}{-0.021649in}}{\pgfqpoint{-0.029463in}{-0.029463in}}%
\pgfpathcurveto{\pgfqpoint{-0.021649in}{-0.037276in}}{\pgfqpoint{-0.011050in}{-0.041667in}}{\pgfqpoint{0.000000in}{-0.041667in}}%
\pgfpathclose%
\pgfusepath{stroke,fill}%
}%
\begin{pgfscope}%
\pgfsys@transformshift{0.745371in}{2.027169in}%
\pgfsys@useobject{currentmarker}{}%
\end{pgfscope}%
\begin{pgfscope}%
\pgfsys@transformshift{1.132549in}{2.093779in}%
\pgfsys@useobject{currentmarker}{}%
\end{pgfscope}%
\begin{pgfscope}%
\pgfsys@transformshift{1.519728in}{2.165104in}%
\pgfsys@useobject{currentmarker}{}%
\end{pgfscope}%
\begin{pgfscope}%
\pgfsys@transformshift{1.906906in}{2.248759in}%
\pgfsys@useobject{currentmarker}{}%
\end{pgfscope}%
\begin{pgfscope}%
\pgfsys@transformshift{2.294085in}{1.717696in}%
\pgfsys@useobject{currentmarker}{}%
\end{pgfscope}%
\begin{pgfscope}%
\pgfsys@transformshift{2.681264in}{1.617389in}%
\pgfsys@useobject{currentmarker}{}%
\end{pgfscope}%
\begin{pgfscope}%
\pgfsys@transformshift{3.068442in}{1.514846in}%
\pgfsys@useobject{currentmarker}{}%
\end{pgfscope}%
\begin{pgfscope}%
\pgfsys@transformshift{3.455621in}{1.420830in}%
\pgfsys@useobject{currentmarker}{}%
\end{pgfscope}%
\begin{pgfscope}%
\pgfsys@transformshift{3.842799in}{1.335670in}%
\pgfsys@useobject{currentmarker}{}%
\end{pgfscope}%
\end{pgfscope}%
\begin{pgfscope}%
\pgfpathrectangle{\pgfqpoint{0.745371in}{0.566590in}}{\pgfqpoint{3.291018in}{1.828724in}}%
\pgfusepath{clip}%
\pgfsetbuttcap%
\pgfsetroundjoin%
\pgfsetlinewidth{1.505625pt}%
\definecolor{currentstroke}{rgb}{0.850000,0.324000,0.098000}%
\pgfsetstrokecolor{currentstroke}%
\pgfsetdash{{5.550000pt}{2.400000pt}}{0.000000pt}%
\pgfpathmoveto{\pgfqpoint{0.745371in}{2.291374in}}%
\pgfpathlineto{\pgfqpoint{0.842165in}{2.016464in}}%
\pgfpathlineto{\pgfqpoint{0.938960in}{2.051305in}}%
\pgfpathlineto{\pgfqpoint{1.035755in}{2.133650in}}%
\pgfpathlineto{\pgfqpoint{1.132549in}{2.162101in}}%
\pgfpathlineto{\pgfqpoint{1.229344in}{2.167663in}}%
\pgfpathlineto{\pgfqpoint{1.326139in}{2.056108in}}%
\pgfpathlineto{\pgfqpoint{1.422933in}{2.036873in}}%
\pgfpathlineto{\pgfqpoint{1.519728in}{1.688899in}}%
\pgfpathlineto{\pgfqpoint{1.616523in}{1.838842in}}%
\pgfpathlineto{\pgfqpoint{1.713317in}{1.768410in}}%
\pgfpathlineto{\pgfqpoint{1.810112in}{1.872564in}}%
\pgfpathlineto{\pgfqpoint{1.906906in}{1.869460in}}%
\pgfpathlineto{\pgfqpoint{2.003701in}{1.916992in}}%
\pgfpathlineto{\pgfqpoint{2.100496in}{1.842146in}}%
\pgfpathlineto{\pgfqpoint{2.197290in}{1.774575in}}%
\pgfpathlineto{\pgfqpoint{2.294085in}{1.747732in}}%
\pgfpathlineto{\pgfqpoint{2.390880in}{1.698894in}}%
\pgfpathlineto{\pgfqpoint{2.487674in}{1.670691in}}%
\pgfpathlineto{\pgfqpoint{2.584469in}{1.639097in}}%
\pgfpathlineto{\pgfqpoint{2.681264in}{1.624687in}}%
\pgfpathlineto{\pgfqpoint{2.778058in}{1.598834in}}%
\pgfpathlineto{\pgfqpoint{2.874853in}{1.585009in}}%
\pgfpathlineto{\pgfqpoint{2.971648in}{1.566553in}}%
\pgfpathlineto{\pgfqpoint{3.068442in}{1.523807in}}%
\pgfpathlineto{\pgfqpoint{3.165237in}{1.496928in}}%
\pgfpathlineto{\pgfqpoint{3.262031in}{1.494434in}}%
\pgfpathlineto{\pgfqpoint{3.358826in}{1.462412in}}%
\pgfpathlineto{\pgfqpoint{3.455621in}{1.437549in}}%
\pgfpathlineto{\pgfqpoint{3.552415in}{1.415647in}}%
\pgfpathlineto{\pgfqpoint{3.649210in}{1.396968in}}%
\pgfpathlineto{\pgfqpoint{3.746005in}{1.363443in}}%
\pgfpathlineto{\pgfqpoint{3.842799in}{1.342755in}}%
\pgfpathlineto{\pgfqpoint{3.939594in}{1.323452in}}%
\pgfpathlineto{\pgfqpoint{4.036389in}{1.297359in}}%
\pgfusepath{stroke}%
\end{pgfscope}%
\begin{pgfscope}%
\pgfpathrectangle{\pgfqpoint{0.745371in}{0.566590in}}{\pgfqpoint{3.291018in}{1.828724in}}%
\pgfusepath{clip}%
\pgfsetbuttcap%
\pgfsetroundjoin%
\definecolor{currentfill}{rgb}{0.850000,0.324000,0.098000}%
\pgfsetfillcolor{currentfill}%
\pgfsetlinewidth{1.003750pt}%
\definecolor{currentstroke}{rgb}{0.850000,0.324000,0.098000}%
\pgfsetstrokecolor{currentstroke}%
\pgfsetdash{}{0pt}%
\pgfsys@defobject{currentmarker}{\pgfqpoint{-0.041667in}{-0.041667in}}{\pgfqpoint{0.041667in}{0.041667in}}{%
\pgfpathmoveto{\pgfqpoint{-0.041667in}{0.000000in}}%
\pgfpathlineto{\pgfqpoint{0.041667in}{0.000000in}}%
\pgfpathmoveto{\pgfqpoint{0.000000in}{-0.041667in}}%
\pgfpathlineto{\pgfqpoint{0.000000in}{0.041667in}}%
\pgfusepath{stroke,fill}%
}%
\begin{pgfscope}%
\pgfsys@transformshift{0.745371in}{2.291374in}%
\pgfsys@useobject{currentmarker}{}%
\end{pgfscope}%
\begin{pgfscope}%
\pgfsys@transformshift{1.035755in}{2.133650in}%
\pgfsys@useobject{currentmarker}{}%
\end{pgfscope}%
\begin{pgfscope}%
\pgfsys@transformshift{1.326139in}{2.056108in}%
\pgfsys@useobject{currentmarker}{}%
\end{pgfscope}%
\begin{pgfscope}%
\pgfsys@transformshift{1.616523in}{1.838842in}%
\pgfsys@useobject{currentmarker}{}%
\end{pgfscope}%
\begin{pgfscope}%
\pgfsys@transformshift{1.906906in}{1.869460in}%
\pgfsys@useobject{currentmarker}{}%
\end{pgfscope}%
\begin{pgfscope}%
\pgfsys@transformshift{2.197290in}{1.774575in}%
\pgfsys@useobject{currentmarker}{}%
\end{pgfscope}%
\begin{pgfscope}%
\pgfsys@transformshift{2.487674in}{1.670691in}%
\pgfsys@useobject{currentmarker}{}%
\end{pgfscope}%
\begin{pgfscope}%
\pgfsys@transformshift{2.778058in}{1.598834in}%
\pgfsys@useobject{currentmarker}{}%
\end{pgfscope}%
\begin{pgfscope}%
\pgfsys@transformshift{3.068442in}{1.523807in}%
\pgfsys@useobject{currentmarker}{}%
\end{pgfscope}%
\begin{pgfscope}%
\pgfsys@transformshift{3.358826in}{1.462412in}%
\pgfsys@useobject{currentmarker}{}%
\end{pgfscope}%
\begin{pgfscope}%
\pgfsys@transformshift{3.649210in}{1.396968in}%
\pgfsys@useobject{currentmarker}{}%
\end{pgfscope}%
\begin{pgfscope}%
\pgfsys@transformshift{3.939594in}{1.323452in}%
\pgfsys@useobject{currentmarker}{}%
\end{pgfscope}%
\end{pgfscope}%
\begin{pgfscope}%
\pgfpathrectangle{\pgfqpoint{0.745371in}{0.566590in}}{\pgfqpoint{3.291018in}{1.828724in}}%
\pgfusepath{clip}%
\pgfsetbuttcap%
\pgfsetroundjoin%
\pgfsetlinewidth{1.505625pt}%
\definecolor{currentstroke}{rgb}{0.000000,0.500000,0.000000}%
\pgfsetstrokecolor{currentstroke}%
\pgfsetdash{{5.550000pt}{2.400000pt}}{0.000000pt}%
\pgfpathmoveto{\pgfqpoint{0.745371in}{2.001458in}}%
\pgfpathlineto{\pgfqpoint{0.842165in}{2.042872in}}%
\pgfpathlineto{\pgfqpoint{0.938960in}{1.953355in}}%
\pgfpathlineto{\pgfqpoint{1.035755in}{2.096811in}}%
\pgfpathlineto{\pgfqpoint{1.132549in}{1.668097in}}%
\pgfpathlineto{\pgfqpoint{1.229344in}{1.983935in}}%
\pgfpathlineto{\pgfqpoint{1.326139in}{1.805632in}}%
\pgfpathlineto{\pgfqpoint{1.422933in}{1.990251in}}%
\pgfpathlineto{\pgfqpoint{1.519728in}{1.902740in}}%
\pgfpathlineto{\pgfqpoint{1.616523in}{1.943785in}}%
\pgfpathlineto{\pgfqpoint{1.713317in}{1.963934in}}%
\pgfpathlineto{\pgfqpoint{1.810112in}{1.895945in}}%
\pgfpathlineto{\pgfqpoint{1.906906in}{1.945982in}}%
\pgfpathlineto{\pgfqpoint{2.003701in}{1.954688in}}%
\pgfpathlineto{\pgfqpoint{2.100496in}{1.886453in}}%
\pgfpathlineto{\pgfqpoint{2.197290in}{1.758865in}}%
\pgfpathlineto{\pgfqpoint{2.294085in}{1.729934in}}%
\pgfpathlineto{\pgfqpoint{2.390880in}{1.680386in}}%
\pgfpathlineto{\pgfqpoint{2.487674in}{1.654748in}}%
\pgfpathlineto{\pgfqpoint{2.584469in}{1.614033in}}%
\pgfpathlineto{\pgfqpoint{2.681264in}{1.602892in}}%
\pgfpathlineto{\pgfqpoint{2.778058in}{1.561568in}}%
\pgfpathlineto{\pgfqpoint{2.874853in}{1.546598in}}%
\pgfpathlineto{\pgfqpoint{2.971648in}{1.539999in}}%
\pgfpathlineto{\pgfqpoint{3.068442in}{1.495706in}}%
\pgfpathlineto{\pgfqpoint{3.165237in}{1.465665in}}%
\pgfpathlineto{\pgfqpoint{3.262031in}{1.456903in}}%
\pgfpathlineto{\pgfqpoint{3.358826in}{1.426451in}}%
\pgfpathlineto{\pgfqpoint{3.455621in}{1.405458in}}%
\pgfpathlineto{\pgfqpoint{3.552415in}{1.379218in}}%
\pgfpathlineto{\pgfqpoint{3.649210in}{1.368202in}}%
\pgfpathlineto{\pgfqpoint{3.746005in}{1.325441in}}%
\pgfpathlineto{\pgfqpoint{3.842799in}{1.315618in}}%
\pgfpathlineto{\pgfqpoint{3.939594in}{1.284524in}}%
\pgfpathlineto{\pgfqpoint{4.036389in}{1.267021in}}%
\pgfusepath{stroke}%
\end{pgfscope}%
\begin{pgfscope}%
\pgfpathrectangle{\pgfqpoint{0.745371in}{0.566590in}}{\pgfqpoint{3.291018in}{1.828724in}}%
\pgfusepath{clip}%
\pgfsetbuttcap%
\pgfsetmiterjoin%
\definecolor{currentfill}{rgb}{0.000000,0.000000,0.000000}%
\pgfsetfillcolor{currentfill}%
\pgfsetfillopacity{0.000000}%
\pgfsetlinewidth{1.003750pt}%
\definecolor{currentstroke}{rgb}{0.000000,0.500000,0.000000}%
\pgfsetstrokecolor{currentstroke}%
\pgfsetdash{}{0pt}%
\pgfsys@defobject{currentmarker}{\pgfqpoint{-0.041667in}{-0.041667in}}{\pgfqpoint{0.041667in}{0.041667in}}{%
\pgfpathmoveto{\pgfqpoint{-0.041667in}{-0.041667in}}%
\pgfpathlineto{\pgfqpoint{0.041667in}{-0.041667in}}%
\pgfpathlineto{\pgfqpoint{0.041667in}{0.041667in}}%
\pgfpathlineto{\pgfqpoint{-0.041667in}{0.041667in}}%
\pgfpathclose%
\pgfusepath{stroke,fill}%
}%
\begin{pgfscope}%
\pgfsys@transformshift{0.745371in}{2.001458in}%
\pgfsys@useobject{currentmarker}{}%
\end{pgfscope}%
\begin{pgfscope}%
\pgfsys@transformshift{1.229344in}{1.983935in}%
\pgfsys@useobject{currentmarker}{}%
\end{pgfscope}%
\begin{pgfscope}%
\pgfsys@transformshift{1.713317in}{1.963934in}%
\pgfsys@useobject{currentmarker}{}%
\end{pgfscope}%
\begin{pgfscope}%
\pgfsys@transformshift{2.197290in}{1.758865in}%
\pgfsys@useobject{currentmarker}{}%
\end{pgfscope}%
\begin{pgfscope}%
\pgfsys@transformshift{2.681264in}{1.602892in}%
\pgfsys@useobject{currentmarker}{}%
\end{pgfscope}%
\begin{pgfscope}%
\pgfsys@transformshift{3.165237in}{1.465665in}%
\pgfsys@useobject{currentmarker}{}%
\end{pgfscope}%
\begin{pgfscope}%
\pgfsys@transformshift{3.649210in}{1.368202in}%
\pgfsys@useobject{currentmarker}{}%
\end{pgfscope}%
\end{pgfscope}%
\begin{pgfscope}%
\pgfpathrectangle{\pgfqpoint{0.745371in}{0.566590in}}{\pgfqpoint{3.291018in}{1.828724in}}%
\pgfusepath{clip}%
\pgfsetbuttcap%
\pgfsetroundjoin%
\pgfsetlinewidth{1.505625pt}%
\definecolor{currentstroke}{rgb}{0.494000,0.184000,0.556000}%
\pgfsetstrokecolor{currentstroke}%
\pgfsetdash{{5.550000pt}{2.400000pt}}{0.000000pt}%
\pgfpathmoveto{\pgfqpoint{0.745371in}{2.072721in}}%
\pgfpathlineto{\pgfqpoint{0.842165in}{2.057889in}}%
\pgfpathlineto{\pgfqpoint{0.938960in}{1.963091in}}%
\pgfpathlineto{\pgfqpoint{1.035755in}{2.234764in}}%
\pgfpathlineto{\pgfqpoint{1.132549in}{2.033821in}}%
\pgfpathlineto{\pgfqpoint{1.229344in}{2.215223in}}%
\pgfpathlineto{\pgfqpoint{1.326139in}{1.905907in}}%
\pgfpathlineto{\pgfqpoint{1.422933in}{1.939406in}}%
\pgfpathlineto{\pgfqpoint{1.519728in}{1.967303in}}%
\pgfpathlineto{\pgfqpoint{1.616523in}{2.121703in}}%
\pgfpathlineto{\pgfqpoint{1.713317in}{2.121102in}}%
\pgfpathlineto{\pgfqpoint{1.810112in}{1.894597in}}%
\pgfpathlineto{\pgfqpoint{1.906906in}{1.921200in}}%
\pgfpathlineto{\pgfqpoint{2.003701in}{1.908290in}}%
\pgfpathlineto{\pgfqpoint{2.100496in}{2.001078in}}%
\pgfpathlineto{\pgfqpoint{2.197290in}{1.807839in}}%
\pgfpathlineto{\pgfqpoint{2.294085in}{1.566828in}}%
\pgfpathlineto{\pgfqpoint{2.390880in}{1.526031in}}%
\pgfpathlineto{\pgfqpoint{2.487674in}{1.503909in}}%
\pgfpathlineto{\pgfqpoint{2.584469in}{1.467611in}}%
\pgfpathlineto{\pgfqpoint{2.681264in}{1.454796in}}%
\pgfpathlineto{\pgfqpoint{2.778058in}{1.417602in}}%
\pgfpathlineto{\pgfqpoint{2.874853in}{1.406849in}}%
\pgfpathlineto{\pgfqpoint{2.971648in}{1.392484in}}%
\pgfpathlineto{\pgfqpoint{3.068442in}{1.346051in}}%
\pgfpathlineto{\pgfqpoint{3.165237in}{1.327202in}}%
\pgfpathlineto{\pgfqpoint{3.262031in}{1.312389in}}%
\pgfpathlineto{\pgfqpoint{3.358826in}{1.282389in}}%
\pgfpathlineto{\pgfqpoint{3.455621in}{1.261706in}}%
\pgfpathlineto{\pgfqpoint{3.552415in}{1.233599in}}%
\pgfpathlineto{\pgfqpoint{3.649210in}{1.222691in}}%
\pgfpathlineto{\pgfqpoint{3.746005in}{1.183154in}}%
\pgfpathlineto{\pgfqpoint{3.842799in}{1.170236in}}%
\pgfpathlineto{\pgfqpoint{3.939594in}{1.142735in}}%
\pgfpathlineto{\pgfqpoint{4.036389in}{1.122774in}}%
\pgfusepath{stroke}%
\end{pgfscope}%
\begin{pgfscope}%
\pgfpathrectangle{\pgfqpoint{0.745371in}{0.566590in}}{\pgfqpoint{3.291018in}{1.828724in}}%
\pgfusepath{clip}%
\pgfsetbuttcap%
\pgfsetroundjoin%
\definecolor{currentfill}{rgb}{0.494000,0.184000,0.556000}%
\pgfsetfillcolor{currentfill}%
\pgfsetlinewidth{1.003750pt}%
\definecolor{currentstroke}{rgb}{0.494000,0.184000,0.556000}%
\pgfsetstrokecolor{currentstroke}%
\pgfsetdash{}{0pt}%
\pgfsys@defobject{currentmarker}{\pgfqpoint{-0.041667in}{-0.041667in}}{\pgfqpoint{0.041667in}{0.041667in}}{%
\pgfpathmoveto{\pgfqpoint{-0.041667in}{-0.041667in}}%
\pgfpathlineto{\pgfqpoint{0.041667in}{0.041667in}}%
\pgfpathmoveto{\pgfqpoint{-0.041667in}{0.041667in}}%
\pgfpathlineto{\pgfqpoint{0.041667in}{-0.041667in}}%
\pgfusepath{stroke,fill}%
}%
\begin{pgfscope}%
\pgfsys@transformshift{0.745371in}{2.072721in}%
\pgfsys@useobject{currentmarker}{}%
\end{pgfscope}%
\begin{pgfscope}%
\pgfsys@transformshift{1.132549in}{2.033821in}%
\pgfsys@useobject{currentmarker}{}%
\end{pgfscope}%
\begin{pgfscope}%
\pgfsys@transformshift{1.519728in}{1.967303in}%
\pgfsys@useobject{currentmarker}{}%
\end{pgfscope}%
\begin{pgfscope}%
\pgfsys@transformshift{1.906906in}{1.921200in}%
\pgfsys@useobject{currentmarker}{}%
\end{pgfscope}%
\begin{pgfscope}%
\pgfsys@transformshift{2.294085in}{1.566828in}%
\pgfsys@useobject{currentmarker}{}%
\end{pgfscope}%
\begin{pgfscope}%
\pgfsys@transformshift{2.681264in}{1.454796in}%
\pgfsys@useobject{currentmarker}{}%
\end{pgfscope}%
\begin{pgfscope}%
\pgfsys@transformshift{3.068442in}{1.346051in}%
\pgfsys@useobject{currentmarker}{}%
\end{pgfscope}%
\begin{pgfscope}%
\pgfsys@transformshift{3.455621in}{1.261706in}%
\pgfsys@useobject{currentmarker}{}%
\end{pgfscope}%
\begin{pgfscope}%
\pgfsys@transformshift{3.842799in}{1.170236in}%
\pgfsys@useobject{currentmarker}{}%
\end{pgfscope}%
\end{pgfscope}%
\begin{pgfscope}%
\pgfpathrectangle{\pgfqpoint{0.745371in}{0.566590in}}{\pgfqpoint{3.291018in}{1.828724in}}%
\pgfusepath{clip}%
\pgfsetbuttcap%
\pgfsetroundjoin%
\pgfsetlinewidth{1.505625pt}%
\definecolor{currentstroke}{rgb}{0.635000,0.078000,0.184000}%
\pgfsetstrokecolor{currentstroke}%
\pgfsetdash{{5.550000pt}{2.400000pt}}{0.000000pt}%
\pgfpathmoveto{\pgfqpoint{0.745371in}{2.152911in}}%
\pgfpathlineto{\pgfqpoint{0.842165in}{2.124002in}}%
\pgfpathlineto{\pgfqpoint{0.938960in}{2.126408in}}%
\pgfpathlineto{\pgfqpoint{1.035755in}{2.107480in}}%
\pgfpathlineto{\pgfqpoint{1.132549in}{2.163312in}}%
\pgfpathlineto{\pgfqpoint{1.229344in}{2.072782in}}%
\pgfpathlineto{\pgfqpoint{1.326139in}{2.113127in}}%
\pgfpathlineto{\pgfqpoint{1.422933in}{2.103039in}}%
\pgfpathlineto{\pgfqpoint{1.519728in}{2.197823in}}%
\pgfpathlineto{\pgfqpoint{1.616523in}{2.210235in}}%
\pgfpathlineto{\pgfqpoint{1.713317in}{2.084063in}}%
\pgfpathlineto{\pgfqpoint{1.810112in}{2.204177in}}%
\pgfpathlineto{\pgfqpoint{1.906906in}{2.216222in}}%
\pgfpathlineto{\pgfqpoint{2.003701in}{2.112543in}}%
\pgfpathlineto{\pgfqpoint{2.100496in}{2.218319in}}%
\pgfpathlineto{\pgfqpoint{2.197290in}{2.123404in}}%
\pgfpathlineto{\pgfqpoint{2.294085in}{1.994185in}}%
\pgfpathlineto{\pgfqpoint{2.390880in}{1.983167in}}%
\pgfpathlineto{\pgfqpoint{2.487674in}{1.969094in}}%
\pgfpathlineto{\pgfqpoint{2.584469in}{1.369952in}}%
\pgfpathlineto{\pgfqpoint{2.681264in}{1.352827in}}%
\pgfpathlineto{\pgfqpoint{2.778058in}{1.315753in}}%
\pgfpathlineto{\pgfqpoint{2.874853in}{1.305704in}}%
\pgfpathlineto{\pgfqpoint{2.971648in}{1.288349in}}%
\pgfpathlineto{\pgfqpoint{3.068442in}{1.249427in}}%
\pgfpathlineto{\pgfqpoint{3.165237in}{1.227099in}}%
\pgfpathlineto{\pgfqpoint{3.262031in}{1.205288in}}%
\pgfpathlineto{\pgfqpoint{3.358826in}{1.181490in}}%
\pgfpathlineto{\pgfqpoint{3.455621in}{1.157539in}}%
\pgfpathlineto{\pgfqpoint{3.552415in}{1.131879in}}%
\pgfpathlineto{\pgfqpoint{3.649210in}{1.123048in}}%
\pgfpathlineto{\pgfqpoint{3.746005in}{1.087495in}}%
\pgfpathlineto{\pgfqpoint{3.842799in}{1.072094in}}%
\pgfpathlineto{\pgfqpoint{3.939594in}{1.043941in}}%
\pgfpathlineto{\pgfqpoint{4.036389in}{1.020014in}}%
\pgfusepath{stroke}%
\end{pgfscope}%
\begin{pgfscope}%
\pgfpathrectangle{\pgfqpoint{0.745371in}{0.566590in}}{\pgfqpoint{3.291018in}{1.828724in}}%
\pgfusepath{clip}%
\pgfsetbuttcap%
\pgfsetmiterjoin%
\definecolor{currentfill}{rgb}{0.000000,0.000000,0.000000}%
\pgfsetfillcolor{currentfill}%
\pgfsetfillopacity{0.000000}%
\pgfsetlinewidth{1.003750pt}%
\definecolor{currentstroke}{rgb}{0.635000,0.078000,0.184000}%
\pgfsetstrokecolor{currentstroke}%
\pgfsetdash{}{0pt}%
\pgfsys@defobject{currentmarker}{\pgfqpoint{-0.035355in}{-0.058926in}}{\pgfqpoint{0.035355in}{0.058926in}}{%
\pgfpathmoveto{\pgfqpoint{-0.000000in}{-0.058926in}}%
\pgfpathlineto{\pgfqpoint{0.035355in}{0.000000in}}%
\pgfpathlineto{\pgfqpoint{0.000000in}{0.058926in}}%
\pgfpathlineto{\pgfqpoint{-0.035355in}{0.000000in}}%
\pgfpathclose%
\pgfusepath{stroke,fill}%
}%
\begin{pgfscope}%
\pgfsys@transformshift{0.745371in}{2.152911in}%
\pgfsys@useobject{currentmarker}{}%
\end{pgfscope}%
\begin{pgfscope}%
\pgfsys@transformshift{1.035755in}{2.107480in}%
\pgfsys@useobject{currentmarker}{}%
\end{pgfscope}%
\begin{pgfscope}%
\pgfsys@transformshift{1.326139in}{2.113127in}%
\pgfsys@useobject{currentmarker}{}%
\end{pgfscope}%
\begin{pgfscope}%
\pgfsys@transformshift{1.616523in}{2.210235in}%
\pgfsys@useobject{currentmarker}{}%
\end{pgfscope}%
\begin{pgfscope}%
\pgfsys@transformshift{1.906906in}{2.216222in}%
\pgfsys@useobject{currentmarker}{}%
\end{pgfscope}%
\begin{pgfscope}%
\pgfsys@transformshift{2.197290in}{2.123404in}%
\pgfsys@useobject{currentmarker}{}%
\end{pgfscope}%
\begin{pgfscope}%
\pgfsys@transformshift{2.487674in}{1.969094in}%
\pgfsys@useobject{currentmarker}{}%
\end{pgfscope}%
\begin{pgfscope}%
\pgfsys@transformshift{2.778058in}{1.315753in}%
\pgfsys@useobject{currentmarker}{}%
\end{pgfscope}%
\begin{pgfscope}%
\pgfsys@transformshift{3.068442in}{1.249427in}%
\pgfsys@useobject{currentmarker}{}%
\end{pgfscope}%
\begin{pgfscope}%
\pgfsys@transformshift{3.358826in}{1.181490in}%
\pgfsys@useobject{currentmarker}{}%
\end{pgfscope}%
\begin{pgfscope}%
\pgfsys@transformshift{3.649210in}{1.123048in}%
\pgfsys@useobject{currentmarker}{}%
\end{pgfscope}%
\begin{pgfscope}%
\pgfsys@transformshift{3.939594in}{1.043941in}%
\pgfsys@useobject{currentmarker}{}%
\end{pgfscope}%
\end{pgfscope}%
\begin{pgfscope}%
\pgfpathrectangle{\pgfqpoint{0.745371in}{0.566590in}}{\pgfqpoint{3.291018in}{1.828724in}}%
\pgfusepath{clip}%
\pgfsetrectcap%
\pgfsetroundjoin%
\pgfsetlinewidth{1.505625pt}%
\definecolor{currentstroke}{rgb}{0.000000,0.447000,0.741000}%
\pgfsetstrokecolor{currentstroke}%
\pgfsetdash{}{0pt}%
\pgfpathmoveto{\pgfqpoint{0.745371in}{1.596224in}}%
\pgfpathlineto{\pgfqpoint{0.980444in}{1.608118in}}%
\pgfpathlineto{\pgfqpoint{1.215516in}{1.601099in}}%
\pgfpathlineto{\pgfqpoint{1.450589in}{1.602571in}}%
\pgfpathlineto{\pgfqpoint{1.685662in}{1.557974in}}%
\pgfpathlineto{\pgfqpoint{1.920734in}{1.498958in}}%
\pgfpathlineto{\pgfqpoint{2.155807in}{1.469642in}}%
\pgfpathlineto{\pgfqpoint{2.390880in}{1.230194in}}%
\pgfpathlineto{\pgfqpoint{2.625952in}{1.176928in}}%
\pgfpathlineto{\pgfqpoint{2.861025in}{1.130770in}}%
\pgfpathlineto{\pgfqpoint{3.096098in}{1.070243in}}%
\pgfpathlineto{\pgfqpoint{3.331170in}{1.003743in}}%
\pgfpathlineto{\pgfqpoint{3.566243in}{0.967012in}}%
\pgfpathlineto{\pgfqpoint{3.801316in}{0.908534in}}%
\pgfpathlineto{\pgfqpoint{4.036389in}{0.851911in}}%
\pgfusepath{stroke}%
\end{pgfscope}%
\begin{pgfscope}%
\pgfpathrectangle{\pgfqpoint{0.745371in}{0.566590in}}{\pgfqpoint{3.291018in}{1.828724in}}%
\pgfusepath{clip}%
\pgfsetbuttcap%
\pgfsetroundjoin%
\definecolor{currentfill}{rgb}{0.000000,0.000000,0.000000}%
\pgfsetfillcolor{currentfill}%
\pgfsetfillopacity{0.000000}%
\pgfsetlinewidth{1.003750pt}%
\definecolor{currentstroke}{rgb}{0.000000,0.447000,0.741000}%
\pgfsetstrokecolor{currentstroke}%
\pgfsetdash{}{0pt}%
\pgfsys@defobject{currentmarker}{\pgfqpoint{-0.041667in}{-0.041667in}}{\pgfqpoint{0.041667in}{0.041667in}}{%
\pgfpathmoveto{\pgfqpoint{0.000000in}{-0.041667in}}%
\pgfpathcurveto{\pgfqpoint{0.011050in}{-0.041667in}}{\pgfqpoint{0.021649in}{-0.037276in}}{\pgfqpoint{0.029463in}{-0.029463in}}%
\pgfpathcurveto{\pgfqpoint{0.037276in}{-0.021649in}}{\pgfqpoint{0.041667in}{-0.011050in}}{\pgfqpoint{0.041667in}{0.000000in}}%
\pgfpathcurveto{\pgfqpoint{0.041667in}{0.011050in}}{\pgfqpoint{0.037276in}{0.021649in}}{\pgfqpoint{0.029463in}{0.029463in}}%
\pgfpathcurveto{\pgfqpoint{0.021649in}{0.037276in}}{\pgfqpoint{0.011050in}{0.041667in}}{\pgfqpoint{0.000000in}{0.041667in}}%
\pgfpathcurveto{\pgfqpoint{-0.011050in}{0.041667in}}{\pgfqpoint{-0.021649in}{0.037276in}}{\pgfqpoint{-0.029463in}{0.029463in}}%
\pgfpathcurveto{\pgfqpoint{-0.037276in}{0.021649in}}{\pgfqpoint{-0.041667in}{0.011050in}}{\pgfqpoint{-0.041667in}{0.000000in}}%
\pgfpathcurveto{\pgfqpoint{-0.041667in}{-0.011050in}}{\pgfqpoint{-0.037276in}{-0.021649in}}{\pgfqpoint{-0.029463in}{-0.029463in}}%
\pgfpathcurveto{\pgfqpoint{-0.021649in}{-0.037276in}}{\pgfqpoint{-0.011050in}{-0.041667in}}{\pgfqpoint{0.000000in}{-0.041667in}}%
\pgfpathclose%
\pgfusepath{stroke,fill}%
}%
\begin{pgfscope}%
\pgfsys@transformshift{0.745371in}{1.596224in}%
\pgfsys@useobject{currentmarker}{}%
\end{pgfscope}%
\begin{pgfscope}%
\pgfsys@transformshift{0.980444in}{1.608118in}%
\pgfsys@useobject{currentmarker}{}%
\end{pgfscope}%
\begin{pgfscope}%
\pgfsys@transformshift{1.215516in}{1.601099in}%
\pgfsys@useobject{currentmarker}{}%
\end{pgfscope}%
\begin{pgfscope}%
\pgfsys@transformshift{1.450589in}{1.602571in}%
\pgfsys@useobject{currentmarker}{}%
\end{pgfscope}%
\begin{pgfscope}%
\pgfsys@transformshift{1.685662in}{1.557974in}%
\pgfsys@useobject{currentmarker}{}%
\end{pgfscope}%
\begin{pgfscope}%
\pgfsys@transformshift{1.920734in}{1.498958in}%
\pgfsys@useobject{currentmarker}{}%
\end{pgfscope}%
\begin{pgfscope}%
\pgfsys@transformshift{2.155807in}{1.469642in}%
\pgfsys@useobject{currentmarker}{}%
\end{pgfscope}%
\begin{pgfscope}%
\pgfsys@transformshift{2.390880in}{1.230194in}%
\pgfsys@useobject{currentmarker}{}%
\end{pgfscope}%
\begin{pgfscope}%
\pgfsys@transformshift{2.625952in}{1.176928in}%
\pgfsys@useobject{currentmarker}{}%
\end{pgfscope}%
\begin{pgfscope}%
\pgfsys@transformshift{2.861025in}{1.130770in}%
\pgfsys@useobject{currentmarker}{}%
\end{pgfscope}%
\begin{pgfscope}%
\pgfsys@transformshift{3.096098in}{1.070243in}%
\pgfsys@useobject{currentmarker}{}%
\end{pgfscope}%
\begin{pgfscope}%
\pgfsys@transformshift{3.331170in}{1.003743in}%
\pgfsys@useobject{currentmarker}{}%
\end{pgfscope}%
\begin{pgfscope}%
\pgfsys@transformshift{3.566243in}{0.967012in}%
\pgfsys@useobject{currentmarker}{}%
\end{pgfscope}%
\begin{pgfscope}%
\pgfsys@transformshift{3.801316in}{0.908534in}%
\pgfsys@useobject{currentmarker}{}%
\end{pgfscope}%
\begin{pgfscope}%
\pgfsys@transformshift{4.036389in}{0.851911in}%
\pgfsys@useobject{currentmarker}{}%
\end{pgfscope}%
\end{pgfscope}%
\begin{pgfscope}%
\pgfpathrectangle{\pgfqpoint{0.745371in}{0.566590in}}{\pgfqpoint{3.291018in}{1.828724in}}%
\pgfusepath{clip}%
\pgfsetrectcap%
\pgfsetroundjoin%
\pgfsetlinewidth{1.505625pt}%
\definecolor{currentstroke}{rgb}{0.850000,0.324000,0.098000}%
\pgfsetstrokecolor{currentstroke}%
\pgfsetdash{}{0pt}%
\pgfpathmoveto{\pgfqpoint{0.745371in}{1.609699in}}%
\pgfpathlineto{\pgfqpoint{0.980444in}{1.592832in}}%
\pgfpathlineto{\pgfqpoint{1.215516in}{1.569673in}}%
\pgfpathlineto{\pgfqpoint{1.450589in}{1.540094in}}%
\pgfpathlineto{\pgfqpoint{1.685662in}{1.568335in}}%
\pgfpathlineto{\pgfqpoint{1.920734in}{1.486608in}}%
\pgfpathlineto{\pgfqpoint{2.155807in}{1.554386in}}%
\pgfpathlineto{\pgfqpoint{2.390880in}{1.265335in}}%
\pgfpathlineto{\pgfqpoint{2.625952in}{1.214579in}}%
\pgfpathlineto{\pgfqpoint{2.861025in}{1.165757in}}%
\pgfpathlineto{\pgfqpoint{3.096098in}{1.114432in}}%
\pgfpathlineto{\pgfqpoint{3.331170in}{1.041999in}}%
\pgfpathlineto{\pgfqpoint{3.566243in}{0.998560in}}%
\pgfpathlineto{\pgfqpoint{3.801316in}{0.944146in}}%
\pgfpathlineto{\pgfqpoint{4.036389in}{0.879306in}}%
\pgfusepath{stroke}%
\end{pgfscope}%
\begin{pgfscope}%
\pgfpathrectangle{\pgfqpoint{0.745371in}{0.566590in}}{\pgfqpoint{3.291018in}{1.828724in}}%
\pgfusepath{clip}%
\pgfsetbuttcap%
\pgfsetroundjoin%
\definecolor{currentfill}{rgb}{0.850000,0.324000,0.098000}%
\pgfsetfillcolor{currentfill}%
\pgfsetlinewidth{1.003750pt}%
\definecolor{currentstroke}{rgb}{0.850000,0.324000,0.098000}%
\pgfsetstrokecolor{currentstroke}%
\pgfsetdash{}{0pt}%
\pgfsys@defobject{currentmarker}{\pgfqpoint{-0.041667in}{-0.041667in}}{\pgfqpoint{0.041667in}{0.041667in}}{%
\pgfpathmoveto{\pgfqpoint{-0.041667in}{0.000000in}}%
\pgfpathlineto{\pgfqpoint{0.041667in}{0.000000in}}%
\pgfpathmoveto{\pgfqpoint{0.000000in}{-0.041667in}}%
\pgfpathlineto{\pgfqpoint{0.000000in}{0.041667in}}%
\pgfusepath{stroke,fill}%
}%
\begin{pgfscope}%
\pgfsys@transformshift{0.745371in}{1.609699in}%
\pgfsys@useobject{currentmarker}{}%
\end{pgfscope}%
\begin{pgfscope}%
\pgfsys@transformshift{0.980444in}{1.592832in}%
\pgfsys@useobject{currentmarker}{}%
\end{pgfscope}%
\begin{pgfscope}%
\pgfsys@transformshift{1.215516in}{1.569673in}%
\pgfsys@useobject{currentmarker}{}%
\end{pgfscope}%
\begin{pgfscope}%
\pgfsys@transformshift{1.450589in}{1.540094in}%
\pgfsys@useobject{currentmarker}{}%
\end{pgfscope}%
\begin{pgfscope}%
\pgfsys@transformshift{1.685662in}{1.568335in}%
\pgfsys@useobject{currentmarker}{}%
\end{pgfscope}%
\begin{pgfscope}%
\pgfsys@transformshift{1.920734in}{1.486608in}%
\pgfsys@useobject{currentmarker}{}%
\end{pgfscope}%
\begin{pgfscope}%
\pgfsys@transformshift{2.155807in}{1.554386in}%
\pgfsys@useobject{currentmarker}{}%
\end{pgfscope}%
\begin{pgfscope}%
\pgfsys@transformshift{2.390880in}{1.265335in}%
\pgfsys@useobject{currentmarker}{}%
\end{pgfscope}%
\begin{pgfscope}%
\pgfsys@transformshift{2.625952in}{1.214579in}%
\pgfsys@useobject{currentmarker}{}%
\end{pgfscope}%
\begin{pgfscope}%
\pgfsys@transformshift{2.861025in}{1.165757in}%
\pgfsys@useobject{currentmarker}{}%
\end{pgfscope}%
\begin{pgfscope}%
\pgfsys@transformshift{3.096098in}{1.114432in}%
\pgfsys@useobject{currentmarker}{}%
\end{pgfscope}%
\begin{pgfscope}%
\pgfsys@transformshift{3.331170in}{1.041999in}%
\pgfsys@useobject{currentmarker}{}%
\end{pgfscope}%
\begin{pgfscope}%
\pgfsys@transformshift{3.566243in}{0.998560in}%
\pgfsys@useobject{currentmarker}{}%
\end{pgfscope}%
\begin{pgfscope}%
\pgfsys@transformshift{3.801316in}{0.944146in}%
\pgfsys@useobject{currentmarker}{}%
\end{pgfscope}%
\begin{pgfscope}%
\pgfsys@transformshift{4.036389in}{0.879306in}%
\pgfsys@useobject{currentmarker}{}%
\end{pgfscope}%
\end{pgfscope}%
\begin{pgfscope}%
\pgfpathrectangle{\pgfqpoint{0.745371in}{0.566590in}}{\pgfqpoint{3.291018in}{1.828724in}}%
\pgfusepath{clip}%
\pgfsetrectcap%
\pgfsetroundjoin%
\pgfsetlinewidth{1.505625pt}%
\definecolor{currentstroke}{rgb}{0.000000,0.500000,0.000000}%
\pgfsetstrokecolor{currentstroke}%
\pgfsetdash{}{0pt}%
\pgfpathmoveto{\pgfqpoint{0.745371in}{1.569772in}}%
\pgfpathlineto{\pgfqpoint{0.980444in}{1.595179in}}%
\pgfpathlineto{\pgfqpoint{1.215516in}{1.593324in}}%
\pgfpathlineto{\pgfqpoint{1.450589in}{1.557054in}}%
\pgfpathlineto{\pgfqpoint{1.685662in}{1.580027in}}%
\pgfpathlineto{\pgfqpoint{1.920734in}{1.508134in}}%
\pgfpathlineto{\pgfqpoint{2.155807in}{1.437534in}}%
\pgfpathlineto{\pgfqpoint{2.390880in}{1.240043in}}%
\pgfpathlineto{\pgfqpoint{2.625952in}{1.182968in}}%
\pgfpathlineto{\pgfqpoint{2.861025in}{1.125801in}}%
\pgfpathlineto{\pgfqpoint{3.096098in}{1.073882in}}%
\pgfpathlineto{\pgfqpoint{3.331170in}{1.016654in}}%
\pgfpathlineto{\pgfqpoint{3.566243in}{0.968146in}}%
\pgfpathlineto{\pgfqpoint{3.801316in}{0.907483in}}%
\pgfpathlineto{\pgfqpoint{4.036389in}{0.850601in}}%
\pgfusepath{stroke}%
\end{pgfscope}%
\begin{pgfscope}%
\pgfpathrectangle{\pgfqpoint{0.745371in}{0.566590in}}{\pgfqpoint{3.291018in}{1.828724in}}%
\pgfusepath{clip}%
\pgfsetbuttcap%
\pgfsetmiterjoin%
\definecolor{currentfill}{rgb}{0.000000,0.000000,0.000000}%
\pgfsetfillcolor{currentfill}%
\pgfsetfillopacity{0.000000}%
\pgfsetlinewidth{1.003750pt}%
\definecolor{currentstroke}{rgb}{0.000000,0.500000,0.000000}%
\pgfsetstrokecolor{currentstroke}%
\pgfsetdash{}{0pt}%
\pgfsys@defobject{currentmarker}{\pgfqpoint{-0.041667in}{-0.041667in}}{\pgfqpoint{0.041667in}{0.041667in}}{%
\pgfpathmoveto{\pgfqpoint{-0.041667in}{-0.041667in}}%
\pgfpathlineto{\pgfqpoint{0.041667in}{-0.041667in}}%
\pgfpathlineto{\pgfqpoint{0.041667in}{0.041667in}}%
\pgfpathlineto{\pgfqpoint{-0.041667in}{0.041667in}}%
\pgfpathclose%
\pgfusepath{stroke,fill}%
}%
\begin{pgfscope}%
\pgfsys@transformshift{0.745371in}{1.569772in}%
\pgfsys@useobject{currentmarker}{}%
\end{pgfscope}%
\begin{pgfscope}%
\pgfsys@transformshift{0.980444in}{1.595179in}%
\pgfsys@useobject{currentmarker}{}%
\end{pgfscope}%
\begin{pgfscope}%
\pgfsys@transformshift{1.215516in}{1.593324in}%
\pgfsys@useobject{currentmarker}{}%
\end{pgfscope}%
\begin{pgfscope}%
\pgfsys@transformshift{1.450589in}{1.557054in}%
\pgfsys@useobject{currentmarker}{}%
\end{pgfscope}%
\begin{pgfscope}%
\pgfsys@transformshift{1.685662in}{1.580027in}%
\pgfsys@useobject{currentmarker}{}%
\end{pgfscope}%
\begin{pgfscope}%
\pgfsys@transformshift{1.920734in}{1.508134in}%
\pgfsys@useobject{currentmarker}{}%
\end{pgfscope}%
\begin{pgfscope}%
\pgfsys@transformshift{2.155807in}{1.437534in}%
\pgfsys@useobject{currentmarker}{}%
\end{pgfscope}%
\begin{pgfscope}%
\pgfsys@transformshift{2.390880in}{1.240043in}%
\pgfsys@useobject{currentmarker}{}%
\end{pgfscope}%
\begin{pgfscope}%
\pgfsys@transformshift{2.625952in}{1.182968in}%
\pgfsys@useobject{currentmarker}{}%
\end{pgfscope}%
\begin{pgfscope}%
\pgfsys@transformshift{2.861025in}{1.125801in}%
\pgfsys@useobject{currentmarker}{}%
\end{pgfscope}%
\begin{pgfscope}%
\pgfsys@transformshift{3.096098in}{1.073882in}%
\pgfsys@useobject{currentmarker}{}%
\end{pgfscope}%
\begin{pgfscope}%
\pgfsys@transformshift{3.331170in}{1.016654in}%
\pgfsys@useobject{currentmarker}{}%
\end{pgfscope}%
\begin{pgfscope}%
\pgfsys@transformshift{3.566243in}{0.968146in}%
\pgfsys@useobject{currentmarker}{}%
\end{pgfscope}%
\begin{pgfscope}%
\pgfsys@transformshift{3.801316in}{0.907483in}%
\pgfsys@useobject{currentmarker}{}%
\end{pgfscope}%
\begin{pgfscope}%
\pgfsys@transformshift{4.036389in}{0.850601in}%
\pgfsys@useobject{currentmarker}{}%
\end{pgfscope}%
\end{pgfscope}%
\begin{pgfscope}%
\pgfpathrectangle{\pgfqpoint{0.745371in}{0.566590in}}{\pgfqpoint{3.291018in}{1.828724in}}%
\pgfusepath{clip}%
\pgfsetrectcap%
\pgfsetroundjoin%
\pgfsetlinewidth{1.505625pt}%
\definecolor{currentstroke}{rgb}{0.494000,0.184000,0.556000}%
\pgfsetstrokecolor{currentstroke}%
\pgfsetdash{}{0pt}%
\pgfpathmoveto{\pgfqpoint{0.745371in}{1.626516in}}%
\pgfpathlineto{\pgfqpoint{0.980444in}{1.632945in}}%
\pgfpathlineto{\pgfqpoint{1.215516in}{1.605463in}}%
\pgfpathlineto{\pgfqpoint{1.450589in}{1.574322in}}%
\pgfpathlineto{\pgfqpoint{1.685662in}{1.540583in}}%
\pgfpathlineto{\pgfqpoint{1.920734in}{1.518333in}}%
\pgfpathlineto{\pgfqpoint{2.155807in}{1.550465in}}%
\pgfpathlineto{\pgfqpoint{2.390880in}{1.255082in}}%
\pgfpathlineto{\pgfqpoint{2.625952in}{1.187591in}}%
\pgfpathlineto{\pgfqpoint{2.861025in}{1.136453in}}%
\pgfpathlineto{\pgfqpoint{3.096098in}{1.086674in}}%
\pgfpathlineto{\pgfqpoint{3.331170in}{1.030707in}}%
\pgfpathlineto{\pgfqpoint{3.566243in}{0.965938in}}%
\pgfpathlineto{\pgfqpoint{3.801316in}{0.910875in}}%
\pgfpathlineto{\pgfqpoint{4.036389in}{0.857415in}}%
\pgfusepath{stroke}%
\end{pgfscope}%
\begin{pgfscope}%
\pgfpathrectangle{\pgfqpoint{0.745371in}{0.566590in}}{\pgfqpoint{3.291018in}{1.828724in}}%
\pgfusepath{clip}%
\pgfsetbuttcap%
\pgfsetroundjoin%
\definecolor{currentfill}{rgb}{0.494000,0.184000,0.556000}%
\pgfsetfillcolor{currentfill}%
\pgfsetlinewidth{1.003750pt}%
\definecolor{currentstroke}{rgb}{0.494000,0.184000,0.556000}%
\pgfsetstrokecolor{currentstroke}%
\pgfsetdash{}{0pt}%
\pgfsys@defobject{currentmarker}{\pgfqpoint{-0.041667in}{-0.041667in}}{\pgfqpoint{0.041667in}{0.041667in}}{%
\pgfpathmoveto{\pgfqpoint{-0.041667in}{-0.041667in}}%
\pgfpathlineto{\pgfqpoint{0.041667in}{0.041667in}}%
\pgfpathmoveto{\pgfqpoint{-0.041667in}{0.041667in}}%
\pgfpathlineto{\pgfqpoint{0.041667in}{-0.041667in}}%
\pgfusepath{stroke,fill}%
}%
\begin{pgfscope}%
\pgfsys@transformshift{0.745371in}{1.626516in}%
\pgfsys@useobject{currentmarker}{}%
\end{pgfscope}%
\begin{pgfscope}%
\pgfsys@transformshift{0.980444in}{1.632945in}%
\pgfsys@useobject{currentmarker}{}%
\end{pgfscope}%
\begin{pgfscope}%
\pgfsys@transformshift{1.215516in}{1.605463in}%
\pgfsys@useobject{currentmarker}{}%
\end{pgfscope}%
\begin{pgfscope}%
\pgfsys@transformshift{1.450589in}{1.574322in}%
\pgfsys@useobject{currentmarker}{}%
\end{pgfscope}%
\begin{pgfscope}%
\pgfsys@transformshift{1.685662in}{1.540583in}%
\pgfsys@useobject{currentmarker}{}%
\end{pgfscope}%
\begin{pgfscope}%
\pgfsys@transformshift{1.920734in}{1.518333in}%
\pgfsys@useobject{currentmarker}{}%
\end{pgfscope}%
\begin{pgfscope}%
\pgfsys@transformshift{2.155807in}{1.550465in}%
\pgfsys@useobject{currentmarker}{}%
\end{pgfscope}%
\begin{pgfscope}%
\pgfsys@transformshift{2.390880in}{1.255082in}%
\pgfsys@useobject{currentmarker}{}%
\end{pgfscope}%
\begin{pgfscope}%
\pgfsys@transformshift{2.625952in}{1.187591in}%
\pgfsys@useobject{currentmarker}{}%
\end{pgfscope}%
\begin{pgfscope}%
\pgfsys@transformshift{2.861025in}{1.136453in}%
\pgfsys@useobject{currentmarker}{}%
\end{pgfscope}%
\begin{pgfscope}%
\pgfsys@transformshift{3.096098in}{1.086674in}%
\pgfsys@useobject{currentmarker}{}%
\end{pgfscope}%
\begin{pgfscope}%
\pgfsys@transformshift{3.331170in}{1.030707in}%
\pgfsys@useobject{currentmarker}{}%
\end{pgfscope}%
\begin{pgfscope}%
\pgfsys@transformshift{3.566243in}{0.965938in}%
\pgfsys@useobject{currentmarker}{}%
\end{pgfscope}%
\begin{pgfscope}%
\pgfsys@transformshift{3.801316in}{0.910875in}%
\pgfsys@useobject{currentmarker}{}%
\end{pgfscope}%
\begin{pgfscope}%
\pgfsys@transformshift{4.036389in}{0.857415in}%
\pgfsys@useobject{currentmarker}{}%
\end{pgfscope}%
\end{pgfscope}%
\begin{pgfscope}%
\pgfpathrectangle{\pgfqpoint{0.745371in}{0.566590in}}{\pgfqpoint{3.291018in}{1.828724in}}%
\pgfusepath{clip}%
\pgfsetrectcap%
\pgfsetroundjoin%
\pgfsetlinewidth{1.505625pt}%
\definecolor{currentstroke}{rgb}{0.635000,0.078000,0.184000}%
\pgfsetstrokecolor{currentstroke}%
\pgfsetdash{}{0pt}%
\pgfpathmoveto{\pgfqpoint{0.745371in}{1.770097in}}%
\pgfpathlineto{\pgfqpoint{0.980444in}{1.766521in}}%
\pgfpathlineto{\pgfqpoint{1.215516in}{1.757720in}}%
\pgfpathlineto{\pgfqpoint{1.450589in}{1.736080in}}%
\pgfpathlineto{\pgfqpoint{1.685662in}{1.710169in}}%
\pgfpathlineto{\pgfqpoint{1.920734in}{1.688379in}}%
\pgfpathlineto{\pgfqpoint{2.155807in}{1.627487in}}%
\pgfpathlineto{\pgfqpoint{2.390880in}{1.447096in}}%
\pgfpathlineto{\pgfqpoint{2.625952in}{1.388399in}}%
\pgfpathlineto{\pgfqpoint{2.861025in}{1.326831in}}%
\pgfpathlineto{\pgfqpoint{3.096098in}{1.271369in}}%
\pgfpathlineto{\pgfqpoint{3.331170in}{1.221643in}}%
\pgfpathlineto{\pgfqpoint{3.566243in}{1.162464in}}%
\pgfpathlineto{\pgfqpoint{3.801316in}{1.109592in}}%
\pgfpathlineto{\pgfqpoint{4.036389in}{1.058344in}}%
\pgfusepath{stroke}%
\end{pgfscope}%
\begin{pgfscope}%
\pgfpathrectangle{\pgfqpoint{0.745371in}{0.566590in}}{\pgfqpoint{3.291018in}{1.828724in}}%
\pgfusepath{clip}%
\pgfsetbuttcap%
\pgfsetmiterjoin%
\definecolor{currentfill}{rgb}{0.000000,0.000000,0.000000}%
\pgfsetfillcolor{currentfill}%
\pgfsetfillopacity{0.000000}%
\pgfsetlinewidth{1.003750pt}%
\definecolor{currentstroke}{rgb}{0.635000,0.078000,0.184000}%
\pgfsetstrokecolor{currentstroke}%
\pgfsetdash{}{0pt}%
\pgfsys@defobject{currentmarker}{\pgfqpoint{-0.035355in}{-0.058926in}}{\pgfqpoint{0.035355in}{0.058926in}}{%
\pgfpathmoveto{\pgfqpoint{-0.000000in}{-0.058926in}}%
\pgfpathlineto{\pgfqpoint{0.035355in}{0.000000in}}%
\pgfpathlineto{\pgfqpoint{0.000000in}{0.058926in}}%
\pgfpathlineto{\pgfqpoint{-0.035355in}{0.000000in}}%
\pgfpathclose%
\pgfusepath{stroke,fill}%
}%
\begin{pgfscope}%
\pgfsys@transformshift{0.745371in}{1.770097in}%
\pgfsys@useobject{currentmarker}{}%
\end{pgfscope}%
\begin{pgfscope}%
\pgfsys@transformshift{0.980444in}{1.766521in}%
\pgfsys@useobject{currentmarker}{}%
\end{pgfscope}%
\begin{pgfscope}%
\pgfsys@transformshift{1.215516in}{1.757720in}%
\pgfsys@useobject{currentmarker}{}%
\end{pgfscope}%
\begin{pgfscope}%
\pgfsys@transformshift{1.450589in}{1.736080in}%
\pgfsys@useobject{currentmarker}{}%
\end{pgfscope}%
\begin{pgfscope}%
\pgfsys@transformshift{1.685662in}{1.710169in}%
\pgfsys@useobject{currentmarker}{}%
\end{pgfscope}%
\begin{pgfscope}%
\pgfsys@transformshift{1.920734in}{1.688379in}%
\pgfsys@useobject{currentmarker}{}%
\end{pgfscope}%
\begin{pgfscope}%
\pgfsys@transformshift{2.155807in}{1.627487in}%
\pgfsys@useobject{currentmarker}{}%
\end{pgfscope}%
\begin{pgfscope}%
\pgfsys@transformshift{2.390880in}{1.447096in}%
\pgfsys@useobject{currentmarker}{}%
\end{pgfscope}%
\begin{pgfscope}%
\pgfsys@transformshift{2.625952in}{1.388399in}%
\pgfsys@useobject{currentmarker}{}%
\end{pgfscope}%
\begin{pgfscope}%
\pgfsys@transformshift{2.861025in}{1.326831in}%
\pgfsys@useobject{currentmarker}{}%
\end{pgfscope}%
\begin{pgfscope}%
\pgfsys@transformshift{3.096098in}{1.271369in}%
\pgfsys@useobject{currentmarker}{}%
\end{pgfscope}%
\begin{pgfscope}%
\pgfsys@transformshift{3.331170in}{1.221643in}%
\pgfsys@useobject{currentmarker}{}%
\end{pgfscope}%
\begin{pgfscope}%
\pgfsys@transformshift{3.566243in}{1.162464in}%
\pgfsys@useobject{currentmarker}{}%
\end{pgfscope}%
\begin{pgfscope}%
\pgfsys@transformshift{3.801316in}{1.109592in}%
\pgfsys@useobject{currentmarker}{}%
\end{pgfscope}%
\begin{pgfscope}%
\pgfsys@transformshift{4.036389in}{1.058344in}%
\pgfsys@useobject{currentmarker}{}%
\end{pgfscope}%
\end{pgfscope}%
\begin{pgfscope}%
\pgfsetrectcap%
\pgfsetmiterjoin%
\pgfsetlinewidth{0.803000pt}%
\definecolor{currentstroke}{rgb}{0.000000,0.000000,0.000000}%
\pgfsetstrokecolor{currentstroke}%
\pgfsetdash{}{0pt}%
\pgfpathmoveto{\pgfqpoint{0.745371in}{0.566590in}}%
\pgfpathlineto{\pgfqpoint{0.745371in}{2.395314in}}%
\pgfusepath{stroke}%
\end{pgfscope}%
\begin{pgfscope}%
\pgfsetrectcap%
\pgfsetmiterjoin%
\pgfsetlinewidth{0.803000pt}%
\definecolor{currentstroke}{rgb}{0.000000,0.000000,0.000000}%
\pgfsetstrokecolor{currentstroke}%
\pgfsetdash{}{0pt}%
\pgfpathmoveto{\pgfqpoint{4.036389in}{0.566590in}}%
\pgfpathlineto{\pgfqpoint{4.036389in}{2.395314in}}%
\pgfusepath{stroke}%
\end{pgfscope}%
\begin{pgfscope}%
\pgfsetrectcap%
\pgfsetmiterjoin%
\pgfsetlinewidth{0.803000pt}%
\definecolor{currentstroke}{rgb}{0.000000,0.000000,0.000000}%
\pgfsetstrokecolor{currentstroke}%
\pgfsetdash{}{0pt}%
\pgfpathmoveto{\pgfqpoint{0.745371in}{0.566590in}}%
\pgfpathlineto{\pgfqpoint{4.036389in}{0.566590in}}%
\pgfusepath{stroke}%
\end{pgfscope}%
\begin{pgfscope}%
\pgfsetrectcap%
\pgfsetmiterjoin%
\pgfsetlinewidth{0.803000pt}%
\definecolor{currentstroke}{rgb}{0.000000,0.000000,0.000000}%
\pgfsetstrokecolor{currentstroke}%
\pgfsetdash{}{0pt}%
\pgfpathmoveto{\pgfqpoint{0.745371in}{2.395314in}}%
\pgfpathlineto{\pgfqpoint{4.036389in}{2.395314in}}%
\pgfusepath{stroke}%
\end{pgfscope}%
\begin{pgfscope}%
\pgfsetbuttcap%
\pgfsetmiterjoin%
\definecolor{currentfill}{rgb}{1.000000,1.000000,1.000000}%
\pgfsetfillcolor{currentfill}%
\pgfsetfillopacity{0.800000}%
\pgfsetlinewidth{1.003750pt}%
\definecolor{currentstroke}{rgb}{0.800000,0.800000,0.800000}%
\pgfsetstrokecolor{currentstroke}%
\pgfsetstrokeopacity{0.800000}%
\pgfsetdash{}{0pt}%
\pgfpathmoveto{\pgfqpoint{0.832871in}{0.629090in}}%
\pgfpathlineto{\pgfqpoint{1.857596in}{0.629090in}}%
\pgfpathquadraticcurveto{\pgfqpoint{1.882596in}{0.629090in}}{\pgfqpoint{1.882596in}{0.654090in}}%
\pgfpathlineto{\pgfqpoint{1.882596in}{1.513118in}}%
\pgfpathquadraticcurveto{\pgfqpoint{1.882596in}{1.538118in}}{\pgfqpoint{1.857596in}{1.538118in}}%
\pgfpathlineto{\pgfqpoint{0.832871in}{1.538118in}}%
\pgfpathquadraticcurveto{\pgfqpoint{0.807871in}{1.538118in}}{\pgfqpoint{0.807871in}{1.513118in}}%
\pgfpathlineto{\pgfqpoint{0.807871in}{0.654090in}}%
\pgfpathquadraticcurveto{\pgfqpoint{0.807871in}{0.629090in}}{\pgfqpoint{0.832871in}{0.629090in}}%
\pgfpathclose%
\pgfusepath{stroke,fill}%
\end{pgfscope}%
\begin{pgfscope}%
\pgfsetbuttcap%
\pgfsetroundjoin%
\definecolor{currentfill}{rgb}{0.000000,0.000000,0.000000}%
\pgfsetfillcolor{currentfill}%
\pgfsetfillopacity{0.000000}%
\pgfsetlinewidth{1.003750pt}%
\definecolor{currentstroke}{rgb}{0.000000,0.447000,0.741000}%
\pgfsetstrokecolor{currentstroke}%
\pgfsetdash{}{0pt}%
\pgfsys@defobject{currentmarker}{\pgfqpoint{-0.041667in}{-0.041667in}}{\pgfqpoint{0.041667in}{0.041667in}}{%
\pgfpathmoveto{\pgfqpoint{0.000000in}{-0.041667in}}%
\pgfpathcurveto{\pgfqpoint{0.011050in}{-0.041667in}}{\pgfqpoint{0.021649in}{-0.037276in}}{\pgfqpoint{0.029463in}{-0.029463in}}%
\pgfpathcurveto{\pgfqpoint{0.037276in}{-0.021649in}}{\pgfqpoint{0.041667in}{-0.011050in}}{\pgfqpoint{0.041667in}{0.000000in}}%
\pgfpathcurveto{\pgfqpoint{0.041667in}{0.011050in}}{\pgfqpoint{0.037276in}{0.021649in}}{\pgfqpoint{0.029463in}{0.029463in}}%
\pgfpathcurveto{\pgfqpoint{0.021649in}{0.037276in}}{\pgfqpoint{0.011050in}{0.041667in}}{\pgfqpoint{0.000000in}{0.041667in}}%
\pgfpathcurveto{\pgfqpoint{-0.011050in}{0.041667in}}{\pgfqpoint{-0.021649in}{0.037276in}}{\pgfqpoint{-0.029463in}{0.029463in}}%
\pgfpathcurveto{\pgfqpoint{-0.037276in}{0.021649in}}{\pgfqpoint{-0.041667in}{0.011050in}}{\pgfqpoint{-0.041667in}{0.000000in}}%
\pgfpathcurveto{\pgfqpoint{-0.041667in}{-0.011050in}}{\pgfqpoint{-0.037276in}{-0.021649in}}{\pgfqpoint{-0.029463in}{-0.029463in}}%
\pgfpathcurveto{\pgfqpoint{-0.021649in}{-0.037276in}}{\pgfqpoint{-0.011050in}{-0.041667in}}{\pgfqpoint{0.000000in}{-0.041667in}}%
\pgfpathclose%
\pgfusepath{stroke,fill}%
}%
\begin{pgfscope}%
\pgfsys@transformshift{0.982871in}{1.444368in}%
\pgfsys@useobject{currentmarker}{}%
\end{pgfscope}%
\end{pgfscope}%
\begin{pgfscope}%
\definecolor{textcolor}{rgb}{0.000000,0.000000,0.000000}%
\pgfsetstrokecolor{textcolor}%
\pgfsetfillcolor{textcolor}%
\pgftext[x=1.207871in,y=1.400618in,left,base]{\color{textcolor}\rmfamily\fontsize{9.000000}{10.800000}\selectfont \(\displaystyle \gamma_{21} = \) -0.11}%
\end{pgfscope}%
\begin{pgfscope}%
\pgfsetbuttcap%
\pgfsetroundjoin%
\definecolor{currentfill}{rgb}{0.850000,0.324000,0.098000}%
\pgfsetfillcolor{currentfill}%
\pgfsetlinewidth{1.003750pt}%
\definecolor{currentstroke}{rgb}{0.850000,0.324000,0.098000}%
\pgfsetstrokecolor{currentstroke}%
\pgfsetdash{}{0pt}%
\pgfsys@defobject{currentmarker}{\pgfqpoint{-0.041667in}{-0.041667in}}{\pgfqpoint{0.041667in}{0.041667in}}{%
\pgfpathmoveto{\pgfqpoint{-0.041667in}{0.000000in}}%
\pgfpathlineto{\pgfqpoint{0.041667in}{0.000000in}}%
\pgfpathmoveto{\pgfqpoint{0.000000in}{-0.041667in}}%
\pgfpathlineto{\pgfqpoint{0.000000in}{0.041667in}}%
\pgfusepath{stroke,fill}%
}%
\begin{pgfscope}%
\pgfsys@transformshift{0.982871in}{1.270062in}%
\pgfsys@useobject{currentmarker}{}%
\end{pgfscope}%
\end{pgfscope}%
\begin{pgfscope}%
\definecolor{textcolor}{rgb}{0.000000,0.000000,0.000000}%
\pgfsetstrokecolor{textcolor}%
\pgfsetfillcolor{textcolor}%
\pgftext[x=1.207871in,y=1.226312in,left,base]{\color{textcolor}\rmfamily\fontsize{9.000000}{10.800000}\selectfont \(\displaystyle \gamma_{22} = \) -0.12}%
\end{pgfscope}%
\begin{pgfscope}%
\pgfsetbuttcap%
\pgfsetmiterjoin%
\definecolor{currentfill}{rgb}{0.000000,0.000000,0.000000}%
\pgfsetfillcolor{currentfill}%
\pgfsetfillopacity{0.000000}%
\pgfsetlinewidth{1.003750pt}%
\definecolor{currentstroke}{rgb}{0.000000,0.500000,0.000000}%
\pgfsetstrokecolor{currentstroke}%
\pgfsetdash{}{0pt}%
\pgfsys@defobject{currentmarker}{\pgfqpoint{-0.041667in}{-0.041667in}}{\pgfqpoint{0.041667in}{0.041667in}}{%
\pgfpathmoveto{\pgfqpoint{-0.041667in}{-0.041667in}}%
\pgfpathlineto{\pgfqpoint{0.041667in}{-0.041667in}}%
\pgfpathlineto{\pgfqpoint{0.041667in}{0.041667in}}%
\pgfpathlineto{\pgfqpoint{-0.041667in}{0.041667in}}%
\pgfpathclose%
\pgfusepath{stroke,fill}%
}%
\begin{pgfscope}%
\pgfsys@transformshift{0.982871in}{1.095757in}%
\pgfsys@useobject{currentmarker}{}%
\end{pgfscope}%
\end{pgfscope}%
\begin{pgfscope}%
\definecolor{textcolor}{rgb}{0.000000,0.000000,0.000000}%
\pgfsetstrokecolor{textcolor}%
\pgfsetfillcolor{textcolor}%
\pgftext[x=1.207871in,y=1.052007in,left,base]{\color{textcolor}\rmfamily\fontsize{9.000000}{10.800000}\selectfont \(\displaystyle \gamma_{23} = \) -0.16}%
\end{pgfscope}%
\begin{pgfscope}%
\pgfsetbuttcap%
\pgfsetroundjoin%
\definecolor{currentfill}{rgb}{0.494000,0.184000,0.556000}%
\pgfsetfillcolor{currentfill}%
\pgfsetlinewidth{1.003750pt}%
\definecolor{currentstroke}{rgb}{0.494000,0.184000,0.556000}%
\pgfsetstrokecolor{currentstroke}%
\pgfsetdash{}{0pt}%
\pgfsys@defobject{currentmarker}{\pgfqpoint{-0.041667in}{-0.041667in}}{\pgfqpoint{0.041667in}{0.041667in}}{%
\pgfpathmoveto{\pgfqpoint{-0.041667in}{-0.041667in}}%
\pgfpathlineto{\pgfqpoint{0.041667in}{0.041667in}}%
\pgfpathmoveto{\pgfqpoint{-0.041667in}{0.041667in}}%
\pgfpathlineto{\pgfqpoint{0.041667in}{-0.041667in}}%
\pgfusepath{stroke,fill}%
}%
\begin{pgfscope}%
\pgfsys@transformshift{0.982871in}{0.921451in}%
\pgfsys@useobject{currentmarker}{}%
\end{pgfscope}%
\end{pgfscope}%
\begin{pgfscope}%
\definecolor{textcolor}{rgb}{0.000000,0.000000,0.000000}%
\pgfsetstrokecolor{textcolor}%
\pgfsetfillcolor{textcolor}%
\pgftext[x=1.207871in,y=0.877701in,left,base]{\color{textcolor}\rmfamily\fontsize{9.000000}{10.800000}\selectfont \(\displaystyle \gamma_{24} = \) -0.12}%
\end{pgfscope}%
\begin{pgfscope}%
\pgfsetbuttcap%
\pgfsetmiterjoin%
\definecolor{currentfill}{rgb}{0.000000,0.000000,0.000000}%
\pgfsetfillcolor{currentfill}%
\pgfsetfillopacity{0.000000}%
\pgfsetlinewidth{1.003750pt}%
\definecolor{currentstroke}{rgb}{0.635000,0.078000,0.184000}%
\pgfsetstrokecolor{currentstroke}%
\pgfsetdash{}{0pt}%
\pgfsys@defobject{currentmarker}{\pgfqpoint{-0.035355in}{-0.058926in}}{\pgfqpoint{0.035355in}{0.058926in}}{%
\pgfpathmoveto{\pgfqpoint{-0.000000in}{-0.058926in}}%
\pgfpathlineto{\pgfqpoint{0.035355in}{0.000000in}}%
\pgfpathlineto{\pgfqpoint{0.000000in}{0.058926in}}%
\pgfpathlineto{\pgfqpoint{-0.035355in}{0.000000in}}%
\pgfpathclose%
\pgfusepath{stroke,fill}%
}%
\begin{pgfscope}%
\pgfsys@transformshift{0.982871in}{0.747146in}%
\pgfsys@useobject{currentmarker}{}%
\end{pgfscope}%
\end{pgfscope}%
\begin{pgfscope}%
\definecolor{textcolor}{rgb}{0.000000,0.000000,0.000000}%
\pgfsetstrokecolor{textcolor}%
\pgfsetfillcolor{textcolor}%
\pgftext[x=1.207871in,y=0.703396in,left,base]{\color{textcolor}\rmfamily\fontsize{9.000000}{10.800000}\selectfont \(\displaystyle \gamma_{25} = \) -0.18}%
\end{pgfscope}%
\end{pgfpicture}%
\makeatother%
\endgroup%
}
					\caption{Cluster V}
				    \label{SubFig:Cluster_VI_real}
				\end{subfigure}
				\caption{RMSE coeficientes de amortiguamiento en función de SNR para los diferentes conjuntos. Lineas punteadas corresponden a \cite{Andersson2014}; lineas solidas corresponden a \emph{Shift-and-Zoom}.}
				\label{Fig:RMSE_Cluster_gamma}
			\end{figure}

    %\section{Conclusiones}

     %   Se analizo la estabilidad  numérica de las técnicas de estimación espectral. Se demostró que las estimaciones se vuelven vulnerables a pequeñas perturbaciones en los en los datos observados cuando las energía asociada a las frecuencias es pequeña y/o las frecuencias están muy cerca entre sí. La   técnica \emph{shift-and-zoom} hace frente a este problema cuando se dispone de información a priori. Se demostró que esta técnica supera a los enfoque basados en la estructura de la matriz de Hankel que no se benefician de la mejora en el espaciado entre las frecuencias complejas introducido por la  técnica \emph{shift-and-zoom}. Se demostró que este esquema es mas eficiente que los esquemas tradicionales cunado de trabaja en un régimen de SNR bajas. En cambio cuando la SNR es alta esta nueva técnica es tan eficiente como los métodos tradicionales. En este caso, la disminución en el número de muestras debido al paso de decimación se vuelve relevante y \emph{shift-and-zoom} requiere una cantidad de datos más grande para converger a limite de Cramér-Rao. Por lo tanto, existe una compensación entre el número de muestras y el factor de diezmado. 


			
		