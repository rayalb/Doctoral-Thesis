\chapter{Introducción}\label{chap:Introduccion}

\section{Marco del Proyecto}
El presente trabajo se desarrolló en el Centro de Simulación Computacional para Aplicaciones Tecnológicas (CSC) del CONICET. Este trabajo se enmarca en diferentes proyectos de investigación donde se destaca el Convenio de Investigación y Desarrollo entre el CONICET y INTI  (Expte. Nº6932/16) para el desarrollo de capacidades técnicas conjuntas con el fin de diseñar un prototipo de sensor UWB y técnicas de detección y clasificación de material dieléctrico. Dicho convenio incluyó otros proyectos de investigación y desarrollo por parte tanto de personal del CONICET como de INTI.

Adicionalmente, parte del trabajo ha sido financiado por el proyecto PICT 2016-1925 de la ANPCyT titulado ``Desarrollo de técnicas de sensado no-invasivo mediante señales UWB y su implementación en una plataforma experimental''. Y los proyectos UBACyt 20020150200248BA: ``Análisis, diseño e implementación de una plataforma inalámbrica de ancho de banda ultra grande (UWB) con aplicación a radares'' y UBACyT 20020170200283BA ``Desarrollo de una plataforma inalámbrica UWB y de algoritmos de inferencia estadística para aplicaciones de radar''.



\section{Descripción del problema a estudiar}

En ingeniería es común trabajar con señales que son modeladas como mezclas de señales sinusoidales en un entorno ruidoso. Estos últimos pueden resultar importantes para la clasificación de señales. Entre los muchos ejemplos donde aparecen señales que decaen exponencialmente se encuentran, señales de espectroscopia \cite{Gudmundson2012}, el análisis de señales sísmicas \cite{Soltaninejad2020}, señales de radar \cite{Cuyt2020, Sarkar2000}, señales biomédicas \cite{Bannis2014, Chon2001}, detección de dirección de arribo (DOA) \cite{Knaepkens2020}, identificación por radiofrecuencia \cite{Rezaiesarlak2013}, señales musicales \cite{Laroche1993}, por nombrar solo algunos. Un ejemplo donde la identificación de las frecuencias complejas se vuelve relevante para clasificación de señales es el caso de la dispersión recogida de un objeto dieléctrico iluminado por un pulso electromagnético, donde la mezcla de señales sinusoidales es una propiedad que caracteriza al objeto.


A pesar de su eficiencia computacional y amplia aplicabilidad, el análisis de Fourier tiene algunas limitaciones bien conocidas, como su resolución limitada y el \emph{leakage} en el dominio de la frecuencia. Estas restricciones complican el análisis de señales que presentan un decaimiento exponencial en el tiempo. El análisis de Fourier, que representa una señal como una suma de funciones periódicas, no es muy adecuado para la descomposición de señales aperiódicas, como las que decaen exponencialmente. El amortiguamiento provoca un ensanchamiento de los picos espectrales, lo que a su vez conduce a que los picos se superpongan y enmascaren los picos de amplitud más pequeños. 

Para el caso de frecuencias sin decaimiento exponencial, existen métodos que exhiben grandes ventajas sobre el análisis de Fourier, como robustez al ruido, y no requieren conocer la cantidad de frecuencias a estimar a priori. Estos métodos conocidos como dispersos (\emph{sparse}). Estos métodos pueden dividirse en dos categorías, por un lado, están los que discretizan el dominio de la frecuencia en una cuadrícula y las estimaciones están restringidas a esta cuadrícula, por lo que la precisión de la estimación está limitada. Por otro lado, están los métodos que realiza la estimación en el dominio de la frecuencia continua, resolviendo un problema de optimización no convexo, por lo que la precisión de las estimaciones también es afectada. 

Los métodos de subespacios basados en el cálculo de autovectores o descomposición de valores singulares son otra clase de enfoque para la estimación de frecuencias complejas. Estos algoritmos tienen capacidad de ``súper-resolución'', es decir, tiene una gran precisión cuando el nivel de ruido es suficientemente pequeño. Sin embargo, la cantidad de frecuencias debe ser conocida y a medida que el nivel de ruido aumenta, el desempeño de los métodos puede deteriorarse.


En base a este conocimiento, se observan dos problemas muy importantes. Por un lado, la cantidad de parámetros a estimar es desconocido; y por otro, las frecuencias asociadas a la mezcla de señales sinusoidales suelen estar muy cercanas entre sí, lo que puede resultar en problemas mal condicionados y afectar la estabilidad numérica de las estimaciones.

En esta tesis se estudiarán las dos problemáticas mencionadas anteriormente cuando se utilizan técnicas de estimación espectral para estimar los parámetros de sinusoides complejas corrompidas por ruido. En una primera instancia se analizarán señales sintéticas con el fin de proponer alternativas superadoras para estos dos problemas. Luego, los algoritmos propuestos se usarán para casos con señales reales.

\subsection{Selección del orden del modelo}

Un paso sensible en los método de estimación espectral basados en subespacios, es estimar con precisión el orden del modelo.  Las técnicas más empleadas para estimar el orden del modelo son los criterios de la información \cite{Stoica2004}. Entre los más conocidos se encuentran el Criterio de información de Akaike, MDL, así como enfoques más reciente desarrollados en \cite{Mariani2015, Nielsen2013}, que garantizan un buen desempeño en el caso asintótico. Sin embargo, para registros de datos pequeños, estos métodos ya no son óptimos y su desempeño se ve deteriorado a medida que disminuye la relación señal a ruido. Además, proponen soluciones para problemas generales y no tienen en cuenta el modelo multiexponencial. 

Alternativamente, en este trabajo, se estudia como primera medida la relación existente entre el modelo exponencial y el rango de la matriz de Hankel asociada a una suma de senoides. Esta relación viene dada por el Teorema de Kronecker \cite{Gantmacher1960} que establece una correspondencia uno a uno entre una combinación lineal de $r$ exponenciales complejas y una matriz de Hankel con rango $r$. Este teorema establece la solución al problema de encontrar una función racional cuyo numerador y denominador sean polinomios de grado $r-1$ y $r$ respectivamente  \cite{Fuhrmann2011}. Este problema está relacionado con los problemas de aproximaciones de Padé \cite{Gonnet2013} y la teoría de polinomios ortogonales \cite{Szego1939}. Desafortunadamente, esta relación falla cuando la señal está contaminada por ruido. Sin embargo, utilizando una descomposición en valores singulares de la matriz de Hankel, es posible descomponer su espacio columna en un subespacio relacionado a la señal y un subespacio de ruido. La dimensión del subespacio de señal se establece por el número de valores singulares prominentes. En el régimen de baja relación señal-ruido, no hay un corte claro entre los valores singulares y, por lo tanto, determinar cuáles son los valores singulares relevantes se convierte en una tarea compleja. 

En primera instancia, se estudian las matrices aleatorias con estructura Hankel para establecer un umbral para determinar la dimensión del espacio de señal.  Para obtener este umbral se analizará la distribución de los valores singulares de una matriz aleatoria con estructura Hankel, ya que si no se tiene en cuenta esta estructura, los umbrales obtenidos pueden conducir a una estimación deficiente. Además, se obtendrán cotas para la norma espectral de una matriz de Hankel aleatoria para estimar el rango de la matriz. Por ejemplo, en \cite{Qiao2020} se propone un umbral para separar los valores singulares significativos de los correspondientes a ruido cuando se trabaja con matrices reales aleatorias con entradas sub-gaussianas y estructura Toeplitz. Un umbral similar se obtuvo utilizando desigualdades de concentración para una matriz aleatoria de Hankel con entradas Gaussianas \cite{tropp2015}. Para matrices de Hankel más generales, en \cite{Hokanson2020} se propuso un umbral como cota superior de la norma espectral de una matriz de Hankel aleatoria. No obstante, todos estos umbrales son conservativos y no logran detectar el rango verdadero de la matriz de Hankel con entradas aleatorias, lo que dificulta el uso de cualquiera de ellos para determinar el orden del modelo. 

En este trabajo, se puso especial énfasis en el análisis de problemas en régimen del alto nivel de ruido. Dos métodos recientes con buen desempeño frente a un número pequeño de muestras son conocidos como ESTER \cite{Badeau2006} y SAMOS \cite{Papy2007}. Estos métodos se basan en las propiedades algebraicas de la matriz de Hankel y presentan muy buen desempeño cuando la relación señal a ruido es alta, pero fallan consistentemente a medida que aumenta el nivel de ruido. Para analizar este fenómeno, utilizamos los ángulos que forman distintos subespacios asociados a la matriz de Hankel para obtener  cotas de los errores que se comenten a medida que aumenta el nivel de ruido. Con esta nueva interpretación se muestra que los métodos basados en propiedades algebraicas son muy sensibles a pequeñas perturbaciones y como consecuencia presentan un rendimiento deficiente cuando el nivel de ruido es alto.
%Dos métodos recientes  frente a un número pequeño muestras son conocidos como ESTER \cite{Badeau2006} y SAMOS \cite{Papy2007}. 
%En este trabajo, se puso especial énfasis en el análisis de problemas en el régimen de alto nivel de ruido. Especialmente en dos técnicas, conocidas como ESTER \cite{Badeau2006} y SAMOS \cite{Papy2007}, basadas en la propiedades algebraicas de la matriz de Hankel tienen muy buen desempeño cuando la relación señal a ruido es alta, pero fallan consistentemente a medida que aumenta el nivel de ruido. Para analizar este fenómeno, utilizamos los ángulos que forman distintos subespacios asociados a la matriz de Hankel para obtener  cotas de los errores que se comenten a medida que aumenta el nivel de ruido. Con esta nueva interpretación se muestra que los métodos basados en propiedades algebraicas son muy sensibles a pequeñas perturbaciones y como consecuencia presentan un rendimiento deficiente cuando el nivel de ruido es alto.
 
Teniendo en cuenta estas dificultades, se propone una nueva estrategia mixta mediante un problema de optimización, donde se busca minimizar la función que mide el/los ángulos entre subespacios de señal generados por la matriz de Hankel, con una restricción sobre los valores singulares correspondiente al subespacio de ruido. %Esta nueva técnica de estimación se valida con simulaciones numéricas y se compara con los métodos previamente analizados. 

\subsection{Estimación espectral}

Para realizar la estimación paramétrica del modelo multiexponencial, necesitamos obtener los valores de las frecuencias de cada senoidal.  Un enfoque diferente al análisis de Fourier, entre otros, son los métodos paramétricos, donde se incluyen MUSIC \cite{Schmidt1986}, ESPRIT \cite{Roy1989}, el método de haz matricial (\emph{Matrix Pencil}, MPM) \cite{Hua1990}, factorización QR simultánea  \cite{Golub1999} o el método de Prony aproximado APM \cite{Potts2010}.

En general, los métodos paramétricos, así como el análisis de Fourier, utilizan el Teorema de Shannon-Nyquist \cite{Nyquist1928, Shannon1949} para obtener muestras de una señal. Este teorema afirma que la frecuencia de muestreo debe ser al menos el doble de la frecuencia máxima presente en la señal. Una frecuencia de muestreo más baja provoca \emph{aliasing}, identificando frecuencias más altas con frecuencias más bajas sin poder distinguirlas. En el último tiempo, enfoques alternativos han demostrado que la reconstrucción de señales también es posible a partir de mediciones obtenidas con tasas sub-Nyquist, si se conoce información adicional sobre la estructura de la señal, como por ejemplo su dispersión (\emph{sparsity}). De hecho, muchas señales son dispersas en algún dominio como el tiempo, la frecuencia o el espacio, lo que significa que la mayoría de las muestras de la señal o de su transformada en otro dominio pueden considerarse cero. 

Sin embargo, una de las principales preocupaciones a la hora de desarrollar estas técnicas ha sido hacer frente al ruido y a las incertidumbres del modelo. Para ello, \cite{Andersson2014} los autores formulan un problema de cuadrados mínimos no lineales en términos de una restricción sobre el rango de la matriz de Hankel asociada a la matriz observada. Mientras que en \cite{Grussler2018} se plantea el problema de aproximación de bajo rango contemplando la estructura Hankel de la matriz observada. Un enfoque similar se propone en \cite{Ying2018} donde la estimación de los parámetros se realiza minimizando la norma nuclear de la matriz de Hankel. En \cite{Yang2018} se propone un enfoque diferente donde los autores consideran sumas de exponenciales no amortiguadas. En este caso, usando un conjunto de frecuencias discretas previamente definido, se consideran técnicas de \emph{Compressed Sensing} \cite{Donoho2006} para estimar los parámetros desconocidos. En \cite{Yang2015,Yang2016} se presentan algunas mejoras a este enfoque donde la solución se obtiene minimizando la norma atómica de la señal estimada. En estos casos, se garantiza la recuperación exacta de la señal observada cuando las frecuencias están separadas adecuadamente \cite{Candes2014}. Desafortunadamente, estos métodos no son adecuados para suma de exponenciales complejas amortiguadas.

Como se señaló en \cite{Halder1997}, el rendimiento de los métodos de estimación espectral se degradan cuando las frecuencias están muy cercanas entre sí. No obstante, una estimación precisa de los parámetros espectrales para suma de exponenciales complejas amortiguadas se realiza mediante la solución de un problema de autovalores generalizados. Sin embargo, estos problemas suelen estar mal condicionados por lo que se requiere un esfuerzo adicional para ejecutar de manera confiable los algoritmos de estimación correspondientes. El objetivo será analizar el comportamiento de la estimación de la frecuencia cuando los datos observados están sujetos a pequeñas perturbaciones. En particular, se estudia la estabilidad numérica del problema de autovalores generalizados que se construye a partir de la matriz con estructura Hankel asociada a la señal observada. El estudio de la estabilidad numérica  de los autovalores generalizados fue abordado antes en \cite{Golub1999, Beckermann2007}. En estos trabajos se ha observado que la sensibilidad de cada autovalor es inversamente proporcional a la amplitud asociada a la frecuencia amortiguada correspondiente al autovalor. En esta tesis se extiende este resultado y se demuestra que la aproximación de primer orden de los autovalores perturbados también es inversamente proporcional a la distancia entre ellos. Por lo tanto, los autovalores que están muy cerca uno del otro son propensos a tener estimaciones que se desvían en gran medida de sus valores verdaderos, incluso cuando la señal observada está ligeramente perturbada.

Por lo tanto, queda claro que resolver problemas de autovalores generalizados con las características mencionadas es un desafío desde la perspectiva algorítmica y puede volverse aún más difícil cuando el número de frecuencias en la señal original es muy grande y estas frecuencias están agrupadas en pequeñas regiones del plano complejo \cite{Cuyt2018, BATENKOV2018, Li1997}. Otra forma de mejorar la estabilidad numérica es la aplicación de enfoques numéricos más sofisticados a matrices y su conexión con los métodos de aproximación racional, de modo que se puedan utilizar algoritmos estables existentes \cite{Derevianko2022}. Sin embargo, este método sólo es válido cuando las frecuencias son puramente imaginarias.

Para sobreponerse a este problema se propone aumentar artificialmente la separación de las frecuencias realizando una decimación de la señal para luego aplicar alguno de los métodos de estimación espectral estudiados. Esta técnica es mencionada en varios trabajos, sin embargo, la decimación se realiza asegurando que no se introduce \emph{aliasing}. Por lo tanto, la señal original debe estar sobre muestreada. Basados es estos resultados, se propone la siguiente estrategia para pre-procesar la señal observada. Si se asume que el contenido espectral de la señal observada se concentra en diferentes bandas de frecuencias, para mejorar la estimación se realiza un \emph{Zoom} sobre esas bandas por separado. Luego, para cada banda se obtiene el equivalente en banda base de la señal y se aplica la decimación. Para evitar \emph{aliasing} el equivalente en banda base es filtrado antes de realizar la decimación. 


\subsection{Aplicación: Clasificación de materiales}

Un problema relacionado con una mezcla de señales senoidales poco espaciadas en un entorno ruidoso se observa en la dispersión medida de un objeto cuando es iluminado por una señal electromagnética UWB, donde la señal reflejada se puede caracterizar mediante el modelo exponencial \cite{Baum1971} donde sus parámetros transportan una gran cantidad de información sobre las características del objeto. En particular, se ha demostrado que las frecuencias complejas solo dependen del tamaño, forma y propiedades eléctricas del objeto iluminado. Por lo tanto, se ha propuesto tomar estas frecuencias como características para la clasificación de diferentes blancos \cite{Knochel2005, Bannis2014, LAUNAY2013, Altieri2020}.  Este procedimiento equivale a resolver un problema de estimación espectral construido a partir de las muestras del campo disperso. Sin embargo, se ha observado que la energía asociada con cada una de las frecuencias depende de las posiciones del punto donde se envía el campo incidente, el punto de observación y el objeto analizado \cite{Baum1976}. Este hecho implica que es posible que alguna frecuencia no se observe de manera significativa, o que varias frecuencias se agrupen en una región particular del plano complejo muy cerca entre sí, lo que hace que la tarea de identificación sea un problema de estimación espectral desafiante. En particular, se requieren técnicas de estimación espectral capaces de hacer frente a un orden de modelo incierto y una baja relación señal a ruido. 

Sin embargo, como se observa a lo largo del trabajo, se trata de un problema mal condicionado cuya solución está sesgada ante pequeñas perturbaciones. En este trabajo se propone un nuevo criterio de clasificación de señales explotando la relación entre el problema de autovalores generalizados y el rango numérico \cite{Chorianopoulos2009} de un par de matrices rectangulares. En particular, la prueba de clasificación se formula como un problema de inclusión de conjuntos y no se requiere estimación del espectro. Esta técnica se aplica en simulaciones numéricas de la dispersión de diferentes materiales dieléctricos.

%Finalmente, dado los avances de los métodos de aprendizaje profundo, se debe mencionar que estos puede usarse para la estimación del orden del modelo \cite{Moon2021} y la estimación de los parámetros \cite{Mhaska2020}. En particular, los modelos de entrenamiento para el aprendizaje requieren grandes cantidades de datos para minimizar el error de generalización. Desafortunadamente, muchas veces los datos de entrenamiento no está disponibles para todas las aplicaciones.


\section{Organización de la Tesis}


El resto de esta tesis está organizada de la siguiente manera:

\begin{itemize}
	\item Capítulo~\ref{chap:ModeloSumExp} se presenta una descripción de los métodos utilizados para la extracción de frecuencias, y se analiza la estabilidad numérica de los métodos para obtener los parámetros asociados al modelo exponencial. 
	\item[] 
	\item Capítulo~\ref{chap:OrdenModelo} hace hincapié en la selección del orden del modelo. Se presenta una descripción de los diferentes métodos encontrados en la literatura y se analizan sus dificultades. 
	\item[] 
    \item Capítulo ~\ref{chap:RandomHankel} se analiza la distribución de los valores singulares para los casos de matrices aleatorias con y sin estructura Hankel. Además, se presentan cotas para la distribución de la norma espectral para el caso de matrices con estructura Hankel. 
    \item[]
	\item Capítulo ~\ref{chap:CriterioOrdenOpt} teniendo en cuenta las dificultades encontradas en los métodos estudiados tanto en el Capítulo \ref{chap:OrdenModelo} como en el Capítulo \ref{chap:RandomHankel} se propone una nueva alternativa basada en las propiedades algebraicas y estadísticas que presenta el modelo.
	\item[] 
	\item Capítulo~\ref{chap:EstabilidadNumerica} se extiende el resultado sobre estabilidad numérica descrito en el Capítulo \ref{chap:ModeloSumExp}. Además, se propone una estrategia para pre-procesar la señal observada mediante un procesamiento de múltiples tasas de muestreo. Esto tiene la capacidad de acomodar las frecuencias complejas para un cálculo más estable.
	\item[] 
	\item Capítulo~\ref{chap:sinalUWB} se muestran resultados de los métodos desarrollados aplicados a mediciones tomadas en el laboratorio utilizando señales de UltraWide Band (UWB).
	\item[] 
	\item Capítulo ~\ref{chap:RangoNumerico} se diseña una estrategia novedosa para clasificar mezclas de señales sinusoidales basada en el rango numérico de matrices. En particular, la clasificación se formula como un problema de inclusión de conjuntos, evitando así la necesidad de calcular autovalores de matrices de grandes dimensiones mal condicionadas.
	\item[]
	\item Capítulo~\ref{chap:conclusiones} se presentan las conclusiones finales. Resume los principales logros de esta tesis y describe futuras direcciones de investigación.
\end{itemize}

\section{Contribuciones}

Las principales contribuciones se pueden resumir de la siguiente manera:
\begin{itemize}
	\item Con respecto a los algoritmos de estimación paramétrica, se obtuvieron cotas para el error que se comete cuando los modos tienen poca energía y/o la distancia entre ellos es pequeña. Estos resultados fueron reportados en ``\textit{Spectrum estimation using frequency shifting and decimation}''. Albert, Galarza. IET Signal Processing \cite{Albert2020}. 
	\item[] 
	\item Para hacer frente al problema de estabilidad numérica se propuso una nueva estrategia para aumentar artificialmente la separación de frecuencias cambiando la frecuencia de muestreo de la señal ruidosa antes de aplicar los métodos de estimación espectral. Se muestra que esta metodología mejora significativamente el rendimiento de la estimación, especialmente en el régimen de baja relación señal/ruido. Esta técnica fue propuesta en ``\textit{Spectrum estimation using frequency shifting and decimation}''. Albert, Galarza. IET Signal Processing \cite{Albert2020}.
	\item[] 
    \item Para el problema de selección de orden del modelo utilizando los métodos ESTER y SAMOS, se obtuvieron expresiones para las funciones costo que permitieron analizar el comportamiento de los mismos cuando la señal observada es ruidosa. Estos resultados se reportaron en  ``\textit{A constraint optimization problem for model order estimation}''. Albert, Galarza. Signal Processing \cite{ALBERT2023}.
    \item[]
    \item Se obtuvieron cotas para la norma espectral de una matriz aleatoria con estructura Hankel considerando la estructura de su valor singular máximo. Estos resultados se reportan en ``\textit{A constraint optimization problem for model order estimation}''. Albert, Galarza. Signal Processing \cite{ALBERT2023}, y en ``\textit{Model order selection for sum of complex exponential}''. Albert, Galarza. IEEE URUCON\cite{Albert2021}. 
    \item[] 
    \item A partir de la relación entre el principio de invariancia rotacional y los ángulos principales entre subespacios se obtienen resultados sobre las perturbaciones para el problema de autovalores generalizados del haz matricial. Estos resultados se presentan en ``\textit{A constraint optimization problem for model order estimation}''. Albert, Galarza. Signal Processing \cite{ALBERT2023}.
    \item[]
	\item Se presenta un nuevo método para estimar el número de términos en la mezcla señales sinusoidales. En particular, se propone combinar las propiedades algebraicas y estadísticas de los métodos basados en subespacios. Esta nueva técnica de estimación  muestra mejoras significativas sobre los métodos basados en subespacios. Esta técnica  se reporta en ``\textit{A constraint optimization problem for model order estimation}''. Albert, Galarza. Signal Processing \cite{ALBERT2023}, y en ``\textit{Model order selection for sum of complex exponential}''. Albert, Galarza. IEEE URUCON\cite{Albert2021}. 
	\item[]
	\item Se propone un nuevo criterio de clasificación de señales explotando la relación entre el problema de autovalores generalizados y el rango numérico de un par de matrices rectangulares. En particular, el problema de clasificación se formula como un problema de inclusión de conjuntos, y no se requiere las estimación de la frecuencias. Este criterio se describe en ``\textit{Dielectric classification by sensing scattering field}''. Albert, et.al. XVIII RPIC \cite{Albert2017} y en ``\textit{Classification of matrices using their numerical range}''. Albert, Galarza. ARGECON \cite{Albert2018}
	\item[] 
\end{itemize}




%Además, suelen presentarse otros obstáculos. Por un lado, la adquisición de mediciones pueden resultar muy costosa y por lo tanto ser limitada. En otros casos, estas mediciones pueden estar contaminadas con ruido. Muchos ejemplos se encuentran en imágenes por resonancia magnética \cite{Spencer2020}, espectroscopia \cite{Gudmundson2012}, en el análisis de señales sísmicas \cite{Soltaninejad2020}, señales de radar \cite{Cuyt2020, Sarkar2000}, señales biomédicas \cite{Bannis2014, Chon2001}, detección de dirección de arribo (DOA) \cite{Knaepkens2020}, identificación por radio frecuencia \cite{Rezaiesarlak2013}, señales musicales \cite{Laroche1993}, por nombrar solo algunos. 

%En general, estas señales se suelen representar como una suma de exponenciales complejas que puede ser amortiguadas o no amortiguadas. Los métodos de subespacios \cite{Stoica2005} son herramientas adecuadas para la estimación espectral, ya que ofrecen una alternativa al enfoque de \emph{Compress Censing} (CS) \cite{Yang2018}. Una ventaja esencial de los métodos basados en  subespacios es que no necesita mediciones recopiladas aleatoriamente, sino que funciona con un muestreo determinista basado en un esquema de muestreo que se adapta al modelo de señal no lineal.

