\chapter{Introducción}\label{chap:Introduccion}

A pesar de su eficiencia computacional y amplia aplicabilidad, el análisis de Fourier tiene algunas limitaciones bien conocidas, como su resolución limitada y el manchado espectral o \emph{leakage} en el dominio de la frecuencia. Estas restricciones complican el análisis de señales que presentan un decaimiento exponencial en el tiempo. El análisis de Fourier que representa una señal como una suma de funciones periódicas, no es muy adecuado para la descomposición de señales aperiódicas, como las que decaen exponencialmente. El amortiguamiento provoca un ensanchamiento de los picos espectrales, lo que a si vez conduce a que los picos se superpongan y enmascaren los picos de amplitud más pequeños. Estos últimos pueden resultar importantes para la clasificación de señales. 

Muchas ejemplos donde aparecen señales que decaen exponencialmente se encuentran, señales de espectroscopia \cite{Gudmundson2012}, en el análisis de señales sísmicas \cite{Soltaninejad2020}, señales de radar \cite{Cuyt2020, Sarkar2000}, señales biomédicas \cite{Bannis2014, Chon2001}, detección de dirección de arribo (DOA) \cite{Knaepkens2020}, identificación por radio frecuencia \cite{Rezaiesarlak2013}, señales musicales \cite{Laroche1993}, por nombrar solo algunos. 

Un enfoque diferente al análisis espectral, entre otros, son los métodos paramétricos. Los métodos paramétricos ampliamente utilizados que suponen un modelo exponencial incluyen MUSIC \cite{Schmidt1986}, ESPRIT \cite{Roy1989}, el método de haz matricial (\emph{Matrix Pencil}, MPM) \cite{Hua1990}, factorización QR simultanea  \cite{Golub1999} o el método de Prony aproximado APM \cite{Potts2010}.

En general, los métodos paramétricos, así como el análisis de Fourier, muestran a una frecuencia dad por el Teorema de Shannon-Nyquist \cite{Nyquist1928, Shannon1949}. Este teorema afirma que la frecuencia de muestreo debe ser al menos el doble de la frecuencia máxima presente en la señal. Una frecuencia de muestreo más baja provoca \emph{aliasing}, identificando frecuencias más altas con frecuencias más bajas sin poder distinguirlas. En el último tiempo, enfoques alternativos han demostrado que la reconstrucción de señales también es posible a partir de mediciones sub-Nyquist, si se conoce información adicional sobre la estructura de la señal, como por ejemplo su dispersión (\emph{sparsity}). De hecho, muchas señales son dispersas en algún dominio como el tiempo, la frecuencia o el espacio, lo que significa que la mayoría de las muestras de la señal o de su transformada en otro dominio pueden considerarse cero. Entre otros métodos, se pueden citar aquellos que se basan en \emph{Compress Sensing} \cite{Donoho2006}. Una ventaja esencial de los métodos paramétricos es que no necesita mediciones recopiladas aleatoriamente, sino que funciona con un muestreo determinista basado en un esquema de muestreo que se adapta al modelo de señal no lineal.

Sin embargo, una de las principales preocupaciones a la hora de desarrollar estas técnicas ha sido hacer frente al ruido y a las incertidumbres del modelo. Últimamente, se han propuesto nuevas técnicas de optimización \cite{Andersson2014, Grussler2018}. En \cite{Andersson2014} los autores formulan un problema de cuadrados mínimos no lineales en términos de una restricción sobre el rango de la matriz de Hankel asociada a la matriz observada. Mientras que en \cite{Grussler2018} se plantea el problema de aproximación de bajo rango contemplando la estructura Hankel de la matriz observada. Un enfoque similar se propone en \cite{Ying2018} donde la estimación de los parámetros se realiza minimizando la norma nuclear de la matriz de Hankel. En \cite{Yang2018} se propone un enfoque diferente donde los autores consideran sumas de exponenciales no amortiguadas. En este caso, usando un conjunto de frecuencias discretas previamente definido, se consideran técnicas de \emph{Compressed Sensing} para estimar los parámetros desconocidos. En \cite{Yang2015,Yang2016} se presentan algunas mejoras a este enfoque donde la solución se obtiene minimizando la norma atómica de la señal estimada. En estos casos, se garantiza la recuperación exacta de la señal observada cuando la frecuencias están separadas adecuadamente \cite{Candes2014}. Desafortunadamente, estos métodos no son adecuados para suma de exponenciales complejas amortiguadas.

Como se señalo en \cite{Halder1997}, el rendimiento de los métodos de estimación espectral, se degradan cuando se tiene frecuencias muy cercanas entre sí. No obstante, una estimación precisa de la parámetros espectrales para suma de exponenciales complejas amortiguadas se realiza mediante la solución del problema de autovalores generalizados. Estos problemas suelen estar mal condicionados y muchas veces las matrices asociadas no suelen ser cuadradas. Resolver problemas de autovalores generalizados con estas características es un desafío desde la perspectiva algorítmica. Esta dificultad se vuelve más difícil cuando el número de frecuencias en la señal original es muy grande y estas frecuencias están agrupadas en pequeñas regiones del plano complejo \cite{Cuyt2018, BATENKOV2018, Li1997}. Un problema relacionado se ha abordado antes en \cite{Golub1999, Beckermann2007}. En particular, se ha observado que la sensibilidad de cada autovalor es inversamente proporcional a la amplitud asociada a la frecuencias amortiguada correspondiente al autovalor. Este problema está estrechamente relacionado con la teoría de aproximación racional y la aproximación de bajo rango de matrices. Sin embrago, estas conexiones aún no se comprenden completamente y pueden conducir a algoritmos de reconstrucción mejores. Otro forma de mejorar la estabilidad numérica es la aplicación de enfoques numéricos mas sofisticados a matrices y su conexión con los métodos de aproximación racional, de modo que se puedan utilizar algoritmos estables existentes \cite{Derevianko2022}.

Por otro lado, un paso sensible en cualquier método de estimación espectral paramétrica es estimar con precisión el orden del modelo.  Las técnicas más empleada para estimar el orden del modelo son los criterios de la información \cite{Stoica2004}. Entre los más conocidos se encuentran el Criterio de información de Akaike, MDL, así como enfoques más reciente desarrollados en \cite{Mariani2015, Nielsen2013}, que garantizan un buen desempeño en el caso asintótico. Sin embargo, para registros de datos pequeños, estos métodos ya no son óptimos y su desempeño se ve deteriorada a medida que disminuye la relación señal a ruido (SNR).

Una estrategia alternativa utiliza el Teorema de Kronecker \cite{Gantmacher1960} que establece un correspondencia uno a uno entre una combinación lineal de $r$ exponenciales complejas y una matriz de Hankel con rango $r$. Este teorema establece la solución al problema de encontrar una función racional cuyo numerador y denominador sean polinomios de grado $r-1$ y $r$ respectivamente  \cite{Fuhrmann2011}. Desafortunadamente, estos resultados se  dedujeron para señales sin perturbaciones y se utilizan como una aproximación cuando la señal observada está contaminada por ruido. Cuando hay ruido presente, el espacio columna de la matriz de Hankel se puede descomponer en la suma directa del subespacio de señal y el subespacio de ruido. Por lo general, la dimensión del subespacio de señal se establece por el número de valores singulares prominentes de la matriz de Hankel. Sin embrago, a medida que la SNR disminuye, la brecha entre valores singulares consecutivas disminuye, lo que dificulta determinar cuántos valores singulares relevantes tiene la matriz de Hankel.

El principio de invariancia rotacional \cite{Roy1989} explota la estructura de los datos cuando se trabaja con una suma de exponenciales complejas. Continuado por este camino, se propusieron dos técnicas conocidas como ESTER y SAMOS \cite{Badeau2006, Papy2007} para la estimación del orden del modelo. Si bien se demostró que SAMOS es más robusto que ESTER, ambas técnicas presentan un buen desempeño en régimen de SNR alto. Sin embargo, ambos métodos se basan en suposiciones sin ruido, no logran obtener buenos resultados en condiciones más generales. 

Por otro lado, algunos autores han considerado la aleatoriedad de las señales ruidosas explotando sus propiedades estadísticas. Por ejemplo, en \cite{Qiao2020} se propone un umbral para separar los valores singulares significativos de los correspondiente a ruido cuando se trabaja con matrices reales aleatorias con entradas sub-gaussianas y estructura Toeplitz. Un umbral similar se obtuvo utilizando desigualdades de concentración para una matriz aleatoria de Hankel con entradas Gaussianas \cite{tropp2015}. Para matrices de Hankel más generales, en \cite{Hokanson2020} se propuso un umbral como cota superior de la norma espectral de una matriz de Hankel aleatoria. No obstante, todos estos umbrales son conservativos y no logran detectar el rango verdadero de la matriz de Hankel con entradas aleatorias, lo que dificulta el uso de cualquiera de ellos para determinar el orden del modelo. 

Finalmente, dado los avances de los métodos de aprendizaje profundo, debemos mencionar que estos puede usarse para la estimación del orden del modelo \cite{Moon2021} y la estimación de los parámetros \cite{Mhaska2020}. En particular, los modelos de entrenamiento para el aprendizaje requieren grandes cantidades de datos para minimizar el error de generalización. Desafortunadamente, muchas veces los datos de entrenamiento no está disponibles para todas las aplicaciones.

\section{Contribuciones}

El resto de esta tesis está organizada de la siguiente manera:

\begin{itemize}
	\item Capítulo~\ref{chap:ModeloSumExp} se presenta una descripción de los métodos utilizados para la extracción de frecuencias, y se analiza la estabilidad numérica de los métodos para obtener los parámetros asociado al modelo exponencial. 
	\item[] 
	\item Capítulo~\ref{chap:OrdenModelo} hace hincapié en la selección del orden del modelo. Se presenta una descripción de los diferentes métodos encontrados en la literatura y se analizan sus dificultades. 
	\item[] 
    \item Capítulo ~\ref{chap:RandomHankel} se analiza la distribución de los valores singulares para los casos de matrices aleatorias con y sin estructura Hankel. Además se presentan cotas para la distribución de la norma espectral para el caso de matrices con estructura Hankel. 
    \item[]
	\item Capítulo ~\ref{chap:CriterioOrdenOpt} teniendo en cuenta la dificultades encontradas en los métodos estudiados tanto en el Capítulo \ref{chap:OrdenModelo} como en el Capítulo \ref{chap:RandomHankel} se propone una nueva alternativa basado en las propiedades algebraicas y estadísticas que presenta el modelo.
	\item[] 
	\item Capítulo~\ref{chap:EstabilidadNumerica} se extiende el resultado sobre estabilidad numérica descrito en el Capítulo \ref{chap:ModeloSumExp}. Además, se propone un estrategia para pre-procesar la señal observada mediante un procesamiento muti-frecuencia de muestreo. Esto tiene la capacidad de acomodar las frecuencias complejas para una cálculo más estable.
	\item[] 
	\item Capítulo~\ref{chap:sinalUWB} se muestran resultados de los métodos desarrollados aplicado a mediciones tomadas en el laboratorio utilizando señales de UltraWide Band (UWB).
	\item[] 
	\item Capítulo ~\ref{chap:RangoNumerico} se diseña una estrategia novedosa para clasificar mezclas de señales sinusoidales basado en el rango numérico de matrices. En particular, la clasificación se formula como un problema de inclusión de conjuntos, evitando así la necesidad de calcular autovalores de matrices de grandes dimensiones mal condicionadas.
	\item[]
	\item Capítulo~\ref{chap:conclusiones} se presentan las conclusiones finales y trabajos a futuro, 
\end{itemize}

Las principales contribuciones se pueden resumir de la siguiente manera:
\begin{itemize}
	\item Se resumen los distintos métodos de estimación espectral aplicados a mezclas de señales sinusoidales. Se demostró que las estimaciones se vuelven vulnerables a pequeñas perturbaciones en las señales observadas cuando los modos tiene poca energía y/o la distancia entre ellos es pequeña.
	\item[] 
	\item Para hacer frente al problema de estabilidad numérica se propone una nueva estrategia para aumentar artificialmente la separación de frecuencias cambiando la frecuencia de muestre de la señal ruidosa antes de aplicar los métodos de estimación espectral. Se muestra que esta metodología mejora significativamente el rendimiento de la estimación, especialmente en el régimen de baja relación señal/ruido. Se comparan los resultados obtenidos en distintas publicaciones.
	\item[] 
	\item Se presenta un nuevo método para estimar el número de términos en la mezcla señales sinusoidales. En particular, se propone combinar las propiedades algebraicas y estadísticas de los métodos basados en subespacios. Esta nueva técnica de estimación  muestra mejoras significativas sobre los métodos basados en subespacios. En particular, cuando no es posible tener una buena separación entre los subespacio de ruido y de señal, esta nueva metodología supera a las técnicas conocidas. El desempeño se evaluó usando experimentos numéricos y comparando con resultado previos encontrados en la bibliografía.
	\item[]
	\item Se propone un nuevo criterio de clasificación de señales explotando la relación entre el problema de autovalores generalizados y el rango numérico de un par de matrices rectangulares. En particular, el problema de clasificación se formula como un problema de inclusión de conjuntos, y no se requiere la estimación de la frecuencias.
\end{itemize}


%Además, suelen presentarse otros obstáculos. Por un lado, la adquisición de mediciones pueden resultar muy costosa y por lo tanto ser limitada. En otros casos, estas mediciones pueden estar contaminadas con ruido. Muchos ejemplos se encuentran en imágenes por resonancia magnética \cite{Spencer2020}, espectroscopia \cite{Gudmundson2012}, en el análisis de señales sísmicas \cite{Soltaninejad2020}, señales de radar \cite{Cuyt2020, Sarkar2000}, señales biomédicas \cite{Bannis2014, Chon2001}, detección de dirección de arribo (DOA) \cite{Knaepkens2020}, identificación por radio frecuencia \cite{Rezaiesarlak2013}, señales musicales \cite{Laroche1993}, por nombrar solo algunos. 

%En general, estas señales se suelen representar como una suma de exponenciales complejas que puede ser amortiguadas o no amortiguadas. Los métodos de subespacios \cite{Stoica2005} son herramientas adecuadas para la estimación espectral, ya que ofrecen una alternativa al enfoque de \emph{Compress Censing} (CS) \cite{Yang2018}. Una ventaja esencial de los métodos basados en  subespacios es que no necesita mediciones recopiladas aleatoriamente, sino que funciona con un muestreo determinista basado en un esquema de muestreo que se adapta al modelo de señal no lineal.

