\chapter{Introducción}\label{chap:Introduccion}


En ingeniería se observan regularmente problemas relacionados con una mezcla de señales sinusoidales en un entorno ruidoso. Un ejemplo es el caso de la dispersión recogida de un objeto dieléctrico iluminado por un pulso electromagnético, donde la mezcla particular es una propiedad que caracteriza al objeto. Por lo tanto, la identificación de las frecuencias complejas en la señal dispersada se vuelve relevante para la identificación y clasificación de diferentes objetos. El mejor manejo en la extracción de características consiste en comprender la interacción física entre el pulso electromagnético y el objeto sensado. En base a este conocimiento, se observan dos problemas muy importantes. Por un lado, la cantidad de parámetros a estimar es desconocido; y por otro, las frecuencias asociada a la mezcla de señales sinusoidales suelen estar muy cercanas entre sí lo que puede resultar en problemas mal condicionados y afectar la estabilidad numéricas de las estimaciones.

En esta tesis se estudiaran las dos problemáticas mencionadas anteriormente cuando se utilizan técnicas de estimación espectral para estimar los parámetros de sinusoides complejas corrompidas por ruido. En una primera instancia se analizaran señales sintéticas con el fin de proponer alternativas superadoras para estos dos problemas. Luego, los algoritmos propuestos se usaran para casos con señales reales.

Para estimar el número de términos en una suma de sinusoides amortiguadas exponencialmente embebidas en ruido, se estudiará como primera medida la relación existente entre el modelo exponencial y el rango de la matriz de Hanke construida a partir de la suma de senoides. Desafortunadamente, esta relación falla cuando la señal esta contaminada por ruido. Sin embargo, utilizando una descomposición en valores singulares de la matriz de Hankel, es posible descomponer su espacio columna en un subespacio relacionado a la señal y un subespacio de ruido. La dimensión del subespacio de señal se establece por el número de valores singulares prominentes. En el régimen de baja relación señal-ruido, no hay un corte claro entre los valores singulares y, por lo tanto, determinar cuales son los valores singulares relevantes se convierte en un tarea difícil. Como una primera regla de selección, el rango de la matriz será igual al número de valores singulares que son mayores a cierto umbral. Para obtener este umbral se analizará la distribución de los valores singulares de una matriz aleatoria con estructura Hankel ya que si no se tiene en cuenta esta estructura, los umbrales obtenidos pueden conducir a una estimación deficiente. Además, se obtendrán cotas para la norma espectral de una matriz de Hankel aleatoria para estimar el rango de la matriz.

Alternativamente, se estudiarán métodos que usan propiedades algebraicas de la matriz de Hankel para la estimación del orden y se obtendrá una nueva interpretación en función de los ángulos que forman distintos subespacios asociados a la matriz de Hankel. Con esta nueva interpretación se muestra que los métodos basados en propiedades algebraicas serán muy sensibles a pequeñas perturbaciones y como consecuencia presentarán un rendimiento deficiente cuando el nivel de ruido es alto. 

Teniendo en cuenta estas dificultades, en este trabajo, se propone una nueva estrategia mixta mediante un problema de optimización, donde se busca minimizar la función que mide el/los ángulos entre subespacios de señal generados por la matriz de Hankel, con una restricción sobre los valores singulares correspondiente al subespacio de ruido.  Esta nueva técnica de estimación se validará con simulaciones numéricas y se comparará con los métodos previamente analizados. 


El otro problema analizado en esta tesis es la sensibilidad de los autovalores a la hora de aplicar algoritmos conocidos para su estimación. Es por ello que se estudiarán más detalladamente los diferentes algoritmos de estimación espectral encontrados en la literatura. Todos estos algoritmos tiene la particularidad que para estimar los parámetros del modelo exponencial resuelven un problema de autovalores generalizados. El objetivo será analizar el comportamiento de la estimación de la frecuencias cuando los datos observados están sujetos a pequeñas perturbaciones. En particular, se estudia la estabilidad  numérica del problemas de autovalores generalizados que se construye a partir de la matriz con estructura Hankel asociada a la señal observada. En este trabajo se extenderán los resultados previamente publicados sobre estabilidad  de los autovalores generalizados y se demostrará que  la aproximación de primer orden de los autovalores perturbados depende de la distancia entre los autovalores. Por lo tanto, los autovalores que están muy cerca uno del otro son propensos a tener estimaciones que se desvían en gran medida de sus valores verdaderos incluso cuando la señal observada está ligeramente perturbada. Para sobreponerse a este problema la idea será aumentar artificialmente la separación de las frecuencias realizando una decimación de la señal para luego aplicar alguno de los métodos de estimación espectral estudiados. Está técnica es mencionada en varios trabajos, sin embargo, la decimación se realiza asegurándose que no se introduce \emph{aliasing}. Por lo tanto, la señal original debe estar sobremuestreada. Basados es estos resultados se propone la siguiente estrategia para pre-procesar la señal observada. Si se asume que el contenido espectral de la señal observada se concentra en diferentes bandas de frecuencias, para mejorar la estimación de realiza un \emph{zoom} sobre esas bandas por separado. Luego, para cada banda se obtiene el equivalente en banda base de la señal y se aplica la decimación. Para evitar \emph{aliasing} el equivalente en banda base es filtrado antes de realizar la decimación. Esta técnica se validará con experimentos numéricos y se comparará con resultados obtenidos en publicaciones previas. %En particular, se mostró que el esquema propuesto es más eficiente que esquemas tradicionales cuando se trabaja en un régimen de baja relación señal a ruido. Cuando la relación señal a ruido es alta, esta nueva técnica es tan eficiente como los métodos más tradicionales. En este último caso la disminución en el número de muestras debido a la decimación se vuelve relevante, y se requiere una cantidad de muestras mayor para podes converger a la cota de Cramér-Rao. Por lo tanto, existe un compromiso entre la cantidad de muestras necesarias y el factor de decimación. 

Por otro lado, la identificación de las frecuencias complejas se vuelve relevante para la identificación y clasificación de distintos materiales en el caso del problema de dispersión electromagnética. Este problema tiene la particularidad que las frecuencias a estimar se encuentran muy cercanas entre si y además se desconoce la cantidad. Estos dos problemas son motivación suficiente para aplicar las técnicas presentadas en esta tesis a señales obtenidas a partir de un sistema de radar impulsivo UWB (\emph{Ultra WideBand}). 

Finalmente, se propone un nuevo criterio de clasificación de señales explotando la relación entre el problema de autovalores generalizados y el rango numérico del haz. En particular, este prueba de clasificación se formula como un problema de inclusión de conjuntos por lo que no es necesario estimación del espectro. Esta técnica se aplicará en simulaciones numéricas de la dispersión de diferentes materiales dieléctricos.



%El método del lápiz matricial (MPM) es una técnica bien conocida para estimar los parámetros de sinusoides amortiguadas exponencialmente en ruido resolviendo un problema de valores propios generalizado. Sin embargo, en varios casos, se trata de un problema mal condicionado cuya solución está muy sesgada ante pequeñas perturbaciones. Cuando la estimación se realiza para clasificar la señal observada en dos categorías, los errores de estimación inducen varias clasificaciones erróneas. En este trabajo proponemos un nuevo criterio de clasificación de señales explotando la relación entre el problema de valores propios generalizados planteado en el MPM y el rango numérico de un par de matrices rectangulares. En particular, la prueba de clasificación se formula como un problema de inclusión de conjuntos y no se requiere estimación del espectro. La técnica se aplica a un problema de dispersión electromagnética para clasificar materiales dieléctricos utilizando la señal de dispersión observada cuando un objetivo es iluminado por una señal de banda ultraancha. El rendimiento del esquema de clasificación se evalúa en términos de tasa de error y se compara con otra técnica de clasificación, la prueba de tasa de verosimilitud generalizada.

\section{Estado del arte}
A pesar de su eficiencia computacional y amplia aplicabilidad, el análisis de Fourier tiene algunas limitaciones bien conocidas, como su resolución limitada y el \emph{leakage} en el dominio de la frecuencia. Estas restricciones complican el análisis de señales que presentan un decaimiento exponencial en el tiempo. El análisis de Fourier que representa una señal como una suma de funciones periódicas, no es muy adecuado para la descomposición de señales aperiódicas, como las que decaen exponencialmente. El amortiguamiento provoca un ensanchamiento de los picos espectrales, lo que a si vez conduce a que los picos se superpongan y enmascaren los picos de amplitud más pequeños. Estos últimos pueden resultar importantes para la clasificación de señales. 

Muchas ejemplos donde aparecen señales que decaen exponencialmente se encuentran, señales de espectroscopia \cite{Gudmundson2012}, en el análisis de señales sísmicas \cite{Soltaninejad2020}, señales de radar \cite{Cuyt2020, Sarkar2000}, señales biomédicas \cite{Bannis2014, Chon2001}, detección de dirección de arribo (DOA) \cite{Knaepkens2020}, identificación por radio frecuencia \cite{Rezaiesarlak2013}, señales musicales \cite{Laroche1993}, por nombrar solo algunos. 

Un enfoque diferente al análisis espectral, entre otros, son los métodos paramétricos, donde se incluyen MUSIC \cite{Schmidt1986}, ESPRIT \cite{Roy1989}, el método de haz matricial (\emph{Matrix Pencil}, MPM) \cite{Hua1990}, factorización QR simultanea  \cite{Golub1999} o el método de Prony aproximado APM \cite{Potts2010}.

En general, los métodos paramétricos, así como el análisis de Fourier, utilizan el Teorema de Shannon-Nyquist \cite{Nyquist1928, Shannon1949} para obtener muestras de una señal. Este teorema afirma que la frecuencia de muestreo debe ser al menos el doble de la frecuencia máxima presente en la señal. Una frecuencia de muestreo más baja provoca \emph{aliasing}, identificando frecuencias más altas con frecuencias más bajas sin poder distinguirlas. En el último tiempo, enfoques alternativos han demostrado que la reconstrucción de señales también es posible a partir de mediciones obtenidas con tasas sub-Nyquist, si se conoce información adicional sobre la estructura de la señal, como por ejemplo su dispersión (\emph{sparsity}). De hecho, muchas señales son dispersas en algún dominio como el tiempo, la frecuencia o el espacio, lo que significa que la mayoría de las muestras de la señal o de su transformada en otro dominio pueden considerarse cero. %Entre otros métodos, se pueden citar aquellos que se basan en \emph{Compress Sensing} \cite{Donoho2006}. Una ventaja esencial de los métodos paramétricos es que no necesita mediciones recopiladas aleatoriamente, sino que funciona con un muestreo determinista basado en un esquema de muestreo que se adapta al modelo de señal no lineal.

Sin embargo, una de las principales preocupaciones a la hora de desarrollar estas técnicas ha sido hacer frente al ruido y a las incertidumbres del modelo. Últimamente, se han propuesto nuevas técnicas de optimización \cite{Andersson2014, Grussler2018}. En \cite{Andersson2014} los autores formulan un problema de cuadrados mínimos no lineales en términos de una restricción sobre el rango de la matriz de Hankel asociada a la matriz observada. Mientras que en \cite{Grussler2018} se plantea el problema de aproximación de bajo rango contemplando la estructura Hankel de la matriz observada. Un enfoque similar se propone en \cite{Ying2018} donde la estimación de los parámetros se realiza minimizando la norma nuclear de la matriz de Hankel. En \cite{Yang2018} se propone un enfoque diferente donde los autores consideran sumas de exponenciales no amortiguadas. En este caso, usando un conjunto de frecuencias discretas previamente definido, se consideran técnicas de \emph{Compressed Sensing} \cite{Donoho2006} para estimar los parámetros desconocidos. En \cite{Yang2015,Yang2016} se presentan algunas mejoras a este enfoque donde la solución se obtiene minimizando la norma atómica de la señal estimada. En estos casos, se garantiza la recuperación exacta de la señal observada cuando la frecuencias están separadas adecuadamente \cite{Candes2014}. Desafortunadamente, estos métodos no son adecuados para suma de exponenciales complejas amortiguadas.

Como se señalo en \cite{Halder1997}, el rendimiento de los métodos de estimación espectral, se degradan cuando se tiene frecuencias muy cercanas entre sí. No obstante, una estimación precisa de la parámetros espectrales para suma de exponenciales complejas amortiguadas se realiza mediante la solución del problema de autovalores generalizados. Sin embargo, estos problemas suelen estar mal condicionados por lo que se requiere un esfuerzo adicional para ejecutar de manera confiable los algoritmos de estimación correspondientes. el estudio de la estabilidad numérica  de los autovalores generalizados se ha abordado antes en \cite{Golub1999, Beckermann2007}. En estos trabajos se ha observado que la sensibilidad de cada autovalor es inversamente proporcional a la amplitud asociada a la frecuencias amortiguada correspondiente al autovalor. Resolver problemas de autovalores generalizados con estas características es un desafío desde la perspectiva algorítmica y puede volverse aun más difícil cuando el número de frecuencias en la señal original es muy grande y estas frecuencias están agrupadas en pequeñas regiones del plano complejo \cite{Cuyt2018, BATENKOV2018, Li1997}. Otro forma de mejorar la estabilidad numérica es la aplicación de enfoques numéricos más sofisticados a matrices y su conexión con los métodos de aproximación racional, de modo que se puedan utilizar algoritmos estables existentes \cite{Derevianko2022}.

Por otro lado, un paso sensible en cualquier método de estimación espectral paramétrica es estimar con precisión el orden del modelo.  Las técnicas más empleada para estimar el orden del modelo son los criterios de la información \cite{Stoica2004}. Entre los más conocidos se encuentran el Criterio de información de Akaike, MDL, así como enfoques más reciente desarrollados en \cite{Mariani2015, Nielsen2013}, que garantizan un buen desempeño en el caso asintótico. Sin embargo, para registros de datos pequeños, estos métodos ya no son óptimos y su desempeño se ve deteriorada a medida que disminuye la relación señal a ruido (SNR).

Una estrategia alternativa utiliza el Teorema de Kronecker \cite{Gantmacher1960} que establece un correspondencia uno a uno entre una combinación lineal de $r$ exponenciales complejas y una matriz de Hankel con rango $r$. Este teorema establece la solución al problema de encontrar una función racional cuyo numerador y denominador sean polinomios de grado $r-1$ y $r$ respectivamente  \cite{Fuhrmann2011}. Este problema está relacionado con los problemas de aproximaciones de Padé \cite{Gonnet2013} y la teoría de polinomios ortogonales \cite{Szego1939}.   Desafortunadamente, estos resultados se  dedujeron para señales sin perturbaciones y se utilizan como una aproximación cuando la señal observada está contaminada por ruido. Cuando hay ruido presente, el espacio columna de la matriz de Hankel se puede descomponer en la suma directa del subespacio de señal y el subespacio de ruido. Por lo general, la dimensión del subespacio de señal se establece por el número de valores singulares prominentes de la matriz de Hankel. Sin embrago, a medida que la SNR disminuye, la brecha entre valores singulares consecutivas disminuye, lo que dificulta determinar cuántos valores singulares relevantes tiene la matriz de Hankel.

Continuado con el análisis de la matriz de Hankel asociada a la suma de exponenciales complejas, se propusieron dos técnicas conocidas como ESTER \cite{Badeau2006} y SAMOS \cite{Papy2007} para la estimación del orden del modelo. Estos métodos se basan en la propiedad de invariancia rotacional del subespacio de señal generado por la matriz de Hankel \cite{Roy1989}. Si bien se demostró que SAMOS es más robusto que ESTER, ambas técnicas presentan un buen desempeño en régimen de SNR alto. Sin embargo, ambos métodos se basan en suposiciones donde la matriz de Hankel está libre de ruido, por lo que no logran obtener buenos resultados en condiciones más generales. 

Por otro lado, algunos autores han considerado la aleatoriedad de las señales ruidosas explotando sus propiedades estadísticas. Por ejemplo, en \cite{Qiao2020} se propone un umbral para separar los valores singulares significativos de los correspondiente a ruido cuando se trabaja con matrices reales aleatorias con entradas sub-gaussianas y estructura Toeplitz. Un umbral similar se obtuvo utilizando desigualdades de concentración para una matriz aleatoria de Hankel con entradas Gaussianas \cite{tropp2015}. Para matrices de Hankel más generales, en \cite{Hokanson2020} se propuso un umbral como cota superior de la norma espectral de una matriz de Hankel aleatoria. No obstante, todos estos umbrales son conservativos y no logran detectar el rango verdadero de la matriz de Hankel con entradas aleatorias, lo que dificulta el uso de cualquiera de ellos para determinar el orden del modelo. 

Un problema relacionado con una mezcla de señales senoidales poco espaciadas en un entorno ruidoso se observa en la dispersión medida de un objeto cuando es iluminado por una señal electromagnética UWB, donde la señal reflejada se puede caracterizar mediante el modelo exponencial \cite{Baum1971} donde sus parámetros transportan una gran cantidad de información sobre las características del objeto. En particular se ha demostrado que las frecuencias complejas solo dependen del tamaño, forma y propiedades eléctricas del objeto iluminado. Por lo tanto se ha propuesto tomar estas frecuencias como características para la clasificación de diferentes blancos \cite{Knochel2005, Bannis2014, LAUNAY2013, Altieri2020}.  Este procedimiento equivale a resolver un problema de estimación espectral construido a partir de las muestras del campo disperso. Sin embargo, se ha observado que la energía asociada con cada una de las frecuencias depende de las posiciones del punto donde se envía el campo incidente, el punto de observación y el objeto analizado \cite{Baum1976}. Este hecho implica que es posible que alguna frecuencia no se observe de manera significativa, o que varias frecuencias se agrupen en una región particular del plano complejo muy cerca entre sí, lo que hace que la tarea de identificación sea un problema de estimación espectral desafiante. En particular, se requieren técnicas de estimación espectral capaces de hacer frente a un orden de modelo incierto y una baja relación señal a ruido. 

Finalmente, dado los avances de los métodos de aprendizaje profundo, se debe mencionar que estos puede usarse para la estimación del orden del modelo \cite{Moon2021} y la estimación de los parámetros \cite{Mhaska2020}. En particular, los modelos de entrenamiento para el aprendizaje requieren grandes cantidades de datos para minimizar el error de generalización. Desafortunadamente, muchas veces los datos de entrenamiento no está disponibles para todas las aplicaciones.

El resto de esta tesis está organizada de la siguiente manera:

\begin{itemize}
	\item Capítulo~\ref{chap:ModeloSumExp} se presenta una descripción de los métodos utilizados para la extracción de frecuencias, y se analiza la estabilidad numérica de los métodos para obtener los parámetros asociado al modelo exponencial. 
	\item[] 
	\item Capítulo~\ref{chap:OrdenModelo} hace hincapié en la selección del orden del modelo. Se presenta una descripción de los diferentes métodos encontrados en la literatura y se analizan sus dificultades. 
	\item[] 
    \item Capítulo ~\ref{chap:RandomHankel} se analiza la distribución de los valores singulares para los casos de matrices aleatorias con y sin estructura Hankel. Además se presentan cotas para la distribución de la norma espectral para el caso de matrices con estructura Hankel. 
    \item[]
	\item Capítulo ~\ref{chap:CriterioOrdenOpt} teniendo en cuenta la dificultades encontradas en los métodos estudiados tanto en el Capítulo \ref{chap:OrdenModelo} como en el Capítulo \ref{chap:RandomHankel} se propone una nueva alternativa basado en las propiedades algebraicas y estadísticas que presenta el modelo.
	\item[] 
	\item Capítulo~\ref{chap:EstabilidadNumerica} se extiende el resultado sobre estabilidad numérica descrito en el Capítulo \ref{chap:ModeloSumExp}. Además, se propone un estrategia para pre-procesar la señal observada mediante un procesamiento muti-frecuencia de muestreo. Esto tiene la capacidad de acomodar las frecuencias complejas para una cálculo más estable.
	\item[] 
	\item Capítulo~\ref{chap:sinalUWB} se muestran resultados de los métodos desarrollados aplicado a mediciones tomadas en el laboratorio utilizando señales de UltraWide Band (UWB).
	\item[] 
	\item Capítulo ~\ref{chap:RangoNumerico} se diseña una estrategia novedosa para clasificar mezclas de señales sinusoidales basado en el rango numérico de matrices. En particular, la clasificación se formula como un problema de inclusión de conjuntos, evitando así la necesidad de calcular autovalores de matrices de grandes dimensiones mal condicionadas.
	\item[]
	\item Capítulo~\ref{chap:conclusiones} se presentan las conclusiones finales. Resume los principales logros de esta tesis y describe futuras direcciones de investigación.
\end{itemize}

\section{Contribuciones}

Las principales contribuciones se pueden resumir de la siguiente manera:
\begin{itemize}
	\item Se resumen los distintos métodos de estimación espectral aplicados a mezclas de señales sinusoidales. Se demostró que las estimaciones se vuelven vulnerables a pequeñas perturbaciones en las señales observadas cuando los modos tiene poca energía y/o la distancia entre ellos es pequeña.
	\item[] 
	\item Para hacer frente al problema de estabilidad numérica se propone una nueva estrategia para aumentar artificialmente la separación de frecuencias cambiando la frecuencia de muestre de la señal ruidosa antes de aplicar los métodos de estimación espectral. Se muestra que esta metodología mejora significativamente el rendimiento de la estimación, especialmente en el régimen de baja relación señal/ruido. Se comparan los resultados obtenidos en distintas publicaciones.
	\item[] 
    \item Para el problema de estimación de orden se analizaron las dificultades que presentan los métodos basados en el principio de invariancia rotacional. Para ello, se expresaron las reglas de estimación en términos de los ángulos principales entre subespacios, que muestran severas fluctuaciones cunado el ruido aumenta. También, se obtuvieron cotas para la norma espectral de una matriz aleatoria con estructura Hankel considerando la estructura de su valor singular máximo. También, a partir de la relación entre el principio de invariancia rotacional y los ángulos principales entre subespacios se obtienen resultados sobre las perturbaciones para el problema de autovalores generalizados del haz matricial. % VER 
    \item[]
	\item Se presenta un nuevo método para estimar el número de términos en la mezcla señales sinusoidales. En particular, se propone combinar las propiedades algebraicas y estadísticas de los métodos basados en subespacios. Esta nueva técnica de estimación  muestra mejoras significativas sobre los métodos basados en subespacios. En particular, cuando no es posible tener una buena separación entre los subespacio de ruido y de señal, esta nueva metodología supera a las técnicas conocidas. El desempeño se evaluó usando experimentos numéricos y comparando con resultado previos encontrados en la bibliografía.
	\item[]
	\item Se propone un nuevo criterio de clasificación de señales explotando la relación entre el problema de autovalores generalizados y el rango numérico de un par de matrices rectangulares. En particular, el problema de clasificación se formula como un problema de inclusión de conjuntos, y no se requiere la estimación de la frecuencias.
\end{itemize}


%Además, suelen presentarse otros obstáculos. Por un lado, la adquisición de mediciones pueden resultar muy costosa y por lo tanto ser limitada. En otros casos, estas mediciones pueden estar contaminadas con ruido. Muchos ejemplos se encuentran en imágenes por resonancia magnética \cite{Spencer2020}, espectroscopia \cite{Gudmundson2012}, en el análisis de señales sísmicas \cite{Soltaninejad2020}, señales de radar \cite{Cuyt2020, Sarkar2000}, señales biomédicas \cite{Bannis2014, Chon2001}, detección de dirección de arribo (DOA) \cite{Knaepkens2020}, identificación por radio frecuencia \cite{Rezaiesarlak2013}, señales musicales \cite{Laroche1993}, por nombrar solo algunos. 

%En general, estas señales se suelen representar como una suma de exponenciales complejas que puede ser amortiguadas o no amortiguadas. Los métodos de subespacios \cite{Stoica2005} son herramientas adecuadas para la estimación espectral, ya que ofrecen una alternativa al enfoque de \emph{Compress Censing} (CS) \cite{Yang2018}. Una ventaja esencial de los métodos basados en  subespacios es que no necesita mediciones recopiladas aleatoriamente, sino que funciona con un muestreo determinista basado en un esquema de muestreo que se adapta al modelo de señal no lineal.

